\documentclass[fleqn]{article}
\usepackage{german}
\begin{document}
\setlength{\mathindent}{2cm}

Die reduzierte kubische Gleichung $y^3 + 3py +2q = 0$ hat f"ur
$D = q^2 + p^3 > 0$
eine reelle und zwei komplexe L"osungen. Diese lassen sich mit den Abk"urzungen
\[ u = \sqrt[3]{-q + \sqrt{q^2+p^3}},\qquad v = \sqrt[3]{-q - \sqrt{q^2+p^3}} \]
nach der \emph{Cardanischen} Formel als
\begin{equation} y_1 = u + v \end{equation}
\begin{equation}
   y_2 = -\frac{u+v}{2} + \frac{i}{2}\sqrt{3}(u - v)
\end{equation}
\begin{equation}
   y_3 = -\frac{u+v}{2} - \frac{i}{2}\sqrt{3}(u-v)
\end{equation}
darstellen.
\end{document}

