\documentclass{article}
\usepackage{german}
\begin{document}
Die Vereinigung zweier Mengen $\mathcal{A}$ und
$\mathcal{B}$ ist die Menge aller Elemente, die in wenigstens einer der beiden
Mengen vorkommen, und wird als $\cal A\cup B$ gekennzeichnet. Diese
Operation ist kommutativ $\cal A\cup B = B\cup A$ und assoziativ
$\cal(A\cup B)\cup C = A\cup(B\cup C)$. Ist $\cal A\subseteq B$, dann gilt
$\cal A\cup B = B$. Daraus folgt $\cal A\cup A = A$,
$\cal A\cup\emptyset = A$ und $\cal J\cup A = J$. (\,$\emptyset$ steht
f"ur die leere Menge.)
\end{document}
