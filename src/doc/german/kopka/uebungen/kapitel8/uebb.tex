\section{Die \LaTeX-Folgeb"ande 2 und 3}
\subsection{Inhaltsbeschreibung von Band 2}
Band 2 der Buchserie "uber \LaTeX\ \cite{hk2} beginnt mit der Vorstellung von 
\LaTeX-Erg"anzungen, die in ihrer Wirkung in Zukunft Bestandteil von
\LaTeX~3 werden und die bereits heute in Form allgemein zug"anglicher
Erg"anzungspakete genutzt werden k"onnen. Die Standardzeichens"atze einer
\LaTeX-Installation waren prim"ar auf Anwendungen aus dem
mathematisch-naturwissenschaft"-lichen Bereich zugeschnitten. 
Mit der Ausbreitung auch auf geisteswissenschaftliche Anwendungen treten
zunehmend Anforderungen auf, die damit nicht zu erf"ullen sind. Inzwischen
existieren f"ur nahezu alle Sprachen und Sonderf"alle, wie z.~B. zur
Schachdokumentation oder zum Musiknotensatz, geeignete Zeichens"atze.
Band 2 stellt eine Vielzahl solcher erg"anzender Zeichens"atze vor, wobei sich
die Erg"anzungen in Richtung \LaTeX~3 als ganz besonders hilfreich erweisen.
Auch die Einbeziehung von PostScript-Zeichens"atzen in die \LaTeX-Bearbeitung
wird angesichts der immer h"aufiger verwendeten PostScript-f"ahigen Drucker
in Kapitel 4 vorgestellt. Die Nutzung von Zeichens"atzen f"ur den Musiknotensatz
zusammen mit einem geeigneten Erg"an"-zungspaket wird in  Kapitel 3 beschrieben. 

Band 2 stellt weiterhin die M"oglichkeiten zur Einbindung von Bildern und
Grafiken vor, die entweder aus v"ollig anderen Programmquellen stammen oder
mit \TeX-eigenen Mitteln, wie mit PiC\TeX, erzeugt werden.
Er schlie"st ab mit einer Einf"uhrung in METAFONT in Kapitel 7. Angesichts der
Vielzahl der vorgestellten Zusatz-Zeichens"atze ist dies eine folgerichtige
Erg"anzung, da die Installation der Zusatz-Zeichens"atze bei vielen Anwendern
aus den Quelldateien zu erfolgen hat, womit der Programmaufruf von METAFONT
mit geeigneten Einstellparametern zwingend notwendig wird.
Band 2 wendet sich also an Anwender, die "uber die M"oglichkeiten 
einer Standard-\LaTeX-Installation hinausgehen wollen, ohne hierzu
in die Tiefen der Programmierung zur Entwicklung von Eigenerweiterungen
steigen zu m"ussen.

\subsection{Inhaltsbeschreibung von Band 3}
F"ur solche Entwicklungen ist schlie"slich der Band 3 gedacht \cite{hk3}.
Er stellt \LaTeX\ in seinen internen Strukturen vor, erg"anzt um eine
Darstellung der wichtigsten \TeX-Strukturen. Mit diesen Kenntnissen werden dann
anschlie"send Beispiele f"ur anwendereigene \LaTeX-Erweiterungen vorgestellt.
Ebenso werden Interna des Bib\TeX-Pro"-gramms angesprochen, aus denen
der Anwender weitere \textsc{Bib}\TeX-Stilfiles zur variablen Gestaltung
von Literaturverzeichnissen erstellen kann.

Jede \TeX-Installation kennt weitere  \TeX-Zusatzwerkzeuge, von denen ich hier
beispielhaft das Programm patgen nenne. Mit diesem Programm kann man
f"ur jede Sprache ein \TeX-spezifisches Trennmusterfile erstellen,
indem als Eingabe ein Trennlexikon der entsprechenden Sprache herangezogen
wird. Band 3 stellt in seinem Anhang alle \TeX-Standard-Zusatzwerkzeuge vor und
beschreibt deren Anwendung und Eigenschaften.
