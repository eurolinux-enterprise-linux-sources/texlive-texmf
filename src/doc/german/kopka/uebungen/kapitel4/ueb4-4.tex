\documentclass{article}
\usepackage{german}
\begin{document}
\begin{itemize}
\item Die Markierung der ersten Stufe ist ein dicker schwarzer Punkt.
\begin{itemize}
\item Die der zweiten Stufe ein Streckenstrich.
\begin{itemize}
\item Die der dritten Stufe ein Stern.
\begin{itemize}
\item Und die der vierten Stufe ist ein hochgestellter Punkt.
\item Gleichzeitig vermindert sich der vertikale Abstand mit
      zunehmender Schachtelungstiefe.
\end{itemize}
\item Zur"uck zur dritten Stufe.
\item In jeder Stufe d"urfen mehrere Aufz"ahlungspunkte der gleichen
      Stufe aufeinander folgen,
\item wie hier f"ur die dritte Stufe demonstriert wird.
\end{itemize}
\item Und hier sind wir wieder bei der zweiten Stufe.
\end{itemize}
\item Und hier wieder bei der ersten.
\item Die standardm"a"sigen Markierungssymbole der verschiedenen
     Stufen k"on"-nen vom Anwender abge"andert werden,
\item wie mit "Ubung 4.6 gezeigt wird.
\end{itemize}

\begin{enumerate}
\item Die Numerierung der ersten Stufe erfolgt in arabischen Ziffern,
      gefolgt von einem Punkt.
\begin{enumerate}
\item Die Numerierung der zweiten Stufe erfogt in Kleinbuchstaben,
      die in () gesetzt sind.
\begin{enumerate}
\item Die Numerierung der dritten Stufe erfolgt in kleinen r"omischen
      Ziffern, gefolgt von einem Punkt.
\begin{enumerate}
\item Die Numerierung der vierten Stufe erfolgt in Gro"sbuchstaben, gefolgt
      von einem Punkt.
\item Eine "Anderung der standardm"a"sigen Markierung ist m"oglich und wird
      mit "Ubung 4.7 demonstriert.
\end{enumerate}
\item Hier sind wir wieder bei der dritten Stufe.
\end{enumerate}
\item Und hier bei der zweiten.
\item Auch bei dieser Aufz"ahlung sind in jeder Stufe mehrere Punkte
      der gleichen Stufe erlaubt,
\item wie hier f"ur die zweite Stufe demonstriert wird.
\end{enumerate}
\item Jede Stufe sollte mindestens zwei Aufz"ahlungspunkte enthalten,
\item auch wenn formal weniger Aufz"ahlungspunkte pro Stufe erlaubt sind!
\end{enumerate}
\end{document}

