\documentclass{article}
\usepackage{german}
\renewcommand{\labelitemi}{---}
\renewcommand{\labelitemii}{--}
\renewcommand{\labelitemiii}{-}
\begin{document}
\begin{itemize}
\item Die Markierung der ersten Stufe ist ein dicker schwarzer Punkt.
\begin{itemize}
\item Die der zweiten Stufe ein Streckenstrich.
\begin{itemize}
\item Die der dritten Stufe ein Stern.
\begin{itemize}
\item Und die der vierten Stufe ist ein hochgestellter Punkt.
\item Gleichzeitig vermindert sich der vertikale Abstand mit
      zunehmender Schachtelungstiefe.
\end{itemize}
\item Zur"uck zur dritten Stufe.
\item In jeder Stufe d"urfen mehrere Aufz"ahlungspunkte der gleichen
      Stufe aufeinander folgen,
\item wie hier f"ur die dritte Stufe demonstriert wird.
\end{itemize}
\item Und hier sind wir wieder bei der zweiten Stufe.
\end{itemize}
\item Und hier wieder bei der ersten.
\item Die standardm"a"sigen Markierungssymbole der verschiedenen
     Stufen k"on"-nen vom Anwender abge"andert werden,
\item wie mit "Ubung 4.6 gezeigt wird.
\end{itemize}
\end{document}
