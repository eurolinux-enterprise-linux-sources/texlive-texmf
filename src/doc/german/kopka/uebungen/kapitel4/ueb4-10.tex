\documentclass{article}
\usepackage{german}
\newcounter{num}
\newenvironment{muster}{\begin{list}{\textbf{Muster \Alph{num}}\hfill}%
   {\usecounter{num} \setlength{\labelwidth}{20mm}
    \setlength{\leftmargin}{22mm} \setlength{\rightmargin}{5mm}
    \setlength{\itemsep}{1ex plus0.5ex minus 0.4ex}
    \setlength{\listparindent}{1em}
    \setlength{\parsep}{0ex plus0.5ex} }}
   {\end{list}}
\begin{document}
Die nachfolgende anwendereigene \texttt{muster}-Umgebung zeigt die Anordung
der ausgegebenen Marke und der linken und rechten Texteinr"uckung: 
\begin{muster}
\item ohne Wert.
\item bei Einsendung des beigef"ugten Gutscheins wird der Betrag von 5.-- DM
   gutgeschrieben.
\end{muster}
\end{document}
