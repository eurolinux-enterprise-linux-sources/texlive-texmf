\documentclass{article}
\usepackage{german}
\setlength{\textwidth}{130mm}
\begin{document}
\begin{table}
\begin{tabular*}{130mm}%
   {@{}ll@{\extracolsep{\fill}}r@{\hspace{1em}}@{\extracolsep{1em}}rr@{}}
\hline
\multicolumn{2}{@{}l}{Energietr"ager} & 1975 & 1980 & 1986\\ \hline
\multicolumn{2}{@{}l}{Gesamtverbrauch}&&& \\
\multicolumn{2}{@{}l}{(in Mio.\ t SKE)} & 347.7 & 390.2 & 385.0 \\
davon & (Anteile in \%)\\
& Mineral"ol & 52.1 & 47.6 & 43.2 \\
& Steinkohle   & 19.1 & 19.8 & 20.0 \\
& Braunkohle   &  9.9 & 10.0 &  8.6 \\
& Erdgas       & 14.2 & 16.5 & 15.1 \\
& Kernenergie  &  2.0 &  3.7 & 10.1 \\
& Sonstiges    &  2.7 &  2.3 &  3.0 \\ \hline
\end{tabular*}

\emph{Quelle:} Arbeitsgemeinschaft Energiebilanzen, Essen 1987.
\end{table}

\noindent \textbf{Antworten der "Ubungsfragen}:

\begin{enumerate}
\item Mit \verb=@{}= am Anfang und Ende des Formatierungsfeldes wird
      horizontaler Leerraum von der H"alfte des standardm"a"sigen 
      Spaltenzwischenraums vor und nach der Tabelle unterdr"uckt.
\item Mit \verb=@{\extracolsep{\hfill}}= am Anfang des Formatierungsfeldes
      w"urde zwischen allen nachfolgenden Spalten bis zum Auftreten eines
      neuen \texttt{\@}-Ausdrucks der gleiche Spaltenzwischenraum eingef"ugt,
      um die geforderte Tabellenbreite zu erhalten. Bitte als Sonder"ubung
      nachvollziehen!
\item Der Ausdruck \verb=@{\extracolsep{\hfill}}= steht f"ur die 
      "Ubungstabelle nach dem zweiten Spaltenparameter. Die diesen Ausdruck
      aufhebende Befehlsgruppe \verb=@{\hspace{1em}@{\extracolsep{1em}}= steht
      nach dem dritten Spaltenparameter.
\end{enumerate}
\end{document}

