\documentclass{article}
\usepackage{german}
\renewcommand{\labelenumi}{\Roman{enumi}}
\renewcommand{\labelenumii}{\labelenumi--\arabic{enumii}}
\begin{document}
\begin{enumerate}
\item Die Numerierung der ersten Stufe erfolgt in arabischen Ziffern,
      gefolgt von einem Punkt.
\begin{enumerate}
\item Die Numerierung der zweiten Stufe erfogt in Kleinbuchstaben,
      die in () gesetzt sind.
\begin{enumerate}
\item Die Numerierung der dritten Stufe erfolgt in kleinen r"omischen
      Ziffern, gefolgt von einem Punkt.
\begin{enumerate}
\item Die Numerierung der vierten Stufe erfolgt in Gro"sbuchstaben, gefolgt
      von einem Punkt.
\item Eine "Anderung der standardm"a"sigen Markierung ist m"oglich und wird
      mit dieser "Ubung demonstriert.
\end{enumerate}
\item Hier sind wir wieder bei der dritten Stufe.
\end{enumerate}
\item Und hier bei der zweiten.
\end{enumerate}
\item Und schlie"slich wieder bei der ersten
\begin{enumerate}
\item Auch bei dieser Aufz"ahlung sind in jeder Stufe mehrere Punkte
      der gleichen Stufe erlaubt,
\item wie hier f"ur die zweite Stufe demonstriert wird.
\end{enumerate}
\item Jede Stufe sollte mindestens zwei Aufz"ahlungspunkte enthalten,
\item auch wenn formal weniger Aufz"ahlungspunkte pro Stufe erlaubt sind!
\end{enumerate}
\end{document}
