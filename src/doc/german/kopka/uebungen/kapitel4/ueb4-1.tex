\documentclass{article}
\usepackage{german}
\begin{document}
Beispiel f"ur die quote-Umgebung aus Band 1 der \LaTeX-Buchserie. Siehe
dort Abschnitt 4.2.3:
\begin{quote}
Zwischen dem einger"uckten Text und dem vorangehenden bzw. folgenden
Normaltext wird ober- und unterhalb des eingr"uckten Textes zus"atzlicher
vertikaler Zwischenraum eingef"ugt.

Der einger"uckte Text darf beliebig lang sein. Er kann aus einem Teilsatz,
einem ganzen Absatz oder gar aus mehreren Abs"atzen bestehen.

Mehrere Abs"atze werden bei der Eingabe wie "ublich durch eine Leerzeile
getrennt, jedoch sollte eine Leerzeile zu Beginn und Ende des einger"uckten
Textes entfallen, da hier automatisch zus"atzlicher Zwischenraum eingef"ugt
wird.
\end{quote}

Und hier folgt das Beispiel f"ur eine quotation-Umgbung aus dem gleichen
Abschnitt wie oben zitiert:
\begin{quotation}
Bei der \texttt{quotation}-Umgebung werden Abs"atze durch zus"atzliches
Einr"ucken der ersten Zeile gekennzeichnet, w"ahrend bei der 
\texttt{quote}-Umgebung die Abs"atze durch zus"atzlichen vertikalen
Zwischenraum voneinander getrennt wurden.

Demzufolge ist der hier einger"uckte Text durch die 
\texttt{quotation}-Umgebung erzeugt worden, der weiter oben einger"uckte
Text dagegen mit der \texttt{quote}-Umgebung.

Die Umgebung \texttt{quotation} wird man sinnvollerweise nur verwenden,
wenn auch im sonstigen Text Abs"atze durch Einr"ucken der ersten Zeile
gekennzeichnet sind.
\end{quotation}
\end{document}

