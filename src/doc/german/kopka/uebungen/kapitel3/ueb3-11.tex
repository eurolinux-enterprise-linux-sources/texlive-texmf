\documentclass[twoside]{report}
\usepackage{a4,german}
\pagestyle{headings} 
\setcounter{secnumdepth}{1}
\begin{document}

\chapter[Die \LaTeX-Buchserie]{Inhaltsbeschreibung der \LaTeX-Buchserie}
 
\section{Entwicklungsgeschichte der \LaTeX-Buchserie}
Der Vorl"aufer dieses Buches mit dem Titel \LaTeX\ -- Eine Einf"uhrung
erlebte zwischen 1988 und 1993 vier Auflagen, wobei die vierte Auflage
mit den Erscheinungsjahren 1992 und 1993 eigentlich aus zwei
eigenst"andigen Ausgaben bestand, zwischen denen es deutliche
Textumstellungen und Erg"anzungen gab. Als Erg"anzung zur Einf"uhrung
erschien 1990 ein zweites Buch mit dem Titel
\LaTeX-Erweiterungsm"oglichkeiten, das zwischen 1990 und 1993 in drei jeweils
deutlich erweiterten Auflagen herauskam. Dabei kam es zunehmend zu 
"Uberschneidungen zwischen beiden B"uchern, die "uberdies immer volumin"oser
wurden und in ihren Erg"anzungen "uber die Zielrichtung der
Buchtitel hinausgingen. Dies hatte zur Folge, da"s Leser/innen der B"ucher
auch mit Darstellungsstoff konfrontiert wurden, der dem aktuellen 
Nutzungsbed"urfnis nicht oder noch nicht zum Zeitpunkt des Einstiegs in \LaTeX\ 
entsprach.

Auf Anregung des damaligen Pr"asidenten der deutschsprachigen
\TeX-Anwender"-vereinigung, Joachim Lammarsch, schlug mir der Verlag vor,
\LaTeX\ in Form einer dreib"andigen Buchserie zu pr"asentieren: 
Band 1 -- \LaTeX-Einf"uhrung, Band 2 -- \LaTeX-Erg"anzungen, Band 3 --
\LaTeX-Erweiterungen.

\section{Inhaltsbeschreibung von Band 1}
Der vorliegende Band 1 geht weitgehend auf das urspr"ungliche Buch 
\LaTeX\ -- Eine Einf"uhrung zur"uck und beschr"ankt sich nun auf die
Vorstellung des internationalen \LaTeX-Standards, lediglich erg"anzt
um die Darstellung von german.sty zur Einbindung deutscher 
Besonderheiten in die \LaTeX-Bearbeitung. Letztere mu"s man f"ur 
deutschsprachige Anwender, und an diese richtet sich das Buch, als zu
unserem Standard geh"orend betrachten. 

In bezug auf den internationalen \LaTeX-Standard ist die Vorstellung
in Band 1 aber vollst"an"-dig. Sie schlie"st somit Bib\TeX\
und MakeIndex ein, da diese \LaTeX-Erg"anzungen
vom Programmautor Leslie Lamport selbst stammen oder unter seiner
aktiven Mithilfe entstanden und von ihm zum Bestandteil des \LaTeX-Gesamtpakets 
erkl"art wurden.

Unmittelbar nach Drucklegung der 1. Auflage von Band 1 erschien auf den
"offentlichen \TeX-Fileservern die neue \LaTeXe-Version, zun"achst als
Probeversion und ausdr"ucklich als solche gekennzeichnet. Nach einer
halbj"ahrigen Erprobungsphase mit Behebung einer Reihe von Fehlern und
Schw"achen wurde im Juni 1994 die \LaTeXe-Probeversion zur
Standard-\LaTeX-Version erkl"art. Die neuen Eigenschaften von \LaTeXe wurden
deshalb zun"achst in Kapitel 1 von Band 2 nachgetragen. Mit den Neuauflagen
von Band 1 erscheinen die Grundeigenschaften
von \LaTeXe\ nunmehr sachgerecht in der Einf"uhrung. 

\section{\LaTeX-Installationshinweise}
Der Einf"uhrungsband 1 schlie"st ab mit Hinweisen zur \TeX-Installation und der
Erstellung der erforderlichen Formatfiles im Anhang F. 
Bei der Zuf"ugung dieses 46seitigen Anhangs befand ich mich in einem Konflikt:
Er geh"ort im engeren Sinne sicherlich nicht zum Stoffbereich einer 
\LaTeX-Einf"uhrung.
Alle mir bekannten B"ucher "uber \TeX\ und seine Makropakete gehen
stillschweigend von der Annahme aus, da"s ein lauff"ahiges \TeX-Programm
mit den erforderlichen Zusatzwerkzeugen im Rechner des Anwenders existiert.
Dies war in den Anfangsjahren von \TeX\ und \LaTeX\ auch sachgerecht, da
\TeX\ damals zun"achst in den Rechenzentren der Hochschulen und 
Forschungsinstitute bereitgestellt wurde. Alle bei der Installation eines
\TeX-Systems vorausgesetzten Kenntnisse und auftretenden Probleme stellten
sich nicht dem Anwender, sondern dem entsprechenden Experten des
Rechenzentrums.

Inzwischen hat sich das Anwenderprofil deutlich ge"andert. Die Mehrzahl
der \TeX- und \LaTeX-Anwender betreibt das Programm auf einem PC. 
F"ur nahezu alle Individualrechner (IBM-PCs und kompatible, Atari, Amiga,
Macintosh und UNIX-Workstations) stehen sowohl kommerzielle wie auch
kostenlose PD- (Public Domain) oder SW- (Shareware) \TeX-Pakete zur 
Verf"ugung. Dokumentation und Installationshilfen sind je nach Programmquelle
unterschiedlich hilfreich. 

Der typische \TeX-Einsteiger will das Programm auf dem eigenen 
Individualrechner betreiben und die eigenen Kenntnisse auf die 
Nutzungsbed"urfnisse beschr"anken und nicht mit komplexen Wechselbeziehungen
zwischen den diversen Programmteilen einer \TeX- und \LaTeX-Installation
konfrontiert werden. Das Programmsystem wurde vermutlich als Diskettensatz
beschafft oder von einem Kollegen kopiert, was bei den PD- und SW-Produkten
auch erlaubt ist. Je nach beigef"ugter Dokumentation und Installationshilfe
kann sich die Installation f"ur den Einsteiger als schwierige H"urde erweisen.
Auch wenn die eigentliche Installation ohne Probleme gelingt, meldet das
Programm eventuell beim ersten Aufruf, da"s es gewisse Teile nicht findet und
damit die Bearbeitung abbricht.          

Die Ursache f"ur eine solche Meldung kann tats"achlich darin liegen, da"s
der Diskettensatz f"ur die Installation zwar das ausf"uhrbare \TeX-Programm
bereitstellt, die f"ur den praktischen Ablauf aber zwingend erforderlichen
Zusatzwerkzeuge, wie bestimmte Zeichensatzfiles und Makropakete, aus anderen
Quellen beschafft werden m"ussen, ohne da"s dies in der beigef"ugten
Dokumentation explizit gesagt wird. H"aufig liegt die Ursache f"ur die
genannte Fehlermeldung aber darin, da"s bestimmte Programmteile zwar vorhanden
sind, aber f"ur die Nutzung mit dem beigef"ugten Spezialprogramm INITEX, 
von dem der Einsteiger bis dahin noch nie etwas geh"ort hat, aufbereitet werden
m"ussen.  Ich hoffe, mit dem Anhang F auch dem Anf"anger 
bei seinem Einstieg in \TeX\ auf dem eigenen PC behilflich  zu sein.

\section{Die \LaTeX-Folgeb"ande 2 und 3}
\subsection{Inhaltsbeschreibung von Band 2}
Band 2 der Buchserie "uber \LaTeX\ beginnt mit der Vorstellung von 
\LaTeX-Erg"anzungen, die in ihrer Wirkung in Zukunft Bestandteil von
\LaTeX~3 werden und die bereits heute in Form allgemein zug"anglicher
Erg"anzungspakete genutzt werden k"onnen. Die Standardzeichens"atze einer
\LaTeX-Installation waren prim"ar auf Anwendungen aus dem
mathematisch-naturwissenschaft"-lichen Bereich zugeschnitten. 
Mit der Ausbreitung auch auf geisteswissenschaftliche Anwendungen treten
zunehmend Anforderungen auf, die damit nicht zu erf"ullen sind. Inzwischen
existieren f"ur nahezu alle Sprachen und Sonderf"alle, wie z.~B. zur
Schachdokumentation oder zum Musiknotensatz, geeignete Zeichens"atze.
Band 2 stellt eine Vielzahl solcher erg"anzender Zeichens"atze vor, wobei sich
die Erg"anzungen in Richtung \LaTeX~3 als ganz besonders hilfreich erweisen.
Auch die Einbeziehung von PostScript-Zeichens"atzen in die \LaTeX-Bearbeitung
wird angesichts der immer h"aufiger verwendeten PostScript-f"ahigen Drucker
in Kapitel 4 vorgestellt. Die Nutzung von Zeichens"atzen f"ur den Musiknotensatz
zusammen mit einem geeigneten Erg"anzungspaket wird in  Kapitel 3 beschrieben. 

Band 2 stellt weiterhin die M"oglichkeiten zur Einbindung von Bildern und
Grafiken vor, die entweder aus v"ollig anderen Programmquellen stammen oder
mit \TeX-eigenen Mitteln, wie mit PiC\TeX, erzeugt werden.
Er schlie"st ab mit einer Einf"uhrung in METAFONT in Kapitel 7. Angesichts der
Vielzahl der vorgestellten Zusatz-Zeichens"atze ist dies eine folgerichtige
Erg"anzung, da die Installation der Zusatz-Zeichens"atze bei vielen Anwendern
aus den Quelldateien zu erfolgen hat, womit der Programmaufruf von METAFONT
mit geeigneten Einstellparametern zwingend notwendig wird.
Band 2 wendet sich also an Anwender, die "uber die M"oglichkeiten 
einer Standard-\LaTeX-Installation hinausgehen wollen, ohne hierzu
in die Tiefen der Programmierung zur Entwicklung von Eigenerweiterungen
steigen zu m"ussen.

\subsection{Inhaltsbeschreibung von Band 3}
F"ur solche Entwicklungen ist schlie"slich der Band 3 gedacht. Er stellt
\LaTeX\ in seinen internen Strukturen vor, erg"anzt um eine Darstellung
der wichtigsten \TeX-Strukturen. Mit diesen Kenntnissen werden dann
anschlie"send Beispiele f"ur anwendereigene \LaTeX-Erweiterungen vorgestellt.
Ebenso werden Interna des Bib\TeX-Pro"-gramms angesprochen, aus denen
der Anwender weitere \textsc{Bib}\TeX-Stilfiles zur variablen Gestaltung
von Literaturverzeichnissen erstellen kann.

Jede \TeX-Installation kennt weitere  \TeX-Zusatzwerkzeuge, von denen ich hier
beispielhaft das Programm patgen nenne. Mit diesem Programm kann man
f"ur jede Sprache ein \TeX-spezifisches Trennmusterfile erstellen,
indem als Eingabe ein Trennlexikon der entsprechenden Sprache herangezogen
wird. Band 3 stellt in seinem Anhang alle \TeX-Standard-Zusatzwerkzeuge vor und
beschreibt deren Anwendung und Eigenschaften.
\end{document}
