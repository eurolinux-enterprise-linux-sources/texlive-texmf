\documentclass[11pt]{article}
\usepackage{german}
\begin{document}
Der vorliegende Band 1 geht weitgehend auf das urspr"ungliche Buch 
\LaTeX\ -- Eine Einf"uhrung zur"uck und beschr"ankt sich nun auf die
Vorstellung des internationalen \LaTeX-Standards, lediglich erg"anzt
um die Darstellung von german.sty zur Einbindung deutscher 
Besonderheiten in die \LaTeX-Bearbeitung. Letztere mu"s man f"ur 
deutschsprachige Anwender, und an diese richtet sich das Buch, als zu
unserem Standard geh"orend betrachten. 

In bezug auf den internationalen \LaTeX-Standard ist die Vorstellung
in Band 1 aber vollst"andig. Sie schlie"st somit Bib\TeX\
und MakeIndex ein, da diese \LaTeX-Erg"anzungen
vom Programmautor Leslie Lamport selbst stammen oder unter seiner
aktiven Mithilfe entstanden und von ihm zum Bestandteil des \LaTeX-Gesamtpakets 
erkl"art wurden.

Unmittelbar nach Drucklegung der 1. Auflage von Band 1 erschien auf den
"offentlichen \TeX-Fileservern die neue \LaTeXe-Version, zun"achst als
Probeversion und ausdr"ucklich als solche gekennzeichnet. Nach einer
halbj"ahrigen Erprobungsphase mit Behebung einer Reihe von Fehlern und
Schw"achen wurde im Juni 1994 die \LaTeXe-Probeversion zur
Standard-\LaTeX-Version erkl"art. Die neuen Eigenschaften von \LaTeXe wurden
deshalb zun"achst in Kapitel 1 von Band 2 nachgetragen. Mit den Neuauflagen
von Band 1 erscheinen die Grundeigenschaften
von \LaTeXe\ nunmehr sachgerecht in der Einf"uhrung. 

Der Einf"uhrungsband 1 schlie"st ab mit Hinweisen zur \TeX-Installation und der
Erstellung der erforderlichen Formatfiles im Anhang F. 
Bei der Zuf"ugung dieses 46seitigen Anhangs befand ich mich in einem Konflikt:
Er geh"ort im engeren Sinne sicherlich nicht zum Stoffbereich einer 
\LaTeX-Einf"uhrung.
Alle mir bekannten B"ucher "uber \TeX\ und seine Makropakete gehen
stillschweigend von der Annahme aus, da"s ein lauff"ahiges \TeX-Programm
mit den erforderlichen Zusatzwerkzeugen im Rechner des Anwenders existiert.
Dies war in den Anfangsjahren von \TeX\ und \LaTeX\ auch sachgerecht, da
\TeX\ damals zun"achst in den Rechenzentren der Hochschulen und 
Forschungsinstitute bereitgestellt wurde. Alle bei der Installation eines
\TeX-Systems vorausgesetzten Kenntnisse und auftretenden Probleme stellten
sich nicht dem Anwender, sondern dem entsprechenden Experten des
Rechenzentrums.

Inzwischen hat sich das Anwenderprofil deutlich ge"andert. Die Mehrzahl
der \TeX- und \LaTeX-Anwender betreibt das Programm auf einem PC. 
F"ur nahezu alle Individualrechner (IBM-PCs und kompatible, Atari, Amiga,
Macintosh und UNIX-Workstations) stehen sowohl kommerzielle wie auch
kostenlose PD- (Public Domain) oder SW- (Shareware) \TeX-Pakete zur 
Verf"ugung. Dokumentation und Installationshilfen sind je nach Programmquelle
unterschiedlich hilfreich. 

Der typische \TeX-Einsteiger will das Programm auf dem eigenen 
Individualrechner betreiben und die eigenen Kenntnisse auf die 
Nutzungsbed"urfnisse beschr"anken und nicht mit komplexen Wechselbeziehungen
zwischen den diversen Programmteilen einer \TeX- und \LaTeX-Installation
konfrontiert werden. Das Programmsystem wurde vermutlich als Diskettensatz
beschafft oder von einem Kollegen kopiert, was bei den PD- und SW-Produkten
auch erlaubt ist. Je nach beigef"ugter Dokumentation und Installationshilfe
kann sich die Installation f"ur den Einsteiger als schwierige H"urde erweisen.
Auch wenn die eigentliche Installation ohne Probleme gelingt, meldet das
Programm eventuell beim ersten Aufruf, da"s es gewisse Teile nicht findet und
damit die Bearbeitung abbricht.          

Die Ursache f"ur eine solche Meldung kann tats"achlich darin liegen, da"s
der Diskettensatz f"ur die Installation zwar das ausf"uhrbare \TeX-Programm
bereitstellt, die f"ur den praktischen Ablauf aber zwingend erforderlichen
Zusatzwerkzeuge, wie bestimmte Zeichensatzfiles und Makropakete, aus anderen
Quellen beschafft werden m"ussen, ohne da"s dies in der beigef"ugten
Dokumentation explizit gesagt wird. H"aufig liegt die Ursache f"ur die
genannte Fehlermeldung aber darin, da"s bestimmte Programmteile zwar vorhanden
sind, aber f"ur die Nutzung mit dem beigef"ugten Spezialprogramm INITEX, 
von dem der Einsteiger bis dahin noch nie etwas geh"ort hat, aufbereitet werden
m"ussen.  Ich hoffe, mit dem Anhang F auch dem Anf"anger 
bei seinem Einstieg in \TeX\ auf dem eigenen PC behilflich  zu sein.
\end{document}
