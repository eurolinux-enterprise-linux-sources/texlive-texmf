\documentclass{article}
\usepackage{german}
\usepackage[ansinew]{inputenc}
\begin{document}
Der vorliegende Band 1 geht weitgehend auf das urspr�ngliche Buch
\LaTeX\ -- Eine Einf�hrung zur�ck und beschr�nkt sich nun auf die
Vorstellung des internationalen \LaTeX-Standards, lediglich erg�nzt
um die Darstellung von german.sty zur Einbindung deutscher
Besonderheiten in die \LaTeX-Bearbeitung. Letztere mu� man f�r
deutschsprachige Anwender, und an diese richtet sich das Buch, als zu
unserem Standard geh�rend betrachten.

In bezug auf den internationalen \LaTeX-Standard ist die Vorstellung
in Band 1 aber vollst�ndig. Sie schlie�t somit Bib\TeX\
und MakeIndex ein, da diese \LaTeX-Erg�nzungen
vom Programmautor Leslie Lamport selbst stammen oder unter seiner
aktiven Mithilfe entstanden und von ihm zum Bestandteil des \LaTeX-Gesamtpakets
erkl�rt wurden.

Unmittelbar nach Drucklegung der 1. Auflage von Band 1 erschien auf den
�ffentlichen \TeX-Fileservern die neue \LaTeXe-Version, zun�chst als
Probeversion und ausdr�cklich als solche gekennzeichnet. Nach einer
halbj�hrigen Erprobungsphase mit Behebung einer Reihe von Fehlern und
Schw�chen wurde im Juni 1994 die \LaTeXe-Probeversion zur
Standard-\LaTeX-Version erkl�rt. Die neuen Eigenschaften von \LaTeXe wurden
deshalb zun�chst in Kapitel 1 von Band 2 nachgetragen. Mit den Neuauflagen
von Band 1 erscheinen die Grundeigenschaften
von \LaTeXe\ nunmehr sachgerecht in der Einf�hrung.

Der Einf�hrungsband 1 schlie�t ab mit Hinweisen zur \TeX-Installation und der
Erstellung der erforderlichen Formatfiles im Anhang F.
Bei der Zuf�gung dieses 46seitigen Anhangs befand ich mich in einem Konflikt:
Er geh�rt im engeren Sinne sicherlich nicht zum Stoffbereich einer
\LaTeX-Einf�hrung.
Alle mir bekannten B�cher �ber \TeX\ und seine Makropakete gehen
stillschweigend von der Annahme aus, da� ein lauff�higes \TeX-Programm
mit den erforderlichen Zusatzwerkzeugen im Rechner des Anwenders existiert.
Dies war in den Anfangsjahren von \TeX\ und \LaTeX\ auch sachgerecht, da
\TeX\ damals zun�chst in den Rechenzentren der Hochschulen und
Forschungsinstitute bereitgestellt wurde. Alle bei der Installation eines
\TeX-Systems vorausgesetzten Kenntnisse und auftretenden Probleme stellten
sich nicht dem Anwender, sondern dem entsprechenden Experten des
Rechenzentrums.

Inzwischen hat sich das Anwenderprofil deutlich ge�ndert. Die Mehrzahl
der \TeX- und \LaTeX-Anwender betreibt das Programm auf einem PC.
F�r nahezu alle Individualrechner (IBM-PCs und kompatible, Atari, Amiga,
Macintosh und UNIX-Workstations) stehen sowohl kommerzielle wie auch
kostenlose PD- (Public Domain) oder SW- (Shareware) \TeX-Pakete zur
Verf�gung. Dokumentation und Installationshilfen sind je nach Programmquelle
unterschiedlich hilfreich.
\end{document}
