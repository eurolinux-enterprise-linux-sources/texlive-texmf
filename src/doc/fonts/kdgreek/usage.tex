% =====================================================================
% ==                   Greek Fonts & Format                          ==
% ==             An example of the use of KD fonts & macros          ==
% ==             in composing papers containing greek text           ==
% ==                                                                 ==
% ==  (C) Copyright 1991 K J Dryllerakis                             ==
% ==                                                                 ==
% ==  Last Revision : Sep 23, 1991                                   ==
% =====================================================================
%
\input greektex		%   This is superfluous if greek format is used
%	  ^ Use Version 3.0[b] or higher
%
% =====================================================================
% ==            Font Declarations                                    ==
% =====================================================================
%
\catcode`@=11 % Access Plain TeX and its Macros
%
% The following point-setting macros are based on D Knuth's TeXBook
%
% Load standard fonts and their specifications
%
\font\ninerm=cmr9  \font\eightrm=cmr8  \font\sixrm=cmr6
\font\ninei=cmmi9  \font\eighti=cmmi8  \font\sixi=cmmi6
\font\ninesy=cmsy9 \font\eightsy=cmsy8 \font\sixsy=cmsy6
\font\ninebf=cmbx9 \font\eightbf=cmbx8 \font\sixbf=cmbx6
\font\ninett=cmtt9 \font\eighttt=cmtt8 
\font\nineit=cmti9 \font\eightit=cmti8
\font\ninesl=cmsl9 \font\eightsl=cmsl8
\skewchar\ninei='177  \skewchar\eighti='177  \skewchar\sixi='177
\skewchar\ninesy='60  \skewchar\eightsy='60  \skewchar\sixsy='60
\hyphenchar\ninett=-1 \hyphenchar\eighttt=-1 \hyphenchar\tentt=-1
\font\csc=cmcsc10
\font\titlefont=cmbx10 scaled\magstep1
%
%	Define Font Point Families
%
\newskip\ttglue					% for listings
\def\tenpoint{\def\rm{\fam0\tenrm}%
 \gdef\t@size{\grtenpoint} % We want greek to follow our conventions
 \textfont0=\tenrm  \scriptfont0=\sevenrm  \scriptscriptfont0=\fiverm
 \textfont1=\teni   \scriptfont1=\seveni   \scriptscriptfont1=\fivei
 \textfont2=\tensy  \scriptfont2=\sevensy  \scriptscriptfont2=\fivesy
 \textfont3=\tenex  \scriptfont3=\tenex    \scriptscriptfont3=\tenex
 \textfont\itfam=\tenit  \def\it{\fam\itfam\tenit}
 \textfont\slfam=\tensl  \def\sl{\fam\slfam\tensl}
 \textfont\ttfam=\tentt  \def\tt{\fam\ttfam\tentt}
 \textfont\bffam=\tenbf  \def\bf{\fam\bffam\tenbf}
  \scriptfont\bffam=\sevenbf \scriptscriptfont\bffam=\fivebf
 \tt \ttglue=.5em plus .25em minus.15em
 \normalbaselineskip=.6cm
 \setbox\strutbox=\hbox{\vrule height8.5pt width0pt depth4.5pt}%
 \let\sc=\eightrm \normalbaselines\rm}
%
\def\ninepoint{\def\rm{\fam0\ninerm}%
 \gdef\t@size{\grninepoint} % We want greek to follow our conventions
 \textfont0=\ninerm  \scriptfont0=\sixrm  \scriptscriptfont0=\fiverm
 \textfont1=\ninei   \scriptfont1=\sixi   \scriptscriptfont1=\fivei
 \textfont2=\ninesy  \scriptfont2=\sixsy  \scriptscriptfont2=\fivesy
 \textfont3=\tenex  \scriptfont3=\tenex    \scriptscriptfont3=\tenex
 \textfont\itfam=\nineit  \def\it{\fam\itfam\nineit}
 \textfont\slfam=\ninesl  \def\sl{\fam\slfam\ninesl}
 \textfont\ttfam=\ninett  \def\tt{\fam\ttfam\ninett}
 \textfont\bffam=\ninebf  \def\bf{\fam\bffam\ninebf}
  \scriptfont\bffam=\sixbf \scriptscriptfont\bffam=\fivebf
 \tt \ttglue=.5em plus .25em minus.15em
 \normalbaselineskip=.52cm
 \setbox\strutbox=\hbox{\vrule height8pt width0pt depth3pt}%
 \let\sc=\sevenrm \normalbaselines\rm}
%
\def\eightpoint{\def\rm{\fam0\eightrm}%
 \gdef\t@size{\greightpoint} % We want greek to follow our conventions
 \textfont0=\eightrm  \scriptfont0=\sixrm  \scriptscriptfont0=\fiverm
 \textfont1=\eighti   \scriptfont1=\sixi   \scriptscriptfont1=\fivei
 \textfont2=\eightsy  \scriptfont2=\sixsy  \scriptscriptfont2=\fivesy
 \textfont3=\tenex  \scriptfont3=\tenex    \scriptscriptfont3=\tenex
 \textfont\itfam=\eightit  \def\it{\fam\itfam\eightit}
 \textfont\slfam=\eightsl  \def\sl{\fam\slfam\eightsl}
 \textfont\ttfam=\eighttt  \def\tt{\fam\ttfam\eighttt}
 \textfont\bffam=\eightbf  \def\bf{\fam\bffam\eightbf}
  \scriptfont\bffam=\sixbf \scriptscriptfont\bffam=\fivebf
 \tt \ttglue=.5em plus .25em minus.15em
 \normalbaselineskip=.35cm
 \setbox\strutbox=\hbox{\vrule height7pt width0pt depth2pt}%
 \let\sc=\sixrm \normalbaselines\rm}
%
% =====================================================================
% ==            Format (Page) Definitions                            ==
% ==		New Output routine				     ==
% =====================================================================
\newdimen\pagewidth \newdimen\pageheight \newdimen\ruleht
\hsize=16.5truecm \vsize=23truecm \maxdepth=2.5pt
\parindent=10pt \parskip=0pt
\pagewidth=\hsize \pageheight=\vsize \ruleht=1pt
\abovedisplayskip=6pt plus 3pt minus 1pt
\belowdisplayskip=6pt plus 3pt minus 1pt
\abovedisplayshortskip=0pt plus 3pt
\belowdisplayshortskip=4pt plus 3pt
\hfuzz=1pt	% Do not make fuss for over 1pt of overfull hbox!
\overfullrule=0pt % and certainly don't show these boxes to us!
%
%	Define New Insert for Footnotes
%
%
\def\footnote#1{\edef\@sf{\spacefactor\the\spacefactor}#1\@sf
	\insert\footins\bgroup\eightpoint
	\interlinepenalty100 \let\par=\endgraf
	\leftskip=0pt \rightskip=0pt
	\splittopskip=10pt plus 1pt minus 1pt \floatingpenalty=20000
	\smallskip\textindent{#1}\bgroup\strut\aftergroup\@foot\let\next}
\skip\footins=12pt plus 2pt minus 4pt 	% space added when footnote exists
\dimen\footins=30pc  			% maximum footnotes per page
\def\footnt{\advance\footno by1\footnote{$^{\number\footno}$}}
%
%
\newif\ifnopagenums\nopagenumsfalse	% Shall we include pagenumbers
\def\nopagenumbers{\nopagenumstrue}
\def\pagenumbers{\global\nopagenumsfalse}
%
%	Headlines
%
\def\rhead{}	% Running Head
\def\leftheadline{\hbox to\pagewidth{%
	\vbox to 10pt{}% Strut to position the baseline
	\tenit\rhead\hfil%
	\ifnopagenums\else\rlap{\kern0.1pc\tenbf\folio}\fi}}% Running Head left
\def\rightheadline{\hbox to\pagewidth{%
	\vbox to 10pt{}% Strut to position the baseline
	\tenit\rhead\hfil%
	\ifnopagenums\else\rlap{\kern0.1pc\tenbf\folio}\fi}}%Running Head right
%
% 	Output routine
%
\def\onepageout#1{\shipout\vbox{
	\offinterlineskip
	\vbox to 3pc{
		\ifnum\pageno>1
			\ifodd\pageno\rightheadline\else\leftheadline\fi
		\fi
	\vfill}
	\vbox to \pageheight{
		#1
		\ifvoid\footins\else
			\vskip\skip\footins \kern-3pt
			\hrule height\ruleht width3cm \kern-\ruleht \kern3pt
			\unvbox\footins
		\fi
		\boxmaxdepth=\maxdepth}}
	\advancepageno}
\output{\onepageout{\unvbox255}}
\newcount\footno\footno=0		% Footnote Number
\def\section#1{%
	\medbreak\bigskip
	\centerline{{\bf #1}}\nobreak
	\bigskip}
%
\def\titleline#1{\line{\hss\titlefont #1\hss}}
\def\beginFine{\par\bgroup\eightpoint}
\def\endFine{\egroup\par}
%
%	Verbatim Listings
%
\def\uncatcodespecials{\def\do##1{\catcode`##1=12}\dospecials}
\def\ttverbatim{\begingroup
\uncatcodespecials\obeyspaces\obeylines\tt}
{\obeyspaces\gdef {\ }}
\outer\def\begintt{\let\par=\endgrapf \ttverbatim\parskip=0pt
                    \ttfinish}

{\catcode`\|=0 |catcode`|\=12
|obeylines
|gdef|ttfinish#1^^M#2\endtt{#1|vbox{#2}|endgroup}}
\catcode`\|=\active % When not in a greek group
{\obeylines\gdef|{\ttverbatim\spaceskip=\ttglue\let^^M=\ \let|=\endgroup}}
% =====================================================================
% ==            Miscellaneous Definitions                            ==
% =====================================================================
\def\today{\ifcase\month\or
January\or February\or March\or April\or May\or June\or July\or
August \or September\or October\or November\or December\fi
\space\number\day, \number\year}

\def\boxit#1{\vbox{\hrule\hbox{\vrule\kern3pt\vbox{\kern3pt#1\kern3pt}%
\kern3pt\vrule}\hrule}}
\def\topboxit#1{\vtop{\hrule\hbox{\vrule\kern3pt\vbox{\kern3pt#1\kern3pt}%
\kern3pt\vrule}\hrule}}
\catcode`@=12
% New defs
\def\LaTeX{{\rm L\kern-.36em\raise.3ex\hbox{\csc a}\kern-.15em
    T\kern-.1667em\lower.7ex\hbox{E}\kern-.125emX}}
\def\greektex{{\csc GreeK}\TeX}%
\chardef\|`\|
%
\newdimen\lefthsize
\newdimen\exdepth
\newcount\testcount
\newbox\exbox
%
% Macro For typesetting examples
%
\def\exparagraph{%
\hangindent\lefthsize\testcount=0%
\loop \advance\testcount by1\ifdim\exdepth>\testcount\baselineskip \repeat
\advance\testcount by-1
\hangafter-\testcount
\rlap{\hbox to\hangindent{\hss\vbox to0pt{\box\exbox\vss}\hss}}
\ignorespaces}

\def\contparagraph#1{%
\hangindent\lefthsize
\hangafter-#1\ignorespaces}
%
% =====================================================================
%
\greekdelims{dollar}	% keep bar for ttverbatim
\vskip2truecm
\titleline{Typesetting Greek Texts with \greektex}
\bigskip
\centerline{\csc K J Dryllerakis}
\bigskip
\centerline{Imperial College}
\centerline{Department of Computing}
\centerline{London}
\centerline{(|kd@doc.ic.ac.uk|)}
\vskip1truecm

	This document describes \greektex, yet another something-\TeX\
application, a package for typesetting greek texts. The following
pages are intended to be a quick tutorial on the package and not a
detailed account of the package. It is assumed that the \greektex\ 
package is already installed at you site and both the greek format
file (|greek.fmt|) and the alternative \greektex\ macros
(|greektex.tex|)  are accesible to the user. We also assume that all
the recomended |kd| fonts exist in the font path. If this is not the
case please consult the installation procedure document which gives
all the information needed to install the package in your site. In
the discussion to follow, more \TeX nical material is presented in a
finer print.

\section{The Greek Mode And Fonts}

	In order to typeset greek text, you have to advise \TeX\ to
enter a ``Greek Mode''\footnt{The idea of a ``greek environment'' was first
introduced by Sylvio Levi in his greek fonts; the top level macros of
our package are inherited from this first approach.}. Entering the
greek mode is achieved by typing |\begingreek| at the position were
we want the greek text to begin. The greek mode will be active until
an |\endgreek| is typed, when we return to the previous state. Inside
the greek mode, the latin characters typed correspond to the greek
alphabet and diacritic marcs. The family changing commands (|\sl|,
|\bf|, |\tt|, |\it|) are automatically switched to correspond to the
proper greek font families and the classical greek font corresponds
to the |\gr| or |\rg| command. Roman letters are still available
through the |\rm| and |\l|{\it oldsequence\/} commands (e.g. |\lsl|).
When we re-enter the greek mode, the font used last is remembered and
typesetting continues from the state we left of in the previous greek
mode.
\beginFine
	\TeX nically, entering the greek mode means {\it i\/}. to redifine the
|\catcode| values for several characters (mostly accents and
breathing symbols), {\it ii\/}. to declare that we are currently
typesetting in greek and use the correct hyphenation tables, {\it
iii\/}. switch to the point size and font used last, and {\it iv\/}.
redifine the control sequences corresponding to font family switching
(e.g. |\sl| etc.).
\endFine
The font families that are currently avaible are all based on the
original designs by Sylvio Levi and Haralambous. The families (and
the corresponding font switching commands) are: classical greek
(|\gr| or |\rg|), slanted greek (|\sl|), greek italics (|\it|),
boldface greek (|\bf|) and typewriter greek (|\tt|). Remember that
these families are only available inside greek mode.
\beginFine
	A set of point-setting commands are made available from the
greek format. Three sizes are prespesified for ten, nine, and eight
points respectevly. The corresponding commands are |\grtenpoint|,
|\grninepoint|, and |\greightpoint|. Since not all fonts exist in the
sizes need, scaled fonts are used instead. Remeber that re-entry in
the greek mode means picking up from exactly where we previously left
i.e. from the same point size. If point-size changes are to remain
local the user is encouraged to enclose them in groups.

	The real names for the control sequences switching to greek
font families are |\gr|, |\git|, |\gsl|, |\gbf|, |gtt|. These
commands  can be used even outside greek mode to typeset greek short
texts but it is not recomended. Note that the \LaTeX\ version of
\greektex\ (the greek style file |greek.sty|) uses only these font
changing commands.
\endFine
Since the commands |\begingreek| and |\endgreek| can be quite tedious
to repeat, they can be both substituted either by the character |$| or
{\tt\|} if at the beginning of your file you specify
|\greekdelims{dollar}| or |\greekdelims{bar}| repsectively. In case
the |$| character is used, math mode is accessed by the control
sequence |\math|.

\section{The Greek Alphabet}

	Since \TeX\ was designed to help portability of files, most
implemetations support only 128 character input (file transfer is
also based on the 128 ascii set). That means we have to use the Latin
alphanumeric characters to represent the greek ones. The following
scheme is used to represent greek characters inside the greek mode:
\math\math
\hbox{\valign{
  \hbox to 10pt{\hfil\strut$#$\hfil}&\hbox to 10pt{\hfil\strut\tt#\hfil}\cr
  a&a\cr b&b\cr g&g\cr d&d\cr e&e\cr z&z\cr h&h\cr j&j\cr
  i&i\cr k&k\cr l&l\cr m&m\cr n&n\cr x&x\cr o&o\cr p&p\cr
  r&r\cr c&s\cr t&t\cr u&u\cr f&f\cr q&q\cr y&y\cr w&w\cr
  c&c\cr}}
\math\math
\beginFine
	It is a common practise in Greece to substitue the characters
corresponding to the higher
ascii codes by greek characters so that users are able to see the
greek characters directly on the screen. It is fairly easy to write a
program to transform any such ascii output to the corresponding
``latin'' representation of the greek characters and feed it directly
to \TeX 's mouth. For example such a translation program is supplied
with the current distribution of the em\TeX\ package for DOS based
machines.
\endFine

\section{Accents, Breathing And Punctuation}

	Three different accents are encountered in greek texts. In
order to accent a vowel simply type |'|, |`|, or |~| to get an acute,
grave or circumflex accent repsectively. Breathing signs are achieved
in the same way by preceding the vowel (and any accent that it may
have) with |<| for rough and |>| for
smooth breathing. Breathings can also be used before a greek rho
($r$). Iota subscript is achieved by typing {\tt \|} {\it after\/}
vowel. The diairesis sign is achieved by presiding a (maybe accented) 
vowel by |"|. Greek punctutation marcs are achieved according to the
following table:
\math\math
\hbox{\valign
  {\hbox to 15pt{\hfil\strut$#$\hfil}&\hbox to 15pt{\hfil\strut\tt#\hfil}\cr
  .&.\cr ,&,\cr ;&;\cr :&:\cr !&!\cr ?&?\cr ''&''\cr ((&((\cr ))&))\cr}}
\math\math
For example, concider the following quotation from Xenophon :
\bigskip
{\narrower\narrower\noindent$>En o>udem'ia g'ar p'olei t`o b`eltiston e>'unoun >est`i t~w|
d~hmw|, >all`a t`o k'akiston >en <ek'asth| >est`i p'olei e>'unoun
t~w| d~hmw|; o<i g'ar <'omoioi to~ic <omo'ioic e>'unoi e>isi.$\par}
\bigskip
\noindent was typeset by \par
\medskip
\math\math
\vbox{\advance\hsize by-40pt\parindent=0pt\noindent
\obeylines
|>En o>udem'ia g'ar p'olei t`o b`eltiston e>'unoun >est`i t~w|{\tt\|}
|d~hmw|{\tt\|}|, >all`a t`o k'akiston >en <ek'asth|{\tt\|}| >est`i|
|p'olei e>'unoun t~w|{\tt\|}| d~hmw|{\tt\|}|; o<i g'ar <'omoioi to~ic|
|<omo'ioic e>'unoi e>isi.|
}
\math\math
\noindent Note the use of punctuation and breathing marks before the
vowels.
\beginFine
All accents and breathings are recognised as normal letters insode
the greek mode. They will print for themselves when they stand alone
but form ligatures when preciding a vowel. The ligatures do all the
work for us; the kerning between accents or breathing and capital
vowels is also done through kerning. This simplifies the macros used
and the greek character set is usable with only a few |\catcode|
definitions. All the rest of the mechanisms supplied in the greek
format are only to facilitate the usage of the fonts in standard text 
and macro creation.
\endFine

\section{Hyphenation}

	If you are working with the greek format file, then correct
hyphenation is assured for greek text. The hyphenation patterns
supplied in the file are the ones compiled by Haralambous in France.
If you are working with the file |greektex.tex| then no hyphenation
patterns are loaded; this fact makes the macros of |greektex.tex|
suitable for only small greek texts. \greektex, takes full advantage
of the multilingual characteristics of \TeX\ version 3.0 or higher by
declaring a new language and identifying the correct patterns for this
new language. The result is correct hyphenation for texts containing
both greek and latin texts.
\beginFine
In case you want to use \greektex with another multilingual package,
it is important to find out the limitation of your \TeX version. It
is possible that the number of allowed hyphenation patterns exceeds
the number needed. Remember that plain \TeX\ uses 4447 patterns and
\greektex\ needs a futher 1170.

It is possible to see the proposed hyphenation for a word using a
macro similar to |\showhyphens| (supplied with plain \TeX). The macro
|\showgreekhyphens| can be used succesfully only inside greek mode.
This is due to the fact that the argument for the macro has to be
scanned with the correct |\catcode| values active.
\endFine
\bigskip
%
%%%% TEST

\section{Building Macros}

	Let us now turn to some examples that will demonstrate the
use of the \greektex\ macros within your own definitions.

%% Define macros for the current section

\def\verse#1#2{\line{#1\hfil#2}}


\setbox\exbox=\vtop{\hrule height0pt
\hrule\hbox{\vrule\kern6pt\vbox{\kern6pt%
    \vbox{\hsize=7truecm
	$\greightpoint
	\centerline{\bf PALAIOJEN ELLHNIS}
	\smallskip
	\verse{Kauqi'etai <h >Anti'oqeia}{gi`a t`a lampr'a thc kt'iria,}
	\verse{ka`i to'uc <wra'iouc thc dr'omouc;}{gi`a t`hn per`i a>ut'hn}
	\verse{jaum'asian >exoq'hn,}{ka`i gi`a t`o m'ega pl~hjoc}
	\verse{t~wn >en a>ut~h| kato'ikwn.}{Kauqi'etai po`u e>~in'' <h <'edra}
	\verse{>end'oxwn basil'ewn;}{ka`i gi`a to`uc kallit'eqnac}
	\verse{ka`i to`uc sofo'uc po`u >'eqei,}{ka`i gi`a to`uc bajuplo'utouc}
	\verse{ka`i gnwstiko`uc >emp'orouc.}{M`a pi`o pol`u >asugkr'itwc}
	\verse{>ap'' <'ola, <h >Anti'oqeia}{kauqi'etai po`u e>~inai p'olic}
	\verse{palai'ojen <ellhn'ic;}{to~u >'Argouc suggen'hc:}
	\verse{>ap'' t`hn >I'wnh po`u}{<idr'ujh <up`o >Arge'iwn}
	\verse{>apo'ikwn pr`oc tim`hn}{t~hc k'orhc to~u >In'aqou.}
	\bigskip
	\leftline{K. Kab'afhc}$
     }
 \kern6pt}\kern6pt\vrule}\hrule}
\lefthsize=\wd\exbox\advance\lefthsize by20pt%
\exdepth=\dp\exbox\advance\exdepth by\ht\exbox
%% End of example macros

\exparagraph Let us try to typeset this poem written by a greek poet
of the 20th century. The peculiar format is the original one required 
by the poet. It is immediatly apparent that a command taking two
arguments can be used to typeset each verse; the arguments will be
the left and right part of each verse. The only important problem
encountered is the following: when \TeX\ is reading the
arguments of a command, it immedietly assigns |\catcode| values to
the tokens read. So, if the correct |\catcode| values are not active
at the time the arguments are scanned, \TeX\ is much confused.
Consequently, macros that take as arguments greek text must be called
from within greek mode. Having that in mind we can define a macro
|\verse| that will correctly typeset one verse of the poem. The macro
is defined as follows:
\smallskip
\hbox to\hsize{\hss|\def\verse#1#2{\line{#1\hfill#2}}|\hss}
\smallskip
\noindent It is assumed of course that the |\hsize| required
for the document is already set. The rest is easy.
\math\math
\vbox{\narrower\narrower\parindent=0pt\obeylines
	|$|
	|\centerline{\bf PALAIOJEN ELLHNIS}|
	|\smallskip|
	|\verse{Kauqi'etai <h >Anti'oqeia}{gi`a t`a lampr'a thc kt'iria,}|
	\centerline{\vdots}
	|\verse{>apo'ikwn pr`oc tim`hn}{t~hc k'orhc to~u >In'aqou.}|
	|\bigskip|
	|\leftline{K. Kab'afhc}|
	|$|}
\math\math

%% Begin Example macros
{\setlatin % Switch to normal TeX Mode (Deactivate both | and $)
\greekdelims{bar}
\global\setbox\exbox=\vtop{\hrule height0pt
\hrule\hbox{\vrule\kern6pt\vbox{\kern6pt%
    \vbox{\hsize=9.6truecm\eightpoint
	|\greightpoint\gr
	\hbox to\hsize{\hfil {\tengri 7.2 T'o s'usthma K'entrou %
M'azac 172}}
	\bigskip
	\noindent {\bf 7.2 To s'usthma k'entrou M'azac}
	\medskip
	E'inai suqn'a bolik'o na perigr'afetai h k'inhsh tou sust'hmatoc se
	pla'isio anafor'ac sto opo'io to k'entro m'azac hreme'i sthn arq'h twn
	ax'onwn. (Se barutik'o ped'io, to s'usthma aut'o e'inai epitagqun'omeno, mh
	adraneiak'o, par'ola aut'a e'inai p'ali qr'hsimo.) Kale'itai to s'usthma
	{\it k'entrou m'azac} (KM). Ja sumbol'izoume ta meg'ejh pou andaf'erontai
	s''aut'o me 'enan aste\-r'i\-sko.

	H sqetik'h j'esh $r$  e'inai b'ebaia anex'arthth ap'o to p'wc 
	epil'egetai h arq'h, 'etsi 'wste j'etontac $R^*=0$  st'hn 
	(7.7) br'iskoume 
	$${\lbf r}^*_1={m_2\over M}{\lbf r},\qquad {\lbf r}^*_2=-{m_1\over
	M}{\lbf r}\eqno(7.12)$$
	St'o s'usthma aut'o, oi orm'ec twn d'uo swmat'iwn e'inai 
	('isec kai) ant'ijetec,
	$$m_ir^*_i = -m_2r^*_2=\mu r = p^*\eqno(7.13)$$
	'Opwc ja do'ume kajar'a arg'otera, e'inai suqn'a bolik'o 
	na epil'uoume ena pr'oblhma pr'wta sto s'usthma KM. %Gia na 
%	bro'ume thn l'ush se k'apoio 'allo
%	s'usthma, qreiaz'omaste t'ote tic sq'eseic an'amesa 
%	stic orm'ec sta d'uo
%	sust'hmata. Ac jewr'hsoume 'ena s'usthma sto opo'io to k'entro m'azac
%	kine'itai me taq'uthta ${\lbf R}$. T'ote oi taq'uthtec twn d'uo
%	swmatid'iwn e'inai: 
	|
	\smallskip
	\hbox to\hsize{\hss\vdots\hss}
     }
 \kern6pt}\kern6pt\vrule}\hrule}
}
\lefthsize=\wd\exbox\advance\lefthsize by20pt%
\exdepth=\dp\exbox\advance\exdepth by\ht\exbox
%% End of example macros
\exparagraph \tolerance=2000 Let us now turn to our second example which will demonstrate
scientific typesetting in greek. The example is taken from the book
{\it Classical Mechanics\/} by T.W.B. Kibble in its greek
translation. Let us suppose that we are trying to typeset this book;
how would we go about to build the macros needed? First of all we
will be using a lot of math formulae and therefore we are in
desperate need of the start and end math symbol. Since we are using
modern greek and no iota ligatures are needed, we can redifine the
bar {\tt\|} to stand for the begin and end greek by
|\greekdelims{bar}|. This might turn out to be convenient if we
want to include latin words in the text. Now for the macros.
Starting with the
headline appearing at the top of the page we need to define the
|\headline| macro. This must be done within greek mode to correctly
interprent the tokens scanned:
\math\math
\vbox{\parindent=0pt\obeylines
|\begingreek|
|\global\headline={\hfil {\git T'o s'usthma K'entrou M'azac}\ \folio}|
|\endgreek|}
\math\math
It might be wiser to change the |\git| command either to 
|\grtenpoint\git| or |\tengri| to ensure that the correct font is
selected when the output routine is called.

To typeset math formulae you only have to use the standard \TeX\
rules. Even within greek mode entering the math mode has exactly the
same effect as from the plain \TeX\ mode. This is not entirely true
since the font changing sequences are redefined. If you want boldface
characters within the math mode you have to use the |\lbf| command
and the same for any other family. See how is it really is:
\math\math
\vbox{\eightpoint\parindent=0pt\obeylines
	|\hsize=12truecm|
	|\begingreek|
	|\grtenpoint	% Switch to ten point characters|
	|\gr		% Default font|
	|\noindent {\bf 7.2 To s'usthma k'entrou M'azac}|
	|\medskip|
	|E'inai suqn'a bolik'o na perigr'afetai h k'inhsh tou sust'hmatoc se|
	|pla'isio anafor'ac sto opo'io to k'entro m'azac hreme'i sthn arq'h twn|
	|ax'onwn. (Se barutik'o ped'io, to s'usthma aut'o e'inai epitagqun'omeno, mh|
	|adraneiak'o, par'ola aut'a e'inai p'ali qr'hsimo.) Kale'itai to s'usthma|
	|{\it k'entrou m'azac} (KM). Ja sumbol'izoume ta meg'ejh pou andaf'erontai|
	|s''aut'o me 'enan aster'isko.|

	|H sqetik'h j'esh $r$  e'inai b'ebaia anex'arthth ap'o to p'wc|
	|epil'egetai h arq'h, 'etsi 'wste j'etontac $R^*=0$  st'hn|
	|(7.7) br'iskoume|
	|$${\lbf r}^*_1={m_2\over M}{\lbf r},|
	|       \qquad {\lbf r}^*_2=-{m_1\over M}{\lbf r}\eqno(7.12)$$|
	|St'o s'usthma aut'o, oi orm'ec twn d'uo swmat'iwn e'inai |
	|('isec kai) ant'ijetec,|
	|$$m_ir^*_i = -m_2r^*_2=\mu r = p^*\eqno(7.13)$$|
	|'Opwc ja do'ume kajar'a arg'otera, e'inai suqn'a bolik'o |
	|na epil'uoume ena pr'oblhma pr'wta sto s'usthma KM.|
	|\endgreek|
}
\math\math

\section{Final Remarks}

	Except from the macros explained above, the greek format also
has a set ofgeneral use macros. It is possinble that in certain
cases, greek mode will be the dominant mode. To switch to a permenant
greek mode use the command |\setgreek|. If you want to restore the
original \TeX\ mode (including restoring any reassignents of |$| or 
{\tt\|}) use the command |\setlatin|. 
\beginFine
	These commands are based on the internal commands
|\gr@@km@de| and |\l@tinmode|. Both define the |\catcode|s for the
mode. A macro-builder can use them to create different environments
e.g. a command to switch to english from a global greek mode. Just
remeber that you will be on your own so you will have to switch to
the enclish language manually by a command |\language0|.
\endFine

\bye

