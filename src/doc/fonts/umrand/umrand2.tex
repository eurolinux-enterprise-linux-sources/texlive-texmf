\newpage\setcounter{page}{1}
\parindent=0pt \parskip=0pt
\font\G=umrandb at 20pt
\font\F=umranda at 20pt
\fussy
\Thema{Neue Umrandungen}
\Nr{10}
\def\C{\char'}

Angeregt vom letzten Stammtisch habe ich einen weiteren Font mit Rahmen
erstellt. Da das {\tt .mf}--File f"ur eine Vorstellung zu lang ist,
gebe ich hier nur die m"oglichen Rahmen zum Besten.

Zun"achst wird definiert:
\begin{verbatim}
\def\C{\char'} % Oktaldarstellung
\font\G = umrandb at 20pt
\end{verbatim}
Und nun kommen die Rahmen, zun"achst jene mit 6 Linien:\vskip 2mm

\begin{bsp}
\RandBox {A}
         font {\G} [0pt]
         (\C5) ([\C1]) (\C4)
         ([\C2])        ([\C0])
         (\C6) ([\C3]) (\C7)
\end{bsp}
\make

\begin{bsp}
\RandBox {B}
         font {\G} [0pt]
         (\C11) ([\C1]) (\C10)
         ([\C2])        ([\C0])
         (\C12) ([\C3]) (\C13)
\end{bsp}
\make

\begin{bsp}
\RandBox {C}
         font {\G} [0pt]
         (\C15) ([\C1]) (\C14)
         ([\C2])        ([\C0])
         (\C16) ([\C3]) (\C17)
\end{bsp}
\make

\begin{bsp}
\RandBox {D}
         font {\G} [0pt]
         (\C21) ([\C1]) (\C20)
         ([\C2])        ([\C0])
         (\C22) ([\C3]) (\C23)
\end{bsp}
\make

\begin{bsp}
\RandBox {E}
         font {\G} [0pt]
         (\C25) ([\C1]) (\C24)
         ([\C2])        ([\C0])
         (\C26) ([\C3]) (\C27)
\end{bsp}
\make

\begin{bsp}
\RandBox {F}
         font {\G} [0pt]
         (\C31) ([\C1]) (\C30)
         ([\C2])        ([\C0])
         (\C32) ([\C3]) (\C33)
\end{bsp}
\make

\begin{bsp}
\RandBox {G}
         font {\G} [0pt]
         (\C35) ([\C1]) (\C34)
         ([\C2])        ([\C0])
         (\C36) ([\C3]) (\C37)
\end{bsp}
\make

\begin{bsp}
\RandBox {H}
         font {\G} [0pt]
         (\C41) ([\C1]) (\C40)
         ([\C2])        ([\C0])
         (\C42) ([\C3]) (\C43)
\end{bsp}
\make

\begin{bsp}
\RandBox {I}
         font {\G} [0pt]
         (\C45) ([\C1]) (\C44)
         ([\C2])        ([\C0])
         (\C46) ([\C3]) (\C47)
\end{bsp}
\make

\begin{bsp}
\RandBox {J}
         font {\G} [0pt]
         (\C51) ([\C1]) (\C50)
         ([\C2])        ([\C0])
         (\C52) ([\C3]) (\C53)
\end{bsp}
\make

\begin{bsp}
\RandBox {K}
         font {\G} [0pt]
         (\C55) ([\C1]) (\C54)
         ([\C2])        ([\C0])
         (\C56) ([\C3]) (\C57)
\end{bsp}
\make

\begin{bsp}
\RandBox {L}
         font {\G} [0pt]
         (\C61) ([\C1]) (\C60)
         ([\C2])        ([\C0])
         (\C62) ([\C3]) (\C63)
\end{bsp}
\make

\begin{bsp}
\RandBox {M}
         font {\G} [0pt]
         (\C65) ([\C1]) (\C64)
         ([\C2])        ([\C0])
         (\C66) ([\C3]) (\C67)
\end{bsp}
\make

\begin{bsp}
\RandBox {N}
         font {\G} [0pt]
         (\C71) ([\C1]) (\C70)
         ([\C2])        ([\C0])
         (\C72) ([\C3]) (\C73)
\end{bsp}
\make

\begin{bsp}
\RandBox {O}
         font {\G} [0pt]
         (\C75) ([\C1]) (\C74)
         ([\C2])        ([\C0])
         (\C76) ([\C3]) (\C77)
\end{bsp}
\make

\vskip2mm
Nat"urlich k"onnen die Ecken auch gemischt werden, es ergeben sich so $15^4=50625$
m"ogliche Rahmen; inwieweit diese auch gut aussehen, bleibt jedem selbst
"uberlassen. Einen der $50610$ noch nicht abgedruckten Rahmen gibt's hier
noch als Draufgabe:
\begin{bsp}
\RandBox {P}
         font {\G} [0pt]
         (\C25) ([\C1]) (\C30)
         ([\C2])        ([\C0])
         (\C32) ([\C3]) (\C27)
\end{bsp}
\make

\newpage
Ein paar Rahmen mit vier Linien.
\begin{bsp}
\RandBox {a}
         font {\G} [0pt]
         (\C105) ([\C101]) (\C104)
         ([\C102])        ([\C100])
         (\C106) ([\C103]) (\C107)
\end{bsp}
\make

\begin{bsp}
\RandBox {b}
         font {\G} [0pt]
         (\C111) ([\C101]) (\C110)
         ([\C102])        ([\C100])
         (\C112) ([\C103]) (\C113)
\end{bsp}
\make

\begin{bsp}
\RandBox {c}
         font {\G} [0pt]
         (\C115) ([\C101]) (\C114)
         ([\C102])        ([\C100])
         (\C116) ([\C103]) (\C117)
\end{bsp}
\make

\begin{bsp}
\RandBox {d}
         font {\G} [0pt]
         (\C121) ([\C101]) (\C120)
         ([\C102])        ([\C100])
         (\C122) ([\C103]) (\C123)
\end{bsp}
\make

\begin{bsp}
\RandBox {e}
         font {\G} [0pt]
         (\C125) ([\C101]) (\C124)
         ([\C102])        ([\C100])
         (\C126) ([\C103]) (\C127)
\end{bsp}
\make

\begin{bsp}
\RandBox {f}
         font {\G} [0pt]
         (\C131) ([\C101]) (\C130)
         ([\C102])        ([\C100])
         (\C132) ([\C103]) (\C133)
\end{bsp}
\make

\begin{bsp}
\RandBox {g}
         font {\G} [0pt]
         (\C135) ([\C101]) (\C134)
         ([\C102])        ([\C100])
         (\C136) ([\C103]) (\C137)
\end{bsp}
\make

\begin{bsp}
\RandBox {h}
         font {\G} [0pt]
         (\C141) ([\C101]) (\C140)
         ([\C102])        ([\C100])
         (\C142) ([\C103]) (\C143)
\end{bsp}
\make

\begin{bsp}
\RandBox {i}
         font {\G} [0pt]
         (\C145) ([\C101]) (\C144)
         ([\C102])        ([\C100])
         (\C146) ([\C103]) (\C147)
\end{bsp}
\make

\vskip2mm
Auch hier gibt es wieder eine ganze Reihe von m"oglichen Kombinationen,
und zwar genau $6561$.

\newpage

Nun noch ein paar Rahmen mit 5 Linien:\\[1mm]
{
\def\make{\RandBox {4}
         font {\F} [0pt]
         (\C136) ([\C137]) (\C140)
         ([\C145])         ([\C141])
         (\C144) ([\C143]) (\C142)

          \vskip 2mm
		  \hrule
          \verbatiminput{b}
		  \vskip 2mm\hrule\hrule\vskip 1mm}
\begin{bsp}
\RandBox {\hspace*{1cm}1\hspace*{1cm}}
         font {\G} [0pt]
         (\C161)    ([\C151\\\C155]\C151)     (\C160)
         ([\C152\\\C156]\C152)  ([\C150\\\C154]\C150)
         (\C162)    ([\C153\\\C157]\C153)     (\C163)
\end{bsp}
\make

\begin{bsp}
\RandBox {\hspace*{1cm}1\hspace*{1cm}}
         font {\G} [0pt]
         (\C165)    ([\C151\\\C155]\C151)     (\C164)
         ([\C152\\\C156]\C152)  ([\C150\\\C154]\C150)
         (\C166)    ([\C153\\\C157]\C153)     (\C167)
\end{bsp}
\make

\begin{bsp}
\RandBox {\hspace*{1cm}1\hspace*{1cm}}
         font {\G} [0pt]
         (\C171)    ([\C151\\\C155]\C151)     (\C170)
         ([\C152\\\C156]\C152)  ([\C150\\\C154]\C150)
         (\C172)    ([\C153\\\C157]\C153)     (\C173)
\end{bsp}
\make

}

Nun gibt es noch einen Rahmen, der selbst den alten Pharaonen gefallen
w"urde. Um die 128-Zeichen-Grenze nicht zu "uberschreiten stehen die hierzu
geh"orenden Zeichen im Font {\tt umranda.mf}.
\begin{bsp}
\RandBox {4}
         font {\F} [0pt]
         (\C136) ([\C137]) (\C140)
         ([\C145])         ([\C141])
         (\C144) ([\C143]) (\C142)
\end{bsp}
\make


\vskip2mm
Sollten die Rahmen beim Ausdrucken nicht exakt aneinanderpassen, so kann
dies daran liegen, da"s beim Druckertreiber eine gewisse Positionsabweichung
der Zeichen erlaubt ist (die sogenannte {\tt maxdrift}), da bei "ublichen
{\tt .dvi}--Files eine gewisse Abweichung nicht auff"allt. Bei diesen Fonts
darf jedoch keine Verschiebung auftreten, also mu"s der {\tt maxdrift}--Wert
auf 0 gesetzt werden. F"ur die em\TeX--Druckertreiber z.B.\ geht dies mit
der Option {\tt /fd0}.
\endinput
