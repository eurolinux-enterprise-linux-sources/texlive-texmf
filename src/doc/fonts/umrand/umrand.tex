\documentstyle[german,umrand,stamm,struts,verbatim,bsp]{article}

\Nr{9a}
\Thema{Anleitung zu UMRAND.STY}

\def\C{\char'}
\font\UA=umranda
\font\UB=umrandb at 20pt

\def\make{%
    \begin{flushleft}
        \begin{minipage}[c]{0.58\linewidth}
            \verbatiminput{\bspname}
        \end{minipage}
        \hfill\vrule\hfill
        \begin{minipage}[c]{0.38\linewidth}
            \input{\bspname}
        \end{minipage}
    \end{flushleft}}

\begin{document}
{\bf Als Nachtrag hier eine kleine Anleitung zu UMRAND.STY:}\\[3mm]
Notwendige Dateien:
\[ \begin{tabular}{|l|l|}
   \hline
   \oustrut Datei & Bedeutung \\ \hline\hline
   \ostrut umranda.mf & 1.\,Font \\
           umrandb.mf & 2.\,Font \\
   \ustrut umrand.mf  & MF-Makros \\ \hline
   \ostrut umrand.sty & \TeX-Makros
   \ustrut \\ \hline
   \end{tabular} \]
Achtung! Die Fonts funktionieren zur Zeit (Version 1.2) nur bei
Ausgabeger"aten mit gleicher horizontaler und vertikaler Auf"|l"osung!
Vielleicht "andert sich das irgendwann.

\verb|umrand.sty| dient zum Zeichnen von Umrandungen. Grunds"atzlich wird die
Gr"o"se des Rahmens von dessen Inhalt (und den Rahmenzeichen) bestimmt. Sie
wird automatisch berechnet. Die allgemeine Form des Aufrufs sieht so aus:
\begin{verbatim}
\RandBox {...}
         font {...}
         [...]
         (...)         (...[...]...)         (...)
         (...[...]...)               (...[...]...)
         (...)         (...[...]...)         (...)
\end{verbatim}
Der erste Parameter ist der Inhalt, der in eine \verb|\hbox| gesetzt wird. Mit
dem Wort \verb|font| eingeleitet wird ein Parameter, der den Font f"ur den
Rahmen ausw"ahlt. In den eckigen Klammern \verb|[...]| steht eine L"angenangabe,
die den Abstand der einzelnen Zeichen voneinander angibt. Mit den folgenden
sechs runden Klammern wird die Struktur definiert. Hier k"onnen Zeichen des
Rahmen-Fonts eingetragen werden. Die Zeichenfolgen, die in den eckigen
Klammern hier stehen, werden sooft wiederholt wie notwendig, um die
Rahmengr"o"se zu erzeugen. Zeichen in Folgen werden durch \verb|\\| voneinander
getrennt. Die runden Klammern bilden in ihrer Reihenfolge die Rahmenteile:
\begin{itemize}
\item linke obere Ecke
\item oberer Rand
\item rechte obere Ecke
\item linker Rand
\item rechter Rand
\item linke untere Ecke
\item unterer Rand
\item rechte untere Ecke
\end{itemize}
Beispiele:
\begin{verbatim}
(G\\H[I\\K]L)
\end{verbatim}
hei"st: Setze G und H, dann I und K abwechselnd, bis die Rahmengr"o"se
erreicht ist, dann setze L.
\begin{verbatim}
([A\\B]A)
\end{verbatim}
hei"st: A und B abwechselnd, am Schlu"s ein A.

Wenn man die Fonts aus \verb|umranda.mf| und \verb|umrandb.mf| benutzt,
geht man normalerweise von einer Fonttabelle (\verb|testfont.tex|
benutzen) aus. Hier sind die Zeichen oktal beziffert. Setzt man zum
Beispiel \verb|\def\C{\char'}|, \verb|\font\UA=umranda at 20pt| und
\verb|\font\UB=umrandb at 20pt| in seine Datei, so kann man die Fonts
folgenderma"sen ansprechen:
\begin{bsp}
\RandBox {Testinhalt} font {\UA} [-0.3em]
         (\C113)  ([\C65]) (\C113)
         ([\C67])          ([\C67])
         (\C113)  ([\C65]) (\C113)
\end{bsp}
\make
Die Ecken bestehen aus dem Zeichen Oktal\,113, die waagerechten Kanten aus
Oktal\,65, die senkrechten aus Oktal\,67.

Jetzt ein Beispiel mit wechselnden Zeichen:
\begin{bsp}
\RandBox {Eine Wiese}
         font {\UA} [0pt]
         () ([\C75\\\C77]\C75) ()
         (\C76[\C100\\\C76])
         ([\C100\\\C76]\C100)
         () (\C77[\C75\\\C77]) ()
\end{bsp}
\make

Bei den Linienfonts (\verb|umrandb.mf|) mu"s man auf die Lage der Elemente
achten. Linienelemente an unterschiedlichen Positionen sind auch
unterschiedliche Zeichen. Hier gibt es aber die nach dem Beispiel
beschriebene Kurzform.
\begin{bsp}
\RandBox {Ein Test}
         font {\UB} [0pt]
         (\C5) ([\C1]) (\C4)
         ([\C2]) ([\C0])
         (\C6) ([\C3]) (\C7)
\end{bsp}
\make

Als Kurzform kann man jeweils das erste Zeichen der Eckenteile und der
Linienteile angeben:
\begin{bsp}
\LinienBox {Ein Test}
           font {\UB}
           ecken 4 linien 0
\end{bsp}
\make

Hier noch eine Zusammenstellung anderer Umrandungen:
\begin{flushleft}
\def\TAB#1{Mit\begin{tabular}[t]{l}#1\end{tabular}}
\LinienBox {\TAB{ecken 10\\linien 0}} font {\UB} ecken 10 linien 0
\LinienBox {\TAB{ecken 14\\linien 0}} font {\UB} ecken 14 linien 0
\LinienBox {\TAB{ecken 20\\linien 0}} font {\UB} ecken 20 linien 0
\LinienBox {\TAB{ecken 24\\linien 0}} font {\UB} ecken 24 linien 0
\LinienBox {\TAB{ecken 30\\linien 0}} font {\UB} ecken 30 linien 0
\LinienBox {\TAB{ecken 34\\linien 0}} font {\UB} ecken 34 linien 0
\LinienBox {\TAB{ecken 40\\linien 0}} font {\UB} ecken 40 linien 0
\LinienBox {\TAB{ecken 44\\linien 0}} font {\UB} ecken 44 linien 0
\LinienBox {\TAB{ecken 50\\linien 0}} font {\UB} ecken 50 linien 0
\LinienBox {\TAB{ecken 54\\linien 0}} font {\UB} ecken 54 linien 0
\LinienBox {\TAB{ecken 60\\linien 0}} font {\UB} ecken 60 linien 0
\LinienBox {\TAB{ecken 64\\linien 0}} font {\UB} ecken 64 linien 0
\LinienBox {\TAB{ecken 70\\linien 0}} font {\UB} ecken 70 linien 0
\LinienBox {\TAB{ecken 74\\linien 0}} font {\UB} ecken 74 linien 0
\LinienBox {\TAB{ecken 104\\linien 100}} font {\UB} ecken 104 linien 100
\LinienBox {\TAB{ecken 110\\linien 100}} font {\UB} ecken 110 linien 100
\LinienBox {\TAB{ecken 114\\linien 100}} font {\UB} ecken 114 linien 100
\LinienBox {\TAB{ecken 120\\linien 100}} font {\UB} ecken 120 linien 100
\LinienBox {\TAB{ecken 124\\linien 100}} font {\UB} ecken 124 linien 100
\LinienBox {\TAB{ecken 130\\linien 100}} font {\UB} ecken 130 linien 100
\LinienBox {\TAB{ecken 134\\linien 100}} font {\UB} ecken 134 linien 100
\LinienBox {\TAB{ecken 140\\linien 100}} font {\UB} ecken 140 linien 100
\LinienBox {\TAB{ecken 144\\linien 100}} font {\UB} ecken 144 linien 100
\end{flushleft}

Ein Meander-Rand ist mit dem folgenden Befehl zu erzeugen:
\begin{bsp}
\MeanderBox {Ein Meander-Rand} at 20pt
\end{bsp}
\make

Einfache R"ander, die nur aus einem einzigen Zeichen bestehen, erh"alt man
auf die folgende Weise:
\begin{bsp}
\EinfachRand {Inhalt}
             font {\Large}
			 char {$\ast$} [0.1em]
\end{bsp}
\make
Hierbei sieht man auch, da"s Zeichen aus beliebigen Fonts f"ur die Rahmen
benutzt werden k"onnen. Problematisch wird es dann, wenn sie
unterschiedliche Breiten/H"ohen haben.

\[ \MeanderBox {Weitere Beispiele findet man in
               {\tt ob1.tex} und {\tt ob2.tex}} at 10pt \]

Eine Font-Tabelle liefert die \TeX-Datei \verb|testfont.tex|. Man ruft sie
unter {\sc Plain}-\TeX\ auf (z.\,B. \verb|tex &plain testfont|). Als Font
gibt man bei der Frage \verb|umranda at 20pt| oder etwas "ahnliches an.
Wenn der \verb|*| erscheint, tippt man \verb|\table| ein, beim n"achsten
\verb|*| schreibt man \verb|\end|. Nun mu"s nur noch \verb|testfont.dvi|
gedruckt werden.

Falls der Druckertreiber eine Einstellung von |maxdrift| erlaubt, sollte
diese auf Null gesetzt werden. Sonst kann es vorkommen, da"s
Linienanschl"usse nicht 100\,\%ig stimmen.

\newpage\setcounter{page}{1}
\font\f=umranda
\font\F=umranda at 20pt
{\def\make{\RandBox {4}
         font {\F} [0pt]
         (\C136) ([\C137]) (\C140)
         ([\C145])         ([\C141])
         (\C144) ([\C143]) (\C142)

          \vskip 2mm
		  \hrule
          \verbatiminput{b}
		  \vskip 2mm\hrule\hrule\vskip 1mm}
\fussy
\Thema{Umrandungen}
\Nr{9}

Heute gehts um Umrandungen, die Makros und der Font sind ziemlich
umfangreich, so da"s beide hier nur in der Anwendung gezeigt werden sollen.
Es sind definiert:
\begin{verbatim}
\def\C{\char'} % Oktaldarstellung
\font\f = umranda
\font\F = umranda at 20pt
\end{verbatim}
Und los gehts:
\vskip 2mm

\begin{bsp}
\RandBox {Dieses ist eine Quadrat-Testbox}
         font {\f} [0pt]
         (\C114) ([\C71\\\C73]\C71) (\C114)
         (\C72[\C74\\\C72])        ([\C74\\\C72]\C74)
         (\C114) (\C73[\C71\\\C73]) (\C114)
\end{bsp}
\make

\begin{bsp}
\RandBox {Es w"achst Gras "uber die Sache}
         font {\f} [0pt]
         () ([\C75\\\C77]\C75) ()
         (\C76[\C100\\\C76])           ([\C100\\\C76]\C100)
         () (\C77[\C75\\\C77]) ()
\end{bsp}
\make



\begin{bsp}
\RandBox {Ist das nicht zum Kringeln?}
         font {\f} [-0.3em]
         () ([\C65]) ()
         ([\C67])   ([\C67])
         () ([\C65]) ()
\end{bsp}
\make



\begin{bsp}
\RandBox {Eine sich schl"angelnde Schlange.}
         font {\F} [-0.3em]
         () ([\C127]) ()
         ([\C131])   ([\C135])
         () ([\C133]) ()
\end{bsp}
\make



\begin{bsp}
\RandBox {Dieses ist eine riesige Box mit Fischen}
         font {\f} [0.1em]
         (\C113) ([\C102]) (\C113)
         ([\C104])         ([\C110])
         (\C113) ([\C106]) (\C113)
\end{bsp}
\make



\begin{bsp}
\RandBox {Wellenlinien mit verschiedenen Ecken}
         font {\f} [0pt]
         (\C115) ([\C111]) (\C114)
         ([\C112])         ([\C112])
         (\C115) ([\C111]) (\C114)
\end{bsp}
\make



\begin{bsp}
\RandBox {\parbox{7cm}{Dieses ist eine riesige und gro"se Testbox
                       mit hoffentlich zwei Zeilen! Der Rand besteht
                       aus sich "uberlappenden Wellenteilen.}}
         font {\f} [-0.5em]
         (\C113) ([\C111]) (\C113)
         ([\C112])         ([\C112])
         (\C113) ([\C111]) (\C113)
\end{bsp}
\make



\begin{bsp}
\EinfachRand {Wieder mal eine (andere) Box} font {\F} char {\C43} [0pt]
\end{bsp}
\make



\begin{bsp}
\EinfachRand {Noch eine Box, diesmal mit Stacheldraht.}
    font {\F} char {\C41} [0.2em]
\end{bsp}
\make



\begin{bsp}
\EinfachRand {Und man sieht hier noch einen Kasten.}
    font {\F} char {\C42} [0.2em]
\end{bsp}
\make



\begin{bsp}
\EinfachRand {Yin und Yang} font {\f} char {\C10} [0.2em]
\end{bsp}
\make



\begin{bsp}
\EinfachRand {Nach einer Grafik von Apollonio.}
    font {\F} char {\C37} [0.2em]
\end{bsp}
\make



\begin{bsp}
\RandBox {Test in einer Box mit Buchstaben.}
         font {\tt} [0.2em]
         (W)        ([B\\C]B)       (X)
         ([D\\E\\F])                ([G\\H\\I])
         (Y)        ([J\\K]J)           (Z)
\end{bsp}
\make



\begin{bsp}
\RandBox {Wir machen das Unm"ogliche m"oglich.}
         font {\F} [0.1em]
         (\C3)              ([\C2\\\C6]\C2)         (\C1)
         ([\C4\\\C0]\C4)                        ([\C0\\\C4]\C0)
         (\C5)              ([\C6\\\C2]\C6)     (\C7)
\end{bsp}
\make



\begin{bsp}
\RandBox {Wunder dauern etwas l"anger.}
         font {\F} [0.2em]
         ()                 ([\C2\\\C6]\C2)         ()
         ([\C4\\\C0]\C4)                        ([\C0\\\C4]\C0)
         ()                 ([\C6\\\C2]\C6)     ()
\end{bsp}
\make



\begin{bsp}
\RandBox {Eine Box wie vorne, aber mit Abst"anden.}
         font {\F} [0.2em]
         ()         ([\C43])        ()
         ([\C43])                       ([\C43])
         ()         ([\C43])        ()
\end{bsp}
\make



\begin{bsp}
\RandBox {Mischmasch mit gedrehtem Yin und Yang.}
         font {\f} [0.2em]
         (\C7)      ([\C10\\\C12\\\C14\\\C16\\\C20\\\C22])      (\C5)
         ([\C21])                                               ([\C13])
         (\C1)      ([\C10\\\C12\\\C14\\\C16\\\C20\\\C22])      (\C3)
\end{bsp}
\make

\begin{bsp}
\EinfachRand {\parbox{7cm}{Dieses hier ist eine Parbox mit mehreren Zeilen
Text, die sp"ater umrahmt werden soll. Der Rahmen besteht nicht aus Linien
sondern aus Schmucksymbolen, die von \TeX\ automatisch zu einem Rahmen in
der erforderlichen Gr"o"se zusammengesetzt werden.}}
    font {\f} char {\C66} [-0.7em]
\end{bsp}
\make
}
\endinput
\newpage\setcounter{page}{1}
\parindent=0pt \parskip=0pt
\font\G=umrandb at 20pt
\font\F=umranda at 20pt
\fussy
\Thema{Neue Umrandungen}
\Nr{10}
\def\C{\char'}

Angeregt vom letzten Stammtisch habe ich einen weiteren Font mit Rahmen
erstellt. Da das {\tt .mf}--File f"ur eine Vorstellung zu lang ist,
gebe ich hier nur die m"oglichen Rahmen zum Besten.

Zun"achst wird definiert:
\begin{verbatim}
\def\C{\char'} % Oktaldarstellung
\font\G = umrandb at 20pt
\end{verbatim}
Und nun kommen die Rahmen, zun"achst jene mit 6 Linien:\vskip 2mm

\begin{bsp}
\RandBox {A}
         font {\G} [0pt]
         (\C5) ([\C1]) (\C4)
         ([\C2])        ([\C0])
         (\C6) ([\C3]) (\C7)
\end{bsp}
\make

\begin{bsp}
\RandBox {B}
         font {\G} [0pt]
         (\C11) ([\C1]) (\C10)
         ([\C2])        ([\C0])
         (\C12) ([\C3]) (\C13)
\end{bsp}
\make

\begin{bsp}
\RandBox {C}
         font {\G} [0pt]
         (\C15) ([\C1]) (\C14)
         ([\C2])        ([\C0])
         (\C16) ([\C3]) (\C17)
\end{bsp}
\make

\begin{bsp}
\RandBox {D}
         font {\G} [0pt]
         (\C21) ([\C1]) (\C20)
         ([\C2])        ([\C0])
         (\C22) ([\C3]) (\C23)
\end{bsp}
\make

\begin{bsp}
\RandBox {E}
         font {\G} [0pt]
         (\C25) ([\C1]) (\C24)
         ([\C2])        ([\C0])
         (\C26) ([\C3]) (\C27)
\end{bsp}
\make

\begin{bsp}
\RandBox {F}
         font {\G} [0pt]
         (\C31) ([\C1]) (\C30)
         ([\C2])        ([\C0])
         (\C32) ([\C3]) (\C33)
\end{bsp}
\make

\begin{bsp}
\RandBox {G}
         font {\G} [0pt]
         (\C35) ([\C1]) (\C34)
         ([\C2])        ([\C0])
         (\C36) ([\C3]) (\C37)
\end{bsp}
\make

\begin{bsp}
\RandBox {H}
         font {\G} [0pt]
         (\C41) ([\C1]) (\C40)
         ([\C2])        ([\C0])
         (\C42) ([\C3]) (\C43)
\end{bsp}
\make

\begin{bsp}
\RandBox {I}
         font {\G} [0pt]
         (\C45) ([\C1]) (\C44)
         ([\C2])        ([\C0])
         (\C46) ([\C3]) (\C47)
\end{bsp}
\make

\begin{bsp}
\RandBox {J}
         font {\G} [0pt]
         (\C51) ([\C1]) (\C50)
         ([\C2])        ([\C0])
         (\C52) ([\C3]) (\C53)
\end{bsp}
\make

\begin{bsp}
\RandBox {K}
         font {\G} [0pt]
         (\C55) ([\C1]) (\C54)
         ([\C2])        ([\C0])
         (\C56) ([\C3]) (\C57)
\end{bsp}
\make

\begin{bsp}
\RandBox {L}
         font {\G} [0pt]
         (\C61) ([\C1]) (\C60)
         ([\C2])        ([\C0])
         (\C62) ([\C3]) (\C63)
\end{bsp}
\make

\begin{bsp}
\RandBox {M}
         font {\G} [0pt]
         (\C65) ([\C1]) (\C64)
         ([\C2])        ([\C0])
         (\C66) ([\C3]) (\C67)
\end{bsp}
\make

\begin{bsp}
\RandBox {N}
         font {\G} [0pt]
         (\C71) ([\C1]) (\C70)
         ([\C2])        ([\C0])
         (\C72) ([\C3]) (\C73)
\end{bsp}
\make

\begin{bsp}
\RandBox {O}
         font {\G} [0pt]
         (\C75) ([\C1]) (\C74)
         ([\C2])        ([\C0])
         (\C76) ([\C3]) (\C77)
\end{bsp}
\make

\vskip2mm
Nat"urlich k"onnen die Ecken auch gemischt werden, es ergeben sich so $15^4=50625$
m"ogliche Rahmen; inwieweit diese auch gut aussehen, bleibt jedem selbst
"uberlassen. Einen der $50610$ noch nicht abgedruckten Rahmen gibt's hier
noch als Draufgabe:
\begin{bsp}
\RandBox {P}
         font {\G} [0pt]
         (\C25) ([\C1]) (\C30)
         ([\C2])        ([\C0])
         (\C32) ([\C3]) (\C27)
\end{bsp}
\make

\newpage
Ein paar Rahmen mit vier Linien.
\begin{bsp}
\RandBox {a}
         font {\G} [0pt]
         (\C105) ([\C101]) (\C104)
         ([\C102])        ([\C100])
         (\C106) ([\C103]) (\C107)
\end{bsp}
\make

\begin{bsp}
\RandBox {b}
         font {\G} [0pt]
         (\C111) ([\C101]) (\C110)
         ([\C102])        ([\C100])
         (\C112) ([\C103]) (\C113)
\end{bsp}
\make

\begin{bsp}
\RandBox {c}
         font {\G} [0pt]
         (\C115) ([\C101]) (\C114)
         ([\C102])        ([\C100])
         (\C116) ([\C103]) (\C117)
\end{bsp}
\make

\begin{bsp}
\RandBox {d}
         font {\G} [0pt]
         (\C121) ([\C101]) (\C120)
         ([\C102])        ([\C100])
         (\C122) ([\C103]) (\C123)
\end{bsp}
\make

\begin{bsp}
\RandBox {e}
         font {\G} [0pt]
         (\C125) ([\C101]) (\C124)
         ([\C102])        ([\C100])
         (\C126) ([\C103]) (\C127)
\end{bsp}
\make

\begin{bsp}
\RandBox {f}
         font {\G} [0pt]
         (\C131) ([\C101]) (\C130)
         ([\C102])        ([\C100])
         (\C132) ([\C103]) (\C133)
\end{bsp}
\make

\begin{bsp}
\RandBox {g}
         font {\G} [0pt]
         (\C135) ([\C101]) (\C134)
         ([\C102])        ([\C100])
         (\C136) ([\C103]) (\C137)
\end{bsp}
\make

\begin{bsp}
\RandBox {h}
         font {\G} [0pt]
         (\C141) ([\C101]) (\C140)
         ([\C102])        ([\C100])
         (\C142) ([\C103]) (\C143)
\end{bsp}
\make

\begin{bsp}
\RandBox {i}
         font {\G} [0pt]
         (\C145) ([\C101]) (\C144)
         ([\C102])        ([\C100])
         (\C146) ([\C103]) (\C147)
\end{bsp}
\make

\vskip2mm
Auch hier gibt es wieder eine ganze Reihe von m"oglichen Kombinationen,
und zwar genau $6561$.

\newpage

Nun noch ein paar Rahmen mit 5 Linien:\\[1mm]
{
\def\make{\RandBox {4}
         font {\F} [0pt]
         (\C136) ([\C137]) (\C140)
         ([\C145])         ([\C141])
         (\C144) ([\C143]) (\C142)

          \vskip 2mm
		  \hrule
          \verbatiminput{b}
		  \vskip 2mm\hrule\hrule\vskip 1mm}
\begin{bsp}
\RandBox {\hspace*{1cm}1\hspace*{1cm}}
         font {\G} [0pt]
         (\C161)    ([\C151\\\C155]\C151)     (\C160)
         ([\C152\\\C156]\C152)  ([\C150\\\C154]\C150)
         (\C162)    ([\C153\\\C157]\C153)     (\C163)
\end{bsp}
\make

\begin{bsp}
\RandBox {\hspace*{1cm}1\hspace*{1cm}}
         font {\G} [0pt]
         (\C165)    ([\C151\\\C155]\C151)     (\C164)
         ([\C152\\\C156]\C152)  ([\C150\\\C154]\C150)
         (\C166)    ([\C153\\\C157]\C153)     (\C167)
\end{bsp}
\make

\begin{bsp}
\RandBox {\hspace*{1cm}1\hspace*{1cm}}
         font {\G} [0pt]
         (\C171)    ([\C151\\\C155]\C151)     (\C170)
         ([\C152\\\C156]\C152)  ([\C150\\\C154]\C150)
         (\C172)    ([\C153\\\C157]\C153)     (\C173)
\end{bsp}
\make

}

Nun gibt es noch einen Rahmen, der selbst den alten Pharaonen gefallen
w"urde. Um die 128-Zeichen-Grenze nicht zu "uberschreiten stehen die hierzu
geh"orenden Zeichen im Font {\tt umranda.mf}.
\begin{bsp}
\RandBox {4}
         font {\F} [0pt]
         (\C136) ([\C137]) (\C140)
         ([\C145])         ([\C141])
         (\C144) ([\C143]) (\C142)
\end{bsp}
\make


\vskip2mm
Sollten die Rahmen beim Ausdrucken nicht exakt aneinanderpassen, so kann
dies daran liegen, da"s beim Druckertreiber eine gewisse Positionsabweichung
der Zeichen erlaubt ist (die sogenannte {\tt maxdrift}), da bei "ublichen
{\tt .dvi}--Files eine gewisse Abweichung nicht auff"allt. Bei diesen Fonts
darf jedoch keine Verschiebung auftreten, also mu"s der {\tt maxdrift}--Wert
auf 0 gesetzt werden. F"ur die em\TeX--Druckertreiber z.B.\ geht dies mit
der Option {\tt /fd0}.
\endinput

\originalTeX
% stolen from testfont.tex
\tracinglostchars=0
\tolerance=1000
\raggedbottom
\parindent=0pt
\newlinechar=`@
\hyphenpenalty=200
\doublehyphendemerits=30000
\newcount\m \newcount\n \newcount\p \newdimen\dim
\chardef\other=12
\def\today{\ifcase\month\or
  January\or February\or March\or April\or May\or June\or
  July\or August\or September\or October\or November\or December\fi
  \space\number\day, \number\year}
\def\hours{\n=\time \divide\n 60
  \m=-\n \multiply\m 60 \advance\m \time
  \twodigits\n\twodigits\m}
\def\twodigits#1{\ifnum #1<10 0\fi \number#1}
\def\setbaselineskip{\setbox0=\hbox{\n=0
\loop\char\n \ifnum \n<255 \advance\n 1 \repeat}
\baselineskip=6pt \advance\baselineskip\ht0 \advance\baselineskip\dp0 }
\def\setchar#1{{\escapechar-1\message{\string#1 character = }%
  \def\do##1{\catcode`##1=\other}\dospecials
  \read-1 to\next
  \expandafter\finsetchar\next\next#1}}
\def\finsetchar#1#2\next#3{\global\chardef#3=`#1
  \ifnum #3=`\# \global\chardef#3=#2 \fi}
\def\promptthree{\setchar\background
  \setchar\starting \setchar\ending}
\def\init#1{%\message{@Name of the font to test = }
 % \read-1 to\fontname
  \def\fontname{#1}
  \font\testfont=\fontname
  \leftline{\sevenrm Test of \fontname\unskip\ on \today\ at \hours}
  \medskip
  \testfont \setbaselineskip
  \ifdim\fontdimen6\testfont<10pt \rightskip=0pt plus 20pt
  \else\rightskip=0pt plus 2em \fi
  \spaceskip=\fontdimen2\testfont % space between words (\raggedright)
  \xspaceskip=\fontdimen2\testfont \advance\xspaceskip by\fontdimen7\testfont
  %\message{Now type a test command (\string\help\space for help):}
 }
\def\oct#1{\hbox{\rm\'{}\kern-.2em\it#1\/\kern.05em}} % octal constant
\def\hex#1{\hbox{\rm\H{}\tt#1}} % hexadecimal constant
\def\setdigs#1"#2{\gdef\h{#2}% \h=hex prefix; \0\1=corresponding octal
 \m=\n \divide\m by 64 \xdef\0{\the\m}%
 \multiply\m by-64 \advance\m by\n \divide\m by 8 \xdef\1{\the\m}}
\def\testrow{\setbox0=\hbox{\penalty 1\def\\{\char"\h}%
 \\0\\1\\2\\3\\4\\5\\6\\7\\8\\9\\A\\B\\C\\D\\E\\F%
 \global\p=\lastpenalty}} % \p=1 if none of the characters exist
\def\oddline{\cr
  \noalign{\nointerlineskip}
  \multispan{19}\hrulefill&
  \setbox0=\hbox{\lower 2.3pt\hbox{\hex{\h x}}}\smash{\box0}\cr
  \noalign{\nointerlineskip}}
\newif\ifskipping
\def\evenline{\loop\skippingfalse
 \ifnum\n<256 \m=\n \divide\m 16 \chardef\next=\m
 \expandafter\setdigs\meaning\next \testrow
 \ifnum\p=1 \skippingtrue \fi\fi
 \ifskipping \global\advance\n 16 \repeat
 \ifnum\n=256 \let\next=\endchart\else\let\next=\morechart\fi
 \next}
\def\morechart{\cr\noalign{\hrule\penalty5000}
 \chartline \oddline \m=\1 \advance\m 1 \xdef\1{\the\m}
 \chartline \evenline}
\def\chartline{&\oct{\0\1x}&&\:&&\:&&\:&&\:&&\:&&\:&&\:&&\:&&}
\def\chartstrut{\lower4.5pt\vbox to14pt{}}
\def\table{$$\global\n=0
  \halign to\hsize\bgroup
    \chartstrut##\tabskip0pt plus10pt&
    &\hfil##\hfil&\vrule##\cr
    \lower6.5pt\null
    &&&\oct0&&\oct1&&\oct2&&\oct3&&\oct4&&\oct5&&\oct6&&\oct7&\evenline}
\def\endchart{\cr\noalign{\hrule}
  \raise11.5pt\null&&&\hex 8&&\hex 9&&\hex A&&\hex B&
  &\hex C&&\hex D&&\hex E&&\hex F&\cr\egroup$$\par}
\def\:{\setbox0=\hbox{\char\n}%
  \ifdim\ht0>7.5pt\reposition
  \else\ifdim\dp0>2.5pt\reposition\fi\fi
  \box0\global\advance\n 1 }
\def\reposition{\setbox0=\vbox{\kern2pt\box0}\dim=\dp0
  \advance\dim 2pt \dp0=\dim}
\def\centerlargechars{
  \def\reposition{\setbox0=\hbox{$\vcenter{\kern2pt\box0\kern2pt}$}}}
%
\newpage
\ifx\selectfont\undefined
  \font\sevenrm=cmr7
 \else
  \def\sevenrm{\fontfamily{cmr}\fontshape{n}\fontseries{m}\fontsize{7}{7pt}\selectfont}
\fi
\pagestyle{empty}
\init{umranda at 20pt}
\table
\newpage
\init{umrandb at 20pt}
\table
\endinput

\end{document}

