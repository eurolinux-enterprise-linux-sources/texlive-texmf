%%
%% This is file `examples.tex',
%% generated with the docstrip utility.
%%
%% The original source files were:
%%
%% barcodes.dtx  (with options: `example')
%% As this is a generated file, you may perhaps not want to edit it.
%% This file belongs to the barcode package.
%% It may be of no great use without the rest of the barcode package.
%% \CharacterTable
%%  {Upper-case    \A\B\C\D\E\F\G\H\I\J\K\L\M\N\O\P\Q\R\S\T\U\V\W\X\Y\Z
%%   Lower-case    \a\b\c\d\e\f\g\h\i\j\k\l\m\n\o\p\q\r\s\t\u\v\w\x\y\z
%%   Digits        \0\1\2\3\4\5\6\7\8\9
%%   Exclamation   \!     Double quote  \"     Hash (number) \#
%%   Dollar        \$     Percent       \%     Ampersand     \&
%%   Acute accent  \'     Left paren    \(     Right paren   \)
%%   Asterisk      \*     Plus          \+     Comma         \,
%%   Minus         \-     Point         \.     Solidus       \/
%%   Colon         \:     Semicolon     \;     Less than     \<
%%   Equals        \=     Greater than  \>     Question mark \?
%%   Commercial at \@     Left bracket  \[     Backslash     \\
%%   Right bracket \]     Circumflex    \^     Underscore    \_
%%   Grave accent  \`     Left brace    \{     Vertical bar  \|
%%   Right brace   \}     Tilde         \~}
%%
%%%%%%%%%%%%%%%%%%%%%%%%%%%%%%%%%%%%%%%%%%%%%%%%%%%%%%
%%
%% This file documents look and use of
%% the barcodes this package belongs to.
%% It may be freely used without any
%% further permission.
%% You should have received this file as part of
%% the barcode package.
%%
%% Author: Peter Willadt
%% Date: 1997-11-29
%%
%%%%%%%%%%%%%%%%%%%%%%%%%%%%%%%%%%%%%%%%%%%%%%%%%%%%%%
%%
%% Note:
%% 1. This file has already been run through codean.pl
%%
%% 2. You need to have the fonts installed, of course.
%%
%%%%%%%%%%%%%%%%%%%%%%%%%%%%%%%%%%%%%%%%%%%%%%%%%%%%%%
%% Fonts
%%
\font\xlix=wlc39 scaled 2000
\font\itf=wlitf  scaled 2000
\font\cdb=wlcdb  scaled 2000
\font\eanfont=WLEAN scaled 1200
\font\fntcxx=wlc128 scaled \magstep3
%%%%%%%%%%%%%%%%%%%%%%%%%%%%%%%%%%%%%%%%%%%%%%%%%%%%%%
%% Inputs
%%
\input code39.tex
%%%%%%%%%%%%%%%%%%%%%%%%%%%%%%%%%%%%%%%%%%%%%%%%%%%%%%
%% Def's
%%
%% for EAN
\def\ean#1{\message{Call codean.pl}}
\def\eean#1{\message{Call codean.pl}}
\def\isbn#1{\message{Call codean.pl}}
\def\embed#1{}
\def\EAN#1{\vbox{\hsize=0.4\hsize\vskip10pt\eanfont#1\vskip10pt}}
%% for Code 128
\def\CXXVIII{\bgroup\fntcxx\let\next\hexchar\next}
\def\hexchar#1#2{\if#1@\global\let\next\egroup\else\char"#1#2\fi\next}
%%
%%%%%%%%%%%%%%%%%%%%%%%%%%%%%%%%%%%%%%%%%%%%%%%%%%%%%%
%% Here we go
\parindent0pt
Hello, this is a test sheet of the barcode fonts.

At first, we deal with code 39. Code 39 is represented in this package
both in form of a native font (Metafont-source) and in form of
\TeX{}-macros. code 39 features low-density alphanumeric encoding.

Here you may see how HELLO looks like in code 39 (the start and stop
sign is mapped to @, if you're curious, so in the source to this sheet
I have written {\tt@HELLO@} after having selected the proper
font). This first example uses the font.

{\xlix@HELLO@}

Another approach is to use \TeX{} macros to make up bars. Here is the
same HELLO with macros:

\begincodethirtynine{HELLO}\endcodethirtynine

Interleaved two-of-five (ITF for short) features high-density
numerical-only encoding. Your code has to have an even number of
digits. The start sign is mapped to $+$, the end sign to $-$.
So, to code 0123456789, you type {\tt+0123456789-}, and the result
looks like this:

{\itf+0123456789-}

If you still have not got enough of barcoding, here is codabar. Here
you got four sets of start/stop signs that get decoded together with
the numbers. The start/stop sign pairs are a/t, b/n, c/*, and d/e. So
{\tt a12345t} looks like this:\bigskip

{\cdb a12345t}

Now you should have a look at code 128. The bars itself look sometimes
disrupted; this is due to the fact that the widest elements are four
times as wide as the narrowest. Code 128 enables you to code any 7-bit
ascii character. With digits only, it is as efficent as itf. The bad
news is the preprocessing required, so you have to read the docs. The
following bars mean {\tt Hallo123456}\bigskip

\CXXVIII 6828414C4C4F630C2238506A@@  % Code128(Hallo123456)

And, last and perhaps most important, the EAN font. Read {\tt
eandoc.tex} to find out how these are coded, here is just the output
of the example code mentioned there:
\embed{2500000000000}

\line{
\EAN{4 +AcFHaa-KKLKNK+} % or, without checksum: %(4025700001030)
\hfil
\EAN{4 +AcFHaa-KKLKNK+} %(402570000103)
}
\line{
\EAN{2 +FAaaAa-KKLMNT+} % embedded(123)
\hfil
\EAN{9 +HiaCaB-LNOOSN+} % ISBN(0201134489)
}

And that's all. Perhaps you may think that this is not a beautiful
document---but barcodes aren't beautiful. As long as reading devices
do not have \ae{}sthetic feelings, I don't regard this as a problem.
\bye
\endinput
%%
%% End of file `examples.tex'.
