% mmfuture.tex
% (c) 1992 Jeroen Hellingman
% Last edit: 09-DEC-1992


\beginsection On the future of \TeX

Working on Malayalam-\TeX, I met with more and more limitations of 
\TeX, which needed ad-hoc solutions, that will not win an elegance 
price. It made me dream of a future \TeX-like system, that would 
incorporate all scripts in a more elegant way. Several people have 
made suggestions on the future of \TeX, but most of them, again, 
focussed on the Western languages. I think a transistion to the 
emerging 16-bit Unicode (ISO 10646) standard will be the most natural 
next step.

\TeX\ was orignally designed for two `scripts', English and 
Mathematics, and it does a very good job of typesetting both of them, 
but one gets in problems when one tries to typeset another script with  
it, for example Malayalam, or one of the various other scripts in use 
for writing the spoken languages of the world, or to name the most 
difficult of all, typesetting music. A future-\TeX\ should be build up 
in some modular manner, so that other scripts can be added more 
easily.

Apart from being a typesetting system, \TeX\ is a fullfledged 
programming language, but a horrible ackward one. As Frank Mittelbach, 
suggested, it would be nice to replace it with a more conventional 
one, although his suggestion to use lisp fills me with new horrors. I 
think a more modern functional language like miranda would be a much  
candidate. A functional language with some very basic functions needed 
for typesetting would be a good kernel in my eyes. All logic needed 
for typesetting a script could then be coded in the language itself, 
instead of being hardwired into the system.

Adding a new script would then be as easy as designing a font for the 
script (A difficult enough task in itself) and writing the logic for 
composing the script from a character-stream, which then could be 
loaded into the system.

An anochronism that has annoyed me more and more when I started making 
several variants and sizes of my font is the use of bitmaps, with took 
up more and more space on my harddisk.

I would propose to join the powers of \TeX\ and \MF into one program, 
that use the speak the same language, so that one can have direct 
access to character-programs, to create the letter in the size and 
style wanted at the a certain moment.


\bye
