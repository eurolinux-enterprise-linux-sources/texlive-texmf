\documentclass[11pt,a4paper,twoside]{article}
  \usepackage[T1]{fontenc}
  % \usepackage[applemac]{inputenc}
  \usepackage[latin1]{inputenc}

  %%%%%%%%%%%%%%%%%%%%%%%%%%%%%%%%%%%%%%%%%%%%%%%%%%%%%%%%%%%%%
  % Format de la  mise en page
  \setlength{\hoffset}{-1in}
  \setlength{\voffset}{-1in}
  \setlength{\oddsidemargin}{2cm}
  \setlength{\evensidemargin}{2cm}
  \setlength{\topmargin}{1.8cm}
  % \setlength{\headheight}{0cm}
  % \setlength{\headsep}{0cm}
  \setlength{\textwidth}{16.5cm}
  \setlength{\textheight}{24.0cm}


  \usepackage{pst-uml}
  
  
  % pour les environnement d'exemple Latex...
  \usepackage{fancyvrb}
  \usepackage[pstricks]{fvrb-ex}
  % option pour le package fancyverb (pour \VerbatimInput)
  \fvset{%
         frame=single,%
         numbers=left,%
		 baselinestretch=0.9,%
		 gobble=0,% nbr de caratere de debut a ignorer
		 fontsize=\footnotesize%
  }
  
  % \providecommand{\showgrid}{\psgrid[subgriddiv=0, griddots=10]}
  
  \DefineShortVerb{\|}
  % \UndefineShortVerb{\|}  % pour annuler 

  \pagestyle{headings}
  

  %%%%%%%%%%%%%%%%%%%%%%%%%%%%%%%%%%%%%%%%%%%%%%%%%%%%%%%%%%%%%%%%%%%%%% 
  % Quelques commandes locales
  %%%%%%%%%%%%%%%%%%%%%%%%%%%%%%%%%%%%%%%%%%%%%%%%%%%%%%%%%%%%%%%%%%%%%% 
  % \printtime
  % 
  % commande d'impression de l'heure courante
  % 
  % Exemple : "Fichier compil\'{e} le \today{} \`{a} \printtime."
  \usepackage{calc}
  \usepackage{ifthen}
  \newcounter{hours}\newcounter{minutes}
  \newcommand{\printtime}{%
    \setcounter{hours}{\time/60}%
    \setcounter{minutes}{\time-\value{hours}*60}%
    \thehours h%
    % on veut obtenir 15h03mn et non 15h3mn...
    \ifthenelse{\theminutes<10}{0}{}\theminutes mn%
  }
  
  %%%%%%%%%%%%%%%%%%%%%%%%%%%%%%%%%%%%%%%%%%%%%%%%%%%%%%%%%%%%%%%%%%%%%% 
  % Quelques abbreviations
  %%%%%%%%%%%%%%%%%%%%%%%%%%%%%%%%%%%%%%%%%%%%%%%%%%%%%%%%%%%%%%%%%%%%%% 
  
  % Conventions g\'{e}n\'{e}rales pour les formats de pr\'{e}sentation
  \newcommand{\strong}[1]{\textbf{\emph{#1}}} % plus fort que \emph
  \newcommand{\tech}[1]{\textsf{#1}}          % terme technique
  \newcommand{\file}[1]{\texttt{#1}} % noms de fichiers et de r\'{e}pertoires
  \newcommand{\menu}[1]{\fbox{#1}}       % nom d'un menu/sous-menu
  \newcommand{\key}[1]{\fbox{\textbf{#1}}}  % touche du clavier
  
  % abreviations locales a ce document :
  \newcommand{\uml}{\textsc{uml}} 
  \newcommand{\pstricks}{\texttt{PSTricks}} 
  \newcommand{\postscript}{\texttt{PostScript}} 
  \newcommand{\pstuml}{\texttt{pst-uml}} 
  % \newcommand{\bs}{\backslash}
  % \newcommand{\bs}{\backslash}
  
  % Pour les noms de commande TeX
  % Exemple : \cs{fbox} => \fbox
  \DeclareRobustCommand\cs[1]{\texttt{\char`\\#1}}

\usepackage[colorlinks,linktocpage]{hyperref}
%\usepackage{french}
\usepackage[francais]{babel} % idem frenchb mais PAS french !
  % La suite evite que Babel impose un espace devant ":" mais n'est
  % pas disponible sur les vielles versions de Babel (comme � l'ENSTA).
\NoAutoSpaceBeforeFDP
      
%%%%%%%%%%%%%%%%%%%%%%%%%%%%%%%%%%%%%%%%%%%%%%%%%%%%%%%%%%%%%%%%%%%%%%%%
%%%%%%%%%%%%%%%%%%%%%%%%%%%%%%%%%%%%%%%%%%%%%%%%%%%%%%%%%%%%%%%%%%%%%%%%
%%%%%%%%%%%%%%%%%%%%%%%%%%%%%%%%%%%%%%%%%%%%%%%%%%%%%%%%%%%%%%%%%%%%%%%%
\begin{document}

%%%%%%%%%%%%%%%%%%%%%%%%%%%%%%%%%%%%%%%%%%%%%%%%%%%%%%%%%%%%%%%%%%%%%%%%
%%%%%%%%%%%%%%%%%%%%%%%%%%%%%%%%%%%%%%%%%%%%%%%%%%%%%%%%%%%%%%%%%%%%%%%%
\title{Exemple de diagrammes utilisant  \pstuml}
\author{%
   Maurice \textsc{Diamantini}%
   \thanks{avec l'aide pr�cieuse de Denis \textsc{Girou}} %
   (email : \texttt{diam@ensta.fr})
}
\date{%
   Compil� le \today{} � \printtime{}.%
}
\maketitle

\tableofcontents
\clearpage

%%%%%%%%%%%%%%%%%%%%%%%%%%%%%%%%%%%%%%%%%%%%%%%%%%%%%%%%%%%%%%%%%%%%%%%%
%%%%%%%%%%%%%%%%%%%%%%%%%%%%%%%%%%%%%%%%%%%%%%%%%%%%%%%%%%%%%%%%%%%%%%%%
%%%%%%%%%%%%%%%%%%%%%%%%%%%%%%%%%%%%%%%%%%%%%%%%%%%%%%%%%%%%%%%%%%%%%%%%
\clearpage
\section{Exemple de diagramme de classe}

% \documentclass[11pt,a4paper,twoside]{article}
%   \usepackage[T1]{fontenc}
%   \usepackage[applemac]{inputenc}
%   % \usepackage[latin1]{inputenc}
%   \usepackage{pst-uml}
% \begin{document}

%%%%%%%%%%%%%%%%%%%%%%%%%%%%%%%%%%%%%%%%%%%%%%%%%%%%%%%%%%%%%%%%%%%%%%
% Placement des objet} 
%%%%%%%%%%%%%%%%%%%%%%%%%%%%%%%%%%%%%%%%%%%%%%%%%%%%%%%%%%%%%%%%%%%%%%

\newcommand{\drawClassi}{%
  \umlClass{Classe1}{%
     umlClassWidth = 0 \\
     (par d�faut) \\\hline
     Attribut2 Tres tres longue ligne \\ \hline 
     Attribut3 \\ \hline %
     M�thode1%
}}

\newcommand{\drawClassii}{%
  \umlClass[umlClassWidth=4cm,umlParameter={\ T\ }]{Classe2}{%
     largeur = 4cm \\ \hline
     Attribut2 \\ \hline %
     Methode1\\
     Methode2%
}}

\newcommand{\drawClassiii}{%
  \umlClass[umlClassWidth=2.7]{Classe3}{%
     attribut 1 \\
     attribut 1\\ \hline 
     Methode1 \\
     Methode2%
}}

\newcommand{\drawClassiv}{%
  \umlClass{Classe4}{%
     MonAttribut 1 \\
     MonAttribut 2 \\ \hline 
     Methode1%
}}

% Classe5 : largeur automatique et titre seul
\newcommand{\drawClassv}{%
  \umlClass[umlClassWidth=0]{Classe5}{}}

%%%%%%%%%%%%%%%%%%%%%%%%%%%%%%%%%%%%%%%%%%%%%%%%%%%%%%%%%%%%%%%%%%%%%%
% Placement des objets
%%%%%%%%%%%%%%%%%%%%%%%%%%%%%%%%%%%%%%%%%%%%%%%%%%%%%%%%%%%%%%%%%%%%%%

\begin{pspicture}(18,15)\psgrid
  \rput(3,13){\rnode{Class1}{\drawClassi}}
  \pnode(17.5,13){pnode1}
  \rput(9,10){\rnode{Class2}{\drawClassii}}
  \rput(2,5){\rnode{Class3}{\drawClassiii}}
  \rput(12,5){\rnode{Class4}{\drawClassiv}}
  \rput(5.5,5.5){\rnode{Class5}{\drawClassv}}
  %
  \rput(16,11){\rnode{Actor1}{\umlActor{Acteur(�) 1}}}
\end{pspicture}
%%%%%%%%%%%%%%%%%%%%%%%%%%%%%%%%%%%%%%%%%%%%%%%%%%%%%%%%%%%%%%%%%%%%%%
% Dessin des liens et labels
%%%%%%%%%%%%%%%%%%%%%%%%%%%%%%%%%%%%%%%%%%%%%%%%%%%%%%%%%%%%%%%%%%%%%%
% La grande boucle en deux �tapes :
\ncline{Class1}{pnode1}
\ncputicon[npos=0.7,nrot=:U]{umlV}
\naput{ncline}\naput[npos=1,ref=r]{Node "P1"}
\ncSXE[armA=11.5]{pnode1}{Class3}
\nbput{SXE (armA=11.5)}
\ncputicon{umlV}% debut
\ncputicon[npos=1.9999,nrot=:U]{umlV}
\ncputicon[npos=2,nrot=:U]{umlV}
\ncputicon[npos=5,nrot=:U]{umlV}% fin ERREUR si nrot=4 ok pour 5!!!!
%
\ncSE{Class1}{Class2}
\naput[npos=1.5]{\{ncSE npos=1.5\}}
\ncSE[offset=-1]{Class1}{Class2}
\ncputicon{umlAgreg} % debut
\ncputicon[npos=2,nrot=:U]{umlCompos}% fin
\nbput[npos=0.3]{0..*}
\naput[npos=1.8]{0..2}
\naput[npos=1.4]{ncSE,offset=-1}
%
\ncSHS[armA=1.5]{Class2}{Class4}\naput{ncSHS}
\ncSHS[armA=1.5]{Class2}{Class3}\nbput{ncSHS}
\ncputicon{umlHerit}%       h�ritage au debut
\ncputicon[npos=3,nrot=:U]{umlV}% V en fin
%
\ncSHN[arm=.7]{Class3}{Class4}
\naput{ncSHN (3 vers 4)}
\ncputicon[npos=1.8,nrot=:U]{umlV}% fleche au milieu vers destination !
%
% \ncE[npos=0.4]{Class5}{Class4}\naput{ncE,npos=0.4}
\ncE{Class5}{Class4}\naput[npos=0.4]{ncE,npos=0.4}
\ncputicon{umlCompos}
%
% Essai de d�finition d'un style personnalis�
\newpsstyle{umlDependance}{%
     linestyle=dashed,
     arrows=->,
     arrowscale=3,
     arrowinset=0.6
}
\ncline[style=umlDependance,offset=-0.5]{Class3}{Class4}
\naput{ncline}
\ncputicon{umlV}% fleche au debut
\nbput[npos=0.15]{1..*}
%
% % % On peut coller n'importe quoi par rapport � un node :
% % % Essai pour mettre un template sur une classe : pr�voir 
% % % une option du style [umlTemplate=myString]
% % \nput*[labelsep=-0.8,offset=1.4]%
% %    {0}{Class2}{\psframebox%
% %    [fillstyle=solid,fillcolor=white,linestyle=dashed]%
% %    {\LARGE\textbf{\ T\ }}}
% % % 
% Lien de Class2 et Class4 vers l'acteur :
\ncline[linestyle=dashed]{Class2}{Actor1}
\naput{ncline}
\ncputicon{umlAgreg}
\ncputicon[npos=0.7,nrot=:U]{umlAgreg}
\ncputicon[npos=1,nrot=:U]{umlCompos} % en fin
%  
\nccurve[linestyle=dashed, angleA=75,offsetA=-1,angleB=-45]{Class4}{Actor1}
\ncputicon{umlHerit} % debut
\ncputicon[npos=0.7,nrot=:U]{umlHerit}
\ncputicon[npos=1,nrot=:U]{umlHerit}% en fin

% \end{document}

\VerbatimInput{diagClass.tex}

%%%%%%%%%%%%%%%%%%%%%%%%%%%%%%%%%%%%%%%%%%%%%%%%%%%%%%%%%%%%%%%%%%%%%%%%
%%%%%%%%%%%%%%%%%%%%%%%%%%%%%%%%%%%%%%%%%%%%%%%%%%%%%%%%%%%%%%%%%%%%%%%%
%%%%%%%%%%%%%%%%%%%%%%%%%%%%%%%%%%%%%%%%%%%%%%%%%%%%%%%%%%%%%%%%%%%%%%%%
\clearpage
\section{Exemple de diagramme des cas d'utilisation}


% \documentclass[11pt,a4paper,twoside]{article}
%   \usepackage[T1]{fontenc}
%   \usepackage[applemac]{inputenc}
%   % \usepackage[latin1]{inputenc}
%   \usepackage{pst-uml}
% \begin{document}

\begin{center}
   % \scalebox{0.5}{%}  Fonctionne egalement
   \resizebox{0.9\linewidth}{!}{%
     \begin{pspicture}(0,0.5)(15,13.5)%\psgrid
     \psset{framesep=0}
     % 
     \psframe[linewidth=0.5pt, linestyle=dashed](3,14)(12,0.5)
     \rput(7.5,1){\Large Syst�me � d�velopper}
     % 
     \rput(1,12){\rnode{acCL}{\umlActor{Client}}}
     \rput(1,6){\rnode{acSC}{\umlActor{Service\\Client�le}}}
     \rput(14,4){\rnode{acST}{\umlActor{Service\\Technique}}}
     \rput(14,11.5){\rnode{acSU}{\umlActor{Superviseur}}}
     % 
     % 
     % \umlPutCase{5,13}{VISU}{\\[0mm]Visualiser\\[0mm]}
     \umlPutCase{5,13}{VISU}{Visualiser}
     \umlPutCase{5,5}{SR}{Saisir\\Risques}
     \umlPutCase{5,3}{SD}{Saisir\\Demandes}
     \umlPutCase{5,8}{RD}{Routage\\demandes}
     \umlPutCase{10,10}{ICB}{%
              Identifier\\
              contraintes\\
              bloquantes}
     \umlPutCase{10,3}{MR}{Modifie\\�R�seau�}
     \umlPutCase{5,11}{VAD}{V�rifier\\�acceptation\\demande�}
     \umlPutCase{10,5}{IR}{Indiquer\\risques�}
     \umlPutCase{10,13}{PM}{Pr�parer\\modif�.}
     %
     \ncline{acCL}{VAD}
     \ncline{acCL}{VISU}
     \ncline{acSC}{SR}
     \ncline{acSC}{SD}
     \ncline{acSC}{RD}
     \ncline{acSU}{ICB}
     \ncline{acSU}{PM}
     \ncline{acST}{IR}
     \ncline{acST}{MR}
     % 
     \ncline{RD}{ICB}\naput[nrot=:U]{\umlStereoType{uses}}
     \ncputicon{umlHerit}
     \ncline{IR}{SR}\nbput[nrot=:D,npos=0.65]{\umlStereoType{uses}}
     \ncputicon{umlHerit}
     \end{pspicture}
   }%end resizeORscalebox
\end{center}


% \end{document}

\VerbatimInput{diagCase.tex}

%%%%%%%%%%%%%%%%%%%%%%%%%%%%%%%%%%%%%%%%%%%%%%%%%%%%%%%%%%%%%%%%%%%%%%%%
%%%%%%%%%%%%%%%%%%%%%%%%%%%%%%%%%%%%%%%%%%%%%%%%%%%%%%%%%%%%%%%%%%%%%%%%
%%%%%%%%%%%%%%%%%%%%%%%%%%%%%%%%%%%%%%%%%%%%%%%%%%%%%%%%%%%%%%%%%%%%%%%%
\clearpage
\section{Exemple de diagramme de s�quences}


% \documentclass[11pt,a4paper,twoside]{article}
%   \usepackage[T1]{fontenc}
%   \usepackage[applemac]{inputenc}
%   % \usepackage[latin1]{inputenc}
%   \usepackage{pst-uml}
% \begin{document}

%%%%%%%%%%%%%%%%%%%%%%%%%%%%%%%%%%%%%%%%%%%%%%%%%%%%%%%%%%%%%%%%%%%%%%
% Placement des objet} 
%%%%%%%%%%%%%%%%%%%%%%%%%%%%%%%%%%%%%%%%%%%%%%%%%%%%%%%%%%%%%%%%%%%%%%

\begin{center}
\resizebox{\linewidth}{!}{%
  \begin{psmatrix}[colsep=0.2,rowsep=0.5]
     % 
     % la ligne 1 contient le nom des objets 
       [name=client]\umlClass{\underline{:Client}}{}
     & [name=reseau]\umlClass{\underline{:R\'eseau}}{} 
     & [name=demande]\umlClass{\underline{:Demande}}{} 
     & [name=route]\umlClass{\underline{:Route}}{}     
     & [name=arete]\umlClass{\underline{:Ar\^ete}}{}   
     & [name=noeud]\umlClass{\underline{:Noeud}}{}     
     & [name=fenetre]\umlClass{\underline{:Fen\^etre}}{} 
     \\[+0.5cm] %1
     % ATTENTION les lignes vides telle que :
     %     &  &  &  &  &  &  \\
     % sont inaccessible par (3,2)
     % 
     {} & {} & {} & {} & {} & {} & {} \\
     {} & {} & {} & {} & {} & {} & {} \\
     {} & {} & {} & {} & {} & {} & {} \\
     {} & {} & {} & {} & {} & {} & {} \\ % 5
         %
     {} & {} & {} & {} & {} & {} & {} \\
     {} & {} & {} & {} & {} & {} & {} \\
     {} & {} & {} & {} & {} & {} & {} \\
     {} & {} & {} & {} & {} & {} & {} \\[-0.5cm]
     {} & {} & {} & {} & {} & {} & {} \\[-0.5cm] % 10
         %
     {} & {} & {} & {} & {} & {} & {} \\[-0.5cm]
     {} & {} & {} & {} & {} & {} & {} \\[-0.5cm]
     {} & {} & {} & {} & {} & {} & {} \\[+0.5cm]
     {} & {} & {} & {} & {} & {} & {} \\
     {} & {} & {} & {} & {} & {} & {} \\[+0.5cm] % 15
         %
     {} & {} & {} & {} & {} & {} & {} \\
     {} & {} & {} & {} & {} & {} & {} \\
     {} & {} & {} & {} & {} & {} & {} \\[-0.5cm]
     {} & {} & {} & {} & {} & {} & {} \\
     {} & {} & {} & {} & {} & {} & {} \\ % 20
     % 
     {} & {} & {} & {} & {} & {} & {} \\[0cm] % 21 ([0cm] n�cessaire : bug ?)
     %
     % Les noms pour les fins d'objets (invariant si nouvelles lignes)
     [name=clientEnd]{}
       & [name=reseauEnd]{} 
       & [name=demandeEnd]{}
       & [name=routeEnd]{}   
       & [name=areteEnd]{}  
       & [name=noeudEnd]{}  
       & [name=fenetreEnd]{}
       & \\[-0.5cm] % Saut de ligne sans vertic pour corrig� probl�me 
     % 
     % Le trait d'axe pour l'�chelle des temps :
     \ncline[linewidth=0.5pt,linestyle=solid,offset=-1.7,nodesep=0.0]%
            {->}{client}{clientEnd}
     \naput[npos=1]{\emph{t}}
     % 
     % Les pointill�s verticaux
     \ncline[linestyle=dashed]{client}{clientEnd}
     \ncline[linestyle=dashed]{reseau}{reseauEnd}
     \ncline[linestyle=dashed]{demande}{demandeEnd}
     \ncline[linestyle=dashed]{route}{routeEnd}
     \ncline[linestyle=dashed]{arete}{areteEnd}
     \ncline[linestyle=dashed]{noeud}{noeudEnd}
     \ncline[linestyle=dashed]{fenetre}{fenetreEnd}
     %
     % Les connexions horisontales ave leur commentaires associ�s
     \small\ttfamily% Fonctionne
     \psset{labelsep=1.5mm}
     \ncline{->}{2,1}{2,3}\naput*{listerDemandes()}
     \ncline{->}{3,3}{3,1}\nbput*{demandes}
     \ncline{->}{4,1}{4,2}\naput*{* visualiser(demande)}
     \ncline{->}{5,2}{5,1}\nbput*{[d�j�Rout�(r�seau) = false]}
     \ncline{->}{6,2}{6,1}\nbput*{[accept�e(demande) = false]}
     \ncline{->}{7,2}{7,4}\naput*{[accept�e(demande) = true] parcourir()}
     \ncline{->}{8,4}{8,5}\naput*{lister()}
     \ncline{->}{9,5}{9,6}\naput*{listerExtr�mit�s()}
     \ncline{->}{10,6}{10,5}
     \ncline{->}{11,5}{11,4}
     \ncline{->}{12,4}{12,2}
     \ncline{->}{13,2}{13,1}
     \ncline{->}{14,1}{14,7}\naput*{[accept�e(demande) = true] %
                                        afficher(demande)}
     \ncline{->}{15,7}{15,1}\nbput*{dessinerSurTerminal()}
     \ncline{->}{16,1}{16,7}\naput*{* zoomer(zone)}   
     \ncline{->}{17,7}{17,5}\nbput*{ar�tesInZone}
     \ncline{->}{18,5}{18,3}\nbput*{estConcern�e(demande)}
     \ncline{->}{19,3}{19,5}
     \ncline{->}{20,5}{20,7}\naput*{ar�tesConcern�es}
     \ncline{->}{21,7}{21,1}\nbput*{rafraichirEcran()}
     % \ncEVW[armA=2]{->}{4,3}{10,3}  % Est Vertical West
     % 
  \end{psmatrix}
}%end resizeORscalebox
\end{center}


% \end{document}

\VerbatimInput{diagSeq.tex}

%%%%%%%%%%%%%%%%%%%%%%%%%%%%%%%%%%%%%%%%%%%%%%%%%%%%%%%%%%%%%%%%%%%%%%%%
%%%%%%%%%%%%%%%%%%%%%%%%%%%%%%%%%%%%%%%%%%%%%%%%%%%%%%%%%%%%%%%%%%%%%%%%
%%%%%%%%%%%%%%%%%%%%%%%%%%%%%%%%%%%%%%%%%%%%%%%%%%%%%%%%%%%%%%%%%%%%%%%%
\clearpage
\section{Exemple de diagramme d'�tats}


% \documentclass[11pt,a4paper,twoside]{article}
%   \usepackage[T1]{fontenc}
%   \usepackage[applemac]{inputenc}
%   % \usepackage[latin1]{inputenc}
%   \usepackage{pst-uml}
% \begin{document}

%%%%%%%%%%%%%%%%%%%%%%%%%%%%%%%%%%%%%%%%%%%%%%%%%%%%%%%%%%%%%%%%%%%%%%
% d�finition des objets
%%%%%%%%%%%%%%%%%%%%%%%%%%%%%%%%%%%%%%%%%%%%%%%%%%%%%%%%%%%%%%%%%%%%%%

\newcommand{\StateGlobal}{%
  \umlState{�tat global de l'objet \texttt{Graphe}}{\umlEmptyBox{13cm}{16cm}}%
}
\newcommand{\StateNRSA}{%
  \umlState{non rout� \\ sans ar�tes}{\space}%
}
\newcommand{\StateNRI}{%
  \umlState{non rout� \\ incomplet}{\space}%
}
\newcommand{\StateNRC}{%
  \umlState{non rout� \\ complet}{\space}%
}
\newcommand{\StateROU}{%
  \umlState{rout� \\ \mbox{}}{\space}%
}
\newcommand{\StateVisu}{%
  \umlState{Visualisable \\ \mbox{}}{do/superviser()}%
}
\newcommand{\StateAnu}{%
  \umlState{GrapheAnnulable}{%
    \hspace*{2.25cm}
    \rmfamily% car normalement un corps d'�tat est en ttfamily
    \begin{psmatrix}[colsep=1,rowsep=1.5,mnode=r]
                                       \\[-1.4cm]
       [name=StateInAnu] \umlStateIn   \\[-0.5cm]
       [name=StateNRSA]   \StateNRSA     \\[0cm]
       [name=StateNRI]  \StateNRI    \\[1cm]
       [name=StateNRC]   \StateNRC     \\[0.5cm]
       [name=StateROU]   \StateROU   
          &   \umlPutStateOut{0,0}{StateOutAnu}  \\[-1.5cm]
          {} % boite vide NECESSAIRE sur la derni�re ligne si vide !
    \end{psmatrix}%
    \hspace*{1.5cm}
    %
    % Connection propre � chaque node
    % 
    {\ttfamily\small
      % 
      % \nput{75}{StateInAnu}{Entr�eAnnulable}
      % \nput{-75}{StateOutAnu}{SortieAnnulable}
      % 
      \ncEXS[offsetA=0.25,offsetB=0.5]{StateNRSA}{StateNRSA}%
      \ncput*[npos=1.7]{ajouterSommet}
      \ncWXS[offsetA=-0.25,offsetB=-0.5]{StateNRSA}{StateNRSA}%
      \ncput*[npos=1.7]{retirerSommet}
      %
      \ncEXS[offsetA=0.25,offsetB=0.5]{StateNRI}{StateNRI}%
      \ncput*[npos=1.7]{ajouterAr�te}
      \ncWXS[offsetA=-0.25,offsetB=-0.5]{StateNRI}{StateNRI}%
      \ncput*[npos=1.7]{retirerAr�te}
      \ncEXN[offsetA=-0.0,offsetB=-0.5]{StateNRI}{StateNRI}%
      \ncput*[npos=1.7]{ajouterSommet}
      \ncWXN[offsetA=0.0,offsetB=0.5]{StateNRI}{StateNRI}%
      \ncput*[npos=1.7]{retirerSommet}
      %
      \ncEXS[offsetA=0.25,offsetB=0.5]{StateNRC}{StateNRC}%
      \ncput*[npos=1.7]{ajouterRoute}
      \ncWXS[offsetA=-0.25,offsetB=-0.5]{StateNRC}{StateNRC}%
      \ncput*[npos=1.7]{retirerRoute}
      % 
      \ncWXS[offsetA=-0.25,offsetB=-0.5]{StateROU}{StateROU}%
      \ncput*[npos=1.7]{r�optimiser}
      % 
      % Connections INTERNODE interne au macro-�tat
      %
      \ncline{->}{StateInAnu}{StateNRSA}%
         \naput[npos=0.3]{}%
      %
      \ncline{->}{StateNRSA}{StateNRI}%
         \naput[npos=0.3]{ajouterAr�te}%
      %
      \ncline{->}{StateNRI}{StateNRC}%
         % \naput[npos=0.3]{graphComplet}%
         \naput{graphComplet}%
      %
      \ncline{->}{StateNRC}{StateROU}%
         \ncput*[npos=0.3]{\umlStack{[ClientPrioritaireSatisfait] DemandeFin}}
      %
      \ncline{->}{StateROU}{StateOutAnu}%
      %
    }%
  }%
}

%%%%%%%%%%%%%%%%%%%%%%%%%%%%%%%%%%%%%%%%%%%%%%%%%%%%%%%%%%%%%%%%%%%%%%
% Placement des objets
%%%%%%%%%%%%%%%%%%%%%%%%%%%%%%%%%%%%%%%%%%%%%%%%%%%%%%%%%%%%%%%%%%%%%%
\begin{center}
  % \scalebox{0.5}{%}  Fonctionne egalement
  % \resizebox{0.9\linewidth}{!}{%}
  \resizebox{!}{13cm}{%
    \begin{pspicture}(-5,-9.5)(9.5,8)%\psgrid
      %
      \psset{%
         linearc=0.3,%
         % arm=1.2,%
         armA=1.2,%
         armB=0.8,%
         arrows=->,%
         arrowscale=2,%
         ncurv=2,% instead of 0.67
      }%
      \rput(2.4,-0.75){\rnode{StateGlobal}{\StateGlobal}}%
      %
      \umlPutStateIn{6,5}{StateIn}%
      %
      \rput(0,-1){\rnode{StateAnu}{\StateAnu}}%
      %
      \rput(6,-3){\rnode{StateVisu}{\StateVisu}}%
      %
      \rput(6,0){\rnode{StateOut}{\umlStateOut}}%
      %
      % LES CONNECTIONS INTERNODE
      %
      {\ttfamily\small
        %
        \ncEXN[offsetA=-0.25,offsetB=-0.5,armA=0.5]{StateVisu}{StateVisu}%
          \ncput*[npos=1.7]{zoomer}
        % 
        \ncSW[offsetB=-5]{->}{StateIn}{StateAnu}%
          \naput[npos=1.3]{NewGraphAsked}
        %
        \ncEN{->}{StateOutAnu}{StateVisu}%
          \nbput[npos=0.9]{/Sauvegarder}
        \ncline{->}{StateVisu}{StateOut}%
        %
        \ncES[offsetA=3]{->}{StateAnu}{StateOut}%
          % \naput[npos=0.6]{Annuler}
		  % on ajoute un espace en d�but de chaque ligne " Annuler"..
          \naput[npos=0.99]{\umlStack[umlAlign=l]%
                  {\ Annuler\\\ /DemanderConfirmation}}
      }
    \end{pspicture}%
  }%end resizeORscalebox
\end{center}


% \end{document}

\VerbatimInput{diagState.tex}

%%%%%%%%%%%%%%%%%%%%%%%%%%%%%%%%%%%%%%%%%%%%%%%%%%%%%%%%%%%%%%%%%%%%%%%%
%%%%%%%%%%%%%%%%%%%%%%%%%%%%%%%%%%%%%%%%%%%%%%%%%%%%%%%%%%%%%%%%%%%%%%%%
%%%%%%%%%%%%%%%%%%%%%%%%%%%%%%%%%%%%%%%%%%%%%%%%%%%%%%%%%%%%%%%%%%%%%%%%

% pour connaitre le numero de la derniere page
\label{verylast}\mbox{}

\end{document}


