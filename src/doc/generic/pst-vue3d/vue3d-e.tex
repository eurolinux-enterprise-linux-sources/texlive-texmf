\listfiles
\documentclass[english]{article}
\usepackage[T1]{fontenc}
\usepackage[latin1]{inputenc}
\usepackage{lmodern}% only for PDF output
\usepackage[a4paper,bmargin=2cm,tmargin=2cm]{geometry}
\usepackage{url}
\usepackage{morefloats}
\setcounter{totalnumber}{10}
\setcounter{dbltopnumber}{10}
\renewcommand{\textfraction}{0}
\usepackage{subfigure}
% Mluque5130@aol.com
% 17 octobre 2003
% Herbert Voss <voss@pstricks.de>
% September 2004
\def\UrlFont{\small\ttfamily}
\makeatletter
\def\verbatim@font{\footnotesize\normalfont\ttfamily}
\makeatother
\usepackage[colorlinks,linktocpage]{hyperref}
\usepackage[english]{babel}
\usepackage{pstricks,multido,pst-grad}
\usepackage{pst-vue3d}
\let\VueFversion\fileversion
\usepackage{pst-example}
%
\definecolor{GrisClair} {rgb}{0.6,0.7,0.8}
\definecolor{GrisTresClair} {rgb}{0.8,0.9,0.7}
\definecolor{GrayA} {rgb}{0.35,0.95,0.95}
\definecolor{GrayB} {rgb}{0.85,0.85,0.35}
\definecolor{GrayC} {rgb}{0.75,0.35,0.55}
\definecolor{GrayD} {rgb}{0.65,0.65,0.65}
\definecolor{GrayE} {rgb}{0.7,0.9,0.65}
\definecolor{LightBlue}{rgb}{.68,.85,.9}
%
\newcommand\tapis{%
	\psset{normaleLatitude=90,normaleLongitude=0}
	\FrameThreeD[fillcolor=green,fillstyle=solid](0,0,-5)(-20,-20)(20,20)
	\QuadrillageThreeD[grille=10](0,0,-5)(-20,-20)(20,20)%
}
%
\def\Table{{%
    \CubeThreeD[A=30,B=30,C=2,CubeColorFaceOne={.7 .6 .5}](0,0,-2)
    \psset{normaleLongitude=0,normaleLatitude=90}
    \QuadrillageThreeD[linewidth=0.2mm,linecolor=white,%
		grille=5](0,0,0)(-30,-30)(30,30)
}}
%
\def\DessusTable{{%
    \psset{normaleLongitude=0,normaleLatitude=90}
    \QuadrillageThreeD[linewidth=0.2mm,linecolor=gray,%
		grille=5](0,0,0)(-30,-30)(30,30)%
}}
\def\PlansOXYZ{{%
	\psset{normaleLongitude=0,normaleLatitude=90}
	\FrameThreeD[fillstyle=solid,fillcolor=GrisClair](0,0,0)(-50,0)(0,50)
	\QuadrillageThreeD[linewidth=0.2mm,grille=10](0,0,0)(-50,0)(0,50)%
	\psset{normaleLongitude=90,normaleLatitude=0}
	\FrameThreeD[fillstyle=solid,fillcolor=GrisTresClair](0,0,0)(0,0)(50,-50)
	\QuadrillageThreeD[linewidth=0.2mm,grille=10](0,0,0)(0,-50)(50,0)%
	\psset{normaleLongitude=0,normaleLatitude=0}
	\FrameThreeD[fillstyle=solid,fillcolor=GrisTresClair](0,0,0)(-50,0)(0,-50)
	\QuadrillageThreeD[linewidth=0.2mm,grille=10](0,0,0)(-50,-50)(0,0)%
	}}
\psset{CubeColorFaceOne=1 1 1,%
       CubeColorFaceTwo=1 0 0,%
       CubeColorFaceThree=0 1 0,%
       CubeColorFaceFour=0 0 1,%
       CubeColorFaceFive=1 1 0,%
       CubeColorFaceSix=0 1 1}
%
\def\hexagon{%
\begin{pspicture}(-2.2,-2.2)(2.2,2)
    \Table
    \pNodeThreeD(-8.66,-5,0){A6}
    \pNodeThreeD(-8.66,5,0){A1}
    \pNodeThreeD(0,10,0){A2}
    \pNodeThreeD(8.66,5,0){A3}
    \pNodeThreeD(8.66,-5,0){A4}
    \pNodeThreeD(0,-10,0){A5}%
    \psclip{\pspolygon[fillstyle=solid,fillcolor=GrisClair,%
        linestyle=none](A1)(A2)(A3)(A4)(A5)(A6)}
        \DessusTable
    \endpsclip
    \psset{A=5,B=5,C=5}
    \CubeThreeD[RotZ=60](-6.83,-11.830,5)%6
    \CubeThreeD[RotZ=120](6.83,-11.830,5)%5
    \CubeThreeD(-13.86,0,5)%1
    \CubeThreeD[RotZ=-60](-6.83,11.830,5)%2
    \CubeThreeD[RotZ=-120](6.83,11.830,5)%3
    \CubeThreeD[RotZ=180](13.86,0,5)%4
\end{pspicture}%
}
%
\def\stardodecagon{%
    \begin{pspicture}(-2.2,-2)(2.2,2.2)
    \Table
    \pNodeThreeD(-6.83,-11.83,0){A6}%
    \pNodeThreeD(-13.86,0,0){A1}%
    \pNodeThreeD(-6.83,11.83,0){A2}%
    \pNodeThreeD(6.83,11.83,0){A3}%
    \pNodeThreeD(13.86,0,0){A4}%
    \pNodeThreeD(6.83,-11.83,0){A5}%
    \psclip{\pspolygon[fillstyle=solid,fillcolor=GrisClair,%
        linestyle=none](A1)(A2)(A3)(A4)(A5)(A6)}
        \DessusTable
    \endpsclip%
    \psset{A=5,B=5,C=5}
    \CubeThreeD[RotZ=105](-10.6066,6.12372,5)%2
    \CubeThreeD[RotZ=45](0,12.2474,5)%1
    \CubeThreeD[RotZ=345](10.6066,6.12372,5)%6
    \CubeThreeD[RotZ=165](-10.6066,-6.12372,5)%3
    \CubeThreeD[RotZ=225](0,-12.2474,5)%4
    \CubeThreeD[RotZ=285](10.6066,-6.12372,5)%5
\end{pspicture}}
%
\def\pentagon{%
    \begin{pspicture}(-2.2,-2.2)(2.2,2.2)
    \Table
    \pNodeThreeD(8.5065,0,0){A1}%
    \pNodeThreeD(2.6287,8.09,0){A2}%
    \pNodeThreeD(-6.882,5,0){A3}%
    \pNodeThreeD(-6.882,-5,0){A4}%
    \pNodeThreeD(2.6287,-8.09,0){A5}%
    \psclip{\pspolygon[fillstyle=solid,fillcolor=GrisClair,%
        linestyle=none](A1)(A2)(A3)(A4)(A5)}
        \DessusTable
    \endpsclip%
    \psset{A=5,B=5,C=5}
    \CubeThreeD(-11.88,0,5)%1
    \CubeThreeD[RotZ=72](-3.617,-11.3,5)%5
    \CubeThreeD[RotZ=-72](-3.617,11.3,5)%2
    \CubeThreeD[RotZ=-144](9.61267,6.984,5)%3
    \CubeThreeD[RotZ=144](9.61267,-6.984,5)%4
\end{pspicture}}
%
\def\stardecagon{%
    \begin{pspicture}*(-2.2,-1.75)(2.2,2.2)
    \Table
    \pNodeThreeD(-12.03,0,0){A1}%
    \pNodeThreeD(-3.7178,-11.44,0){A2}%
    \pNodeThreeD(9.7325,-7.071,0){A3}%
    \pNodeThreeD(9.7325,7.071,0){A4}%
    \pNodeThreeD(-3.7178,11.44,0){A5}%
    \psclip{\pspolygon[fillstyle=solid,fillcolor=GrisClair,%
        linestyle=none](A1)(A2)(A3)(A4)(A5)}
        \DessusTable
    \endpsclip%
    \psset{A=5,B=5,C=5}
    \CubeThreeD[RotZ=81](-7.87375,-5.72061,5)%4
    \CubeThreeD[RotZ=9](-7.87375,5.72061,5)%3
    \CubeThreeD[RotZ=153](3.0075,-9.2561,5)%5
    \CubeThreeD[RotZ=-63](3.0075,9.25615,5)%2
    \CubeThreeD[RotZ=-135](9.73249,0,5)%1
\end{pspicture}%
}
\def\octogon{%
    \begin{pspicture}(-2.2,-2.2)(2.2,2.2)
    \Table
    \pNodeThreeD(12.07,5,0){A1}%
    \pNodeThreeD(5,12.07,0){A2}%
    \pNodeThreeD(-5,12.07,0){A3}%
    \pNodeThreeD(-12.07,5,0){A4}%
    \pNodeThreeD(-12.07,-5,0){A5}%
    \pNodeThreeD(-5,-12.071,0){A6}%
    \pNodeThreeD(5,-12.07,0){A7}%
    \pNodeThreeD(12.07,-5,0){A8}%
    \psclip{\pspolygon[fillstyle=solid,fillcolor=GrisClair,%
        linestyle=none](A1)(A2)(A3)(A4)(A5)(A6)(A7)(A8)}
        \DessusTable
    \endpsclip%
    \psset{A=5,B=5,C=5}
    \CubeThreeD(-17.07,0,5)%5
    \CubeThreeD[RotZ=45](-12.07,-12.07,5)%6
    \CubeThreeD[RotZ=90](0,-17.07,5)%7
    \CubeThreeD[RotZ=135](12.07,-12.07,5)%8
    \CubeThreeD[RotZ=-45](-12.07,12.07,5)%4
    \CubeThreeD[RotZ=-90](0,17.07,5)%3
    \CubeThreeD[RotZ=-135](12.07,12.07,5)%2
    \CubeThreeD[RotZ=180](17.07,0,5)%1
\end{pspicture}%
}
%
\def\starhexadecagon{%
    \begin{pspicture}(-2.2,-2)(2.2,2.2)
    \Table
    \pNodeThreeD(17.07,7.07,0){A1}%
    \pNodeThreeD(7.07,17.07,0){A2}%
    \pNodeThreeD(-7.07,17.07,0){A3}%
    \pNodeThreeD(-17.07,7.07,0){A4}%
    \pNodeThreeD(-17.07,-7.07,0){A5}%
    \pNodeThreeD(-7.07,-17.07,0){A6}%
    \pNodeThreeD(7.07,-17.07,0){A7}%
    \pNodeThreeD(17.07,-7.07,0){A8}%
    \psclip{\pspolygon[fillstyle=solid,fillcolor=GrisClair,%
        linestyle=none](A1)(A2)(A3)(A4)(A5)(A6)(A7)(A8)}
        \DessusTable
    \endpsclip%
    \psset{A=5,B=5,C=5}
    \CubeThreeD[RotZ=225](-17.07,0,5)%5
    \CubeThreeD[RotZ=-90](-12.07,-12.07,5)%6
    \CubeThreeD[RotZ=-45](0,-17.07,5)%7
    \CubeThreeD(12.07,-12.07,5)%8
    \CubeThreeD[RotZ=180](-12.07,12.07,5)%4
    \CubeThreeD[RotZ=135](0,17.07,5)%3
    \CubeThreeD[RotZ=90](12.07,12.07,5)%2
    \CubeThreeD[RotZ=45](17.07,0,5)%1
\end{pspicture}}
%
\def\DecorSable{%
  \FrameThreeD[normaleLongitude=0,normaleLatitude=90,%
     fillstyle=solid,fillcolor=GrayE](0,0,0)(-60,-60)(60,60)
  \QuadrillageThreeD[normaleLongitude=0,normaleLatitude=90,%
     linecolor=GrayA,linewidth=0.2mm,grille=10](0,0,0)(-60,-60)(60,60)%
}
\newpsstyle{GradGrayWhite}{fillstyle=gradient,%
    gradbegin=blue,gradend=white,linewidth=0.1mm}%

\begin{document}

\title{3D views with \texttt{pst-vue3d}\\[3ex]
 \normalsize (v. \VueFversion)}
\author{Manuel Luque\thanks{\url{mluque5130 _at_ aol.com}}\ 
and Herbert Vo\ss\thanks{\url{voss _at_ pstricks.de}}}

\maketitle
\tableofcontents
\clearpage

\section{Presentation}
The 3D representation of an object or a landscape is one of the
most interesting subject in computer science and have many
industrial applications (car and plane design, video game
etc\ldots). In a smaller way, one can obtain very didactic
realizations using PSTricks with two peculiarities:
\begin{itemize}
  \item using PostScript;
  \item being manageable through \LaTeX.
\end{itemize}
Package \texttt{pst-key} of David \textsc{Carlisle} allows to
write commands with parameters. Using this as an interface, one
can observe the result of little modifications of some parameters.
Our parameters being here: the position of the watcher, the choice
of an solid (cube, sphere etc\ldots) and many other things. I want
to signal that
\begin{itemize}
\item
Regarding 3D representation, one does not forget the package pst-3d by Timothy Van Zandt
who has used the best part of Post\-Script. Althrought limited to parallel projections,
this package allows to draw very interesting 3D figure.\footnote{A lot of different examples
for 3D images are available at: \url{http://members.aol.com/Mluque5130/}}
\item Thanks to Denis \textsc{Girou}, i have discovered the
package \texttt{pst-xkey} and I have learned it.
\item I have written another package for drawing picture reflecting
in  spherical mirrors.%
\footnote{\url{http://melusine.eu.org/syracuse/mluque/BouleMiroir/boulemiroir.html}}

It is a french paper which illustrate a study of Pr. Henri
\textsc{Bouasse} from this book \textit{Optique sup\'erieure}, edited in $1917$ by  Delagrave.
\end{itemize}


\section{Aims}
First, we want to draw the 3D representation with elimination of
the hidden parts of some objects.

The position of  the watcher will be defined by its spherical
coordinates: the distances from the origin, the longitude $\theta$
and the latitude $\phi$. We will choose, too, the distance of the
projection screen from this point.

Second, we want to define some $3D$ elements of the scene: the bricks.

The following bricks are already defined
\begin{itemize}
\item A box given by its three dimensions \verb+A,B,C+: it could
be turn into a cube or a dice.
\item A point which can be defined it two ways
\begin{itemize}
\item By cartesian coordinates $(x,y,z)$
\item Or by  spherical coordinates $(R,\theta,\phi)$
 ($\theta$, $\phi$ are, respectively, longitude and latitude).
\end{itemize}
\item A rectangle.
\item A circle defined by the normal line to its plane, its center
and its radius. An arc is defined as the circle with two limit
angles.
\item A tetrahedron given by the coordinates of the center of its
base and the radius of the circle containing the vertex of each
faces.  We can make it rotate.
\item A square pyramid given by the half of the length of the side
of its base and its height. We can make it rotate and move.
\item A sphere given by the coordinates of its center \verb+\SphereThreeD(x,y,z){Radius}+
and its radius. We can make it rotate with the parameters
\verb+RotX=...+, \verb+RotY=...+, \verb+RotZ=...+ We can choose to
draw only some meridians and parallel circles.
 \item A  solid or empty half-sphere (same parameters than a sphere)
 \item A vertical cylinder defined by its radius and its height. We
 can make it rotate using the parameters \verb+RotX=...+, \verb+RotY=...+, \verb+RotZ=...+
 An we can choose the center of its base in the same way than the Sphere.
\item A cone and a truncated cone defined by the radius of their
base, the height and the height of the truncature.
\end{itemize}

\vspace*{1cm}
To construct a scene, one may choose himself the order of the
objects. For example, if an object 1 is partially hidden by an
object 2, we write, in the list of commands, first object 1 and
second object 2.

\section{Rotating in the 3D space}

A 3D object can be rotated around every axes with the \verb+RotX+, \verb+RotY+ and
\verb+RotZ+ option. They can be mixed in every combination. Figure~\ref{fig:rot} shows
how a rotation around the z-axes works.

\begin{figure}[!htb]
\multido{\iRotZ=0+45}{8}{%
  \begin{pspicture}(-1.5,-1.5)(1.5,1.5)
    \psset{THETA=70,PHI=30,Dobs=200,Decran=10}
    \psset{A=5,B=5,C=A,fillstyle=solid,fillcolor=GrisClair,%
       linecolor=red, RotZ=\iRotZ}
    \tapis\DieThreeD(0,0,0)%
    \LineThreeD[linecolor=red,linestyle=dashed,arrows=->](0,0,0)(0,0,25)
    \pNodeThreeD(0,0,12.5){Z'}
    \uput[180](Z'){\texttt{RotZ=\iRotZ}}
  \end{pspicture}\hfill %
}

\psset{THETA=-10,PHI=20,Dobs=200,Decran=10}
\multido{\iCX=0+30}{8}{%
  \begin{pspicture}(-1.5,-1.5)(1.5,1.5)
    \AxesThreeD{->}(50,20,20)
    \psset{A=20,B=5,C=10,fillstyle=solid,fillcolor=LightBlue,linecolor=gray}
    \psset{RotZ=0,RotY=0,RotX=\iCX}
    \CubeThreeD(0,0,0)%
    \psset{linestyle=dashed}
  \end{pspicture}\hfill%
}%
\caption{Diffenerent views of a die and a cube\label{fig:rot}}
\end{figure}


\section{Location of the cube in the space}
Suppose that one wants to place a 10-units edge cube at the point
$(x=40,y=40,z=35)$. First, the half edge of the cube will be
define by the parameters : \verb+A=5,B=5,C=5+, and next the
coordinates of its center by \texttt{(40,40,35)}. On the
figure, the period of the grid is 10~units
(figure~\ref{coordinates}).

\begin{figure}[!htb]
\centering
\begin{Beispiel}[colwidth=0.45\linewidth]
\psset{THETA=30,PHI=30,Dobs=200,Decran=12}
\begin{pspicture}(-2.8,-3)(3.5,3.5)
    \PlansOXYZ
    \pNodeThreeD(40,40,35){G}
    \pNodeThreeD(40,40,0){G_XY}
    \pNodeThreeD(40,0,0){G_X}
    \pNodeThreeD(0,40,0){G_Y}
    \pNodeThreeD(0,0,35){G_Z}
    \pNodeThreeD(0,40,35){G_YZ}
    \pNodeThreeD(40,0,35){G_XZ}
    \psdots(G)(G_XY)(G_XZ)(G_YZ)(G_X)(G_Y)(G_Z)
    \psline(G)(G_XY)
    \psline(G)(G_XZ)
    \psline(G)(G_YZ)
    \psline(G_Z)(G_XZ)
    \psline(G_Z)(G_YZ)
    \AxesThreeD{->}(55)
\end{pspicture}
\end{Beispiel}
\caption{\label{coordinates}Origin \texttt{(40,40,35)}}
\end{figure}

\begin{figure}[!ht]
\centering
\begin{Beispiel}[colwidth=0.45\linewidth]
\psset{THETA=30,PHI=30,Dobs=200,Decran=12}
\begin{pspicture}(-2.8,-3)(3.5,3.5)
  \PlansOXYZ
  \pNodeThreeD(40,40,35){G}
  \pNodeThreeD(40,40,0){G_XY}
  \pNodeThreeD(40,0,0){G_X}
  \pNodeThreeD(0,40,0){G_Y}
  \pNodeThreeD(0,0,35){G_Z}
  \pNodeThreeD(0,40,35){G_YZ}
  \pNodeThreeD(40,0,35){G_XZ}
  \psdots(G)(G_XY)(G_XZ)(G_YZ)(G_X)(G_Y)(G_Z)
  \psline(G)(G_XY)
  \psline(G)(G_XZ)
  \psline(G)(G_YZ)
  \psline(G_Z)(G_XZ)
  \psline(G_Z)(G_YZ)
  \psset{A=5,B=5,C=5}
  \DieThreeD(40,40,35)%
  \AxesThreeD{->}(55)
\end{pspicture}
\end{Beispiel} 
\caption{\label{CubeOne}The placed cube.}
\end{figure}


To make it rotate of  around  $OX$ , one adds the parameter \verb+RotX=90+(figure~\ref{RotX}).

\begin{figure}[!ht]
\begin{Beispiel}[colwidth=0.45\linewidth]
\psset{THETA=30,PHI=30,Dobs=200,Decran=12}
\begin{pspicture}(-2.8,-3)(3.5,3.5)
  \PlansOXYZ
  \AxesThreeD{->}(55)
  \psset{A=5,B=5,C=5,RotX=90}
  % projections sur les plaans
  \DieThreeD(40,40,5)%
  \DieThreeD(5,40,35)%
  \DieThreeD(40,5,35)%
  \pNodeThreeD(40,40,35){G}
  \pNodeThreeD(40,40,10){G_XY}
  \pNodeThreeD(10,40,35){G_YZ}
  \pNodeThreeD(40,10,35){G_XZ}
  \psline(G)(G_XY)
  \psline(G)(G_XZ)
  \psline(G)(G_YZ)
  \DieThreeD(40,40,35)%
\end{pspicture}
\end{Beispiel}
\caption{\label{RotX} 90\textsuperscript{o} rotation around $OX$ and plane projections.}
\end{figure}


Three successive rotations around three axes with: \verb+RotX=60,RotY=20,RotZ=110+, are illustrate in figure~\ref{RotXYZ}.

\begin{figure}[!ht]
\begin{Beispiel}[colwidth=0.45\linewidth]
\psset{THETA=30,PHI=30,Dobs=200,Decran=12}
\begin{pspicture}(-2.8,-3)(3.5,3.5)
  \PlansOXYZ
  \AxesThreeD(55)
  \DieThreeD[A=5,B=5,C=5,RotX=30,RotY=20,RotZ=150](40,40,35)%
\end{pspicture}
\end{Beispiel}
\caption{\label{RotXYZ}rotations around $OX$, $OY$ et $OZ$: \texttt{RotX=60,RotY=20,RotZ=110}.}
\end{figure}

\section{Constructions using cubes}
This section was done after a book first published in  1873 and
titled:

\begin{figure}[!ht]
\centering
\psframebox{%
\begin{pspicture}(-3.1,-3.8)(3.1,3)
\rput(0,2.6){M\'ETHODE INTUITIVE}
\rput(0,2){\Large EXERCICES ET TRAVAUX}
\rput(0,1.5){POUR LES ENFANTS}
\rput(0,1){\tiny SELON LA M\'ETHODE ET LES PROC\'ED\'ES}
\rput(0,0){de \textbf{PESTALOZZI et FR\OE{}BEL}}
\rput(0,-1){M\textsuperscript{me} FANNY DELON}
\rput(0,-1.5){\tiny Directrice d'une \'Ecole professionnelle \`a Paris}
\rput(0,-2){M. CH. DELON}
\rput(0,-2.5){\tiny Licenci\'e \`es sciences}
\rput(0,-3){PARIS}
\rput(0,-3.5){1873}
\end{pspicture}}
\end{figure}

for children at infant school! One can not be surprised that
theses kinds of pedagogue gave rise to the generation of Eintein,
Maxwell, Bohr etc.



\begin{figure}[ht]
\begin{Beispiel}[colwidth=0.45\linewidth]
\psset{THETA=15,PHI=50,Dobs=200,Decran=15}
\hexagon
\end{Beispiel}
\caption{\label{hexagone}hexagon.}
\end{figure}

\begin{figure}[ht]
\begin{Beispiel}[colwidth=0.45\linewidth]
\psset{THETA=15,PHI=50,Dobs=200,Decran=15}%
\stardodecagon
\end{Beispiel}
\caption{\label{dodecagone}star dodecagon.}
\end{figure}

\begin{figure}[ht]
\begin{Beispiel}[colwidth=0.45\linewidth]
\psset{THETA=-15,PHI=50,Dobs=200,Decran=15}
\pentagon
\end{Beispiel}
\caption{\label{pentagone}pentagon.}
\end{figure}

\begin{figure}[ht]
\begin{Beispiel}[colwidth=0.45\linewidth]
\psset{THETA=-15,Decran=10,Dobs=100,PHI=75}
\stardecagon
\end{Beispiel}
\caption{\label{decagone}star decagon.}
\end{figure}

\begin{figure}[ht]
\begin{Beispiel}[colwidth=0.45\linewidth]
\psset{THETA=20,PHI=75,Decran=10,Dobs=100}
\begin{pspicture*}(-2.5,-2.5)(2.5,2)
\Table
\psset{A=5,B=5,C=5}
\CubeThreeD(-7.88675,0,5)%1
\CubeThreeD[RotZ=-120](3.94338,6.83,5)%2
\CubeThreeD[RotZ=120](3.94338,-6.83,5)%3
\end{pspicture*}
\end{Beispiel}
\caption{\label{triangle}triangle.}
\end{figure}


\begin{figure}[ht]
\begin{Beispiel}[colwidth=0.45\linewidth]
\psset{THETA=-15,PHI=50,Decran=10,Dobs=150}
\octogon
\end{Beispiel}
\caption{\label{octogone}octogon.}
\end{figure}


\begin{figure}[ht]
\begin{Beispiel}[colwidth=0.45\linewidth]
\psset{THETA=-15,Decran=10,Dobs=150,PHI=75}
\starhexadecagon
\end{Beispiel}
\caption{\label{hexadecagon}star hexadecagon.}
\end{figure}

\begin{figure}[ht]
\begin{Beispiel}[colwidth=0.45\linewidth]
\psset{THETA=-15,Decran=10,Dobs=150,PHI=75}
\begin{pspicture}(-2.2,-1.75)(2.2,2.2)
  \Table
  \pNodeThreeD(-8.66,-5,0){A6}
  \pNodeThreeD(-8.66,5,0){A1}
  \pNodeThreeD(0,10,0){A2}
  \pNodeThreeD(8.66,5,0){A3}
  \pNodeThreeD(8.66,-5,0){A4}
  \pNodeThreeD(0,-10,0){A5}%
  \psclip{\pspolygon[fillstyle=solid,fillcolor=GrisClair,%
    linestyle=none](A1)(A2)(A3)(A4)(A5)(A6)}
    \DessusTable
  \endpsclip
  \psset{A=5,B=5,C=5}
  \DieThreeD[RotZ=60,RotX=-90](-6.83,-11.83,5)%
  \DieThreeD[RotZ=120,RotY=-90](6.83,-11.83,5)%
  \DieThreeD[RotX=90](-13.86,0,5)%
  \DieThreeD[RotZ=-60,RotY=90](-6.83,11.83,5)%
  \DieThreeD[RotZ=-120,RotY=180](6.83,11.83,5)%
  \DieThreeD[RotZ=180](13.86,0,5)%
\end{pspicture}
\end{Beispiel}
\caption{\label{pentagoneDie}hexagon with dices.}
\end{figure}

Observing figure from off :
\begin{verbatim}
\psset{PHI=90,THETA=0}
\end{verbatim}
 one obtains classical geometric
figures :

(\ref{hexagonePlan}) (\ref{dodecagonePlan}) (\ref{pentagonePlan}) (\ref{decagonePlanStar})
(\ref{trianglePlan}) (\ref{octogonePlan}) (\ref{hexadecagonePlan}) (\ref{hexagonePlanDie}).

\begin{figure}[ht]
\centering
\psset{THETA=0,Decran=10,Dobs=125,PHI=90}
\hexagon
\caption{\label{hexagonePlan}``flat'' hexagon.}
\end{figure}


\begin{figure}[ht]
\centering
\psset{Decran=10,Dobs=100}
\psset{PHI=90,THETA=0}
\stardecagon
\caption{\label{dodecagonePlan}``flat'' star dodecagone.}
\end{figure}
%
\begin{figure}[ht]
\centering
\psset{Decran=10,Dobs=125}
\psset{PHI=90,THETA=0}
\pentagon
\caption{\label{pentagonePlan}``flat'' pentagon.}
\end{figure}


\begin{figure}[ht]
\centering
\psset{THETA=0,Decran=10,Dobs=125,PHI=90}
\stardecagon
\caption{\label{decagonePlanStar}``flat'' star decagon.}
\end{figure}
%


%
\begin{figure}[ht]
\centering
\psset{PHI=90,THETA=0,Decran=10,Dobs=100}
\begin{pspicture}*(-2.2,-2.2)(2.2,2.2)
\Table
\psset{A=5,B=5,C=5}
\CubeThreeD(-7.88675,0,5)%1
\CubeThreeD[RotZ=-120](3.94338,6.83,5)%2
\CubeThreeD[RotZ=120](3.94338,-6.83,5)%3
\end{pspicture}
\caption{\label{trianglePlan}``flat'' triangle.}
\end{figure}


\begin{figure}[ht]
\centering
\psset{PHI=90,THETA=0,Decran=10,Dobs=125}
\octogon
\caption{\label{octogonePlan}``flat'' octogon.}
\end{figure}



\begin{figure}[ht]
\centering
\psset{PHI=90,THETA=0,Decran=10,Dobs=125}
\starhexadecagon
\caption{\label{hexadecagonePlan}``flat'' star hexadecagon.}
\end{figure}

\begin{figure}[ht]
\centering
\psset{PHI=90,THETA=0,Decran=10,Dobs=125}
\begin{pspicture}(-2.2,-2.2)(2.2,2.2)
\Table
\pNodeThreeD(-8.66,-5,0){A6}
\pNodeThreeD(-8.66,5,0){A1}
\pNodeThreeD(0,10,0){A2}
\pNodeThreeD(8.66,5,0){A3}
\pNodeThreeD(8.66,-5,0){A4}
\pNodeThreeD(0,-10,0){A5}%
\psclip{\pspolygon[fillstyle=solid,fillcolor=GrisClair,linestyle=none](A1)(A2)(A3)(A4)(A5)(A6)}
\DessusTable
\endpsclip
\psset{A=5,B=5,C=5}
\DieThreeD[RotZ=60,RotX=-90](-6.83,-11.83,5)%
\DieThreeD[RotZ=120,RotY=-90](6.83,-11.83,5)%
\DieThreeD[RotX=90](-13.86,0,5)%
\DieThreeD[RotZ=-60,RotY=90](-6.83,11.83,5)%
\DieThreeD[RotZ=-120,RotY=180](6.83,11.83,5)%
\DieThreeD[RotZ=180](13.86,0,5)%
\end{pspicture}
\caption{\label{hexagonePlanDie}``flat'' hexagon with dices.}
\end{figure}




\clearpage


\section{Sphere, part of sphere, half-sphere, parallels and meridians}

Beside \verb+sphereThreeD+ there exist several macro for spheres:

\begin{itemize}
\item \verb|SphereInverseThreeD|
\item \verb|\SphereCercleThreeD|
\item \verb|\SphereMeridienThreeD|
\item \verb|\DemiSphereThreeDThreeD|
\item \verb|\SphereCreuseThreeD|
\item \verb|\PortionSphereThreeD|
\end{itemize}


The macro:
\begin{verbatim}
\SphereThreeD(10,30,20){20}
\end{verbatim}
draws the sphere defined by the coordinates of its centre and its radius which is shown in
figure~\ref{sphere} together with the macro
\begin{verbatim}
\PortionSphereThreeD(0,0,0){20}
\end{verbatim}
and some more additional lines.

\begin{verbatim}
\begin{pspicture}(-3,-3.5)(3,5)
\psset{THETA=30,PHI=30,Dobs=100,Decran=10}
{\psset{style=GradGrayWhite}%
\SphereThreeD(0,0,0){20}
\psset{fillstyle=solid,fillcolor=gray}
\PortionSphereThreeD(0,0,0){20}
\pNodeThreeD(20;10;10){C1}
\pNodeThreeD(40;10;10){D1}
\psline(C1)(D1)
\pNodeThreeD(20;10;-10){C2}
\pNodeThreeD(40;10;-10){D2}
\psline(C2)(D2)
\pNodeThreeD(20;-10;-10){C3}
\pNodeThreeD(40;-10;-10){D3}
\psline(C3)(D3)
\pNodeThreeD(20;-10;10){C4}
\pNodeThreeD(40;-10;10){D4}
\psline(C4)(D4)
\PortionSphereThreeD%
	[style=GradGrayWhite](0,0,0){40}}
% PhiCercle=latitude of the cercle
% \SphereCercle[PhiCercle=...]{radius}
\psset{linecolor=white,PhiCercle=45}
\SphereCercleThreeD(0,0,0){20}
% ThetaMeridien=longitude of the meridian
% \SphereMeridien[ThetaMeridien=...]{radius}
\SphereMeridienThreeD%
	[ThetaMeridien=45](0,0,0){20}
\pNodeThreeD(20;45;45){A}
\pNodeThreeD(50;45;45){B}
\psline[linecolor=black]{->}(A)(B)
\pNodeThreeD(20;0;90){Nord}
\pNodeThreeD(40;0;90){Nord1}
\psline[linecolor=black]{->}(Nord)(Nord1)
\SphereCercleThreeD[PhiCercle=0](0,0,0){20}
\SphereMeridienThreeD%
	[ThetaMeridien=0](0,0,0){20}
\end{pspicture}
\end{verbatim}



\begin{figure}[!htb]
\begin{pspicture}(-3,-3.5)(3,5)
\psset{THETA=30,PHI=30,Dobs=100,Decran=10}
\bgroup
  \psset{style=GradGrayWhite}%
  \SphereThreeD(0,0,0){20}
  \psset{fillstyle=solid,fillcolor=gray}
  \PortionSphereThreeD(0,0,0){20}
  \pNodeThreeD(20;10;10){C1}
  \pNodeThreeD(40;10;10){D1}
  \psline(C1)(D1)
  \pNodeThreeD(20;10;-10){C2}
  \pNodeThreeD(40;10;-10){D2}
  \psline(C2)(D2)
  \pNodeThreeD(20;-10;-10){C3}
  \pNodeThreeD(40;-10;-10){D3}
  \psline(C3)(D3)
  \pNodeThreeD(20;-10;10){C4}
  \pNodeThreeD(40;-10;10){D4}
  \psline(C4)(D4)
  \PortionSphereThreeD[style=GradGrayWhite](0,0,0){40}
\egroup
% PhiCercle=latitude of the cercle
% \SphereCercle[PhiCercle=...]{radius}
  \psset{linecolor=white,PhiCercle=45}
  \SphereCercleThreeD(0,0,0){20}
% ThetaMeridien=longitude of the meridian
% \SphereMeridien[ThetaMeridien=...]{radius}
  \SphereMeridienThreeD[ThetaMeridien=45](0,0,0){20}
% \pNodeThreeD(radius}{longitude}{latitude}{name of the point}
  \pNodeThreeD(20;45;45){A}
  \pNodeThreeD(50;45;45){B}
  \psline[linecolor=black]{->}(A)(B)
  \pNodeThreeD(20;0;90){Nord}
  \pNodeThreeD(40;0;90){Nord1}
  \psline[linecolor=black]{->}(Nord)(Nord1)
  \SphereCercleThreeD[PhiCercle=0](0,0,0){20}
  \SphereMeridienThreeD[ThetaMeridien=0](0,0,0){20}
\end{pspicture}
\caption{\label{sphere}A Sphere.}
\end{figure}


\begin{figure}[!htb]
\centering
\begin{pspicture}(-3,-2)(3,5)
  \psset{THETA=60,PHI=30,Dobs=100,Decran=10}
% \DemiSphereThreeD(x,y,z){radius}
  \DemiSphereThreeD[RotX=180,style=GradGrayWhite](0,0,0){20}
  \SphereCreuseThreeD[RotX=180,linecolor=white,style=GradGrayWhite](0,0,0){20}
  \AxesThreeD[linestyle=dashed](30,30,40)
\end{pspicture}
\caption{\label{halfsphere}half-sphere.}
\end{figure}


\begin{figure}[!htb]
\centering
\begin{pspicture}(-3,-2)(3,2)
\psset{THETA=60,PHI=20,Dobs=100,Decran=10}
\psset{style=GradGrayWhite}%
\SphereThreeD(0,0,0){10}%
\DemiSphereThreeD[RotX=180](0,0,0){20}%
\begin{psclip}{%
\SphereCreuseThreeD[RotX=180,linecolor=white](0,0,0){20}}%
\SphereThreeD(0,0,0){10}
\end{psclip}%
\end{pspicture}
\caption{\label{egg} levitation}
\end{figure}


\section{A Hole in a sphere}

\begin{figure}[!htb]
\centering
\psset{THETA=10,PHI=30,Dobs=100,Decran=10}
\begin{pspicture}*(-3,-3)(3,3)
  \SphereThreeD[style=GradGrayWhite,gradmidpoint=0.2](0,0,0){40}%
  \begin{psclip}{\PortionSphereThreeD[PortionSpherePHI=40,%
       DeltaPHI=30,DeltaTHETA=30,linewidth=4\pslinewidth](0,0,0){40}}%
    \SphereInverseThreeD[style=GradGrayWhite](0,0,0){40}%
    \SphereThreeD[style=GradGrayWhite](0,0,0){30}%
    \begin{psclip}{\PortionSphereThreeD[PortionSpherePHI=30,%
         DeltaPHI=30,DeltaTHETA=30,linewidth=4\pslinewidth](0,0,0){30}}%
      \SphereInverseThreeD[style=GradGrayWhite](0,0,0){30}%
      \SphereThreeD[style=GradGrayWhite](0,0,0){20}%
      \begin{psclip}{\PortionSphereThreeD[PortionSpherePHI=30,%
           DeltaPHI=30,DeltaTHETA=30,linewidth=4\pslinewidth](0,0,0){20}}%
        \SphereInverseThreeD[style=GradGrayWhite](0,0,0){20}%
        \SphereThreeD[style=GradGrayWhite](0,0,0){10}%
        \begin{psclip}{%
          \PortionSphereThreeD[PortionSpherePHI=30,%
              DeltaPHI=30,DeltaTHETA=30,linewidth=4\pslinewidth](0,0,0){10}}%
          \SphereInverseThreeD[style=GradGrayWhite](0,0,0){10}%
          \SphereThreeD[style=GradGrayWhite](0,0,0){5}%
        \end{psclip}%
      \end{psclip}%
    \end{psclip}%
  \end{psclip}%
\end{pspicture}
\caption{\label{Holeinasphere}A Hole in a sphere.}
\end{figure}

It is a rectangular hole whose the size are meridian and parallels
arcs (figure~\ref{Holeinasphere}).

We  define the part of the sphere setting its radius, the center
of the sphere and the $\Delta\phi$ and $\Delta\theta$.
\begin{verbatim}
\PortionSphereThreeD[PortionSpherePHI=45,%
              PortionSphereTHETA=0,%
                       DeltaPHI=45,%
                 DeltaTHETA=20](0,0,0){20}
\end{verbatim}

There are the parameters of the first hole. The radius is
\texttt{20}.
\begin{verbatim}
{\psset{fillstyle=gradient,%
           gradbegin=white,%
              gradend=blue,%
          gradmidpoint=0.2,%
            linecolor=cyan,%
            linewidth=0.1mm}
\SphereThreeD(0,0,0){20}}%
\begin{psclip}{%
\PortionSphereThreeD[PortionSpherePHI=45,%
    DeltaPHI=45,DeltaTHETA=20](0,0,0){20}}
\SphereInverseThreeD[fillstyle=solid,%
                 fillcolor=red,%
                linecolor=blue](0,0,0){20}%
\end{psclip}%
\end{verbatim}

This is the tricks to see the inner of the sphere.

\verb+\SphereInverse+ define the hidden part of the sphere.


\section{Drawing a cylinder}
A cylinder is defined by the radius of its base and its height.
The center of the base is set in the usual way, and
\textsf{RotX,RotY,RotZ} make it rotate around the axes.

\verb+\CylindreThreeD(x,y,z){radius}{hauteur}+

\begin{figure}[!htb]
\centering
\begin{pspicture}(-3.5,-2)(3,4.5)
\psset{THETA=5,PHI=40,Dobs=150,Decran=6.5,fillstyle=solid,linewidth=0.1mm}
% plan horizontal
{\psset{normaleLongitude=0, normaleLatitude=90}
\FrameThreeD[fillstyle=solid,fillcolor=GrisClair](0,0,0)(-50,0)(50,50)
\FrameThreeD[fillstyle=solid,fillcolor=GrisClair](0,0,0)(-50,0)(50,-50)
\QuadrillageThreeD(0,0,0)(-50,-50)(50,50)}
\multido{\iCY=-45+90}{2}{%
    \CylindreThreeD(-45,\iCY,0){5}{50}
    \DemiSphereThreeD(-45,\iCY,50){5}%
}
\CylindreThreeD(0,0,0){10}{15}
\CylindreThreeD(0,0,15){20}{5}
\DemiSphereThreeD[RotX=180](0,0,35){20}
\SphereCreuseThreeD[RotX=180](0,0,35){20}
{\psset{RotY=90,RotX=0,RotZ=30}
\CylindreThreeD(15,15,5){5}{20}}
\multido{\iCY=-45+90}{2}{%
\CylindreThreeD(45,\iCY,0){5}{50}
\DemiSphereThreeD(45,\iCY,50){5}}
\end{pspicture}
\caption{\label{cylinder}cylinders.}
\end{figure}

\begin{verbatim}
\CylindreThreeD(0,0,-5){10}{15}}
\psset{RotY=90}
\CylindreThreeD(15,15,-5){5}{20}
\end{verbatim}


\section{Tetrahedron, cone and square pyramid}
\subsection{square pyramid}
\begin{verbatim}
\psset{A=...,Hpyramide=...}
\Pyramide
\end{verbatim}

See the examples of figures~(\ref{Pyramid})~(\ref{Obelisque}).

\begin{figure}[!htb]
\centering
\psset{ColorFaceD=GrayD,ColorFaceA=GrayA,%
    ColorFaceB=GrayB,ColorFaceC=GrayC,ColorFaceE=GrayE}
\psframebox[fillstyle=solid,fillcolor=GrayB,framesep=0pt]{%
\begin{pspicture}*(-3,-4)(3,4)
\psset{THETA=-70,PHI=60,Dobs=200,Decran=15}
\DecorSable
\psset{RotZ=45,fillstyle=solid,linecolor=black,A=9}
\PyramideThreeD(5,35,0){10}
\psset{A=10}
\PyramideThreeD(0,0,0){13}
\psset{A=7}
\PyramideThreeD(10,-35,0){8.7}
\end{pspicture}}
\caption{\label{Pyramid}Pyramids of Egypt.}
\end{figure}


\begin{figure}[!htb]
\centering
\psframebox[fillstyle=solid,fillcolor=GrayB,framesep=0pt]{%
    \begin{pspicture}*(-2.5,-2)(2.5,5.5)
    \psset{THETA=30,PHI=30,Dobs=400,Decran=12}
    \DecorSable
    \CubeThreeD[A=15,B=15,C=15](0,0,15)%
    \psset{A=10,fillstyle=solid}
    \PyramideThreeD[fracHeight=0.8](0,0,30){150}%
    \psset{A=2}
    \PyramideThreeD(0,0,150){5}%
    \end{pspicture}%
}
\caption{\label{Obelisque}Obelisk of Egypt.}
\end{figure}


\subsection{Cone}
\begin{verbatim}
\ConeThreeD[fracHeight=...]
           (x,y,z){radius}{Height}
\end{verbatim}
by default \verb+fracHeight=1+ : figure~\ref{Cone}.

\begin{figure}[!htb]
\centering
\psframebox[fillstyle=solid,fillcolor=GrayB,framesep=0pt]{%
\begin{pspicture}*(-3,-5)(3,4)
\psset{THETA=30,PHI=40,Dobs=200,Decran=12,fillstyle=solid,%
    fillcolor=GrisClair,linewidth=0.25\pslinewidth}
\DecorSable
\CylindreThreeD(0,0,0){10}{50}
\ConeThreeD[fillcolor=GrayB](0,0,50){10}{10}
\CylindreThreeD[RotY=90,RotZ=150](40,20,10){10}{50}
\ConeThreeD[fracHeight=0.5](20,-20,0){10}{10}
\CylindreThreeD(20,-20,5){5}{50}
\ConeThreeD[fracHeight=0.5](50,50,0){10}{10}
\CylindreThreeD(50,50,5){5}{50}
\end{pspicture}}
\caption{\label{Cone}Cones and cylinders.}
\end{figure}

\section{Points and lines}
The command allowing to mark points and thus to draw lines
and polygons can be used of two manners, either with the Cartesian coordinates
 \begin{verbatim}
\pNodeThreeD(x,y,z){name}
\end{verbatim}
 or with the spherical coordinates :
\begin{verbatim}
\pNodeThreeD(radius;longitude;latitude)%
            {name of the point}
\end{verbatim}

For example \verb+\pNodeThreeD(25,-25,25){A}+, the point $A(25,25,25)$ places.
Points being positioned, just to write \verb+\psline(A)(B)+, to draw the segment $AB$.

 On the figure~\ref {points}, one drew a cube with its diagonals.
\begin{figure}[!htb]
\centering
\psset{unit=1cm}
 \psset{THETA=70,PHI=30,Dobs=150,Decran=10}
 \begin{pspicture}(-3,-3)(3,4)
  \AxesThreeD[linecolor=red,linestyle=dashed](50,60,50)
  \pNodeThreeD(25,-25,25){A}
  \pNodeThreeD(25,25,25){B}
  \pNodeThreeD(25,25,-25){C}
  \pNodeThreeD(25,-25,-25){D}
  \pNodeThreeD(-25,-25,25){E}
  \pNodeThreeD(-25,25,25){F}
  \pNodeThreeD(-25,25,-25){G}
  \pNodeThreeD(-25,-25,-25){H}
  \pspolygon(A)(B)(C)(D)
  \pspolygon(E)(F)(G)(H)
  \psline(A)(E)
  \psline(B)(F)
  \psline(C)(G)
  \psline(D)(H)
  \psset{linestyle=dashed}
  \psline(A)(G)
  \psline(B)(H)
  \psline(C)(E)
  \psline(D)(F)
% routine page 49 in "pr�sentation de PSTricks"
% D.Girou "cahier 16 Gutengerg"
    \newcounter{lettre}
    \multido{\i=1+1}{8}{%
        \setcounter{lettre}{\i}
        \psdot[linecolor=red](\Alph{lettre})
        \uput[90](\Alph{lettre}){\Alph{lettre}}
    }
\end{pspicture}
\caption{\label{points}Points and lines.}
\end{figure}


\section{Circles}
A circle is defined by a vector normal for its plan by  $(\theta,\varphi)$, with the following parameters for example:
\begin{verbatim}
normaleLongitude=60,normaleLatitude=90
\end{verbatim}
The coordinates of his centre as well as his radius.
\begin{verbatim}
\CircleThreeD(x,y,z){radius}
\end{verbatim}

The circles of the figure~\ref{circles}, were drawn with the following
commands:

\begin{figure}[!htb]
\centering
\psframebox{%
    \begin{pspicture}(-2.5,-3.5)(3.5,1.5)
    \psset{THETA=50,PHI=50,Dobs=250,Decran=10}
    \multido{\iX=-70+10}{15}{%
        \pNodeThreeD(\iX,0,0){X1}
        \pNodeThreeD(\iX,50,0){X2}
        \psline(X1)(X2)
    }
    \multido{\iY=0+10}{6}{%
        \pNodeThreeD(-70,\iY,0){Y1}
        \pNodeThreeD(70,\iY,0){Y2}
        \psline(Y1)(Y2)%
    }
    \psset{normaleLongitude=0,normaleLatitude=90}
    \multido{\iXorigine=-65+10}{14}{%
        \multido{\iYorigine=5+10}{5}{%
            \CircleThreeD[linecolor=red](\iXorigine,\iYorigine,0){5}%
        }%
    }
    \end{pspicture}%
}
\caption{\label{circles}circles.}
\end{figure}

\begin{verbatim}
\psset{normaleLongitude=0,%
        normaleLatitude=90}
\multido{\iXorigine=-65+10}{14}{%
    \multido{\iYorigine=5+10}{5}{%
        \CircleThreeD[linecolor=red]%
            (\iXorigine,\iYorigine,0){5}}}
\end{verbatim}

\section{The macros and the options}
\subsection{The colors of the cube, the pyramid and tetraedre}

The predefined colors for the different sides of a cube are
always set in the \verb+rgb+ mode :
\begin{verbatim}
CubeColorFaceOne=1 1 0,%
CubeColorFaceTwo=0.9 0.9 0,%
CubeColorFaceThree=0.8 0.8 0,%
CubeColorFaceFour=0.7 0.7 0,%
CubeColorFaceFive=0.65 0.65 0,%
CubeColorFaceSix=0.75 0.75 0
\end{verbatim}

The colors for the pyramid and the tetraedre are taken from the predefined ones:
\begin{verbatim}
ColorFaceD=cyan,
ColorFaceA=magenta,
ColorFaceB=red,
ColorFaceC=blue,
ColorFaceE=yellow
\end{verbatim}

They can be changed in the usual way with the \verb+\psset+ macro.


\subsection{Common parameters}
\verb+RotX=<value>, RotY=<value>, RotZ=<value>+

The predefined value is zero, means no rotation.

\subsection{Cube}
The following command places a parallelepiped with a length of $a=40$, $b=20$ and $c=10$ units
and it is placed with its center at the point $x=25$, $y=25$ and $z=25$

\begin{verbatim}
\CubeThreeD[A=20,B=10,C=5](25,25,25)
\end{verbatim}

\begin{figure}[!htb]
\centering
\begin{pspicture}(-3,-3)(3,3.5)
\psset{PHI=30,THETA=45,Dobs=200}
\PlansOXYZ\AxesThreeD(55)
\FrameThreeD[normaleLongitude=0,%
             normaleLatitude=90,%
             fillstyle=vlines,hatchsep=0.4mm](25,25,0)(-10,-15)(10,15)
\FrameThreeD[normaleLongitude=0,%
             normaleLatitude=0,%
             fillstyle=vlines,hatchsep=0.4mm](0,25,25)(-10,-5)(10,5)
\FrameThreeD[normaleLongitude=90,%
             normaleLatitude=0,%
             fillstyle=vlines,hatchsep=0.4mm](25,0,25)(-15,-5)(15,5)
\CubeThreeD[A=15,B=10,C=5](25,25,25)%
\end{pspicture}
\caption{\label{Prisme}Parallelepiped}
\end{figure}

In other words: the length of the sides is \verb+2A,2B,2C+ (see figure~\ref{Prisme}).

For rotations, let us consider the result of a rotation around one of the axes, while knowing that it is possible to combine them.  The corresponding rotation of projection on the horizontal level is obtained with the parameter: \verb+normaleLongitude=<degrees>+ (figure~\ref{PrismeRotZ}).

\begin{figure}[!htb]
\centering
\begin{pspicture}(-3,-3)(3,3.5)
\psset{PHI=30,THETA=45,Dobs=200,RotZ=60}
\PlansOXYZ\AxesThreeD(55)
% la projection sur le plan Oxy
\FrameThreeD[normaleLongitude=60,%
             normaleLatitude=90,%
             fillstyle=vlines,hatchsep=0.4mm](25,25,0)(-10,-15)(10,15)
\CubeThreeD[A=15,B=10,C=5](25,25,25)%
\end{pspicture}
\caption{\label{PrismeRotZ}The same parallelepiped rotated with \texttt{RotZ=60}.}
\end{figure}

There is no difference to a die, except that all sides have the same length.


\begin{figure}[!htb]
\centering
\begin{pspicture}(-3,-3)(3,3.5)
\psset{PHI=30,THETA=45,Dobs=200,RotZ=60,,RotX=90}
\PlansOXYZ\AxesThreeD(55)
% la projection sur le plan Oxy
\FrameThreeD[normaleLongitude=60,%
             normaleLatitude=90,%
             fillstyle=vlines,hatchsep=0.4mm](25,25,0)(-5,-15)(5,15)
\CubeThreeD[A=15,B=10,C=5](25,25,25)%
\end{pspicture}
\caption{\label{PrismeRotXRotZ}The same parallelepiped, rotated with the values \texttt{RotX=90,RotZ=60}}
\end{figure}


\subsection{Cylinder and circle}
In addition to the already quoted optional parameters the cylinder  requires the obligatory parameters:
\begin{verbatim}
\CylindreThreeD[...](x,y,z){radius}{height}
\end{verbatim}

Projection on the horizontal level is obtained with the following values:

\begin{verbatim}
\CircleThreeD[normaleLongitude=0,%
             normaleLatitude=90,%
             fillstyle=vlines,%
             hatchsep=0.4mm](30,30,0){10}
\end{verbatim}

The circle macro needs the following parameters:

\begin{verbatim}
\CircleThreeD[...](x,y,z){radius}
\end{verbatim}

Figure~\ref{CylindreDemo} shows an example of the above macros.

\begin{figure}[!ht]
\centering
\begin{pspicture}(-3,-3)(3,3.5)
\psset{PHI=30,THETA=45,Dobs=200}
\PlansOXYZ\AxesThreeD(55)
% la projection sur le plan Oxy
\CircleThreeD[normaleLongitude=0,%
             normaleLatitude=90,%
             fillstyle=vlines,%
             hatchsep=0.4mm](30,30,0){10}
\CylindreThreeD[fillstyle=solid,fillcolor=yellow,%
   linewidth=0.1mm](30,30,20){10}{30}%
\end{pspicture}
\caption{\label{CylindreDemo}A cylinder with a radius of $10$ units and a
    height of $50$ units
    with its base center at \texttt{(30,30,20)}.%
}
\end{figure}


\section{See the interior of a cube}
The following option makes it possible to visualize the interior of the box, the result is seen in the figure~\ref{Cube inside} :

\begin{verbatim}
\DieThreeD(0,0,0)%
\begin{psclip}{%
\FrameThreeD[normaleLongitude=0,%
             normaleLatitude=90]%
			 (0,0,10)(-10,-10)(10,10)}%
\DieThreeD[CubeInside=true](0,0,0)%
\end{psclip}%
\end{verbatim}

\begin{figure}
\centering
\begin{pspicture}(-2,-2)(2,3.5)
  \psset{A=10,B=10,C=10,PHI=60,THETA=-60}
  \DieThreeD(0,0,0)%
  \begin{psclip}{%
    \FrameThreeD[normaleLongitude=0,%
       normaleLatitude=90](0,0,10)(-10,-10)(10,10)}%
    \DieThreeD[CubeInside=true](0,0,0)%
  \end{psclip}%
  \FrameThreeD[normaleLongitude=0,%
    normaleLatitude=90,linewidth=1mm](0,0,10)(-10,-10)(10,10)%
\end{pspicture}
\caption{\label{Cube inside}An empty box.}
\end{figure}



\end{document}
