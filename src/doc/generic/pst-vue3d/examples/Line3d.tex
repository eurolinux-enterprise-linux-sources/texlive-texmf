\documentclass[12pt,a4paper]{article}
\usepackage{pstricks}
\usepackage{pst-vue3d}
\usepackage{multido}
\usepackage{geometry}
%
\begin{document}

 \psset{unit=1cm}
 \begin{pspicture}(-3,-3)(3,4)
  \psset{THETA=70,PHI=30,Dobs=150,Decran=10}
  \AxesThreeD[linecolor=red,linestyle=dashed](50,60,50)
  \pNodeThreeD(25,-25,25){A}
  \pNodeThreeD(25,25,25){B}
  \pNodeThreeD(25,25,-25){C}
  \pNodeThreeD(25,-25,-25){D}
  \pNodeThreeD(-25,-25,25){E}
  \pNodeThreeD(-25,25,25){F}
  \pNodeThreeD(-25,25,-25){G}
  \pNodeThreeD(-25,-25,-25){H}
  \pspolygon(A)(B)(C)(D)
  \pspolygon(E)(F)(G)(H)
  \psline(A)(E)
  \psline(B)(F)
  \psline(C)(G)
  \psline(D)(H)
  \psset{linestyle=dashed}
  \psline(A)(G)
  \psline(B)(H)
  \psline(C)(E)
  \psline(D)(F)
% routine page 49 in "pr�sentation de PSTricks"
% D.Girou "cahier 16 Gutengerg"
  \newcounter{lettre}
  \multido{\i=1+1}{8}{%
    \setcounter{lettre}{\i}
    \psdot[linecolor=red](\Alph{lettre})
    \uput[90](\Alph{lettre}){\Alph{lettre}}
  }
\end{pspicture}



    \begin{pspicture}(-2.5,-3.5)(3.5,1.5)
    \psset{THETA=50,PHI=50,Dobs=250,Decran=10}
    \multido{\iX=-70+10}{15}{%
        \pNodeThreeD(\iX,0,0){X1}
        \pNodeThreeD(\iX,50,0){X2}
        \psline(X1)(X2)
    }
    \multido{\iY=0+10}{6}{%
        \pNodeThreeD(-70,\iY,0){Y1}
        \pNodeThreeD(70,\iY,0){Y2}
        \psline(Y1)(Y2)%
    }
    \psset{normaleLongitude=0,normaleLatitude=90}
    \multido{\iXorigine=-65+10}{14}{%
        \multido{\iYorigine=5+10}{5}{%
            \CircleThreeD[linecolor=red](\iXorigine,\iYorigine,0){5}%
        }%
    }
    \end{pspicture}%



\end{document}
