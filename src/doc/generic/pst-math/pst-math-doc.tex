\documentclass[fleqn,10pt]{article}

\usepackage{multicol}
\usepackage[a4paper,margin=1.5cm,includeheadfoot]{geometry}
\usepackage{fancyhdr}
\usepackage[baw,pstricks]{fvrb-ex}
\usepackage{pst-infixplot,pst-math}
\usepackage{amsmath,amssymb}
\usepackage{pstcol}
\makeatletter

\renewcommand{\Begin@Example}{%
\parindent=0pt
\multiply\topsep by 2 \VerbatimEnvironment
\begin{VerbatimOut}[codes={\catcode`\�=12\catcode`\/=12\catcode`\&=12%
                           \catcode`\"=12}]{\jobname.tmp}}

\renewcommand{\Below@Example}[1]{%
\VerbatimInput[gobble=0,commentchar=�,commandchars=/&",numbersep=3pt]%
              {\jobname.tmp}
\catcode`\�=9\relax%
\NoHighlight@Attributes % To suppress possible highlighting
\ifFvrbEx@Grid\vspace{5pt}\fi
#1%
\ifFvrbEx@Grid\vspace{5pt}\fi
\par}

\renewcommand{\SideBySide@Example}[1]{%
\@tempdimb=\FV@XRightMargin \advance\@tempdimb -5mm
\begin{minipage}[c]{\@tempdimb}
  \fvset{xrightmargin=0pt}
  \catcode`\�=9\relax%
  \NoHighlight@Attributes % To suppress possible highlighting
  #1
\end{minipage}%
\@tempdimb=\textwidth \advance\@tempdimb -\FV@XRightMargin \advance\@tempdimb 5mm
\begin{minipage}[c]{\@tempdimb}
  \VerbatimInput[gobble=0,commentchar=�,commandchars=/&",numbersep=3pt,
                 xleftmargin=5mm,xrightmargin=0pt]{\jobname.tmp}
\end{minipage}}

% The \NoHighlight@Attributes from `hbaw' and `hcolors' packages
% must be modified too
\def\NoHighlight@Attributes{%
\catcode`\/=0\relax%
\catcode`\&=1\relax%
\catcode`\"=2\relax%
\def\HLa##1{##1}%
\def\HLb##1{##1}%
\def\HLc##1{##1}%
\def\HLd##1{##1}%
\def\HLe##1{##1}%
\def\HLf##1{##1}%
\def\HLBFa##1{##1}%
\def\HLBFb##1{##1}%
\def\HLBFc##1{##1}%
\def\HLBFd##1{##1}%
\def\HLBFe##1{##1}%
\def\HLBFf##1{##1}%
\def\HLITa##1{##1}%
\def\HLITb##1{##1}%
\def\HLITc##1{##1}%
\def\HLITd##1{##1}%
\def\HLITe##1{##1}%
\def\HLITf##1{##1}%
\def\HLCBBa##1{##1}%
\def\HLCBBb##1{##1}%
\def\HLCBBc##1{##1}%
\def\HLCBBd##1{##1}%
\def\HLCBBe##1{##1}%
\def\HLCBBf##1{##1}%
\def\HLCBBz##1{##1}%
\def\HLCBWa##1{##1}%
\def\HLCBWb##1{##1}%
\def\HLCBWc##1{##1}%
\def\HLCBWd##1{##1}%
\def\HLCBWe##1{##1}%
\def\HLCBWf##1{##1}%
\def\HLCBWz##1{##1}%
}

\makeatother

\fvset{numbers=none,frame=single,labelposition=topline,gobble=4}

\DeclareRobustCommand\cs[1]{\texttt{\char`\\#1}}

\newcommand{\MathPackage}{\textbf{`pst-math'}}

\lhead{\MathPackage}\rhead{A PSTricks package for enhancing mathematical operators in PSTricks}
\pagestyle{fancy}

\psset{subgriddiv=1,griddots=10}%
\showgrid

\makeatletter

\def\DefOfOperator{\@ifstar{\DefOfOperator@}{\DefOfOperator@@}}

\def\DefOfOperator@#1#2#3#4{{\operator@font#1}:\left\{\begin{array}{ccc} #2&\to&#3\\
x&\mapsto&#4\end{array}\right.}

\def\DefOfOperator@@#1#2#3{{\operator@font#1}:\left\{\begin{array}{ccc} #2&\to&#3\\
x&\mapsto&{\operator@font#1}(x)\end{array}\right.}

\makeatother

\begin{document}

\title{\MathPackage\\ A PSTricks package for enhancing mathematical operators in PSTricks\\ \normalsize ver. 0.2}
\author{Christophe \textsc{Jorssen} \texttt{<christophe.jorssen@libre.fr.invalid>}\\ \footnotesize `libre' is the french word for `free'}
\date{04/07/14}
\maketitle

\setlength{\columnseprule}{0.6pt}
\begin{multicols}{2}
{\parskip 0pt \tableofcontents}
\end{multicols}

\section{Trigonometry}

\MathPackage{} introduces natural trigonometric postscript operators COS, SIN and TAN defined by
\[\DefOfOperator{cos}{\mathbb R}{[-1,1]}\]
\[\DefOfOperator{sin}{\mathbb R}{[-1,1]}\]
\[\DefOfOperator{tan}{\mathbb R\backslash\{k\frac{\pi}2,k\in\mathbb Z\}}{\mathbb R}\]
where $x$ is in \emph{radians}. TAN does \emph{not} produce PS error\footnote{TAN is defined with
Div PSTricks operator rather than with div PS operator.} when $x=k\frac{pi}{2}$.
\begin{center}
\begin{tabular}{c|c|c|c}
  \textbf{Stack} & \textbf{Operator} & \textbf{Result} & \textbf{Description} \\
    \hline
  \textsf{\textsl{num}} & \textsf{\textbf{COS}} & \textsf{\textsl{real}} & Return cosine of
  \textsf{\textsl{num}} radians \\
    \hline
  \textsf{\textsl{num}} & \textsf{\textbf{SIN}} & \textsf{\textsl{real}} & Return sine of
  \textsf{\textsl{num}} radians \\
    \hline
  \textsf{\textsl{num}} & \textsf{\textbf{TAN}} & \textsf{\textsl{real}} & Return tangent of
  \textsf{\textsl{num}} radians
\end{tabular}
\end{center}

\begin{SideBySideExample}[xrightmargin=10.5cm]
    \begin{pspicture}*(-5,-2)(5,2)
    \SpecialCoor % For label positionning
    \psaxes[labels=y,Dx=/HLCBWz&\pstPI2"]{->}%
        (0,0)(-5,-2)(5,2)
    \uput[-90](!/HLCBWz&PI" 0){$\pi$}
    \uput[-90](!/HLCBWz&PI" neg 0){$-\pi$}
    \uput[-90](!/HLCBWz&PI" 2 div 0){$\frac{\pi}2$}
    \uput[-90](!/HLCBWz&PI" 2 div neg 0)%
        {$-\frac{\pi}2$}
    \psplot[linecolor=blue]{-5}{5}{x /HLCBWz&COS"}
    \psplot[linecolor=red]{-5}{5}{x /HLCBWz&SIN"}
    \psplot[linecolor=green]{-5}{5}{x /HLCBWz&TAN"}
    \end{pspicture}
\end{SideBySideExample}

\MathPackage{} introduces natural trigonometric postscript operators ACOS, ASIN and ATAN defined by
\[\DefOfOperator{acos}{[-1,1]}{[0,\pi]}\]
\[\DefOfOperator{asin}{[-1,1]}{[-\frac{\pi}2,\frac{\pi}2]}\]
\[\DefOfOperator{atan}{\mathbb R}{]-\frac{\pi}2,\frac{\pi}2[}\]

\begin{center}
\begin{tabular}{c|c|c|c}
  \textbf{Stack} & \textbf{Operator} & \textbf{Result} & \textbf{Description} \\
    \hline
  \textsf{\textsl{num}} & \textsf{\textbf{ACOS}} & \textsf{\textsl{angle}} & Return arccosine of
  \textsf{\textsl{num}} in radians \\
    \hline
  \textsf{\textsl{num}} & \textsf{\textbf{ASIN}} & \textsf{\textsl{angle}} & Return arcsine of
  \textsf{\textsl{num}} in radians \\
    \hline
  \textsf{\textsl{num}} & \textsf{\textbf{ATAN}} & \textsf{\textsl{angle}} & Return arctangent of
  \textsf{\textsl{num}} in radians \\
\end{tabular}
\end{center}

\textbf{Important :} ATAN is \emph{not} defined as PS operator atan. ATAN needs only \emph{one}
argument on the stack.

\begin{SideBySideExample}[xrightmargin=10.5cm]
    \begin{pspicture}(-5,-2)(5,4)
    \SpecialCoor % For label positionning
    \psaxes[labels=x,Dy=/HLCBWz&\pstPI2"]{->}%
        (0,0)(-5,-2)(5,4)
    \uput[0](!0 /HLCBWz&PI"){$\pi$}
    \uput[0](!0 /HLCBWz&PI" 2 div){$\frac{\pi}2$}
    \uput[0](!0 /HLCBWz&PI" 2 div neg)%
        {$-\frac{\pi}2$}
    \psplot[linecolor=blue]{-1}{1}%
        {x /HLCBWz&ACOS"}
    \psplot[linecolor=red]{-1}{1}%
        {x /HLCBWz&ASIN"}
    \psplot[linecolor=green]{-5}{5}%
        {x /HLCBWz&ATAN"}
    \end{pspicture}
\end{SideBySideExample}

\section{Hyperbolic trigonometry}

\MathPackage{} introduces hyperbolic trigonometric postscript operators COSH, SINH and TANH defined
by
\[\DefOfOperator{cosh}{\mathbb R}{[1,+\infty[}\]
\[\DefOfOperator{sinh}{\mathbb R}{\mathbb R}\]
\[\DefOfOperator{tanh}{\mathbb R}{]-1,1[}\]
\begin{center}
\begin{tabular}{c|c|c|c}
  \textbf{Stack} & \textbf{Operator} & \textbf{Result} & \textbf{Description} \\
    \hline
  \textsf{\textsl{num}} & \textsf{\textbf{COSH}} & \textsf{\textsl{real}} & Return hyperbolic cosine of
  \textsf{\textsl{num}} \\
    \hline
  \textsf{\textsl{num}} & \textsf{\textbf{SINH}} & \textsf{\textsl{real}} & Return hyperbolic sine of
  \textsf{\textsl{num}} \\
    \hline
  \textsf{\textsl{num}} & \textsf{\textbf{TANH}} & \textsf{\textsl{real}} & Return hyperbolic tangent of
  \textsf{\textsl{num}}
\end{tabular}
\end{center}

\begin{SideBySideExample}[xrightmargin=10.5cm]
    \begin{pspicture}*(-5,-5)(5,5)
    \psaxes{->}(0,0)(-5,-5)(5,5)
    \psplot[linecolor=blue]{-5}{5}{x /HLCBWz&COSH"}
    \psplot[linecolor=red]{-5}{5}{x /HLCBWz&SINH"}
    \psplot[linecolor=green]{-5}{5}{x /HLCBWz&TANH"}
    \end{pspicture}
\end{SideBySideExample}

\MathPackage{} introduces reciprocal hyperbolic trigonometric postscript operators ACOSH, ASINH and
ATANH defined by
\[\DefOfOperator{acosh}{[1,+\infty[}{\mathbb R}\]
\[\DefOfOperator{asinh}{\mathbb R}{\mathbb R}\]
\[\DefOfOperator{atanh}{]-1,1[}{\mathbb R}\]
\begin{center}
\begin{tabular}{c|c|c|c}
  \textbf{Stack} & \textbf{Operator} & \textbf{Result} & \textbf{Description} \\
    \hline
  \textsf{\textsl{num}} & \textsf{\textbf{ACOSH}} & \textsf{\textsl{real}} & Return reciprocal hyperbolic cosine of
  \textsf{\textsl{num}} \\
    \hline
  \textsf{\textsl{num}} & \textsf{\textbf{ASINH}} & \textsf{\textsl{real}} & Return reciprocal hyperbolic sine of
  \textsf{\textsl{num}} \\
    \hline
  \textsf{\textsl{num}} & \textsf{\textbf{ATANH}} & \textsf{\textsl{real}} & Return reciprocal hyperbolic tangent of
  \textsf{\textsl{num}}
\end{tabular}
\end{center}

\begin{SideBySideExample}[xrightmargin=10.5cm]
    \begin{pspicture}(-5,-4)(5,4)
    \psaxes{->}(0,0)(-5,-4)(5,4)
    \psplot[linecolor=blue]{1}{5}%
        {x /HLCBWz&ACOSH"}
    \psplot[linecolor=red]{-5}{5}%
        {x /HLCBWz&ASINH"}
    \psplot[linecolor=green]{-.999}{.999}%
        {x /HLCBWz&ATANH"}
    \end{pspicture}
\end{SideBySideExample}

\section{Other operators}

\MathPackage{} introduces postscript operator EXP defined by
\[\DefOfOperator{exp}{\mathbb R}{\mathbb R}\]
\begin{center}
\begin{tabular}{c|c|c|c}
  \textbf{Stack} & \textbf{Operator} & \textbf{Result} & \textbf{Description} \\
    \hline
  \textsf{\textsl{num}} & \textsf{\textbf{EXP}} & \textsf{\textsl{real}} & Return exponential of
  \textsf{\textsl{num}}
\end{tabular}
\end{center}

\begin{SideBySideExample}[xrightmargin=10.5cm]
    \begin{pspicture}*(-5,-1)(5,5)
    \psaxes{->}(0,0)(-5,-0.5)(5,5)
    \psplot[linecolor=blue,
        plotpoints=1000]{-5}{5}{x /HLCBWz&EXP"}
    \end{pspicture}
\end{SideBySideExample}

\MathPackage{} introduces postscript operator GAUSS defined by
\[\DefOfOperator*{gauss}{\mathbb R}{\mathbb R}{\displaystyle\frac{1}{\sqrt{2\pi\sigma^2}}\exp-\frac{(x-\overline x)^2}{2\sigma^2}}\]
\begin{center}
\begin{tabular}{c|c|c|c}
  \textbf{Stack} & \textbf{Operator} & \textbf{Result} & \textbf{Description} \\
    \hline
  \textsf{\textsl{num}${}_1$} \textsf{\textsl{num}${}_2$} \textsf{\textsl{num}${}_3$} &
  \textsf{\textbf{GAUSS}} & \textsf{\textsl{real}} & \parbox{4cm}{Return gaussian
  of \textsf{\textsl{num}${}_1$} with mean \textsf{\textsl{num}${}_2$} and standart deviation \textsf{\textsl{num}${}_3$}}
\end{tabular}
\end{center}

\begin{SideBySideExample}[xrightmargin=10.5cm]
    \psset{yunit=5}
    \begin{pspicture}(-5,-.1)(5,1.1)
    \psaxes{->}(0,0)(-5,-.1)(5,1.1)
    \psplot[linecolor=blue,
        plotpoints=1000]%
        {-5}{5}{x 2 2 /HLCBWz&GAUSS"}
    \psplot[linecolor=red,
        plotpoints=1000]%
        {-5}{5}{x 0 .5 /HLCBWz&GAUSS"}
    \end{pspicture}
\end{SideBySideExample}

\MathPackage{} introduces postscript operator SINC defined by
\[\DefOfOperator*{sinc}{\mathbb R}{\mathbb R}{\displaystyle\frac{\sin x}x}\]
\begin{center}
\begin{tabular}{c|c|c|c}
  \textbf{Stack} & \textbf{Operator} & \textbf{Result} & \textbf{Description} \\
    \hline
  \textsf{\textsl{num}} & \textsf{\textbf{SINC}} & \textsf{\textsl{real}} & Return cardinal sine of
  \textsf{\textsl{num}} radians
\end{tabular}
\end{center}

\begin{SideBySideExample}[xrightmargin=10.5cm]
    \psset{xunit=.25,yunit=3}
    \begin{pspicture}(-20,-.5)(20,1.5)
    \SpecialCoor % For label positionning
    \psaxes[labels=y,Dx=/HLCBWz&\pstPI1"]{->}%
        (0,0)(-20,-.5)(20,1.5)
    \uput[-90](!/HLCBWz&PI" 0){$\pi$}
    \uput[-90](!/HLCBWz&PI" neg 0){$-\pi$}
    \psplot[linecolor=blue,
        plotpoints=1000]{-20}{20}{x /HLCBWz&SINC"}
    \end{pspicture}
\end{SideBySideExample}

\MathPackage{} introduces postscript operator GAMMALN defined by
\[\DefOfOperator*{\ln\Gamma}{]0,+\infty[}{\mathbb R}{\ln\displaystyle\int_0^t t^{x-1}\mathrm e^{-t}\,\mathrm d t}\]
\begin{center}
\begin{tabular}{c|c|c|c}
  \textbf{Stack} & \textbf{Operator} & \textbf{Result} & \textbf{Description} \\
    \hline
  \textsf{\textsl{num}} & \textsf{\textbf{GAMMALN}} & \textsf{\textsl{real}} & Return logarithm of
  $\Gamma$ function of \textsf{\textsl{num}}
\end{tabular}
\end{center}

\begin{SideBySideExample}[xrightmargin=10.5cm]
    \begin{pspicture}(-.5,-.5)(6,6)
    \psaxes{->}(0,0)(-.5,-.5)(6,6)
    \psplot[linecolor=blue,
        plotpoints=1000]{.1}{6}{x /HLCBWz&GAMMALN"}
    \end{pspicture}
\end{SideBySideExample}

\begin{SideBySideExample}[xrightmargin=10.5cm]
    \psset{xunit=.25,yunit=3}
    \begin{pspicture}(-20,-.5)(20,1.5)
    \psaxes[Dx=5,Dy=.5]{->}%
        (0,0)(-20,-.5)(20,1.5)
    \psplot[linecolor=blue,
        plotpoints=1000]{-20}{20}%
        {x /HLCBWz&BESSEL_J0"}
    \psplot[linecolor=red,
        plotpoints=1000]{-20}{20}%
        {x /HLCBWz&BESSEL_J1"}
    \end{pspicture}
\end{SideBySideExample}

\begin{SideBySideExample}[xrightmargin=10.5cm]
    \psset{xunit=.5,yunit=3}
    \begin{pspicture}*(-1.5,-.75)(19,1.5)
    \psaxes[Dx=5,Dy=.5]{->}%
        (0,0)(-1,-.75)(19,1.5)
    \psplot[linecolor=blue,
        plotpoints=1000]{0.0001}{20}%
        {x /HLCBWz&BESSEL_Y0"}
    \psplot[linecolor=red,
        plotpoints=1000]{0.0001}{20}%
        {x /HLCBWz&BESSEL_Y1"}
    %    \psplot[linecolor=green,
    %        plotpoints=1000]{0.0001}{20}%
    %        {x 2 /HLCBWz&BESSEL_Yn"}
    \end{pspicture}
\end{SideBySideExample}

\section{\texttt{Infix-RPN} and \texttt{pst-infixplot} support}

You can now use the operators defined in \MathPackage{} with the infix notation, using the
\texttt{infix-RPN} package. The packages must be read in the fellowing order:

\begin{Verbatim}[label={\LaTeX{} preamble}]
    \usepackage{infix-RPN,pst-math}
\end{Verbatim}

If you want to use \MathPackage{} with \texttt{pst-infixplot}, then read the packages in the
fellowing order:

\begin{Verbatim}[label={\LaTeX{} preamble}]
    \usepackage{pst-infixplot,pst-math}
\end{Verbatim}

\begin{SideBySideExample}[xrightmargin=10.5cm]
    \psset{xunit=.5,yunit=3}
    \begin{pspicture}*(-1.5,-.75)(19,1.5)
    \psaxes[Dx=5,Dy=.5]{->}%
        (0,0)(-1,-.75)(19,1.5)
    \psPlot[linecolor=red,
      plotpoints=1000]{0.0001}{20}%
      {BESSEL_Y0(x)-BESSEL_Y1(x)}
    \end{pspicture}
\end{SideBySideExample}

\section{Credits}

Many thanks to Jacques L'helgoualc'h and Herbert Voss.


\end{document}
