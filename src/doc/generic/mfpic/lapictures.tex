%%% File: lapictures.tex
%%% A part of mfpic 0.9 2006/05/26
%%%
% Magnify to same scale as pictures.tex :
% (Use of \mag is against the law of LaTeX, but Bugs Bunny set a precedent.)
\mag=1200
\documentclass{article}

%%!!!!!!!!!!!!!!!!!!!!!!!!!!!
% adjust these to your liking:
\setlength{\paperheight}{11truein}
\setlength{\paperwidth}{8.5truein}

% Default plainTeX margins
\setlength{\textwidth}{\paperwidth}
\addtolength{\textwidth}{-2truein}
\setlength{\textheight}{\paperheight}
\addtolength{\textheight}{-2.1truein}
\setlength{\headheight}{0in}
\setlength{\headsep}{0in}
\setlength{\oddsidemargin}{0in}
\setlength{\evensidemargin}{\oddsidemargin}
\setlength{\footskip}{\baselineskip}

% Use the metafont option if you prefer.
\usepackage[metapost,raggedcaptions]{mfpic}

\ifx\pdfoutput\UndEfInEd
\else
  \setlength{\pdfpageheight}{\paperheight}
  \setlength{\pdfpagewidth}{\paperwidth}
\fi

\opengraphsfile{pics}

\setlength{\mfpframesep}{0pt}
\headshape{1}{1}{true}
\newcommand{\vs}{\bigskip\filbreak}
%\renewcommand\raggedright{\rightskip 0pt plus2em \spaceskip.3333em \xspaceskip.5em\relax}
\begin{document}
\mftitle{Commutative Diagram example.}

%  A-----D
%  |\   /|
%  | C-F |
%  |/   \|
%  B.....E

\noindent
\begin{mfpframe}
\begin{mfpic}[12]{0}{10}{0}{10}
\tlabeljustify{cc}
\tcaption[2.0,1.0]{{\it Figure 1:}  Commutative diagram example.}
\tlabels{(1,9){A}
         (1,1){B}
         (3,5){C}
         (9,9){D}
         (9,1){E}
         (7,5){F}}
\setrender{\arrow\draw\trimpath{6pt}}
\lines{(1,9), (1,1)}  % A -> B.
\lines{(1,9), (3,5)}  % A -> C.
\lines{(3,5), (1,1)}  % C -> B.
\lines{(1,9), (9,9)}  % A -> D.
\lines{(9,9), (9,1)}  % D -> E.
\lines{(9,9), (7,5)}  % D -> F.
\lines{(7,5), (9,1)}  % F -> E.
\lines{(3,5), (7,5)}  % C -> F.
% B- - ->E :
%\dotted\arrow[r90][b-12pt]\arrow[b15pt]\reverse\arrow\lines{(1.5,1), (8.5,1)}
\arrow\arrow[b10pt]\reverse\arrow\dotted\trimpath{6pt}\lines{(1,1), (9,1)}
\end{mfpic}
\end{mfpframe}

\vs

\mftitle{Function Plot with Cartesian Axes.}

\noindent
\begin{mfpframe}
\begin{mfpic}[20]{-3}{3}{-3}{3}
\axes
\function{-2,2,0.1}{((x**3)-x)/3}
\tcaption{{\it Figure 2:}  Function Plot with Cartesian Axes.}
\end{mfpic}
\end{mfpframe}

\vs

\mftitle{Parametric Function Plot, and Filled Circle.}

\noindent
\begin{mfpframe}
\begin{mfpic}[30]{-1.5}{1.5}{-1}{1}
\parafcn{0,6,0.1}{cosd(150t)*dir(90t)}
\gfill\circle{(0,0),0.25}
\tcaption{{\it Figure 3:}  Parametric Function Plot, and Filled Circle.}
\end{mfpic}
\end{mfpframe}

\vs

\mftitle{Bar Graph.}

\noindent
\begin{mfpframe}
\begin{mfpic}[20]{-0.5}{4}{-0.5}{4}
\axes
\shade\draw\rect{(0,0),(1,0.5)}
\darkershade
\shade\draw\rect{(1,0),(2,1)}
\hatch\draw\rect{(2,0),(3,2)}
\tcaption{{\it Figure 4:}  Bar Graph.}
\end{mfpic}
\end{mfpframe}

\vs

\mftitle{Pie Chart.}

\noindent
\begin{mfpframe}
\begin{mfpic}[30]{-1.3}{1.7}{-1}{1.1}
\gfill\sector{(0.3,0.2), 1, 0,60}
\shade\sector{(0,0), 1, 60,105}
\turtle{(0,0), \plr{(1,105)}}
\sector{(0,0), 1, 60,360}
\tcaption{{\it Figure 5:}  Pie Chart.}
\end{mfpic}
\end{mfpframe}

\noindent Unindented text here.

\vs

\mftitle{Circle with Arrow.}

\noindent
\begin{mfpframe}
\begin{mfpic}[20]{-2}{2}{-1}{1}
\arrow\circle{(0,0),1}
\tcaption{{\it Figure 6:}  Circle with Arrow.}
\end{mfpic}
\end{mfpframe}

\vs

\mftitle{Use of hatch, draw, lclosed, connect, curve, point, lines,
  dotted, reverse.}

\noindent
\begin{mfpframe}
\begin{mfpic}[20]{-3}{3}{-3}{3}
\hatch\draw\lclosed\connect
\curve{(1,0), (1,0.5), (1,1), (0,1.5)}
\point{(0,0)}
\endconnect
\lines{(-1,1), (-1,-1), (1,-1.5)}
\point{(0,0)}
\dotted\reverse\lines{(-2,2), (-2,-2), (2,-3)}
\tcaption{{\it Figure 7:}  Use of hatch, draw, lclosed, connect,
  curve, point, lines, dotted, reverse.}
\end{mfpic}
\end{mfpframe}

\vs

\mftitle{Simpler variant of the previous figure.}

\noindent
\begin{mfpframe}
\begin{mfpic}[40]{-1}{1}{-1}{1}  % Was `[20]'.
\tcaption{{\it Figure 8:} Simpler variant of the previous figure.}
\hatch\draw\lclosed\connect
\curve{(1,0), (0.5,0.25), (0.5,0.5), (0,0.75)}
\point{(0,0)}
\endconnect
\reverse\lines{(-0.5,0.5), (-0.5,-0.5), (0.5,-0.75)}
\tcaption{{\it Figure 8:} Simpler variant of the previous figure.}
\end{mfpic}
\end{mfpframe}

\vs

\mftitle{Graph of data from file.}

\noindent
\begin{mfpframe}
\begin{mfpic}[6]{-10}{10}{-2}{12}
  \makepercentother
    \using{#1% #2 #3}{(#1,#2)}
  \makepercentcomment
  \mfpdatacomment\#
  \fcncurve\datafile{data.dat}
  \axes
\tcaption{{\it Figure 9:} Example of a graph drawn from data in a file}
\end{mfpic}
\end{mfpframe}

\closegraphsfile

\end{document}

%%%
%%%  EOF  lapictures.tex
%%%
