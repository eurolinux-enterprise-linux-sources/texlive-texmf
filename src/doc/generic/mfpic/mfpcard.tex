%%% File: mfpcard.tex
%%% A part of mfpic 0.9 2006/05/26
%%%
% (Ask your driver to print so that longer dimension is horizontal.)
\newdimen\paperlongside
    \paperlongside 11 truein
\newdimen\papershortside
    \papershortside 8.5 truein

\ifx\pdfoutput\undefined
\else\ifx\pdfoutput\relax
\else\ifnum\pdfoutput>0
  \pdfpagewidth\paperlongside
  \pdfpageheight\papershortside
\fi\fi\fi

\font\manual=cmcsc8
\def\MF{{\manual meta}\-{\manual font}}
\def\MP{{\manual meta}\-{\manual post}}
\def\TeX{{\rm T\kern-.1667em\lower.5ex\hbox{E}\kern-.125emX\null}}
\def\plainTeX{{\rm plain\kern -.15em\TeX}}
\def\LaTeX{{\rm L\raise.42ex\hbox{\kern-.36em \manual a}\kern-.15em\TeX}}
%\def\LaTeXe{\LaTeX$2_{\textstyle\varepsilon}$}
\def\gobble#1{}
\def\bsl{{\tt\char\escapechar}}

\def\cs#1{{\tt\char`\\#1}}
\def\marg#1{{\tt\char`\{#1\char`\}}}
\def\oarg#1{{\tt[#1]}}
\def\parg#1{{\tt(#1)}}

% logical markup to typeset filenames, program names, MF code,
% grafbase identifiers, environment names, email addresses, and
% emphasis.
\def\file#1{{\tt #1}}
\def\prog#1{{\sf #1}} % didn't want everything in \tt
\def\mfc#1{{\tt #1}}
\def\opt#1{{\tt #1}}  % LaTeX2e option names
\def\gbc#1{{\tt #1}}
\def\env#1{{\tt #1}}
\def\mail#1{{\tt #1}}
\def\emph#1{{\it #1\/}}

% Forced, Clean, Linebreak.

\def\brk{\nobreak\hfil\break}

\newcount\columnsperpage

% Entries are too wide for 3 columns.  Print 2 columns landscape
\columnsperpage=2

% The rest of this header is nicked from gnuplot (which took it from
% Emacs) but was drastically adapted for mfpic.
%
% This file is intended to be processed by plainTeX.
%
%    You will need to inform your dvi driver to print it in landscape
%    and, optionally, double sided.
% Author of GNU Emacs Refcard:
%  Stephen Gildea
%  UUCP: mit-erl!gildea
%  Internet: gildea@erl.mit.edu
%
% Person who ripped off the formatter for gnuplot:
%  Alex Woo
%  NASA Ames Research Center
%  Internet: woo@ames.arc.nasa.gov
%
% Modified on 9 Dec 1992 by:
%  Daniel S. Lewart
%  University of Illinois
%  Internet: d-lewart@uiuc.edu
%
% Modified on 29 Sep 1999, 11 Jul 2002, 29 Sep 2003 (for mfpic) by:
%  Daniel H. Luecking
%  University of Arkansas
%  luecking@uark.edu

\newdimen\intercolumnskip
\newbox\columna
\newbox\columnb

\hsize4.75truein
\vsize 7.25truein

% Equal left right, right and middle margins, about .5in on USletter
\intercolumnskip=\paperlongside
\advance\intercolumnskip -2\hsize
\intercolumnskip .33333\intercolumnskip
\hoffset\intercolumnskip
\advance\hoffset -1truein

% Equal top and bottom margins, about .4in on A4 paper
\voffset\papershortside
\advance\voffset-\vsize
\voffset .5\voffset
\advance\voffset -1truein

\topskip8truept

\font\manual=cmcsc10 at 8pt
\font\titlefont=cmbx10 scaled \magstep1
\font\headingfont=cmbx10
\font\smallfont=cmr6
\font\smallsy=cmsy6
\font\eighti=cmmi8
\font\sixi=cmmi6
\font\eightex=cmex10 at 8pt
\font\eightrm=cmr8
\font\eightbf=cmbx8
\let\subheadingfont\eightbf
\font\sixbf=cmbx6
\font\eightit=cmti8
\font\eighttt=cmtt8
\font\eightsy=cmsy8
\font\sixsy=cmsy6
\font\eightsf=cmss10 at 8pt
\font\eightsl=cmsl8
\font\tensf=cmss10
\textfont0=\eightrm \scriptfont0=\smallfont
\textfont1=\eighti \scriptfont1=\sixi
\textfont2=\eightsy \scriptfont2=\sixsy
\textfont3=\eightex \scriptfont3=\eightex \scriptscriptfont3=\eightex
\textfont\bffam\eightbf \scriptfont\bffam\sixbf
\def\rm{\fam0\eightrm}
\def\bf{\fam\bffam\eightbf}
\def\it{\eightit}
\def\tt{\eighttt}
\def\sf{\eightsf}
\def\sl{\eightsl}
\smallskipamount 2pt plus 1pt minus 0pt
\medskipamount 5pt plus 2pt minus 2pt
\bigskipamount 10pt plus 3pt minus 3pt
\normalbaselineskip=10pt
\normalbaselines
\rm
\setbox\strutbox=\hbox{\vrule height6.8pt depth3.2pt width0pt}

\def\abc{a}
\output={%
    \if b\abc \multicolumnformat \global\def\abc{a}
    \else\if a\abc
      \global\setbox\columna\columnbox \global\def\abc{b}
      \global\setbox\columnb\hbox to -\intercolumnskip{}
    \fi\fi}
\def\multicolumnformat{\shipout\vbox{%
    \hbox{\box\columna\hskip\intercolumnskip
      \box\columnb\hskip\intercolumnskip\columnbox}}\advancepageno}
\def\columnbox{\leftline{\pagebody}}

\def\bye{\par\vfill\supereject
  \if a\abc \else\null\vfill\eject\fi
  \if a\abc \else\null\vfill\eject\fi
  \end}

% Typewriter type braces (redefinitions of escaped braces) :
\def\{{{\tt \char`\{}}
\def\}{{\tt\char`\}}}


\parindent 0pt

\def\title#1{{\titlefont\centerline{#1}}\vskip 1ex plus .5ex}

\def\section#1{\par
  \penalty-100
  \vskip 2.25ex plus 2ex minus 2ex {\headingfont #1}%
  \par \nobreak
  \vskip 1.25ex plus 1ex minus 1.5ex}

\def\subsection#1{\par
  \ifdim\lastskip<1.5ex
    \vskip -\lastskip
    \vskip 1.5ex plus 1ex minus 1ex
  \fi
  {\subheadingfont #1}%
  \par \nobreak
  \vskip .75ex plus .5ex minus .5ex}

\newdimen\describeindent
\describeindent=0em

\def\meta#1{$\langle\hbox{\it #1\/}\rangle$}
\def\alt{\unskip$\;|\;$\hfilbreak}
\def\hfilbreak{\hfil\allowbreak\hfilneg}

\newdimen\secondkeyindent

\def\describe#1#2{\leavevmode\vtop
  {\hsize=\firstkeyfrac\hsize\rightskip=1em plus 1 fill\relax
   \secondkeyindent\describeindent \advance\secondkeyindent 1em
   \hangindent\secondkeyindent \hangafter1
   \hskip\describeindent\relax\strut#1\nobreak\hskip0pt\strut}%
   \vtop
  {\hsize=\secondkeyfrac\hsize\rightskip=0pt plus 1 fill\relax
   \leavevmode\strut#2\unskip\strut}\par}
\def\code#1{{\sf #1}}

\title{MFPIC Quick Reference}

\centerline{(Copyright 2000--2006 by Daniel Luecking)}
\medskip
This information was prepared for version 0.9 of \prog{mfpic}.

\section{Preamble commands}

\def\firstkeyfrac{.4}
\def\secondkeyfrac{.6}

\describe{Load \prog{mfpic} package (\LaTeX)}
        {\cs{usepackage}\oarg{\meta{options}}\marg{mfpic}}
\describe{Options}{\opt{metafont} \alt \opt{metapost}, \opt{mplabels},
        \opt{overlaylabels}, \opt{centeredcaptions},
        \opt{raggedcaptions}, \opt{clip}, \opt{truebbox}, \opt{draft},
        \opt{final}, \opt{nowrite}, \opt{mfpreadlog}}
\describe{Load \prog{mfpic}; activate options (\plainTeX)}
        {\cs{input} \file{mfpic}. \cs{usemetafont} \alt
        \cs{usemetapost}, \cs{usemplabels}, \cs{overlaylabels}
        \cs{usecenteredcaptions}, \cs{useraggedcaptions},
        \cs{clipmfpic}, \cs{usetruebbox}, \cs{mfpicdraft},
        \cs{mfpicfinal}, \cs{mfpicnowrite}, \cs{mfpreadlog}}
\describe{Turn off some options}
        {\cs{nomplabels}, \cs{nooverlaylabels},
        \cs{nocenteredcaptions}, \cs{noraggedcaptions}, \cs{noclipmfpic},
        \cs{notruebbox}}
\describe{Set up/close the output file}
        {\cs{opengraphsfile}\marg{\meta{base name}}$\ldots$\cs{closegraphsfile}}

\section{The {\tensf mfpic} environment}

\def\firstkeyfrac{.30}
\def\secondkeyfrac{.70}

\describe{Start an \prog{mfpic} figure}
    {\cs{mfpic}\oarg{\meta{xscale}}\oarg{\meta{yscale}}%
    \marg{\meta{$x_{\min}$}}\allowbreak\marg{\meta{$x_{max}$}}\allowbreak
    \marg{\meta{$y_{min}$}}\allowbreak\marg{\meta{$y_{max}$}}\brk
    \null\quad\meta{mfpic commands}\brk
    \cs{endmfpic}}
\describe{\quad\LaTeX{} (optional)}
        {\cs{begin}\marg{mfpic}${}\equiv{}$\cs{mfpic},
          \cs{end}\marg{mfpic}${}\equiv{}$\cs{endmfpic}}


\section{Dimensions (lengths)}

\def\firstkeyfrac{.58}
\def\secondkeyfrac{.42}

\describe{\sl Purpose; where used:} {\sl Name and default value:}
\vskip1pt
\describe{Unit of length; \cs{mfpic}} {\cs{mfpicunit}, \tt 1pt}
\describe{Size of a symbol; \cs{point}, \cs{plot}, and \cs{plotsymbol}}
        {\cs{pointsize}, \tt 2pt}
\describe{Darkness of shading; \cs{shade}} {\cs{shadespace}, \tt 1pt }
\describe{Space between dots; \cs{polkadot}}
    {\cs{polkadotspace}, \tt 10pt}
\describe{Space between hatch lines; hatching macros}
    {\cs{hatchspace}, \tt 3pt}
\describe{Size of arrowhead; \cs{arrow}} {\cs{headlen}, \tt 3pt}
\describe{Size of x-, y-axis arrowhead; xy-axes macros}
    {\cs{axisheadlen}, \tt 5pt}
\describe{Size of border axis arrowhead; side axis macros}
    {\cs{sideheadlen}, \tt 0pt}
\describe{Size of marks on axes; axis marks} {\cs{hashlen}, \tt 4pt}
\describe{Size of dashes; \cs{dashed}} {\cs{dashlen}, \tt 4pt}
\describe{Space between dashes; \cs{dashed}} {\cs{dashspace}, \tt 4pt}
\describe{Size of dots; \cs{dotted}} {\cs{dotsize}, \tt 0.5pt}
\describe{Space between dots; \cs{dotted}} {\cs{dotspace}, 3pt}
\describe{Space between symbols; \cs{plot}} {\cs{symbolspace}, \tt 5pt}

\medskip
The following commands are used to change the size of some dimension
parameters:

\smallskip
\describe{{\sl Purpose} ({\sl default\/}):} {\sl Command:}
\vskip1pt
\describe{Set diameter of drawing pen ({\tt 0.5pt})}
        {\cs{penwd}\marg{\meta{dimen}}}
\describe{Set diameter of shading dots ({\tt 0.5pt})}
        {\cs{shadewd}\marg{\meta{dimen}}}
\describe{Set diameter of polkadot ({\tt 5pt})}
        {\cs{polkadotwd}\marg{\meta{dimen}}}
\describe{Set diameter of hatching pen ({\tt 0.5pt})}
        {\cs{hatchwd}\marg{\meta{dimen}}}
\describe{Multiply \cs{shadespace} by 1.2} {\cs{lightershade}}
\describe{Divide \cs{shadespace} by 1.2} {\cs{darkershade}}

%\vfil\eject
\section{Colors}

\def\firstkeyfrac{.45}
\def\secondkeyfrac{.55}

\smallskip
\describe{Set color for curves} {\cs{drawcolor}\marg{\meta{color}}}
\describe{Set color for fills} {\cs{fillcolor}\marg{\meta{color}}}
\describe{Set color for points, symbols} {\cs{pointcolor}\marg{\meta{color}}}
\describe{Set color for hatching} {\cs{hatchcolor}\marg{\meta{color}}}
\describe{Set color for arrowheads} {\cs{headcolor}\marg{\meta{color}}}
\describe{Set color for tlabels} {\cs{tlabelcolor}\marg{\meta{color}}}
\describe{Set color used by \cs{gclear}} {\cs{backgroundcolor}\marg{\meta{color}}}
\describe{\quad \LaTeX{} syntax}
        {\cs{drawcolor}\oarg{\meta{model}}\marg{\meta{clrspec}}, etc.}
\describe{Define a color name} {\cs{mfpdefinecolor}\marg{\meta{name}}\marg{\meta{model}}\marg{\meta{clrspec}}}


\section{Common geometric figures}

Drawing commands that operate on a variable length list in braces may
be followed by \cs{datafile}\marg{\meta{filename}} instead of the list.

\subsection{Points}

\def\firstkeyfrac{.42}
\def\secondkeyfrac{.58}

\describe{Place a symbol at given point(s)}
        {\cs{plotsymbol}\oarg{\meta{size}}\marg{\meta{name}}%
         \marg{($x_0,y_0$),($x_1,y_1$),$\ldots$}}
\describe{\quad Available symbol names} {{\tt Triangle}, {\tt Square}, {\tt Circle}, {\tt
        Diamond}, {\tt Star}, {\tt SolidTriangle}, {\tt SolidSquare}, {\tt SolidCircle},
        {\tt SolidDiamond}, \tt{SolidStar}, {\tt Plus}, {\tt Cross},
        {\tt Asterisk}}
\describe{Points (filled or unfilled circles)}
        {\cs{point}\oarg{\meta{size}}\marg{($x_0,y_0$),($x_1,y_1$),$\ldots$}}
\describe{Force filled/open circles in \cs{point}:}
        {\cs{pointfilltrue}/\cs{pointfillfalse}}

\subsection{Lines}

\def\firstkeyfrac{.45}
\def\secondkeyfrac{.55}

\describe{Connect points with lines}
        {\cs{polyline}\marg{($x_0,y_0$),($x_1,y_1$),$\ldots$},
         or (\cs{lines})}
\describe{Closed polygon}
        {\cs{polygon}\marg{($x_0,y_0$),($x_1,y_1$),$\ldots$}}
\describe{Concatenate vectors}
        {\cs{turtle}\marg{\meta{initialpoint},\meta{$v_1$},\meta{$v_2$},$\ldots$}}
\describe{Rectangle (upright) with given corners}
        {\cs{rect}\marg{($x_0,y_0$),($x_1,y_1$)}}

\subsection{Circles, arcs and ellipses}

\def\firstkeyfrac{.55}
\def\secondkeyfrac{.45}

\describe{Circles} {}
\describe{\qquad polar form (default):}
        {\cs{circle}\oarg{p}\marg{\meta{center},\meta{radius}}}
\describe{\qquad three-point form:}
        {\cs{circle}\oarg{t}\marg{($x_0,y_0$),($x_1,y_1$),($x_2,y_2$)}}
\describe{\qquad center-point form:}
        {\cs{circle}\oarg{c}\marg{\meta{center},\meta{point}}}
\describe{\qquad point-sweep form:}
        {\cs{circle}\oarg{s}\marg{($x_0,y_0$),($x_1,y_1$),\meta{angle}}}
\describe{Arcs} {}
\describe{\qquad polar form:}
        {\cs{arc}\oarg{p}\marg{\meta{center},\meta{$\theta_1$},\meta{$\theta_2$},%
        \meta{radius}}}
\describe{\qquad three-point form:}
        {\cs{arc}\oarg{t}\marg{($x_0,y_0$),($x_1,y_1$),($x_2,y_2$)}}
\describe{\qquad center-point-angle form:}
        {\cs{arc}\oarg{c}\marg{\meta{center},\meta{point},\meta{angle}}}
\describe{\qquad point-sweep form (default):}
        {\cs{arc}\oarg{s}\marg{($x_0,y_0$),($x_1,y_1$),\meta{angle}}}
\describe{Ellipse, center $(x_0,y_0)$, radii \meta{$r_x$}, \meta{$r_y$}, angle
\meta{$\theta$}}
        {\cs{ellipse}\oarg{\meta{$\theta$}}%
        \tt\{($x_0,y_0$),\meta{$r_x$},\meta{$r_y$}\}}

\section{General curves}

\def\firstkeyfrac{.45}
\def\secondkeyfrac{.55}

A \meta{spec} can be {\tt p} (for polyline) or {\tt s} (for smooth)
followed by a number for the tension.

\smallskip
\describe{Smooth curve through points}
        {\cs{curve}\oarg{\meta{tension}}\marg{($x_0,y_0$),($x_1,y_1$),$\ldots$}}
\describe{Graph of $y=f(x)$}
        {\cs{function}\oarg{\meta{spec}}\marg{$x_{\min}$,$x_{\max}$,%
         $\Delta x$}\marg{$f(\hbox{\tt x})$}}
\describe{Graph of parametric curve $(x(t),y(t))$}
        {\cs{parafcn}\oarg{\meta{spec}}\marg{$t_{\min}$,$t_{\max}$,$\Delta t$}%
         \marg{($x(\hbox{\tt t})$, $y(\hbox{\tt t})$)}}
\describe{Graph of $r = f(\theta)$}
        {\cs{plrfcn}\oarg{\meta{spec}}%
        \marg{$\theta_{\min}$,$\theta_{\max}$,$\Delta\theta$}\marg{$f(\hbox{\tt t})$}}
\describe{Interpolate with a smooth {\it function}}
        {\cs{fcncurve}\oarg{\meta{tension}}\marg{($x_0,y_0$),($x_1,y_1$),$\ldots$}}
\describe{Curve from data in a file}
        {\cs{datafile}\oarg{\meta{spec}}\marg{\meta{file}}}
\describe{\quad Set how \cs{datafile} processes a line}
        {\cs{using}\marg{\meta{read\_pattern}}\marg{\meta{write\_pattern}}\brk
        Default is \cs{using}\marg{\#1 \#2 \#3}\marg{(\#1,\#2)}}


\section{Regions}

\def\firstkeyfrac{.45}
\def\secondkeyfrac{.55}

Curves are not necessarily `closed' even if the start and end are the
same. The following are closed (may be filled), as are \cs{rect},
\cs{polygon}, \cs{circle}, and \cs{ellipse}.

\smallskip
\describe{Closed curve through given points}
        {\cs{cyclic}\oarg{\meta{tension}}\marg{$(x_1,y_1),(x_2,y_2),\ldots$}}
\describe{Circular sector (pie slice)}
        {\cs{sector}\marg{\meta{center},\meta{radius},\meta{$\theta_1$},%
         \meta{$\theta_2$}}}
\describe{Region between two functions}
        {\cs{btwnfcn}\oarg{\meta{spec}}\marg{$x_{\min}$,%
         $x_{\max}$,$\Delta x$}\marg{$f(\hbox{\tt x})$}\marg{$g(\hbox{\tt x})$}}
\describe{Region in polar coordinates}
        {\cs{plrregion}\oarg{\meta{spec}}\marg{$\theta_{\min}$,$\theta_{\max}$,%
        $\Delta \theta$}\marg{$f(\hbox{\tt t})$}}
\describe{Curves surrounding text}
        {\cs{tlabelrect}\oarg{\meta{radius}}\parg{\meta{x},\meta{y}}\marg{\meta{text}}\brk
         \cs{tlabeloval}\oarg{\meta{mult}}\parg{\meta{x},\meta{y}}\marg{\meta{text}}\brk
         \cs{tlabelellipse}\oarg{\meta{ratio}}\parg{\meta{x},\meta{y}}\marg{\meta{text}}\brk
         \meta{radius}: round corners. \meta{mult}: stretch horizontally.
         \meta{ratio}: width/height of ellipse}


\section{Prefix macros}

\subsection{Drawing curves}

\def\firstkeyfrac{.4}
\def\secondkeyfrac{.6}

\describe{Dashed path}{\cs{dashed}\oarg{\meta{length},\meta{gap}}$\ldots$}
\describe{Dotted path}{\cs{dotted}\oarg{\meta{size},\meta{gap}}$\ldots$}
\describe{Trace a path with symbols}
        {\cs{plot}\oarg{\meta{size},\meta{gap}}\marg{\meta{symbol}}$\ldots$}
\describe{Generalized dashes}{\cs{gendashed}\marg{\meta{patname}}$\ldots$}
\describe{\quad Define a named dash pattern}
        {\cs{dashpattern}\marg{\meta{patname}}\marg{\meta{len$_1$},\meta{len$_2$},$\ldots$,%
            \meta{len$_{2n}$}}}
\describe{Place a symbol at all nodes}
        {\cs{plotnodes}\oarg{\meta{size}}\marg{\meta{symbol}}$\ldots$}
\describe{Solid curve}{\cs{draw}\oarg{\meta{color}}$\ldots$}


\subsection{Closing a curve}

\def\firstkeyfrac{.45}
\def\secondkeyfrac{.55}

These turn any path into a `closed' path (result may be filled).

\smallskip
\describe{Close with a straight line,}{\cs{lclosed}$\ldots$}
\describe{Close with a smooth join, like \cs{cycle},}{\cs{sclosed}$\ldots$}
\describe{Close letting \MF{} choose}{\cs{bclosed}$\ldots$}

\subsection{Filling closed curves}

\def\firstkeyfrac{.4}
\def\secondkeyfrac{.6}

These filling prefixes turn off drawing of the curve.

\smallskip
\describe{Solid fill} {\cs{gfill}\oarg{\meta{color}}$\ldots$}
\describe{Unfill} {\cs{gclear}$\ldots$}
\describe{Hatched fills}
        {\cs{thatch}\oarg{\meta{space},\meta{angle}}\oarg{\meta{color}}$\ldots$}
\describe{\qquad \meta{angle}${}=45\,$deg}
        {\cs{rhatch}\oarg{\meta{space}}\oarg{\meta{color}}$\ldots$}
\describe{\qquad \meta{angle}${}=-45\,$deg}
        {\cs{lhatch}\oarg{\meta{space}}\oarg{\meta{color}}$\ldots$}
\describe{\qquad crosshatching}
        {\cs{xhatch}\oarg{\meta{space}}\oarg{\meta{color}}$\ldots$}
\describe{}  {\cs{hatch} $=$ \cs{xhatch}}
\describe{Shading}
        {\cs{shade}\oarg{\meta{space}}$\ldots$}
\describe{Polkadot fill}{\cs{polkadot}\oarg{\meta{space}}$\ldots$}
\describe{Fill with copies of a tile} {\cs{tess}\marg{\meta{tile}}$\ldots$}
\describe{\quad Define a tile*}
        {\cs{tile}\marg{\meta{name},\meta{unit},\meta{width},\meta{height},\meta{clip}}
        \meta{drawing commands} \cs{endtile}}

\smallskip
* Creates a mini-mfpic, clipped if \meta{clip}${}={}${\tt true}.

\subsection{Storing and reusing a path}

\describe{Store a path} {\cs{store}\marg{\meta{name}}$\ldots$}
\describe{\quad reusing a stored path} {\cs{mfobj}\marg{\meta{name}}}

\subsection{Subpaths}

\describe{Subpath by fraction of length}
        {\cs{partpath}\marg{\meta{frac1},\meta{frac2}}$\ldots$}
\describe{Subpath by node number} {\cs{subpath}\marg{\meta{m},\meta{n}}$\ldots$}
\describe{Cutting by another path}
        {\cs{cutoffafter}\marg{\meta{obj}}$\ldots$,
        \cs{cutoffbefore}\marg{\meta{obj}}$\ldots$.\brk
        \meta{obj} is a name created with \cs{store}}
\describe{Trim the ends of a path}
        {\cs{trimpath}\marg{\meta{dim1},\meta{dim2}}$\ldots$}


\subsection{Modifying a curve}

\def\firstkeyfrac{.4}
\def\secondkeyfrac{.6}

\describe{Add arrowhead to the end}
        {\cs{arrow}\oarg{l\meta{length}}\oarg{r\meta{angle}}%
         \oarg{b\meta{backset}}\oarg{c\meta{color}}$\ldots$}
\describe{\quad Define arrowhead shape}
        {\cs{headshape}\marg{\meta{ratio}}\marg{\meta{tension}}\marg{\meta{filled}}}
\describe{Reverse a curve} {\cs{reverse}$\ldots$}
\describe{Double arrow} {\cs{arrow}\cs{reverse}\cs{arrow}$\ldots$}
\describe{Rotate around a point} {\cs{rotatepath}\marg{($x_0$,$y_0$),\meta{angle}}$\ldots$}
\describe{Reflect about a line} {\cs{reflectpath}\marg{($x_0$,$y_0$),($x_1$,$y_1$)}$\ldots$}
\describe{Shift} {\cs{shiftpath}\marg{($dx$,$dy$)}$\ldots$}
\describe{Scale around a point} {\cs{scalepath}\marg{($x_0$,$y_0$),\meta{scale}}$\ldots$}
\describe{xscale about line $x=x_0$} {\cs{xscalepath}\marg{$x_0$,\meta{scale}}$\ldots$}
\describe{yscale about line $y=y_0$} {\cs{yscalepath}\marg{$y_0$,\meta{scale}}$\ldots$}
\describe{slant, pivoting on line $y=y_0$} {\cs{slantpath}\marg{$y_0$,\meta{slant}}$\ldots$}
\describe{yslant, pivoting on line $x=x_0$} {\cs{yslantpath}\marg{$x_0$,\meta{slant}}$\ldots$}
\describe{Swap x and y} {\cs{xyswappath}$\ldots$}


\section{Axes}

\def\firstkeyfrac{.30}
\def\secondkeyfrac{.70}

\describe{Draw x- and/or y-axes} {\cs{axes}\oarg{\meta{headlen}},
        \cs{xaxis}\oarg{\meta{headlen}}, \cs{yaxis}\oarg{\meta{headlen}}}
\describe{Draw various axes}
        {\cs{axis}\oarg{\meta{headlen}}\marg{\meta{axis}}, \ \meta{axis} is one of
        {\tt x}, {\tt y}, {\tt l}, {\tt b}, {\tt r}, or {\tt t}.}
\describe{Draw many axes} {\cs{doaxes}\oarg{\meta{headlen}}\marg{\meta{list}}, \
        \meta{list} of letters, no commas.}
\describe{Shift border axis inward}
        {\cs{axismargin}\marg{\meta{axis}}\marg{\meta{amt}}, \ \meta{amt} is
        in graph units.}
\describe{Add hashmarks to axes}{%
        \cs{axismarks}\marg{\meta{axis}}\oarg{\meta{len}}\marg{$c_1$,$c_2$,$\ldots$}, \
        $c_j$ are positions.\brk Abbrev. by
        \cs{xmarks} for \cs{axismarks}\marg{x}, etc.}
\describe{Change position of hash marks}
        {\cs{setaxismarks}\marg{\meta{axis}}\marg{\meta{pos}} \ \meta{pos} is one of
            {\tt inside}, {\tt outside}, {\tt centered}, {\tt ontop},
            {\tt onbottom}, {\tt onleft}, or {\tt onright}.}

\section{Miscellaneous}

\def\firstkeyfrac{.30}
\def\secondkeyfrac{.70}

\describe{Text labels}
        {\cs{tlabel}\oarg{\meta{pos}\meta{$\theta$}}%
                \parg{\meta{x},\meta{y}}\marg{\meta{\TeX{} text}}\brk
         \cs{tlabels}\marg{\meta{args$_1$}\meta{args$_2$}$\ldots$}\brk
         \cs{axislabels}\marg{\meta{axis}}\oarg{\meta{pos}\meta{$\theta$}}%
                \marg{\marg{\meta{txt$_1$}}\meta{$n_1$},%
                \marg{\meta{txt$_2$}}\meta{$n_2$},$\ldots$}\brk
            \meta{pos} is a two-letter sequence, \meta{$\theta$} the
            angle* of rotation in degrees;
            \meta{args$_j$} is an entire set of arguments as in \cs{tlabel};
            \meta{axis} is a letter, \meta{txt$_j$} is label, \meta{$n_j$}
            is coordinate on axis}
\describe{Clipping to a path} {\cs{gclip}$\ldots$}
\describe{Polar conversion}
        {\cs{plr}\marg{($r_0$,$\theta_0$),($r_1$,$\theta_1$),$\ldots$}}
\describe{Connect paths} {\cs{connect} \meta{path1} \meta{path2} $\ldots$
        \cs{endconnect}}
\describe{Draw many curves from one datafile}
        {\cs{plotdata}\oarg{\meta{spec}}\marg{\meta{file}}, \ \meta{spec} is {\tt
        p} or {\tt s\meta{num}} where \meta{num} is the (optional) tension
        in the smooth curve}
\describe{Set how \cs{plotdata} draws curves$^{\dag}$}{\cs{dashedlines}
        \ (different dash patterns)\brk
        \cs{coloredlines} \ (different colors, \MP{} only)\brk
        \cs{pointedlines} \ (different symbols, like \cs{plot})\brk
        \cs{datapointsonly} \ (different symbols, like \cs{plotnodes})}

\smallskip
* The angle is optional, and ignored unless option \opt{mplabels} is
in effect.\brk
$^{\dag}$ \cs{plotdata} also respects the \cs{using} setting (see
\cs{datafile} in section {\bf General curves}).

\bye

% Other commands:
\cs{setrender}[\<\TeX{} commands\>]

\cs{coords}$\ldots$\cs{endcoords}
\cs{applyT}[\<transformer\>]
\cs{rotate}[$\theta$]
\cs{rotatearound}[$p$][$\theta$]
\cs{turn}[?$p$][$\theta$]
\cs{mirror}[$p_1$][$p_2$]
\cs{reflectabout}[$p_1$][$p_2$]
\cs{shift}[$p$]
\cs{scale}[$s$]
\cs{xscale}[$s$]
\cs{yscale}[$s$]
\cs{zscale}[$p$]
\cs{xslant}[$s$]
\cs{yslant}[$s$]
\cs{zslant}[$s$]
\cs{boost}[$\chi$]
\cs{xyswap}

\cs{fdef}[\<fcn\>]%
  (\<param1\>,\<param2\>,$\ldots$)%
  [\<mf-expr\>]

\cs{tcaption}[?\<maxwd\>,\<linewd\>][\<caption text\>]

\cs{savepic}\bsl\<picname\>
\cs{newsavepic}
\cs{usepic}

\cs{dashlineset}, \cs{dotlineset}

\cs{mfsrc}[\<metafont code\>]
\cs{noship}

\cs{store}[\<path variable\>][\<path\>]
\cs{mfobj}[\<path expression\>]
\cs{patharr}[\<pv\>]$\ldots$\cs{endpatharr}

\cs{mftitle}[\<title\>]
\cs{tmtitle}[\<title\>]

\cs{newfdim}[\<fdim\>]

\cs{setmfpicgraphic}
\cs{getmfpicoffset}
\cs{setfilename}

\bye
