\documentclass[english]{article}
%
\usepackage[T1]{fontenc}
\usepackage[latin9]{inputenc}
\listfiles
\usepackage[scaled]{luximono}
\usepackage{lmodern}
\usepackage{xspace,setspace,caption}
\usepackage[bottom]{footmisc}
\usepackage{tabularx,amsmath,eurosym}
\usepackage{longtable}
\usepackage[NewCommands,NewParameters]{ragged2e}
\usepackage[dvipsnames]{pstricks}
\usepackage{pst-text,pst-grad}
\let\GRADfileversion\fileversion
\let\GRADfiledate\filedate
\usepackage{graphicx,multido,framed}
\definecolor{hellgelb}{rgb}{1,1,0.8}
\makeatletter
\newcount\pst@cntm%
\newcount\pst@cntn%
\newcount\pst@cnto%
\newcount\pst@cntp%
\def\modulo#1#2#3{%
  \begingroup%
  \pst@cntm=#1\pst@cntn=#2\relax%
  \pst@cnto=\pst@cntm%
  \divide\pst@cntm by \pst@cntn%
  \multiply\pst@cntn by \pst@cntm%
  \advance\pst@cnto by -\pst@cntn%
  \edef\value{\endgroup\def\noexpand#3{\number\pst@cnto}}\value%
}
\makeatother
%
\def\PST{{\texttt{PSTricks}}\xspace}
\def\PDF{{\texttt{PDF}}\xspace}
\def\pst{{\texttt{pstricks}}\xspace}
\def\PS{PostScript\xspace}
\newcommand*\CMD[1]{{\UrlFont\texttt{\textbackslash #1}}}
%
\def\tIndex#1{\index{#1@{\UrlFont\texttt{#1}}}}
\def\cIndex#1{\index{#1@\CMD{#1}}}
\def\pIndex#1{\index{Parameter@\textbf{Parameter}!{\UrlFont\texttt{#1}}}}
\def\ppIndex#1{\index{Parameter@\textbf{Parameter}!{#1}}}
\def\sIndex#1{\index{Syntax@\textbf{Syntax}!\CMD{#1}}}
\def\csIndex#1{\sIndex{#1}\cIndex{#1}}
\def\PIndex#1{\index{Paket@\textbf{Paket}!\texttt{#1}}}
\def\mIndex#1{\texttt{#1}\tIndex{#1}\pIndex{#1}}
%
\pretolerance=500
\tolerance=1000 
\hbadness=3000
\vbadness=3000
\hyphenpenalty=400

\usepackage{showexpl}% not a real PSTricks package
\usepackage{babel}
\usepackage{makeidx}
\makeindex
\usepackage[dvips,colorlinks,linktocpage]{hyperref} % PDF-support
%
\renewcommand{\ttdefault}{ul9}% Luxi Mono
\lstset{keywordstyle=\small\ttfamily\bfseries,preset=\raggedright} 
\lstset{language=PSTricks,moredelim=**[is][\bf\color{blue}]{�}{�}}%
%

\begin{document}
%
\title{\texttt{pst-grad}:Gradients \\[10pt]\footnotesize v. \GRADfileversion\ -- \GRADfiledate}
\author{Herbert Vo�\thanks{Thanks to Lars Kotthoff and Angelo Rossi for translating this documentation!}
}
\maketitle

\begin{abstract}
\PIndex{pst-grad}\tIndex{pst-grad}\verb+pst-grad+ 
is also one of the older and smaller
packages. It  provides only one fill style. A
gradient could be created with the macros known from \PST, too, the use of
\verb+pst-grad+ offers advantages though, since one does not need to take care
of the calculation of the intermediate colour values.

This version of \verb+pst-grad+ integrates the function of the
\verb+pst-ghsb+ package, which supports the HSB color model.

\end{abstract}

\tableofcontents

\clearpage

\section{Introduction}


All parameters are only available when \textbf{gradient} is used as
		fill style.
There are further packages which support such fill styles, especially for
circular gradients (\verb+pst-slpe+\PIndex{pst-slpe}).


% ---------------------------------------------------------------------------------------
\section{Parameters}\label{sec:pstgrad:parameter}
% ---------------------------------------------------------------------------------------
Table~\ref{tab:pst-grad:parameter} shows a compilation of the special parameters
valid for \verb+pst-grad+.

\begin{longtable}{>{\ttfamily}l>{\ttfamily}l>{\ttfamily}l}
\caption{Summary of all parameters for \texttt{pst-grad} and
\texttt{pst-ghsb}}\label{tab:pst-grad:parameter}\\[-5pt]
\textrm{name} & \textrm{values}  & \textrm{default}\\\hline
\endfirsthead
\textrm{name} & \textrm{values}  & \textrm{default}\\\hline
\endhead
gradbegin     &  <colour>        & gradbegin\\
gradend       &  <colour>        & gradend\\
gradlines     &  <value>         & 500\\
gradmidpoint  &  <value>         & 0.9\\
gradangle     &  <angle>       & 0\\
gradientHSB   &  false|true     & false\\
GradientCircle&  false|true     & false\\
GradientScale &  <value>         & 1.0\\
GradientPos   &  <(x,y)>        & (0,0)\\
\end{longtable}
\tIndex{gradbegin}\tIndex{gradend}\tIndex{gradlines}\tIndex{gradmidpoint}\tIndex{gradangle}
\pIndex{gradbegin}\pIndex{gradend}\pIndex{gradlines}\pIndex{gradmidpoint}\pIndex{gradangle}
\tIndex{gradientHSB}\pIndex{gradientHSB}%
\tIndex{GradientCircle}\pIndex{GradientCircle}%
\tIndex{GradientPos}\pIndex{GradientPos}%
\tIndex{GradientScale}\pIndex{GradientScale}


% ---------------------------------------------------------------------------------------
\subsection{\texttt{gradbegin}}\label{subsec:pstgrad:gradbegin}
% ---------------------------------------------------------------------------------------
\tIndex{gradbegin}\verb+gradbegin+ denotes the parameter as well as the starting
colour, which is a little bit confusing here.
\begin{verbatim}
\newrgbcolor{gradbegin}{0 .1 .95} % default
\end{verbatim}

Consequently this starting colour can be changed by redefining the colour or by
an assignment through the parameter.
\begin{verbatim}
\newrgbcolor{gradbegin}{0 0 1} 
\definecolor{rgb}{gradbegin}{0 0 1} % requires color/xcolor package
\psset{gradbegin=blue}
\end{verbatim}


\medskip\noindent
\begin{LTXexample}[width=5cm]
\begin{pspicture}(5,3.5)
\psframe[fillstyle=gradient,gradbegin=white](5,1.5)
\newrgbcolor{gradbegin}{0 1 1}
\psframe[fillstyle=gradient](0,2)(5,3.5)
\end{pspicture}
\end{LTXexample}


\medskip
%\begin{shaded}
\begin{itemize}
	\item \verb+gradbegin+ should be defined as \verb+RGB+ colour\tIndex{RGB},
		since a faultless function for \verb+CMYK+ or gray scales is not
		warranted in every case.
\item Con\TeX{}t users change the colour with\\
    \verb+\definecolor{rgb}{gradbegin}{r=0,g=0,b=1}+
\end{itemize}
%\end{shaded}



% ---------------------------------------------------------------------------------------
\subsection{\texttt{gradend}}\label{subsec:pstgrad:gradend}
% ---------------------------------------------------------------------------------------
\tIndex{gradend}\verb+gradend+ is \textbf{not} the counterpart to
\verb+gradbegin+, for it is the colour which is reached at the relative point
\verb+gridmidpoint+. In every case it is ambiguous as \verb+gradbegin+ again.
\begin{verbatim}
\newrgbcolor{gradend}{0 1 1} % default
\end{verbatim}


Changes can be made differently again.
\begin{verbatim}
\newrgbcolor{gradend}{1 0 0} 
\definecolor{rgb}{gradend}{1 0 0} % requires color/xcolor package
\psset{gradend=red}
\end{verbatim}


\medskip\noindent
\begin{LTXexample}[width=5cm]
\begin{pspicture}(5,3.5)
\psframe[fillstyle=gradient,gradend=white](5,1.5)
\newrgbcolor{gradend}{1 0 0}
\psframe[fillstyle=gradient](0,2)(5,3.5)
\end{pspicture}
\end{LTXexample}


\medskip
%\begin{shaded}
\begin{itemize}
	\item \verb+gradend+ should be defined as \verb+RGB+ colour\tIndex{RGB},
		since a faultless function for \verb+CMYK+ or gray scales is not
		warranted in every case.
\item Con\TeX{}t users change the colour with\\
\verb+\definecolor{rgb}{gradend}{r=1,g=1,b=0}+
\end{itemize}
%\end{shaded}



% ---------------------------------------------------------------------------------------
\subsection{\texttt{gradlines}}\label{subsec:pstgrad:gradlines}
% ---------------------------------------------------------------------------------------
\tIndex{gradlines}A gradient is nothing but a string of coloured lines. The
width of those depends only on the resolution of the monitor resp. the printer
in the end. But since this is very user-specific, \verb+pst-grad+ allows any
number of lines, which can be changed through \verb+gradlines+.



\medskip\noindent
\begin{LTXexample}[width=5cm]
\begin{pspicture}(5,4)
\psset{fillstyle=gradient,linestyle=none}
\psframe[gradlines=5](5,1)
\psframe(0,1.5)(5,2.5)
\psframe[gradlines=1000](0,3)(5,4)
\end{pspicture}
\end{LTXexample}




% ---------------------------------------------------------------------------------------
\subsection{\texttt{gradmidpoint}}\label{subsec:pstgrad:gradmidpoint}
% ---------------------------------------------------------------------------------------
\tIndex{gradmidpoint}Denotes the relative point where the colour \verb+gradend+
is reached. Then it is proceeded in reverse order.


\medskip\noindent
\begin{LTXexample}[width=5cm]
\begin{pspicture}(5,4)
\psset{fillstyle=gradient,linestyle=none}
\psframe[gradmidpoint=0](5,1)
\psframe[gradmidpoint=0.5](0,1.5)(5,2.5)
\psframe[gradmidpoint=1](0,3)(5,4)
\end{pspicture}
\end{LTXexample}



% ---------------------------------------------------------------------------------------
\subsection{\texttt{gradangle}}\label{subsec:pstgrad:gradangle}
% ---------------------------------------------------------------------------------------
\tIndex{gradangle}\verb+gradangle+ determines the gradient angle of the straight
line.

\medskip\noindent
\begin{LTXexample}[width=5cm]
\begin{pspicture}(5,4)
\psset{fillstyle=gradient,linestyle=none,gradmidpoint=0.5}
\psframe[gradangle=0](5,1)
\psframe[gradangle=45](0,1.5)(5,2.5)
\psframe[gradangle=90](0,3)(5,4)
\end{pspicture}
\end{LTXexample}

% ---------------------------------------------------------------------------------------
\subsection{\texttt{GradientCircle}, \texttt{GradientScale} and \texttt{GradientPos}   }\label{subsec:pstgrad:GradientCircle}
% ---------------------------------------------------------------------------------------
\tIndex{GradientCircle}\tIndex{GradientScale}\tIndex{GradientPos}With the option
\verb+GradientCircle+ circular gradients can be created. The radius can be
influenced through \verb+GradientScale+ and the centre with \verb+GradientPos+.
The specification of the coordinates refers to the based coordinate system,
which is given by the \verb+pspicture+ environment as a rule.

\medskip\noindent
\begin{LTXexample}[width=5cm]
\begin{pspicture}(5,4)
\psset{fillstyle=gradient,linestyle=none}
\psframe[GradientCircle=true](5,1)%
\psframe[GradientCircle=true,GradientScale=3](0,1.5)(5,2.5)%
\psframe[GradientCircle=true,GradientScale=2,%
         GradientPos={(4,3.5)}](0,3)(5,4)%
\end{pspicture}
\end{LTXexample}

\vspace{1cm}
\begin{center}
\bgroup
\DeclareFixedFont{\RM}{T1}{ptm}{b}{n}{3cm}
\DeclareFixedFont{\Rm}{T1}{ptm}{m}{n}{2mm}

\psset{fillstyle=gradient,gradbegin=red,gradend=white}
\begin{pspicture}(\linewidth,3cm)
\resizebox{\linewidth}{!}{\pscharpath[gradangle=90]{\RM PostScript}}
\end{pspicture}\\
\begin{pspicture}(\linewidth,3cm)
\resizebox{\linewidth}{!}{\pscharpath[gradangle=0]{\RM PostScript}}
\end{pspicture}\\
\begin{pspicture}(\linewidth,3cm)
\resizebox{\linewidth}{!}{\pscharpath[gradmidpoint=0,gradangle=90]{\RM PostScript}}
\end{pspicture}
\egroup
\captionof{figure}{Shadow games\ldots}
\end{center}


\clearpage
\subsection{\texttt{GradientHSB}}\label{subsec:pstgrad:gradientHSB}

\medskip\noindent
\begin{LTXexample}[width=5.5cm]
\newcommand{\Fig}[1][]{%
\begin{pspicture}(5.5,5.5)
  \psframe[#1](5,5)
\end{pspicture}}
\newhsbcolor{ColorA}{0 0 0.7}
\newhsbcolor{ColorB}{0 1 0.7}
\newhsbcolor{ColorC}{.5 0.8 0}
\newhsbcolor{ColorD}{.5 0.8 1}
\psset{fillstyle=gradient,gradientHSB=true}
\Fig[gradmidpoint=1,gradbegin=ColorA,gradend=ColorB]
\Fig[gradmidpoint=0.5,gradbegin=ColorC,gradend=ColorD]
\end{LTXexample}


\newcommand{\Fig}[1][]{%
\begin{pspicture}(5.5,5.5)
  \psframe[#1](5,5)
\end{pspicture}}
\definecolor{ColorA}{hsb}{0.7, 0.1, 0.8}
\definecolor{ColorB}{hsb}{0.7, 0.9, 0.8}
\definecolor{ColorC}{hsb}{0, 0, 0}
\definecolor{ColorD}{hsb}{0, 0, 1}
\definecolor{ColorE}{hsb}{0, 0, 0.5}
\definecolor{ColorF}{hsb}{0, 1, 0.5}
\definecolor{ColorG}{hsb}{0, 0, 0.5}
\definecolor{ColorH}{hsb}{0.99999, 0, 0.5}   % As it's cycli� 1=0 !
\definecolor{ColorI}{hsb}{1, 1, 1}
\definecolor{ColorJ}{hsb}{1, 0, 0}
\definecolor{ColorK}{hsb}{0.99999, 1, 1}     % As it's cyclic 1=0 !
\definecolor{ColorL}{hsb}{0, 1, 0}
\definecolor{ColorM}{hsb}{0.99999, 1, 1}     % As it's cyclic 1=0 !
\definecolor{ColorN}{hsb}{0, 0, 1}
\definecolor{ColorO}{hsb}{0, 0.6, 0.7}
\definecolor{ColorP}{hsb}{0.99999, 0.7, 0.7} % As it's cyclic 1=0 !
\definecolor{ColorQ}{hsb}{0.3, 0, 0.8}
\definecolor{ColorR}{hsb}{0.3, 1, 0.8}
\definecolor{ColorS}{hsb}{0.6, 0.3, 0}
\definecolor{ColorT}{hsb}{0.6, 0.3, 1}
\psset{fillstyle=gradient,gradmidpoint=1}
\Fig[gradbegin=yellow,gradend=green]
\Fig[gradientHSB=true,gradbegin=ColorA,gradend=ColorB]

\Fig[gradbegin=green,gradend=yellow]
\psset{gradientHSB=true}
\Fig[gradbegin=ColorC,gradend=ColorD]

\Fig[gradbegin=ColorE,gradend=ColorF]
\Fig[gradbegin=ColorG,gradend=ColorH]

\Fig[gradbegin=ColorI,gradend=ColorJ]
\Fig[gradbegin=ColorK,gradend=ColorL]

\Fig[gradbegin=ColorM,gradend=ColorN]
\Fig[gradbegin=ColorO,gradend=ColorP]

\Fig[gradbegin=ColorQ,gradend=ColorR]
\Fig[gradbegin=ColorS,gradend=ColorT]

\begin{lstlisting}
\definecolor{ColorA}{hsb}{0.7, 0.1, 0.8}
\definecolor{ColorB}{hsb}{0.7, 0.9, 0.8}
\definecolor{ColorC}{hsb}{0, 0, 0}
\definecolor{ColorD}{hsb}{0, 0, 1}
\definecolor{ColorE}{hsb}{0, 0, 0.5}
\definecolor{ColorF}{hsb}{0, 1, 0.5}
\definecolor{ColorG}{hsb}{0, 0, 0.5}
\definecolor{ColorH}{hsb}{0.99999, 0, 0.5}   % As it's cycli� 1=0 !
\definecolor{ColorI}{hsb}{1, 1, 1}
\definecolor{ColorJ}{hsb}{1, 0, 0}
\definecolor{ColorK}{hsb}{0.99999, 1, 1}     % As it's cyclic 1=0 !
\definecolor{ColorL}{hsb}{0, 1, 0}
\definecolor{ColorM}{hsb}{0.99999, 1, 1}     % As it's cyclic 1=0 !
\definecolor{ColorN}{hsb}{0, 0, 1}
\definecolor{ColorO}{hsb}{0, 0.6, 0.7}
\definecolor{ColorP}{hsb}{0.99999, 0.7, 0.7} % As it's cyclic 1=0 !
\definecolor{ColorQ}{hsb}{0.3, 0, 0.8}
\definecolor{ColorR}{hsb}{0.3, 1, 0.8}
\definecolor{ColorS}{hsb}{0.6, 0.3, 0}
\definecolor{ColorT}{hsb}{0.6, 0.3, 1}
\psset{fillstyle=gradient,gradmidpoint=1}
\Fig[gradbegin=yellow,gradend=green]
\Fig[gradientHSB=true,gradbegin=ColorA,gradend=ColorB]

\Fig[gradbegin=green,gradend=yellow]
\psset{gradientHSB=true}
\Fig[gradbegin=ColorC,gradend=ColorD]

\Fig[gradbegin=ColorE,gradend=ColorF]
\Fig[gradbegin=ColorG,gradend=ColorH]

\Fig[gradbegin=ColorI,gradend=ColorJ]
\Fig[gradbegin=ColorK,gradend=ColorL]

\Fig[gradbegin=ColorM,gradend=ColorN]
\Fig[gradbegin=ColorO,gradend=ColorP]

\Fig[gradbegin=ColorQ,gradend=ColorR]
\Fig[gradbegin=ColorS,gradend=ColorT]
\end{lstlisting}


\nocite{*}
\bgroup
\raggedright
\bibliographystyle{plain}
\bibliography{\jobname}
\egroup

\printindex


\end{document}
