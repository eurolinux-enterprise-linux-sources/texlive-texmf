\documentclass[a4paper,12pt]{article}
\usepackage[
 a4paper,
 nohead,
 nofoot,
 top=2cm,
 bottom=2cm,
 left=2cm,
 right=2cm,
 pdftex]{geometry}
% \usepackage{array}
\usepackage{fancyvrb}
\usepackage{url}
\usepackage{charter}
\usepackage{multicol}
\usepackage[colorlinks,bookmarks=false]{hyperref}
\usepackage[vietnam,english]{babel}

% \pagestyle{empty}
\def\arraystretch{1.3}
\DefineShortVerb{\|}
\parskip.5\baselineskip
\parindent0pt
\raggedbottom

\input{abbr.tex}
\begin{document}

\title{\bfseries Minimal steps to typeset Vietnamese}
\author{\fontencoding{T5}\selectfont H\`an Th\'\ecircumflex{} Th\`anh}
\maketitle

This document tries to answer the question that has been often asked:
\textit{How can I typeset \textbf{just a few Vietnamese words} in my
document, which is in English (or French/German/...)?}

The answer depends very much on a particular scenario, however I
assume that you are in a hurry, you don't want to bother with issues like
how to display and write Vietnamese in your \<TEX>  editor. You only have a few
Vietnamese words in your \<LATEX> file and you would like to see them
properly displayed in your final \<PDF> or \<PS> file.

\begin{enumerate}
\item As the very first requirement, you must have some minimal \<LATEX>
support for Vietnamese:
\begin{itemize}
\item Check whether you have \<VNTEX> installed. \<VNTEX> is included
in \<TETEX>, \<MIKTEX> and \<TEXLIVE>. 

\item If the above is not the case, try to download and install \<VNTEX> by
following the instructions at \url{http://vntex.org/download/vntex}.

\item If you can't install \<VNTEX>, you must have at least Latin Modern
fonts installed. Then download
\url{http://vntex.org/download/vntex-support/t5enc.def} and put it to the
directory where your \<LATEX> file is.

\item If all the above fails, try to get help from someone else to solve
at least one of those issues.
\end{itemize}

\item  Make sure you have package |fontenc| loaded with T5 encoding. For
example, if your document contains European languague(s) only, then you
should have a line saying

\begin{verbatim}
\usepackage[T1,T5]{fontenc}
\end{verbatim}

in your preamble.

\item An example how to input Vietnamese words may look like this:

\begin{verbatim}
{\fontencoding{T5}\selectfont Ti\'\ecircumflex{}ng Vi\d\ecircumflex{}t}
\end{verbatim}

which gives the output as \texttt{\fontencoding{T5}\selectfont
Ti\'\ecircumflex{}ng Vi\d\ecircumflex{}t}.

\item The following table contains all Vietnamese letters for your
reference:

{\fontencoding{T5}\selectfont
% Copyright 2003-2005 Han The Thanh <hanthethanh@gmx.net>.
% This file is part of vntex.  License: LPPL, version 1.3 or newer,
% according to http://www.latex-project.org/lppl.txt

\large
\def\X#1{\hbox to 2em{\hss#1\hss}}
\begin{multicols}{2}
\noindent
\X{\ABREVE} \verb|\ABREVE| \\
\X{\Abreve} \verb|\Abreve| \\
\X{\ACIRCUMFLEX} \verb|\ACIRCUMFLEX| \\
\X{\Acircumflex} \verb|\Acircumflex| \\
\X{\ECIRCUMFLEX} \verb|\ECIRCUMFLEX| \\
\X{\Ecircumflex} \verb|\Ecircumflex| \\
\X{\Ocircumflex} \verb|\Ocircumflex| \\
\X{\OCIRCUMFLEX} \verb|\OCIRCUMFLEX| \\
\X{\OHORN} \verb|\OHORN| \\
\X{\Ohorn} \verb|\Ohorn| \\
\X{\UHORN} \verb|\UHORN| \\
\X{\Uhorn} \verb|\Uhorn| \\
\X{\abreve} \verb|\abreve| \\
\X{\acircumflex} \verb|\acircumflex| \\
\X{\dj} \verb|\dj| \\
\X{\ecircumflex} \verb|\ecircumflex| \\
\X{\i} \verb|\i| \\
\X{\ocircumflex} \verb|\ocircumflex| \\
\X{\ohorn} \verb|\ohorn| \\
\X{\uhorn} \verb|\uhorn| \\
\X{\'A} \verb|\'A| \\
\X{\'E} \verb|\'E| \\
\X{\'I} \verb|\'I| \\
\X{\'O} \verb|\'O| \\
\X{\'U} \verb|\'U| \\
\X{\'Y} \verb|\'Y| \\
\X{\'\ABREVE} \verb|\'\ABREVE| \\
\X{\'\Abreve} \verb|\'\Abreve| \\
\X{\'\ACIRCUMFLEX} \verb|\'\ACIRCUMFLEX| \\
\X{\'\Acircumflex} \verb|\'\Acircumflex| \\
\X{\'\ECIRCUMFLEX} \verb|\'\ECIRCUMFLEX| \\
\X{\'\Ecircumflex} \verb|\'\Ecircumflex| \\
\X{\'\OCIRCUMFLEX} \verb|\'\OCIRCUMFLEX| \\
\X{\'\Ocircumflex} \verb|\'\Ocircumflex| \\
\X{\'\OHORN} \verb|\'\OHORN| \\
\X{\'\Ohorn} \verb|\'\Ohorn| \\
\X{\'\UHORN} \verb|\'\UHORN| \\
\X{\'\Uhorn} \verb|\'\Uhorn| \\
\X{\'\abreve} \verb|\'\abreve| \\
\X{\'\acircumflex} \verb|\'\acircumflex| \\
\X{\'\ecircumflex} \verb|\'\ecircumflex| \\
\X{\'\ocircumflex} \verb|\'\ocircumflex| \\
\X{\'\ohorn} \verb|\'\ohorn| \\
\X{\'\uhorn} \verb|\'\uhorn| \\
\X{\'a} \verb|\'a| \\
\X{\'e} \verb|\'e| \\
\X{\'i} \verb|\'i| \\
\X{\'o} \verb|\'o| \\
\X{\'u} \verb|\'u| \\
\X{\'y} \verb|\'y| \\
\X{\^A} \verb|\^A| \\
\X{\^E} \verb|\^E| \\
\X{\^O} \verb|\^O| \\
\X{\^a} \verb|\^a| \\
\X{\^e} \verb|\^e| \\
\X{\^o} \verb|\^o| \\
\X{\`A} \verb|\`A| \\
\X{\`E} \verb|\`E| \\
\X{\`I} \verb|\`I| \\
\X{\`O} \verb|\`O| \\
\X{\`U} \verb|\`U| \\
\X{\`Y} \verb|\`Y| \\
\X{\`\ABREVE} \verb|\`\ABREVE| \\
\X{\`\Abreve} \verb|\`\Abreve| \\
\X{\`\ACIRCUMFLEX} \verb|\`\ACIRCUMFLEX| \\
\X{\`\Acircumflex} \verb|\`\Acircumflex| \\
\X{\`\ECIRCUMFLEX} \verb|\`\ECIRCUMFLEX| \\
\X{\`\Ecircumflex} \verb|\`\Ecircumflex| \\
\X{\`\OCIRCUMFLEX} \verb|\`\OCIRCUMFLEX| \\
\X{\`\Ocircumflex} \verb|\`\Ocircumflex| \\
\X{\`\OHORN} \verb|\`\OHORN| \\
\X{\`\Ohorn} \verb|\`\Ohorn| \\
\X{\`\UHORN} \verb|\`\UHORN| \\
\X{\`\Uhorn} \verb|\`\Uhorn| \\
\X{\`\abreve} \verb|\`\abreve| \\
\X{\`\acircumflex} \verb|\`\acircumflex| \\
\X{\`\ecircumflex} \verb|\`\ecircumflex| \\
\X{\`\ocircumflex} \verb|\`\ocircumflex| \\
\X{\`\ohorn} \verb|\`\ohorn| \\
\X{\`\uhorn} \verb|\`\uhorn| \\
\X{\`a} \verb|\`a| \\
\X{\`e} \verb|\`e| \\
\X{\`i} \verb|\`i| \\
\X{\`o} \verb|\`o| \\
\X{\`u} \verb|\`u| \\
\X{\`y} \verb|\`y| \\
\X{\d A} \verb|\d A| \\
\X{\d E} \verb|\d E| \\
\X{\d I} \verb|\d I| \\
\X{\d O} \verb|\d O| \\
\X{\d U} \verb|\d U| \\
\X{\d Y} \verb|\d Y| \\
\X{\d \ABREVE} \verb|\d \ABREVE| \\
\X{\d \Abreve} \verb|\d \Abreve| \\
\X{\d \ACIRCUMFLEX} \verb|\d \ACIRCUMFLEX| \\
\X{\d \Acircumflex} \verb|\d \Acircumflex| \\
\X{\d \ECIRCUMFLEX} \verb|\d \ECIRCUMFLEX| \\
\X{\d \Ecircumflex} \verb|\d \Ecircumflex| \\
\X{\d \OCIRCUMFLEX} \verb|\d \OCIRCUMFLEX| \\
\X{\d \Ocircumflex} \verb|\d \Ocircumflex| \\
\X{\d \OHORN} \verb|\d \OHORN| \\
\X{\d \Ohorn} \verb|\d \Ohorn| \\
\X{\d \UHORN} \verb|\d \UHORN| \\
\X{\d \Uhorn} \verb|\d \Uhorn| \\
\X{\d \abreve} \verb|\d \abreve| \\
\X{\d \acircumflex} \verb|\d \acircumflex| \\
\X{\d \ecircumflex} \verb|\d \ecircumflex| \\
\X{\d \ocircumflex} \verb|\d \ocircumflex| \\
\X{\d \ohorn} \verb|\d \ohorn| \\
\X{\d \uhorn} \verb|\d \uhorn| \\
\X{\d a} \verb|\d a| \\
\X{\d e} \verb|\d e| \\
\X{\d i} \verb|\d i| \\
\X{\d o} \verb|\d o| \\
\X{\d u} \verb|\d u| \\
\X{\d y} \verb|\d y| \\
\X{\h A} \verb|\h A| \\
\X{\h E} \verb|\h E| \\
\X{\h I} \verb|\h I| \\
\X{\h O} \verb|\h O| \\
\X{\h U} \verb|\h U| \\
\X{\h Y} \verb|\h Y| \\
\X{\h \ABREVE} \verb|\h \ABREVE| \\
\X{\h \Abreve} \verb|\h \Abreve| \\
\X{\h \ACIRCUMFLEX} \verb|\h \ACIRCUMFLEX| \\
\X{\h \Acircumflex} \verb|\h \Acircumflex| \\
\X{\h \ECIRCUMFLEX} \verb|\h \ECIRCUMFLEX| \\
\X{\h \Ecircumflex} \verb|\h \Ecircumflex| \\
\X{\h \OCIRCUMFLEX} \verb|\h \OCIRCUMFLEX| \\
\X{\h \Ocircumflex} \verb|\h \Ocircumflex| \\
\X{\h \OHORN} \verb|\h \OHORN| \\
\X{\h \Ohorn} \verb|\h \Ohorn| \\
\X{\h \UHORN} \verb|\h \UHORN| \\
\X{\h \Uhorn} \verb|\h \Uhorn| \\
\X{\h \abreve} \verb|\h \abreve| \\
\X{\h \acircumflex} \verb|\h \acircumflex| \\
\X{\h \ecircumflex} \verb|\h \ecircumflex| \\
\X{\h \ocircumflex} \verb|\h \ocircumflex| \\
\X{\h \ohorn} \verb|\h \ohorn| \\
\X{\h \uhorn} \verb|\h \uhorn| \\
\X{\h a} \verb|\h a| \\
\X{\h e} \verb|\h e| \\
\X{\h i} \verb|\h i| \\
\X{\h o} \verb|\h o| \\
\X{\h u} \verb|\h u| \\
\X{\h y} \verb|\h y| \\
\X{\~A} \verb|\~A| \\
\X{\~E} \verb|\~E| \\
\X{\~I} \verb|\~I| \\
\X{\~O} \verb|\~O| \\
\X{\~U} \verb|\~U| \\
\X{\~Y} \verb|\~Y| \\
\X{\~\ABREVE} \verb|\~\ABREVE| \\
\X{\~\Abreve} \verb|\~\Abreve| \\
\X{\~\ACIRCUMFLEX} \verb|\~\ACIRCUMFLEX| \\
\X{\~\Acircumflex} \verb|\~\Acircumflex| \\
\X{\~\ECIRCUMFLEX} \verb|\~\ECIRCUMFLEX| \\
\X{\~\Ecircumflex} \verb|\~\Ecircumflex| \\
\X{\~\OCIRCUMFLEX} \verb|\~\OCIRCUMFLEX| \\
\X{\~\Ocircumflex} \verb|\~\Ocircumflex| \\
\X{\~\OHORN} \verb|\~\OHORN| \\
\X{\~\Ohorn} \verb|\~\Ohorn| \\
\X{\~\UHORN} \verb|\~\UHORN| \\
\X{\~\Uhorn} \verb|\~\Uhorn| \\
\X{\~\abreve} \verb|\~\abreve| \\
\X{\~\acircumflex} \verb|\~\acircumflex| \\
\X{\~\ecircumflex} \verb|\~\ecircumflex| \\
\X{\~\ocircumflex} \verb|\~\ocircumflex| \\
\X{\~\ohorn} \verb|\~\ohorn| \\
\X{\~\uhorn} \verb|\~\uhorn| \\
\X{\~a} \verb|\~a| \\
\X{\~e} \verb|\~e| \\
\X{\~i} \verb|\~i| \\
\X{\~o} \verb|\~o| \\
\X{\~u} \verb|\~u| \\
\X{\~y} \verb|\~y| \\
\end{multicols}

}

\item If you have quite a lot of Vietnamese words, then it can be somewhat
tedious to translate them to the above form (often called as \<LATEX>
Internal Character Representation -- LICR). On \<WINDOWS> you can use the
package \url{http://vntex.org/download/vntex-support/tovntex.zip}
to translate text in clipboard from VIQR or UTF-8 to LICR by one key press.

The same (or close) convenience could be made for \<UNIX>/\<LINUX>
users, but at somewhat higher cost due to deficiencies of \<UNIX>-like
systems. So if you don't use \<WINDOWS> then you are out of luck, sorry.
However, if you use \<VIM>, you can still download the package mentioned
above, and use the vim script inside the zip archive to do the conversion.
If you want to make this easier for \<UNIX> users then let me know.

\item If you still have questions, join the \<VNTEX> mailing list
at \url{https://lists.sourceforge.net/lists/listinfo/vntex-users}.

\end{enumerate}

Good luck!

% \DefineShortVerb{\|}
\end{document}
