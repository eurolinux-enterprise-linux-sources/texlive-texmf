\documentclass{article}

\usepackage{multicol}
\usepackage[a4paper,margin=1.5cm]{geometry}
\usepackage{fancyhdr}
\usepackage[baw,pstricks]{fvrb-ex}
\usepackage{pst-circ}
\usepackage{pstcol}
\usepackage{url}
\makeatletter
\def\@UrlFont{\small\ttfamily}
\renewenvironment{description}
  {\list{}{\labelwidth\z@ \itemindent-\leftmargin
    \itemsep0pt \parsep0pt
    \let\makelabel\descriptionlabel}}
  {\endlist}

\renewcommand{\Begin@Example}{%
\parindent=0pt
\multiply\topsep by 2 \VerbatimEnvironment
\begin{VerbatimOut}[codes={\catcode`\�=12\catcode`\/=12\catcode`\&=12%
                           \catcode`\"=12}]{\jobname.tmp}}

\renewcommand{\Below@Example}[1]{%
\VerbatimInput[gobble=0,commentchar=�,commandchars=/&",numbersep=3pt]%
              {\jobname.tmp}
\catcode`\�=9\relax%
\NoHighlight@Attributes % To suppress possible highlighting
\ifFvrbEx@Grid\vspace{5pt}\fi
#1%
\ifFvrbEx@Grid\vspace{5pt}\fi
\par}

\renewcommand{\SideBySide@Example}[1]{%
\@tempdimb=\FV@XRightMargin \advance\@tempdimb -5mm
\begin{minipage}[c]{\@tempdimb}
  \fvset{xrightmargin=0pt}
  \catcode`\�=9\relax%
  \NoHighlight@Attributes % To suppress possible highlighting
  #1
\end{minipage}%
\@tempdimb=\textwidth \advance\@tempdimb -\FV@XRightMargin \advance\@tempdimb 5mm
\begin{minipage}[c]{\@tempdimb}
  \VerbatimInput[gobble=0,commentchar=�,commandchars=/&",numbersep=3pt,
                 xleftmargin=5mm,xrightmargin=0pt]{\jobname.tmp}
\end{minipage}}
% The \NoHighlight@Attributes from `hbaw' and `hcolors' packages
% must be modified too
\def\NoHighlight@Attributes{%
\catcode`\/=0\relax%
\catcode`\&=1\relax%
\catcode`\"=2\relax%
\def\HLa##1{##1}%
\def\HLb##1{##1}%
\def\HLc##1{##1}%
\def\HLd##1{##1}%
\def\HLe##1{##1}%
\def\HLf##1{##1}%
\def\HLBFa##1{##1}%
\def\HLBFb##1{##1}%
\def\HLBFc##1{##1}%
\def\HLBFd##1{##1}%
\def\HLBFe##1{##1}%
\def\HLBFf##1{##1}%
\def\HLITa##1{##1}%
\def\HLITb##1{##1}%
\def\HLITc##1{##1}%
\def\HLITd##1{##1}%
\def\HLITe##1{##1}%
\def\HLITf##1{##1}%
\def\HLCBBa##1{##1}%
\def\HLCBBb##1{##1}%
\def\HLCBBc##1{##1}%
\def\HLCBBd##1{##1}%
\def\HLCBBe##1{##1}%
\def\HLCBBf##1{##1}%
\def\HLCBBz##1{##1}%
\def\HLCBWa##1{##1}%
\def\HLCBWb##1{##1}%
\def\HLCBWc##1{##1}%
\def\HLCBWd##1{##1}%
\def\HLCBWe##1{##1}%
\def\HLCBWf##1{##1}%
\def\HLCBWz##1{##1}%
}

\makeatother

\fvset{numbers=none,frame=single,labelposition=topline}

\DeclareRobustCommand\cs[1]{\texttt{\char`\\#1}}

\newcommand{\CircPackage}{\textbf{`pst-circ'}}

\lhead{\CircPackage}\rhead{A PSTricks package for drawing electric circuits}
\pagestyle{fancy}

\psset{subgriddiv=0,griddots=5,gridlabels=7pt}%
\showgrid
\usepackage[colorlinks,linktocpage]{hyperref}

\begin{document}

\title{\CircPackage\\ A PSTricks package for drawing electric circuits\\\normalsize ver. 1.2b}
\author{{Christophe \textsc{Jorssen} \url{<christophe.jorssen@noos.fr>}}\\
Herbert \textsc{Voss} \url{<voss@perce.de>}}
\date{28 August 2003}
\maketitle

\begin{center}
\psframebox[framearc=0.3,framesep=5mm,linewidth=.7mm]{%
\parbox{15cm}{%
{\textbf{Abstract:} \CircPackage{} is a PSTricks package to draw easily electric circuits. Most
dipoles, tripoles and quadrupoles used in classical electrotechnical
circuits are provided as graphical units which can readily be
interconnectedd to produce circuit diagrams of a reasonable level of complexity}}}
\end{center}

\setlength{\columnseprule}{0.6pt}
\begin{multicols}{2}
{\parskip 0pt \tableofcontents}
\end{multicols}

\section{Introduction}

The package \CircPackage{} is a collection of graphical elements based
on PStricks that can be used to facilitate display of electronic
circuit elements. For example, an equivalent circuit of a voltage
source, its source impedance, and a connected load can easily be
constructed along with arrows indicating current flow and potential
differences. The emphasis is upon the circuit elements and the
details of the exact placement are hidden as much as possible so the
author can focus on the circuitry without the distraction of sorting
out the underlying vector graphics.

\section{Usage}

\subsection{Parameters}

There are specific paramaters defined to change easily the behaviour of the pst-circ
objects you are drawing.

%\begin{multicols}{2}
\begin{description}
\item[\HLTTc{intensity} (boolean):] (\emph{default:~false})
\item[\HLTTc{intensitylabel} (string):] (\emph{default:~\texttt{\cs{empty}}})
\item[\HLTTc{intensitylabeloffset} (dimension):] (\emph{default:~0.5})
\item[\HLTTc{intensitycolor} (PSTricks color):] (\emph{default:~black})
\item[\HLTTc{intensitylabelcolor} (PSTricks color):] (\emph{default:~black})
\item[\HLTTc{intensitywidth} (dimension):] (\emph{default:~\texttt{\cs{pslinewidth}}})
\item[\HLTTc{tension} (boolean):] (\emph{default:~false})
\item[\HLTTc{tensionlabel} (string):] (\emph{default:~\texttt{\cs{empty}}})
\item[\HLTTc{tensionoffset} (dimension):] (\emph{default:~1})
\item[\HLTTc{tensionlabeloffset} (dimension):] (\emph{default:~1.2})
\item[\HLTTc{tensioncolor} (PSTricks color):] (\emph{default:~black})
\item[\HLTTc{tensionlabelcolor} (PSTricks color):] (\emph{default:~black})
\item[\HLTTc{tensionwidth} (dimension):] (\emph{default:~\texttt{\cs{pslinewidth}}})
\item[\HLTTc{labeloffset} (dimension):] (\emph{default:~0.7})
\item[\HLTTc{labelangle} (PSTricks label angle):] (\emph{default:~0})
\item[\HLTTc{dipoleconvention}:] (\emph{default:~receptor})
\item[\HLTTc{directconvetion} (boolean):] (\emph{default:~true})
\item[\HLTTc{dipolestyle} (string):] (\emph{default:~normal})
\item[\HLTTc{variable} (boolean):] (\emph{default:~false})
\item[\HLTTc{parallel} (boolean):] (\emph{default:~false})
\item[\HLTTc{parallelarm} (dimension):] (\emph{default:~1.5})
\item[\HLTTc{parallelsep} (real):] (\emph{default:~0})
\item[\HLTTc{parallelnode} (boolean):] (\emph{default:~false})
\item[\HLTTc{intersect} (boolean):] (\emph{default:~false})
\item[\HLTTc{intersectA} (node):]
\item[\HLTTc{intersectB} (node):]
\item[\HLTTc{OAinvert} (boolean):] (\emph{default:~true})
\item[\HLTTc{OAperfect} (boolean):] (\emph{default:~true})
\item[\HLTTc{OAiplus} (boolean):] (\emph{default:~false})
\item[\HLTTc{OAiminus} (boolean):] (\emph{default:~false})
\item[\HLTTc{OAiout} (boolean):] (\emph{default:~false})
\item[\HLTTc{OAipluslabel} (string):] (\emph{default:~\texttt{\cs{empty}}})
\item[\HLTTc{OAiminuslabel} (string):] (\emph{default:~\texttt{\cs{empty}}})
\item[\HLTTc{OAioutlabel} (string):] (\emph{default:~\texttt{\cs{empty}}})
\item[\HLTTc{transistorcircle} (boolean):] (\emph{default:~true})
\item[\HLTTc{transistorinvert} (boolean):] (\emph{default:~false})
\item[\HLTTc{transistoribase} (boolean):] (\emph{default:~false})
\item[\HLTTc{transistoricollector} (boolean):] (\emph{default:~false})
\item[\HLTTc{transistoriemitter} (boolean):] (\emph{default:~false})
\item[\HLTTc{transistoribaselabel} (string):] (\emph{default:~\texttt{\cs{empty}}})
\item[\HLTTc{transistoricollectorlabel} (string):] (\emph{default:~\texttt{\cs{empty}}})
\item[\HLTTc{transistoriemitterlabel} (string):] (\emph{default:~\texttt{\cs{empty}}})
\item[\HLTTc{transistortype} (string):] (\emph{default:~PNP})
\item[\HLTTc{primarylabel} (string):] (\emph{default:~\texttt{\cs{empty}}})
\item[\HLTTc{secondarylabel} (string):] (\emph{default:~\texttt{\cs{empty}}})
\item[\HLTTc{transformeriprimary} (boolean):] (\emph{default:~false})
\item[\HLTTc{transformerisecondary} (boolean):] (\emph{default:~false})
\item[\HLTTc{transformeriprimarylabel} (string):] (\emph{default:~\texttt{\cs{empty}}})
\item[\HLTTc{transformerisecondarylabel} (string):] (\emph{default:~\texttt{\cs{empty}}})
\item[\HLTTc{tripolestyle} (string):] (\emph{default:~normal})
\end{description}
%\end{multicols}

\section{Macros}

\subsection{Dipole macros}

\begin{PSideBySideExample}[xrightmargin=5.5cm](3,2)
  \pnode(0,1){A}
  \pnode(3,1){B}
  /HLCBWz&\resistor"(A)(B){$R$}
\end{PSideBySideExample}

\begin{PSideBySideExample}[xrightmargin=5.5cm](3,2)
  \pnode(0,1){A}
  \pnode(3,1){B}
  /HLCBWz&\capacitor"(A)(B){$C$}
\end{PSideBySideExample}

\begin{PSideBySideExample}[xrightmargin=5.5cm](3,2)
  \pnode(0,1){A}
  \pnode(3,1){B}
  /HLCBWz&\battery"(A)(B){$E$}
\end{PSideBySideExample}

\begin{PSideBySideExample}[xrightmargin=5.5cm](3,2)
  \pnode(0,1){A}
  \pnode(3,1){B}
  /HLCBWz&\coil"(A)(B){$L$}
\end{PSideBySideExample}

\begin{PSideBySideExample}[xrightmargin=5.5cm](3,2)
  \pnode(0,1){A}
  \pnode(3,1){B}
  /HLCBWz&\Ucc"(A)(B){$E$}
\end{PSideBySideExample}

\begin{PSideBySideExample}[xrightmargin=5.5cm](3,2)
  \pnode(0,1){A}
  \pnode(3,1){B}
  /HLCBWz&\Icc"(A)(B){$\eta$}
\end{PSideBySideExample}

\begin{PSideBySideExample}[xrightmargin=5.5cm](3,2)
  \pnode(0,1){A}
  \pnode(3,1){B}
  /HLCBWz&\switch"(A)(B){$K$}
\end{PSideBySideExample}

\begin{PSideBySideExample}[xrightmargin=5.5cm](3,2)
  \pnode(0,1){A}
  \pnode(3,1){B}
  /HLCBWz&\diode"(A)(B){$D$}
\end{PSideBySideExample}

\begin{PSideBySideExample}[xrightmargin=5.5cm](3,2)
  \pnode(0,1){A}
  \pnode(3,1){B}
  /HLCBWz&\Zener"(A)(B){$D$}
\end{PSideBySideExample}

\begin{PSideBySideExample}[xrightmargin=5.5cm](3,2)
  \pnode(0,1){A}
  \pnode(3,1){B}
  /HLCBWz&\lamp"(A)(B){$\mathcal L$}
\end{PSideBySideExample}

\begin{PSideBySideExample}[xrightmargin=5.5cm](3,2)
  \pnode(0,1){A}
  \pnode(3,1){B}
  /HLCBWz&\circledipole"(A)(B){$\mathcal G$}
\end{PSideBySideExample}

\begin{PSideBySideExample}[xrightmargin=5.5cm](3,2)
  \pnode(0,1){A}
  \pnode(3,1){B}
  /HLCBWz&\LED"(A)(B){$\mathcal D$}
\end{PSideBySideExample}

\subsection{Tripole macros}

Obviously, tripoles are not node connections. So \CircPackage{} tries its best to adjust the
position of the tripole regarding the three nodes. Internally, the connections are done by the
\cs{ncangle} pst-node macro. However, the auto-positionning and the auto-connections are not always
well chosen\footnote{This is something we are working on. I think that auto-positionning and
auto-connections should be done at PostScript level and not at PSTricks level. If someone has any
ideas, please mail us.}, so don't try to use tripole macros in strange situations!

\begin{PSideBySideExample}[xrightmargin=5.5cm](5,3)
  \pnode(0,0){A}
  \pnode(0,3){B}
  \pnode(5,1.5){C}
  /HLCBWz&\OA"(B)(A)(C)
\end{PSideBySideExample}

\begin{PSideBySideExample}[xrightmargin=5.5cm](5,3)
  \pnode(0,1.5){A}
  \pnode(5,3){B}
  \pnode(5,0){C}
  /HLCBWz&\transistor"(A)(B)(C)
\end{PSideBySideExample}

\begin{PSideBySideExample}[xrightmargin=5.5cm](5,3)
  \pnode(0,2){A}
  \pnode(5,2){B}
  \pnode(0,0){C}
  /HLCBWz&\Tswitch"(A)(B)(C){$K$}
\end{PSideBySideExample}

\begin{PSideBySideExample}[xrightmargin=3.5cm](3,3)
  \pnode(0,1){A}
  \pnode(3,1){B}
  \pnode(3,2.25){C}
  /HLCBWz&\potentiometer"[labeloffset=0pt](A)(B)(C){$P$}
\end{PSideBySideExample}

\subsection{Quadrupole macros}

\begin{PSideBySideExample}[xrightmargin=5.5cm](5,5)
  \pnode(0,5){A}
  \pnode(0,0){B}
  \pnode(5,5){C}
  \pnode(5,0){D}
  /HLCBWz&\transformer"(A)(B)(C)(D){$\mathcal T$}
\end{PSideBySideExample}

\begin{PSideBySideExample}[xrightmargin=4.5cm](4,3)
  \pnode(0,2.5){A}
  \pnode(0,0.5){B}
  \pnode(4,2.5){C}
  \pnode(4,0.5){D}
  /HLCBWz&\optoCoupler"(A)(B)(C)(D){$OC$}
\end{PSideBySideExample}

\subsection{Multidipole}



\cs{multidipole} is a macro that allows multiple dipoles to be drawn between two specified nodes.
\cs{multidipole} takes as many arguments as you want. \textbf{Note the \rnode{Dot}{dot} that is
after the last dipole.}

\def\HLrnode#1{\rnode{Dot2}{#1}}
\begin{PSideBySideExample}[xrightmargin=8.5cm](8,8)
  \pnode(0,0){A}
  \pnode(8,8){B}
  /HLCBWz&\multidipole"(A)(B)\resistor{$R$}%
    \capacitor[linecolor=red]{$C$}%
    \diode{$D$}/HLrnode&/HLCBWz."
\end{PSideBySideExample}

\ncangles[linestyle=dashed,linecolor=gray,nodesep=3pt,armA=.5cm,angleA=-90,armB=4cm,angleB=0]{->}{Dot}{Dot2} 
Important: for the time being, \cs{multidipole} takes optional arguments but does not 
restore original values. We recommand not using it.


\subsection{Wire}

\begin{PSideBySideExample}[xrightmargin=3.5cm](3,2)
  \pnode(0,1){A}
  \pnode(3,1){B}
  /HLCBWz&\wire"(A)(B)
\end{PSideBySideExample}

\subsection{Potential}

\begin{PSideBySideExample}[xrightmargin=3.5cm](3,2)
  \pnode(0,1){A}
  \pnode(3,1){B}
  /HLCBWz&\tension"(A)(B){$u$}
\end{PSideBySideExample}

\subsection{ground}

\begin{PSideBySideExample}[xrightmargin=3.5cm](3,2)
  \pnode(0.5,1){A}
  \pnode(1,1){B}
  \pnode(2.5,1){C}
  /HLCBWz&\ground"(A)
  \ground{/HLCBWz&135"}(B)
  \ground[/HLCBWz&linecolor"=blue]{180}(C)
\end{PSideBySideExample}

\section{Parameters}

\subsection{Label parameters}

\begin{PSideBySideExample}[xrightmargin=3.5cm](3,2)
  \pnode(0,.5){A}
  \pnode(3,.5){B}
  \resistor[/HLCBWz&labeloffset"=0](A)(B){$R$}
\end{PSideBySideExample}

\begin{PSideBySideExample}[xrightmargin=3.5cm](3,2)
  \pnode(0,0){A}
  \pnode(3,2){B}
  \resistor[/HLCBWz&labelangle"=:U](A)(B){$R$}
\end{PSideBySideExample}

\begin{PSideBySideExample}[xrightmargin=3.5cm](3,2)
  \pnode(0,0){A}
  \pnode(3,2){B}
  \resistor[/HLCBWz&labelangle"=0](A)(B){$R$}
\end{PSideBySideExample}

\begin{PSideBySideExample}[xrightmargin=5.5cm](5,5)
  \pnode(0,5){A}
  \pnode(0,0){B}
  \pnode(5,5){C}
  \pnode(5,0){D}
  \transformer[/HLCBWz&primarylabel"=$n_1$,
    /HLCBWz&secondarylabel"=$n_2$](A)(B)(C)(D){$\mathcal T$}
\end{PSideBySideExample}


\subsection{Current intensity and electrical potential parameters}

If the \HLTTc{intensity} parameter is set to \texttt{true}, an arrow is drawn on the wire
connecting one of the nodes to the dipole. If the \HLTTc{tension} parameter is set to \texttt{true},
an arrow is drawn parallel to the dipole.

The way those arrows are drawn is set by \HLTTc{dipoleconvention} and \HLTTc{directconvention}
parameters. \HLTTc{dipoleconvention} can take two values~: \texttt{generator} or \texttt{receptor}.
\HLTTc{directconvention} is a boolean.

\begin{PSideBySideExample}[xrightmargin=3.5cm](3,2)
  \pnode(0,.5){A}
  \pnode(3,.5){B}
  \resistor[/HLCBWz&intensity",/HLCBWz&tension"](A)(B){}
\end{PSideBySideExample}

\begin{PSideBySideExample}[xrightmargin=3.5cm](3,2)
  \pnode(0,.5){A}
  \pnode(3,.5){B}
  \resistor[intensity,tension,
    /HLCBWz&dipoleconvention"=generator](A)(B){}
\end{PSideBySideExample}

\begin{PSideBySideExample}[xrightmargin=3.5cm](3,2)
  \pnode(0,.5){A}
  \pnode(3,.5){B}
  \resistor[intensity,tension,
    /HLCBWz&directconvention"=false](A)(B){}
\end{PSideBySideExample}

\begin{PSideBySideExample}[xrightmargin=3.5cm](3,2)
  \pnode(0,.5){A}
  \pnode(3,.5){B}
  \resistor[intensity,tension,
    dipoleconvention=generator,directconvention=false](A)(B){}
\end{PSideBySideExample}

If \HLTTc{intensitylabel} is set to an non empty argument, then \HLTTc{intensity} is automatically
set to true. If \HLTTc{tensionlabel} is set to an non empty argument, then \HLTTc{tension} is
automatically set to true.

\begin{PSideBySideExample}[xrightmargin=3.5cm](3,2)
  \pnode(0,.5){A}
  \pnode(3,.5){B}
  \resistor[/HLCBWz&intensitylabel"=$i$,/HLCBWz&tensionlabel"=$u$](A)(B){}
\end{PSideBySideExample}

\begin{PSideBySideExample}[xrightmargin=3.5cm](3,2)
  \pnode(0,1.5){A}
  \pnode(3,1.5){B}
  \resistor[intensitylabel=$i$,/HLCBWz&intensitylabeloffset"=-0.5,
    tensionlabel=$u$,/HLCBWz&tensionlabeloffset"=-1.2,
    /HLCBWz&tensionoffset"=-1](A)(B){}
\end{PSideBySideExample}

\begin{PSideBySideExample}[xrightmargin=3.5cm](3,2)
  \pnode(0,.5){A}
  \pnode(3,.5){B}
  \resistor[intensitylabel=$i$,/HLCBWz&intensitywidth"=3\pslinewidth,
    /HLCBWz&intensitycolor"=red,/HLCBWz&intensitylabelcolor"=yellow,
    tensionlabel=$u$,/HLCBWz&tensionwidth"=2\pslinewidth,
    /HLCBWz&tensioncolor"=green,/HLCBWz&tensionlabelcolor"=blue](A)(B){}
\end{PSideBySideExample}

Some specific intensity parameters are available for tripoles and quadrupoles.

\begin{PSideBySideExample}[xrightmargin=5.5cm](5,3)
  \pnode(0,0){A}
  \pnode(0,3){B}
  \pnode(5,1.5){C}
  \OA[/HLCBWz&OAipluslabel"=$i_+$,
    /HLCBWz&OAiminuslabel"=$i_-$,
    /HLCBWz&OAioutlabel"=$i_o$](B)(A)(C)
\end{PSideBySideExample}

\begin{PSideBySideExample}[xrightmargin=5.5cm](5,3)
  \pnode(0,1.5){A}
  \pnode(5,3){B}
  \pnode(5,0){C}
  \transistor[/HLCBWz&transistoribaselabel"=$i_B$,
    /HLCBWz&transistoricollectorlabel"=$i_C$,
    /HLCBWz&transistoriemitterlabel"=$i_E$](A)(B)(C)
\end{PSideBySideExample}

\begin{PSideBySideExample}[xrightmargin=5.5cm](5,5)
  \pnode(0,5){A}
  \pnode(0,0){B}
  \pnode(5,5){C}
  \pnode(5,0){D}
  \transformer[/HLCBWz&transformeriprimarylabel"=$i_1$,
    /HLCBWz&transformerisecondarylabel"=$i_2$]%
    (A)(B)(C)(D){$\mathcal T$}
\end{PSideBySideExample}


\subsection{Parallel parameters}

If the \HLTTc{parallel} parameter is set to \texttt{true}, the dipole is drawn parallel to the line
connecting the nodes.

\begin{PSideBySideExample}[xrightmargin=3.5cm](3,3)
  \pnode(0,.5){A}
  \pnode(3,.5){B}
  \resistor(A)(B){}
  \resistor[/HLCBWz&parallel"](A)(B){}
\end{PSideBySideExample}

\begin{PSideBySideExample}[xrightmargin=3.5cm](3,3)
  \pnode(0,.5){A}
  \pnode(3,.5){B}
  \resistor(A)(B){}
  \resistor[parallel,/HLCBWz&parallelsep"=.5](A)(B){}
\end{PSideBySideExample}

\begin{PSideBySideExample}[xrightmargin=3.5cm](3,3)
  \pnode(0,.5){A}
  \pnode(3,.5){B}
  \resistor(A)(B){}
  \resistor[parallel,parallelsep=.3,
    /HLCBWz&parallelarm"=2](A)(B){}
\end{PSideBySideExample}

\begin{PSideBySideExample}[xrightmargin=3.5cm](3,3)
  \pnode(0,.5){A}
  \pnode(3,.5){B}
  \resistor(A)(B){}
  \resistor[parallel,parallelsep=.3,
    parallelarm=2,/HLCBWz&parallelnode"](A)(B){}
\end{PSideBySideExample}

\begin{PSideBySideExample}[xrightmargin=8.5cm](8,8)
  \pnode(0,0){A}
  \pnode(8,8){B}
  \multidipole(A)(B)\resistor{$R$}%
    \capacitor[linecolor=red]{$C$}%
    \coil[parallel,parallelsep=.1]{$L$}%
    \diode{$D$}.
\end{PSideBySideExample}

Note: When used with \cs{multidipole}, the  parallel \HLTTc{parameter}
must not be set for the first dipole.

\subsection{Wire intersections}

\begin{PSideBySideExample}[xrightmargin=3.5cm](3,3)
  \pnode(0,0){A}
  \pnode(3,3){B}
  \pnode(0,3){C}
  \pnode(3,0){D}
  \wire(A)(B)
  \wire[/HLCBWz&intersect",/HLCBWz&intersectA"=A,/HLCBWz&intersectB"=B](C)(D)
\end{PSideBySideExample}

Wire intersect parameters work also with \cs{multidipole}.

\begin{PSideBySideExample}[xrightmargin=8.5cm](8,8)
  \pnode(0,0){A}
  \pnode(8,8){B}
  \pnode(0,8){C}
  \pnode(8,0){D}
  \wire(A)(B)
  \multidipole(C)(D)\resistor{$R$}%
    \wire[intersect,intersectA=A,intersectB=B]%
    \capacitor{$C$}.
\end{PSideBySideExample}



\subsection{Dipole style parameters}

\begin{PSideBySideExample}[xrightmargin=3.5cm](3,2)
  \pnode(0,1){A}
  \pnode(3,1){B}
  \resistor[/HLCBWz&dipolestyle"=/HLCBWz&zigzag"](A)(B){$R$}
\end{PSideBySideExample}

\begin{PSideBySideExample}[xrightmargin=3.5cm](3,2)
  \pnode(0,1){A}
  \pnode(3,1){B}
  \capacitor[dipolestyle=/HLCBWz&chemical"](A)(B){$C$}
\end{PSideBySideExample}

\begin{PSideBySideExample}[xrightmargin=3.5cm](3,2)
  \pnode(0,1){A}
  \pnode(3,1){B}
  \capacitor[dipolestyle=/HLCBWz&elektor"](A)(B){$C$}
\end{PSideBySideExample}

\begin{PSideBySideExample}[xrightmargin=3.5cm](3,2)
  \pnode(0,1){A}
  \pnode(3,1){B}
  \capacitor[dipolestyle=/HLCBWz&elektorchemical"](A)(B){$C$}
\end{PSideBySideExample}

\begin{PSideBySideExample}[xrightmargin=3.5cm](3,2)
  \pnode(0,1){A}
  \pnode(3,1){B}
  \coil[dipolestyle=/HLCBWz&rectangle"](A)(B){$L$}
\end{PSideBySideExample}

\begin{PSideBySideExample}[xrightmargin=3.5cm](3,2)
  \pnode(0,1){A}
  \pnode(3,1){B}
  \coil[dipolestyle=/HLCBWz&curved"](A)(B){$L$}
\end{PSideBySideExample}

\begin{PSideBySideExample}[xrightmargin=3.5cm](3,2)
  \pnode(0,1){A}
  \pnode(3,1){B}
  \coil[dipolestyle=/HLCBWz&elektor"](A)(B){$L$}
\end{PSideBySideExample}

\begin{PSideBySideExample}[xrightmargin=3.5cm](3,2)
  \pnode(0,1){A}
  \pnode(3,1){B}
  \coil[dipolestyle=/HLCBWz&elektorcurved"](A)(B){$L$}
\end{PSideBySideExample}

\begin{PSideBySideExample}[xrightmargin=3.5cm](3,2)
  \pnode(0,1){A}
  \pnode(3,1){B}
  \diode[dipolestyle=/HLCBWz&thyristor"](A)(B){$T$}
\end{PSideBySideExample}

\begin{PSideBySideExample}[xrightmargin=3.5cm](3,2)
  \pnode(0,1){A}
  \pnode(3,1){B}
  \diode[dipolestyle=/HLCBWz&GTO"](A)(B){$T$}
\end{PSideBySideExample}

\begin{PSideBySideExample}[xrightmargin=3.5cm](3,2)
  \pnode(0,1){A}
  \pnode(3,1){B}
  \diode[dipolestyle=/HLCBWz&triac"](A)(B){$T$}
\end{PSideBySideExample}

\begin{PSideBySideExample}[xrightmargin=3.5cm](3,2)
  \pnode(0,1){A}
  \pnode(3,1){B}
  \resistor[/HLCBWz&variable"](A)(B){$R$}
\end{PSideBySideExample}

\begin{PSideBySideExample}[xrightmargin=3.5cm](3,2)
  \pnode(0,1){A}
  \pnode(3,1){B}
  \capacitor[/HLCBWz&variable"](A)(B){$C$}
\end{PSideBySideExample}

\begin{PSideBySideExample}[xrightmargin=3.5cm](3,2)
  \pnode(0,1){A}
  \pnode(3,1){B}
  \coil[/HLCBWz&variable"](A)(B){$L$}
\end{PSideBySideExample}

\begin{PSideBySideExample}[xrightmargin=3.5cm](3,2)
  \pnode(0,1){A}
  \pnode(3,1){B}
  \battery[/HLCBWz&variable"](A)(B){$U$}
\end{PSideBySideExample}

\begin{PSideBySideExample}[xrightmargin=3.5cm](3,2)
  \pnode(0,1){A}
  \pnode(3,1){B}
  \coil[dipolestyle=/HLCBWz&elektor",variable](A)(B){$L$}
\end{PSideBySideExample}

In the following example the parameter \verb|dipolestyle| is used for a tripole and quadrupole, because
the coils are drawn as rectangles and the resistor as a zigzag.

\begin{PSideBySideExample}[xrightmargin=3.5cm](3,3)
  \pnode(0,0){A}
  \pnode(3,3){B}
  \pnode(3,1.5){C}
  \potentiometer[,dipolestyle=/HLCBWz&zigzag",%
  	labelangle=:U](A)(B)(C){$P$}
\end{PSideBySideExample}

\begin{PSideBySideExample}[xrightmargin=4.5cm](4,4)
  \pnode(0,4){A}
  \pnode(0,0){B}
  \pnode(4,4){C}
  \pnode(4,0){D}
  /HLCBWz&\transformer"[dipolestyle=/HLCBWz&rectangle"](A)(B)(C)(D){$\mathcal T$}
\end{PSideBySideExample}


\subsection{Tripole style parameters}

\begin{PSideBySideExample}[xrightmargin=5.5cm](5,3)
  \pnode(0,2){A}
  \pnode(5,2){B}
  \pnode(0,0){C}
  \Tswitch[tripolestyle=left](A)(B)(C){$K$}
\end{PSideBySideExample}

\begin{PSideBySideExample}[xrightmargin=5.5cm](5,3)
  \pnode(0,2){A}
  \pnode(5,2){B}
  \pnode(0,0){C}
  \Tswitch[tripolestyle=right](A)(B)(C){$K$}
\end{PSideBySideExample}

\begin{PSideBySideExample}[xrightmargin=5.5cm](5,3)
  \pnode(0,3){A}
  \pnode(0,0){B}
  \pnode(5,1.5){C}
  \OA[tripolestyle=french](A)(B)(C)
\end{PSideBySideExample}

\subsection{Other Parameters}

\begin{PSideBySideExample}[xrightmargin=5.5cm](5,3)
  \pnode(0,0){A}
  \pnode(0,3){B}
  \pnode(5,1.5){C}
  \OA[/HLCBWz&OAinvert"=false](B)(A)(C)
\end{PSideBySideExample}

\begin{PSideBySideExample}[xrightmargin=5.5cm](5,3)
  \pnode(0,0){A}
  \pnode(0,3){B}
  \pnode(5,1.5){C}
  \OA[/HLCBWz&OAperfect"=false](B)(A)(C)
\end{PSideBySideExample}

\begin{PSideBySideExample}[xrightmargin=5.5cm](5,3)
  \pnode(0,1.5){A}
  \pnode(5,3){B}
  \pnode(5,0){C}
  \transistor[/HLCBWz&transistorinvert",/HLCBWz&transistorcircle"=false](A)(B)(C)
\end{PSideBySideExample}


\section{Examples}

\begin{SideBySideExample}[xrightmargin=8cm]
  \begin{pspicture}(-1.5,-1)(6,5)
  \psgrid[subgriddiv=1,griddots=10]
  % Node definitions
  \pnode(0,0){A}
  \pnode(0,3){B}
  \pnode(4.5,3){C}
  \pnode(4.5,0){D}
  % Dipole node connection
  \Ucc[tension,dipoleconvention=generator](A)(B){$E$}
  \multidipole(B)(C)%
    \switch[intensitylabel=$i$]{$K$}%
    \resistor[labeloffset=0,tensionlabel=$u_R$]{$R$}.
  \capacitor[tensionlabel={$u_C$},
    tensionlabeloffset=-1.2,tensionoffset=-1,
    directconvention=false](D)(C){$C$}
  % Wire to complete circuit
  \wire(A)(D)
  % Ground
  \ground(D)
  \end{pspicture}
\end{SideBySideExample}

\begin{SideBySideExample}[xrightmargin=8cm]
  \begin{pspicture}(-0.5,0)(7,8)
  \psgrid[subgriddiv=1,griddots=10]
  % Node definitions
  \pnode(0.5,1){A}
  \pnode(3.5,1){B}
  \pnode(6.5,1){C}
  \pnode(0.5,4){D}
  \pnode(3.5,4){Minus}
  \pnode(3.5,3){Plus}
  \pnode(6.5,5){S}
  \pnode(3.5,5){E}
  % Dipole node connections
  \resistor(D)(Minus){$R_2$}
  \capacitor(E)(S){$C$}
  \resistor[parallel,parallelarm=2](E)(S){$R_1$}
  \OA[intensity](Minus)(Plus)(S)
  % Wires
  \wire(Minus)(E)
  \wire(Plus)(B)
  % Tensions
  \tension(A)(D){$u_E$}
  \makeatletter % (special tricks see below)
  \tension(C)(S@@){$u_S$}
  \tension[linecolor=blue](Plus@@)(Minus@@){$\epsilon$}
  \makeatother
  % Grounds
  \ground(A)
  \ground(B)
  \ground(C)
  \end{pspicture}
\end{SideBySideExample}

\begin{SideBySideExample}[xrightmargin=8.5cm]
  \begin{pspicture}(-1,0)(7,8)
  \psgrid[subgriddiv=1,griddots=10]
  % Node definitions
  \pnode(1,1){A}
  \pnode(1,7){B}
  \pnode(3,1){C}
  \pnode(3,7){D}
  % Dipole node connections
  \Ucc[tensionlabel=$E$](A)(B){}
  \resistor(B)(D){$R$}
  \coil(D)(C){$L$}
  \capacitor[parallel,parallelarm=2.5](D)(C){$C$}
  % Wire
  \wire(A)(C)
  \end{pspicture}
\end{SideBySideExample}

\begin{SideBySideExample}[xrightmargin=8.5cm]
  \begin{pspicture}(6,6)
  \psgrid[subgriddiv=1,griddots=10]
  % Node definitions
  \pnode(0,3){A}
  \pnode(3,3){B}
  \pnode(6,3){C}
  % Dipole node connections
  \coil[intensitylabel=$i$](A)(B){$L$}
  \coil[intensitylabel=$i'$,intensitycolor=green,%
    parallel,parallelarm=2](B)(C){$L'$}
  \capacitor[parallel,parallelarm=-2](B)(C){$C$}
  \end{pspicture}
\end{SideBySideExample}

\begin{SideBySideExample}[xrightmargin=8.5cm]
  \begin{pspicture}(6,6)
  \psgrid[subgriddiv=1,griddots=10]
  % Node definitions
  \pnode(0,0){A}\pnode(6,0){B}
  \pnode(0.3,4){Cprime}\pnode(5.7,4){Dprime}
  \pnode(2.5,4){Gprime}\pnode(2.5,0){Hprime}
  \pnode(0,4){C}\pnode(6,4){D}
  \pnode(0.3,6){E}\pnode(5.7,6){F}
  \pnode(4,6){G}\pnode(4,0){H}
  \multidipole(G)(H)%
    \wire[intersect,
      intersectA=C,intersectB=D]
    \resistor{$R'_3$}.
  \resistor(E)(G){$R'_1$}
  \resistor(G)(F){$R'_2$}
  \multidipole(C)(D)\resistor{$R_1$}%
    \wire\resistor{$R_2$}.
  \wire(A)(B)\wire(Cprime)(E)
  \wire(Dprime)(F)
  \resistor(Hprime)(Gprime){$R_3$}
  \end{pspicture}
\end{SideBySideExample}



\begin{SideBySideExample}[xrightmargin=9.5cm]
  \begin{pspicture}(9,11)
  \psgrid[subgriddiv=1,griddots=10]
  % Node definitions
  \pnode(0,0){A}
  \pnode(9,0){B}
  \pnode(0,6){C}
  \pnode(9,6){D}
  \pnode(4.5,1){E}
  \pnode(4.5,10.5){F}
  %
  \switch(A)(C){$K$}
  \multidipole(A)(B)\resistor{$R$}%
    \battery[intensitylabel=$i$]{$V$}.
  \wire(B)(D)
  \multidipole(C)(D)\diode{$D$}\wire.
  \resistor[tensionlabel=$U_1$](C)(F){$R_1$}
  \resistor(C)(E){$R_4$}
  \capacitor[parallel,parallelarm=1.2,
    parallelsep=1.5](C)(E){$C_2$}
  \coil(E)(D){$L$}
  \capacitor[parallel,parallelarm=1.2,
    parallelsep=1.5](E)(D){$C_3$}
  \capacitor[tensionlabel=$U_2$](F)(D){$C_1$}
  \multidipole(E)(F)\wire%
    \wire[intersect,
    intersectA=C,intersectB=D]%
    \circledipole[labeloffset=-0.7]{$E$}%
    \resistor[parallel,
      parallelsep=.6,parallelarm=.8]{$R$}.
  \end{pspicture}
\end{SideBySideExample}


\begin{CenterExample}
  \begin{pspicture}(13,8)
    \psset{intensitycolor=red,intensitylabelcolor=red,%
        tensioncolor=green,tensionlabelcolor=green,%
        intensitywidth=3pt}
    \psgrid[griddots=5,gridlabels=7pt,subgriddiv=0]
    \circledipole[
         tension,%
         tensionlabel=$U_0$,%
         tensionoffset=0.75,%
         labeloffset=0](0,0)(0,6){\LARGE\textbf{=}}
    \wire[intensity,intensitylabel=$i_0$](0,6)(2.5,6)
    \diode[dipolestyle=thyristor](2.5,6)(4.5,6){$T_1$}
    \wire[intensity,intensitylabel=$i_1$](4.5,6)(6.5,6)
    \multidipole(6.5,7.5)(2.5,7.5)%
        \coil[dipolestyle=rectangle,labeloffset=-0.75]{$L_5$}%
        \diode[labeloffset=-0.75]{$D_5$}.
    \wire[intensity,intensitylabel=$i_5$](6.5,6)(6.5,7.5)
    \wire(2.5,7.5)(2.5,3)
    \wire[intensity,intensitylabel=$i_c$](2.5,4.5)(2.5,6)
    \qdisk(2.5,6){2pt}\qdisk(6.5,6){2pt}
    \diode[dipolestyle=thyristor](2.5,4.5)(4.5,4.5){$T_2$}
    \wire[intensity,intensitylabel=$i_2$](4.5,4.5)(6.5,4.5)
    \capacitor[tension,tensionlabel=$u_c$,%
        tensionoffset=-0.75,tensionlabeloffset=-1](6.5,4.5)(6.5,6){$C_k$}
    \qdisk(2.5,4.5){2pt}\qdisk(6.5,4.5){2pt}
    \wire[intensity,intensitylabel=$i_3$](6.5,4.5)(6.5,3)
    \multidipole(6.5,3)(2.5,3)%
        \coil[dipolestyle=rectangle,labeloffset=-0.75]{$L_3$}%
        \diode[labeloffset=-0.75]{$D_3$}.
    \wire(6.5,6)(9,6)\qdisk(9,6){2pt}
    \diode(9,0)(9,6){$D_4$}
    \wire[intensity,intensitylabel=$i_4$](9,3.25)(9,6)
    \wire[intensity,intensitylabel=$i_a$](9,6)(11,6)
    \multidipole(11,6)(11,0)%
        \resistor{$R_L$}
        \coil[dipolestyle=rectangle]{$L_L$}%
    \circledipole[labeloffset=0,%
            tension,tensionoffset=0.7,%
            tensionlabel=$U_B$]{\LARGE\textbf{=}}.
    \wire(0,0)(11,0)\qdisk(9,0){2pt}
    \pnode(12.5,5.5){A}\pnode(12.5,0.5){B}
    \tension(A)(B){$u_a$}
  \end{pspicture}
\end{CenterExample}


\makeatletter
%
\def\REG{\@ifnextchar[{\pst@REG}{\pst@REG[]}}
%
\def\pst@REG[#1](#2)(#3)(#4)#5{{%
  \setkeys{psset}{#1,dimen=middle,arm=0}%
  \pst@getcoor{#2}\pst@tempa
  \pst@getcoor{#3}\pst@tempb
  \pst@getcoor{#4}\pst@tempc
  \pnode(!%
    \pst@tempa /Y1 exch \pst@number\psyunit div def
    /X1 exch \pst@number\psxunit div def
    \pst@tempb /Y2 exch \pst@number\psyunit div def
    /X2 exch \pst@number\psxunit div def
    \pst@tempc /Y3 exch \pst@number\psyunit div def
    /X3 exch \pst@number\psxunit div def
    /XC X1 X2 add 2 div def
    /YC Y1 2 mul Y3 add 3 div def
    /Xin XC 1 sub def
    /Yin YC 0.5 add def
    /Xout XC 1 add def
    /Yout Yin def
    /Xref XC def
    /Yref YC 1 sub def
    XC YC){C@}
  \pnode(! Xin Yin){in@}
  \pnode(! Xout Yout){out@}
  \pnode(! Xref Yref){ref@}
  \rput(C@){\pst@draw@REG}
  \ncangle{#2}{in@}
  \ncangle{#3}{out@}
  \ncangle{#4}{ref@}
  \rput(C@){#5}
  }\ignorespaces}
%
\def\pst@draw@REG{%
  \begingroup
  \psset{linewidth=1.5\pslinewidth}%
  \psframe(-1,-0.5)(1,0.75)
  \psline(-1.5,0.5)(-1,0.5)
  \psline(1.5,0.5)(1,0.5)
  \psline(0,-0.5)(0,-1)
  \endgroup
  }
%
\makeatother

The fellowing example was written by Manuel Luque.
\begin{CenterExample}
  \begin{pspicture}(14,4)
  \psgrid[subgriddiv=1,griddots=10]
  \pnode(0,0){B}\pnode(0,3){A}
  \pnode(2.5,3.5){C}\pnode(2.5,-0.5){D}
  \pnode(5,3){E}\pnode(6.5,1.5){F}
  \pnode(5,0){G}\pnode(3.5,1.5){H}
  \pnode(8,2.5){I}\pnode(8,1){J}
  \pnode(10,2.5){K}\pnode(10,1){L}
  \pnode(14,2.5){M}\pnode(12,1){N}
  \pnode(3,1){H'}\pnode(14,2.5){O}
  \pnode(14,1){P}\pnode(13.5,1){Q}
  \transformer[transformeriprimarylabel=$i_1$,
    transformerisecondarylabel=$i_2$,
    primarylabel=$n_1$,secondarylabel=$n_2$]%
    (A)(B)(C)(D){$T_1$}
  {\psset{fillstyle=solid,fillcolor=black}
  \diode(H)(E){}\diode(H)(G){}
  \diode(E)(F){}\diode(G)(F){}}
  \capacitor[dipolestyle=chemical](I)(J){}
  \capacitor(K)(L){}
  \REG(K)(M)(N)%
    {\shortstack{\textsf{%
    \textbf{\large LM7805}}\\\textbf{+5V}}}
  \ncangle{I}{F}\psline(I)(K)
  \ncangle{E}{C}\ncangle{G}{D}
  \ncangle[arm=0]{P}{Q}
  \ncangle[arm=0]{H}{H'}
  \ground(H')\ground(J)
  \ground(L)\ground(N)
  \ground(Q)\qdisk(I){1.5pt}
  \qdisk(K){1.5pt}\qdisk(E){1.5pt}
  \qdisk(G){1.5pt}\qdisk(H){1.5pt}
  \qdisk(F){1.5pt}
  \pscircle[fillstyle=solid](A){0.075}
  \pscircle[fillstyle=solid](B){0.075}
  \pscircle[fillstyle=solid](P){0.075}
  \pscircle[fillstyle=solid](O){0.075}
  \end{pspicture}
\end{CenterExample}

The fellowing example was written by Lionel Cordesses.

\begin{CenterExample}
  \begin{pspicture}(11,3)
  \psset{dipolestyle=elektor}
  \pnode(1,2){Vin}\pnode(0.5,2){S}\pnode(0.5,0){Sm}
  \pnode(2.5,2){A}\pnode(4.5,2){B}\pnode(6.5,2){C}
  \pnode(8,2){Cd}\pnode(8.5,2){D}\pnode(9.5,2){E}
  \pnode(2.5,0){Am}\pnode(4.5,0){Bm}\pnode(6.5,0){Cm}
  \pnode(8.5,0){Dm}\pnode(9.5,0){Em}
  \Ucc[labeloffset=0.9](Sm)(S){$V_{in}$}\resistor(Vin)(A){$R$}
  \capacitor(A)(Am){$C_1$} \capacitor(B)(Bm){$C_3$}
  \capacitor[labeloffset=-0.7](D)(Dm){$C_n$}\resistor(E)(Em){$R$}
  \coil(A)(B){$L_2$}\coil(B)(C){$L_4$}
  \wire(Am)(Bm)\wire(Bm)(Cm)\wire(Cm)(Dm)\wire(Dm)(Em)\wire(D)(E)
  \wire(Cd)(D)\psline[linestyle=dashed](C)(Cd)
  \wire(S)(Vin)\wire(Sm)(Am)
  \pscircle*(D){2\pslinewidth} \pscircle*(Dm){2\pslinewidth}
  \pscircle*(A){2\pslinewidth} \pscircle*(Am){2\pslinewidth}
  \pscircle*(B){2\pslinewidth} \pscircle*(Bm){2\pslinewidth}
  \end{pspicture}
\end{CenterExample}

\section{Adding new components}

Adding new components is not simple. As a matter of fact, because of the complex
mechanism of \cs{multidipole}, there are multiple steps. The easiest way to proceed is
to draw the component, send it to me (\texttt{\makeatletter
christophe.jorssen@noos.fr\makeatother}) and I'll do the programming work regarding your
component. Nevertheless, it can take some time\dots

If you want to modify the code, you need to know the following
things. For a dipole, you first
need to define the following items \fvset{commandchars=/&"}

\begin{Verbatim}
  \def\/emph&component/_name"{\@ifnextchar[{\pst@/emph&component/_name"}{\pst@/emph&component/_name"[]}}
  %
  \def\pst@/emph&component/_name"[#1](#2)(#3)#4{{%
    \pst@draw@dipole{#1}{#2}{#3}{#4}\pst@draw@/emph&component/_name"
    }\ignorespaces}
  %
  \def\pst@multidipole@/emph&component/_name"{\@ifnextchar[{\pst@multidipole@/emph&component/_name"@}%
    {\pst@multidipole@/emph&component/_name"@[]}}
  %
  \def\pst@multidipole@/emph&component/_name"@[#1]#2{%
    \expandafter\def\csname pst@circ@tmp@\number\pst@circ@count@iii\endcsname{#2}%
    {\setkeys{psset}{#1}%
    \ifPst@circ@parallel\aftergroup\advance\aftergroup\pst@circ@count@i\aftergroup\m@ne\fi}%
    \pst@circ@count@ii=\pst@circ@count@i%
    \advance\pst@circ@count@ii\@ne%
    \toks0\expandafter{\pst@multidipole@output}%
    \edef\pst@multidipole@output{%
      \the\toks0%
      \pst@multidipole@def@coor%
      \noexpand\/emph&component/_name"[#1]%
    (! X@\the\pst@circ@count@i\space Y@\the\pst@circ@count@i)%
    (! X@\the\pst@circ@count@ii\space Y@\the\pst@circ@count@ii)%
        {\noexpand\csname pst@circ@tmp@\number\pst@circ@count@iii\endcsname}%
    }%
    \pst@multidipole@
  }
  %
  \def\pst@draw@/emph&component/_name"{%
    % The PSTricks code for your component
    % The center of the component is at (0,0)
    \pnode(/emph&component/_left/_end",0){dipole@1}
    \pnode(/emph&component/_right/_end",0){dipole@2}
  }
\end{Verbatim}

Then, you have to make some changes in the \cs{multidipole} core code\dots In the definition
of \Verb+\pst@multidipole+, look for the last \Verb+\ifx+ test
\begin{Verbatim}
  % ...
  % Extract from \pst@multidipole
                      \else
                        \ifx\circledipole #4%
                          \let\next\pst@multidipole@circledipole
                        \else
                          \ifx\LED #4%
                            \let\next\pst@multidipole@LED
                          \else
                          % Put your modification here
                            \let\next\ignorespaces
                          \fi
                        \fi
                      \fi
  % Extract form \pst@multidipole
  % ...
\end{Verbatim}
and add (in red)
\begin{Verbatim}
  % ...
  % Extract from \pst@multidipole
                      \else
                        \ifx\circledipole #4%
                          \let\next\pst@multidipole@circledipole
                        \else
                          \ifx\LED #4%
                            \let\next\pst@multidipole@LED
                          \else
                            /red&\ifx\/emph&component/_name" #4%"
                              /red&\let\next\pst@multidipole@/emph&component/_name""
                            /red&\else"
                              \let\next\ignorespaces
                            /red&\fi"
                          \fi
                        \fi
                      \fi
  % Extract form \pst@multidipole
  % ...
\end{Verbatim}
Do the same in \Verb+\pst@multidipole@+
\begin{Verbatim}
  % ...
  % Extract from \pst@multidipole@
                      \else
                        \ifx\circledipole #1%
                          \let\next\pst@multidipole@circledipole
                        \else
                          \ifx\LED #1%
                            \let\next\pst@multidipole@LED
                          \else
                            /red&\ifx\/emph&component/_name" #1%"
                              /red&\let\next\pst@multidipole@/emph&component/_name""
                            /red&\else"
                              \let\next\ignorespaces
                              \pst@multidipole@output
                            /red&\fi"
                          \fi
                        \fi
                      \fi
  % Extract form \pst@multidipole@
  % ...
\end{Verbatim}
and that's it! All you have to do then is send your modified \texttt{pst-circ.tex} to 
me and  it will become part of the official release of \CircPackage.

\textbf{Important:} Pay attention to the comment character \Verb+%+
at the end of lines. They are \emph{very} important in order to avoid spurious blanks.

\section{Acknowledgements}

We thank of course Manuel Luque for his original work on pst-circ and for his circuit
drawings: this wouldn't have been possible without him. As usual, Denis Girou gave us a 
precious hand with some dark tricks of \TeX{} and PSTricks. Jean-C\^ome Charpentier
wrote the outline of \cs{multidipole} (a story about riri, fifi and loulou\dots).
Thanks also to Douglas Waud.

\end{document}
