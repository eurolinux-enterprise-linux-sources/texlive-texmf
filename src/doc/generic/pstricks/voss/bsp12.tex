%%%%%%%%%%%%%%%%%%%%%%%%%%%%%%%%%%%%%%%%%%%%%%%%%%%%%%%%%%%%%%%%%%%%%%%%%%%
%% bsp12.tex ---   beoetigt Donald Arseneau's shapepar.sty ,
%% Autor           : Herbert Voss <voss@perce.de> (nach einer Idee von
%%                   Thomas Siegel)
%% Datum           : 2004-04-25 
%%%%%%%%%%%%%%%%%%%%%%%%%%%%%%%%%%%%%%%%%%%%%%%%%%%%%%%%%%%%%%%%%%%%%%%%%%%
\documentclass[ngerman]{article}
\usepackage[T1]{fontenc}
\usepackage[latin1]{inputenc}
\usepackage{lmodern}          
\usepackage{shapepar}
\usepackage{pstricks}
\usepackage{babel}

\newsavebox\PBox
\pagestyle{empty}

\begin{document}

\savebox\PBox{
\begin{minipage}{\linewidth}
\shapepar{\nutshape}
Warum ist die Natur so nahezu symmetrisch? Niemand hat
eine Idee, warum. Das einzige, was wir vorschlagen k�nnten, ist etwas wie
dies: Es gibt ein Tor in Japan, ein Tor in Neiko, das die Japaner manchmal
das sch�nste Tor in ganz Japan nennen; es wurde zu einer Zeit gebaut, als
die chinesische Kunst gro�en Einflu� hatte. Dies Tor ist sehr kunstvoll
gearbeitet mit zahlreichen Giebeln und sch�nen Schnitzereien und vielen
S�ulen und Drachenk�pfen und F�rsten, die in die S�ulen eingemei�elt sind,
usw. Aber wenn man genau hinschaut, sieht man, da� in dem kunstvollen und
komplizierten Muster entlang einer der S�ulen eines der kleinen
Musterelemente kopf�ber gemei�elt ist; sonst ist alles vollst�ndig
symmetrisch. Wenn man danach fragt, warum dies so ist, ist die Erkl�rung,
da� es kopf�ber gemei�elt worden ist, damit die G�tter auf die
Vollkommenheit des Menschen nicht eifers�chtig sind. So machten sie
absichtlich einen Fehler, damit die G�tter nicht eifers�chtig und auf die
Menschen zornig sein w�rden. Wir k�nnten den Gedanken umdrehen und daran
denken, da� die wahre Erkl�rung der n�herungsweisen Symmetrie der Natur
diese ist: Gott machte die Gesetze nur ungef�hr symmetrisch, damit wir nicht
auf seine Vollkommenheit eifers�chtig sind!

\end{minipage}
}

\begin{center}
\begin{pspicture}(-6,-5.5)(6,5.5)
 \rput(0,0){\usebox\PBox}
  \pscustom*{%
   \psarc(0,1.1){1.1}{270}{450}
   \psarc(0,0){2.2}{90}{270}
   \psarcn(0,-1.1){1.1}{270}{450}
  }
  \pscircle*(0,-1.1){0.3}
  \pscircle[fillstyle=solid](0,1.1){0.3}
%  \psarc(4,4){2.4}{270}{450}
\end{pspicture}

\vspace{2cm}
\textit{Richard Feynman}
\end{center}

\end{document}

