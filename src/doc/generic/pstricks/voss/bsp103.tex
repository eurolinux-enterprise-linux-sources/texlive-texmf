%%%%%%%%%%%%%%%%%%%%%%%%%%%%%%%%%%%%%%%%%%%%%%%%%%%%%%%%%%%%%%%%%%%%%%%%%%%
%% bsp103.tex ---
%% Autor           : Herbert Voss <voss@perce.de>
%% Datum           : 2004-04-25 
%%%%%%%%%%%%%%%%%%%%%%%%%%%%%%%%%%%%%%%%%%%%%%%%%%%%%%%%%%%%%%%%%%%%%%%%%%%
\documentclass[12pt]{article}
\usepackage{pstricks}
\pagestyle{empty}

\begin{document}

\begin{pspicture}(-0.5\linewidth,-4)(0.5\linewidth,4)
  \begin{psclip}{\psellipse[linecolor=blue,doubleline=true](.5\linewidth,4)}
    \rput(0,0){\Huge\red\sf%
       \parbox{\linewidth}{%
,,\texttt{PSTricks} -- mehr als nur ein alter Hut``{}, war ein Vortrag auf der DANTE Tagung
in Darmstadt betitelt. Er sollte den Teilnehmern vor Augen f�hren, dass
\texttt{PSTricks}, als eines der ersten f�r Plain \TeX{} entwickelten Pakete nichts an seiner
Aktualit�t und vor allem Professionalit�t verloren hat. Die Qualit�t der Grafikausgabe,
die mit \texttt{PSTricks} erreicht werden kann, sucht sicherlich ihresgleichen. Dabei darf
nicht vergessen werden, dass alles seine Grenzen hat, so auch \texttt{PSTricks} mit seinen
vielf�ltigen Paketen, denn die Grafiken m�ssen komplett in \TeX- und somit in 
       }}
  \end{psclip}
\end{pspicture}


\end{document}
