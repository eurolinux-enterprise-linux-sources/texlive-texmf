\documentclass[]{article}
\usepackage{pstricks}
\usepackage{pst-vue3d}
\usepackage{longtable}
\pagestyle{empty}
\parindent=0pt

\begin{document}
\bgroup
\def\arraystretch{0.5}
\begin{longtable}{p{0.4\linewidth}p{0.4\linewidth}}
\begin{pspicture}(-1.5,-0.75)(2.5,2.25)
	\AxesThreeD(20)
	\pNodeThreeD(15,18,15){P}
	\uput[0](P){P}
	\psset{linecolor=red}
	\qdisk(P){3pt}
	\showCoorThreeD[linecolor=red](15,18,15)
\end{pspicture}
&
\begin{pspicture}(-1.5,-0.75)(2.5,2.25)
	\AxesThreeD(20)
	\FrameThreeD[linecolor=red](5,10,10)(10,0)(0,10)
	\showCoorThreeD[linecolor=red](5,10,10)
\end{pspicture}
\tabularnewline

\begin{pspicture}(-1.5,-0.75)(2.5,2.25)
	\AxesThreeD(20)
	\psset{normaleLongitude=0,normaleLatitude=90}
	\FrameThreeD(0,0,0)(0,15)(-15,0)
	\QuadrillageThreeD[linewidth=0.1mm,grille=5](0,0,0)(-15,0)(0,15)%
\end{pspicture}
&
\begin{pspicture}(-1.5,-0.75)(2.5,2.25)
	\AxesThreeD(20)
	\CircleThreeD[linecolor=red](5,10,10){5}
	\showCoorThreeD[linecolor=red](5,10,10)
\end{pspicture}
\tabularnewline
%
\begin{pspicture}(-1.5,-0.5)(2.5,2)
	\AxesThreeD(20,10,15)
	\bgroup
	\psset{CubeColorFaceOne= 1 1 0 ,
		CubeColorFaceTwo= 0.9 0.9 0 ,
		CubeColorFaceThree= 0.8 0.8 0 ,
		CubeColorFaceFour= 0.7 0.7 0 ,
		CubeColorFaceFive= 0.6 0.6 0 ,
		CubeColorFaceSix= 0.5 0.5 0 }
	\CubeThreeD[A=5,B=7.5,C=5](5,7.5,5)
	\egroup
\end{pspicture}
&
\begin{pspicture}(-1.5,-0.5)(2.5,2)
	\AxesThreeD(20,10,15)
	\bgroup
	\psset{CubeColorFaceOne= 1 1 0 ,
		CubeColorFaceTwo= 0.9 0.9 0 ,
		CubeColorFaceThree= 0.8 0.8 0 ,
		CubeColorFaceFour= 0.7 0.7 0 ,
		CubeColorFaceFive= 0.6 0.6 0 ,
		CubeColorFaceSix= 0.5 0.5 0 }
	\DieThreeD[A=5,B=5,C=5](5,5,5)
	\egroup
\end{pspicture}
\tabularnewline
%
\begin{pspicture}(-1.5,-0.5)(2.5,2.25)
	\AxesThreeD(20)
	\psset{A=7.5,B=7.5,C=7.5}
	\PyramideThreeD[fillstyle=solid](7.5,7.5,0){18}
\end{pspicture}
&

\begin{pspicture}(-1.5,-0.5)(2.5,2.25)
	\AxesThreeD(20)
	\psset{A=7.5,B=7.5,C=7.5}
	\PyramideThreeD[fillstyle=solid,fracHeight=0.5](7.5,7.5,0){18}
\end{pspicture}
\tabularnewline

\begin{pspicture}(-1.5,-0.75)(2.5,2.25)
	\AxesThreeD(20)
	\TetraedreThreeD[Rtetraedre=12,fillcolor=blue,fillstyle=solid](12,12,0)
\end{pspicture}
&
\begin{pspicture}(-1.5,-0.75)(2.5,2.25)
	\AxesThreeD(20)
	\DodecahedronThreeD[fillstyle=solid,fillcolor=cyan,linewidth=0.1pt](7.5,10,5)
\end{pspicture}
\tabularnewline
%
\begin{pspicture}(-1.5,-0.5)(2.5,2.25)
	\AxesThreeD(20)
	\ConeThreeD[fillstyle=solid,fillcolor=cyan,linewidth=0.1pt](7.5,10,0){7.5}{18}
\end{pspicture}
&
\begin{pspicture}(-1.5,-0.5)(2.5,2.25)
	\AxesThreeD(20)
	\ConeThreeD[fillstyle=solid,fillcolor=yellow,%
		fracHeight=0.5,linewidth=0.1pt](7.5,10,0){7.5}{18}
\end{pspicture}
\tabularnewline
%
\begin{pspicture}(-1.5,-0.75)(2.5,2.25)
	\AxesThreeD(20)
	\CylindreThreeD[fillstyle=solid,fillcolor=cyan,%
		linewidth=0.1pt](7.5,10,0){7.5}{18}
\end{pspicture}
&

\begin{pspicture}(-1.5,-0.75)(2.5,2.25)
	\AxesThreeD(20)
	\SphereThreeD[fillstyle=solid,fillcolor=yellow,%
		linewidth=0.1pt](7.5,7.5,7.5){10}
\end{pspicture}
\tabularnewline

\begin{pspicture}(-1.5,-0.75)(2.5,2.25)
	\AxesThreeD(20)
	\SphereInverseThreeD[fillstyle=solid,fillcolor=yellow,%
		linewidth=0.1pt](7.5,7.5,7.5){10}
\end{pspicture}
&
\begin{pspicture}(-1.5,-0.75)(2.5,2.25)
	\AxesThreeD(20)
	\SphereCercleThreeD[linecolor=red,linewidth=0.1pt](7.5,7.5,7.5){10}
\end{pspicture}
\tabularnewline

\begin{pspicture}(-1.5,-0.75)(2.5,2.25)
	\AxesThreeD(20)
	\SphereMeridienThreeD[linecolor=red,linewidth=0.1pt](7.5,7.5,7.5){10}
\end{pspicture}
&
\begin{pspicture}(-1.5,-0.75)(2.5,2.25)
	\AxesThreeD(20)
	\psset{PHI=20}
	\DemiSphereThreeD[fillstyle=solid,fillcolor=yellow,%
		linewidth=0.1pt](7.5,7.5,7){10}
\end{pspicture}
\tabularnewline
\begin{pspicture}(-1.5,-0.75)(2.5,2)
	\AxesThreeD(20,20,15)
	\psset{PHI=20}
	\DemiSphereThreeD[fillstyle=solid,fillcolor=yellow,%
		linewidth=0.1pt,RotX=180](7.5,7.5,10){10}
\end{pspicture}
&
\begin{pspicture}(-1.5,-0.75)(2.5,2)
	\AxesThreeD(20,10,15)
	\psset{PHI=20}
	\DemiSphereThreeD[fillstyle=solid,fillcolor=cyan,%
		linewidth=0.1pt,RotX=180](7.5,7.5,10){10}
	\SphereCreuseThreeD[fillstyle=solid,fillcolor=cyan,%
		linewidth=0.1pt,RotX=180](7.5,7.5,10){10}
\end{pspicture}
\tabularnewline
\begin{pspicture}(-1.5,-1)(2.5,2.25)
	\AxesThreeD(20)
	\psset{PortionSphereTHETA=60,PortionSpherePHI=45}
	\SphereThreeD[fillstyle=solid,fillcolor=cyan,%
		linewidth=0.1pt](0,0,0){10}
	\PortionSphereThreeD[fillstyle=solid,fillcolor=red](0,0,0){10}
\end{pspicture}
\tabularnewline[15pt]
\end{longtable}
\egroup

\end{document}
