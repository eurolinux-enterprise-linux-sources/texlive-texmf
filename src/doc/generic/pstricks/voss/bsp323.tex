\documentclass[a4paper]{article}% 
\usepackage{pstricks,pst-node,pst-grad,pst-math,multido}
\SpecialCoor 

\parindent0pt
\pagestyle{empty}
\begin{document}
\centering
\psframebox[fillcolor=black!45,fillstyle=solid]{%
\begin{pspicture*}(-6,0)(7,7.5)     % dieser code wird irgendwann nach
\pnode(0,0){O}                      % nach pst-optic wandern
\pnode(!
    /AnglePrisme 30 def
    /AnglePlan1 19 def
    /AnglePlan2 54 def
% le point C1 sur la droite 1
    /C1x -8 def
    /C1y 7 def
% le point C2 sur la droite 2
    /C2x 11 def
    /C2y 5 def
% donne la distance C1E1
    /u 1.5 def
%
    /g1x AnglePrisme sin neg def % -sin(A/2)
    /g1y AnglePrisme cos def     %  cos(A/2)
    /u1x AnglePlan1 sin neg def
    /u1y AnglePlan1 cos neg def
% le point E �metteur
    /E1x C1x u u1x mul add def
    /E1y C1y u u1y mul add def
    %
    /n1x AnglePlan1 cos def
    /n1y AnglePlan1 sin neg def
    /Lambda {E1x g1y mul E1y g1x mul neg add
             n1y g1x mul neg n1x g1y mul add
             div neg} bind def
% point I1
    /i1x {E1x Lambda n1x mul add} bind def
    /i1y {E1y Lambda n1y mul add} bind def
    0 0){Stockage_parametres_prisme}
\pspolygon[fillstyle=gradient,gradbegin=cyan,gradend=white,gradangle=60,gradmidpoint=0.5](O)%
        (! 7 90 AnglePrisme add cos mul 7 90 AnglePrisme add sin mul)
        (! 7 90 AnglePrisme sub cos mul 7 90 AnglePrisme sub sin mul)
\multido{\iLAMBDA=400+3}{115}{%
  \pstVerb{
    /lambda \iLAMBDA\space def
  }%
\definecolor{prisme}{wave}{\iLAMBDA}%
\pnode(!
% Les datas
% Sellmeier's
% glass sf15 : verre flint lourd
% n=sqrt(1+B1*L^2/(l^2-C1)+B2*L^2/(l^2-C2)+B3*L^2/(l^2-C3))
% Cauchy : /N {1.606 6545 1 mul lambda dup mul div add} bind def
    /L2 {lambda 1e-3 mul dup mul} bind def
    /N {1
        1.539259 L2 mul L2 0.011931 sub div
         add
         0.247621 L2 mul L2 0.055608 sub div
         add
         1.038164 L2 mul L2 116.416755 sub div
         add
         sqrt} bind def
    /alpha1 AnglePlan1 AnglePrisme add def
    /sinB1 alpha1 sin N div def
    /B1 sinB1 ASIN RadtoDeg def
    /Delta1 AnglePrisme B1 sub def
%%%
    /g2x AnglePrisme sin def
    /g2y AnglePrisme cos def
    /d12x Delta1 cos def % d12x
    /d12y Delta1 sin def % d12y
    /Lambda2 {i1x g2y mul i1y g2x mul sub
            d12y g2x mul d12x g2y mul sub
            div} bind def
% point I2
    /i2x {i1x Lambda2 d12x mul add} bind def
    /i2y {i1y Lambda2 d12y mul add} bind def
%
    /B2  AnglePrisme 2 mul B1 sub def
    /sinA2 N B2 sin mul def
    /alpha2 sinA2 ASIN RadtoDeg def
    /u2x AnglePlan2 sin def
    /u2y AnglePlan2 cos neg def
    /Delta2 alpha2 AnglePrisme sub def
    /d2x Delta2 cos def
    /d2y Delta2 sin def
    /s2x i2x C2x sub def
    /s2y i2y C2y sub def
    /dA d2x u2y mul d2y u2x mul sub def
    /dM d2x s2y mul d2y s2x mul sub def
% le point R2
    /r2x C2x dM dA div u2x mul add def
    /r2y C2y dM dA div u2y mul add def
0 0){factice}
\pnode(! C1x C1y){C1}
\pnode(! C2x C2y){C2}
\pnode(! E1x E1y){E1}
\pnode(! i1x i1y){I1}
\pnode(! i2x i2y){I2}
\pnode(! r2x r2y){R2}
\psline[linecolor=prisme](I1)(I2)(R2)}
\psline[linecolor=white,linewidth=0.5mm](E1)(I1)
\psline[linecolor=white,linewidth=0.5mm,arrowscale=2]{->}(E1)(!i1x E1x add 2 div i1y E1y add 2 div)
\end{pspicture*}}

\end{document}
