%%%%%%%%%%%%%%%%%%%%%%%%%%%%%%%%%%%%%%%%%%%%%%%%%%%%%%%%%%%%%%%%%%%%%%%%%%%
%% bsp108.tex ---
%% Autor           : Herbert Voss <voss@perce.de>
%% Datum           : 2004-04-25 
%%%%%%%%%%%%%%%%%%%%%%%%%%%%%%%%%%%%%%%%%%%%%%%%%%%%%%%%%%%%%%%%%%%%%%%%%%%
\documentclass[12pt]{article}
\usepackage{pstricks}
\usepackage{pst-char}
\usepackage{pst-grad}
\usepackage{graphicx}

\parindent=0pt
\pagestyle{empty}

\begin{document}

\begin{center}
\bgroup
\DeclareFixedFont{\RM}{T1}{ptm}{b}{n}{3cm}
\DeclareFixedFont{\Rm}{T1}{ptm}{m}{n}{2mm}
%\psset{shadow=true,blur=true,shadowsize=10pt,blurradius=5pt}

\psset{fillstyle=gradient,gradbegin=red,gradend=white}
\begin{pspicture}(\linewidth,3cm)
\resizebox{\linewidth}{!}{\pscharpath[gradangle=90]{\RM PostScript}}
\end{pspicture}\\
\begin{pspicture}(\linewidth,3cm)
\resizebox{\linewidth}{!}{\pscharpath[gradangle=0]{\RM PostScript}}
\end{pspicture}\\
\begin{pspicture}(\linewidth,3cm)
\resizebox{\linewidth}{!}{\pscharpath[gradmidpoint=0,gradangle=90]{\RM PostScript}}
\end{pspicture}
\egroup
%\figcaption{\texttt{bsp108.tex}: Schattenspiele \ldots}

\end{center}

\end{document}
