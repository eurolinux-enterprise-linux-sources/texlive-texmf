\documentclass{article}% bsp300.tex
\listfiles
\usepackage[dvipsnames,prologue]{pstricks}
\usepackage{pst-labo}

\begin{document}

\sffamily
\begin{pspicture}(0,-4)(7,4)
  \rput(3,0){\pstDosage[niveauReactifBurette=25,niveauLiquide1=30,%
    aspectLiquide1=Champagne,glassType=becher,phmetre,unit=0.5]}
  \rput(2,3){B\"urette}   
  \rput(4.7,3.6){25 mL}
  \rput(5.2,-2.2){H$_3$O$^+$+Cl$^-$}
  \rput(.8,-3){PH-Messer} 
  \rput(5,-2.8){20 mL}
  \rput(5,1){Na$^+$+OH$^-$} \rput(6.4,-3.6){Heizplatte}
  \psline{->}(2.7,2.9)(3.4,2.9)
\end{pspicture}
\begin{pspicture}(-3,-2)(2,3)
  \psset{unit=0.5cm}
  \rput(-4.5,4.0){\pstEprouvette[tubePenche=-60,niveauLiquide1=90,niveauLiquide2=50]}
  \rput(.5,0){\pstEntonnoir[glassType=flacon,niveauLiquide1=30]}
  \rput(.5,7.5){
    \framebox{\begin{minipage}{3.2cm}Nach der Dekantation
    sind die einzelnen Phasen getrennt, das Leichteste sammelt man durch Filtrieren.
    \end{minipage}}}
\end{pspicture}

\end{document}