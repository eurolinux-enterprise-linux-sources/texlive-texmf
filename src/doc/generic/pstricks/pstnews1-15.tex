\documentclass[12pt]{article}
\usepackage[T1]{fontenc}
\usepackage[latin1]{inputenc}
%\usepackage{geometry}
\usepackage{pst-all}
\usepackage{showexpl}
\usepackage{tabularx,array,longtable}
\SpecialCoor
\usepackage[colorlinks,linktocpage]{hyperref}
\def\UrlFont{\small\ttfamily}
\makeatletter
\def\verbatim@font{\small\normalfont\ttfamily}
\makeatother
%\usepackage{color}
\definecolor{hellgelb}{rgb}{1,1,0.8}
\definecolor{colKeys}{rgb}{0,0,1}
\definecolor{colIdentifier}{rgb}{0,0,0}
\definecolor{colComments}{rgb}{1,0,0}
\definecolor{colString}{rgb}{0,0.5,0}
%
\usepackage{listings}
\lstset{%
    language=PSTricks,%
    float=hbp,%
    basicstyle=\ttfamily\small, %
    identifierstyle=\color{colIdentifier}, %
    keywordstyle=\color{colKeys}, %
    stringstyle=\color{colString}, %
    commentstyle=\color{colComments}, %
    columns=flexible, %
    tabsize=4, %
    frame=single, %
    extendedchars=true, %
    showspaces=false, %
    showstringspaces=false, %
    numbers=left, %
    numberstyle=\tiny, %
    breaklines=true, %
%    backgroundcolor=\color{hellgelb}, %
    breakautoindent=true, %
    captionpos=b,%
	xleftmargin=0pt%
}

%\parindent=0pt
\newcommand\verbI[1]{{\small\texttt{#1}}}
\newcommand\CMD[1]{{\texttt{\textbackslash#1}}}
\let\Lcs\CMD
%
%\psset{subgriddiv=0,griddots=5,gridlabels=7pt}
%
\DeclareRobustCommand\cs[1]{\texttt{\char`\\#1}}
\def\PS{PostScript}
%
\begin{document}
\title{\texttt{PSTricks -- version 1.15}\\new macros and bugfixes for \texttt{pstricks}}
\author{Herbert Vo�\thanks{%
\url{Herbert.Voss@pstricks.de}}}
\date{\today}

\maketitle

\begin{abstract}
This new version of \texttt{pstricks.tex} depends on the also new prologue file
\texttt{pstricks.pro} (v 1.01) and pst-dots.pro (v. 2.00), which should go into 
the local \TeX-directory  \url{$TEXMFLOCAL/dvips/}.
\end{abstract}


\tableofcontents

\clearpage
%--------------------------------------------------------------------------------------
\section{linecap}
%--------------------------------------------------------------------------------------
This was already introduced with version 1.14, but I forgot to mentioned it.
For the line ends one can use the ,,arrows``{} \texttt{c-c}, but only for the ends. For dashed
lines the option \verb+linecap+ can be set to the value 1 to get the dashes with 
rounded ends, too.

\begin{pspicture}(5,3)
\psline[linestyle=dashed,linewidth=8pt,dash=5mm 5mm,linecap=2](0,2.5)(5,2.5)
\psline[linestyle=dashed,linewidth=8pt,dash=5mm 5mm,linecap=1](0,1.5)(5,1.5)
\psline[linestyle=dashed,linewidth=8pt,dash=5mm 5mm](0,0.5)(5,0.5)
\end{pspicture}

\begin{lstlisting}
\begin{pspicture}(5,3)
\psline[linestyle=dashed,linewidth=8pt,dash=5mm 5mm,linecap=2](0,2.5)(5,2.5)
\psline[linestyle=dashed,linewidth=8pt,dash=5mm 5mm,linecap=1](0,1.5)(5,1.5)
\psline[linestyle=dashed,linewidth=8pt,dash=5mm 5mm](0,0.5)(5,0.5)
\end{pspicture}
\end{lstlisting}

It is obvious that only the value 1 makes sense for the dashed lines.

\section{Dots}
Long time ago, Etienne Riga sent the code for some new dotstyles
which are now part of pstricks.

\psset{dotscale=1.5}
\let\Loption\texttt
\def\multiDot#1{\multido{\nA=0+1}{3}{\psdot[dotstyle=#1]\kern1em}}
\def\cmultiDot#1{\multido{\nA=0+1}{3}{\psdot*[dotstyle=#1]\kern1em}}%}
\begin{longtable}{@{}l@{\kern2em}>{\ttfamily}c>{\ttfamily}c|
		   l@{\kern2em}>{\ttfamily}c>{\ttfamily}c@{}}
\emph{Name} & \Lcs{psdot}    & \Lcs{psdot*} & 
\emph{Name} & \Lcs{psdot}    & \Lcs{psdot*} \\\hline
\endhead
\Loption{*}             & \multiDot{*}        & \cmultiDot{*} &%\\        
\Loption{o}             & \multiDot{o}         & \cmultiDot{o}\\
\Loption{Bo}            & \multiDot{Bo}        & \cmultiDot{Bo}&%\\
\Loption{x}             & \multiDot{x}        & \cmultiDot{x}\\
\Loption{+}             & \multiDot{+}        & \cmultiDot{+}&%\\
\Loption{B+}            & \multiDot{B+}        & \cmultiDot{B+}\\
\Loption{Add}        & \multiDot{Add}    & \cmultiDot{Add}&%\\
\Loption{BoldAdd}        & \multiDot{BoldAdd}    & \cmultiDot{BoldAdd}\\
\Loption{Oplus}      & \multiDot{Oplus}        & \cmultiDot{Oplus}&%\\
\Loption{BoldOplus}     & \multiDot{BoldOplus}        & \cmultiDot{BoldOplus}\\
\Loption{SolidOplus}     & \multiDot{SolidOplus}        & \cmultiDot{SolidOplus} &
\Loption{Hexagon}      & \multiDot{Hexagon}        & \cmultiDot{Hexagon}\\
\Loption{BoldHexagon}     & \multiDot{BoldHexagon}        & \cmultiDot{BoldHexagon}& 
\Loption{SolidHexagon}     & \multiDot{SolidHexagon}        & \cmultiDot{SolidHexagon}\\
\Loption{asterisk}      & \multiDot{asterisk}        & \cmultiDot{asterisk}&%\\
\Loption{Basterisk}     & \multiDot{Basterisk}        & \cmultiDot{Basterisk}\\
\Loption{Asterisk}      & \multiDot{Asterisk}        & \cmultiDot{Asterisk}&%\\
\Loption{BoldAsterisk}     & \multiDot{BoldAsterisk}        & \cmultiDot{BoldAsterisk}\\
\Loption{SolidAsterisk}     & \multiDot{SolidAsterisk}        & \cmultiDot{SolidAsterisk}&
\Loption{oplus}         & \multiDot{oplus}     & \cmultiDot{oplus}\\
\Loption{otimes}        & \multiDot{otimes}    & \cmultiDot{otimes}&
\Loption{Otimes}      & \multiDot{Otimes}        & \cmultiDot{Otimes}\\
\Loption{BoldOtimes}     & \multiDot{BoldOtimes}        & \cmultiDot{BoldOtimes}&
\Loption{SolidOtimes}     & \multiDot{SolidOtimes}        & \cmultiDot{SolidOtimes}\\
\Loption{Mul}        & \multiDot{Mul}    & \cmultiDot{Mul}&%\\
\Loption{BoldMul}        & \multiDot{BoldMul}    & \cmultiDot{BoldMul}\\
\Loption{|}              & \multiDot{|}    & \cmultiDot{|}&%\\
\Loption{B|}\iffalse\index{B*@\texttt{B}|}\fi & \multiDot{B|}  & \cmultiDot{B|}\\
\Loption{Bar}      & \multiDot{Bar}        & \cmultiDot{Bar}&%\\
\Loption{BoldBar}     & \multiDot{BoldBar}        & \cmultiDot{BoldBar}\\
\Loption{Bullet}        & \multiDot{Bullet}    & \cmultiDot{Bullet}&%\\
\Loption{Circle}        & \multiDot{Circle}    & \cmultiDot{Circle}\\
\Loption{BoldCircle}        & \multiDot{BoldCircle}    & \cmultiDot{BoldCircle}&%\\
\Loption{square}        & \multiDot{square}    & \cmultiDot{square}\\
\Loption{Bsquare}       & \multiDot{Bsquare}   & \cmultiDot{Bsquare} &
\Loption{square*}       & \multiDot{square*}    & \cmultiDot{square*}\\
\Loption{Square}      & \multiDot{Square}        & \cmultiDot{Square}&%\\
\Loption{BoldSquare}     & \multiDot{BoldSquare}        & \cmultiDot{BoldSquare}\\
\Loption{SolidSquare}     & \multiDot{SolidSquare}        & \cmultiDot{SolidSquare}&
\Loption{diamond}       & \multiDot{diamond}   & \cmultiDot{diamond}\\
\Loption{Bdiamond}      & \multiDot{Bdiamond}  & \cmultiDot{Bdiamond}&%\\
\Loption{diamond*}      & \multiDot{diamond*}    & \cmultiDot{diamond}\\
\Loption{Diamond}      & \multiDot{Diamond}        & \cmultiDot{Diamond}&%\\
\Loption{BoldDiamond}     & \multiDot{BoldDiamond}        & \cmultiDot{BoldDiamond}\\
\Loption{SolidDiamond}     & \multiDot{SolidDiamond}        & \cmultiDot{SolidDiamond}&
\Loption{triangle}      & \multiDot{triangle}  & \cmultiDot{triangle}\\
\Loption{Btriangle}     & \multiDot{Btriangle} & \cmultiDot{Btriangle}&
\Loption{triangle*}     & \multiDot{triangle*} & \cmultiDot{triangle*}\\
\Loption{Triangle}      & \multiDot{Triangle}        & \cmultiDot{Triangle}&%\\
\Loption{BoldTriangle}     & \multiDot{BoldTriangle}        & \cmultiDot{BoldTriangle}\\
\Loption{SolidTriangle}     & \multiDot{SolidTriangle}        & \cmultiDot{SolidTriangle}&
\Loption{pentagon}      & \multiDot{pentagon}  & \cmultiDot{pentagon}\\
\Loption{Bpentagon}     & \multiDot{Bpentagon} & \cmultiDot{Bpentagon}&%\\
\Loption{pentagon*}     & \multiDot{pentagon*}& \cmultiDot{pentagon*}\\
\Loption{Pentagon}      & \multiDot{Pentagon}        & \cmultiDot{Pentagon}&%\\
\Loption{BoldPentagon}     & \multiDot{BoldPentagon}        & \cmultiDot{BoldPentagon}\\
\Loption{SolidPentagon}     & \multiDot{SolidPentagon}        & \cmultiDot{SolidPentagon}&
\Loption{Hexagon}      & \multiDot{Hexagon}        & \cmultiDot{Hexagon}\\
\Loption{BoldHexagon}     & \multiDot{BoldHexagon}        & \cmultiDot{BoldHexagon}&
\Loption{SolidHexagon}     & \multiDot{SolidHexagon}        & \cmultiDot{SolidHexagon}
\end{longtable}


\section{\texttt{pstricks.pro}}
The PostScript header file defines the following functions and constants, which 
can be used
in arguments for \CMD{psplot} or any other macro, which reads PostScript code. 

\begin{lstlisting}[basicstyle=\ttfamily\footnotesize]
/Sqrt { dup 0 lt { pop 0 } { sqrt } ifelse } def% return 0 for negative arguments 
/Atan { /atan load stopped { pop pop 0 } if } def% return 0 if atan not known
/ATAN1 {neg -1 atan 180 sub } def		% atan(x) (only one parameter)
/Div { dup 0 eq { pop } { div } ifelse } def  	% control the division
/NET { neg exch neg exch T } def	      	% change coordinate system to the negative one		
/Pyth { dup mul exch dup mul add sqrt } def   	% Pythagoras, epects 2 parameter
/PtoC { 2 copy cos mul 3 1 roll sin mul } def 	% Polar to Cartesian
%----------------- hv added 20050516 ---------------
/Pi 3.14159265359 def 
/TwoPi 6.28318530718 def
/Euler 2.71828182846 def 
/RadtoDeg { 180 mul Pi div } bind def % convert from radian to degrees
/DegtoRad { Pi mul 180 div } bind def % viceversa
\end{lstlisting}

\section{Packages}
The package \texttt{pst-char} is now part of \texttt{pst-text} and the
package \texttt{pst-ghsb} is now part of \texttt{pst-grad}. The macros
and enviroments will work in the same way.

The documented source code of \texttt{pstricks} and \texttt{pst-node} is
now available as a pdf file with a source which can be run by pdf\LaTeX.
Rolf Niepraschk modified the old sources from Timothy Van Zandt for
running with \LaTeX\ and/or pdf\LaTeX.

\end{document}
