\documentclass[english]{article}
%
\usepackage[T1]{fontenc}
\usepackage[latin9]{inputenc}
\listfiles
\usepackage[scaled]{luximono}
\usepackage{lmodern}
\usepackage{xspace}
\usepackage[bottom]{footmisc}
\usepackage{tabularx}
\usepackage{longtable}
\usepackage[NewCommands,NewParameters]{ragged2e}
\usepackage[dvipsnames]{pstricks}
\usepackage{pst-node}% muss _nach_ pst-grad geladen werden
\usepackage{pst-coil}
\definecolor{hellgelb}{rgb}{1,1,0.8}
%
\def\PST{{\texttt{PSTricks}}\xspace}
\def\PDF{{\texttt{PDF}}\xspace}
\def\pst{{\texttt{pstricks}}\xspace}
\def\PS{PostScript\xspace}
\newcommand*\CMD[1]{{\UrlFont\texttt{\textbackslash #1}}}
%
\def\tIndex#1{\index{#1@{\UrlFont\texttt{#1}}}}
\def\cIndex#1{\index{#1@\CMD{#1}}}
\def\pIndex#1{\index{Parameter@\textbf{Parameter}!{\UrlFont\texttt{#1}}}}
\def\ppIndex#1{\index{Parameter@\textbf{Parameter}!{#1}}}
\def\sIndex#1{\index{Syntax@\textbf{Syntax}!\CMD{#1}}}
\def\csIndex#1{\sIndex{#1}\cIndex{#1}}
\def\PIndex#1{\index{Paket@\textbf{Paket}!\texttt{#1}}}
\def\mIndex#1{\texttt{#1}\tIndex{#1}\pIndex{#1}}
%
\pretolerance=500
\tolerance=1000 
\hbadness=3000
\vbadness=3000
\hyphenpenalty=400

\usepackage{showexpl}% not a real PSTricks package
\usepackage{babel}
\usepackage{makeidx}
\makeindex
\usepackage[dvips,colorlinks,linktocpage]{hyperref} % PDF-support
%
\renewcommand{\ttdefault}{ul9}% Luxi Mono
\lstset{keywordstyle=\small\fontfamily{ul9}\bfseries} 
\lstset{language=PSTricks,moredelim=**[is][\bf\color{blue}]{�}{�}}% oder andere Begrenzer
%

\begin{document}
%
\title{\texttt{pst-coil}:\newline Coils, springs and zigzag lines}
\author{Herbert Vo�\thanks{Thanks to Eugene Ressler for translating this documentation!}}
\maketitle

\begin{abstract}
\index{Paket@\textbf{Paket}!\texttt{pst-coil}}Coils, springs and zigzag curves
do not really count to the ordinary when creating graphics. They are of some
importance though when dealing with the creation of coilish connections. They
can also be used for node connections which will be shown
at the end of this chapter.

The parameters especially for the \verb+coil+ variants are only easy to
understand when keeping in mind that three dimensional helixes are dealt with,
which are only projected onto the two dimensional paper plane. This will be
elaborated on at the corresponding places.
\end{abstract}

\tableofcontents

\clearpage
% ---------------------------------------------------------------------------------------
\section{Parameters}\label{sec:pstcoil:parameter}
% ---------------------------------------------------------------------------------------
Table~\ref{tab:pst-coil:parameter} shows a compilation of the special parameters
valid for \verb+pst-coil+.

\begin{longtable}{@{}>{\ttfamily}l>{\ttfamily}l>{\ttfamily}l@{}}
\caption{Compilation of all parameters for  \texttt{pst-coil}}\label{tab:pst-coil:parameter}\\[-5pt]
\textrm{name} & \textrm{value}  & \textrm{default}\\\hline
\endfirsthead
\textrm{name} & \textrm{value}  & \textrm{default}\\\hline
\endhead
coilwidth  &  <value[unit]> & 1cm\\
coilheight  &  <value> & 1\\
coilarm  &  <value[unit]> & 0.5cm\\
coilarmA  &  <value[unit]> & 0.5cm\\
coilarmB  &  <value[unit]> & 0.5cm\\
coilaspect  &  <angle> & 45\\
coilinc  &  <angle> & 45\\
\end{longtable}
\tIndex{coilwidth}\tIndex{coilheight}\tIndex{coilarm}\tIndex{coilaspect}\tIndex{coilinc}%
\pIndex{coilwidth}\pIndex{coilheight}\pIndex{coilarm}\pIndex{coilaspect}\pIndex{coilinc}%

% ---------------------------------------------------------------------------------------
\subsection{\texttt{coilwidth}}\label{subsec:pstcoil:coilwidth}
% ---------------------------------------------------------------------------------------
\tIndex{coilwidth}\verb+coilwidth+ denotes the diameter\index{diameter} of the
coil resp. the height of a zigzag line\index{zigzag line}\index{coil}. The
diameter corresponds to the height with a perpendicular parallel projection
transverse to the length.

\medskip\noindent
\begin{minipage}{5cm}
\begin{pspicture}(0,-5)(5,5)
\pscoil(0,4)(3.5,4)
{
\psset{linewidth=0.2pt}
\psline[arrowscale=2,tbarsize=3mm]{|<->|}(4.5,3.5)(4.5,4.5)
\psline[linestyle=dashed](3.25,4.5)(4.5,4.5)
\psline[linestyle=dashed](2.75,3.5)(4.5,3.5)
}
\uput*[0](3.5,4){coilwidth}
\pscoil[coilwidth=0.75cm](0,2.6)(3.6,2.6)
\uput*[0](3.5,2.6){0.75cm}
\pscoil[coilwidth=0.5cm](0,1.8)(3.5,1.8)
\uput*[0](3.5,1.8){0.50cm}
\pscoil[coilwidth=0.25cm](0,1.2)(3.5,1.2)
\uput*[0](3.5,1.2){0.25cm}
\pscoil*[coilwidth=0.5cm](0,0.5)(3.5,0.5)
\uput*[0](3.5,0.5){0.50cm}
% zigzag
\pszigzag(0,-4)(3.5,-4)
{
\psset{linewidth=0.2pt}
\psline[arrowscale=2,tbarsize=3mm]{|<->|}(4.5,-3.5)(4.5,-4.5)
\psline[linestyle=dashed](3.25,-4.5)(4.5,-4.5)
\psline[linestyle=dashed](2.75,-3.5)(4.5,-3.5)
}
\uput*[0](3.5,-4){coilwidth}
\pszigzag[coilwidth=0.75cm](0,-2.6)(3.6,-2.6)
\uput*[0](3.5,-2.6){0.75cm}
\pszigzag[coilwidth=0.5cm](0,-1.8)(3.5,-1.8)
\uput*[0](3.5,-1.8){0.50cm}
\pszigzag[coilwidth=0.25cm](0,-1.2)(3.5,-1.2)
\uput*[0](3.5,-1.2){0.25cm}
\pszigzag*[coilwidth=0.5cm](0,-0.5)(3.5,-0.5)
\uput*[0](3.5,-0.5){0.50cm}
\end{pspicture}
\end{minipage}\hfill%
\begin{minipage}{\linewidth-5cm-1em}
\begin{lstlisting}
\begin{pspicture}(0,-5)(5,5)
\pscoil(0,4)(3.5,4)
{\psset{linewidth=0.2pt}
\psline[arrowscale=2,tbarsize=3mm]{|<->|}(4.5,3.5)(4.5,4.5)
\psline[linestyle=dashed](3.25,4.5)(4.5,4.5)
\psline[linestyle=dashed](2.75,3.5)(4.5,3.5)}
\uput*[0](3.5,4){coilwidth}
\pscoil[coilwidth=0.75cm](0,2.6)(3.6,2.6)
\uput*[0](3.5,2.6){0.75cm}
\pscoil[coilwidth=0.5cm](0,1.8)(3.5,1.8)
\uput*[0](3.5,1.8){0.50cm}
\pscoil[coilwidth=0.25cm](0,1.2)(3.5,1.2)
\uput*[0](3.5,1.2){0.25cm}
\pscoil*[coilwidth=0.5cm](0,0.5)(3.5,0.5)
\uput*[0](3.5,0.5){0.50cm}
% zigzag
\pszigzag(0,-4)(3.5,-4)

[ ... ]
\end{lstlisting}
\end{minipage}



% ---------------------------------------------------------------------------------------
\subsection{\texttt{coilheight}}\label{subsec:pstcoil:coilheight}
% ---------------------------------------------------------------------------------------
\tIndex{coilheight}\verb+coilheight+ is in contrast to \verb+coilwidth+ no
length but only a factor, whereat the distance between two
windings\index{winding} resp. jags is composed as follows:
\begin{equation}dx = coilheight\cdot coilwidth \end{equation}

As can be seen in the following example, not the same physical distance is
yielded. This is caused by the internal three dimensional
representation\index{three dimensional representation} of the coil; it is not
seen under an angle of $90$� but $45$�. (section~\ref{subsec:pstcoil:coilaspect})

\medskip\noindent
\begin{LTXexample}[width=5cm]
\begin{pspicture}(0,-4)(5,5)
\pscoil(0,4)(3.5,4)
{\psset{linewidth=0.2pt}
\psline[arrowscale=2,tbarsize=3mm]{|<->|}(4.5,3.5)(4.5,4.5)
\psline[linestyle=dashed](3.25,4.5)(4.5,4.5)
\psline[linestyle=dashed](2.75,3.5)(4.5,3.5)
\psline[arrowscale=2,tbarsize=3mm]{|<->|}(1.75,3.2)(2.5,3.2)
\uput[-90](2.15,3){\footnotesize\texttt{coilheight$\times$coilwidth}}}
\uput*[0](3.5,4){coilwidth}
\pscoil[coilheight=0.75](0,1.5)(3.6,1.5)
\uput*[0](3.5,1.5){0.75}
\pscoil[coilheight=0.5](0,0)(3.5,0)
\uput*[0](3.5,0){0.50}
\pscoil[coilheight=1.25](0,-1.5)(3.5,-1.5)
\uput*[0](3.5,-1.5){1.25}
\pscoil*[coilheight=0.5](0,-3)(3.5,-3)
\uput*[0](3.5,-3){0.50}
\end{pspicture}
\end{LTXexample}


\medskip\noindent
\begin{LTXexample}[width=5cm]
\begin{pspicture}(0,-4)(5,5)
\pszigzag(0,4)(3.5,4)
{ \psset{linewidth=0.2pt}
\psline[arrowscale=2,tbarsize=3mm]{|<->|}(4.5,3.5)(4.5,4.5)
\psline[linestyle=dashed](2.5,4.5)(4.5,4.5)
\psline[linestyle=dashed](2.75,3.5)(4.5,3.5)
\psline[arrowscale=2,tbarsize=3mm]{|<->|}(1.75,3.2)(2.75,3.2)
\uput[-90](2.15,3){\footnotesize\texttt{coilheight$\times$coilwidth}} }
\uput*[0](3.5,4){coilwidth}
\pszigzag[coilheight=0.75](0,1.5)(3.6,1.5)
\uput*[0](3.5,1.5){0.75}
\pszigzag[coilheight=0.5](0,0)(3.5,0)
\uput*[0](3.5,0){0.50}
\pszigzag[coilheight=1.25](0,-1.5)(3.5,-1.5)
\uput*[0](3.5,-1.5){1.25}
\pszigzag*[coilheight=0.5](0,-3)(3.5,-3)
\uput*[0](3.5,-3){0.50}
\end{pspicture}
\end{LTXexample}



% ---------------------------------------------------------------------------------------
\subsection{\texttt{coilarm}, \texttt{coilarmA} and \texttt{coilarmB}}\label{subsec:pstcoil:coilarm}
% ---------------------------------------------------------------------------------------
\tIndex{coilarm}\tIndex{coilarmA}\tIndex{coilarmB}\verb+coilarm+ denotes the
part of a straight line made left and right. Negative values are possible, but
do not make a lot of sense as a rule. As can be gathered from the example, the
coil is lenghtened by a negative value in this case, so that the part of a
straight line is led back to the specified starting point resp. endpoint and
consequently has a negative ``direction''{}. This parameter is not available for
\CMD{psCoil}.

\medskip\noindent
\begin{LTXexample}[width=4cm]
\begin{pspicture}(4,4.5)
\pscoil(0,4)(4,4)
\psset{coilwidth=0.5}
\pscoil[coilarm=0](0,3)(4,3)
\pscoil[coilarmA=1cm,coilarmB=0.2cm](0,2)(4,2)
\pscoil[coilarm=-10pt](0,1)(4,1)
\end{pspicture}
\end{LTXexample}

\medskip\noindent
\begin{LTXexample}[width=4cm]
\begin{pspicture}(4,4.5)
\pszigzag(0,4)(4,4)
\psset{coilwidth=0.5}
\pszigzag[coilarm=0](0,3)(4,3)
\pszigzag[coilarmA=1cm,coilarmB=0.2cm](0,2)(4,2)
\pszigzag[coilarm=-10pt](0,1)(4,1)
\end{pspicture}
\end{LTXexample}



% ---------------------------------------------------------------------------------------
\subsection{\texttt{coilaspect}}\label{subsec:pstcoil:coilaspect}
% ---------------------------------------------------------------------------------------
\tIndex{coilaspect}In section~\ref{subsec:pstcoil:coilheight} the three
dimensional representation of the coil has been pointed out already. If it was
regarded right-angled to its axis, the windings would not be recognizable. With
\verb+coilaspect+ this ``perspective''{} can be influenced, which is only
possible for the ``coil''{} variants of course. \verb+coilaspect=0+ yields a
sine curve.

\medskip\noindent
\begin{LTXexample}[width=4cm]
\begin{pspicture}(4,4.5)
\pscoil(0,4)(4,4)
\psset{coilwidth=0.75}
\pscoil[coilaspect=0](0,3)(4,3)
\pscoil[coilaspect=30,coilheight=0.3](0,2)(4,2)
\pscoil[coilaspect=-30,coilheight=0.3](0,1)(4,1)
\end{pspicture}
\end{LTXexample}


% ---------------------------------------------------------------------------------------
\subsection{\texttt{coilinc}}\label{subsec:pstcoil:coilinc}
% ---------------------------------------------------------------------------------------
\tIndex{coilinc}The curve is drawn with the \verb+lineto+ procedure of \PS,
whereat \verb+coilinc+ specifies the rotation angle in the angular measure at
which the next point is calculated. It has already been mentioned that the
calculation is done as a three dimensional coil and is projected onto the two
dimensional plane only in the end.

\verb+coilinc+ does also not make sense for zigzag lines, so that this parameter
again only is available for the \verb+coil+ variant. Large angles result in a
polygon line, small angles in harmonic progressions with increased calculation
effort.

\medskip\noindent
\begin{LTXexample}[width=4cm]
\begin{pspicture}(4,4.5)
\psset{coilwidth=0.8}
\pscoil(0,4)(4,4)
\psset{coilinc=0}
\pscoil[coilinc=5](0,3)(4,3)
\pscoil[coilinc=30](0,2)(4,2)
\pscoil[coilinc=60](0,1)(4,1)
\end{pspicture}
\end{LTXexample}


% ---------------------------------------------------------------------------------------
\section{Makros}\label{sec:pstcoil:makros}
% ---------------------------------------------------------------------------------------
\verb+pst-coil+ defines three macros, which make the creation of coil or zigzag
shaped lines possible.\index{zigzag line}
\begin{verbatim}
\pscoil[<parameters>](<x1,y1>)
\pscoil[<parameters>](<x0,y0>)(<x1,y1>)
\pscoil[<parameters>]{<arrows>}(<x0,y0>)(<x1,y1>)
\pscoil*[<parameters>](<x1,y1>)
\pscoil*[<parameters>](<x0,y0>)(<x1,y1>)
\pscoil*[<parameters>]{<arrows>}(<x0,y0>)(<x1,y1>)
\psCoil[<parameters>]{<angle1>}{<angle2>}
\psCoil*[<parameters>]{<angle1>}{<angle2>}
\pszigzag[<parameters>](<x1,y1>)
\pszigzag[<parameters>](<x0,y0>)(<x1,y1>)
\pszigzag[<parameters>]{<arrows>}(<x0,y0>)(<x1,y1>)
\pszigzag*[<parameters>](<x1,y1>)
\pszigzag*[<parameters>](<x0,y0>)(<x1,y1>)
\pszigzag*[<parameters>]{<arrows>}(<x0,y0>)(<x1,y1>)
\end{verbatim}
\csIndex{pscoil}\csIndex{psCoil}\csIndex{pszigzag}

\begin{itemize}
\item If only one coordinate pair is specified, the first point is automatically
	set to the origin of ordinates $(0,0)$.
\item Arrows may be set with the own parameter or with the optional parameter
	with \verb+arrows=<arrows>+.
\item For \verb+psCoil+ the coils are principally drawn without ``arms''{}.
\end{itemize}


The asterisk version are in principle pointless, since they do not allow a
meaningful representation. Some could be seen with the description of
parameters, but will not be described further in the following.


% ---------------------------------------------------------------------------------------
\subsection{\CMD{pscoil}}\label{subsec:pstcoil:pscoil}
% ---------------------------------------------------------------------------------------
Since \verb+pscoil+ is nothing but a polygon line, also all parameters are
available which aplly for lines.

\medskip\noindent
\begin{LTXexample}[width=4cm]
\begin{pspicture}(4,4)
\pscoil[coilarm=.5cm,linewidth=1.5pt,coilwidth=.5cm]{|->}(4,3)
\pscoil[linecolor=red,coilheight=0.25](3,4)(3,1)
\pscoil[doubleline=true,linecolor=cyan,coilheight=0.75](0,4)(3,0)
\end{pspicture}
\end{LTXexample}

% ---------------------------------------------------------------------------------------
\subsection{\CMD{psCoil}}\label{subsec:pstcoil:psCoil}
% ---------------------------------------------------------------------------------------
\verb+\psCoil+ draws starting at the current point to the angle $\alpha$ an
invisible line and then from $\alpha$ to $\beta$ a helix.
\verb+\rput+\cIndex{rput} has to be used if they are to be placed at a
particular position. The advantage is that with \verb+psCoil+ the number of
helixes to draw can be specified exactly.

\medskip\noindent
\begin{LTXexample}[width=4cm]
\begin{pspicture}(4,4)
\psCoil[linewidth=1.5pt,coilwidth=.5cm]{1800}{3600}
\rput{-90}(3,4){\psCoil[linecolor=red,coilheight=0.25]{0}{3600}}
\rput{-45}(0,3.5){\psCoil[doubleline=true,linecolor=cyan,coilheight=0.75]{100}{2700}}
\end{pspicture}
\end{LTXexample}



% ---------------------------------------------------------------------------------------
\subsection{\CMD{pszigzag}}\label{subsec:pstcoil:zigzag}
% ---------------------------------------------------------------------------------------
\cIndex{pszigzag}This macro represents the two dimensional variant and is
correspondly easy to use. Especially the \verb+linearc+ option can bring good
results for \verb+\pszigzag+. The specification of the ``arm length''{} is not
absolute here, since the line is principally ended in the geometrical middle and
then led to the specified arm length.

\medskip\noindent
\begin{LTXexample}[width=4cm]
\begin{pspicture}(4,4)
\pszigzag[coilarm=.5cm,linewidth=1.5pt,coilwidth=.5cm]{|->}(4,3)
\pszigzag[linecolor=red,coilheight=0.2](3,4)(3,1)
\pszigzag[doubleline=true,linecolor=cyan,coilheight=0.75](0,4)(3,0)
\end{pspicture}
\end{LTXexample}


% ---------------------------------------------------------------------------------------
\section{Node connections}\label{sec:pstcoil:knotenverbindungen}
% ---------------------------------------------------------------------------------------
Prerequisite for this macro is that \verb+pst-node+\tIndex{pst-node}
is loaded, then the following node connections are
available:
%
\begin{verbatim}
\nccoil[<parameters>]{<nodeA>}{<nodeB>}
\nccoil[<parameters>]{<arrows>}{<nodeA>}{<nodeB>}
\nccoil*[<parameters>]{<nodeA>}{<nodeB>}
\nccoil*[<parameters>]{<arrows>}{<nodeA>}{<nodeB>}
\nczigzag[<parameters>]{<nodeA>}{<nodeB>}
\nczigzag[<parameters>]{<arrows>}{<nodeA>}{<nodeB>}
\nczigzag*[<parameters>]{<nodeA>}{<nodeB>}
\nczigzag*[<parameters>]{<arrows>}{<nodeA>}{<nodeB>}
\pccoil[<parameters>](<nodeA>)(<nodeB>)
\pccoil[<parameters>]{<arrows>}(<nodeA>)(<nodeB>)
\pccoil*[<parameters>](<nodeA>)(<nodeB>)
\pccoil*[<parameters>]{<arrows>}(<nodeA>)(<nodeB>)
\pczigzag[<parameters>](<nodeA>)(<nodeB>)
\pczigzag[<parameters>]{<arrows>}(<nodeA>)(<nodeB>)
\pczigzag*[<parameters>](<nodeA>)(<nodeB>)
\pczigzag*[<parameters>]{<arrows>}(<nodeA>)(<nodeB>)
\end{verbatim}

These connections behave completely analogue to those dealt with the
package \verb+pst-node+ so that
no further explanations are needed here.

\medskip\noindent
\begin{LTXexample}[width=4cm]
\begin{pspicture}(4,3)
\cnode(0.5,0.5){0.5}{A}
\cnode[fillstyle=solid,fillcolor=lightgray](3.5,2.5){0.5}{B}
\nccoil[coilwidth=0.4,coilaspect=35,coilheight=0.5]{<->}{A}{B}
\end{pspicture}
\end{LTXexample}


\nocite{*}
\bgroup
\raggedright
\bibliographystyle{plain}
\bibliography{\jobname}
\egroup

\printindex


\end{document}
