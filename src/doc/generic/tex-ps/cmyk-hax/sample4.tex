% E  Sample TeX file demonstrating how to separate using CMYK-HAX macros
%    -- bitmap graphics
% P  Przyk/ladowy plik TeX-owy demonstruj/acy separacj/e kolor/ow
%    za pomoc/a makr pakietu CMYK-HAX -- grafika bitmapowa
%
\input epsf     % DVIPS standard distribution
\input colordvi % DVIPS standard distribution
\input cmyk-hax

\nopagenumbers
\hoffset-1in
\voffset-1in
\topskip 0pt
\hsize 210mm
\vsize 297mm

\def\epsfi{\epsfxsize 90mm \epsffile{screen.eps}}
\def\cbox#1{\vbox{\halign{\hss##\hss\cr#1\crcr}}}
\def\chbit{% E change some colors of bitmap
           % P podmiana niekt/orych kolor/ow mapy bitowej
    \forbitmap{23 19 1B 00}\changebitmap{00 00 00 60}
    \forbitmap{DD BC C1 E4}\changebitmap{00 00 00 FF}
    \forbitmap{2A 21 21 00}\changebitmap{00 00 00 E0}
    \forbitmap{69 54 54 01}\changebitmap{00 00 00 90}}
\def\separate#1{\PSbegingroup
  \setCMYKchange
    \forcolor\black\delcolor % E make black unvisible
                             % P uczy/n czarny niewidzialnym
    \chbit
  \useCMYKchange
  \projectCMYK#1%
  \cbox{\epsfi\cr\noalign{\vskip 2mm}\CMYKlabels\rm}
  \PSendgroup}
\def\sepblack{\PSbegingroup
  \setCMYKchange
    \chbit
  \useCMYKchange
  \projectCMYK\black
  \cbox{\epsfi\cr\noalign{\vskip 2mm}\CMYKlabels\rm}
  \PSendgroup}

\vglue 0pt plus 1fil
\centerline{\separate\cyan \hskip 5mm \separate\magenta }
\vskip 5mm
\centerline{\separate\yellow \hskip 5mm \sepblack }
\vglue 0pt plus 1fil

\eject\end
