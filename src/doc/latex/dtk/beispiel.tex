%%*****************************************************************************
%% $Id: beispiel.tex,v 1.4 2001/10/21 15:17:13 gene Exp $
%%*****************************************************************************
%% Author: Gerd Neugebauer
%%-----------------------------------------------------------------------------
%%
%% Dies ist ein Beispieldokument zu einem Artikel f�r "`Die TeXnische
%% Kom"odie"'. Dieses soll die Struktur aufzeigen, die solch ein Artikel
%% haben sollte. Au"serdem sind hier die speziellen Makros im Einsatz
%% zu sehen und es werden einige Regeln vermittelt, die es zu beachten
%% gilt.
%%
%% Diese Datei enth�lt nur den eigentlichen Artikel. Diese Datei kann
%% nicht alleine von LaTeX bearbeitet werden. Hierzu ist der Treiber
%% in der Datei beispiel.ltx notwendig. Die �bersetzung des Beitrags
%% geschieht mit
%%
%%	latex beispiel.ltx
%%
%%-----------------------------------------------------------------------------

%%-----------------------------------------------------------------------------
%% Die ben"otigten Pakete k"onnen hier noch einmal deklariert werden.
%% Diese werden hier gebraucht, da die Pr"aambel sich in der
%% Treiber-Datei befindet, die nicht zur Produktion der
%% Vereinszeitschrift eingesetzt wird.
% \NeedPackage{multicol}
\NeedPackage{booktabs}
%%
%% Das Paket german/ngerman wird automatisch geladen und sollte nicht
%% explizit angefordert werden. 
%%
%% Regel: Es sollten so wenig wie m"oglich zus"atzliche Pakete
%%        angefordert werden, da diese mit anderen Paketen in Konflikt
%%        geraten k"onnten, die in anderen Artikeln unbedingt
%%        ben"otigt werden.
%% Regel: Es sollten nicht unn"otig viele Kommandos oder Umgebungen
%%        definiert werden (aus demselben Grund). N"otigenfalls
%%        sollte die Definition im Artikel und nicht in der Pr"aambel
%%        geschehen. 
%% Regel: Es darf kein spezielles inputencoding verwendet werden. Die
%%        Eingabe muss reines ASCII sein.
%%        Bei Bedarf kann als letzter Schritt die Umwandlung der
%%        Sonderzeichen in pures ASCII erfolgen (z.B. mit recode)
%%

%%-----------------------------------------------------------------------------
%% Angabe des Titels der Arbeit.
\title{Der Titel, der auch so lang sein kann, dass er nicht mehr in die
  Kopfzeile passt}

%%-----------------------------------------------------------------------------
%% Angabe der Autoren. Mehrere Autoren sind durch \and getrennt.
%% Die Vornamen m"oglichst nicht abk"urzen.
\author{A. Utor\and J.E. Mand}

%%-----------------------------------------------------------------------------
%% Angabe der Adressen der Autoren. Jeder Autor ist mit einer eigenen
%% \address-Anweisung vertreten.
%% Das erste Argument enth"alt (Titel und) Vornamen.
%% Das zweite Argument enth"alt die restlichen Namensbestandteile
%% (hiernach wird sortiert).
%% Das dritte Argument enth"at die Adresse. Die Bestandteile der
%% Adresse sind durch \\ voneinander getrennt.
\address{A.}{Utor}{Eine Stra"se 11\\11111 Irgendwo}
\address{J.E.}{Mand}{Weiter Weg 22\\99999 Hinterm Berg}

%%-----------------------------------------------------------------------------
%% Hier wird der Titel gesetzt.
\maketitle

%%-----------------------------------------------------------------------------
%% Die Definition von lokalen Makros und Umgebungen erfolgt hier.
\newcommand{\bs}{\symbol{`\\}}

%%-----------------------------------------------------------------------------
%% Falls der Titel sehr lang ist, muss hier eine kurze Fassung f"ur die
%% Kopfzeile angegeben werden. Ansonsten kann diese Anweisung wegfallen.
\markboth{Der Kurztitel}{Der Kurztitel}

%%-----------------------------------------------------------------------------
\begin{abstract}
  Der Abstract enth"alt eine kurze Zusammenfassung der Problemstellung,
  die in dem Artikel behandelt wird. Hier soll der Leser einen Hinweis
  bekommen, ob der Artikel interessant ist, oder erst sp"ater gelesen
  werden kann.
\end{abstract}

\section{Einleitung}

In der Einleitung sollte das Problem motiviert werden, das in diesem
Artikel behandelt wird. Wir behandeln hier das Vorbereiten von
Artikeln f"ur \DTK. Normalerweise wird das etwas l"anger ausfallen.

\section{Jetzt geht's los}

Es k"onnen alle \LaTeX-Konstrukte verwendet werden, die im
\texttt{article}-Style definiert sind. Dabei sind allerdings einige
Einschr"ankungen zu beachten.

\begin{itemize}
\item Die Abschnitte sind nicht nummeriert. Deshalb macht es keinen
  Unterschied, ob \verb|\section| oder \verb|\section*| verwendet
  wird. Der Einfachheit halber sollte deshalb \verb|\section| benutzt
  werden. Analoges gilt f"ur Unterabschnitte etc.
\item Die Papiergr"o"se ist DIN A5. Das muss ber"ucksichtigt werden, wenn
  Figuren oder Tabellen entworfen werden.
\item Es d"urfen keine globalen Parameter umdefiniert werden, da dies
  andere Artikel st"oren k"onnte. Insbesondere ist die Schriftgr"o"se
  festgelegt.
\item Es sollten keine Randnotizen (\verb|\marginpar|) benutzt werden
  (Es ist einfach kein ausreichender Rand vorhanden).
\end{itemize}

Die Makros in der Tabelle~\ref{tab:logos} stehen zus"atzlich zur
Verf"ugung und sollten an den entsprechenden Stellen auch benutzt
werden.
\begin{table}[t]
  \begin{center}
    \caption{Logos in dtk.cls}\label{tab:logos}

    \begin{tabular}{ll}\toprule
      \emph{Marko}		& \emph{Ergebnis}\\\midrule
      \texttt{\bs AMSLaTeX}	& \AMSLaTeX	\\
      \texttt{\bs AMSTeX}	& \AMSTeX	\\
      \texttt{\bs AMS}		& \AMS		\\
      \texttt{\bs BibTeX}	& \BibTeX	\\
      \texttt{\bs dante}	& \dante	\\
      \texttt{\bs Dante}	& \Dante	\\
      \texttt{\bs DTK}		& \DTK		\\
      \texttt{\bs emTeX}	& \emTeX	\\
      \texttt{\bs eTeX}		& \eTeX		\\
      \texttt{\bs LAmSTeX}	& \LAmSTeX	\\
      \texttt{\bs LaTeXTeX}	& \LaTeXTeX	\\
      \texttt{\bs LaTeXe}	& \LaTeXe	\\
      \texttt{\bs MF}		& \MF		\\
      \texttt{\bs MP}		& \MP		\\
      \texttt{\bs MakeIndex}	& \MakeIndex	\\
      \texttt{\bs NTS}		& \NTS		\\
      \texttt{\bs PS}		& \PS		\\
      \texttt{\bs PiCTeX}	& \PiCTeX	\\
      \texttt{\bs PubliCTeX}	& \PubliCTeX	\\
      \texttt{\bs SliTeX}	& \SliTeX	\\
      \texttt{\bs TeXXeT}	& \TeXXeT	\\
      \texttt{\bs TeXeT}	& \TeXeT	\\\bottomrule
    \end{tabular}
  \end{center}
\end{table}

Beim Setzen von Verzeichnisnamen, E-Mail-Adressen, URLs und "ahnlichem
tritt oft das Problem auf, dass diese nicht getrennt werden k"onnen,
wenn sie in \verb|\texttt| gesetzt sind. Zu diesem Zweck wurde von
dieser Klasse das Makro \verb|\Path| bereitgestellt, das sein Argument
in der Nichtproportionalschrift setzt und die Trennung an gewissen
Stellen erlaubt. Damit ist es dann m"oglich, auch lange URLs wie die
folgende zu setzen:
\Path{http://www.gerd-neugebauer.de/software/TeX/}.

Bei dieser Gelegenheit sind die Tilde \verb|~| und der Unterstrich
\verb|_| keine aktiven Zeichen mehr und k"onnen damit direkt
eingegeben werden.

Leider ist \verb|\Path| fragil. Das hei"st, dass es in Argumenten von
anderen Makros unerwartete Resultate liefern kann und somit dort nicht
gebraucht werden sollte.


\section{Tabellen}

\begin{enumerate}
\item Tabellen sollten normalerweise als Floats ausgef"uhrt werden.
  Damit ist es f"ur \LaTeX{} m"oglich, einen geeigneten Ort f"ur deren
  Plazierung zu finden. In der Regel sollte als Plazierungsargument
  \verb|[tp]| angegeben werden.
\item Tabellen sollten eine Unterschrift (\verb|\caption|) tragen und
  "uber eine Referenz im laufenden Text verankert werden.
\item Auf "uberfl"ussigen Schmuck, wie Rahmen und Linien sollte
  verzichtet werden. Meistens kann man "uber angemessene Formatierung
  bessere Ergebnisse erreichen als durch zus"atzliche Linien.
\item Die mit \verb|\label| vergebenen Namen sind lokal f"ur jeden
  Artikel. Hier braucht keine R"ucksicht auf andere Artikel genommen zu
  werden. 
\end{enumerate}

Ein Beispiel ist in der Tabelle~\ref{tab:logos} zu sehen.


\section{Goldene Regeln}

In diesem Abschnitt werden Regeln f"ur den Satz deutscher Texte
gesammelt. Diese sind von allgemeinerem Interesse. Die Erfahrung hat
gezeigt, dass diese Regeln trotzdem nicht allgemein bekannt sind, oder
angewendet werden.

\begin{enumerate}
\item Es gilt die deutsche Rechtschreibung, wie sie im Duden
  festgelegt ist! Seit der Ausgabe 1/2000 wird die 21. Auf\/lage des
  Duden zu Grunde gelegt.
\item Die Satzzeichen Punkt, Komma, Semikolon, Ausrufezeichen und
  Fragezeichen folgen ohne Zwischenraum auf das davorstehende Wort.
\item Bei Abk"urzungen folgt nach einem inneren Punkt ein kleiner
  Zwischenraum:
  z.\,B. wird als \verb|z.\,B.| eingegeben.

  Am Anfang eines Satzes werden Abk"urzungen ausgeschrieben. Im
  Allgemeinen ist es ratsam, Abk�rzungen nur sehr sparsam einzusetzen.
  So kann man "`z.\,B."' immer durch "`beispielsweise"' ersetzen.
\item Hervorhebungen erfolgen mit \verb|\emph|. \verb|\textbf| ist f"ur
  Hervorhebungen \emph{nicht}\/ zu\-l"assig!
\item Die Gedankenstriche -- werden als \verb|--| eingegeben. Davor
  und danach ist ein Leerraum zu lassen.
\item Ein Bereich von Zahlen wird mit \verb|--| gebildet. Hierbei sind
  davor und danach \emph{keine}\/ Leerzeichen einzuf"ugen:
  21--34
\item Bindestriche zwischen Worten werden mit \verb|-| eingegeben.
  Davor und danach sind keine Leerzeichen zul"assig.
\end{enumerate}


\section{Literatur}

Literatur sollte mit \BibTeX{} verwaltet werden.

\begin{itemize}
\item Es git einen speziellen \BibTeX-Style \texttt{dtk.bst}, der mit
  der Klasse \verb|dtk.cls| verteilt wird. Dieser sollte verwendet
  werden. 
\item Die Literatur muss in einer \BibTeX-Datei gesammelt sein. Diese
  muss denselben Basisnamen haben wie das eigentliche Dokument. 
\item Alternativ kann die Literatur in dem Dokument direkt angegeben
  werden. In diesem Fall k"onnen die Beipiele am Ende dieses Dokuments
  oder in der Zeitschrift "`\DTK"' "ur das Aussehen dienen. Davon wird
  allerdings dringend abgeraten.
\end{itemize}

An dieser Stelle kommen einige Beispiele f"ur Literaturzitate.  Eine
Kurzbeschreibung der Klasse und deren Entstehung wurde in
\cite{neugebauer:klasse} ver"offentlicht.

Angefangen hat alles mit einigen B"uchern von Donald Knuth
\cite{knuth:texbook,knuth:mfbook}. Der Klassiker zu \LaTeX{} stammt
von Leslie Lamport \cite{lamport:latex.2}. Neuerdings gibt es den
"`Companion"' \cite{companion}. Schlie"slich sollen auch noch B"ucher auf
Deutsch nicht verschwiegen werden \cite{kopka:latex,kopka:latex2}.

Schlie"slich muss auch noch die \TeX-Directory-Structure
\cite{twg:tds} erw"ahnt werden, da das ein Beipiel f"ur eine Referenz
ins CTAN enth"alt.


\section{Zusammenfassung}

Hier sollte noch einmal ein Resum\'e gezogen werden. Auch ein Ausblick
oder eine Diskussion von offenen Problemen kann hier erfolgen.

Eine letzte Bitte sei hier noch angebracht. Das Dokument, das zur
Ver"offentlichung f"ur "`\DTK"' eingereicht wird, sollte m"oglichst
ohne Fehler durch \LaTeX{} bearbeitet werden k"onnen. Dazu z"ahlen
auch "`l"a"sliche S"unden"' wie overfull/underfull Boxes.

\bibliography{beispiel}

\endinput%%%%%%%%%%%%%%%%%%%%%%%%%%%%%%%%%%%%%%%%%%%%%%%%%%%%%%%%%%%%%%%%%%%%%%
%
% Local Variables: 
% mode: latex
% TeX-master: "beispiel.ltx"
% End: 
