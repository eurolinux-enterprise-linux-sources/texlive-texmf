%%
%% This is file `rcsinfo2html.tex',
%% generated with the docstrip utility.
%%
%% The original source files were:
%%
%% rcsinfo.dtx  (with options: `header,html')
%% 
%% IMPORTANT NOTICE:
%% 
%% For the copyright see the source file.
%% 
%% Any modified versions of this file must be renamed
%% with new filenames distinct from rcsinfo2html.tex.
%% 
%% For distribution of the original source see the terms
%% for copying and modification in the file rcsinfo.dtx.
%% 
%% This generated file may be distributed as long as the
%% original source files, as listed above, are part of the
%% same distribution. (The sources need not necessarily be
%% in the same archive or directory.)

%%%%%%%%%%%%%%%%%%%%%%%%%%%%%%%%%%%%%%%%%%%%%%%%%%%%%%%%%%%%%%%%%%%%%%%%%%%%%
%%
%% `rcsinfo' package to use with LaTeX2e.
%%
%% This package is used to extract the revision and file information provided
%% by the RCS revision control system.
%% A PERL-package supporting rcsinfo and LaTeX2HTML is provieded too.
%%
%% Copyright (C) 1995 Dr. Juergen Vollmer
%%                    Viktoriastrasse 15, D-76133 Karlsruhe, Germany
%%                    Juergen.Vollmer@informatik-vollmer.de
%% License:
%%   This program can be redistributed and/or modified under the terms
%%   of the LaTeX Project Public License Distributed from CTAN
%%   archives in directory macros/latex/base/lppl.txt; either
%%   version 1 of the License, or any later version.
%%
%% If you find this software useful, please send me a postcard.
%%
%% $Id: rcsinfo.dtx,v 1.7 2005/02/25 08:37:03 vollmer draft vollmer $
%%%%%%%%%%%%%%%%%%%%%%%%%%%%%%%%%%%%%%%%%%%%%%%%%%%%%%%%%%%%%%%%%%%%%%%%%%%%%

%% \CheckSum{397}
%% \CharacterTable
%%  {Upper-case    \A\B\C\D\E\F\G\H\I\J\K\L\M\N\O\P\Q\R\S\T\U\V\W\X\Y\Z
%%   Lower-case    \a\b\c\d\e\f\g\h\i\j\k\l\m\n\o\p\q\r\s\t\u\v\w\x\y\z
%%   Digits        \0\1\2\3\4\5\6\7\8\9
%%   Exclamation   \!     Double quote  \"     Hash (number) \#
%%   Dollar        \$     Percent       \%     Ampersand     \&
%%   Acute accent  \'     Left paren    \(     Right paren   \)
%%   Asterisk      \*     Plus          \+     Comma         \,
%%   Minus         \-     Point         \.     Solidus       \/
%%   Colon         \:     Semicolon     \;     Less than     \<
%%   Equals        \=     Greater than  \>     Question mark \?
%%   Commercial at \@     Left bracket  \[     Backslash     \\
%%   Right bracket \]     Circumflex    \^     Underscore    \_
%%   Grave accent  \`     Left brace    \{     Vertical bar  \|
%%   Right brace   \}     Tilde         \~}

\documentclass{article}
\usepackage{html}
\usepackage{rcsinfo}

\begin{htmlonly}
%%
%% This is file `rcsinfo2html.tex',
%% generated with the docstrip utility.
%%
%% The original source files were:
%%
%% rcsinfo.dtx  (with options: `header,html')
%% 
%% IMPORTANT NOTICE:
%% 
%% For the copyright see the source file.
%% 
%% Any modified versions of this file must be renamed
%% with new filenames distinct from rcsinfo2html.tex.
%% 
%% For distribution of the original source see the terms
%% for copying and modification in the file rcsinfo.dtx.
%% 
%% This generated file may be distributed as long as the
%% original source files, as listed above, are part of the
%% same distribution. (The sources need not necessarily be
%% in the same archive or directory.)

%%%%%%%%%%%%%%%%%%%%%%%%%%%%%%%%%%%%%%%%%%%%%%%%%%%%%%%%%%%%%%%%%%%%%%%%%%%%%
%%
%% `rcsinfo' package to use with LaTeX2e.
%%
%% This package is used to extract the revision and file information provided
%% by the RCS revision control system.
%% A PERL-package supporting rcsinfo and LaTeX2HTML is provieded too.
%%
%% Copyright (C) 1995 Dr. Juergen Vollmer
%%                    Viktoriastrasse 15, D-76133 Karlsruhe, Germany
%%                    Juergen.Vollmer@informatik-vollmer.de
%% License:
%%   This program can be redistributed and/or modified under the terms
%%   of the LaTeX Project Public License Distributed from CTAN
%%   archives in directory macros/latex/base/lppl.txt; either
%%   version 1 of the License, or any later version.
%%
%% If you find this software useful, please send me a postcard.
%%
%% $Id: rcsinfo.dtx,v 1.7 2005/02/25 08:37:03 vollmer draft vollmer $
%%%%%%%%%%%%%%%%%%%%%%%%%%%%%%%%%%%%%%%%%%%%%%%%%%%%%%%%%%%%%%%%%%%%%%%%%%%%%

%% \CheckSum{397}
%% \CharacterTable
%%  {Upper-case    \A\B\C\D\E\F\G\H\I\J\K\L\M\N\O\P\Q\R\S\T\U\V\W\X\Y\Z
%%   Lower-case    \a\b\c\d\e\f\g\h\i\j\k\l\m\n\o\p\q\r\s\t\u\v\w\x\y\z
%%   Digits        \0\1\2\3\4\5\6\7\8\9
%%   Exclamation   \!     Double quote  \"     Hash (number) \#
%%   Dollar        \$     Percent       \%     Ampersand     \&
%%   Acute accent  \'     Left paren    \(     Right paren   \)
%%   Asterisk      \*     Plus          \+     Comma         \,
%%   Minus         \-     Point         \.     Solidus       \/
%%   Colon         \:     Semicolon     \;     Less than     \<
%%   Equals        \=     Greater than  \>     Question mark \?
%%   Commercial at \@     Left bracket  \[     Backslash     \\
%%   Right bracket \]     Circumflex    \^     Underscore    \_
%%   Grave accent  \`     Left brace    \{     Vertical bar  \|
%%   Right brace   \}     Tilde         \~}

\documentclass{article}
\usepackage{html}
\usepackage{rcsinfo}

\begin{htmlonly}
%%
%% This is file `rcsinfo2html.tex',
%% generated with the docstrip utility.
%%
%% The original source files were:
%%
%% rcsinfo.dtx  (with options: `header,html')
%% 
%% IMPORTANT NOTICE:
%% 
%% For the copyright see the source file.
%% 
%% Any modified versions of this file must be renamed
%% with new filenames distinct from rcsinfo2html.tex.
%% 
%% For distribution of the original source see the terms
%% for copying and modification in the file rcsinfo.dtx.
%% 
%% This generated file may be distributed as long as the
%% original source files, as listed above, are part of the
%% same distribution. (The sources need not necessarily be
%% in the same archive or directory.)

%%%%%%%%%%%%%%%%%%%%%%%%%%%%%%%%%%%%%%%%%%%%%%%%%%%%%%%%%%%%%%%%%%%%%%%%%%%%%
%%
%% `rcsinfo' package to use with LaTeX2e.
%%
%% This package is used to extract the revision and file information provided
%% by the RCS revision control system.
%% A PERL-package supporting rcsinfo and LaTeX2HTML is provieded too.
%%
%% Copyright (C) 1995 Dr. Juergen Vollmer
%%                    Viktoriastrasse 15, D-76133 Karlsruhe, Germany
%%                    Juergen.Vollmer@informatik-vollmer.de
%% License:
%%   This program can be redistributed and/or modified under the terms
%%   of the LaTeX Project Public License Distributed from CTAN
%%   archives in directory macros/latex/base/lppl.txt; either
%%   version 1 of the License, or any later version.
%%
%% If you find this software useful, please send me a postcard.
%%
%% $Id: rcsinfo.dtx,v 1.7 2005/02/25 08:37:03 vollmer draft vollmer $
%%%%%%%%%%%%%%%%%%%%%%%%%%%%%%%%%%%%%%%%%%%%%%%%%%%%%%%%%%%%%%%%%%%%%%%%%%%%%

%% \CheckSum{397}
%% \CharacterTable
%%  {Upper-case    \A\B\C\D\E\F\G\H\I\J\K\L\M\N\O\P\Q\R\S\T\U\V\W\X\Y\Z
%%   Lower-case    \a\b\c\d\e\f\g\h\i\j\k\l\m\n\o\p\q\r\s\t\u\v\w\x\y\z
%%   Digits        \0\1\2\3\4\5\6\7\8\9
%%   Exclamation   \!     Double quote  \"     Hash (number) \#
%%   Dollar        \$     Percent       \%     Ampersand     \&
%%   Acute accent  \'     Left paren    \(     Right paren   \)
%%   Asterisk      \*     Plus          \+     Comma         \,
%%   Minus         \-     Point         \.     Solidus       \/
%%   Colon         \:     Semicolon     \;     Less than     \<
%%   Equals        \=     Greater than  \>     Question mark \?
%%   Commercial at \@     Left bracket  \[     Backslash     \\
%%   Right bracket \]     Circumflex    \^     Underscore    \_
%%   Grave accent  \`     Left brace    \{     Vertical bar  \|
%%   Right brace   \}     Tilde         \~}

\documentclass{article}
\usepackage{html}
\usepackage{rcsinfo}

\begin{htmlonly}
%%
%% This is file `rcsinfo2html.tex',
%% generated with the docstrip utility.
%%
%% The original source files were:
%%
%% rcsinfo.dtx  (with options: `header,html')
%% 
%% IMPORTANT NOTICE:
%% 
%% For the copyright see the source file.
%% 
%% Any modified versions of this file must be renamed
%% with new filenames distinct from rcsinfo2html.tex.
%% 
%% For distribution of the original source see the terms
%% for copying and modification in the file rcsinfo.dtx.
%% 
%% This generated file may be distributed as long as the
%% original source files, as listed above, are part of the
%% same distribution. (The sources need not necessarily be
%% in the same archive or directory.)

%%%%%%%%%%%%%%%%%%%%%%%%%%%%%%%%%%%%%%%%%%%%%%%%%%%%%%%%%%%%%%%%%%%%%%%%%%%%%
%%
%% `rcsinfo' package to use with LaTeX2e.
%%
%% This package is used to extract the revision and file information provided
%% by the RCS revision control system.
%% A PERL-package supporting rcsinfo and LaTeX2HTML is provieded too.
%%
%% Copyright (C) 1995 Dr. Juergen Vollmer
%%                    Viktoriastrasse 15, D-76133 Karlsruhe, Germany
%%                    Juergen.Vollmer@informatik-vollmer.de
%% License:
%%   This program can be redistributed and/or modified under the terms
%%   of the LaTeX Project Public License Distributed from CTAN
%%   archives in directory macros/latex/base/lppl.txt; either
%%   version 1 of the License, or any later version.
%%
%% If you find this software useful, please send me a postcard.
%%
%% $Id: rcsinfo.dtx,v 1.7 2005/02/25 08:37:03 vollmer draft vollmer $
%%%%%%%%%%%%%%%%%%%%%%%%%%%%%%%%%%%%%%%%%%%%%%%%%%%%%%%%%%%%%%%%%%%%%%%%%%%%%

%% \CheckSum{397}
%% \CharacterTable
%%  {Upper-case    \A\B\C\D\E\F\G\H\I\J\K\L\M\N\O\P\Q\R\S\T\U\V\W\X\Y\Z
%%   Lower-case    \a\b\c\d\e\f\g\h\i\j\k\l\m\n\o\p\q\r\s\t\u\v\w\x\y\z
%%   Digits        \0\1\2\3\4\5\6\7\8\9
%%   Exclamation   \!     Double quote  \"     Hash (number) \#
%%   Dollar        \$     Percent       \%     Ampersand     \&
%%   Acute accent  \'     Left paren    \(     Right paren   \)
%%   Asterisk      \*     Plus          \+     Comma         \,
%%   Minus         \-     Point         \.     Solidus       \/
%%   Colon         \:     Semicolon     \;     Less than     \<
%%   Equals        \=     Greater than  \>     Question mark \?
%%   Commercial at \@     Left bracket  \[     Backslash     \\
%%   Right bracket \]     Circumflex    \^     Underscore    \_
%%   Grave accent  \`     Left brace    \{     Vertical bar  \|
%%   Right brace   \}     Tilde         \~}

\documentclass{article}
\usepackage{html}
\usepackage{rcsinfo}

\begin{htmlonly}
\input{rcsinfo2html.aux}
\end{htmlonly}

\newcommand{\LatexToHtml}{\LaTeX 2\texttt{HTML}}
\newcommand{\Var}[1]{\texttt{\$rcsinfo::#1}}
\newcommand{\RCS}{\emph{RCS}}

\pagestyle{fancy}

\begin{document}
\rcsInfo $Id: rcsinfo.dtx,v 1.7 2005/02/25 08:37:03 vollmer draft vollmer $

\author{Dr.~J{\"u}rgen Vollmer\\
        Viktoriastra{\ss}e 15, D-76133 Karlruhe, Germany\\
        \small{\texttt{Juergen.Vollmer@informaktik-vollmer.de}}}
\date{\today}
\title{Example for converting a \LaTeX\ document to \texttt{HTML}\\
       using \LatexToHtml\ and the \texttt{rcsinfo}-package}
\maketitle

\section{Notes}
%%%%%%%%%%%%%%%

\begin{itemize}
\item Use at least version \emph{99.1 release (March 30, 1999)} of
      \LatexToHtml.

\item You should \verb|\usepackage{html}|.

\item You must run \LaTeX\ on the input file before running \LatexToHtml.

\item In order to make \LatexToHtml\ read the \verb|.aux| file, you must
      either:
      \begin{itemize}
      \item call \LatexToHtml\ with the option
                 \verb|-show_section_numbers| or
      \item include the \verb|.aux| file explicitly in your \LaTeX-source, by
            adding the lines in the preamble:\\
                \verb|\begin{|\verb|htmlonly}|\\  % looks ulgy, but otherwise
                \verb|\input{|\emph{source}\verb|.aux}|\\
                \verb|\end{|\verb|htmlonly}|\\    % latex2html fails.
            which requires \verb|\usepackage{html}|.
      \end{itemize}
      If you don't do so, the \verb|\rcsInfo...| commands give no value.

\item If using the \LatexToHtml\ tool, only the last \verb|\rcsInfo| takes
      effect, e.g.~if using several input files each having the \verb|\rcsInfo|
      command, only the vales of the last included file are used.
      (If this is a serious problem for you, drop me a mail).

\item If the \verb|fancy| / \verb|fancyhdr| option is given to the
      \verb|rcsinfo| package,
      the date of the \verb|ADDRESS| at the bottom of the \texttt{HTML}
      page is set to the \RCS\ date.

\item If you would like to set your own \verb|ADDRESS| text, you should
      define a procedure, which sets \texttt{perl} \verb|$main::ADDRESS|
      variable in your local \texttt{.latex2html} initialization file.
      The routine may use the \texttt{perl}
      variables shown below. Assign a reference of this procdure to the
      \texttt{perl}-variable
      \verb|$rcsinfo::SetAddressProc|.
      E.g.\ if you have a \texttt{perl} routine \verb|&my_address_field|,
      then \\
      \verb|$rcsinfo::SetAddressProc = &my_address_field|.

      For example the follownig \texttt{perl}-code does the job:
\begin{verbatim}
sub my_address_field
{
 $ADDRESS = '<hr>This file was last updated on ' .
            $rcsinfo::Date  .
            '&nbsp; by ' .
            '<a href="mailto: Juergen.Vollmer@informatik-vollmer.de">' .
            'Dr. Juergen Vollmer ' .
            '&lt;Juergen.Vollmer@informatik-vollmer.de&gt;</a>';
}
$rcsinfo::SetAddressProc = \&my_address_field;
\end{verbatim}
\end{itemize}

\newpage

\section{Example}
%%%%%%%%%%%%%%%%%

\label{sec-examples}
Here are the commands and their output:

\begin{tabular}{lll}
\LaTeX-command              & \texttt{perl} variable & Result for this file \\\hline
\verb|\today|               & \texttt{\$today}       & \today           \\
\verb|\rcsInfoFile|         & \Var{File}             & \rcsInfoFile     \\
\verb|\rcsInfoOwner|        & \Var{Owner}            & \rcsInfoOwner    \\
\verb|\rcsInfoStatus|       & \Var{Status}           & \rcsInfoStatus   \\
\verb|\rcsInfoLocker|       & \Var{Locker}           & \rcsInfoLocker   \\
\verb|\rcsInfoDate|         & \Var{Date}             & \rcsInfoDate     \\
\verb|\rcsInfoLongDate|     & \Var{LongDate}         & \rcsInfoLongDate \\
\verb|\rcsInfoDay|          & \Var{Day}              & \rcsInfoDay      \\
\verb|\rcsInfoMonth|        & \Var{Month}            & \rcsInfoMonth    \\
\verb|\rcsInfoYear|         & \Var{Year}             & \rcsInfoYear     \\
\end{tabular}

\end{document}
\endinput
%%
%% End of file `rcsinfo2html.tex'.

\end{htmlonly}

\newcommand{\LatexToHtml}{\LaTeX 2\texttt{HTML}}
\newcommand{\Var}[1]{\texttt{\$rcsinfo::#1}}
\newcommand{\RCS}{\emph{RCS}}

\pagestyle{fancy}

\begin{document}
\rcsInfo $Id: rcsinfo.dtx,v 1.7 2005/02/25 08:37:03 vollmer draft vollmer $

\author{Dr.~J{\"u}rgen Vollmer\\
        Viktoriastra{\ss}e 15, D-76133 Karlruhe, Germany\\
        \small{\texttt{Juergen.Vollmer@informaktik-vollmer.de}}}
\date{\today}
\title{Example for converting a \LaTeX\ document to \texttt{HTML}\\
       using \LatexToHtml\ and the \texttt{rcsinfo}-package}
\maketitle

\section{Notes}
%%%%%%%%%%%%%%%

\begin{itemize}
\item Use at least version \emph{99.1 release (March 30, 1999)} of
      \LatexToHtml.

\item You should \verb|\usepackage{html}|.

\item You must run \LaTeX\ on the input file before running \LatexToHtml.

\item In order to make \LatexToHtml\ read the \verb|.aux| file, you must
      either:
      \begin{itemize}
      \item call \LatexToHtml\ with the option
                 \verb|-show_section_numbers| or
      \item include the \verb|.aux| file explicitly in your \LaTeX-source, by
            adding the lines in the preamble:\\
                \verb|\begin{|\verb|htmlonly}|\\  % looks ulgy, but otherwise
                \verb|\input{|\emph{source}\verb|.aux}|\\
                \verb|\end{|\verb|htmlonly}|\\    % latex2html fails.
            which requires \verb|\usepackage{html}|.
      \end{itemize}
      If you don't do so, the \verb|\rcsInfo...| commands give no value.

\item If using the \LatexToHtml\ tool, only the last \verb|\rcsInfo| takes
      effect, e.g.~if using several input files each having the \verb|\rcsInfo|
      command, only the vales of the last included file are used.
      (If this is a serious problem for you, drop me a mail).

\item If the \verb|fancy| / \verb|fancyhdr| option is given to the
      \verb|rcsinfo| package,
      the date of the \verb|ADDRESS| at the bottom of the \texttt{HTML}
      page is set to the \RCS\ date.

\item If you would like to set your own \verb|ADDRESS| text, you should
      define a procedure, which sets \texttt{perl} \verb|$main::ADDRESS|
      variable in your local \texttt{.latex2html} initialization file.
      The routine may use the \texttt{perl}
      variables shown below. Assign a reference of this procdure to the
      \texttt{perl}-variable
      \verb|$rcsinfo::SetAddressProc|.
      E.g.\ if you have a \texttt{perl} routine \verb|&my_address_field|,
      then \\
      \verb|$rcsinfo::SetAddressProc = &my_address_field|.

      For example the follownig \texttt{perl}-code does the job:
\begin{verbatim}
sub my_address_field
{
 $ADDRESS = '<hr>This file was last updated on ' .
            $rcsinfo::Date  .
            '&nbsp; by ' .
            '<a href="mailto: Juergen.Vollmer@informatik-vollmer.de">' .
            'Dr. Juergen Vollmer ' .
            '&lt;Juergen.Vollmer@informatik-vollmer.de&gt;</a>';
}
$rcsinfo::SetAddressProc = \&my_address_field;
\end{verbatim}
\end{itemize}

\newpage

\section{Example}
%%%%%%%%%%%%%%%%%

\label{sec-examples}
Here are the commands and their output:

\begin{tabular}{lll}
\LaTeX-command              & \texttt{perl} variable & Result for this file \\\hline
\verb|\today|               & \texttt{\$today}       & \today           \\
\verb|\rcsInfoFile|         & \Var{File}             & \rcsInfoFile     \\
\verb|\rcsInfoOwner|        & \Var{Owner}            & \rcsInfoOwner    \\
\verb|\rcsInfoStatus|       & \Var{Status}           & \rcsInfoStatus   \\
\verb|\rcsInfoLocker|       & \Var{Locker}           & \rcsInfoLocker   \\
\verb|\rcsInfoDate|         & \Var{Date}             & \rcsInfoDate     \\
\verb|\rcsInfoLongDate|     & \Var{LongDate}         & \rcsInfoLongDate \\
\verb|\rcsInfoDay|          & \Var{Day}              & \rcsInfoDay      \\
\verb|\rcsInfoMonth|        & \Var{Month}            & \rcsInfoMonth    \\
\verb|\rcsInfoYear|         & \Var{Year}             & \rcsInfoYear     \\
\end{tabular}

\end{document}
\endinput
%%
%% End of file `rcsinfo2html.tex'.

\end{htmlonly}

\newcommand{\LatexToHtml}{\LaTeX 2\texttt{HTML}}
\newcommand{\Var}[1]{\texttt{\$rcsinfo::#1}}
\newcommand{\RCS}{\emph{RCS}}

\pagestyle{fancy}

\begin{document}
\rcsInfo $Id: rcsinfo.dtx,v 1.7 2005/02/25 08:37:03 vollmer draft vollmer $

\author{Dr.~J{\"u}rgen Vollmer\\
        Viktoriastra{\ss}e 15, D-76133 Karlruhe, Germany\\
        \small{\texttt{Juergen.Vollmer@informaktik-vollmer.de}}}
\date{\today}
\title{Example for converting a \LaTeX\ document to \texttt{HTML}\\
       using \LatexToHtml\ and the \texttt{rcsinfo}-package}
\maketitle

\section{Notes}
%%%%%%%%%%%%%%%

\begin{itemize}
\item Use at least version \emph{99.1 release (March 30, 1999)} of
      \LatexToHtml.

\item You should \verb|\usepackage{html}|.

\item You must run \LaTeX\ on the input file before running \LatexToHtml.

\item In order to make \LatexToHtml\ read the \verb|.aux| file, you must
      either:
      \begin{itemize}
      \item call \LatexToHtml\ with the option
                 \verb|-show_section_numbers| or
      \item include the \verb|.aux| file explicitly in your \LaTeX-source, by
            adding the lines in the preamble:\\
                \verb|\begin{|\verb|htmlonly}|\\  % looks ulgy, but otherwise
                \verb|\input{|\emph{source}\verb|.aux}|\\
                \verb|\end{|\verb|htmlonly}|\\    % latex2html fails.
            which requires \verb|\usepackage{html}|.
      \end{itemize}
      If you don't do so, the \verb|\rcsInfo...| commands give no value.

\item If using the \LatexToHtml\ tool, only the last \verb|\rcsInfo| takes
      effect, e.g.~if using several input files each having the \verb|\rcsInfo|
      command, only the vales of the last included file are used.
      (If this is a serious problem for you, drop me a mail).

\item If the \verb|fancy| / \verb|fancyhdr| option is given to the
      \verb|rcsinfo| package,
      the date of the \verb|ADDRESS| at the bottom of the \texttt{HTML}
      page is set to the \RCS\ date.

\item If you would like to set your own \verb|ADDRESS| text, you should
      define a procedure, which sets \texttt{perl} \verb|$main::ADDRESS|
      variable in your local \texttt{.latex2html} initialization file.
      The routine may use the \texttt{perl}
      variables shown below. Assign a reference of this procdure to the
      \texttt{perl}-variable
      \verb|$rcsinfo::SetAddressProc|.
      E.g.\ if you have a \texttt{perl} routine \verb|&my_address_field|,
      then \\
      \verb|$rcsinfo::SetAddressProc = &my_address_field|.

      For example the follownig \texttt{perl}-code does the job:
\begin{verbatim}
sub my_address_field
{
 $ADDRESS = '<hr>This file was last updated on ' .
            $rcsinfo::Date  .
            '&nbsp; by ' .
            '<a href="mailto: Juergen.Vollmer@informatik-vollmer.de">' .
            'Dr. Juergen Vollmer ' .
            '&lt;Juergen.Vollmer@informatik-vollmer.de&gt;</a>';
}
$rcsinfo::SetAddressProc = \&my_address_field;
\end{verbatim}
\end{itemize}

\newpage

\section{Example}
%%%%%%%%%%%%%%%%%

\label{sec-examples}
Here are the commands and their output:

\begin{tabular}{lll}
\LaTeX-command              & \texttt{perl} variable & Result for this file \\\hline
\verb|\today|               & \texttt{\$today}       & \today           \\
\verb|\rcsInfoFile|         & \Var{File}             & \rcsInfoFile     \\
\verb|\rcsInfoOwner|        & \Var{Owner}            & \rcsInfoOwner    \\
\verb|\rcsInfoStatus|       & \Var{Status}           & \rcsInfoStatus   \\
\verb|\rcsInfoLocker|       & \Var{Locker}           & \rcsInfoLocker   \\
\verb|\rcsInfoDate|         & \Var{Date}             & \rcsInfoDate     \\
\verb|\rcsInfoLongDate|     & \Var{LongDate}         & \rcsInfoLongDate \\
\verb|\rcsInfoDay|          & \Var{Day}              & \rcsInfoDay      \\
\verb|\rcsInfoMonth|        & \Var{Month}            & \rcsInfoMonth    \\
\verb|\rcsInfoYear|         & \Var{Year}             & \rcsInfoYear     \\
\end{tabular}

\end{document}
\endinput
%%
%% End of file `rcsinfo2html.tex'.

\end{htmlonly}

\newcommand{\LatexToHtml}{\LaTeX 2\texttt{HTML}}
\newcommand{\Var}[1]{\texttt{\$rcsinfo::#1}}
\newcommand{\RCS}{\emph{RCS}}

\pagestyle{fancy}

\begin{document}
\rcsInfo $Id: rcsinfo.dtx,v 1.7 2005/02/25 08:37:03 vollmer draft vollmer $

\author{Dr.~J{\"u}rgen Vollmer\\
        Viktoriastra{\ss}e 15, D-76133 Karlruhe, Germany\\
        \small{\texttt{Juergen.Vollmer@informaktik-vollmer.de}}}
\date{\today}
\title{Example for converting a \LaTeX\ document to \texttt{HTML}\\
       using \LatexToHtml\ and the \texttt{rcsinfo}-package}
\maketitle

\section{Notes}
%%%%%%%%%%%%%%%

\begin{itemize}
\item Use at least version \emph{99.1 release (March 30, 1999)} of
      \LatexToHtml.

\item You should \verb|\usepackage{html}|.

\item You must run \LaTeX\ on the input file before running \LatexToHtml.

\item In order to make \LatexToHtml\ read the \verb|.aux| file, you must
      either:
      \begin{itemize}
      \item call \LatexToHtml\ with the option
                 \verb|-show_section_numbers| or
      \item include the \verb|.aux| file explicitly in your \LaTeX-source, by
            adding the lines in the preamble:\\
                \verb|\begin{|\verb|htmlonly}|\\  % looks ulgy, but otherwise
                \verb|\input{|\emph{source}\verb|.aux}|\\
                \verb|\end{|\verb|htmlonly}|\\    % latex2html fails.
            which requires \verb|\usepackage{html}|.
      \end{itemize}
      If you don't do so, the \verb|\rcsInfo...| commands give no value.

\item If using the \LatexToHtml\ tool, only the last \verb|\rcsInfo| takes
      effect, e.g.~if using several input files each having the \verb|\rcsInfo|
      command, only the vales of the last included file are used.
      (If this is a serious problem for you, drop me a mail).

\item If the \verb|fancy| / \verb|fancyhdr| option is given to the
      \verb|rcsinfo| package,
      the date of the \verb|ADDRESS| at the bottom of the \texttt{HTML}
      page is set to the \RCS\ date.

\item If you would like to set your own \verb|ADDRESS| text, you should
      define a procedure, which sets \texttt{perl} \verb|$main::ADDRESS|
      variable in your local \texttt{.latex2html} initialization file.
      The routine may use the \texttt{perl}
      variables shown below. Assign a reference of this procdure to the
      \texttt{perl}-variable
      \verb|$rcsinfo::SetAddressProc|.
      E.g.\ if you have a \texttt{perl} routine \verb|&my_address_field|,
      then \\
      \verb|$rcsinfo::SetAddressProc = &my_address_field|.

      For example the follownig \texttt{perl}-code does the job:
\begin{verbatim}
sub my_address_field
{
 $ADDRESS = '<hr>This file was last updated on ' .
            $rcsinfo::Date  .
            '&nbsp; by ' .
            '<a href="mailto: Juergen.Vollmer@informatik-vollmer.de">' .
            'Dr. Juergen Vollmer ' .
            '&lt;Juergen.Vollmer@informatik-vollmer.de&gt;</a>';
}
$rcsinfo::SetAddressProc = \&my_address_field;
\end{verbatim}
\end{itemize}

\newpage

\section{Example}
%%%%%%%%%%%%%%%%%

\label{sec-examples}
Here are the commands and their output:

\begin{tabular}{lll}
\LaTeX-command              & \texttt{perl} variable & Result for this file \\\hline
\verb|\today|               & \texttt{\$today}       & \today           \\
\verb|\rcsInfoFile|         & \Var{File}             & \rcsInfoFile     \\
\verb|\rcsInfoOwner|        & \Var{Owner}            & \rcsInfoOwner    \\
\verb|\rcsInfoStatus|       & \Var{Status}           & \rcsInfoStatus   \\
\verb|\rcsInfoLocker|       & \Var{Locker}           & \rcsInfoLocker   \\
\verb|\rcsInfoDate|         & \Var{Date}             & \rcsInfoDate     \\
\verb|\rcsInfoLongDate|     & \Var{LongDate}         & \rcsInfoLongDate \\
\verb|\rcsInfoDay|          & \Var{Day}              & \rcsInfoDay      \\
\verb|\rcsInfoMonth|        & \Var{Month}            & \rcsInfoMonth    \\
\verb|\rcsInfoYear|         & \Var{Year}             & \rcsInfoYear     \\
\end{tabular}

\end{document}
\endinput
%%
%% End of file `rcsinfo2html.tex'.
