\documentclass[11pt,twocolumn]{article}

\usepackage{lambda,ifthen,calc}
\usepackage{tabularx}
\usepackage[ps,mover]{skak}

\newcommand{\package}[1]{\textsf{#1}}
\newcommand{\pgn}{PGN\ }
\newcommand{\san}{SAN\ }
\newcommand{\fen}{FEN\ }
\newcommand{\xboard}{xboard}
\newcommand{\metafont}{\textsc{Metafont}\ }
\newcommand{\filename}[1]{\texttt{#1}}
%\usepackage[a5paper]{anysize}
%\tracingmacros=1

\newcommand{\Guido}{Guido Governatori\ }
\newcommand{\Harri}{Harri Haanpaa\ }
\newcommand{\Ingo}{Ingo Sander\ }
\newcommand{\Dirk}{Dirk B\"achle\ }
\newcommand{\Ulrike}{Ulrike Fischer\ }
\newcommand{\Andreas}{Andreas Wilm\ }
 
\title{Typesetting Chess in \LaTeX with the \package{skak} Package\\
Version 1.3}
\author{Torben Hoffmann\\ e-mail: \texttt{Torben.Hoffmann@motorola.com}}
\begin{document}
\maketitle

\begin{abstract}
  This user guide shows how to use an enhancement to the existing
  package for typesetting chess in \LaTeX (\package{skak}). The
  primary improvement over the old package is that it has become
  easier to typeset chess games with this new package.
\end{abstract}

\newpage
\tableofcontents
\newpage

%%%%%%%%%%%%%%%%%%%%%%%%%%%%%%%%%%%%%%%%%%%%%%%%%%%%%%%%%%%%%%%%%%
\section{Change History}
\label{sec:change-history}


\subsection*{Version 1.3}
\label{sec:version-1.3}

\begin{itemize}
\item \Dirk has reworked the typesetting of chess moves:
  \begin{itemize}
  \item A new capture symbol has been introduced to make the spacing around
    it better.
  \item The figurine symbols have been improved so that the spacing around
    them are more uniform --- it looks really good now, at least to my eyes.
  \item A proposal for what syntax the \package{skak} package should handle in
    the future has been included in the distribution.
  \item The Informator symbols ``novelty'', ``comment'' and ``various'' needed
    another pair of curly braces in order to work right within the
    ``mainline'' and ``variation'' environments.
  \item The check for trailing spaces in the macro ``typeset@cmoves'' had to
    be removed because it led to problems at the end of a ``mainline'' or
    ``variation'' environment when an additional Informator symbol was
    involved.      
  \end{itemize}
\item A reference to the \textsf{pgn2ltx} tool created by \Dirk has been added
  to the document.
\item A reference to an URL decribing the \pgn standard has been added.
\end{itemize}


\subsection*{Version 1.2}
\label{sec:version-1.2}

\begin{itemize}
\item \Dirk has --- once again --- done a great job:
  \begin{itemize}
  \item the informator symbols have been improved and now scale as
    they should (I have in the same go done some \metafont clean-up
    associated with the symbols).
  \item he has also discovered that the selection of other fonts for
    the typesetting of moves can be done by redefining the
    \verb|\skakfamily| command!
  \item added the \verb|\movecomment| for flexible typesetting of
    comments --- see the Reference Manual for details.
  \item The PostScript file are now 10 times smaller (for large files)
    and a comprehensive test of the PostScript ornaments has been
    created. 
  \end{itemize}
\item All dead code has been removed from \texttt{skak.sty}.
\end{itemize}


\subsection*{Version 1.1}
\label{sec:version-1.1}

\begin{itemize}
\item \Dirk's enhancements have been incorporated:
  \begin{itemize}
  \item extension of the \verb|\highlight| command; it now accepts X,
    x, O and o as optional arguments and produces a cross and a
    circle on the square, respectively, instead of a frame around the
    square.
  \item unwanted generation of spaces removed.
  \item a set of Informator symbols were added to the fonts and made
    available through commands documented in the
    \texttt{informator.ps} file.
  \end{itemize}
\item Two fixes by \Ulrike have been incorporated:
  \begin{itemize}
  \item a bug in \verb|\ParseCoordinates|.
  \item a problem in the fonts regarding size.
  \end{itemize}
\item \Ulrike reported a problem with opening spaces causing the input
  to be ignored --- this has been fixed and \filename{test/test2.tex}
  verifies this.
\end{itemize}



\subsection*{Version 1.0}
\label{sec:version-1.0}

The major novelty in this release is that the documentation has been
updated and a reference manual has been created.


%%%%%%%%%%%%%%%%%%%%%%%%%%%%%%%%%%%%%%%%%%%%%%%%%%%%%%%%%%%%%%%%%%
\section{Terms of Usage}
\label{sec:terms-usage}

This package is distributed under the terms described in the Latex
Project Public Licence, i.e.,    

\begin{quote}
  This software is copyright but you are granted a license which gives
  you, the ``user'' of the software, legal permission to copy,
  distribute, and/or modify the software. However, if you modify the
  software and then distribute it (even just locally) you must change
  the name of the software to avoid confusion.
\end{quote}

%%%%%%%%%%%%%%%%%%%%%%%%%%%%%%%%%%%%%%%%%%%%%%%%%%%%%%%%%%%%%%%%%%
\section{Acknowledgements}
\label{sec:acknowledgements}

I would like to thank the creator of the \package{lambda} package,
Alan Jeffrey, for making a splendid package that made this package
possible.

For discussions and alpha-testing I send my thanks to \Guido --- some
of his ideas have already been implemented; the rest? Time will show.

Thanks to \Harri for finding a nasty bug in the castling routine.

A big thank goes to \Dirk for adding the Informator symbols to the
fonts and for removing a couple of nasty bugs as well as improving
some of the commands.

I thank \Ulrike for spotting as well as fixing a big problem with the
parsing of moves and for her hints to improving the fonts.

%%%%%%%%%%%%%%%%%%%%%%%%%%%%%%%%%%%%%%%%%%%%%%%%%%%%%%%%%%%%%%%%%%
\section{The Old \package{chess} Package Versus the New \package{skak}
  Package}
\label{sec:old-vs-new}

The main reason for considering an improvement of the \package{chess}
package that Piet Tutelaers made back in 1991 is the cumbersome user
interface the package has when one wants to write about a chess game
and display a diagram every now and then. Typesetting the two opening
moves where white and black move their kingside knights is done as
follows in the \package{chess} package:

\begin{verbatim}
\move g1f3 g8f6
\end{verbatim}

The package then produces a nice typesetting of these moves in figurine
notation, but it is hard to keep track of what is going on because one usually
uses the \san (Short Algebraic Notation, employed in the \pgn standard for
typing the moves --- see \texttt{http://pgn.freeservers.com/standard.txt} for
details) to write down the moves of a chess game. The \san version of the two
moves above is: \verb|1. Nf3 Nf6|, which is much clearer to most chess
players. In the \package{skak} package the author of a chess article is
allowed to use the \san notation as input to the command that updates the
chess board. In addition to making it easier for the author to write about
chess using a familiar notation it also provides an easy way to include moves
generated by a chess program such as \xboard\ in the document---most other
chess programs can also output a \pgn version of a chess game and from that
you can extract the \san recording of the moves.

The \package{skak} package can also input chess board positions given
in the \fen notation (also used in the \pgn standard), which is also quite
standard in the domain of chess programs. 

Apart from a better user interface the \package{skak} chess font
contains three small modifications of the font created by Piet
Tutelaers: the knight now looks a bit more ``youthful'', the contour
of the queen has been smoothened and all the chess pieces have been
shrunk such that they do not fill as much of a square as before. I
find this font nicer to look at, but your milage may vary.




\section{How to use the \package{skak} Package}
\label{sec:how-use-skak-package}

Writing about a chess game can be done straightforward:

\begin{verbatim}
\newgame

\mainline{1. e4 e5 2. Nf3 Bc5}
\end{verbatim}
 
\newgame
\noindent
starts a new game and produces the following in your document: 

\mainline{1. e4 e5 2. Nf3 Bc5}

So far, so good. If you want to show the current position you just
type \verb|\[\showboard\]| in your document to get:

\[\showboard\]

\noindent
(the use of math \verb|\[...\]| is just to make sure the board is centered.)

This is the basic functionality of the \package{skak} package, but it
offers many different ways in which one can talk about chess
games. I think that a good way to show how the \package{skak} package
can be used is to typeset the Fischer--Tal game from the old
\package{chess} package. The result is in the file named
\filename{tuggame.ps}.

Further information about all the bells and whistles the
\package{skak} package provides can be found in the \emph{Reference
  Manual} where all commands of relevance are described.


\subsection{Handling of Variations}
\label{sec:handling-variations}


The \package{skak} package does \emph{not} support
\pgn variations such as
\begin{verbatim}
\mainline{12. Bb4 (12. Ng5 h6) 12... Ra8}
%gives an error
\end{verbatim}

You have to type this as
\begin{verbatim}
\mainline{12. Bb4}
\variation{12. Ng5 h6}
\mainline{12...Ra8}
\end{verbatim}

I do not have the nerves to update the parsing algorithm to cope with
this so the best thing to do would be to craft a tool for translating
\pgn to \TeX (see \ref{sec:transl-pgn-files}).


\subsection{Changing the Font}
\label{sec:changing-font}

(New in Version 1.2)

\Dirk has discovred that you can change the font used for typesetting
of moves if you redefine the \verb|\skakfamily| font.

I do not --- however --- know how to handle different chess fonts, so
if someone could work this out I'll include that in a future version.


%%%%%%%%%%%%%%%%%%%%%%%%%%%%%%%%%%%%%%%%%%%%%%%%%%%%%%%%%%%%%%%%%%
\section{Installing the \package{skak} Package}
\label{sec:install}

I could not get the makefile to do the job, so if someone could help
me out on this I would be very glad indeed.

Basically all you have to do is to follow the guidelines in the
makefile. First you create all the required directories (stated in the
\texttt{install} section of the makefile) and then you run the
\texttt{make install} command. Remember to run \texttt{texhash} after
you have updated your texmf tree.

Sorry that it is not working $100\%$--- I just do not have the time for
it right now.

If you cannot get the things working just put the files where \TeX and
the other tools can find them and you will be just fine. 


\subsection{Using the \package{lambda} Package}
\label{sec:using-lambda}

The \package{skak} package uses the \package{lambda} package to do
some of the hard processing necessary to allow the \pgn notation as
input --- since it is not standard in \LaTeX distributions it is
included in the distribution of the \package{skak} package.


%%%%%%%%%%%%%%%%%%%%%%%%%%%%%%%%%%%%%%%%%%%%%%%%%%%%%%%%%%%%%%%%%%
\section{Future Additions/Wish-list}
\label{sec:future-additions}


\subsection{Choosing Language for \texttt{mainline}}\label{sec:choos-lang-mainline}

(Proposal date: mid 2002.)

It would be nice if one could choose the language for the
\verb|mainline|, \verb|variation| and \verb|\hidemoves| for each invocation
as it allows for easier inclusion of analysis from chess programs when
using a non-english language.


\subsection{Algebraic Notation used for Typesetting}
\label{sec:algebraic-notation-in-typesetting}

(Proposal date: mid 2001.)

\Ingo has suggested that the output of the typesetting should be the
old algebraic notation where the from and to squares always are
given. This requires a modification to the game engine: after each
move has been made you should store the algebraic notation of the move
since you have the to and from squares calculated at that
moment. Changing the game engine is a bit hairy --- even for me --- so I'll
postpone this.


\subsection{Fonts}
\label{sec:ps-font}

(Proposal date: dec-2002.)

\Ulrike has suggested the creation PostScript version of the \metafont
font.

I do not know how to do this, but perhaps someone can help me?

Furthermore, \Ulrike would like to see a more transparent
fonthandling, so that it becomes easy to change to another chess-font.
(Version 1.2 news: see Section~\ref{sec:changing-font}.)


\subsection{Optimisation of the Implementation}
\label{sec:optim-impl}

(Proposal date: 2-Jan-2003.)

I would like to improve some areas of the implementation since they
are unnecessary slow and/or complicated, e.g., \verb|\IsPieceName|
could easily be implemented as a case statement instead of a list
look-up. I have a strong feeling that this would be a lot faster than
the current implementation which has a very functional programming
flavour --- this was very helpful during the development, but it has a
tendency to be a bit inefficient. 

Another great improvement would be to let the commands
\verb|\Mainline| and \verb|\typeset@A| use the same parsing algorithm
--- then one could implement the long algebraic notation as suggested
by \Ingo by letting the action taken on a move be both updating of the
board \emph{and} typesetting of the move!

\subsection{Test Suite}
\label{sec:test-suite}

(Proposal date: 03-Jan-2003.)

It would be a very good idea to create a test suite created using
normal testing techniques. 
Areas of test should include:
\begin{itemize}
\item resolving ambiguous moves, i.e., ensuring that moves like
  \wmove{Rad1} moves the right rook.
\item all legal \san moves are accepted.
\end{itemize}



\subsection{Translating \pgn Files}
\label{sec:transl-pgn-files}

(Proposal date: 03-Jan-2003.)

It would be very nice if one had a command line tool that could
translate \pgn files to a \TeX file.

27-Sep-2003: Take a look at \texttt{http://pgn2ltx.sourceforge.net} for a nice
helper tool.


\subsection{Typesetting \textsl{e.p.} after en passant moves}
\label{sec:typeset-en-passant}

\Andreas asked how one should input en passant moves, especially the addition
of \textsl{e.p.} after the capture. 

This is not part of the \san notation as described in the \pgn standard and it
requires that the improvement described in Section \ref{sec:optim-impl} is in
place before it is possible to add the possibility to add the
\textsl{e.p.}. It has to be this way because I do not want to support more
than what is described by the \pgn standard.


\end{document}





