% This is a skak version of the tugame.ltx which Piet Tutelaers made
% to show how the original chess package could be used.
% I have changed it to use the skak package commands as a
% demonstration of how to use the skak package.
% Author: Torben Hoffmann
%
% Change history
% --------------
% Version          Comments 
%  1.0             Initial version adapted to the skak package.

% The original comments follows below
%===================================================================
%
% TUGgame.ltx
% -----------
% LaTeX source of example in TUGboat article (part of the game Fisher
% lost against Tal), input'ted by TUGboat.ltx.
% Author : Piet Tutelaers (internet: rcpt@urc.tue.nl)
% Version: 1.2 ( 8 Jun 1991)
%    Reflects changes in chess.sty version 1.2
% Version: 1.1 (30 Nov 1990)
%    Improvements over version 1.0:
%     - two typos corrected, thanks Hugo
%===================================================================


\documentclass[11pt,twocolumn]{article}

\usepackage[ps,mover,styleC]{skak}

\title{Example of the LaTeX-input and output of an annotated 
chess game using \texttt{skak.sty}}
\author{Torben Hoffmann}

\begin{document}

\parindent=0pt

\maketitle

\section{The Input}

\begin{verbatim}
\fenboard{1q3kr1/3rb2p/p3Q3/8/%
1p6/8/PPP3PP/4R2K w - - 0 26}

\begin{figure}[htbp]
  \begin{center}
    $$\showboard$$
    \caption{Fischer--Tal after 
      \protect\variation{25... Kf8!}}
    \label{fig:after-25...Kf8}
  \end{center}
\end{figure}


(See figure~\ref{fig:after-25...Kf8}.)

\mainline{26. Qxd7}

Not \variation{26. Rf1+ Kg7 27. Rf7+ 
Kh8}
and if \variation{28. Qxd7 Rd8 29. Qg4 
Qe5}
wins. 


\mainline{26...Qd6 27. Qb7 Rg6}
Within a handful of moves the game
has changed its complexion. Now it
is White who must fight for a draw!

\mainline{28. c3}
Black's extra piece means less with
each pawn that's exchanged.

\mainline{28...a5}
On \variation{28...bxc3 29. Qc8+ Bd8 
30. Qxc3}=.


\mainline{29. Qc8+}
On the wrong track. Right is
\variation{29. cxb4 Qxb4} (if 
\variation{29... axb4 30. a3! bxa3 
31. bxa3 Qxa3} draws) 
\variation{30. Qf3+ Kg7 31. Qe2} draws, 
since Black can't possibly build up a
winning K-side attack and his own
king is to exposed.

\mainline{29...Kg7 30. Qc4 Bd8 
31. cxb4 axb4}
On \variation{31... Qxb4 32. Qe2} 
White should draw with best play.
$$\showboard$$
\end{verbatim}

%%%%%%%%%%%%%%%%%%%%%%%%%%%%%%%%%%%%%%%%%%%%%%%%%%%%%%%%%%%%%%%%%%
\section{The Output}

\fenboard{1q3kr1/3rb2p/p3Q3/8/%
1p6/8/PPP3PP/4R2K w - - 0 26}

\begin{figure}[htbp]
  \begin{center}
    $$\showboard$$
    \caption{Fischer--Tal after 
      \protect\variation{25... Kf8!}}
    \label{fig:after-25...Kf8}
  \end{center}
\end{figure}


(See figure~\ref{fig:after-25...Kf8}.)

\mainline{26. Qxd7}

Not \variation{26. Rf1+ Kg7 27. Rf7+ 
Kh8}
and if \variation{28. Qxd7 Rd8 29. Qg4 
Qe5}
wins. 


\mainline{26...Qd6 27. Qb7 Rg6}
Within a handful of moves the game
has changed its complexion. Now it
is White who must fight for a draw!

\mainline{28. c3}
Black's extra piece means less with
each pawn that's exchanged.

\mainline{28...a5}
On \variation{28...bxc3 29. Qc8+ Bd8 
30. Qxc3}=.


\mainline{29. Qc8+}
On the wrong track. Right is
\variation{29. cxb4 Qxb4} (if 
\variation{29... axb4 30. a3! bxa3 
31. bxa3 Qxa3} draws) 
\variation{30. Qf3+ Kg7 31. Qe2} draws, 
since Black can't possibly build up a
winning K-side attack and his own
king is to exposed.

\mainline{29...Kg7 30. Qc4 Bd8 
31. cxb4 axb4}
On \variation{31... Qxb4 32. Qe2} 
White should draw with best play.
$$\showboard$$


\end{document}






