%%
%% This is file `example.tex',
%% generated with the docstrip utility.
%%
%% The original source files were:
%%
%% typedref.dtx  (with options: `example')
%% 
%%    Package `typedref' for use with LaTeX2e
%%    Copyright (C) 2001 Gregory Seidman (email `gseidman@acm.org').
%%    All rights reserved.  You may not alter the contents of this file unless
%%    you also change its name. Please make a bona fide attempt to submit
%%    patches to me before changing the name of this file.
%% 
%%    Modified from package `saferef'
%%    Copyright (C) 1997 James Ashton (email `James.Ashton@anu.edu.au').
%%    All rights reserved.  You may not alter the contents of this file unless
%%    you also change its name.
%% 
%%    Modifications include proper handling of appendices which are not
%%    chapters (i.e. in a document class other than book) and compatibility
%%    with hyperref.
%% 
\documentclass{article}
\usepackage{amsmath}
\usepackage{amsthm}
\usepackage{typedref}
\newtheorem{theorem}{Theorem}{Theorems}
\newtheorem{lemma}[theorem]{Lemma}{Lemmas}
\begin{document}
\section{The first section}
\label{zero}
We'll use the following in \theoremref{one}.
\begin{equation}
a = b\label{two}
\end{equation}
\begin{theorem}
\label{one}
Equation \eqref{two} has nothing to do with \eqref{three} or \eqref{four}.
See also \lemmaref{six}.
\begin{align}
a^2&=b^2+c^2\label{three}\\
a&=\sqrt{b^2+c^2-2bc\cos A}\label{four}
\end{align}
\end{theorem}
Having had this pointless theorem, why not try for a lemma of similar
class.
\section{Another section}
\label{five}
\begin{lemma}
\label{six}
\theoremref{one} (in \sectionref{zero}) contains two equations.
While both are simple, \eqref{four} is more complex than \eqref{three}.
\end{lemma}
\section{The final section}
\label{seven}
This document includes \sectionref{zero,five,seven}.
\lemmaref{six} is on page \pageref{lemma:six}.
\end{document}
\endinput
%%
%% End of file `example.tex'.
