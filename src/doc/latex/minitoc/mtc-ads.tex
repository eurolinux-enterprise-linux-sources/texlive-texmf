%%
%% This is file `mtc-ads.tex',
%% generated with the docstrip utility.
%%
%% The original source files were:
%%
%% minitoc.dtx  (with options: `mtc-ads')
%% This is a generated file.
%% 
%% Copyright (C) 2005, 2006 by:
%% Jean-Pierre F. Drucbert (JPFD)
%% <jean-pierre.drucbert@onera.fr>
%% 
%% This file may be distributed and/or modified under the conditions of
%% the LaTeX Project Public License, either version 1.3 of this license
%% or (at your option) any later version.  The latest version of this
%% license is in:
%% 
%%    http://www.latex-project.org/lppl.txt
%% 
%% and version 1.3 or later is part of all distributions of LaTeX version
%% 2003/12/01 or later.
%% 
%% This work has the LPPL maintenance status "author-maintained".
%% 
%% This Current Maintainer of this work is Jean-Pierre F. Drucbert (JPFD).
%% 
%% This work consists of all the files listed in the file `minitoc.l'
%% distributed with it.
\documentclass[oneside,12pt,a4paper]{article}
\ProvidesFile{mtc-ads.tex}%
  [2007/01/04]
\usepackage{url,tocbibind,makeidx}
\makeatletter
\newif\ifscan@allowed
\scan@allowedtrue
\makeatother
\def\dotfil{\leaders\hbox to.6em{\hss .\hss}\hfil}%
\def\pfill{\unskip~\dotfill\penalty500\strut\nobreak
           \dotfil~\ignorespaces}%
\usepackage[tight,hints,insection]{minitoc}
%%\usepackage{mtcoff}
\makeindex
\begin{document}
\doparttoc \dopartlof \dopartlot
\dosecttoc \dosectlof \dosectlot
\setcounter{tocdepth}{6}
\setcounter{parttocdepth}{6}
\setcounter{secttocdepth}{6}
\tableofcontents
\mtcaddsection
\listoffigures
\mtcaddsection
\listoftables
\mtcaddsection
\part{Part~1}
\parttoc \mtcskip \partlof \mtcskip \partlot

\section{First section}
\index{section!normal}
\secttoc \mtcskip \sectlof \mtcskip \sectlot
\subsection{First subsection}

\begin{figure}[tp]
\caption{First figure}
\end{figure}
\begin{table}[tp]
\caption{First table}
\end{table}

\subsection{Second subsection}
A small nice citation from~\cite{dark}:\\
\index{small}\index{citation}\index{nice}\index{A}%
\index{a}\index{and}\index{bird}\index{But}%
\index{cannot}\index{claim}\index{great}\index{he}%
\index{I}\index{imagine}\index{it}\index{know}%
\index{land}\index{on}\index{once}\index{that}%
\index{to}\index{tree}\index{would}\index{yes}%
A bird cannot land once on a great tree and claim to know it.
But I imagine that he would, yes.\\
\hbox{}\hfill Iain M. Banks (1993), \textsl{Against a dark background.}%
\index{Iain}\index{Banks}\index{Against}\index{dark}%
\index{background}
\begin{figure}[tp]
\caption{Second figure}
\end{figure}
\begin{table}
\caption{Second table}
\end{table}
\section*{Second section, starred}
\index{section!starred}
%% UNCOMMENT ONE AND ONLY ONE OF THE 4 FOLLOWING LINES
\mtcaddsection[Second section, starred] % OK
%%\mtcaddsection[]     % BAD
%%\mtcaddsection[~]    % produces a (strange) correct result.
%%\mtcaddsection       % BAD
%%%%%%%%%%%%%%%%%%%%%%%%%%%%%%%%%%%%%%%%%%%%%%%%%%%%%%
\index{tests}

This is a starred section; you can test here variations on
the \verb|\mtcaddsection| command. Just uncomment one (and
only one) of the \verb|\mtcaddsection| commands after
\verb|\section*| in the source code of \texttt{mtc-add.tex}.
For each case, look at the \index{Table of Contents}Table of Contents
and at this section.
\index{a}\index{added}\index{after}\index{also}\index{and}%
\index{at}\index{can}\index{case}\index{section}%
\index{code}\index{command}\index{commands}%
\index{Contents}\index{each}\index{entries}\index{For}%
\index{here}\index{I}\index{in}\index{index}\index{is}%
\index{Just}\index{just}\index{look}\index{lot}%
\index{of}\index{on}\index{one}\index{only}\index{source}%
\index{starred}\index{Table}\index{test}\index{the}%
\index{This}\index{this}\index{to}\index{uncomment}%
\index{variations}\index{you}%
I also added a lot of index entries, just to test.

\section{Third section}
\index{section!normal}
\secttoc \mtcskip \sectlof \mtcskip \sectlot
\subsection{Third subsection}

\begin{figure}
\caption{Third figure}
\end{figure}
\begin{table}
\caption{Third table}
\end{table}

\subsection{Fourth subsection}

\begin{figure}
\caption{Fourth figure}
\end{figure}
\begin{table}
\caption{Fourth table}
\end{table}

\subsubsection{Even a sub-sub-section!}
\subsubsection{And yet another one}

\part{Part~2}
\parttoc \mtcskip \partlof \mtcskip \partlot

\section{Fourth section}
\index{section!normal}
\secttoc \mtcskip \sectlof \mtcskip \sectlot
\subsection{Fifth subsection}

\begin{figure}[tp]
\caption{Fifth figure}
\end{figure}
\begin{table}[tp]
\caption{Fifth table}
\end{table}

\subsection{Sixth subsection}
A small nice citation from~\cite{dark}:\\
\index{small}\index{citation}\index{nice}\index{A}%
\index{a}\index{and}\index{bird}\index{But}%
\index{cannot}\index{claim}\index{great}\index{he}%
\index{I}\index{imagine}\index{it}\index{know}%
\index{land}\index{on}\index{once}\index{that}%
\index{to}\index{tree}\index{would}\index{yes}%
A bird cannot land once on a great tree and claim to know it.
But I imagine that he would, yes.\\
\hbox{}\hfill
Iain M. Banks (1993), \textsl{Against a dark background.}%
\index{Iain}\index{Banks}\index{Against}%
\index{dark}\index{background}

\begin{figure}[tp]
\caption{Sixth figure}
\end{figure}
\begin{table}
\caption{Sixth table}
\end{table}

\section*{Fifth section, starred}
\index{section!starred}
%% UNCOMMENT ONE AND ONLY ONE OF THE 4 FOLLOWING LINES
\mtcaddsection[Fifth section, starred] % OK
%%\mtcaddsection[]     % OK
%%\mtcaddsection[~]    % produces a (strange) correct result.
%%\mtcaddsection       % OK
%%%%%%%%%%%%%%%%%%%%%%%%%%%%%%%%%%%%%%%%%%%%%%%%%%%%%%
\index{tests}

This is a starred section; you can test here variations on
the \verb|\mtcaddsection| command. Just uncomment one (and
only one) of the \verb|\mtcaddsection| commands after
\verb|\section*| in the source code of \texttt{mtc-add.tex}.
For each case, look at the \index{Table of Contents}Table of Contents
and at this section.\index{a}%
\index{added}\index{after}\index{also}\index{and}%
\index{at}\index{can}\index{case}\index{section}%
\index{code}\index{command}\index{commands}\index{Contents}%
\index{each}\index{entries}\index{For}\index{here}%
\index{I}\index{in}\index{index}\index{is}%
\index{Just}\index{just}\index{look}\index{lot}%
\index{of}\index{on}\index{one}\index{only}%
\index{source}\index{starred}\index{Table}\index{test}%
\index{the}\index{This}\index{this}\index{to}%
\index{uncomment}\index{variations}\index{you}%
I also added a lot of index entries, just to test.

\section{Sixth section}
\index{section!normal}
\secttoc \mtcskip \sectlof \mtcskip \sectlot
\subsection{Seventh subsection}

\begin{figure}
\caption{Seventh figure}
\end{figure}
\begin{table}
\caption{Seventh table}
\end{table}

\subsection{Eighth subsection}

\begin{figure}
\caption{Eighth figure}
\end{figure}
\begin{table}
\caption{Eighth table}
\end{table}
\nocite*
\def\noopsort#1{\relax}
\bibliographystyle{plain}
\bibliography{mtc-add}
\adjuststc
\printindex
\mtcfixindex % use this OR the 2 following lines
%%\addcontentsline{lof}{xsect}{}
%%\addcontentsline{lot}{xsect}{}
%%\mtcaddsection

\appendix
\section{App.~1}
\index{section!appendix}
\secttoc \mtcskip \sectlof \mtcskip \sectlot
\subsection{Ninth subsection}

\begin{figure}
\caption{Ninth figure}
\end{figure}
\begin{table}
\caption{Ninth table}
\end{table}

\subsection{Tenth subsection}

\begin{figure}
\caption{Tenth figure}
\end{figure}
\begin{table}
\caption{Tenth table}
\end{table}

\section{App.~2}
\index{section!appendix}
%% contains no tables but asks for a sectlot! No sectlot printed.
\secttoc \mtcskip \sectlof \mtcskip \sectlot
\subsection{Eleventh subsection}

\begin{figure}
\caption{Eleventh figure}
\end{figure}
\begin{figure}
\caption{Twelfth figure}
\end{figure}

\subsection{Twelfth subsection}

\begin{figure}
\caption{Thirdteenth figure}
\end{figure}
\begin{figure}
\caption{Fourteenth figure}
\end{figure}

\end{document}
\endinput
%%
%% End of file `mtc-ads.tex'.
