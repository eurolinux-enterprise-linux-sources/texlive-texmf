\documentclass[a4paper]{article}
\usepackage{times,makeglos}
\makeglossary
\title{The makeglos package}
\author{Thomas Henlich (thenlich at arcor dot de)}
\begin{document}
\maketitle
\tableofcontents
\renewcommand{\glossaryintro}{Now follows a very short glossary:}
\printglossary
\addcontentsline{toc}{section}{\glossaryname}

\section{Introduction}

A glossary is a list of words (concepts, terms) together with their
explanations.
\glossary{glossary:A list of words with explanations of these words}
In a glossary you should explain those terms which are
necessary for the understanding of your document, but which some of your
readers might not know. The advantages of using a glossary vs. writing
definitions in the middle of a sentence are:

\begin{itemize}
  \item improved text flow: the explanations don't clutter up the logical
    structure of the text
  \item reader-friendliness: people who already know the meaning of terms
    (e.\,g. people who come from the same professional background as you)
    don't need to read their definitions. That helps them understand the
    text faster.
\end{itemize}

Very often a glossary is sorted alphabetically. An example for a glossary in
a technical or scientific document is a list of mathematical symbols
together with their meanings.

{\tt makeglos} is a \LaTeX{} package to include a glossary into a document.
The glossary must be prepared by an external program, like {\tt xindy} or
{\tt makeindex}.
\glossary{index processor:a software product to prepare an index for
inclusion into a \LaTeX{} document}
\glossary{xindy|is{an index processor, can also be used to process glossaries}}
\glossary{xindy|see{index processor}}
\glossary{xindy|seealso{glossary}}


\section{Features}

\begin{itemize}
  \item Can be easily configured: various aspects of the appearance of the
    glossary can be changed by a simple \verb+\renewcommand+.
  \item Multi-language capability: certain keywords, like the name of the
    glossary, can be made language-dependent.
  \item Flexible: works equally well for document classes with a book- or
    report-like structure (divided into \verb+\chapters+) and for article-like
    documents, like the one you are reading now (based on \verb+\sections+)
  \item Equivalent: \verb+makeglos+ is to glossaries what \verb+makeidx+ is
    to indices.
\end{itemize}


\section{Commands}

\begin{tabular}{lp{.7\textwidth}}
\verb+\printglossary+ & Inserts the glossary file ({\tt *.gls}) here. \\
\verb+\glossaryname+ & The name of the glossary. Defaults to ``Glossary''.\\
\verb+\glossaryintro+ & Something to print before the actual glossary.
Defaults to empty. \\
\verb+\gsee+ & The cross-reference introduction phrase.
Defaults to \verb+\emph{\seename} #1+. \\
\verb+\galso+ & The secondary cross-reference introduction phrase.
Defaults to \verb+\emph{\alsoname} #1+. \\
\verb+\seename+ & The cross-reference introduction phrase.
Defaults to ``see''. Redefined by \verb+babel+ package.\\
\verb+\alsoname+ & The secondary cross-reference introduction phrase.
Defaults to ``see also''. Redefined by \verb+babel+ package.\\
\end{tabular}


\section{Notes}

The glossary environment is by default defined to be a chapter (or section)
which does not appear in the table of contents. If you want the glossary to
appear there (just like I did when I wrote this), just add \\
\verb+\addcontentsline{toc}{section}{\glossaryname}+.

The glossary's environment definition for document classes in which the
\verb+\chapter+ command is undefined (article-like documents) is 
\begin{verbatim}
\newenvironment{theglossary}%
{\section*{\glossaryname}\glossaryintro%
\begin{description}}%
{\end{description}}%
\end{verbatim}

The glossary's environment definition for document classes in which the
\verb+\chapter+ command is defined (report-/book-like documents) is 
\begin{verbatim}
\newenvironment{theglossary}%
{\chapter*{\glossaryname}\glossaryintro%
\begin{description}}%
{\end{description}}%
\end{verbatim}

If you want to change this, \verb+\renewenvironment+ is your friend!

The glossary for this document was prepared with \verb+xindy+, version 2.0d:\\
\verb+xindy -o makeglos.gls -f tex2xindy makeglos.xdy makeglos.glo+

\end{document}

%%% Local Variables: 
%%% mode: latex
%%% TeX-master: t
%%% End: 
