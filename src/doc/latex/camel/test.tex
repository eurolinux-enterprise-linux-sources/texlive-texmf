%%
%% This is file `test.tex',
%% generated with the docstrip utility.
%%
%% The original source files were:
%%
%% camel.dtx  (with options: `testtex')
%% 
%% This file is part of the Camel package.
%% ---------------------------------------
%% This is a generated file.
%% 
%% IMPORTANT NOTICE:
%% 
%% You are not allowed to change this file.  You may however copy
%% this file to a file with a different name and then change the
%% copy if (a) you do not charge for the modified code, (b) you
%% acknowledge Camel and its author(s) in the new file, if it
%% is distributed to others, and (c) you attach these same
%% conditions to the new file.
%% 
%% The above conditions do not apply to the demonstration
%% file test.tex.
%% 
%% You are not allowed to distribute this file alone.  You are not
%% allowed to take money for the distribution or use of this file
%% (or a changed version) except for a nominal charge for copying
%% etc.
%% 
%% You are allowed to distribute this file under the condition that
%% it is distributed with all of its contents, intact.
%% 
%% For error reports, or offers to help make Camel a more powerful,
%% friendlier, and better package, please contact me on
%% `fb' at soas.ac.uk
%% 
\documentclass{article}
\usepackage{camel}

%\citationsubject[o=sta,i=stb]{statutes}{Statutory Materials}
%\citationsubject[o=sec,i=seb]{second}{Secondary Literature}
%\citationsubject[o=cas,i=cab]{case}{Cases}

\begin{document}
\citationstyle{law}
\citationdata{camel}

\section*{{\sc Camel} tests and examples\footnotemark{}{} }
\footnotetext{Note that, in addition to admiring the examples in
  this document, you can tinker with it to produce different
  kinds of bibliographies.  See the comments in the file for
  suggestions.}

We currently have support for articles (including newspaper
articles), items in collections of
essays, books, sections of books, ephemeral booklets, masters theses,
Commonwealth, US and Japanese cases, and statutes from a few
jurisdictions.  An example of each is given below, first without,
then with a pinpoint.  This does not give a complete picture of
the `formatting tree', it''s just a sample.

\begin{itemize}
\item Ordinary articles
  \begin{itemize}
  \item \source[s=second]{macauley}
  \item \source[f,s=second]{macauley}[56]
  \end{itemize}
\item Newspaper articles
  \begin{itemize}
  \item \source[s=second]{appleyard-heed}
  \item \source[f,s=second]{appleyard-heed}[21] (A silly example,
    since there's only one page to the piece!)
  \end{itemize}
\item Articles in collections of essays
  \begin{itemize}
  \item \source[s=second]{haley-land-lease}
  \item \source[f,s=second]{haley-land-lease}[152]
  \end{itemize}
\item Books
  \begin{itemize}
  \item \source[f,s=second]{latex-companion}
  \item \source[f,s=second]{latex-companion}[371]
  \end{itemize}
\item Sections of books
  \begin{itemize}
  \item \source[f,s=second]{companion-bibs}
  \item \source[f,s=second]{companion-bibs}[374-75] (The BibTeX{}
    processing flow shown here is simplified for {\sc Camel} users)
  \end{itemize}
\item Ephemeral booklets
  \begin{itemize}
  \item \source[f,s=second]{sansom-constitution}
  \item \source[f,s=second]{sansom-constitution}[2]
  \end{itemize}
\item Masters theses
  \begin{itemize}
  \item \source[f,s=second]{homma-derivative}
  \item \source[f,s=second]{homma-derivative}[23]
  \end{itemize}
\item Commonwealth law cases
  \begin{itemize}
  \item \source[f,s=case]{heap}
  \item \source[f,s=case]{heap}[578]\footnote{Notice how the
      pinpointed citation gives only as many parallel as are
      specified in the pinpoint.  Compare this with the next example.}
  \end{itemize}
\item US law cases
  \begin{itemize}
  \item \source[f,s=case]{bradshaw-v-us}
  \item \source[f,s=case]{bradshaw-v-us}[145=366]
  \end{itemize}
\end{itemize}
Titles that end in a numeral are a special treat.  {\sc Camel}
is not tricked by them.
\begin{itemize}
  \item \source[s=statutes]{maki-constitution}[23]
  \item \source[f,s=statutes]{maki-constitution}
\end{itemize}
Selective suppression of author, title, or the entire citation is
supported.
\begin{itemize}
  \item \source[a,f,s=second]{homma-derivative}
  \item \source[t,f,s=second]{homma-derivative}
  \item \source[n,s=second]{homma-derivative} (invisible)
\end{itemize}

% If you uncomment the \citationsubject lines above,
% you can comment out the `all' entry below, run
% makeindex in the appropriate way over the output
% files, uncomment the three special bibliography
% declarations below, and print a whole different sort
% of document.  There are a few bugs in the way
% bibliographies are set up.  Sorting them out will
% best be done once BibTeX 1.0 is available, since
% integrating with the new BibTeX will affect the same
% portions of the code.
\printbibliography[labels=false]{all}
%\printbibliography{statutes}
%\printbibliography{case}
%\printbibliography{second}
\end{document}
\endinput
%%
%% End of file `test.tex'.
