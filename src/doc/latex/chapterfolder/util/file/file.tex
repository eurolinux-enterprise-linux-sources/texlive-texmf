La premi\`ere possibilit\'e pour inclure un fichier est d'utiliser la commande {\tt $\backslash$cfinput} qui permet d'inclure un fichier de la m\^eme mani\`ere que {\tt $\backslash$input}, sauf que le chemin d'acc\`es est celui du dossier courant. La deuxi\`eme fa\c{c}on consiste \`a utiliser les commandes {\tt $\backslash$cfcurrentfolder} et {\tt $\backslash$cfcurrentfolderfigure}. La premi`ere retourne le chemin d'acc\`es au dossier courant, alors que la deuxi\`eme retourne le chemin d'acc\`es \`a un dossier figure qui se trouvent dans le dossier courant.

Ainsi, par exemple, la commande {\tt $\backslash$cfcurrentfolder} produit dans notre cas le r\'esultat "\cfcurrentfolder", alors que la commande {\tt $\backslash$cfcurrentfolderfigure} produit "\cfcurrentfolderfigure" .