\documentclass[a4paper,%		% Papiergr��e A4
				12pt, %				%
				BCOR0mm, %		% Bindekorrektur
				DIVcalc, %		% Satzspiegel wird unten neu berechnet
				automark, %		% lebende Kolumnentitel
				pagesize%			% Seitengr��e wird bei dvi und pdf richtig gesetzt
				]{scrartcl}		% entspricht article

\usepackage{ngerman, graphicx, sets} %, listing}

%----------------  Schriften  ------------------------------------------------
\usepackage{mathpazo} % Palatino
\usepackage[scaled=0.95]{helvet} % Sansserif-Schrift: Frutiger
\usepackage{courier}								% TeleType-Schrift Courier

\usepackage{setspace}								% 1.05-facher Zeilenabstand wegen Palatino
\linespread{1.05}

\usepackage[T1]{fontenc}						% T1-Schriften verwenden
\usepackage[latin1]{inputenc}				% Zeichenkodierung latin 1
%-----------------------------------------------------------------------------

%----------------  Hyperref  -------------------------------------------------
\usepackage{hyperref}
\hypersetup{%				% Einstellung der pdf-Dokumenteigenschaften
  pdftitle  = {Die Pakete sets und vhistory},		% Titel
  pdfsubject = {Beschreibung der LaTeX-Pakete},	% Thema
  pdfauthor = {Jochen Wertenauer},		% Autor
  pdfkeywords = {sets, sets.sty, vhistory, vhistory.sty, Beschreibung, Dokumentation, LaTeX}}	% Stichworte
%-----------------------------------------------------------------------------

%----------------  Kopf- und Fu�zeilen  --------------------------------------
\usepackage{scrpage2}
\pagestyle{scrheadings}
%-----------------------------------------------------------------------------

\typearea[current]{calc} % Neuberechnung des Satzspiegels

%----------------  Indexerstellung (Vorbereitung)  ----------------------------
%\usepackage{makeidx}

% Ordnet im Index die Punkte so an, dass sie untereinander stehen
%\newcommand{\symDotfill}{\leaders\hbox to 5pt{\hss.\hss}\hfill}

%\makeindex

%-----------------------------------------------------------------------------

%\renewcommand{\bibname}{Referenzen}


%%%%%%%%%%%%%%%%%%%%%%%%%%%%%%%%%%%%%%%%%%%%%%%%%%%%%%%%%%%%%%%%%%%%%%%%%%%%%%
%%%%%%%%%%%%%%%%%%%%%%%%%%%%%%%%%%%%%%%%%%%%%%%%%%%%%%%%%%%%%%%%%%%%%%%%%%%%%%

\begin{document}

%%%--------  TITELSEITE(N)  -----------------------------------------------%%%

\subject{Beschreibung der \LaTeX-Pakete}				% Typisierung: Unmittelbar �ber Titel
\title{sets und vhistory}					% Titel
\author{Jochen Wertenauer\\
	\normalsize{\href{mailto:jwertenauer@gmx.de}
					{jwertenauer\,@\,gmx.de}}}
\date{\today}						% Datum
\publishers{\vskip 2ex \small Dieses Dokument steht unter der
	\href{http://www.latex-project.org/lppl.txt}{\LaTeX\ Project Public License}.}%
\maketitle
%%%---------- ENDE TITELSEITE(N) ------------------------------------------%%%

\tableofcontents\clearpage

\section{Einleitung}\label{einleitung}
Es ist mir klar, dass die wenigsten Leser Einleitungen lesen. Ich empfehle dennoch, diesen Abschnitt nicht zu �berspringen, da er zu erl�utern versucht, warum die Pakete sets und vhistory entwickelt wurden. So k�nnen Sie fr�hzeitig erkennen, ob sie Ihren Anforderungen gerecht werden.

%\index{Versionshistorie|(}
Bei Softwareprojekten entstehen (hoffentlich) viele Dokumente wie Spezifikation oder Entwurf. Diese Dokumente werden mehrfach �berarbeitet. Um �nderungen direkt nachvollziehen zu k�nnen, sollten diese Dokumente eine sogenannte \emph{Versionshistorie} enthalten. Dabei handelt es sich um eine Tabelle, deren Eintr�ge folgende Daten umfassen:
\begin{itemize}
	\item eine Versionsnummer,
	\item das Datum der �nderung,
	\item die K�rzel der Personen, die die �nderungen vorgenommen haben (die Autoren),
	\item eine Beschreibung der �nderungen.
\end{itemize}

Bestimmte Daten der Versionshistorie sollen h�ufig an anderen Stellen im Dokument wiederholt werden. So soll typischerweise die Titelseite die aktuelle Versionsnummer und alle Autoren auff�hren. Die Versionsnummer sollte au�erdem auf allen Seiten des Dokuments, z.\,B. in einer Fu�zeile, wiederholt werden. Dadurch kann leicht �berpr�ft werden, ob eine Seite zur aktuellsten Version geh�rt oder schon veraltet ist.

Normalerweise werden die Daten, die z.\,B. auf der Titelseite erscheinen, nicht aus der Versionshistorie �bernommen, sondern an anderer Stelle erneut angegeben. Die Eintr�ge der Versionshistorie werden in der Regel immer aktualisiert. In der Hektik wird aber meist vergessen, die Angaben f�r Titelseite etc. zu aktualisieren. Das Ergebnis sind inkonsistente Dokumente. Aus eigener Erfahrung wei� ich, dass die Angaben zu den Autoren praktisch nie stimmen, besonders wenn im Laufe der Zeit mehrere Personen an einem Dokument gearbeitet haben.
%\index{Versionshistorie|)}

Es w�re also sch�n, wenn der Autor eines Dokuments sich nur darum k�mmern m�sste, die Versionshistorie auf dem aktuellen Stand zu halten. Die Informationen auf der Titelseite und in Fu�zeilen sollten automatisch aus der Versionshistorie generiert werden.

Diese Anforderungen sind ohne einen gewissen Aufwand nicht umzusetzen, da beispielsweise die Titelseite erzeugt wird, bevor die Versionshistorie �berhaupt gelesen wurde. Die relevanten Daten m�ssen deshalb in eine Datei geschrieben und noch vor Bearbeitung der Titelseite wieder eingelesen werden. Da f�r manche Anwendungen auch der Zeitpunkt, zu dem die aux-Datei eingelesen wird zu sp�t ist, wird eine eigene Datei mit der Endung hst angelegt.
F�r die Tabelle mit der Versionshistorie ist ebenfalls eine eigene Datei notwendig, doch dazu sp�ter.

Ein anderes Problem stellt die Liste der Autoren dar. Diese Liste kann nicht einfach durch Aneinanderreihung der Autor-Eintr�ge in der Versionshistorie erzeugt werden, da sonst einige Personen mehrfach auftreten w�rden. Dies war die Geburtsstunde des Pakets sets, mit dem einfache Mengen von Text verwaltet werden k�nnen. Die Menge aller Autoren wird bei jedem Eintrag in der Versionshistorie mit der Menge der angegebenen Autoren vereinigt. Die Menge aller Autoren kann dann in alphabetisch sortierter Form an beliebiger Stelle -- eben meist auf der Titelseite -- ausgegeben werden.

Soweit zur Vorrede. Die beiden folgenden Abschnitte beschreiben die beiden Pakete eingehender und zeigen, wie man mit ihnen arbeitet. Dabei wurde darauf verzichtet, den Quellcode der Pakete wiederzugeben. Wer sich daf�r interessiert, kann direkt in die Quellen schauen. Ich habe versucht, den Quellcode so zu strukturieren und zu kommentieren, dass er lesbar ist. 

%Au�erdem habe ich einen Trick angewandt, der sich unter \TeX{}nischen Programmierern leider kaum herumgesprochen hat; ich habe Kommentare sowohl zwischen, als auch innerhalb der Makros eingef�gt.
\clearpage
\section{Das Paket vhistory}
Sinn und Zweck dieses Pakets wurde bereits ausf�hrlich in Abschnitt \ref{einleitung} beschrieben. Ich sagte ja, dass es sich lohnt, die Einleitung zu lesen. Hier soll nun ausf�hrlich erkl�rt werden, wie man mit vhistory arbeitet.

\subsection{Laden des Pakets}\label{laden}
Das Paket wird wie �blich in der Pr�ambel mit dem Befehl\\
\mbox{}\hspace{2em}\verb|\usepackage{vhistory}|\\
geladen. Vhistory setzt \LaTeXe\ und die Pakete sets und ltxtable (welches wiederum longtable und tabularx ben�tigt) voraus. Sollten die Pakete noch nicht geladen worden sein, werden sie von vhistory automatisch geladen.

Das Paket vhistory versteht einige Optionen, um sein Verhalten anzupassen. Diese sind im Folgenden aufgelistet. Ein Aufruf mit Optionen lautet zum Beispiel\\
\mbox{}\hspace{2em}\verb|\usepackage[tocentry, owncaptions]{vhistory}|.

\paragraph{nochapter:} Ist diese Option beim Laden des Pakets angegeben worden, wird f�r die Versionshistorie kein eigenes Kapitel -- bzw. kein eigener Abschnitt, falls die Dokumentenklasse article (oder scrartcl) verwendet wird -- erzeugt.

\paragraph{tocentry:} Mit dieser Option wird veranlasst, dass die Versionshistorie im Inhaltsverzeichnis aufgef�hrt wird. Normalerweise wird dieser Eintrag nicht erzeugt. Ist die Option nochapter aktiviert hat tocentry keine Funktion. Hier sind Sie selbst f�r eventuelle Eintr�ge in das Inhaltsverzeichnis verantwortlich.

\paragraph{owncaptions:} vhistory unterst�tzt die Sprachen Deutsch und Englisch. Verwenden Sie eine nicht unterst�tzte Sprache, werden die �berschriften (z.\,B. "`Versionshistorie"' oder "`�nderungen"') in Englisch ausgegeben. In diesem Fall m�chten Sie vielleicht die �berschriften selbst ver�ndern. Die Option own\-capt\-ions unterst�tzt Sie dabei. N�heres zu diesem Thema finden Sie in Unterabschnitt \ref{sprachen}.

%%%%%%%%%%%%%%%%%%%%%%%%%%%%%%%%%%%%%%%%%%%%%%%%%%%%%%%%%%%%%%%%%%%%%%%%%%%%%%
\subsection{Verwendung}\label{verwendung}
Die Verwendung von vhistory ist denkbar einfach und soll in diesem Unterabschnitt beschrieben werden. Allgemein gilt, dass vhistory zwei Durchl�ufe ben�tigt, da Daten in Dateien geschrieben werden.

\subsubsection{Erzeugen der Versionshistorie}\label{begin}
Die Versionshistorie wird als Umgebung dargestellt:\\
\mbox{}\hspace{2em}\verb|\begin{versionhistory}|\\
\mbox{}\hspace{2em}\verb|<Eintr�ge>|\\
\mbox{}\hspace{2em}\verb|\begin{versionhistory}|

Ein Eintrag hat die allgemeine Gestalt:\\
\mbox{}\hspace{2em}\verb@\vhEntry{<Version>}{<Datum>}{<Autoren>}{<�nderungen>}@

Die Autoren werden in der Mengenschreibweise des Pakets set angegeben, d.\,h. als Trennzeichen wird \texttt{|} verwendet. Ein Eintrag k�nnte also wie folgt aussehen:\\
\mbox{}\hspace{2em}\verb@\vhEntry{1.1}{13.05.04}{JW|AK|KL}{Fehler korrigiert.}@

Durch das \verb|\begin|\ldots wird ein neues Kapitel (beziehungsweise ein neuer Abschnitt, falls article verwendet wird) begonnen, wenn dies nicht durch die Paketoption nochapter wie oben beschrieben abgeschalten wurde.

Die Versionshistorie selbst wird in eine "`ltxtable"' gesetzt. Dadurch kann die Versionshistorie auch mehrere Seiten umfassen. Die Spalten "`Autor(en)"' und "`�nderungen"' werden automatisch umgebrochen. Das Paket ltxtable setzt voraus, dass die Tabelle in einer eigenen Datei liegt. Diese Datei wird von vhistory automatisch erzeugt und hat die Endung "`ver"'.

\subsubsection{Auslesen der aktuellen Versionsnummer}
Die aktuelle Versionsnummer -- genauer: die zuletzt angegebene Versionsnummer -- kann mit dem Befehl\\
\mbox{}\hspace{2em}\verb|\vhCurrentVersion|\\
bestimmt werden. Der Befehl ist ab der Einbindung des Pakets verf�gbar.

\subsubsection{Auslesen des aktuellen Datums}
Analog zur aktuellen Versionsnummer kann auch das Datum der letzten �nderung angezeigt werden. Dies funktioniert mit dem Befehl\\
\mbox{}\hspace{2em}\verb|\vhCurrentDate|.\\
Der Befehl ist ebenfalls ab der Einbindung des Pakets verf�gbar.

\subsubsection{Liste aller Autoren ausgeben}
An eine Liste der Autoren kommen Sie auf zwei Wege. �ber das Kommando\\
\mbox{}\hspace{2em}\verb|\vhAllAuthorsSet|\\
k�nnen Sie sich die Autoren als Menge wie in Abschnitt \ref{sets} beschrieben zur�ckgeben lassen. Der weitaus einfachere Weg liegt im Kommando\\
\mbox{}\hspace{2em}\verb|\vhListAllAuthors|.

Dieses Kommando gibt eine alphabetisch sortierte Liste der Autoren aus. Die einzelnen Eintr�ge werden durch Kommata getrennt. Soll stattdessen ein anderes Trennzeichen -- beispielsweise \texttt{\&} zur Ausgabe in einer Tabelle -- verwendet werden, k�nnen sie den Befehl\\
\mbox{}\hspace{2em}\verb|\setsepararator|\\
aus dem Paket sets umdefinieren (siehe auch Abschnitt \ref{inspektoren}).

Manchmal m�chten Sie vielleicht, dass die Autorenliste komplette Namen statt K�rzeln enth�lt, zum Beispiel
"`Jochen Wer\-ten\-au\-er"' statt des K�rzels "`JW"', das in der Versionshistorie Verwendung findet. Indem Sie das Kommando\\
\mbox{}\hspace{2em}\verb|\vhListAllAuthorsLong|\\
verwenden, erhalten Sei das gew�nschte Verhalten. In diesem Fall schreiben Sie weiterhin ``JW'' im \verb|\entry|-Kommando (Hinweis: Da ist kein backslash an dieser Stelle!), definieren aber zus�tzlich das Makro \verb|\JW| wie unten beschrieben.\\
\mbox{}\hspace{2em}\verb|\newcommand{\JW}{Jochen Wertenauer}|\\
In der Versionshistorie wird weiterhin der Text ``JW'' angezeigt, aber das Kommando \verb|\vhListAllAuthorsLong| verwendet das Makro \verb|\JW|. Ist das Makro undefiniert, wird kein Text (f�r diesen Autor) ausgegeben.

%%%%%%%%%%%%%%%%%%%%%%%%%%%%%%%%%%%%%%%%%%%%%%%%%%%%%%%%%%%%%%%%%%%%%%%%%%%%%%
\subsection{Sprachunterst�tzung}\label{sprachen}
Wie schon in Abschnitt \ref{laden} erw�hnt unterst�tzt vhistory die Sprachen Deutsch und Englisch. Die sprachabh�ngigen Texte mit ihren Voreinstellungen sind in Tabelle \ref{tab:sprachen} aufgelistet. Soll das Dokument in einer Sprache verfasst werden, die vhistory nicht unterst�tzt, wird die englische Variante gew�hlt.

Mit der Paketoption "`owncaptions"' k�nnen eigene �berschriften verwendet werden. Die Option veranlasst vhistory dazu, die Kommandos, die die �berschriften enthalten, mit den Varianten der aktuell gew�hlten Sprache vorzubelegen. Deshalb ist es sinnvoll, in diesem Fall Pakete wie babel oder ngerman vor vhistory zu laden.

�ber das Kommando\\
\mbox{}\hspace{2em}\verb|\renewcommand{<Kommando>}{<gew�nschter Text>}|\\
kann eine �berschrift ver�ndert werden. Wollen Sie beispielsweise statt "`�nderungen"' lieber "`Verbesserungen"' verwenden, schreiben Sie\\
\mbox{}\hspace{2em}\verb|\renewcommand{\vhchangename}{Verbesserungen}|.

\begin{table}%[htb]
\begin{center}
\begin{tabular}{|l|l|l|}\hline
\textbf{Kommando} & \textbf{Deutsch} & \textbf{Englisch}\\ \hline
\verb|\vhhistoryname| & Versionshistorie & History of Versions \\
\verb|\vhversionname| & Version & Version \\
\verb|\vhdatename| & Datum & Date \\
\verb|\vhauthorname| & Autor(en) & Author(s) \\
\verb|\vhchangename| & �nderungen & Changes \\ \hline
\end{tabular}
\caption{sprachabh�ngige Texte}
\label{tab:sprachen}
\end{center}
\end{table}

%%%%%%%%%%%%%%%%%%%%%%%%%%%%%%%%%%%%%%%%%%%%%%%%%%%%%%%%%%%%%%%%%%%%%%%%%%%%%%
\subsection{Beispiel}
%Abbildung \ref{abb:beispiel} zeigt ein kurzes Beispiel f�r die Verwendung von vhistory.
\begin{figure}[htb]
\newcounter{line}\small
\newcommand{\bef}{\\\stepcounter{line}\hbox to 1em{\hfill\footnotesize \arabic{line}:}\quad}
\mbox{}
\bef\verb@\documentclass{scrartcl}@
\bef\verb@\usepackage{ngerman, vhistory, hyperref}@
\bef\verb@@
\bef\verb@\newcommand{\docTitle}{Ein Beispiel f\"ur vhistory}@
\bef\verb@@
\bef\verb@\hypersetup{%@
\bef\verb@  pdftitle  = {\docTitle},@
\bef\verb@  pdfkeywords = {\docTitle, Version \vhCurrentVersion@
\bef\verb@                 vom \vhCurrentDate},@
\bef\verb@  pdfauthor = {\vhAllAuthorsSet}@
\bef\verb@}@
\bef\verb@@
\bef\verb@\usepackage{scrpage2}@
\bef\verb@\pagestyle{scrheadings}@
\bef\verb@\ifoot{\docTitle\ -- Version \vhCurrentVersion}@
\bef\verb@\cfoot[]{}@
\bef\verb@\ofoot[\thepage]{\thepage}@
\bef\verb@@
\bef\verb@\begin{document}@
\bef\verb@@
\bef\verb@\title{\docTitle}@
\bef\verb@\author{\vhListAllAuthors}@
\bef\verb@\date{Version \vhCurrentVersion\ vom \vhCurrentDate}@
\bef\verb@\maketitle@
\bef\verb@@
\bef\verb@\begin{versionhistory}@
\bef\verb@  \vhEntry{1.0}{22.01.04}{JPW|KW}{Erstellung}@
\bef\verb@  \vhEntry{1.1}{23.01.04}{DP|JPW}{Fehlerkorrektur}@
\bef\verb@  \vhEntry{1.2}{03.02.04}{DP|JPW}{\"Uberarbeitung nach Review}@
\bef\verb@\end{versionhistory}@
\bef\verb@@
\bef\verb@\end{document}@

\caption{Beispiel f�r die Verwendung von vhistory}\label{abb:beispiel}
\end{figure}
\clearpage
\section{Das Paket sets}\label{sets}

Wie schon in der Einleitung beschrieben ist sets dazu konzipiert, Mengenoperationen zu unterst�tzen. Die Elemente einer Menge sind normalerweise einfacher Text, Sie k�nnen aber auch Kommandos in Mengen einf�gen. Diese werden -- au�er
bei der Ausgabe -- nicht ausgepackt. Die Verwendung von geschweiften Klammern in Mengen funktioniert leider nicht. In diesem Fall m�ssen Sie sich eine Abk�rzung definieren, die ohne Parameter auskommt. Parameter ohne Klammern funktionieren jedoch. "`\verb|H"agar|"' w�re also ein g�ltiges Element einer Menge. Ein Element "`\verb|\endset|"' darf nicht in einer Menge enthalten sein.

Da ein Dokument nur wenige Autoren hat, wurde auf Effizienz kein besonderer Wert gelegt. Die Mengen sollten deshalb relativ klein sein. Sollten Sie dennoch eine Menge mit hunderten oder gar tausenden von Elementen anlegen wollen, kann es passieren, dass der Stack von \TeX\ �berl�uft.

Normalerweise ist bei einer Menge die Reihenfolge der Elemente egal. Dies ist auch hier bei den meisten Befehlen der Fall. Bei Abweichungen wird darauf hingewiesen.

Das Paket sets ben�tigt \LaTeXe.

\subsection{Verwendung}\label{sets-verwendung}

In diesem Unterabschnitt soll die Verwendung des Pakets sets vorgestellt werden. Dabei werden einige Beispielmengen verwendet werden:
\begin{eqnarray*}
	A &=& \{Alice, Bob, Charly\}\\
	B &=& \{Alice, Bob\}\\
	C &=& \{Bob, Dean\}\\
	D &=& \{Dean\}\\
	L &=& \emptyset
\end{eqnarray*}

%-----------------------------------------------------------------------------
\subsubsection{Konstruktoren}\label{konstruktoren}
Um eine Menge anzulegen, gibt es die Befehle\\
\mbox{}\hspace{2em}\verb$\newset{<Menge>}{<Inhalt>}$\\
und\\
\mbox{}\hspace{2em}\verb$\newsetsimple{<Menge>}{<Inhalt>}$.\\
\texttt{<Menge>} ist ein Kommandoname, unter dem die Menge sp�ter erreichbar sein soll. Die Elemente einer Menge werden durch \texttt{|} getrennt. Die Menge $A$ lie�e sich also wie folgt definieren:\\
\mbox{}\hspace{2em}\verb$\newset{\mA}{Alice|Bob|Charly}$\\
Die Menge $L$ wird mit\\
\mbox{}\hspace{2em}\verb$\newset{\mL}{}$\\
definiert.

\verb$\newset$ legt also eine neue Menge an. Diese wird dabei alphabetisch sortiert und Duplikate werden entfernt. Es w�re also egal gewesen, wenn bei der Definition von $A$ nach "`Charly"' ein weiteres Mal "`Alice"' gestanden h�tte.

Der Aufwand f�r Sortierung und Duplikatentfernung ist an dieser Stelle unn�tig. Will man diese Schritte nicht durchf�hren lassen, kann man eine Menge auch mit \verb$\newsetsimple$ definieren.

Da sie sp�ter noch ben�tigt werden, werden wir nun alle oben genannten Mengen anlegen:\\
\mbox{}\hspace{2em}\verb$\newsetsimple{\mA}{Alice|Bob|Charly}$ \newsetsimple{\mA}{Alice|Bob|Charly}\\
\mbox{}\hspace{2em}\verb$\newsetsimple{\mD}{Alice|Bob}$ \newsetsimple{\mB}{Alice|Bob}\\
\mbox{}\hspace{2em}\verb$\newsetsimple{\mC}{Bob|Dean}$ \newsetsimple{\mC}{Bob|Dean}\\
\mbox{}\hspace{2em}\verb$\newsetsimple{\mD}{Dean}$ \newsetsimple{\mD}{Dean}\\
\mbox{}\hspace{2em}\verb$\newsetsimple{\mL}{}$ \newsetsimple{\mL}{}
%-----------------------------------------------------------------------------
\subsubsection{Inspektoren}\label{inspektoren}
Inspektoren dienen dazu, Eigenschaften von Mengen herauszufinden oder diese auszugeben.

\paragraph{Ausgabe:}Eine Menge l�sst sich �ber\\
\mbox{}\hspace{2em}\verb$\listset$\\
ausgeben. Die Elemente werden in der Reihenfolge, in der sie in der Menge stehen, ausgegeben. Als Trennzeichen zwischen den Elementen wird das Komma verwendet.

\verb$\listset{\mA}$ ergibt also folgende Ausgabe:\\
\centerline{\listset{\mA}}

Manchmal m�chte man die Elemente einer Menge vielleicht auf andere Art und Weise trennen, z.\,B. mit einem \texttt{\&} zur Darstellung in einer Tabelle. Hier hilft einem die (tempor�re) Umdefinition des Kommandos\\
\mbox{}\hspace{2em}\verb$\setseparator$\\
weiter. Normalerweise hat dieses Makro den Ersetzungstext `\verb*$,\ $'.

\paragraph{Gr��enbestimmung:} Der n�chste Inspektor hat die Syntax\\
\mbox{}\hspace{2em}\verb$\sizeofset{$$M$\verb$}\is{<Z�hler>}$\\
\texttt{<Z�hler>} ist dabei der Name eines \LaTeX-Z�hlers, in dem die Anzahl der Elemente der Menge $M$ gespeichert wird. Die Kommandosequenz\\
\mbox{}\hspace{2em}\verb$\newcounter{mycounter}$\newcounter{mycounter}\\
\mbox{}\hspace{2em}\verb$\sizeofset{\mB}\is{mycounter}$\sizeofset{\mB}\is{mycounter}\\
\mbox{}\hspace{2em}\verb$\arabic{mycounter}$\\
f�hrt zur Ausgabe: "`\arabic{mycounter}"'
Wird die Gr��e der Menge $L$ bestimmt, ist das Ergebnis \sizeofset{\mL}\is{mycounter}wie erwartet "`\arabic{mycounter}"'.

\paragraph{Pr�fung auf Mitgliedschaft:}Mit dem Befehl\\
\mbox{}\hspace{2em}\verb$\iselementofset{$$e$\verb$}{$$M$\verb$}$\\
kann �berpr�ft werden, ob $e \in M$ gilt. Der Aufwand ist $O(1)$, da alle Arbeit durch die Mustererkennung von \TeX\ �bernommen wird. Die Befehlssequenz\\
\mbox{}\hspace{2em}\verb$\if \iselementofset{Bob}{\mC}Ja\else Nein\fi$\\
w�rde zur Ausgabe "`\if \iselementofset{Bob}{\mC}Ja\else Nein\fi"' f�hren, der gleiche Test mit Menge $D$ zu "`\if \iselementofset{Bob}{\mD}Ja\else Nein\fi"'.

%-----------------------------------------------------------------------------
\subsubsection{Modifikatoren}\label{modifikatoren}

\paragraph{Mengenvereinigung:}Die Operation $R := M_1 \cup M_2$ wird durch den Befehl\\
\mbox{}\hspace{2em}\verb|\unionsets{|$M_1$\verb|}{|$M_2$\verb|}\to{|$R$\verb|}|\\
realisiert. Ein paar Beispiele sind in Tabelle \ref{tab:ops} aufgef�hrt. Das Ergebnis der Operation ist eine sortierte Menge ohne Duplikate, die die Elemente der Mengen $M_1$ und $M_2$ enth�lt.

\paragraph{Mengendifferenz:}Die Operation $R := M_1 - M_2$ (auch $R := M_1 \backslash M_2$ geschrieben) l�sst sich mit dem Befehl\\
\mbox{}\hspace{2em}\verb|\minussets{|$M_1$\verb|}\minus{|$M_2$\verb|}\to{|$R$\verb|}|\\
durchf�hren. Ist $M_1$ eine sortierte Menge, wird auch $R$ sortiert sein. Enth�lt $M_1$ Duplikate, enth�lt $R$ eventuell ebenfalls diese Duplikate. Tabelle \ref{tab:ops} enth�lt einige Beispiele f�r die Verwendung dieses Befehls.

Die Operation kann man umgangssprachlich wie folgt formulieren: Pr�fe f�r jedes Element $e$ aus $M_1$, ob $e \in M_2$ gilt. Wenn nein, f�ge $e$ zu $R$ hinzu. Und genau so wurde es auch straight forward implementiert!

\paragraph{Mengendurchschnitt:} Die Operation $R := M_1 \cap M_2$ wird durch den Befehl\\
\mbox{}\hspace{2em}\verb|\intersectsets{|$M_1$\verb|}{|$M_2$\verb|}\to{|$R$\verb|}|\\
erm�glicht. Auch hier gilt: Ist $M_1$ eine sortierte Menge, wird auch $R$ sortiert sein. Enth�lt $M_1$ Duplikate, enth�lt $R$ eventuell ebenfalls diese Duplikate.
Tabelle \ref{tab:ops} enth�lt auch Beispiele f�r die Verwendung dieser Operation.

Die Operation kann man umgangssprachlich wie folgt formulieren: Pr�fe f�r jedes Element $e$ aus $M_1$, ob $e \in M_2$ gilt. Wenn ja, f�ge $e$ zu $R$ hinzu. Wenn man dies mit der Formulierung der Mengendifferenz vergleicht, f�llt auf, dass der einzige Unterschied im W�rtchen "`ja"' besteht. In der Implementierung dr�ckt sich dies durch ein fehlendes \verb|\else| aus. Eigentlich verbl�ffend einfach, wenn man bedenkt, dass formal $M_1 \cap M_2 \equiv M_1 \backslash (M_1\backslash M_2)$ gilt, was eine deutlich h�here Komplexit�t erwarten l�sst.

\begin{table}%[htb]
\begin{center}
\begin{tabular}{|l|l|}\hline
\textbf{Befehl} & \textbf{Ergebnis} \\ \hline 
\unionsets{\mA}{\mC}\to{\mR}\global\let\mR\mR
	\verb$\unionsets{\mA}{\mC}\to{\mR}$ & "`\listset{\mR}"' \\ 
\unionsets{\mB}{\mD}\to{\mR}\global\let\mR\mR
	\verb$\unionsets{\mB}{\mD}\to{\mR}$ & "`\listset{\mR}"' \\
\unionsets{\mL}{\mC}\to{\mR}\global\let\mR\mR
	\verb$\unionsets{\mL}{\mC}\to{\mR}$ & "`\listset{\mR}"' \\
\unionsets{\mL}{\mL}\to{\mR}\global\let\mR\mR
	\verb$\unionsets{\mL}{\mL}\to{\mR}$ & "`\listset{\mR}"' \\\hline
%
\minussets{\mA}\minus{\mC}\to{\mR}\global\let\mR\mR
	\verb$\minussets{\mA}\minus{\mC}\to{\mR}$ & "`\listset{\mR}"' \\ 
\minussets{\mD}\minus{\mC}\to{\mR}\global\let\mR\mR
	\verb$\minussets{\mD}\minus{\mC}\to{\mR}$ & "`\listset{\mR}"' \\
\minussets{\mD}\minus{\mB}\to{\mR}\global\let\mR\mR
	\verb$\minussets{\mD}\minus{\mB}\to{\mR}$ & "`\listset{\mR}"' \\
\minussets{\mA}\minus{\mL}\to{\mR}\global\let\mR\mR
	\verb$\minussets{\mA}\minus{\mL}\to{\mR}$ & "`\listset{\mR}"' \\\hline
%
\intersectsets{\mA}{\mB}\to{\mR}\global\let\mR\mR
	\verb$\intersectsets{\mA}{\mB}\to{\mR}$ & "`\listset{\mR}"' \\ 
\intersectsets{\mC}{\mB}\to{\mR}\global\let\mR\mR
	\verb$\intersectsets{\mC}{\mB}\to{\mR}$ & "`\listset{\mR}"' \\
\intersectsets{\mB}{\mD}\to{\mR}\global\let\mR\mR
	\verb$\intersectsets{\mB}{\mD}\to{\mR}$ & "`\listset{\mR}"' \\
\intersectsets{\mA}{\mL}\to{\mR}\global\let\mR\mR
	\verb$\intersectsets{\mA}{\mL}\to{\mR}$ & "`\listset{\mR}"' \\\hline
\end{tabular}
\caption{Mengenoperationen, Beispiele}
\label{tab:ops}
\end{center}
\end{table}

\paragraph{Sortieren:} Eine Menge $M$ kann alphabetisch sortiert werden. Dazu verwendet man den Befehl\\
\mbox{}\hspace{2em}\verb|\sortset{|$M$\verb|}{|$R$\verb|}|.\\
$R$ enth�lt danach die sortierte Menge. Die Sortierung erfolgt nach dem Sortierverfahren Bubblesort, einem Verfahren, das sich auch mit \TeX\ ohne gr��ere Verrenkungen umsetzen l�sst.

Bei der Sortierung werden die Elemente so verglichen, wie Sie sie angegeben haben, d.\,h. eventuell enthaltene Kommandos werden nicht expandiert, sondern nach ihrem Namen verglichen (inklusive des Backslash).

\paragraph{Duplikatentfernung:} Diese Operation funktioniert \emph{nur} auf sortierten Mengen! Man ben�tigt sie aber eigentlich auch kaum, da die Erstellung einer neuen Menge mit \verb|\newset| diese Aufgabe automatisch �bernimmt (indem sie dieses Makro verwendet). Ich habe mich aber dazu entschlossen, die Operation trotzdem verf�gbar zu machen; vielleicht ben�tigt sie ja tats�chlich einmal jemand.
Aufgerufen wird die Duplikateleminierung mit:\\
\mbox{}\hspace{2em}\verb|\deleteduplicates{|$M$\verb|}{|$R$\verb|}|,\\
wobei $R$ die Ergebnis-Menge darstellt und $M$ die sortierte Menge, deren Duplikate entfernt werden sollen.

%------------------------------------------------------------------------------
\subsection{Aufwandsabsch�tzung}
Tabelle \ref{tab:komplexitaet} beschreibt die Komplexit�t der Operationen in O-Notation. Dabei werden folgende Annahmen getroffen:
\begin{itemize}
  \item Die L�nge eines Elements einer Menge sei $m$.
  \item Die Anzahl der Elemente einer Menge sei $n$. Werden f�r eine Operation zwei Mengen ben�tigt, gibt $n_1$ die Kardinalit�t der ersten und $n_2$ die Kardinalit�t der zweiten Menge an.
  \item Zur Vereinfachung sei der Aufwand f�r die Mustererkennung bei der Parameter�bergabe konstant.
\end{itemize}

Die angegebenen Komplexit�tsklassen k�nnen dazu dienen, eine Reihe von Mengenoperationen m�glichst g�nstig anzuordnen. Beispielsweise ist es besser, bei \verb@\intersectsets@ und \verb@\minussets@ als erste Menge die kleinere Menge zu �bergeben.

\begin{table}[htb]
  \begin{center}
  \begin{tabular}{|l|c|}\hline
    \textbf{Operation} & \textbf{Komplexit�tsklasse}\\ \hline
    Element-Vergleich & $m$\\
    \verb@\sizeofset@ & $n$\\
    \verb@\listset@ & $n$\\
    \verb@\iselementofset@ & $1$\\%\hline
    \verb@\sortset@ & $m \cdot n^2$\\
    \verb@\deleteduplicates@ & $n$\\%\hline
    \verb@\newset@ & $m \cdot n^2$\\
    \verb@\newsetsimple@ & $1$\\%\hline
    \verb@\unionsets@ & $m \cdot (n_1 + n_2)^2$\\
    \verb@\intersectsets@ & $n_1$\\
    \verb@\minussets@ & $n_1$\\ \hline
  \end{tabular}
  \caption{Komplexit�tsklassen der Mengenoperationen}%
  \label{tab:komplexitaet}%
  \end{center}
\end{table}

%\printindex %Index

\end{document} 
