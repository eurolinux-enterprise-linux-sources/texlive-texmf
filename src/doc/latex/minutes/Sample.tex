\def\minfileversion{V1.7}     %^^Aof minutes.sty
\def\minfiledate{2001/09/19}   %^^Aof minutes.sty
%%
%% \CharacterTable
%%  {Upper-case    \A\B\C\D\E\F\G\H\I\J\K\L\M\N\O\P\Q\R\S\T\U\V\W\X\Y\Z
%%   Lower-case    \a\b\c\d\e\f\g\h\i\j\k\l\m\n\o\p\q\r\s\t\u\v\w\x\y\z
%%   Digits        \0\1\2\3\4\5\6\7\8\9
%%   Exclamation   \!     Double quote  \"     Hash (number) \#
%%   Dollar        \$     Percent       \%     Ampersand     \&
%%   Acute accent  \'     Left paren    \(     Right paren   \)
%%   Asterisk      \*     Plus          \+     Comma         \,
%%   Minus         \-     Point         \.     Solidus       \/
%%   Colon         \:     Semicolon     \;     Less than     \<
%%   Equals        \=     Greater than  \>     Question mark \?
%%   Commercial at \@     Left bracket  \[     Backslash     \\
%%   Right bracket \]     Circumflex    \^     Underscore    \_
%%   Grave accent  \`     Left brace    \{     Vertical bar  \|
%%   Right brace   \}     Tilde         \~}
%%
\documentclass{report}
%%\documentclass{article}
%%\documentclass{scrreprt}
%%\documentclass{scrarctl}
\usepackage[english,german]{babel}
\usepackage{minutes}

%%\usepackage{Evntlist}%\usepackage{Calendar}
\title{Collection of minutes\\Protokollsammlung\\[1cm]
minutes.sty}
\author{\LaTeXe}
\minutesstyle{
    header   = {list}, %or {table},
    vote     = {list}, %or {table},
    contents = {true}, %or {false}
}
\begin{document}
\maketitle
\tableofcontents
\selectlanguage{english}
\inputminutes{SampleEN}%A english minutes
\selectlanguage{german}
\inputminutes{SampleDE}%Ein deutsches Protokoll
%%-----------------
\appendix
\chapter{Appendix}
\selectlanguage{english}
\section{List of decisions in sample.tex}\listofdecisions
\section{List of open tasks from sample.tex}\listoftasks
\section{list of attachments in sample.tex}\listofattachments
%%\selectlanguage{german}
%%\section{Liste der Beschl{\"u}sse}\beschlussliste
%%\section{Liste offener Aufgaben}\aufgabenliste
%%\section{Liste der Anh{\"a}nge}\anhangsliste
\section{Calendar}
\prepareCal
%%\begin{eventlist}{}{Sample}
%%1 dec 2000 to 31 dec 2000
%%\end{eventlist}
\end{document}
\endinput
%%
%% End of file `Sample.tex'.
