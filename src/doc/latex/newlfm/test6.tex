\documentclass[10pt]{newlfm}

\newlfmP{leftmarginsize=1in,rightmarginsize=1in,addrtoskipbefore=72pt,%
stdletter,dateno,noheadline,topmarginskip=.6in,leftmarginskipleft=35pt,%
leftmarginskipright=.35in,Avery5163,labsize=\small,labrowfrto,addrf=LVB}%

\closeline{Yours in fontal health,}
\Cfooter{\parbox{3in}{{\Large\bfseries\itshape Your font is my font!!}}}
\greetto{Tovarich Pyotr:}
\encllist{Document}
\cclist{George, John, Paul, Ringo}

\makelabels

\setcounter{errorcontextlines}{20}

\begin{document}
\ltrbody{{\tailor} In terms of the relationships between font descriptions and
 musical typography, it seems to me that only in the case of the early
 1800s can even the slightest case be made for a relationship.}
\providecommand{\tailor}{}
\newcounter{tx}
\setcounter{tx}{0}
\whiledo{\thetx < 7}{ 
\addtocounter{tx}{1}
\renewcommand{\tailor}{\thetx: A line that will differ from case to case.}
\oneletter{PIT}
\renewcommand{\tailor}{\thetx: Of course, it doesn't have to, but it will anyway.}%
\oneletter{CCR}
}

\end{document}
