\documentclass[11pt]{article}
\usepackage[tight]{chemcono}
%\usepackage{drftcono}
%\usepackage{showkeysff}
%\renewcommand{\fcite}[1]{\underline{\ffcite{#1}}}

\newcommand{\grade}{$\,^{\circ}$}
\setlength{\parindent}{0pt}
\setlength{\parskip}{5pt plus 2pt minus 1pt}

\begin{document}
\begin{center}
\fbox{\textsf{Uncomment packages in the preamble to see different output.}}

\section*{Composition of the Pheromone System of the male Danaine Butterfly, \emph{Idea
leuconoe}}

S. Schulz$^{*,a}$, R. Nishida$^{b}$ \\ $^{a}$Institute of Organic Chemistry, TU
Braunschweig, Hagenring 30, D-38106~Braunschweig, Germany, phone +49-531-391 7353,
email:~stefan.schulz@tu-bs.de;
\\$^{b}$Pesticide Research Institute, Kyoto University, Kyoto, 606-01, Japan
\end{center}

\begin{center}
Keywords: pheromones, Danainae, lipids, \emph{Idea}, lactones
\end{center}
\begin{center}
\textbf{Abstract}
\end{center}

 Male \emph{Idea leuconoe} butterflies release a complex mixture
of volatiles from their pheromone glands (hairpencils) during courtship. The pheromone
components geranyl methyl thioether (\fcite{f2}), viridifloric-$\beta$-lactone
(\fcite{f3}), and 6-hydroxy-4-dodecanolide (\fcite{f8}) have been synthesized for the
first time. Therefore the structural assignment of these new natural products could be
proved. Related 7-hydroxy-5-alkanolides are also present in the extract. The volatiles
are embedded in a lipidic matrix with more than 150 components. This matrix consists of
alkanes, alkenes, 2,5-dialkyltetrahydrofurans, secondary alkanols and alkenols as well as
alkanones and alkenones. Several regioisomers of the oxidized hydrocarbons occur. The
elucidation of  double bond  positions has been performed by MS using DMDS adducts.

\begin{center}
\textbf{Introduction}

\end{center}
 Male Danaine butterflies possess  striking evertible
pheromone glands, so called hairpencils,  which are used during courtship \cite{c1}.
About 30 years ago the first pheromone component of these butterflies, danaidone
(\fcite{f1}), could be identified \cite{c2} and its function as courtship pheromone
proved  \cite{c3}. Since then, many danaine species have been investigated and the
chemical composition of their male pheromone glands elucidated \cite{c1,c4,c7}. Despite
the fact that some of the scent bouquets  consist of up to 60 components, no pheromonal
function of any other component than \fcite{f1} has been established.


Recently we were able to show that  male \emph{Idea leuconoe} butterflies emit  a complex
mixture of chemicals from their hairpencils. At least three of these components,
danaidone (\fcite{f1}), geranyl methyl thioether (\fcite{f2}), and
(\emph{S,S})-viridifloric-$\beta$-lactone (\fcite{f3}), act as courtship  pheromones
\cite{c5}.  An artificial mixture of hairpencil compounds containing \fcite{f1},
\fcite{f2}, and \fcite{f3} as well as phenol, \emph{p}-cresol, benzoic acid,  a series of
homologue 6-hydroxy-4-alkanolides ranging from C$_{10}$ to C$_{13}$,
(\emph{E,E})-farnesol,  and (\emph{Z})-9-tricosene elicited the same courtship behavior
as a crude hairpencil extract \cite{c5}. In addition, (--)-(\emph{R})-mellein
(8-hydroxy-3-methyl-3,4-dihydroisocoumarin) and another $\beta$-lactone related to
\fcite{f3}, 2-ethyl-2-hydroxy-3-butanolide, could be identified  in the hairpencil
extracts. Besides their function as courtship pheromones, other types of interactions,
like a defensive warning odor or intermale  recognition (for a discussion, see
\cite{c5}), seem also to be associated with these chemicals.  The volatile compounds
identified are embedded into a complex lipidic matrix (see Figure~\ref{f2}).
\marginpar{fig.~\ref{f2} here} In this paper we will report on the synthesis of some
hairpencils constituents, the identification of additional   components, and the
composition of the lipid matrix.

\begin{figure}[h!]
  \centering
  \caption{Figure}\label{f2}
\end{figure}

\begin{center}
\textbf{Results}
\end{center}
The structure of the previously unknown geranyl methyl thioether (\fcite{f2}) could be
deduced from its mass spectrum (see Figure~\ref{f3}). \marginpar{fig.~\ref{f3} here}The
presence of sulfur was detected by high-resolution mass spectroscopy, giving a molecular
formula of C$_{11}$H$_{20}$S
 (M$^{+}_{obs}$ = 184.1289, M$^{+}_{calc}$ = 184.1286).
Ions at \emph{m/z} = 47 (CH$_{3}$S$^{+}$) and 61 (CH$_{3}$SCH$_{2}^{+}$) and the loss of
48 amu (CH$_{3}$SH)  from  M$^{+}$ indicated a thiomethyl group in the molecule. Typical
terpenic ions (\emph{m/z} = 41, 69, 81, 93, 123, and 136) suggested that this molecule
could be geranyl methyl thioether (\fcite{f2}).  For an unambiguous proof, \fcite{f2} was
synthesized by reaction of geranyl chloride with sodium methanolate in boiling ethanol.
The product showed identical mass spectra and gaschromatographic retention times on
different stationary phases as the natural compound which is therefore
(\emph{E})-2,6-dimethyloctadienyl methyl thioether (\fcite{f2}). The corresponding
(\emph{Z})-isomer exhibited a similar mass spectrum, but a shorter retention time on an
apolar phase.


Another pheromone component not reported from nature before is
2-hydroxy-2-(1-methylethyl)-3-butanolide (viridifloric  $\beta$-lactone, \fcite{f3}). Its
identification basing on MS-, IR- and NMR-data has been described \cite{c5}.  A racemic
mixture of both diastereomers of \fcite{f3} was synthesized for structural assignment
according to Figure 4.

\begin{thebibliography}{99}
\bibitem{c1}Ackery, P. R.; Vane-Wright, R. I. \emph{Milkweed
Butterflies: Their Cladistics and Biology}; British Museum (Natural
History): London, 1984.
\bibitem{c2}Meinwald, J.; Meinwald, Y. C.
\emph{J. Am. Chem. Soc.}  \textbf{1966}, \emph{88}, 1305.
\bibitem{c3}Pliske, T. E.; Eisner, T. \emph{Science} \textbf{1969},
\emph{164}, 1170.
\bibitem{c4}Schulz, S.; Boppr\'e, M.; Vane-Wright, R. I.
\emph{Phil. Trans. R. Soc. Lond. B}  \textbf{1993}, \emph{342}, 161.
\bibitem{c7}
Schulz, S.; Francke, W.; Edgar, J.; Schneider, D. \emph{Z. Naturforsch.} \textbf{1988},
\emph{43c}, 99;  Schulz, S.; Francke, W.;  Boppr\'e, M. \emph{Biol.\ Chem.\ Hoppe-Seyler}
\textbf{1988}, \emph{389}, 633; Francke, W.; Bartels, J.; Krohn, S.; Schulz, S.; Baader,
E.; Teng\"o, J.; Schneider, D. \emph{Pure Appl. Chem.} \textbf{1989}, \emph{61}, 539;
Francke, W.; Schulz, S.; Sinnwell, V.; K\"onig, W. A.;  Roisin, Y. \emph{Liebigs Ann.
Chem.} \textbf{1989}, 1195.
\bibitem{c5}Nishida, R.; Schulz, S.; Kim, C. H.; Fukami, H.; Kuwahara,
Y.; Honda, K.; Hayashi, N. \emph{J. Chem. Ecol.} \textbf{1995}, submitted.
\end{thebibliography}

\clearpage
\begin{theffbibliography}{99}
\ffbibitem{f1} danaidone
\ffbibitem{f2} geranyl methyl thioether
\ffbibitem{f3} viridifloric acid lactone
\ffbibitem{f5} ho but sre
\ffbibitem{f6} butensre
\ffbibitem{f7} virid. acid
\ffbibitem{f12} oxobutester
\ffbibitem{f11} hexenolid
\ffbibitem{f9} oxo olid
\ffbibitem{f8} 6-ho-c12-g-lactone
\end{theffbibliography}
 \end{document}
