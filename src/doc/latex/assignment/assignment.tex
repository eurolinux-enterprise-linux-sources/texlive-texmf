\documentclass{assignment}
\coursetitle{Creating assignments}
\courselabel{ASG 101}
\exercisesheet{Home Work 1}{Documentation}
\student{Madhusudan Singh}
\semester{Summer 2004}
\date{July 14, 2004}
%\usepackage[pdftex]{graphicx}
%\usepackage{subfigure}

\begin{document}

\begin{problemlist}
\pbitem What is this package used for ?

\begin{problem}

This package may be used to typeset homework assignments.

\begin{answer}
assignment.cls $\Rightarrow$ Typesetting homework assignments
\end{answer}

\end{problem}

\pbitem What does the preamble contain ?
\begin{problem}

The preamble may contain the following declarations\footnote{Current markup's preamble.} :

\begin{verbatim}
\documentclass{assignment}
\coursetitle{Creating assignments}
\courselabel{ASG 101}
\exercisesheet{Home Work 1}{Documentation}
\student{Madhusudan Singh}
\semester{Summer 2004}
\date{July 14, 2004}
%\usepackage[pdftex]{graphicx}
%\usepackage{subfigure}
\end{verbatim}

\begin{answer}
That was the preamble
\end{answer}

\end{problem}

\pbitem How do I use various features of the article class ?

\begin{problem}

Its possible. Just pass the options in the preamble :

\begin{verbatim}
\documentclass[option1,option2, ...]{assignment}
\end{verbatim}

\begin{answer}
\end{answer}

\end{problem}
\pbitem How are the equations numbered ?

\begin{problem}

Equation numbers refer to the problem number. For instance,

\begin{verbatim}
\begin{eqnarray}
E & = & mc^{2} \label{eqn:emc2} \\
\textrm{That is how equations are numbered} \ldots \label{eqn:numbered} \\
\textrm{Or not numbered} \ldots \nonumber
\end{eqnarray}
\end{verbatim}

\begin{eqnarray}
E & = & mc^{2} \label{eqn:emc2} \\
\textrm{That is how equations are numbered} \ldots \label{eqn:numbered} \\
\textrm{Or not numbered} \ldots \nonumber
\end{eqnarray}


\begin{answer}
That was that.
\end{answer}
\end{problem}

\pbitem How do the answer sections look ?

\begin{problem}

Nice !

\begin{verbatim}
\begin{answer}
\begin{eqnarray}
Answer=f(bold) \nonumber
\end{eqnarray}
\end{answer}
\end{problem}
\end{verbatim}


\begin{answer}
\begin{eqnarray}
Answer=f(bold) \nonumber
\end{eqnarray}
\end{answer}
\end{problem}


\pbitem Can one have more than one answer section for the problem ?

\begin{problem}

Most definitely.

Certain problems have many parts :

\begin{enumerate}
\item Part 1

\begin{answer}
Answer to part one.
\end{answer}

\item Part 2

\begin{answer}
Answer to part two.
\end{answer}

\end{enumerate}

\begin{verbatim}
\begin{enumerate}
\item Part 1

\begin{answer}
Answer to part one.
\end{answer}

\item Part 2

\begin{answer}
Answer to part two.
\end{answer}

\end{enumerate}

\end{verbatim}

\end{problem}


\pbitem What are the copyright conditions for this class file ?

\begin{problem}

This material is subject to the \LaTeX\ Project Public License. See http://www.ctan.org/tex-archive/help/Catalogue/licenses.lppl.html for the details of that license. See the LICENSE file for more details.


\begin{answer}
\begin{eqnarray}
&& Q. E. D. \nonumber
\end{eqnarray}
\end{answer}

\end{problem}


\pbitem How do I get help using this class ?
\begin{problem}


You may post your queries on comp.text.tex . I check it fairly regularly.
\end{problem}

\end{problemlist}

\end{document}
