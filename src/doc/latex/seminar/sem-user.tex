%% BEGIN sem-user.tex
\def\FileDate{93/04/01}
\def\FileVersion{1.0}
%%
%% COPYRIGHT 1993, by Timothy Van Zandt, Timothy.VAN-ZANDT@insead.edu
%%
%%
%% This file may be distributed and/or modified under the conditions of
%% the LaTeX Project Public License, either version 1.2 of this license
%% or (at your option) any later version.  The latest version of this
%% license is in:
%% 
%%    http://www.latex-project.org/lppl.txt
%% 
%% and version 1.2 or later is part of all distributions of LaTeX version
%% 1999/12/01 or later.
%%
%%
%%
%% This LaTeX file prints the User's Guide for seminar.sty.
%% You must have my tvz-hax.sty, tvz-user.sty, and fancybox.sty,
%% which are distributed with seminar.sty.
%%
%% This is distributed with the index file sem-user.ind.
%% To make an new index, run, e.g.:
%%   makeindex sem-user.idx
%%
%% STYLE OPTIONS:
%%   `a4'       : For A4 paper.
%%   `twoside'  : For two-sided printing.
%%   `2up'      : For two-up printing. (Uses non-standard font magnifications.
%%                                      See 2up.doc for details.)
%%

\documentstyle[12pt]{tvz-user}

\makeindex

% Used with twoside option:
\def\theheadertitle{seminar.sty: User's Guide}

% No headers, 1in top margin
\setlength{\topmargin}{0pt}
\setlength{\headheight}{0pt}
\setlength{\headsep}{0pt}
\setlength{\textheight}{8.75in}
\setlength{\footskip}{.625in}
\papersizeadjust

% Change part environment:
\renewcommand{\part}[1]{%
  \clearpage
  \refstepcounter{part}
  \addcontentsline{toc}{part}{\protect\numberline{\thepart}#1}%
  \hrule height 1pt\relax
  \vskip 1ex
  \hbox to\columnwidth{\strut\huge\bf\thepart\hfil#1}%
  \vskip 1ex
  \hrule height 1pt\relax
  \vskip 3ex}

\def\MainFont{\tt}            % For macro definitions.

\ShortVerb
\ShortMeta

\let\filedate\FileDate

\begin{document}

\thispagestyle{empty}

\null
\vfill

\begin{center}
  {\huge\bf seminar.sty}\\[9pt]
  {\LARGE\bf A \bLaTeX\ style for slides and notes}\\[14pt]
  {\LARGE\bf User's Guide}\par
  \vskip 1.5em
  \large
  \renewcommand{\thefootnote}{\fnsymbol{footnote}}
  Timothy Van Zandt\\
  {\normalsize\tt Timothy.VAN-ZANDT@insead.edu}
  \vskip 1em
  \thefiledate\\ Version \FileVersion\par
\end{center}
\par
\setlength{\unitlength}{1cm}
\thicklines
\centerline{\lower.8in\hbox{%
\begin{picture}(0,0)
\put(2,-11){\framebox(4,2){}}
\put(2,-11){\line(-2,3){1}}
\put(2,-9){\line(-2,3){1}}
\put(6,-9){\line(-2,3){1}}
\put(1,-9.5){\line(0,1){2}}
\put(1,-7.5){\line(1,0){4}}
\put(5.8,-8.7){\line(0,1){2}}
\put(5.0,-6.7){\oval(1.6,1.6)[tr]}
\put(5.0,-5.9){\line(-1,0){.7}}
\put(3.1,-6.4){\framebox(1.2,1){}}
\put(3.1,-6.4){\line(-1,2){.3}}
\put(3.1,-5.4){\line(-1,3){.3}}
\put(4.3,-5.4){\line(-1,3){.3}}
\put(2.8,-5.8){\line(0,1){1.3}}
\put(2.8,-4.5){\line(1,0){1.2}}
\put(0,6.2){\oval(14,10)}
\put(.02,6.18){\oval(14,10)}
\thinlines
\put(2.8,-5.8){\line(-4,3){9.6}}
\put(2.8,-4.5){\line(-3,5){9.35}}
\put(4,-4.5){\line(1,6){2.6}}
\put(6.8,1.4){\line(-1,-3){2.5}}
\end{picture}}}
\par
\bigskip
\rightskip=0pt plus 2em\relax
\parshape=20
0cm 2.3cm
0cm 2.9cm
0cm 3.5cm
0cm 4.1cm
0cm 4.7cm
0cm 5.3cm
0cm 5.9cm
0cm 6.5cm
0cm 7.1cm
0cm 7.7cm
0cm 8.3cm
0cm 8.3cm
0cm 8.3cm
0cm 7.3cm
0cm 7.3cm
0cm 7.3cm
0cm 7.3cm
0cm 7.3cm
0cm 7.3cm
0cm 7.3cm
seminar.sty is a \LaTeX\ style for typesetting slides or
transparencies, and accompanying notes. Here are some of its special features:
It is compatible with \AmS-\LaTeX, and you can use PostScript and \AmS{}
fonts. Slides can be landscape and portrait. There is support for color and
frames. The magnification can be changed easily.
Overlays can be produced from a single slide environment. Accompanying notes,
such as the text of a presentation, can be put
    outside the slide environments. The slides, notes or both together
    can then be typeset in a variety of formats.

\vfill

\twosideclearpage
\pagenumbering{roman}
\tableofcontents

\twosideclearpage

\section*{Getting Started}
\addcontentsline{toc}{section}{Getting Started}
\markboth{Getting Started}{Getting Started}

"seminar.sty" is a \LaTeX{} document style for typesetting slides, and more.

You should know how to use \LaTeX, as described in Leslie Lamport's \LaTeX:
{\em User's Guide and Reference Manual}.

Let's get started:
\begin{enumerate}
\item If you are installing "seminar.sty", read the the accompanying read-me
file, and put the input files where your \TeX\ looks for inputs.

\item Typeset the sample file, "semsamp1.tex", to see that everything is
working. You can use \LaTeX\ or \AmS-\LaTeX.

\item Read Appendix \ref{short-fonts}, on page \pageref{short-fonts}.

\item To start making landscape slides, use
\begin{LVerbatim}
  \documentstyle{seminar}
  \begin{document}
  \begin{slide}
    foo
  \end{slide}
  \end{document}
\end{LVerbatim}
and print out your document in landscape mode.

\item To start making portrait slides, include the \o{portrait} style option,
and use the \e{slide*} environment instead of \e{slide}:
\begin{LVerbatim}
  \documentstyle[portrait]{seminar}
  \begin{document}
  \begin{slide*}
    foo
  \end{slide*}
  \end{document}
\end{LVerbatim}

\item If you have used \SliTeX, see Section \ref{slitex}.

\item For A4 paper, use the "a4" style option.

\item When you are ready to explore "seminar.sty"'s special features, skim the
{\em User's Guide}, including the appendices.

\item Play around with the sample file "semsamp2.tex" to try out some of
"seminar.sty"'s special features.

\item When you run into problems, look for help in Part \ref{Help}.

\end{enumerate}

\twosideclearpage

\pagenumbering{arabic}

\part{Just slides}

\section{Landscape and portrait slides}

"seminar.sty" is a \LaTeX\ (or \AmS-\LaTeX) document style. Thus, begin your
document with
\begin{LVerbatim}
  \documentstyle{seminar}
\end{LVerbatim}

The slide environments are\MainEnvIndex{slide}\MainEnvIndex{slide*}%
\begin{LVerbatim}
  \begin{slide}    ...    \end{slide}
  \begin{slide*}   ...    \end{slide*}
\end{LVerbatim}
\e{slide} is for landscape slides and \e{slide*} is for portrait slides.

By default, the document is typeset in landscape mode, but if you include the
\O{portrait} style option, the document is typeset in portrait mode.
Typesetting the document in landscape mode is different from printing it in
landscape mode; you may have to take care of the latter when printing with
your dvi driver (see Appendix \ref{S-landscape}).

If you have both landscape and portrait slides in your file, there are two
ways to print the slides:
\begin{itemize}
  \item If you dvi driver supports rotation, then you can print all the slides
at once. See Appendix \ref{S-landscape} for details.
  \item You can first print your landscape slides by putting the command
\begin{MD}
  \landscapeonly
\end{MD}
in the preamble, and then print the portrait slides by inserting instead the
command
\begin{MD}
  \portraitonly
\end{MD}
and including the \o{portrait} style option.
\end{itemize}

\section{The height and width of slides\label{slidedim}}

The dimensions of the slides are set by the lengths
\begin{MD}
  \slidewidth\\
  \slideheight
\end{MD}
The ``width'' refers to the width of a slide when looking at it in landscape
orientation, whether it is a landscape or portrait slide (and the same goes
for ``height''). The default width is 8.5 inches and the default height is 6.3
inches.

The slide environments have an optional argument that lets you change the
dimensions of a single slide, as in
\begin{LVerbatim}
  \begin{slide*}[7.5in,6in]
\end{LVerbatim}
The first dimension is the slide's ``width'', and the second dimension is the
slides ``height''. If you remember what we said about what ``width'' and
``height'' mean, you will see that the above example begins a portrait slide
that is 7.5 inches high and 6 inches wide.


\section{Margins within a slide}

When a slide is not full, the material is vertically centered within the
slide. The command
\begin{MD}
  \centerslidesfalse
\end{MD}
cause the material to be flush to the top, instead. The command
\begin{MD}
  \centerslidestrue
\end{MD}
switches back to vertical centering.

The right margin in slides is ragged by default. You can change this with the
command
\begin{MD}
  \raggedslides[len]
\end{MD}
<len> should be the maximum space between the end of the line and the right
margin. The argument is optional; \n\raggedslides\ is equivalent to
\begin{LVerbatim}
  \raggedslides[1fil]
\end{LVerbatim}
which gives a ragged right margin, as in \LaTeX's "flushleft" environment (the
default). On the other hand,
\begin{LVerbatim}
  \raggedslides[0pt]
\end{LVerbatim}
gives a justified margin, and
\begin{LVerbatim}
  \raggedslides[2em]
\end{LVerbatim}
gives a semi-ragged margin. Note that as a margin becomes less ragged,
hyphenation becomes more likely and more material fits on a slide.

\section{Page breaking within a slide environment\label{S-pagebreak}}

A slide environment can contain more than one ``page'' of slides; \TeX\ will
break slides into pages automatically.

If the mere idea disturbs you, put the command
\begin{LVerbatim}
  \extraslideheight{10in}
\end{LVerbatim}
in your document. You can then divide slides into pages yourself by starting
new slide environments or using the
\begin{MD}
  \newslide
\end{MD}
command within a slide environment. And you need read this section no further.

If instead you use the command
\begin{LVerbatim}
  \extraslideheight{0pt}
\end{LVerbatim}
then \TeX\ will break pages the way you would expect. This is a quick and
dirty way to break a whole paper or a long proof into slides.

However, dividing material into slides is usually too delicate a matter to be
left up to \TeX, and ultimately you will make all the page breaks yourself. On
the other hand, automatic page breaking can still be helpful at the early
stages, letting \TeX\ find preliminary page breaks for you.

Setting the "\extraslideheight" to "0pt" doesn't give you much flexibility
about where to put the page breaks. Of course, you can always put a page break
earlier than the one found by \TeX, but occasionally you will prefer to let a
slide overflow by a small amount rather than rewrite the whole slide.

Therefore, by default, "seminar.sty" uses\MainIndex{\extraslideheight}%
\begin{LVerbatim}
  \extraslideheight{10pt}
\end{LVerbatim}
This adds on an extra "10pt" to the target slide height (as determined by
"\slideheight" for landscape slides and "\slidewidth" for portrait slides)
{\em for the purpose of page-breaking only}.

If the resulting slide exceeds the slide height, you will get a message like
\begin{LVerbatim}
  LaTeX Error: Slide 3 overfull by 9pt.
\end{LVerbatim}
"seminar.sty" then tries to reduce the slide to its maximum height by
squeezing out rubber vertical space (e.g., tightening up the interline
spacing). If the slide is still too full, you will get another warning like:
\begin{LVerbatim}
  Overfull \vbox by 4.4pt while output was active.
\end{LVerbatim}
You can go back and look for a way to make the slide shorter, or you can
insert a "\newslide" command to change the page-break, or you can just ignore
these warnings.

The recommended value to give in the argument of "\extraslideheight" is the
largest length by which you might be willing to let  a slide overflow.

After you have decided on the page breaks, you can shut up the warnings about
overfull slides and "\vbox"'s with
\begin{LVerbatim}
  \renewcommand{\slidefuzz}{1in}
\end{LVerbatim}
The value of \n\slidefuzz\ (which should be length, even though \n\slidefuzz\
is an ordinary command sequence) is the threshold above which "seminar.sty"
gives a warning about an overfull slide. The default definition is "2pt".


\section{Margins on the page\label{S-slidemargins}}

After "seminar.sty" makes a slide (or a slide page) as described above, it
(optionally) frames the slide. This is described in Section \ref{S-frames}.
For making and framing the slide, "seminar.sty" does not need to know anything
about the paper you are printing on, or the margins you want. Hence, none of
\LaTeX's standard page parameters are relevant within a slide environment.
(However, for the sake of consistency, "seminar.sty" sets "\textwidth" to the
width of the slide and "\textheight" to the height of the slide within a slide
environment.)

Now "seminar.sty" has to do something with the finished slide. As you read
this User's Guide, you will find that there are various options. However,
right now we are making slides for printing on transparencies, and so we have
to position the slide on the transparency and add headers and footers.
"seminar.sty" does not use \LaTeX's page parameters for this either. Instead,
it uses the following parameters:
\begin{center}
  \begin{tabular}{lc}
  {\em Parameter:} & \em Default:\\
    \N\slideleftmargin & .5in\\
    \N\sliderightmargin & .5in\\
    \N\slidetopmargin & .5in\\
    \N\slidebottommargin & .5in
  \end{tabular}
\end{center}
There are commands, to be set with "\renewcommand".\footnote{The only true
length parameters (meaning that thay should be set with "\setlength") in
"seminar.sty" are \n\slidewidth, \n\slideheight, \n\slideframewidth,
\n\slideframesep, \n\semin, and \n\semcm.}

\begin{figure}
\hrule  height 1pt\relax
\begin{center}
\setlength\unitlength{.7cm}
\begin{picture}(13,11.5)(-1,-1.5)
\thicklines
\put(0,0){\framebox(11,8.5){\Huge My slide}}
\put(5.5,4.25){\fancyoval(8.5,6){}}

\thinlines
\put(0,8.65){\line(0,1){.3}}
\put(.75,8.65){\line(0,1){.3}}
\put(.5,8.8){\vector(1,0){.25}}
\put(.5,8.8){\vector(-1,0){.5}}
\put(.375,9.1){\makebox(0,0)[b]{\n\slideleftmargin}}

\put(10.25,8.65){\line(0,1){.3}}
\put(11,8.65){\line(0,1){.3}}
\put(10.75,8.8){\vector(1,0){.25}}
\put(10.75,8.8){\vector(-1,0){.5}}
\put(10.625,9.1){\makebox(0,0)[b]{\n\sliderightmargin}}

\put(0,-.15){\line(0,-1){.3}}
\put(11,-.15){\line(0,-1){.3}}
\put(5.5,-.3){\vector(1,0){5.5}}
\put(5.5,-.3){\vector(-1,0){5.5}}
\put(5.5,-.6){\makebox(0,0)[t]{\n\paperheight}}

\put(-.15,0){\line(-1,0){.3}}
\put(-.15,.75){\line(-1,0){.3}}
\put(-.3,.5){\vector(0,1){.25}}
\put(-.3,.5){\vector(0,-1){.5}}
\put(-.6,.375){\makebox(0,0)[r]{\n\slidebottommargin}}

\put(-.15,8.5){\line(-1,0){.3}}
\put(-.15,7.75){\line(-1,0){.3}}
\put(-.3,8){\vector(0,1){.5}}
\put(-.3,8){\vector(0,-1){.25}}
\put(-.6,8.125){\makebox(0,0)[r]{\n\slidetopmargin}}

\put(11.15,0){\line(1,0){.3}}
\put(11.15,8.5){\line(1,0){.3}}
\put(11.3,4.25){\vector(0,1){4.25}}
\put(11.3,4.25){\vector(0,-1){4.25}}
\put(11.6,4.25){\makebox(0,0)[l]{\n\paperwidth}}

\put(.75,7.75){\makebox(0,0)[bl]{Here is my header}}
\put(10.25,7.75){\makebox(0,0)[br]{9/30/99}}
\put(.75,.75){\makebox(0,0)[tl]{Here is my footer}}
\put(10.25,.75){\makebox(0,0)[tr]{7-9}}
\end{picture}

\caption{Slide margins.\label{SlideMargins}}
\end{center}
\hrule  height 1pt\relax
\end{figure}

Look at Figure \ref{SlideMargins}. Note that the headers and footers lie {\em
inside} the top and bottom margins, respectively. The slide is then centered
horizontally and vertically between the margins.


\section{Magnification and lengths}

"seminar.sty" changes \TeX's magnification so that the output is larger than
when typesetting an article. This means, for example, that if you paste some
input from a paper you are writing with \LaTeX\ into a slide, it will look
pretty much the same in the slide, except that it is magnified. For example,
"\vspace{.5in}" produces a space that gets bigger along with the fonts and
everything else. Of course, it won't look identical, since "seminar.sty" uses
its own spacing parameters and margins.

\TeX's standard magnifications are in magsteps. $n$ magsteps means a
magnification of $1.2^n$. "seminar.sty"'s default magnification is 4 magsteps,
but you can change this with the command
\begin{MD}
  \slidesmag{n}
\end{MD}
<n> should be an integer between $-5$ and 9.

As noted above, lengths grow with the magnification. For spacing, like the
parameter "\parindent" or using "\\[2pt]" to add a little extra space between
lines, this is great, because it is easier to think in unmagnified dimensions.
Setting "\parindent" to ".5cm" will look the same (relative to everything
else) whatever the magnification.

However, if you want to set the unit in a "picture" environment to 1cm, {\em
as it appears on the slide}, use
\begin{LVerbatim}
  \setslidelength{\unitlength}{1cm}
\end{LVerbatim}
"\setslidelength" is like \LaTeX's "\setlength", but it scales the size down
so that the resulting size after magnification is, in this example, 1cm (in
the process, it removes any stretch from rubber lengths).%
\footnote{If you know what {\tt true} dimensions are, you may be wondering why
they haven't been mentioned. These are not recommended because you will not be
able to print the slides two-up or use the {\tt article} format.}
There is also a
\begin{MD}
  \addtoslidelength{cmd}{len}
\end{MD}
command.

"seminar.sty" also provides the lengths
\begin{MD}
  \semin\\
  \semcm
\end{MD}
which are equal to an inch and a centimeter, scaled down to ``before
magnification'' size. For example,
\begin{LVerbatim}
  \rule{1pt}{4\semcm}
\end{LVerbatim}
makes a line 4 centimeters long on the transparency,.\footnote{Suppose you
want to use another unit, such as millimeters, instead of centimeters. Then
try this:
\begin{LVerbatim}
  \newcommand{\semmm}{\semcm}
  \renewcommand{\semcmlength}{1mm}
  \setslidelength{\semmm}{1mm}
\end{LVerbatim}}

"\textwidth", "\textheight", "\columnwidth" and "\linewidth" all have the
expected values in slides. (But "\slidewidth" and "\slideheight" do not.)
Thus, this would
\begin{LVerbatim}
  \rule{1pt}{.5\textwidth}
\end{LVerbatim}
make a line that is 1/2 the width of the slide.

For help getting an EPS file to be the right size, see page \pageref{EPS}.

Finally, there are some parameters that "seminar.sty" scales for you, and
hence you can (and should) set them at their magnified values (the actual size
on transparencies):
\begin{itemize}
  \item "\slidewidth" and "\slideheight"
  \item "\slideframewidth" and "\slideframesep" (see Section \ref{S-frames}).
  \item The slide margin parameters (see Section \ref{S-slidemargins}).
  \item The \LaTeX\ page parameters, {\em when they are set in the
preamble}.\footnote{The \LaTeX\ page parameters are listed on page 163 of
Lamport's {\em \LaTeX\ User's Guide and Reference Manual}. They are used for
notes (see Section \ref{S-notes}) and the \o{article} option (see Sections
\ref{S-twoup} and \ref{S-article}), but not for printing slides onto
transparencies. Lamport warns that resetting page parameters in the middle of
the document is tricky; here it is more so, because you do have to worry about
scaling them. But you can use "\setslidelength" for this.}
\end{itemize}

Note that the warnings about overfull slides (see Section \ref{S-pagebreak})
report unmagnified dimensions.


\section{Font sizes}

The "\slidesmag" command lets you change the magnification of your document.
You can effectively fine tune the magnification by using the "11pt" and "12pt"
style options. For example, instead of increasing the magnification from 4 to
5, you can switch to the "12pt" style option. There are differences---e.g.,
magnification affects the space you get from "\hspace{1in}"---but the fonts
will at least be roughly the same size either way. The "11pt" option gives a
half-step, somewhat like "\slidesmag{4.5}", if the latter were permitted.

Sometimes you want to use smaller font sizes for a single slide, in order to
fit in that little bit of extra material, or to use larger font sizes, so that
the few things you have to say on a slide don't look too lonely. You can't
changing the magnification in the middle of the document, but you can change
the size of the fonts with the command
\begin{MD}
  \ptsize{n}
\end{MD}
This switches to the font sizes that are in effect when you use the "<n>pt"
option. Actually, <n> can be 8, 9, 10, 11, 12, 14 or 17, whether you have
"art<n>.sty" or not.


\section{Spacing parameters\label{S-spacing}}

\begin{figure}
\hrule height 1pt
\begin{center}
\begin{tabular}{llc}
{\em Command:} & {\em Initializes:} & {\em Default} \\[2pt]
\N\slideparskip & "\parskip" & "1ex minus .2ex"\\
\N\slideparindent & "\parindent" & "0pt"\\
\N\slidefootnotesep & "\footnotesep" & "1.2ex"\\
\N\slideleftmargini & "\leftmargini" & "1.8em" \\
\N\slideleftmarginii & "\leftmarginii" & "1.4em" \\
\N\slideleftmarginiii & "\leftmarginiii" & "1em" \\
\N\slidelabelsep & "\labelsep" & ".5em" \\
\N\slideitemsep & "\itemsep" & ".8ex minus .2ex" \\
\N\slidepartopsep & "\partopsep" & "1ex minus .2ex" \\
\N\slidestretch & "\baselinestretch" & "1.2"\\
\N\slidearraystretch & "\arraystretch" & "1.2"
\end{tabular}

\caption{Slide spacing parameters.\label{SpacingDefaults}}
\end{center}
\hrule height 1pt
\end{figure}

The commands in Figure \ref{SpacingDefaults} are used to initialize some of
the spacing parameters at the beginning of each slide environment and when
"\ptsize" is used in a slide environment. These are spacing parameters that
should depend on the size of fonts and that in \LaTeX's "article" style would
be set in "art10.sty", etc. These commands should all be changed with
"\renewcommand", even though, except for the "stretch" parameters, their
values should be lengths.
Note that "ex" and "em" units are used because these are the units that depend
of font sizes.

Outside a slide environment, redefine the commands on the left if you want to
change these spacing parameters. Within a slide environment, reset the
parameters directly, or redefine the commands on the left and then use the
"\ptsize" command.

In other \LaTeX\ styles, the extra distance between lines that is inserted
when "\baselinestretch" exceeds 1 is eaten up by tall or deep lines. E.g., if
the line  contains a table or a large math operator, there is probably no
extra space at all. The advantage of this system is that the distance between
baselines does not fluctuate with every tilde. The disadvantage is that lines
can end up too close.

In "seminar.sty", on the other hand, fraction
\begin{MD}
  \slideskip
\end{MD}
of the extra space cannot be eaten, but up to fraction
\begin{MD}
  \slideshrink
\end{MD}
of this extra space that cannot be eaten can be removed if there is too much
material on the slide. This gives you added flexibility about how much
material to include on a slide. Both \n\slideskip\ and \n\slideshrink\ can be
set with "\renewcommand" to a number between 0 and 1. The default value of
\n\slideskip\ is ".75" and the default value of \n\slideshrink\ is ".25". Set
\n\slideskip\ to "0" to revert to \LaTeX's usual behavior, as described above.

For example, suppose you are using a 10pt font, \n\slidestretch\ is "1.2",
\n\slideskip\ is ".75", and \n\slideshrink\ is ".25". The "\baselineskip" for
a 10pt font is normally 12pt, leaving a little space between the lines. Then
"\baselineskip" is set to
\begin{center}
  "\slidestretch" $\times$ "12pt" $=$ 1.2$\times$12pt $=$ 14.4pt
\end{center}
The extra space between lines that is inserted is thus 2.4pt, and
\begin{center}
  "\slideskip" $\times$ 2.4pt $=$ .75$\times$2.4pt $=$ 1.8pt
\end{center}
of this cannot be eaten by tall or deep lines. However, the space between
lines can be reduced by up to
\begin{center}
  "\slideshrink" $\times$ 1.8pt $=$ .25$\times$1.8pt $=$ .45pt
\end{center}
if the slide would otherwise be too long.


\section{Slide frames\label{S-frames}}

Slides can be framed. The command
\begin{MD}
  \slideframe[commands]{style}
\end{MD}
specifies the frame style to use. Valid frame styles are "none" and "plain",
unless you use macros that define additional styles. For example, the
\o{fancybox} style option defines the frames "shadow", "double", "oval" and
"Oval" (corresponding to the "\shadowbox", "\doublebox", "\ovalbox" and
"\Ovalbox" commands defined in that style option). The \o{semcolor}
 option defines the styles "scplain", "scdouble", and "scshadow".

All the frame styles use the lengths
\begin{MD}
  \slideframewidth\\
  \slideframesep
\end{MD}
which are the (magnified) width of the line (default 4pt) and the distance
between the slide and the frame (default .4in), respectively.

\n\slideframe's optional argument is for commands that you want to use to
customize the slide frame style. For example:
\begin{LVerbatim}
  \slideframe[\setlength{shadowsize}{12pt}]{shadow}\\
  \slideframe[\psset{fillstyle=gradient}]{scplain}
\end{LVerbatim}

If you want to build your own custom slide frame, use the command
\begin{MD}
  \newslideframe{style}[commands]{frame command}
\end{MD}
<style> is the name of the frame, "[<commands>]", which is optional, will be
inserted before the frame command and before the <commands> given by
"\slideframe"'s optional argument. These commands can be used to set some
default parameter values. Then the final argument should frame "#1".

For example, if "\myframe{foo}" frames "foo", and if "\myframe" uses the
length "\baldness" as a parameter, then you might write
\begin{LVerbatim}
  \newslideframe{wildframe}[\setlength{\baldness}{.2cm}]%
    {\myframe{#1}}
\end{LVerbatim}
You can still override the default value of "\baldness", as in
\begin{LVerbatim}
  \slideframe[\setlength{\baldness}{.1cm}]{wildframe}
\end{LVerbatim}

There is a starred version of \n\slideframe\ that adds the frame to previously
specified frames.\footnote{If using color or PostScript images, note that each
frame gets added to the background.} This can be used for special tricks. For
example, if you are using the \o{fancybox} and \o{semcolor} options, then
\begin{LVerbatim}
  \newslideframe{draft}%
    {\boxput{\rput{30}{\Huge\gray DRAFT}}{#1}}
  \slideframe{draft}
  \slideframe*{scdouble}
\end{LVerbatim}
puts the word ``DRAFT'' in the background of each slide, gray and rotated 30
degrees, and then adds a double frame.\footnote{Use "\boxput*" to put "DRAFT"
in the foreground. See "fancybox.doc" for details.}


\part{Advanced Features}

\section{Counters}

The counter for slides is \C{slide}. The default definition of \N\theslide\ is
"\arabic{slide}". You can use "\label" and "\ref" to cross-reference slides,
and the page number in slide environments is set to
\n\theslide.\footnote{However, these page or slide cross-references are not
always accurate when you let \TeX\ break pages for you within a slide.} Thus,
you can write, for example,
\begin{LVerbatim}
  See equation (\ref{foo}) on Slide \pageref{foo}.
\end{LVerbatim}

You may want some counters, such as equation counters, to be reset with each
new slide environment or \n\newslide\ command. By default, only the "footnote"
counter is reset this way, but you can specify your own (comma-separated) list
of counters to be reset with the command
\begin{MD}
  \slidereset{list}
\end{MD}
If you want to reset additional counters, rather than replace the list
entirely, use
\begin{MD}
  \addtoslidereset{list}
\end{MD}

There is another way in which the footnote counter gets special treatment. The
command
\begin{MD}
  \theslidefootnote
\end{MD}
is used for the counter text instead of "\thefootnote". The default definition
is
\begin{LVerbatim}
  \alph{footnote}
\end{LVerbatim}

\section{Selectively including or excluding slides}

The commands
\begin{MD}
  \onlyslides{list}\\
  \notslides{list}
\end{MD}
can be used to include or exclude only those slides in the given list. The
argument should expand to a comma-separated list of numbers or ranges. The
numbers do not need to be in order, the list can contain numbers that do not
correspond to any slide, and there can be duplicate numbers. Negative numbers
should be enclosed in curly braces. Since the argument is first expanded, you
can use the "\ref" command in the argument. For example,
\begin{LVerbatim}
  \onlyslides{{-2},\ref{dp}-10,\ref{chart},0,17-999}
\end{LVerbatim}
is legal. If "\label{dp}" appears in slide 5 and "\label{chart}" appears in
slide 12, this is equivalent to:
\begin{LVerbatim}
  \onlyslides{{-2},5-10,12,0,17-999}
\end{LVerbatim}

\section{Printing your slides two-up\label{S-twoup}}

Printing your slides two-up is useful both for previewing slides and for
making hard copies to distribute or for proofreading.

One way to print your slides two-up is to include the \O{article} style
option. This is described further below.

Another way is to include the command
\begin{MD}
  \twoup[n]
\end{MD}
in the preamble. This inputs "2up.tex", which contains generic macros for
two-up printing, and sets the parameters to values that are likely to work.
Including the optional argument "[<n>]" increases the two-up magnification by
$n$ magsteps. $n$ can be a positive or negative integer. If you are not able
to get the right layout, then include the \O{2up} style option, read the
documentation of "2up.tex", and set the parameters yourself, rather than using
the \n\twoup\ command.

The \o{article} style option is also called the article {\em format} (as
opposed to the slides format). In the article format without the \o{portrait}
option, the slides are centered horizontally and vertically, two to a page (if
they fit---landscape slides do fit by default). With the portrait option, the
slides are printed side-by-side, two to a page. In the \o{article} format, you
can mix landscape and portrait slides with different orientations, but this
does not work well with the \o{portrait} option.

You can change the article format's magnification with
\begin{MD}
  \articlemag{n}
\end{MD}
This command works like the "\slidesmag" command (page \pageref{+slidesmag}).
The default is
\begin{LVerbatim}
  \articlemag{0}
\end{LVerbatim}
Increase the article magnification if the slides look too lonely; decrease it
if they are not coming out two-up.

The size of the slides depends on the {\em difference} between the
magnifications in the slides and article format. E.g., since the default
slides magnification is 4 magsteps, the slides are scaled down in the article
format by 4 magsteps. When you change the slides magnification with
\n\slidesmag, you also have to change the article magnification with
\n\articlemag\ by the same amount if you want the size of the slides in the
article format to remain the same. However, when the article magnification in
magsteps is negative , you will be using non-standard font magnifications
(which will produce terrible results if you cannot generate the needed
fonts---see Appendix \ref{bitmaps} for advice).

Whether or not landscape slides (or rotated portrait slides) come out two-up
in the article format without the \o{portrait} option depends on (i) the
difference between the slides and article magnification, (ii) the values of
\n\slidewidth\ and \n\slideheight\, (iii) the size of the slide frame, and
(iv) the article format's page parameters. With the default values of
(ii)--(iv), the slides will still come out two-up if you change the difference
between the slides and article magnification to 3 (e.g., increase the value of
\n\articlemag\ by 1). Try this if you want the slides to be larger.

If you use the \o{article} option {\em and} the \n\twoup\ command, then your
slides should be printed four-up!

You will notice labels on the side or bottom of each slide in the \o{article}
format. The command
\begin{MD}
  \slidestyle{style}
\end{MD}
determines where these labels go. There are three predefined slide styles:
\begin{description}
\item[\tt empty] No captions or labels are used.
\item[\tt left]  The labels go on the left of each slide (the default when the
\o{portrait} option is not used).
\item[\tt bottom] The labels go on the bottom (the default with the
\o{portrait} option).
\end{description}
The label you get is the value of \N\slidelabel. The default definition of
\n\slidelabel\ is
\begin{LVerbatim}
  \bf Slide \theslide
\end{LVerbatim}

\section{Notes\label{S-notes}}

In addition to slides, you can include a few comments following each slide to
remind you of what to say, or even the entire text of your presentation for
your own benefit or to be distributed to others. These comments or text,
referred to in this documentation as ``notes'', do not go inside any special
environment. However, a slide cannot go in the middle of a paragraph of notes.

To include notes this way, you have to use one of the following style options,
which determine what is printed:
\begin{quote}
  \begin{tabular}{ll}
    \O{slidesonly} & Only the slides are printed.\\
    \O{notes} & Both notes and slides are printed.\\
    \O{notesonly} & Only the notes are printed.\\
  \end{tabular}
\end{quote}
These style options are referred to as {\em selections} in this documentation.

The pages of notes following slide 5 are numbered 5.1, 5.2, etc.

You can use the "portrait" option to print out your notes, even if you don't
use this option when printing the slides.

With the "notesonly" selection, the slides are not printed, but they are
processed. This means that one can still refer in the notes to slides or to
equations in slides.

The fact that various spacing parameters are initialized at the beginning of
each slide (see Section \ref{S-spacing}) means that when you reset these
parameters outside a slide environment, only the notes are affected. This lets
you use different values for notes and slides. The command \N\slidefonts\ can
be used for any special font commands that should apply only to slides. Also,
the command \N\everyslide\ is executed at the beginning of every slide, and
you can use this for any other customizations you want to include.

You can also use the "11pt" or "12pt" style option for your notes, without
changing the size of the fonts in the slides, by using the "\ptsize" command
to set the font sizes for slides. For example, suppose that the "12pt"
document style option is used, but the preamble contains  the command
\begin{LVerbatim}
  \ptsize{11}
\end{LVerbatim}
and the second slide begins with the command
\begin{LVerbatim}
  \ptsize{9}
\end{LVerbatim}
Then the notes use the "12pt" font size declarations, all the slides but the
second one use the "11pt" font size declarations, and the second slide uses
the "9pt" font size declarations.

Notes are typeset using \LaTeX's standard output routines, and \LaTeX's
standard page parameters.

 Finally, it is even possible to use a different page style for notes and
slides. The page style for notes is set by the "\pagestyle" and
"\thispagestyle" commands. The command
\begin{MD}
  \slidepagestyle{style}
\end{MD}
sets a different page style for the slides. If the argument is empty, then the
page style for slides reverts to the one for notes.\footnote{There is no
"\thisslidepagestyle" command.}

With the command
\begin{MD}
  \onlynotestoo
\end{MD}
the "\onlyslides" and "\notslides" commands apply to the notes as well.


\section{How notes are omitted\label{omit}}

In the previous section, it was stated that notes are not enclosed in a
special environment. "seminar.sty" omits notes using special
macros\footnote{Defined in {\tt xcomment.sty}} that comment out everything
{\em outside} the slide environments (and a few other environments mentioned
below). Global declarations that should be processed even when slides are
omitted go in \e{allversions*} environments, described below.

"seminar.sty" also let you use a more conventional approach to notes. If you
put the command
\begin{MD}
  \noxcomment
\end{MD}
in the preamble, then the notes go inside \E{note} environments, and you can
put all the global declarations outside any environment. "xcomment.sty" is
used to make \e{note} a comment environment when note should be omitted, and
otherwise the \e{note} environment does nothing. This mode of operation is
more robust than omitting everything outside the slide environments, but
remembering to insert the
\begin{LVerbatim}
  \begin{note} ... \end{note}
\end{LVerbatim}
is more tedious.

Here are a few technicalities that have to do with "xcomment.sty" and that
apply whether or not you use \n\noxcomment.
\begin{enumerate}
\item The text that follows the beginning of a slide environment (when not
using \n\noxcomment) or the end of a \e{note} environment {\em must have
balanced curly braces}.
\item "\input" and "\include" commands are followed, even when found in
omitted notes, but you must use the \LaTeX\ syntax "\input{<file>}" (as
opposed to \verb*+\input <file> +), and the inputted file must end with
"\endinput" (this is a good practice anyway).
\item In omitted text, "%" is still a comment character (hence it is possible
to comment out a slide or note environment).
\item A temporary file, "\jobname.tmp", is created (this is of no
consequence---just in case you wanted to know where it comes from).
\end{enumerate}

The rest of this section deals with special considerations when omitting
everything outside the slide environments.

If you might want to make a global change to one of the slide parameters after
the document preamble, you cannot include it in a slide environment because
the change will be local, and you cannot include it in the notes because the
change won't be processed when the notes are omitted.

To get around this problem, an environment \E{allversions*} is provided. It is
processed even when notes are omitted, and any parameter changes or command
definitions made within the environment are global. Don't generate any output
within this environment.

On a rare occasion you might want to include some text that should be typeset
even when the other notes are omitted (e.g.,  a list of the slides, or a cover
page). The \E{allversions} environment is provided for this purpose.

If you do any serious hacking, you might want to add to the list of
environments that should be included with the "slidesonly" selection. Do this
with the command
\begin{MD}
  \addtoslidelist{list}
\end{MD}
where <list> is a comma separated list of environments, without spaces.

\section{The article format\label{S-article}}

The \o{article} style option was described in Section \ref{S-twoup} as a way
to print your slides two-up. This option is also a good way to print your
notes.

The \o{article} option gets its name because your document is typeset somewhat
like it would be with \LaTeX's "article" style. (It uses \LaTeX's standard
page parameters, but some of the default values are different.) It is also
called the article {\em format}; the slides format is what you get without the
\o{article} option.

You probably don't want to use the \o{portrait} style option with the
\o{article} format except when you are only printing portrait slides. But you
are welcome to experiment.

With the \o{notesonly} selection, you can make the notes as big as you want
using the \n\articlemag\ command (e.g., for easy reading while giving a
presentation). Just as in the \o{slides} format, the page parameters are
scaled at the beginning of the document so that they can be set with their
true dimensions in the preamble.

The commands
\begin{MD}
  \setartlength{cmd}{len}\\
  \addtoartlength{cmd}{len}
\end{MD}
are analogous to "\setslidelength" and "\addtoslidelength", but they scale the
lengths so that they end up at the specified size after magnification in the
"article" format. You never would use such a command in slide environments or
to set slide parameters, but you might use these in the notes if you were
planning on typesetting your notes with the "article" format rather than the
"slides" format.

You can change the placement of slides in the \o{article} format using the
\begin{MD}
  \slideplacement{name}
\end{MD}
command. Here are the valid placement names:
\begin{description}
  \item[float] The slides are floated. This is the default with the \o{notes}
selection and the \o{portrait} option.
  \item[float*] Like "float", but if the notes are printed in a two-column
format the slides extend across both columns (e.g., like "table" versus
"table*"). This is the default with the \o{notes} selection without the
\o{portrait} option.
  \item[onepercol] Each slide is centered horizontally and vertically within a
single column. This is the default when only slides are printed and the
\o{portrait} option is used.
  \item[twopercol] Each slide is centered horizontally, two or one per column,
depending on how many fit. This is the default when only slides are printed
and the \o{portrait} option is not used.
  \item[here] Each slide is centered horizontally, separated from adjoining
text or other slides by the rubber length \N\slidesep.  The default value of
\n\slidesep\ is "\intextskip". This is useful with the \o{notes} selection
when you want the slides to precede accompanying comments and when you do not
mind large spaces at the bottom of pages.
  \item[here*] This is like "here*", but with "here*" the length \n\slidesep\
is not lost when it falls at the beginning of a page or after a slide.
\end{description}

If you want each slide environment to begin a new page (with any of the
selections), put the command
\begin{MD}
  \slideclearpagetrue
\end{MD}
in the preamble.

The \o{notesonly} selection has a starred version, \O{notesonly*}, which
produces a slide marker for each slide environment, like this one:
\begin{center}
\leavevmode
\vbox{%
\hrule height 1pt
\kern 8pt
\hbox to \linewidth{\hss \LARGE\bf Slide 4 \hss}%
\kern 8pt
\hrule height 1pt}
\end{center}


Commands that are irrelevant in the "article" format are simply ignored. The
idea is that it should be possible to switch back and forth between the two
formats without making any other changes in the document. However, this is not
entirely possible. Changing "\parindent" or "\textwidth" in the preamble
affects both formats. You can get around this using the \n\ifarticle\
conditional. For example,
\begin{LVerbatim}
  \ifarticle
    blah blah
  \else
    blee blee
  \fi
\end{LVerbatim}
The "\else" part is optional.


\section{Page styles}

The page styles "empty", "plain", "headings" and "myheadings" work like in
\LaTeX's article style. There is also a page style "align" which puts "+"
signs in the corners, like in \SliTeX.\footnote{See question \ref{Pagestyle}
in Section \ref{tips} for suggestions on defining new page styles.} You can
change the size of the fonts used in headers and footers in the slides format
by setting the commands
\begin{MD}
  \slideheadfont\\
  \slidefootfont
\end{MD}
to the desired size, using "\renewcommand" (but any explicit font declarations
in a page style override these commands). The default definition of these
commands is "\scriptsize".


\section{Other style options\label{options}}

The style options described so far, "portrait", "article", "slidesonly",
"notes", and "notesonly", were designed specifically for "seminar.sty". There
are other such style options described elsewhere in this documentation: 
"semrot" (page \pageref{o+semrot}), \o{semlayer} (page \pageref{o+semlayer}).
\o{semcolor} (page \pageref{o+semcolor}), "semhelv" (page
\pageref{o+semhelv}), "semlcmss" (page \pageref{o+semlcmss}), and "a4" (page
\pageref{o+a4}).

Here are two more style options that are part of the "seminar.sty" package,
but that can be used with other document styles:
\begin{description}
\item[fancybox] "fancybox.sty" is specific to slides, but it contains commands
and documentation that are useful for making slides.

\item[slidesec] This sets up some sectioning/heading commands for slides, and
lets you print a list of slides or a table of contents. See the file
"slidesec.sty" for details.
\end{description}

It is impossible to predict, much less describe, the consequences of using all
the other available \LaTeX{} style option with each selection and format.
Since this document style is an extension of \LaTeX's "article" style, style
options and commands that do not work with \SliTeX{} may well work here. Just
about any style option that works with \LaTeX's "article" style will work with
the "article" format. When unsure of the effect of an option, just try it and
see for yourself what happens.

\section{Overlays\label{S-overlays}}

For overlays, you must use both the \O{semlayer} and \o{semcolor} style
options (see page \pageref{o+semcolor}).

Overlays have two purposes. First, you can use overlays to gradually add
layers of information on a slide during a presentation. Second, you can use
overlays to make color layers; each layer is printed in black-on-white on
paper, and then you use some kind of color copying service to put each layer
on a transparency in a different color. This is not nearly as nice as having a
color printer or using a color printing service, but it's the next best thing.

"seminar.sty" lets you make both types of overlays (and the main slide) from a
single slide environment. The command\footnote{You can also write
\begin{Ex}
  "\begin{overlay}{<n>}" $\ldots$ "\end{overlay}"
\end{Ex}}
\begin{MD}
  \overlay{n}
\end{MD}
puts whatever is in its scope (\TeX\ group) on overlay $n$, where
$n=0,\ldots,9$. Actually, overlay 0 refers to the main slide, but you might
use "\overlay{0}" because overlay commands can be nested. As implemented by
the "semcolor" style option, these overlay commands can be used just about
anywhere, including in math mode, tables, and around an included graphics file
(if it is a conforming EPS file, at least). Also, it is all right to use
non-consecutive overlay numbers.

For color layers, you have to start by defining some colors using the command
\begin{MD}
  \colorlayers{colors}
\end{MD}
<colors> should be a comma separated list of color names, without spaces, as
in
\begin{LVerbatim}
  \colorlayers{red,green,blue}
\end{LVerbatim}
Then you can use the command "\red" just like the command \n\overlay"{<n>}";
everything in its scope goes on a red color layer.

The "\colorlayers" command obeys the usual rules on scope. You can use this
command any time, including in a slide environment. The command is cumulative,
meaning that previously defined color layers continue to exist. The command
was purposely defined so that it does not complain when a color name is
already defined; this makes it easier to switch from some other color system
to a layer system. However, you should be careful not to inadvertently
redefine some command that you need. Fortunately, there are no \TeX{}
primitives whose names are the names of colors.

When you print out the slides, the main slide is printed, followed by each of
the color layers (if any) for the main slide. Then each of the overlays is
printed, together with each of its color layers. Only overlays or color layers
that are actually used (i.e., that are not empty) are printed.

You can turn overlays and color layers on and off with the commands
\begin{MD}
  \overlaystrue\\
  \overlaysfalse\\
  \layerstrue\\
  \layersfalse
\end{MD}
These commands can be used at any time, and they obey the usual scoping rules.
The default is for overlays to be active in the "slides" format and suppressed
in the "article" format.

The counter \C{overlay} keeps track of the overlays. The default definition of
\N\theoverlay\ is:
\begin{LVerbatim}
  \theslide-\alph{overlay}
\end{LVerbatim}
Overlays can be cross-referenced.

The command \N\currlayer\ is set to the name of the current color layer.
\N\thelayer\ makes a label for layers; its default definition is:
\begin{LVerbatim}
  \theoverlay-\currlayer
\end{LVerbatim}
Color layers cannot be cross-referenced.

For example, if slide 7 has overlays 1 and 2 and colors "red" and "green",
then the main slide is numbered 7, followed by layers 7-red and 7-green,
followed by overlay 7-a, followed by layers 7-a-red and 7-a-green, followed by
overlay 7-b, followed by layers 7-b-red and 7-b-green.

The caption used in the slide styles is \n\overlaylabel\ for overlays and
\N\layerlabel\ for layers. The defaults are, respectively,
\begin{LVerbatim}
  \bf Overlay \theoverlay
  \bf Layer \thelayer
\end{LVerbatim}

By default the overlays and layers use the same page styles in the "slides"
format as their ``owner.'' You can specify special page styles with the
commands:
\begin{MD}
  \overlaypagestyle{style}\\
  \layerpagestyle{style}
\end{MD}

Finally, by default the overlays and layers use the same frame style as their
``owner.'' You can specify special frame styles with the commands:
\begin{MD}
  \overlayframe[commands]{style}\\
  \layerframe[commands]{style}
\end{MD}


\part{Help\label{Help}}

Road map:
\begin{itemize}
  \item If you just want to figure out how to do something, check Section
\ref{tips}, ``Tips and tricks.''
  \item If you are trying to decipher an error message, check Section
\ref{errors}, ``Errors.''
  \item If you are trying to solve some system-dependent problem that has
arisen, check Section \ref{trouble}, ``Troubleshooting.''
  \item To convert \SliTeX\ files, see Section \ref{slitex}, ``Converting
\SliTeX\ files''.
\end{itemize}

\section{Tips and tricks\label{tips}}

\faq{How can I fit more material in the slides?}

Or alternatively, make the little material you have fill up the slide.

Let us count the ways:
\begin{enumerate}
\item Change the magnification with \n\slidesmag"{<n>}" (affects all the
slides).
\item Change the height and/or width of one or all the slides.
\item Use "\ptsize{<n>}".
\item Change \n\slidestretch.
\item Use the \n\raggedslides\ command, which changes the propensity to
hyphenate.
\end{enumerate}

\faq{How do I include an Encapsulated PostScript figure?\label{EPS}}

The only thing tricky about including EPS files is getting the size right. For
example, suppose the you want to include a postscript file in a slide using
the "epsf.sty" macros, and you want it to be 6 inches wide. Then this will do
the trick:
\begin{LVerbatim}
  \setslidelength{\epsfxsize}{6in}
  \epsffile{mypic.eps}
\end{LVerbatim}

If you include the command
\begin{MD}
  \espfslidesize
\end{MD}
(e.g., in the preamble), then "epsf.sty" will take care of scaling the size
for you, and so you can just set "\epsfxsize" and "\epsfysize" to their
magnified sizes (using "\setlength" rather than "\setslidelength"). Or you can
not set these parameters at all, and then the eps file  appears at its natural
size on the slide.

Either way, this will have the expected effect:
\begin{LVerbatim}
  \setlength{\epsfysize}{.8\textheight}
  \epsffile{mypic.eps}
\end{LVerbatim}

"psfig.tex" does not handle magnification properly. By using magnified
dimensions, without "\setslidelength", as in
\begin{LVerbatim}
  \psfig{file=mypic.eps,width=6in}
\end{LVerbatim}
the picture should appear correctly within a slide, but then it will not scale
properly if you try to use the \o{article} option or the \n\twoup\ command.
The solution, if using Rokicki's "dvips", is to use "epsf.sty" (written by the
man himself).

\faq{How can I print just selected overlays, layers or pages of notes in the
{\tt slides} format?}

The "\onlyslides" and "\notslides" commands affect only slides. If a slide is
omitted, so are all its overlays and color layers. To be more selective, you
have to use your dvi driver to select the pages to be printed.

In the "slides" format, the page numbers recognized by dvi drivers correspond
to the numbers of the slides. This means that a slide and all the overlays,
layers and notes that correspond to that slide have the same page number. Some
drivers allow one to select occurrence $n$ of a page number. E.g., with
"dvips",
\begin{LVerbatim}
  dvips -p2.1 -l2.4 myslides
\end{LVerbatim}
will print the first through fourth page of overlays or notes that follow
slide 2.

If your dvi driver does not support such selection, and you would prefer that
the driver recognize physical page numbers, then put the command
\N\truepagenumbers\ in the preamble.


\faq{How do I define custom page styles?}
\label{Pagestyle}

There is nothing special about defining new page styles in "seminar.sty".
However, to make it easier to do this in the preamble, the commands
\begin{MD}
  \newpagestyle{style}{header}{footer}\\
  \renewpagestyle{style}{header}{footer}
\end{MD}
are provided.\footnote{%
  You can also use the macros in "fancyheadings.sty", which is available from
various archives. However, you have to set the pagestyle {\em after}
"\begin{document}".
Otherwise, the dimensions get screwed up. If using the \o{slidesonly}
selection, then you also need to enclose the "\pagestyle" command in
an \e{allversions*} environment. E.g.,
\begin{LVerbatim}
  \begin{allversions*}
    \pagestyle{fancy}
  \end{allversions*}
\end{LVerbatim}}

<style> is the name of the page style, and <header> is the header, and
<footer> is the footer. These can only be used for simple page styles that are
the same for odd and even pages, and that do not do anything special with
section marks.

Headers and footers are set in an "\hbox" the width of the page
("\textwidth"). You can use stretchable space such as "\hspace*{\fill}" or
"\hfil" to center some information or put it flush against the margins. See
the definitions of page styles in "latex.tex" and "article.sty" for examples.

Here is an example: Professor Starr wants lots of information in the headers
and footers for the slides, and so she defines the page style
"mypagestyle":\footnote{{\tt\string\thedate} is set with the {\tt\string\date}
command.}
\begin{LVerbatim}
  \newpagestyle{mypagestyle}%
    {\sl Big U \hfil \thedate \hfil \thepage}%
    {\hfil File \jobname.tex; printed \today\hfil}
\end{LVerbatim}
Prof.\ Starr wants to use the standard "headings" page style for notes and the
"mypagestyle" page style for slides, and she wants overlays to just have the
overlay number. Therefore, she defines another page style for the overlays:
\begin{LVerbatim}
  \newpagestyle{myoverlays}{\hfil \thepage}{}
\end{LVerbatim}
and she puts
\begin{LVerbatim}
  \pagestyle{headings}
  \slidepagestyle{mypagestyle}
  \overlaypagestyle{myoverlays}
\end{LVerbatim}
in the preamble.

\faq{How do I change a parameter only for the {\tt slides} format?}
\label{conditionals}

The conditionals
\begin{MD}
  \ifarticle\ $\ldots$ "\else" $\ldots$ "\fi"\\
  \ifslidesonly\ $\ldots$ "\else" $\ldots$ "\fi"\\
  \ifnotes\ $\ldots$ "\else" $\ldots$ "\fi"\\
  \ifnotesonly\ $\ldots$ "\else" $\ldots$ "\fi"\\
  \ifportrait\ $\ldots$ "\else" $\ldots$ "\fi"
\end{MD}
allow one to select for what versions material is to be processed. "\ifnotes"
is true if and only if the "notes" selection is in effect, and so on. The
"\else" clause is optional. See {\em The \TeX book}, Chapter 20, for more
information about using conditionals.

For example, the first line below sets the slide rotation to "right" in the
"article" format only. The second line changes the page style for the "notes"
and "notesonly" selections and "article" format:
\begin{LVerbatim}
  \ifarticle\sliderotation{right}\fi
  \ifarticle\ifslidesonly\else\pagestyle{myheadings}\fi\fi
\end{LVerbatim}

\faq{Why does extra space get inserted at the top of a slide when I begin the
slide with a color or overlay command (when using the \o{semcolor} option)?}

Color and overlay commands with the "semcolor" option use "\special"'s. \TeX{}
adds the space "\parskip" between the "\special"'s and the first material in
the slide.

Here is the workaround: If that material is an ordinary paragraph, put the
command "\leavevmode" just after the color or overlay command. In other cases,
if you are sure this is the problem, put "\vskip-\parskip" just before the
color or overlay command.

\section{Errors\label{errors}}

There are several errors that identify themselves as being from "seminar.sty",
but the error messages are so self-explanatory that that there is no need to
describe them here. Instead, this section explains a few especially cryptic
\TeX{} error messages that can arise when using "seminar.sty". These errors
may also arise for reasons that are not particular to the "seminar.sty"
macros. See your favorite \TeX{} and \LaTeX{} manuals for more help in
debugging your documents.

\error"! File ended while scanning use of \next."

You are missing an "\end{document}" on the main file or an "\endinput" on a
file that is input.


\error"! File ended within \read."

You may have an unmatched curly brace following "\begin{slide}" on the same
line. See Section \ref{omit}.


\error"! Paragraph ended before \begin@slide was complete."

You have not specified the optional argument for the "slide" or "slide*"
environment correctly (see Section \ref{slidedim}), or the first character in
a slide environment is a "[" (put a pair "{}" of braces before the "[").

\section{Troubleshooting\label{trouble}}
\setcounter{faq}{0}

\faq{A few slides do not come out two-up with the \o{article} option.}

First, read Section \ref{S-twoup} carefully. If the slides still do not come
out twoup, it might be that there is extraneous output between the slides. Try
putting the commands
\begin{LVerbatim}
  \slideplacement{here*}
  \setlength{\slidesep}{8pt plus 1fill}
\end{LVerbatim}
in the preamble ("8pt" should be one-half the minimum distance between
slides). If this solves the problem, look for the extraneous output, or just
leave those commands in the preamble.


\faq{What kind of incompatibilities are there between {\tt seminar.sty} and
other macros?}

Of course, whenever you load macros that are not part of the standard \LaTeX{}
distribution and that were not designed to work with "seminar.sty", problems
may arise because of name conflicts. What are listed here are changes made to
\LaTeX{} commands that may conflict with other macros that also redefine these
command. The problems listed here are very unlikely to occur, however, unless
you have a habit of seriously hacking the standard \LaTeX{} macro files.

\begin{enumerate}
\item This style modifies the definitions of the \LaTeX{} primitive
"\document". It generally will not be upset by, nor will it void,
modifications to "\document" made before "seminar.sty" is input. However,
subsequent modifications of "\document" may cause problems.

\item "xcomment.sty", which is input by "seminar.sty" with the "slidesonly"
selection, modifies the definition of "\end" within included environments.
This is generally compatible with modifications to "\end" made outside
environments, but may conflict with modifications to "\end" made inside
environments.
\end{enumerate}

\faq{Why are the landscape slides displayed sideways and the portrait slides
displayed upside down on my previewer?}

This document style makes frequent use of landscape mode. Some DVI-to-PS
converters, such as older versions of Rokicki's "dvips", use their own
PostScript macros to print a landscape document, rather than simply
instructing PostScript to use landscape mode.
If the PostScript output of such a converter is viewed using a PostScript
previewer that does not allow you to choose the orientation of the display,
the output will be positioned correctly on the page, but the page will always
be displayed in portrait mode. When viewing slides without the "portrait"
option, the landscape slides will be sideways, and the portrait slides will be
upside-down!

There is nothing this style can do to coerce the page to be displayed in
landscape mode. There are various ways to minimize neck strain, however:
\begin{enumerate}
\item The direction in which the portrait slides are rotated can be reversed,
so that they end up right-side-up. Just put
\begin{LVerbatim}
  \sliderotation{right}
\end{LVerbatim}
before the beginning of the document (assuming that you are using rotation
macros).
\item You can also get the portrait slides to be displayed right-side-up by
using the "portrait" style option (Section \ref{options}).
\item In the "article" format, the document is typeset in portrait mode
(unless the "portrait" option is used), and so the landscape slides are
right-side-up. You can use this format when composing and proof-reading the
slides.
\end{enumerate}

\faq{Why is my dvi driver soooo slow?}

Probably it cannot find the right font bitmaps, and so it is either
automatically generating new ones, or it is scaling the ones it can find.
Either way, as long as you can somehow generate the missing font bitmaps, this
problem is transitory. See Appendix \ref{bitmaps} for details.

\faq{Why does \TeX{} complain about missing circle fonts?}

Older versions of the NFSS use the names "circle10" and "circlew10" for the
\LaTeX circle fonts, instead of the otherwise standard names "lcircle10" and
"lcirclew10". You can copy your ".pk" and ".tfm" files to the new names, or
get a new version of the NFSS.

\section{Converting \SliTeX\ files}\label{slitex}

"seminar.sty" can do everything \SliTeX\ can do, and much
more\footnote{However, color layers and overlays require PSTricks and a
PostScript printer.} Here is a brief and incomplete description of how to do
with "seminar.sty" what you can do with \SliTeX, and how to convert \SliTeX\
files.

\begin{itemize}
  \item Use "seminar" as your document style, instead of "slides".

  \item Run \LaTeX\ or \AmS-\LaTeX, instead of \SliTeX.

  \item With "seminar.sty", the preamble and slides can all go in the same
file.

  \item The default in "seminar.sty" is to get landscape slides. If you want
to convert a \SliTeX\ file containing portrait slides, add the "portrait" 
style option, and replace your "slide" environments by "slide*" environments.

  \item With "seminar.sty", notes do not need to go in a separate environment.
To convert a \SliTeX\ file containing "note" environments, define a note
environment that does nothing:
\begin{LVerbatim}
  \newenvironment{note}{}{}
\end{LVerbatim}

  \item For color layers:
  \begin{itemize}
    \item Use the "\colorlayers" command instead of the "\colors" command.
    \item Delete the argument to the "slide" and "slide*" environments that
lists the color layers.
    \item Include the \o{semlayer} style option.
  \end{itemize}

  \item "seminar.sty" does not use separate environments for overlays:
  \begin{itemize}
    \item Remove the "overlay" environments.
    \item Replace "\invisible" commands by "\overlay{1}", "\overlay{2}", etc.
    \item Include the \o{semlayer} style option.
  \end{itemize}

  \item To actually produce the overlays and color layers, you have to have
PSTricks and a PostScript printer, and you must include the \o{semcolor} style
option.

  \item Use \n\onlyslides"{<list>}" and \n\onlynotestoo\ instead of
"\onlynotes".
\end{itemize}


\begingroup
\def\addcontentsline#1#2#3{}%
\def\thepart{Appendices:}
\part{Configuration}
\endgroup
\begingroup
\def\thepart{}
\addcontentsline{toc}{part}{Appendices:\hspace{1em}Configuration}
\endgroup
\appendix

Before reading this appendix, you should follow the installation instructions
in the file "sem-read.me" that is distributed with "seminar.sty".

\section{The short story about fonts}\label{short-fonts}

"seminar.sty" is a \LaTeX{} style, and you can use whatever fonts that are
compatible with \LaTeX\ (or \AmS-\LaTeX). However, you are likely to want to
use special fonts with "seminar.sty"; see Appendix \ref{fonts} for help.
Furthermore, you are likely to want to use fonts in sizes that are larger than
the standard sizes; see Appendix \ref{bitmaps} for help.

But if you don't want to read these appendices, you can just use whatever
\LaTeX\ fonts you know how to use (e.g., the standard Computer Modern fonts).
To be sure you need only standard font magnifications:\footnote{This is not a
consideration if you are using only PostScript or other scalable fonts.}
\begin{itemize}\label{font-res}
  \item Don't use the "11pt" or "12pt" style options  (or the \n\ptsize{}
command).
  \item Don't change the document magnification (or only use 0--4 magsteps).
  \item Don't use any font size larger than "\large".
  \item Don't use the \n\twoup\ command.
\end{itemize}

\section{Choosing fonts\label{fonts}}

So that you can use different fonts for the notes and the slides (if you
want), the command "\slidefonts" is invoked at the beginning of every slide
environment. Define it to set up any special fonts for the slides.

Here are the font configurations that come ready-to-use with "seminar.sty".
You might also use these as a starting point for your own configurations.
\begin{enumerate}
\item You can just use the regular old Computer Modern fonts that you probably
grew up on. For this, you don't have to do anything at all. Ain't that easy?
But see the next section about font bitmaps.

\item The \O{semhelv} style option sets up the PostScript Helvetica text fonts
for the slides. The Computer Modern fonts are still used for math and for the
notes with the \o{article} format. You need a PostScript printer, a dvi-to-ps
driver that supports PostScript fonts (e.g., Rokicki's "dvips"), and the New
Font Selection Scheme. You should also check the font file names used in
"semhelv.sty", and change them if necessary to match the names on your system.

This combination of fonts is highly recommended because sans serif fonts look
good for slides, Helvetica is a resident font in just about all PostScript
printers, and the fonts are scalable and thus there is no problem of needing
new bitmaps.

\item The \O{semlcmss} style option sets up the \SliTeX\ sans serif fonts for
the slides, and uses the Computer Modern fonts for notes with the \o{article}
format. The \SliTeX\ fonts are ugly, but they might be the only usable sans
serif font you have on your system and you are really dying for that kind of
font. Also, they usually come ready for printing magnified documents, and so
you won't need too many new font bitmaps. You must have the New Font Selection
Scheme. "semlcmss" uses
\begin{LVerbatim}
  \slidesmag{4}
  \ptsize{10}
\end{LVerbatim}
Deviating from this will increase the need for new font bitmaps.
\end{enumerate}

The New Font Selection Scheme (NFSS) mentioned above is a macro package for
\LaTeX{} that greatly simplifies using non-standard fonts. The NFSS was
written by Frank Mittelbach and Rainer Sch\"opf, and is available from various
archives, including:
\begin{center}
"ftp.uni-stuttgart.de"
\end{center}
It is far easier to take the 10 minutes or so that are required (in theory, at
least) to install the NFSS, than it is to try to muck around with \LaTeX's
font primitives.


\section{Font bitmaps\label{bitmaps}}

If you are only using PostScript or other scalable fonts, or if you obey the
restrictions listed on page \pageref{font-res}, then you can ignores this
appendix.

To use "seminar.sty", you may need font bitmap sizes that are not currently
found on your system. This section describes how to avoid this and what to do
about it. First, a few paragraphs about magnification and font bitmaps.

\TeX's Metafont fonts are designed for a type size, such as 5pt. Most font
families are available at least in the sizes 5pt, 6pt, 7pt, 8pt, 9pt and 10pt.
Some are also available in the sizes 12pt and 17pt (and others), but often any
size above 10pt is obtained by scaling the 10pt fonts. Table \ref{font-mag}
lists the possible magnifications for fonts, depending on which option you are
using ("10pt", "11pt" or "12pt"), and depending on the \LaTeX{} type size
declaration that is in effect (e.g., "\small"). The magnifications are given
in magsteps, which is \TeX's standard unit for font magnifications. $n$
magsteps means a magnification of $1.2^n$.

\begin{table}
\hrule  height 1pt\relax
\begin{center}
\begin{tabular}{l|c|c|c|}
\multicolumn{1}{l}{size} &
\multicolumn{1}{c}{default (10pt)} &
  \multicolumn{1}{c}{11pt option}  &
  \multicolumn{1}{c}{12pt option}\\
\cline{2-4} "\tiny"       & 0  & 0 & 0\\
\cline{2-4} "\scriptsize" & 0  & 0 & 0\\
\cline{2-4} "\footnotesize" & 0 & 0 & 0 \\
\cline{2-4} "\small"        & 0 & 0 & 1/2 \\
\cline{2-4} "\normalsize" & 0 & 1/2 & 1 \\
\cline{2-4} "\large"      & 1 & 1 & 2 \\
\cline{2-4} "\Large"      & 2 & 2 & 3 \\
\cline{2-4} "\LARGE"      & 3 & 3 & 4\\
\cline{2-4} "\huge"       & 4 & 4 & 5\\
\cline{2-4} "\Huge"       & 5 & 5 & 5\\
\cline{2-4}
\end{tabular}

\caption{Font magnification in magsteps ($n$ means a magnification of $1.2^n$)
for a font that is available in 5pt, 6pt, 7pt, 8pt, 9pt and 10pt
sizes.\label{font-mag}}
\end{center}
\hrule  height 1pt\relax
\end{table}

Both the "article" and the "slides" formats may also magnify the document. By
default, the magnification of the "slides" format is 4 magsteps, and the
magnification of the "article" format is 0. To find the total font
magnification, add the magnification listed in the table to the magnification
of the document. E.g., in the "slides" format with the "12pt" option (or the
command "\ptsize{12}" at the beginning of a slide), the "\large" command may
invoke fonts that  have a magnification of up to 7 magsteps.

Most systems have font bitmaps for 0, 1/2, 1, 2, 3, 4 and 5 magsteps. If you
obey the restrictions listed on page \pageref{font-res}, then you will only
need fonts in these standard magnifications. However, if you want to use other
font magnifications, then check with your system administrator to determine
which of the following applies to you (or just plunge ahead and see what
happens):
\begin{enumerate}
\item If you have Metafont and "dvips" or some other driver that automatically
generates font bitmaps as needed, and if this feature is enabled, then you
will simply notice that it takes a long time to print documents at first,
because the driver has to wait for the new font bitmaps to be made.
Eventually, you will have generated all the extra bitmaps you need, and this
delay will go away.
\item If you have Metafont, but your dvi driver does not automatically
generate needed bitmaps, then you just need to keep track of what font bitmaps
you are missing as you use "seminar.sty" and occasionally run Metafont to make
them. You will also notice a delay when a font bitmap is missing, because your
driver will probably scale the closest bitmap it finds, and this can take time
on some systems.
\item If you do not even have Metafont, then you have to try to get by without
the extra bitmaps. Scaled fonts look lousy, but an occasional scaled font in a
heading is not so bad.
\end{enumerate}

Of course, PostScript and other scalable fonts do not present any problem, and
so it is a good idea to use these as much as possible. The Computer Modern
fonts are available from Blue Sky Research in PostScript Type I format. If you
do not have PostScript versions of the CM fonts, and you instead use other
PostScript fonts for text, then you are likely to still need bitmapped fonts
for mathematics. However, since mathematics is usually set at "\normalsize" or
smaller, this is not a big problem.

If you are installing "seminar.sty" on a multi-user system, then hopefully you
will make the needed fonts available.

\section{Color}

You can use whatever color commands you ordinarily use with \LaTeX. You might
try the \o{semcolor} style option, which lets you use the PSTricks color
commands for printing on a color PostScript printer. The \o{semcolor} option
combined with the \o{semlayer} option (see Section \ref{S-overlays}) lets you
print color layers.

\section{Landscape printing and slide rotation\label{S-landscape}}

If your dvi driver supports a "\special" for landscape printing, then you can
define \N\printlandscape\ in the preamble of your document to invoke this
command. E.g., for "dvips" put the line
\begin{LVerbatim}
  \renewcommand{\printlandscape}{\special{landscape}}
\end{LVerbatim}
in the preamble. Otherwise, "seminar.sty" will display a message reminding you
to print your document in landscape mode, when appropriate.

You can print both landscape and portrait slides in one shot if you can rotate
the portrait slides when printing in landscape mode or the landscape slides
when printing in portrait mode. If you are using a PostScript printer, you
probably are using a dvi driver that supports rotation; otherwise, you
probably are not. Here are suggestions for setting up rotation:
\begin{itemize}
\item The \o{semcolor} option provides an interface to the rotation macros in
PSTricks (see Appendix \ref{S-semcolor}); it works with many popular dvi-to-ps
drivers.

\item If you are using Rokicki's "dvips" and want rotation but do not want to
load the entire PSTricks package, then use the \O{semrot} option instead of
the \o{semcolor} option.

\item If the \o{semcolor} and \o{semrot} options do not work for you, but you
have your own rotation macros, then you have to define the commands
\begin{MD}
  \leftsliderotation\\
  \rightsliderotation
\end{MD}
so that they rotate something left and right, respectively. For example,
\begin{LVerbatim}
  \renewcommand{\leftsliderotation}[1]{\rotateleft{#1}}
\end{LVerbatim}
\end{itemize}

In any case, you can determine the direction of rotation using
\begin{MD}
  \sliderotation{direction}
\end{MD}
where valid directions are "none", "left" and "right". The default is "left".

By default, the headers and footers aren't rotated, but you can switch between
rotating and not rotating the headers with the commands:
\begin{MD}
  \rotateheaderstrue\\
  \rotateheadersfalse
\end{MD}

\section{The semcolor style option\label{S-semcolor}}

The \O{semcolor} option sets up an interface between "seminar.sty" and the
PSTricks package. PSTricks is a collection of PostScript macros for \TeX. It
works with Rokicki's "dvips", and several other dvi-to-ps drivers. You can
probably get the PSTricks package from wherever you obtained "seminar.sty", or
check the archives listed in the PSTricks read-me file, "read-me.pst", which
is distributed with "seminar.sty".

What the \o{semcolor} style option gives you, compared to just using the
PSTricks package, is:
\begin{description}
\item[Color] A small patch to make the PSTricks color commands more robust in
slides.
\item[Rotation] The rotations "left" and "right" are defined. "left" is the
default.
\item[Framing] The frame styles "scplain" (using "\psframebox"), "scdouble"
(using "\psdblframebox") and "scshadow" (using "\psshadowbox").
\item[Overlays] Overlays and layers, when used in conjunction with the
\o{semlayer} option.
\end{description}

\section{A4 and other paper sizes\label{S-papersizes}}

Use the \O{a4} option when using A4 paper. Note that this option does not
correspond to an independent file.

If you want to configure seminar.sty for A4 paper by default (without having
to include style options), then you can do one of the following:
\begin{enumerate}

  \item Add the following line to "seminar.con" (see Appendix
\ref{S-configfile}):
\begin{LVerbatim}
  %%
%% This file may be distributed and/or modified under the conditions of
%% the LaTeX Project Public License, either version 1.2 of this license
%% or (at your option) any later version.  The latest version of this
%% license is in:
%% 
%%    http://www.latex-project.org/lppl.txt
%% 
%% and version 1.2 or later is part of all distributions of LaTeX version
%% 1999/12/01 or later.
%%
%% BEGIN sem-a4.sty
\def\paperwidth{210mm}
\def\paperheight{297mm}
\input sem-page.sty
\slidewidth 222mm
\slideheight 152mm
\endinput
%% END sem-a4.sty

\end{LVerbatim}

  \item Add the following lines to "seminar.con" (these lines are just the
contents of "sem-a4.sty"):
\begin{LVerbatim}
  \def\paperwidth{210mm}
  \def\paperheight{297mm}
  \input sem-page.sty
  \slidewidth 222mm
  \slideheight 152mm
\end{LVerbatim}

  \item Create a file such as "mysem.sty", to be used as a document {\em
style} (rather than style {\em option}), with the following lines:
\begin{LVerbatim}
  \def\paperwidth{210mm}
  \def\paperheight{297mm}
  \input seminar.sty
  \slidewidth 222mm
  \slideheight 152mm
\end{LVerbatim}

\end{enumerate}

For other paper sizes, you can create a style option by modifying
"sem-a4.sty". Then any of the options described above is available to you,
(with the appropriate parameter values). However, this will give satisfactory
results only for paper sizes that are close to A4 or 8.5in by 11in.

Note that for any paper size, even 8.5in by 11in, the page parameters that are
set this way are just suggested defaults, and most people will want to
customize them. Because of the variety of ways in which "seminar.sty"
documents can be printed, this is a little more complex than with most
document styles. Examine "sem-page.sty" to see what page parameters need to be
set, and when.

\section{Configuration file\label{S-configfile}}

You can put customizations to "seminar.sty" in a file named "seminar.con".
This file is optional. It is loaded by "seminar.sty" if it exists, {\em
before} loading the style option files.  E.g., "seminar.con" might contain the
following lines:
\begin{LVerbatim}
  \input semhelv.sty
  \input semcolor.sty
  \renewcommand{\printlandscape}{\special{landscape}}
  \endinput
\end{LVerbatim}

\clearpage
\PrintUserIndex

\end{document}
%% END sem-user.tex
