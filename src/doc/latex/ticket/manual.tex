\documentclass[a4paper,KOMA,landscape]{powersem}

\usepackage[stmo,button,display]{ifmslide}

%% user definitions
\newcommand{\ticket}{{\code{ticket.sty}}}
\newcommand{\bs}{{\mtt\\}}

\hypersetup{pdfauthor={Thomas Emmel}}
\hypersetup{pdftitle={ticket manual}}
\hypersetup{pdfsubject={ticket.sty}}

\IfFileExists{cmtt.sty}{\usepackage{cmtt}}{}

\usepackage{verbatim}
\usepackage{amssymb}

%%%%%%%%%%%%%%%%%%%%%%%%%%%%%%%%%%%%%%%%%%%%%%%%%%%%%%%%%%%%%%%%%%%%
%%%%%%%%%%%%%%%%%%%%%%%%%%%%%%%%%%%%%%%%%%%%%%%%%%%%%%%%%%%%%%%%%%%%
\begin{document}
\slidepagestyle{panel}
\panelposition{bottom}
\pageTransitionDissolve
\sffamily

\orgname{TU Darmstadt -- Institute of Mechanics -- AG4}

\title{\begin{minipage}[t]{0.98\textwidth}\begin{center}
      {\mdseries \ticket{} v0.3c October 10, 2000}\\[1ex]
      Making labels, visting cards, pins with \LaTeX\\
      A short introduction
    \end{center}\end{minipage}}

\author{\scalebox{1}[1.3]{Thomas Emmel}}

\address{\href{mailto:emmel@mechanik.tu-darmstadt.de}%
  {emmel@mechanik.tu-darmstadt.de}}
\orgurl{http://coulomb.mechanik.tu-darmstadt.de/}
%%%%%%%%%%%%%%%%%%%%%%%%%%%%%%%%%%%%%%%%%%%%%%%%%%%%%%%%%%%%%%%%%%%%
\begin{slide}
  \maketitle
\end{slide}
%%%%%%%%%%%%%%%%%%%%%%%%%%%%%%%%%%%%%%%%%%%%%%%%%%%%%%%%%%%%%%%%%%%%
\slidepagestyle{empty}
\panelposition{outsidebottom}
\hidebackground
\pageTransitionReplace
\centerslidesfalse
\begin{slide}
  \section{A short introduction}
  \ticket{} provides an easy to handle interface to produce 
  visiting cards, labels for your files, stickers, pins and
  other stuff for your office, conferences etc.\\[2ex]
  All you need is a definition of your \lq{}ticket\rq{} included in
  a ticket definition file (\code{.tdf}),\\
  $\rightarrowtail$ see \hyperlink{tdf}{ticket definition file}\\
  and the two commands \code{\bs{}ticketdefault} and \code{\bs{}ticket}.\\
  $\rightarrowtail$ see \hyperlink{ticket}{how to fill your ticket}\\[2ex]
  What you get is shown in some examples:\\
  $\rightarrowtail$ \href{ex_file.pdf}{\code{ex\_file}} back labels for your files...\\
  $\rightarrowtail$ \href{ex_pin.pdf}{\code{ex\_pin}}  pins for a conference...\\
  $\rightarrowtail$ \href{ex_vcard.pdf}{\code{ex\_vcard}} visiting-cards...\\
\end{slide}
%%%%%%%%%%%%%%%%%%%%%%%%%%%%%%%%%%%%%%%%%%%%%%%%%%%%%%%%%%%%%%%%%%%%
\begin{slide}
  \section{ticket definition file}\hypertarget{tdf}{}
A sample ticket definition (\code{lz1680.tdf}) for a product of the german company 
\lq{}LEITZ\rq{} is:
\begin{center}
  \begin{minipage}[t]{0.5\textwidth}
    {\scriptsize
\begin{verbatim}
%%
%% ticket for "Leitz 1680" 
%% pre-cutted label for wide files
%%
\unitlength=1mm

%% tested for an HP5SiMX, adjust it for your printer
\hoffset=-16.2mm
\voffset=-6.6mm

\ticketNumbers{1}{4}

\ticketSize{189}{56.5}     % in unitlength
\ticketDistance{0}{9.8}    % in unitlength
\end{verbatim}
      }
    \end{minipage}
\end{center}
The meaning of \code{\bs{}unitlength} is obvious, \code{\bs{}hoffset} and 
\code{\bs{}voffset} are used to adjust the output.
\code{\bs{}ticketNumbers\{n$_x$\}\{n$_y$\}} are the number of the tickets
on the sheet in horizontal and vertical direction.
\code{\bs{}ticketSize\{width\}\{height\}} is the size of one ticket and
\code{\bs{}ticketDistance\{dist$_x$\}\{dist$_y$\}} are existing distances
between the tickets.
\end{slide}
%%%%%%%%%%%%%%%%%%%%%%%%%%%%%%%%%%%%%%%%%%%%%%%%%%%%%%%%%%%%%%%%%%%%
\begin{slide}
  \section{missing ticket definition file}
If you like to test a new ticket definition or there is no need to 
provide such a file \ticket{} assumes default values which can be
redefined in your file.\\
Simply call \ticket{} without any ticket definition:\\
\code{\bs{}usepackage[other options]\{ticket\}}
\end{slide}
%%%%%%%%%%%%%%%%%%%%%%%%%%%%%%%%%%%%%%%%%%%%%%%%%%%%%%%%%%%%%%%%%%%%
\begin{slide}
  \section{how to fill your ticket}\hypertarget{ticket}{}
  \ticket{} opens a picture-environment for all tickets. You can put things 
  in this picture with \code{\bs{}ticketdefault} and \code{\bs{}ticket}.
  Both commands have one argument.\\
  \code{\bs{}ticketdefault} is used to put a default background to your 
  ticket, which will not changed for all tickets. 
  You have to \code{\bs{}renewcommand} \code{\bs{}ticketdefault} every time
  you change this background:\\[-5ex]
  \begin{center}
    \begin{minipage}[t]{.5\textwidth}
      {\scriptsize
\begin{verbatim}
\renewcommand{\ticketdefault}{%
    \put(100, 5){\includegraphics[width=35mm]{ifmlogoc}}%
    \put(100,50){\begin{rotate}{180}\Huge ...\end{rotate}}%
    }
\end{verbatim}
        \normalsize
        }
    \end{minipage}
  \end{center}
  \code{\bs{}ticket} is used to provide the content which will be changed
  for the individual tickets. The simplest way to do that is to define a new 
  command like this:
  \begin{center}
    \begin{minipage}[t]{0.5\textwidth}
      {\scriptsize
\begin{verbatim}
\newcommand{\myticket}[3]{\ticket{%
    \put(  0, 5){#1}%
    \put( 20,20){{\bfseries\large#2}\\#3}%
    }}
\end{verbatim}
        }
    \end{minipage}
  \end{center}
\end{slide}
%%%%%%%%%%%%%%%%%%%%%%%%%%%%%%%%%%%%%%%%%%%%%%%%%%%%%%%%%%%%%%%%%%%%
\begin{slide}
  \section{special commands}
You can add additional pagebreaks with \code{\bs{}newpage\bs{}ticketreset}. This clears the ticket-buffers and sets the corresponding values to initial values.\\
\end{slide}
%%%%%%%%%%%%%%%%%%%%%%%%%%%%%%%%%%%%%%%%%%%%%%%%%%%%%%%%%%%%%%%%%%%%
\end{document}
%%%%%%%%%%%%%%%%%%%%%%%%%%%%%%%%%%%%%%%%%%%%%%%%%%%%%%%%%%%%%%%%%%%%
%%%%%%%%%%%%%%%%%%%%%%%%%%%%%%%%%%%%%%%%%%%%%%%%%%%%%%%%%%%%%%%%%%%%

