%%
%% This is file `examplen.tex',
%% generated with the docstrip utility.
%%
%% The original source files were:
%%
%% labbook.dtx  (with options: `examplen')
%% 
%% Copyright (C) 2002 Frank K^^fcster.
%% 
%% All rights reserved.
\ProvidesFile{examplen.tex}%
            [2003/05/20 v1.1 LaTeX2e labbook document class]

\documentclass[idxtotoc,hyperref]{labbook}


\usepackage[%
  backref=page,%
  pdfpagelabels=true,%
  plainpages=false,%
  colorlinks=true,%
  bookmarks=true,%
  pdfview=FitB]{hyperref}

%%
%% This is file `boilerplates.tex',
%% generated with the docstrip utility.
%%
%% The original source files were:
%%
%% labbook.dtx  (with options: `boilerplates')
%% 
%% Copyright (C) 2002 Frank K^^fcster.
%% 
%% All rights reserved.
\ProvidesFile{boilerplates.tex}%
            [2003/05/20 v1.1 LaTeX2e labbook document class]
\newcommand*{\sometext}{A rather lengthy description which is just
  here to use some space, so that one can see the effect of the index
  ranges. }
\newcommand*{\Xsometext}{\sometext\sometext\sometext\sometext
  \sometext\sometext\sometext\sometext\sometext\sometext}
\newcommand*{\othertext}{Some shorter Text, the cells will only grow
  if you're not a muggle\dots }
\newcommand*{\Xothertext}{\othertext\othertext\othertext
  \othertext\othertext\othertext\othertext}

\newcommand{\sdsband}{\rule{1cm}{0.15cm}}
\newcommand{\sdsbandh}{\rule[0.2cm]{1cm}{0.15cm}}
\newcommand{\sdsbandl}{\rule[-0.15cm]{1cm}{0.15cm}}
\newcommand*{\sdsgel}{%
    \begin{tabular}{cccccccc}
\sdsband & \sdsbandl & \sdsband & \sdsbandh &
  \sdsbandh & \sdsband & \sdsbandl & \sdsband \\[0.5cm]
\sdsbandl & \sdsband & \sdsbandh & \sdsbandl &
  \sdsband & \sdsbandl & \sdsbandh & \sdsband \\[0.5cm]
\sdsbandh & \sdsbandl & \sdsbandh & \sdsband &
  \sdsbandl & \sdsbandh & \sdsband & \sdsbandl \\[0.5cm]
\sdsband & \sdsbandh & \sdsbandl & \sdsbandl &
  \sdsband & \sdsbandh & \sdsbandl & \sdsbandh \\[0.5cm]
\sdsbandl & \sdsbandl & \sdsbandh & \sdsband &
  \sdsbandh & \sdsbandl & \sdsbandh & \sdsband \\[0.5cm]
    \end{tabular}
}
\endinput
%%
%% End of file `boilerplates.tex'.



\newexperiment{prep_S-peptide}{Preparation of RNase A S-peptide}
\newsubexperiment{subtilisin}[subtilisin digest]{Digestion with
  Subtilisin}

\newexperiment{prep?some1th&^ing}{Preparation of purified something}

\newexperiment{lyoph-A}{Lyophilization of A}
\newexperiment{lyoph-B}{Lyophilization of B}

\begin{document}

\frontmatter
\title{Laboratory Journal from 2003-04-22 to \dots}
\author{Jane Eager}
\maketitle

\printindex
\tableofcontents

\mainmatter

\labday{Tuesday, April 22, 2003}

\experiment{prep_S-peptide}
\subexperiment{subtilisin}

\Xsometext\par
\Xsometext\par
\Xsometext\par
\Xsometext\par
\begin{figure}[htbp]
  \centering
  \sdsgel
  \caption{Some very interesting SDS gel}
  \label{fig:digest-sds}
\end{figure}

\experiment{Expression of Hagridin in \textit{E. coli}}

\subexperiment[Transformation with HagIV plasmid]{Transformation of
  hogwarts-auxotrophic \textit{E. coli} with the HagIV
  plasmid}

\Xothertext\Xothertext

\labday{Wednesday, April 23, 2003}

\experiment{prep_S-peptide}

\subexperiment[subtilisin]{Digestion using the new batch of
  Subtilisin}

\experiment{prep?some1th&^ing}

\experiment{Preparation of A}

A was prepared according to the protocol in Muller \textit{et al.} and
was temporarily stored in the freezer.

\labday{Thursday, April 24, 2003}

\experiment{Preparation of B}

B was prepared according to the protocol by Smith and Fox.


\experiment[Lyophilization of A and B,lyoph-A, lyoph-B ]{Lyophilization
  of A and B}

\end{document}
\endinput
%%
%% End of file `examplen.tex'.
