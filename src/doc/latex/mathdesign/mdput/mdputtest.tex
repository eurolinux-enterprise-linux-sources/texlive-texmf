\documentclass[fleqn]{article}

\usepackage[fraktur,mdput]{mathdesign}

\title{A \LaTeX\ math test document}
\author{for fonts created by Math Design}

\raggedbottom

\newcommand{\testsize}[1]{
   #1 \texttt{\string#1}: \(a_{c_e}, b_{d_f}, C_{E_G}, 0_{1_2},
      a_{0_a}, 0_{a_0},
      \sum_{i=0}^\infty\) \\
}

\newcommand{\testdelims}[3]{\sqrt{
   #1|#1\|#1\uparrow
   #1\downarrow#1\updownarrow#1\Uparrow#1\Downarrow
   #1\Updownarrow#1\lfloor#1\lceil
   #1(#1\{#1[#1\langle
      #3
   #2\rangle#2]#2\}#2)
   #2\rceil#2\rfloor#2\Updownarrow#2\Downarrow
   #2\Uparrow#2\updownarrow#2\downarrow#2\uparrow
   #2\|#2|
}\\}

\newcommand{\testglyphs}[1]{
\begin{quote}
   #1a#1b#1c#1d#1e#1f#1g#1h#1i#1j#1k#1l#1m
   #1n#1o#1p#1q#1r#1s#1t#1u#1v#1w#1x#1y#1z
   #1A#1B#1C#1D#1E#1F#1G#1H#1I#1J#1K#1L#1M
   #1N#1O#1P#1Q#1R#1S#1T#1U#1V#1W#1X#1Y#1Z
   #10#11#12#13#14#15#16#17#18#19
   #1\Gammait#1\Deltait#1\Thetait#1\Lambdait#1\Xiit
   #1\Piit#1\Sigmait#1\Upsilonit#1\Phiit#1\Psiit#1\Omegait
   #1\alpha#1\beta#1\gamma#1\digamma#1\delta#1\epsilon
   #1\varepsilon#1\zeta#1\eta#1\theta#1\vartheta
   #1\iota#1\kappa#1\varkappa#1\lambda#1\mu#1\nu#1\xi#1\omicron
   #1\pi#1\varpi#1\rho#1\varrho
   #1\sigma#1\varsigma#1\tau#1\upsilon#1\phi
   #1\varphi#1\chi#1\psi#1\omega
   #1\Gamma#1\Delta#1\Theta#1\Lambda#1\Xi
   #1\Pi#1\Sigma#1\Upsilon#1\Phi#1\Psi#1\Omega
   #1\alphaup#1\betaup#1\gammaup#1\digammaup#1\deltaup#1\epsilonup
   #1\varepsilonup#1\zetaup#1\etaup#1\thetaup#1\varthetaup
   #1\iotaup#1\kappaup#1\varkappaup#1\lambdaup#1\muup#1\nuup#1\xiup#1\omicron
   #1\piup#1\varpiup#1\rhoup#1\varrhoup
   #1\sigmaup#1\varsigmaup#1\tauup#1\upsilonup#1\phiup
   #1\varphiup#1\chiup#1\psiup#1\omegaup
   #1\partial#1\ell#1\imath#1\jmath#1\wp
\end{quote}
}

\newcommand{\parenthesis}[1]{ $(#1)$ }
\newcommand{\sidebearings}[1]{ $|#1|$  }
\newcommand{\subscripts}[1]{ $#1_\circ$  }
\newcommand{\supscripts}[1]{ $#1^\_$  }
\newcommand{\scripts}[1]{ $#1^2_\circ$  }
\newcommand{\vecaccents}[1]{ $\vec#1$  }
\newcommand{\tildeaccents}[1]{ $\tilde#1$  }


\ifx\omicron\undefined
   \let\omicron=o
\fi

\parindent 0pt
\mathindent 1em

\def\test#1{#1}

\def\testnums{%
  \test 0 \test 1 \test 2 \test 3 \test 4 \test 5 \test 6 \test 7
  \test 8 \test 9 }
\def\testupperi{%
  \test A \test B \test C \test D \test E \test F \test G \test H
  \test I \test J \test K \test L \test M }
\def\testupperii{%
  \test N \test O \test P \test Q \test R \test S \test T \test U
  \test V \test W \test X \test Y \test Z }
\def\testupper{%
  \testupperi\testupperii}

\def\testloweri{%
  \test a \test b \test c \test d \test e \test f \test g \test h
  \test \imath \test \jmath \test k \test l \test m }
\def\testlowerii{%
  \test n \test o \test p \test q \test r \test s \test t \test u
  \test v \test w \test x \test y \test z
  \test\imath \test\jmath }
\def\testlower{%
  \testloweri\testlowerii}

\def\testupgreeki{%
  \test A \test B \test\Gamma \test\Delta \test E \test Z \test H
  \test\Theta \test I \test K \test\Lambda \test M }
\def\testupgreekii{%
  \test N \test\Xi \test O \test\Pi \test P \test\Sigma \test T
  \test\Upsilon \test\Phi \test X \test\Psi \test\Omega
  \test\nabla }
\def\testupgreek{%
  \testupgreeki\testupgreekii}

\def\testlowgreeki{%
  \test\alpha \test\beta \test\gamma \test\delta \test\epsilon
  \test\zeta \test\eta \test\theta \test\iota \test\kappa \test\lambda
  \test\mu }
\def\testlowgreekii{%
  \test\nu \test\xi \test o \test\pi \test\rho \test\sigma \test\tau
  \test\upsilon \test\phi \test\chi \test\psi \test\omega }
\def\testlowgreekiii{%
  \test\varepsilon \test\vartheta \test\varpi \test\varrho
  \test\varsigma \test\varphi}
\def\testlowgreek{%
  \testlowgreeki\testlowgreekii\testlowgreekiii}

\DeclareMathSymbol{\dit}{\mathord}{letters}{`d}
\DeclareMathSymbol{\dup}{\mathord}{operators}{`d}

\newenvironment{boldface}{\bgroup\mathversion{bold}%
  \def\it{\fontseries{b}\fontshape{it}\selectfont}%
  \fontseries{b}\selectfont }{\egroup}

\begin{document}

\maketitle

\section*{Introduction}

This document tests the math capabilities of the mdputpackage, and is
strongly modelled after a similar document by Alan Jeffrey.

This test exercises the {\tt MathDesign mdput} math fonts combined with the
{\tt put} text fonts.

\section*{Math Alphabets}

Math italic:
$$
   ABCDEFGHIJKLMNOPQRSTUVWXYZ
   abcdefghijklmnopqrstuvwxyz
$$
Text italic:
$$
   \mathit{ABCDEFGHIJKLMNOPQRSTUVWXYZ
      abcdefghijklmnopqrstuvwxyz}
$$
Roman:
$$
   \mathrm{ABCDEFGHIJKLMNOPQRSTUVWXYZ
      abcdefghijklmnopqrstuvwxyz}
$$
Bold:
$$
   \mathbf{ABCDEFGHIJKLMNOPQRSTUVWXYZ
      abcdefghijklmnopqrstuvwxyz}
$$
Typewriter:
$$
   \mathtt{ABCDEFGHIJKLMNOPQRSTUVWXYZ
      abcdefghijklmnopqrstuvwxyz}
$$

AMS like Symbol:
$$
   \yen \geqq \circeq \daleth \varkappa \leftarrowtail \because
   \eqslantless  \eqslantgtr \curlyeqprec
$$

Greek:
$$
   \Gamma\Delta\Theta\Lambda\Xi\Pi\Sigma\Upsilon\Phi\Psi\Omega
  \alpha\beta\gamma\delta\epsilon\varepsilon\zeta\eta\theta\vartheta
   \iota\kappa\lambda\mu\nu\xi\omicron\pi\varpi\rho\varrho
   \sigma\varsigma\tau\upsilon\phi\varphi\chi\psi\omega
$$
{\mathversion{bold}
$$
       \Gamma\Delta\Theta\Lambda\Xi\Pi\Sigma\Upsilon\Phi\Psi\Omega
  \alpha\beta\gamma\delta\epsilon\varepsilon\zeta\eta\theta\vartheta
   \iota\kappa\lambda\mu\nu\xi\omicron\pi\varpi\rho\varrho
   \sigma\varsigma\tau\upsilon\phi\varphi\chi\psi\omega
$$}

Calligraphic:
$$A\mathcal{ABCDEFGHIJKLMNOPQRSTUVWXYZ}Z$$
Sans:
$$
   A\mathsf{ABCDEFGHIJKLMNOPQRSTUVWXYZ}Z \quad
      a\mathsf{abcdefghijklmnopqrstuvwxyz}z
$$
Fraktur:
$$
   A\mathfrak{ABCDEFGHIJKLMNOPQRSTUVWXYZ}Z 
$$
$$
   a\mathfrak{abcdefghijklmnopqrstuvwxyz}z
$$

Blackboard Bold:
$$
   A\mathbb{ABCDEFGHIJKLMNOPQRSTUVWXYZ}Z
$$

\section*{Symbols}

$$ \frac{\partial f}{\partial x} $$

$$
   a \hookrightarrow b \hookleftarrow c \longrightarrow d
   \longleftarrow e \Longrightarrow f \Longleftarrow g
   \longleftrightarrow h \Longleftrightarrow i
   \mapsto j
$$
$$\textstyle
   \oint \int \quad
   \bigodot \bigoplus \bigotimes \sum \prod
   \bigcup \bigcap \biguplus \bigwedge \bigvee \coprod
$$
$$
   \oint \int \quad
   \bigodot \bigoplus \bigotimes \sum \prod
   \bigcup \bigcap \biguplus \bigwedge \bigvee \coprod
$$
$$ \bigodot_{i=1}^n \gamma_i = \bigoplus_{i=1}^n \gamma_i
=\bigotimes_{i=1}^n \gamma_i = \sum_{i=1}^n \gamma_i = \prod_{i=1}^n
\gamma_i = \bigcup_{i=1}^n \gamma_i = \bigcap_{i=1}^n \gamma_i =
\biguplus_{i=1}^n \gamma_i = \bigwedge_{i=1}^n \gamma_i=
\bigvee_{i=1}^n \gamma_i = \coprod_{i=1}^n \gamma_i
$$

\clearpage

\section*{Big operators}

\def\testop#1{#1_{i=1}^{n} x^{n} \quad}
\begin{displaymath}
  \testop\sum
  \testop\prod
  \testop\coprod
  \testop\int
  \testop\oint
\end{displaymath}
\begin{displaymath}
  \testop\bigotimes
  \testop\bigoplus
  \testop\bigodot
  \testop\bigwedge
  \testop\bigvee
  \testop\biguplus
  \testop\bigcup
  \testop\bigcap
  \testop\bigsqcup
% \testop\bigsqcap
\end{displaymath}


\section*{Radicals}

\begin{displaymath}
  \sqrt{x+y} \qquad \sqrt{x^{2}+y^{2}} \qquad
  \sqrt{x_{i}^{2}+y_{j}^{2}} \qquad
  \sqrt{\left(\frac{\cos x}{2}\right)} \qquad
  \sqrt{\left(\frac{\sin x}{2}\right)}
\end{displaymath}

\begingroup
\delimitershortfall-1pt
\begin{displaymath}
  \sqrt{\sqrt{\sqrt{\sqrt{\sqrt{\sqrt{\sqrt{x+y}}}}}}}
\end{displaymath}
\endgroup % \delimitershortfall


\section*{Over- and underbraces}

\begin{displaymath}
  \overbrace{x} \quad
  \overbrace{x+y} \quad
  \overbrace{x^{2}+y^{2}} \quad
  \overbrace{x_{i}^{2}+y_{j}^{2}} \quad
  \underbrace{x} \quad
  \underbrace{x+y} \quad
  \underbrace{x_{i}+y_{j}} \quad
  \underbrace{x_{i}^{2}+y_{j}^{2}} \quad
\end{displaymath}


\section*{Normal and wide accents}

\begin{displaymath}
  \dot{x} \quad
  \ddot{x} \quad
  \vec{x} \quad
  \bar{x} \quad
  \overline{x} \quad
  \overline{xx} \quad
  \tilde{x} \quad
  \widetilde{x} \quad
  \widetilde{xx} \quad
  \widetilde{xxx} \quad
  \hat{x} \quad
  \widehat{x} \quad
  \widehat{xx} \quad
  \widehat{xxx} \quad
\end{displaymath}

\def\testwilde#1{
    \begin{displaymath}
        #1{a} \quad
        #1{ab} \quad 
        #1{abc} \quad 
        #1{abcde} \quad 
        #1{abcdefg} \quad 
        #1{abcdefghi} \quad 
        #1{abcdefghijk} \quad 
    \end{displaymath}}

\testwilde\widehat
\testwilde\widetilde
\testwilde\widetriangle
\testwilde\wideparen


\section*{Long arrows}

\begin{displaymath}
    \leftrightarrow \quad
  \longleftarrow  \quad
  \longrightarrow \quad
  \longleftrightarrow \quad
  \Leftrightarrow \quad
  \Longleftarrow  \quad
  \Longrightarrow \quad
  \Longleftrightarrow \quad
\end{displaymath}


\section*{Left and right delimters}

\def\testdelim#1#2{ - #1 f #2 - }
\begin{displaymath}
  \testdelim()
  \testdelim[]
  \testdelim\lfloor\rfloor
  \testdelim\lceil\rceil
  \testdelim\langle\rangle
  \testdelim\{\}
\end{displaymath}

\clearpage
\section*{Big-g-g delimters}

\def\testdelim#1#2{
  -
  \left#1\left#1\left#1\left#1\left#1\left#1\left#1\left#1\left#1\left#1\left#1
  #1 -
  #2 \right#2\right#2\right#2\right#2\right#2\right#2\right#2\right#2\right#2\right#2\right#2 -}

\begingroup
\delimitershortfall-1pt
\begin{displaymath}
  \testdelim\lfloor\rfloor
  \qquad
  \testdelim()
\end{displaymath}
\begin{displaymath}
  \testdelim\lceil\rceil
  \qquad
  \testdelim\{\}
\end{displaymath}
\begin{displaymath}
  \testdelim\llbracket\rrbracket 
  \qquad 
  \testdelim\lwave\rwave
\end{displaymath}
\begin{displaymath}
  \testdelim[]
  \qquad
  \testdelim\lgroup\rgroup
\end{displaymath}
\begin{displaymath}
  \testdelim\langle\rangle
  \qquad
  \testdelim\lmoustache\rmoustache
\end{displaymath}
\begin{displaymath}
  \testdelim\uparrow\downarrow \quad
  \testdelim\Uparrow\Downarrow \quad
\end{displaymath}
\endgroup % \delimitershortfall

\section*{Delimiters}

Each row should be a different size, but within each row the delimiters
should be the same size.  First with \verb|\big|, etc:
$$\begin{array}{c}
   \testdelims\relax\relax{J}
   \testdelims\bigl\bigr{J}
   \testdelims\Bigl\Bigr{J}
   \testdelims\biggl\biggr{J}
   \testdelims\Biggl\Biggr{J}
\end{array}$$
Then with \verb|\left| and \verb|\right|:
$$\begin{array}{c}
   \testdelims\left\right{\begin{array}{c} f \end{array}}
   \testdelims\left\right{\begin{array}{c} a\\f \end{array}}
   \testdelims\left\right{\begin{array}{c} a\\a\\f \end{array}}
   \testdelims\left\right{\begin{array}{c} a\\a\\a\\f \end{array}}
\end{array}$$

\section*{Sizing}

$$
    abcde + x^{abcde} + 2^{x^{abcde}}
$$

The subscripts should be appropriately sized:
\begin{quote}
\testsize\tiny
\testsize\scriptsize
\testsize\footnotesize
\testsize\small
\testsize\normalsize
\testsize\large
\testsize\Large
\testsize\LARGE
\testsize\huge
\testsize\Huge

\end{quote}

\clearpage

\section*{Spacing}

This paragraph should appear to be a monotone grey texture.  Suppose
\(f \in \mathcal{S}_n\) and \(g(x) = (-1)^{|\alpha|}x^\alpha
f(x)\).  Then \(g \in \mathcal{S}_n\); now (\emph{c}) implies
that \(\hat g = D_\alpha \hat f\) and \(P \cdot D_\alpha\hat
f = P \cdot \hat g = (P(D)g)\hat{}\), which is a bounded function,
since \(P(D)g \in L^1(R^n)\).  This proves that \(\hat f \in
\mathcal S_n\).  If \(f_i \rightarrow f\) in \(\mathcal S_n\),
then \(f_i \rightarrow f\) in \(L^1(R^n)\).  Therefore \(\hat
f_i(t) \rightarrow \hat f(t)\) for all \(t \in R^n\).  That \(f
\rightarrow \hat f\) is a \emph{continuous} mapping of
\(\mathcal S_n\) into \(\mathcal S_n\) follows now from the
closed graph theorem.  And thus for \(x_1\) through \(x_i\).
\emph{Functional Analysis}, W.~Rudin,
McGraw--Hill, 1973.

\begin{boldface}
This paragraph should appear to be a monotone dark texture.  Suppose
\(f \in \mathcal{S}_n\) and \(g(x) = (-1)^{|\alpha|}x^\alpha
f(x)\).  Then \(g \in \mathcal{S}_n\); now (\emph{c}) implies
that \(\hat g = D_\alpha \hat f\) and \(P \cdot D_\alpha\hat
f = P \cdot \hat g = (P(D)g)\hat{}\), which is a bounded function,
since \(P(D)g \in L^1(R^n)\).  This proves that \(\hat f \in
\mathcal S_n\).  If \(f_i \rightarrow f\) in \(\mathcal S_n\),
then \(f_i \rightarrow f\) in \(L^1(R^n)\).  Therefore \(\hat
f_i(t) \rightarrow \hat f(t)\) for all \(t \in R^n\).  That \(f
\rightarrow \hat f\) is a \emph{continuous} mapping of
\(\mathcal S_n\) into \(\mathcal S_n\) follows now from the
closed graph theorem.  And thus for \(x_1\) through \(x_i\).
\emph{Functional Analysis}, W.~Rudin, McGraw--Hill, 1973.
\end{boldface}

{\itshape This paragraph should appear to be a monotone grey texture.
Suppose \(f \in \mathcal{S}_n\) and \(g(x) =
(-1)^{|\alpha|}x^\alpha f(x)\).  Then \(g \in \mathcal{S}_n\);
now (\emph{c}) implies that \(\hat g = D_\alpha \hat f\) and
\(P \cdot D_\alpha\hat f = P \cdot \hat g = (P(D)g)\hat{}\),
which is a bounded function, since \(P(D)g \in L^1(R^n)\).  This
proves that \(\hat f \in \mathcal S_n\).  If \(f_i \rightarrow
f\) in \(\mathcal S_n\), then \(f_i \rightarrow f\) in
\(L^1(R^n)\).  Therefore \(\hat f_i(t) \rightarrow \hat f(t)\)
for all \(t \in R^n\).  That \(f \rightarrow \hat f\) is a
\emph{continuous} mapping of \(\mathcal S_n\) into \(\mathcal
S_n\) follows now from the closed graph theorem.  \emph{Functional
Analysis}, W.~Rudin, McGraw--Hill, 1973.}

The text in these boxes should spread out as much as the math does:
$$\begin{array}{c}
   \framebox[.95\width][s]{For example \(x+y = \min\{x,y\}
      + \max\{x,y\}\) is a formula.} \\
   \framebox[.975\width][s]{For example \(x+y = \min\{x,y\}
      + \max\{x,y\}\) is a formula.} \\
   \framebox[\width][s]{For example \(x+y = \min\{x,y\}
      + \max\{x,y\}\) is a formula.} \\
   \framebox[1.025\width][s]{For example \(x+y = \min\{x,y\}
      + \max\{x,y\}\) is a formula.} \\
   \framebox[1.05\width][s]{For example \(x+y = \min\{x,y\}
      + \max\{x,y\}\) is a formula.} \\
   \framebox[1.075\width][s]{For example \(x+y = \min\{x,y\}
      + \max\{x,y\}\) is a formula.} \\
   \framebox[1.1\width][s]{For example \(x+y = \min\{x,y\}
      + \max\{x,y\}\) is a formula.} \\
   \framebox[1.125\width][s]{For example \(x+y = \min\{x,y\}
      + \max\{x,y\}\) is a formula.} \\
\end{array}$$
\end{document}

%% Local Variables:
%% mode: latex
%% End:
