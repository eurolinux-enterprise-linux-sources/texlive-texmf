\documentclass[a4paper]{article}
\usepackage{german}
\setlength{\unitlength}{1mm}

\title{Akletter.cls\\Anleitung}
\author{Axel Kielhorn\\a.kielhorn@tu-bs.de}

\begin{document}

\section{Vorwort}

Als ich 1992 zum ersten Mal mit \TeX\ in Ber"uhrung kam, lag 
diesem ein deutscher Brief-Stil, benannt nach seinem Autor, 
Stefan Lindner, bei. Mit wenig Aufwand habe ich ihn an meine 
Bed"urfnisse angepa"st, ohne zu Verstehen, was ich da eigentlich 
"andere.

Im Fr"uhjahr 1994 hatten sich bereits drei sehr "ahnliche 
Brief-Stile angesammelt und ich wurde des "ofteren gefragt "`Wie 
machst Du das?"' Nachdem ich jedes mal wieder erkl"aren mu"ste, 
was denn nun anzupassen sei und wovon unbedingt die Finger zu 
lassen sein, entschlo"s ich mich, einen einfach konfigurierbaren 
Brief-Stil zu schreiben. (Meine \LaTeX-Kenntnisse hatten 
inzwischen zugenommen.)

Hauptproblem war jedoch das auf meinem Rechner installierte 
\LaTeX\ mit NFSS, dessen F"ahigkeiten ich gerne ausnutzen wollte, 
das aber in der weiten Welt (dem Rechner im Nebenraum) auf 
Unverst"andnis stie"s. Das neue \LaTeXe\ mit integriertem NFSS war 
die L"osung aller Probleme, so entstand der erste deutsche 
Brief-Stil f"ur \LaTeXe. (Inzwischen sind noch ein paar andere 
dazugekommen.)

Und die Benutzerinnen sahen, das es fehlerhaft war, und 
schickten haufenweise Bug-Reports. Wie zu erwarten war befanden 
sich noch viele kleine Fehler in \texttt{myletter.cls} die in 
Zusammenarbeit mit vielen Betatesterinnen jedoch gefunden und 
beseitigt werden konnten.

Seit Version 1.5 hei"st die Klasse nun \texttt{akletter.cls} und hat 
ein leicht ver"andertes Interface.  Au"serdem ist 
\texttt{akletter.cls} jetzt mehrsprachig geworden und sollte auch mit 
\texttt{babel} zusammenarbeiten.\footnote{Ich benutze nur 
\texttt{german.sty}}

\section{Was kann \texttt{akletter.cls}?}

\texttt{akletter} implementiert ein festes Layout mit 
verschiedenen Boxen, deren Inhalt (in gewissen Grenzen) frei 
definierbar ist. Das Layout hat ungef"ahr folgendes Aussehen:

\begin{picture}(100.00,140.00)
\put(5.00,5.00){\framebox(90.00,130.00){}} % Blatt
\put(5.00,5.00){\framebox(90.00,20.00){}}  % Firmfoot
\put(5.00,25.00){\framebox(45.00,5.00){}}  % Closing
\put(5.00,60.00){\framebox(45.00,5.00){}}  % Opening
\put(5.00,70.00){\framebox(60.00,5.00){}}  % Subject
\put(65.00,70.00){\framebox(30.00,5.00){}} % Date
\put(5.00,75.00){\framebox(90.00,5.00){}}  % Refline
\put(65.00,85.00){\framebox(30.00,35.00){}}% Rightfield
\put(5.00,85.00){\framebox(45.00,25.00){}} % Addressfield
\put(5.00,110.00){\framebox(45.00,5.00){}} % Firmreturn
\put(10.00,116,50){\textit{addrfieldsep}}
\put(5.00,120.00){\framebox(90.00,15.00){}}% FIRM
\put(65.00,120.00){\framebox(30.00,15.00){}}% Firmaddress
\put(10.00,126.5){{FIRM}}
\put(10.00,111.5){{Firmreturn}}
\put(10.00,96.5){{Addressfield}}
\put(10.00,81.5){\textit{datefieldsep}}
\put(10.00,76.5){{Ref"|line}}
\put(10.00,71.5){{Subject}}
\put(10.00,66.5){\textit{openingsep}}
\put(10.00,61.5){{Opening}}
\put(10.00,26.5){{Closing}}
\put(10.00,16.5){{Firmfoot}}
\put(70.00,126.5){{Firmaddress}}
\put(70.00,106.5){{Right-}}
\put(70.00,101.5){{box}}
\put(70.00,71.5){{Date}}
\end{picture}

Ein Teil der Felder ist statisch, wird also einmal aufgebaut und 
kann dann nicht mehr ver"andert werden, hierzu geh"oren:
\begin{itemize}
\item FIRM
\item Firmaddress
\item Firmreturn
\item Firmfoot.
\end{itemize}

Die Breite dieser Felder ist abh"anging von \texttt{textwidth}, 
"andert sich, wenn man das Papierformat wechselt. Dies sollte bei der 
Definition des Inhalts beachtet werden.

Der Rest wird f"ur jeden Brief neu aufgebaut (dies ist wichtig 
wenn mehr als ein Brief in einem Dokument ist):
\begin{itemize}
\item ref
\item subject
\item opening
\item closing
\end{itemize}

Seit Version 1.5i ist der Abstand zwischen den Boxen "uber Parameter
einstellbar. Dadurch ist es m"oglich, das Layout an Fensterumschl"age
anzupassen, die nicht der DIN entsprechen. Zus"atzlich wurde eine neue Option
\texttt{reverseaddr} eingef"uhrt, die Addressfield und Rightbox vertauscht.

Das Layout dieser Felder wird in einer \texttt{cfg}-Datei 
festgelegt, welche sich mit dem \verb+\usename+-Befehl ausw"ahlen 
l"a"st. Ohne diesen Befehl wird \texttt{akletter.cfg} geladen. 
Durch dieses Konzept ist es m"oglich mehrere \texttt{cfg}-Dateien 
zu installieren und diese bei einem Update von 
\texttt{akletter.cls} ohne "Anderungen\footnote{Bein Wechsel von 
Version 1.3 auf 1.4 waren leider doch "Anderungen notwendig :-(} 
weiterzuverwenden. Die endg"ultige Anpassung an \texttt{babel} wird
sicher auch noch ein paar "Anderungen erfordern.

Mit der Version 1.5 wurde der Name auf \texttt{akletter} ge"andert. 
Der Pr"afix \texttt{my} sollte f"ur lokale Styles reserviert bleiben. 
Styles, die "uber CTAN verteilt werden, sollten eindeutig 
identifizierbar sein. 

\section{Die \texttt{cfg}-Datei}

Diese Datei stellt die unterste Ebene der 
Benutzerinnenschnittstelle dar, sie wird in der Regel einmalig 
(bei der Installation) bearbeitet und dann vergessen. Daher ist 
es wichtig die \texttt{akletter.cfg}-Datei nicht zu l"oschen, da 
sich darin noch viele Anregungen f"ur sp"ater befinden.

Auf Anregung von Jan Braun hat sich hier einiges getan:
\texttt{akletter.cls} sucht nun nach \texttt{akletter.cfg} wenn 
kein Benutzername angegeben wird. Diese Datei sollte so ge"andert 
werden, das sie mit den "ortlichen Gegebenheiten (Institut/Firma) 
"ubereinstimmt. Die pers"onliche \texttt{cfg}-Datei enth"alt dann 
nur noch die "Anderungen zu diesem Normal, wie z.\,B. Namen, 
Durchwahl, Zimmernummer~\dots\ Die pers"onlche Konfigurationsdatei 
l"adt dann einfach die Firmendatei nach.

Hier zuerst die Firmendatei:

\begin{verbatim}
\makeatletter
\end{verbatim}

Das ist notwendig, damit die internen Befehle definiert werden 
k"onnen.

\begin{verbatim}
\telephone{0800 / 12 34 56 }
\telefax  {0800 / 12 34 57 }
%% You may include logos and other graphic-material into
%% your letterhead:
%%
%%\RequirePackage[dvips]{graphics}
\end{verbatim}

Neben der M"oglichkeit Bilder mit der \texttt{graphics}-Package 
einzubinden gibt es nat"urlich noch viele andere M"oglichkeiten. 
Mit \texttt{graphics} ist es getestet und sollte daher 
funktionieren. (\texttt{graphics.sty} braucht entweder eine 
Option, die den DVI-Treiber angibt oder eine Datei 
\texttt{graphics.cfg})

\begin{verbatim}
%% You may redefine the following variables
%%

%\renewcommand*{\yourrefname}   {Ihr Zeichen:}
%\renewcommand*{\yourmailname}  {Ihr Schreiben vom:}
%\renewcommand*{\myrefname}     {Unser Zeichen:}
%\renewcommand*{\mymailname}    {Unser Schreiben vom:}
%\renewcommand*{\customername}  {Kundennummer }
%\renewcommand*{\invoicename}   {Rechnungsnummer }
%\renewcommand*{\subjectname}   {}  %{Betr.}
%\renewcommand*{\ccname}        {Kopien an:}
%\renewcommand*{\enclname}      {Anlagen:}
%\renewcommand*{\headtoname}    {An}
%\renewcommand*{\datename}      {Datum}
%\renewcommand*{\pagename}      {Seite}
%\renewcommand*{\telephonename} {Telefon}
%\renewcommand*{\telefaxname}   {FAX}
\end{verbatim}

Diese Namen k"onnen in der \texttt{cfg}-Datei an "ortliche 
Gegebenheiten angepa"st werden, au"serdem ist so eine Anpassungen 
an andere Sprachen m"oglich, auch wenn diese nicht explizit von 
\texttt{german.sty}/\texttt{babel.sty} unterst"utzt werden. Die 
Definitionen sind kompatibel zu \texttt{scrlettr.cls} aus dem 
\texttt{Koma-Script}-Paket. Nicht alle Namen werden tats"achlich 
genutzt. Damit das auch mit \texttt{babel} funtioniert, ist noch 
etwas mehr Aufwand notwendig, siehe \texttt{akletter.cfg}.

\begin{verbatim}
%% Alternate definitions for the reference-line.
%% Use this if you do not need a mymail-entry.
%% In fact you can do a lot of things with these
%% 4 entries, e.g. setting \mymailname to {InvoiceNo}
%% and using \mymail to display an invoiceno.
%% You may change these lines but make sure that
%% the combined length of the 4 fields is 1\leftfield
%% or less.
%\def\yref#1{\def\@yref{\parbox[t]{.33\leftfield}
%    {\scriptsize \yourrefname\\ \normalsize #1}}
%     reftrue}
%\def\ymail#1{\def\@ymail{\parbox[t]{.33\leftfield}
%    {\scriptsize \yourmailname\\ \normalsize #1}}
%     reftrue}
%\def\myref#1{\def\@myref{\parbox[t]{.33\leftfield}
%    {\scriptsize \myrefname\\ \normalsize #1}}
%     reftrue}
%\let\mymail=\@gobbleone
\end{verbatim}

Die Referenz-Zeile in \texttt{akletter.cls} ist nach DIN 
676 ausgelegt, kann aber beliebig ver"andert werden, die Summe 
der Eintr"age sollte jedoch in eine Zeile passen.

\begin{verbatim}
%%% Top of first page: left
\sbox{\FIRM}
   {\parbox[t]{\leftfield}
   {\fontsize{17.28}{22pt}\fontseries{bx}\selectfont%
    Mein Name\hfill}}
%%% Top of first page: right
\sbox{\firmaddress}
% Use either Text:
   {\parbox[t]{\rightfield}{%
    \fontsize{9}{10pt}\selectfont\normalfont
    Meine Stra{\ss}e 1\\ 12345 Meindorf}}
% or graphic:
% {\includegraphics{ourlogo}}
% and don't forget to adjust FIRM and firmaddress:
%%% Top of first page: left + right
%%% You may want to adjust the raisebox
\sbox{\FIRMHEAD}
   {\usebox{\FIRM}\raisebox{3mm}{\usebox{\firmaddress}}}
\end{verbatim}

Der Briefkopf der ersten Seite setzt sich aus zwei Elementen 
zusammen, der Box \texttt{FIRM} und der Box 
\texttt{firmaddress}. Die Breite der Boxen wird in 
\texttt{akletter.cls} festgelegt, \texttt{rightfield} ist immer 42 mm 
breit, \texttt{leftfield} ist \texttt{textwidth} $-$ 
\texttt{rightfield}. Beide werden in die Box \texttt{FIRMHEAD} 
gepackt, wobei die horizontale Ausrichtung stattfindet. 
Nat"urlich k"onnte man das auch mit ein paar Parboxen oder 
Minipages machen, da aber \texttt{firmaddress} ein guter Platz 
f"ur eine Graphik ist und diese mit Sicherheit ein 
Fein-Tuning\footnote{Nicht nur der Text, nein, auch die Worte 
sind zweisprachig :-)} erfordert, habe ich diese etwas 
umst"andlich wirkende L"osung beibehalten.

\begin{verbatim}
%%% Top of every other page
\sbox{\firm}
   {\fontsize{10}{12pt}\selectfont\normalfont
    Mein Name}
\end{verbatim}

Die Folgeseiten brauchen kein so aufwendiges Layout. Wer mehr m"ochte, 
dann sich nat"urlich auch ein anderes Layout definieren, seit 1.5e 
sind entsprechende Muster in der cfg-Datei vorhanden:

\begin{verbatim}
% \if@twoside % Twoside definition
%   \def\ps@headings{%
%       \def\@oddfoot{\hfil Text on the odd page}
% 	  \def\@evenfoot{Text on the even page\hfil}
%       \def\@oddhead{\usebox{\firm}
%          \headfont\hfil\@date\hfil\pagename\ \pnumfont\thepage}%
%       \let\@evenhead\@oddhead}
% \else % Oneside definition
%   \def\ps@headings{%
%       \def\@oddfoot{\hfil Text for one-side layout\hfill}
%       \def\@oddhead{\usebox{\firm}
%          \headfont\hfil\@date\hfil\pagename\ \pnumfont\thepage}}
% \fi
\end{verbatim}

\begin{verbatim}
%%% Return address
%%% one line version
\sbox{\firmreturn}
   {\underline{\fontsize{7}{8pt}\selectfont
    \hskip5mm Mein Name $\cdot$ Meine Stra{\ss}e 1
    $\cdot$ D--12345 Meindorf\hskip5mm}}
%%% two line version
%\sbox{\firmreturn}
%   {\underline{
%      \fontsize{7}{8pt}\fontfamily{\sfdefault}\selectfont
%      \hskip3mm\parbox{65mm}
%      {\makebox[65mm][c]
%	  {Fachschaft -- Maschinenbau $\cdot$
%	   \mbox{\raise .75ex \hbox{c}\kern -.15em /\kern -.125em
%	       \smash{\lower .3ex \hbox{o}}}
%		AStA TU Braunschweig}
%	\makebox[65mm][c]
%	   {Katharinenstra{\ss}e 1 $\cdot$ D--38106 Braunschweig}}
%   \hskip3mm}}
\end{verbatim}

Damit unzustellbare Briefe den Weg zur"uck finden, erscheint im 
Fenster die \texttt{firmreturn}-Adresse. An der zweizeiligen 
Version habe ich ein Wochenende gebastelt bis sie mir gefiel.

\begin{verbatim}
%%% Firstpage footer
%%% The code should be controlled by @banklo
\sbox{\firmfoot}
   {\fontsize{9}{11pt}\selectfont
%%% The following line may look nice in a FAX.CFG :-)
%%% If you have a bad telephone-line try snail-mail!\\
     \if@banklo
       \ifnum\language = \l@german
	  Bankverbindung:
	  \hspace{5mm}
	  \parbox[t]{\rightfield}{\raggedright
	  Undeutsche Landesbank Frankfurt,
	  BLZ~123\,456\,78, Konto~1\,111\,007\\}%
	  \hspace{5mm}
	  \parbox[t]{\rightfield}{\raggedright
	  Die Sparkasse Meindorf,
	  BLZ~007\,123\,10, Konto~08\,15\\}
       \fi
     \fi}
\end{verbatim}

Die erste Seite kann auch einen Brief"|fu"s tragen, nach DIN ist 
das der Ort f"ur Bankverbindungen, Gerichtsstand und "ahnliches. 
Damit man auch Briefe ohne diesen Kram verschicken kann, ohne 
eine neue \texttt{cfg}-Datei zu erzeugen, gibt es die Option 
\texttt{banklow}, die den Schalter \verb+@banklo+ setzt. Hiermit 
k"onnen die Bankverbindungen\footnote{Daher der Name} 
eingeschaltet werden. Diese Einstellung gilt f"ur das gesamte 
Dokument.

\begin{verbatim}
%%% Text displayed below the header on the right side
%%% This should be controlled by @bankhi
%%% This is the only place that will be re-evaluated for
%%% every letter! Everything that changes for different
%%% letters (in the same document) must go either here
%%% or in the ref-line.
\newcommand{\rightbox}
    {\parbox[t]{\rightfield}
    {\fontsize{9}{11pt}\selectfont
     \raggedright
     \ifx\fromname\@empty
     \else
       \ifnum \language = \l@german
	  Sachbearbeiter:
       \fi
       \fromname\\
     \fi
     \ifx\telephonenum\@empty
	 \ifnum \language = \l@german
	   \telephonename: 040 / 44\,17\,77
	 \else
	   \telephonename: 49--40--44\,17\,77
	 \fi
     \else
	 \telephonename: \telephonenum
     \fi\\[1.5ex]
     \ifx\telefaxnum\@empty
	 \ifnum \language = \l@german
	   \telefaxname: 040 / 44\,17\,77
	 \else
	   \telefaxname: 49--40--44\,17\,77
	 \fi
     \else
	 \telefaxname: \telefaxnum
     \fi\\[1.5ex]
     \ifx\@email\@empty
	 e-mail Kein Mehl\\[1.5ex]
     \else
	 e-mail: \@email \\[1.5ex]
     \fi
     \if@bankhi
       \ifnum \language = \l@german
	  Bankverbindung:\\[1ex]
	  Undeutsche Landesbank Frankfurt,
	  BLZ~123\,456\,78, Konto~1\,111\,007\\[.5ex]
	  Die Sparkasse L\"uneburg,
	  BLZ~007\,123\,10, Konto~08\,15 \\[.5ex]
       \fi
     \fi
     %% Finally fixed it!
     %% If you want to have the date here, set the
     %% rightdate option.
     %% (With a big excuse to Markus and many others, who
     %% had to wait that long)
     \if@rightdate
       \vspace{0.5ex}
       \normalsize\@date
     \fi    
      }}
\end{verbatim}

Dies ist der komplexeste Code in der \texttt{cfg}-Datei, das 
Feld rechts neben dem Adressfeld. Es wird f"ur jeden Brief neu 
ausgewertet, darf daher nicht in eine Box. Normalerweise soll 
hier Platz f"ur einen Eingangsstempel sein, aber viele 
Amateurdesigner packen hier lieber die Bankverbindungen hin, mit 
der Option \texttt{bankhigh} besteht die M"oglichkeit beides 
vorzusehen und bei Bedarf umzuschalten.

Die Optionen \texttt{banklow} und \texttt{bankhigh} sind \emph{nicht} 
exclusiv.

\begin{verbatim}
%% The labels defined in akletter.cls are what my printer likes:
%% Onecolumn, 100mm * 50mm, 3mm separation.
%% If you have different lables edit here:

%\renewcommand*{\startlabels}{\labelcount\z@
%   \pagestyle{empty}\let\@texttop\relax
% The margins are calculated from the 1in,1in point,
% thus negative values will occur if the margins are smaller
%   \topmargin -60\p@
%   \headsep \z@
%   \oddsidemargin -35\p@
%   \evensidemargin -35\p@
%   \textheight 10in
%   \@colht\textheight	\@colroom\textheight \vsize\textheight
%   \textwidth 550\p@
% You may have to increase columnsep if you use twocolumn
% This is what letter.cls uses.
%   \columnsep 26\p@
% This does almost nothing since there is an explicit 
% fontsize-command in the text written.
%   \ifcase \@ptsize\relax
%     \normalsize
%   \or
%     \small
%   \or
%     \footnotesize
%   \fi
%   \baselineskip \z@
%   \lineskip \z@
%   \boxmaxdepth \z@
%   \parindent \z@
% This is obvious, isn't it?
%   \twocolumn
%   \relax}

%% This is how the label will look like when printed.
%% Note that specialmail will be written to the .aux-file
%% when the letter is processed. The width used here is
%% what letter.cls uses. 

%\renewcommand*{\mlabel}[2]{%
%  \parbox[b][2in][c]{262\p@}{%
%     \strut\ignorespaces\usebox{\firmreturn}\\
%     \fontsize{12}{14pt}\selectfont
%     \parbox[t][1.2in][c]{3in}{#2}
%   }\par%
%}
\end{verbatim}

Um echte Adressaufkleber zu erzeugen ist noch einiges an Arbeit 
notwendig. Das hier angegebene Layout funktioniert mit meinen 
Aufklebern auf einem Deskjet~500 recht gut, auf dem neuen 
Deskjet~540 ist alles um 20~mm verschoben, daf"ur werden die 
Aufkleber besser eingezogen; so viel zum Thema ``Device 
Independend''.

\begin{verbatim}
\makeatother
\endinput
\end{verbatim}

Das wars.

Die einzelnen Mitarbeiterinnen brauchen dann nur noch folgendes:

\begin{verbatim}
%
\makeatletter
%% This is myletter.cfg
%%
%% You should modify this file to correspond to
%% the site where it is used.
%% It may be a good idea to input this file into the
%% the personal configuration file.
%% Thus only names have to be redefined, the official
%% letterhead is kept in this file and changes for everyone
%% when the masterfile is changed.
%%
%% You may include logos and other graphic-material into
%% your letterhead:
%%
%%\RequirePackage[dvips]{graphics}

%% You may redefine the following variables
%%

%\renewcommand*{\yourrefname}   {Ihr Zeichen:}
%\renewcommand*{\yourmailname}  {Ihr Schreiben vom:}
%\renewcommand*{\myrefname}     {Unser Zeichen:}
%\renewcommand*{\mymailname}    {Unser Schreiben vom:}
%\renewcommand*{\customername}  {Kundennummer }
%\renewcommand*{\invoicename}   {Rechnungsnummer }
%\renewcommand*{\subjectname}   {}  %{Betr.}
%\renewcommand*{\ccname}        {Kopien an:}
%\renewcommand*{\enclname}      {Anlagen:}
%\renewcommand*{\headtoname}    {An}
%\renewcommand*{\datename}      {Datum}
%\renewcommand*{\pagename}      {Seite}
%\renewcommand*{\telephonename} {Telefon}
%\renewcommand*{\telefaxname}   {FAX}

% This may be neccessary for babel:
 
% \let\savedcaptions\captionsgerman
% \def\captionsgerman{\savedcaptions
%    \renewcommand*{\myrefname}	{Mein Zeichen:}
%    }


%%% Top of first page: left
\sbox{\FIRM}
   {\parbox[t]{\leftfield}
   {\bfseries\fontsize{17.28}{22pt}\selectfont%
    Mein Name\hfill}}

%%% Top of every other page
\sbox{\firm}
   {\fontsize{10}{12pt}\selectfont\normalfont
    Mein Name}

%% The normal page-layout for the following page does not contain a 
%% footer. In fact the layout is rather primitive. So if you want to
%% include some fancy text, your corporate logo or whatever, you have
%% to redefine the pagestyle. You may have to adjust the textheight if
%% the footer becomes too large. Of course you can use fancyhdr or
%% scrpage if you need a more powerful tool.

% \if@twoside % Twoside definition
%   \def\ps@headings{%
%       \def\@oddfoot{\hfil Text on the odd page}
% 	  \def\@evenfoot{Text on the even page\hfil}
%       \def\@oddhead{\usebox{\firm}
%          \headfont\hfil\@date\hfil\pagename\ \pnumfont\thepage}%
%       \let\@evenhead\@oddhead}
% \else % Oneside definition
%   \def\ps@headings{%
%       \def\@oddfoot{\hfil Text for one-side layout\hfill}
%       \def\@oddhead{\usebox{\firm}
%          \headfont\hfil\@date\hfil\pagename\ \pnumfont\thepage}}
% \fi

	
%%% Top of first page: right
\sbox{\firmaddress}
% Use either Text:
   {\parbox[t]{\rightfield}{%
    \fontsize{9}{10pt}\selectfont\normalfont
    Meine Stra{\ss}e 1\\ 12345 Meindorf}}
% or graphic:
%{\includegraphics{ourlogo}}

%% and remember to adjust FIRM and firmaddress:

%%% Top of first page: left + right
%%% You may want to adjust the raisebox
\sbox{\FIRMHEAD}
   {\usebox{\FIRM}\raisebox{3mm}{\usebox{\firmaddress}}}
  
%%% Return address
%%% one line version
\sbox{\firmreturn}
   {\underline{\fontsize{7}{8pt}\selectfont
   \hskip5mm Mein Name \textperiodcentered Meine Stra{\ss}e 1
   \textperiodcentered D--12345 Meindorf\hskip5mm}}

%%% two line version
%\sbox{\firmreturn}
%   {\underline{
%      \slshape\fontsize{7}{8pt}\selectfont
%      \hskip3mm\parbox{65mm}
%      {\makebox[65mm][c]
%         {Fachschaft -- Maschinenbau \textperiodcentered
%          \mbox{\raise .75ex \hbox{c}\kern -.15em /\kern -.125em
%              \smash{\lower .3ex \hbox{o}}}
%               AStA TU Braunschweig}
%       \makebox[65mm][c]
%          {Katharinenstra{\ss}e 1 \textperiodcentered D--38106 Braunschweig}}
%   \hskip3mm}}

%%% Firstpage footer
%%% The code should be controlled by @banklo
\sbox{\firmfoot}
   {\fontsize{9}{11pt}\selectfont
%%% The following line may look nice in a FAX.CFG :-)
%%% If you have a bad telephone-line try snail-mail!\\
     \if@banklo
       \ifnum\language = \l@german
	  Bankverbindung:
	  \hspace{5mm}
	  \parbox[t]{\rightfield}{\raggedright
	  Undeutsche Landesbank Frankfurt,
	  BLZ~123\,456\,78, Konto~1\,111\,007\\}%
	  \hspace{5mm}
	  \parbox[t]{\rightfield}{\raggedright
	  Die Sparkasse Meindorf,
	  BLZ~007\,123\,10, Konto~08\,15\\}
       \fi
     \fi}

%%% Text displayed below the header on the right side
%%% This should be controlled by @bankhi
%%% This is the only place that will be re-evaluated for
%%% every letter! Everything that changes for different
%%% letters (in the same document) must go either here
%%% or in the ref-line.
\newcommand{\rightbox}%
    {\parbox[t]{\rightfield}%
    {\fontsize{9}{11pt}\selectfont
     \raggedright
     \ifx\fromname\@empty
     \else
       \ifnum \language = \l@german
          Sachbearbeiter:
       \fi
          \fromname\\
     \fi
     \ifx\telephonenum\@empty
         \ifnum \language = \l@german
           \telephonename: 040 / 44\,17\,77
         \else
           \telephonename: 49--40--44\,17\,77
         \fi
     \else
         \telephonename: \telephonenum
     \fi\\[1.5ex]
     \ifx\telefaxnum\@empty
         \ifnum \language = \l@german
           \telefaxname: 040 / 44\,17\,77
         \else
           \telefaxname: 49--40--44\,17\,77
         \fi
     \else
         \telefaxname: \telefaxnum
     \fi\\[1.5ex]
     \ifx\@email\@empty
         e-mail: Mein-Name@ insert.your.domain.here
     \else
         e-mail: \@email
     \fi\\[1.5ex]
     \if@bankhi
       \ifnum \language = \l@german
	  Bankverbindung:\\[1ex]
	  Undeutsche Landesbank Frankfurt,
	  BLZ~123\,456\,78, Konto~1\,111\,007\\[.5ex]
	  Die Sparkasse L\"uneburg,
	  BLZ~007\,123\,10, Konto~08\,15 \\[.5ex]
       \fi
     \fi


     %% Finally fixed it!
     %% If you want to have the date here, set the
     %% rightdate options.
     %% (With a big excuse to Markus and many others, who
     %% had to wait that long)
     \if@rightdate
       \vspace{0.5ex}
       \normalsize\@date
     \fi
     }}


%% The labels defined in myletter.cls are what my printer likes:
%% Onecolumn, 100mm * 50mm, 3mm separation.
%% If you have different lables edit here:

%\renewcommand*{\startlabels}{\labelcount\z@
%   \pagestyle{empty}
%   \let\@texttop\relax
%   \setlength\parsep {0\p@} 
%% The margins are calculated from the 1in,1in point,
%% thus negative values if the margins schould be smaller
%   \topmargin -60\p@
%   \headsep \z@
%   \oddsidemargin -35\p@
%   \evensidemargin -35\p@
%   \textheight 10in
%   \@colht\textheight  \@colroom\textheight \vsize\textheight
%   \textwidth 550\p@
% You may have to increase columnsep if you use twocolumn labels
% This is what letter.cls uses.
%   \columnsep 26\p@
% This does almost nothing since there is an explicit fontsize-command 
% in the text written.
%   \ifcase \@ptsize\relax
%     \normalsize
%   \or
%     \small
%   \or
%     \footnotesize
%   \fi
%   \baselineskip \z@
%   \lineskip     \z@
%   \boxmaxdepth  \z@
%   \parindent    \z@
% This is obvious, isn't it?
%   \twocolumn
%   \relax}

%% This is how the label printed will look like.
%% Note that specialmail will be written to the .aux-file
%% when the letter is processed. The width used here is
%% what letter.cls uses. 

%\renewcommand*{\mlabel}[2]{%
%  \parbox[b][2in][c]{3in}{%
%     \strut\ignorespaces\usebox{\firmreturn}\\
%     \fontsize{12}{14pt}\selectfont
%     \parbox[t][1.2in][c]{3in}{#2}
%   }\par%
%}

\makeatother
\endinput

%
\email{A.Kielhorn@web.de.de}
\mailbox{}
\name{A. Kielhorn}
\signature{(Axel Kielhorn)}
%
\end{verbatim}

\section{Der erste Brief}

Nicht alle Einstellungen k"onnen in der \texttt{cfg}-Datei 
gemacht werden, sonst w"urde man ja immer den selben Brief 
schreiben. Daher gibt es noch viele Einstellungen im 
eigentlichen Text zu machen. (Nat"urlich k"onnen default-Werte in 
der \texttt{cfg}-Datei gesetzt werden.)

Folgende Befehle m"ussen in der Pr"aambel (vor dem 
\verb+\begin{document}+) gesetzt werden.

\begin{description}
\item [\texttt{usename\{default\}}] w"ahlt die \texttt{cfg}-Datei 
	aus.
\item [\texttt{pagestyle\{headings\}}] Folgeseiten mit Kopfzeile.
\item [\texttt{makelabels}] Adressaufkleber erzeugen
\item [\texttt{selectlanguage\{english\}}] Die verwendete 
	Sprache. Wenn der \texttt{german.sty} verwendet wird, 
	Seitenkopf und -fu"s aber die englischen Namen tragen 
	sollen, kann das \emph{hier} (und nur hier) eingestellt 
	werden. Durch die mehrfache Verwendung von 
	\verb+\selectlanguage{}+ kann ein furchtbares 
	Durcheinander entstehen, daher am besten nur eine Sprache pro 
	Dokument verwenden!
\end{description}

Seit Version 1.5e kann der Benutzer das Aussehen noch feiner 
manipulieren:
\begin{description}
	\item[\texttt{reffont}] Mit \verb+\renewcommand*\reffont{}+ kann das 
	Aussehen der von \texttt{Akletter.cls} erzeugten Texte wie "`Ihr 
	Schreiben vom"' ver"andert werden, default ist \texttt{normalfont}

	\item[\texttt{reftextfont}] Mit \verb+\renewcommand*\reftextfont{}+ 
	kann das Aussehen der Argumente der Referenzzeile ver"andert werden,
	default ist \texttt{normalfont}

	\item[\texttt{headfont}] Mit einem 
	\verb+\renewcommand*\headfont{}+ kann das Aussehen der Kopfzeile 
	(ohne Seitenzahl) ver"andert werden, default ist \texttt{slshape}

	\item[\texttt{pnumfont}] Mit einem 
	\verb+\renewcommand*\pnumfont{}+ kann das Aussehen der Seitenzahl 
	ver"andert werden, default ist \texttt{normalfont}
	
\end{description}


Die folgenden Befehle "andern das Aussehen eines Briefes, sie 
sollten also hinter dem \verb+\begin{letter}+ stehen:

\begin{description}
\item [\texttt{telephone\{0800 / 12 34 56\}}] Hiermit 
	"uberschreibe ich die Telephonnummer der 
	\texttt{cfg}-Datei
\item [\texttt{telefax\{0800 / 12 34 57\}}] Gleiches f"ur die Fax-Nummer
\item [\texttt{email\{A.Kielhorn@tu-bs.de\}}] Steht auch 
	schon in der \texttt{cfg}-Datei
\item [\texttt{mailbox\{ \}}]
\item [\texttt{name\{A. Kielhorn\}}] Sinnvoll, wenn die 
	\texttt{cfg}-Datei f"ur eine Abteilung erstellt wurde, 
	dann steht hier die Sachbearbeiterin. Dieser Befehl 
	setzt auch die \texttt{signature}
\item [\texttt{signature\{(Axel Kielhorn)\}}] Wenn die 
	\texttt{signature} nicht mit dem Namen "ubereinstimmt 
	(i.~A.)
\item [\texttt{specialmail\{Einschreiben\}}]
\item [\texttt{refdatename\{Braunschweig, den\}}] Refdatename wird nur 
benutzt, wenn auch tats"achlich eine Referenzzeile vorhanden ist.
\item [\texttt{yref\{ \}}]
\item [\texttt{ymail\{1.6.1994\}}]
\item [\texttt{myref\{ak/AK\}}]
\item [\texttt{mymail\{1.6.1994\}}]
\item [\texttt{subject\{LaTeX2e Letter-style\}}] Die Betreffzeile
\item [\texttt{invoice\{Rechnungsnummer\}}] noch experimentell
\item [\texttt{customer\{Kundennummer\}}] noch experimetell
\item [\texttt{opening\{Dear friends,\}}] Die Anrede, der 
	einzige Befehl der in jedem Brief vorhanden sein 
	\emph{mu"s}, da er den Briefkopf erzeugt. Dabei kann das 
	Argument leer sein, z.\,B. um Vordrucke zu erzeugen.
\end{description}

Hier sollte dann der Text stehen\dots

Der Abschlu"s des Briefes:
\begin{description}
\item [\texttt{closing\{mfg\}}] Auch dieser Befehl darf nicht 
	fehlen, er beendet den Brief.
\item [\texttt{cc\{comp.text.tex\}}] Eine Kopie f"ur das Archiv
\item [\texttt{encl\{akletter.cfg\}}] Und eine Anlage
\item [\texttt{ps}] Der Brief ist zu Ende! Kein langes 
	Postscriptum mehr.
\end{description}

\section{Serienbriefe}

Seit Version 1.4 kann \texttt{akletter} auch Serienbriefe 
erzeugen. Ich habe hier die entsprechenden Programmteile aus der 
\texttt{scrlettr.cls} des Koma-Script-Paketes kopiert. Daher 
k"onnen die \texttt{adr}-Dateien zwischen beiden Systemen 
ausgetauscht werden. Programmierbeispiele hierzu in der 
Anleitung zu Koma-Script.

\section{Optionen}

\texttt{akletter.cls} kennt eine Reihe von Optionen die nicht 
unerw"ahnt bleiben sollen:

\begin{description}
\item [a4paper] W"ahlt das Papierformat aus. Diese Option kann auch
weggelassen werden, da sie default ist.

\item [a4offset] Erzeugt ein um 24 Punkte (ca. 8 mm) nach rechts
verschobenes Layout, das um 24 Punkte schmaler ist. Gedacht ist dieses
Layout f"ur Bewerbungsmappen, bei denen der linke Rand durch eine
Klemmleiste verdeckt ist.

\item[letterpaper] W"ahlt das amerikanische letter Format. Der Brief
wird etwas breiter, daf"ur aber k"urzer. Ob das Layout mit
amerikanischen Fensterumschl"agen zurechtkommt, kann ich nicht
"uberpr"ufen.

\item [banklow] und

\item [bankhigh] sind zwei Schalter zur freien Verf"ugung. Im Original
dienen sie dazu, die Bankverbindungen in Fu"s der Seite (wo sie
hingeh"oren) oder im Feld rechts neben der Anschrift (wo viele
Designer sie gerne hinpacken) ein- und auszuschalten. Beide Schalter
k"onnen unabh"angig voneinander benutzt werden.

\item [foldmarks] schaltet die Falzmarken ein. Ob sie auch wirklich zu
sehen sind, liegt dann am Drucker. Der oben genannte DeskWriter 540
l"a"st einen zu breiten linken Rand, so da"s die Falzmarken nicht
gedruckt werden k"onnen.

\item [refdate] Aktiviert das Datum in der Referenzzeile


\item [subjectdate] Das Datum erscheint in der Subjectzeile (default)

\item [rightdate] Wenn im rightfield entsprechneder Code enthalten
ist, wird das Datum dort ausgegeben. Ansonsten wird kein Datum
gedruckt.
\item [reverseaddr] Vertauscht Anschriftenfeld und Rightfield. F"ur
	ausl"andische Fensterumschl"age.
\end{description}

Da ich ein gro"ser Anh"anger von Normen bin, habe ich jetzt auch das 
Datumsformat nach ISO 8601 definiert. F"ur deutsche Briefe ist es 
relevant, da die DIN 5008 von 1996 (Maschinenschreiben) dieses Format 
empfiehlt. Durch den Befehl \verb+\dateiso+ in der Pr"aamble wird das 
Datum als JJJJ-MM-TT ges"atzt, also z.B. 2000-05-28 f"ur den 28. Mai 
2000. Mit jeder Sprachumschaltung wird das Datum wieder auf den 
Standardwert f"ur diese Sprache zur"uckges"atzt. 

\section{Danksagungen}

Bedanken m"ochte ich mich bei Donald E. Knuth und Leslie Lamport, 
die mit \TeX\ und \LaTeX\ ein sehr brauchbares 
Textverarbeitungssystem geschaffen haben, bei Stefan Lindner f"ur 
die Portierung auf Atari, bei Andrew Trevorrow f"ur die Portierung 
auf MacIntosh und beim \LaTeX3-Team f"ur die Weiterentwicklung von \LaTeX.

Mein spezieller Dank gilt den Betatestern, die sich mit fr"uhen 
Versionen vom \texttt{akletter.cls} rumgeschlagen haben, hier 
ganz besonders Helmut Lichtenberg, dem ich viele Anregungen 
verdanke und Jan Braun, der den Ansto"s zu dieser Anleitung 
gegeben hat und sie Korrektur la"s.

\end{document}

