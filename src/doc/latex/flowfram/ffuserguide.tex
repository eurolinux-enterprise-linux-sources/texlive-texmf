\documentclass[a4paper,twoside]{book}


\usepackage{times}
\usepackage{courier}
\usepackage{helvet}
\usepackage{makeidx}
\usepackage[landscape,margin=1in,top=1in,bottom=1in]{geometry}
\usepackage{color}
\usepackage[colorlinks,
            plainpages=false,
            linkcolor=black,
            bookmarksopen,
            pdfauthor={Nicola Talbot},
            pdftitle={Creating Flow Frames for Posters, Brochures or Magazines using flowfram.sty},
            pdfkeywords={LaTeX;text frames;posters;brochures}]{hyperref}
\usepackage[style=altlist,toc,number=none]{glossary}[2005/07/15]
\usepackage{pgf}
\usepackage[norotate,ttbnotitle,ttbnum]{flowfram}

% Define some commands for consistency and indexing
\newcommand{\sty}[1]{\textsf{#1}\index{#1@\textsf{#1} package}}
\newcommand{\env}[1]{\textsf{#1}\index{#1@\textsf{#1} environment}}
\newcommand{\cmdname}[1]{\texttt{\symbol{92}#1}\index{#1@\texttt{\symbol{92}#1}}}
\newcommand{\meta}[1]{\textnormal{\textless\textit{#1}\textgreater}}
\newcommand{\appname}[1]{\texttt{#1}\index{#1@\texttt{#1}}}
\newcommand{\pkgopt}[1]{\textsf{#1}\index{#1@\textsf{#1} option}}
\newcommand{\key}[1]{#1\index{frame settings!#1}}

% Glossary Stuff

\storeglosentry{typeblock}{name=typeblock,description={The 
area of the page where the main body of the text goes.
The width and height of this area are given by 
\protect\cmdname{textwidth} and \protect\cmdname{textheight}}}

\storeglosentry{flow}{name=flow frame,description={The frames in 
a document such that the contents of the \protect\env{document}
environment flow from one frame to the next in the order 
that they were defined. There must be at least one flow frame on
every page.}}

\storeglosentry{static}{name=static frame,description={Frames
in which text is fixed in place.  The contents are fixed until
explicitly changed.}}

\storeglosentry{dynamic}{name=dynamic frame,description={Frames
in which text is fixed in place, but the contents are re-typeset
after each page.}}

\storeglosentry{frame}{name=frame,description={A rectangular
area of the page in which text can be placed (not to be
confused with a frame making command). There are three types:
flow, static and dynamic.}}

\storeglosentry{fcmd}{name=frame making command,description={A
\LaTeX\ command which places some kind of border around its
argument. For example: \protect\cmdname{fbox}.}}

\storeglosentry{pglist}{name=page list,description={A list of
pages. This can either be a single keyword: \texttt{all},
\texttt{odd}, \texttt{even} or \texttt{none}, or it can
be a comma-separated list of individual page numbers or
page ranges. For example: \texttt{\textless3,5,7-11,\textgreater15}
indicates pages 1,2,5,7,8,9,10,11 and all pages after page 15.
Note that these numbers refer to the actual value of the
page counter, not the absolute physical page number.}}

\storeglosentry{pgrange}{name=page range,description={Page
ranges can be closed, e.g.\ \texttt{5-10}, or open, e.g.
\texttt{\textless7} or \texttt{\textgreater9}.}}

\storeglosentry{bbox}{name=bounding box,description={The smallest
possible rectangle that completely encompasses the object.}}

\newacronym{IDN}{identification number}{description={A unique
number assigned to each frame, which you can use to identify
the frame when modifying its appearance. Example: if you
have defined 3 flow frames, 2 static frames and 1 dynamic
frame, the flow frames will have IDNs 1, 2 and 3, the static
frames will have IDNs 1 and 2, and the dynamic frame will
have IDN 1.}}

\newacronym{IDL}{identification label}{description={A unique
label which can be assigned to a frame, enabling you to
refer to the frame by label instead of by its IDN.}}

\makeglossary
\makeindex

% Page layout stuff

% set up some background frames to liven up the title page
\newlength{\leftwidth}
\newlength{\rightwidth}

\computeleftedgeodd{\leftwidth}
\setlength{\leftwidth}{-\leftwidth}
\addtolength{\leftwidth}{0.4\textwidth}
\setlength{\rightwidth}{\paperwidth}
\addtolength{\rightwidth}{-\leftwidth}
% only defined on page 1 unfortunately the document
% has more than one page 1, so will need to change the settings after the title page
\vtwotone[1]{\leftwidth}{magenta}{backleft}{\rightwidth}{[cmyk]{0,0.48,0,0}}{backright}

% This is for the back cover.
\vtwotone[none]{\rightwidth}{[cmyk]{0,0.48,0,0}}{lastbackright}{\leftwidth}{magenta}{lastbackleft}

\vtwotonetop[odd]{1cm}{\leftwidth}{magenta}{oddtopleft}{\rightwidth}{[cmyk]{0,0.48,0,0}}{oddtopright}
\vtwotonetop[even]{1cm}{\rightwidth}{[cmyk]{0,0.48,0,0}}{eventopleft}{\leftwidth}{magenta}{eventopright}

% Set the margin width
\setlength{\marginparwidth}{2cm}

% now set up main document frames. Each page has a dynamic
% frame for the chapter heading, and a flow frame for the
% text.
\newflowframe{0.6\textwidth}{\textheight}{0pt}{0pt}[main]
\newdynamicframe{0.38\textwidth}{\textheight}{0.62\textwidth}{0pt}[chaphead]

% swap them round on even pages
\setflowframe*{main}{evenx=0.4\textwidth}
\setdynamicframe*{chaphead}{evenx=0pt,clear}

% set the margins to appear on the spine side of the page
\setflowframe*{main}{margin=inner}

% put chapter headings in dynamic frame with IDL chaphead
\dfchaphead*{chaphead}

% append chapter minitocs to same dynamic frame.
\appenddfminitoc*{chaphead}

% change the style of the chapter headings

% numbered chapters:
\renewcommand{\DFchapterstyle}[1]{%
\raggedright\sffamily\bfseries\Huge\color{blue}\thechapter. #1\par
}

% unnumbered chapters:
\renewcommand{\DFschapterstyle}[1]{\raggedright\sffamily\bfseries\Huge\color{blue} #1\par
}

% Make thumb tabs (specify each tab to be 0.75in high)

\makethumbtabs{0.75in}
\enableminitoc

% Thumbtabs are grey by default which looks a bit boring,
% so change the colours
\setthumbtab{1}{backcolor=[rgb]{0.15,0.15,1}}
\setthumbtab{2}{backcolor=[rgb]{0.2,0.2,1}}
\setthumbtab{3}{backcolor=[rgb]{0.25,0.25,1}}
\setthumbtab{4}{backcolor=[rgb]{0.3,0.3,1}}
\setthumbtab{5}{backcolor=[rgb]{0.35,0.35,1}}
\setthumbtab{6}{backcolor=[rgb]{0.4,0.4,1}}
\setthumbtab{7}{backcolor=[rgb]{0.45,0.45,1}}

% change the text style on the thumbtabs
\newcommand{\thumbtabstyle}[1]{\textsc{\sffamily #1}}
\setthumbtab{all}{style=thumbtabstyle,textcolor=white}

% set default page style

\pagestyle{plain}

% Put headers and footers in dynamic frames
\makedfheaderfooter

\newlength{\xoffset}
\computerightedgeodd{\xoffset}
\addtolength{\xoffset}{-2cm}
\newlength{\yoffset}
\computebottomedge{\yoffset}

\newcommand{\footstyle}[1]{\bfseries\LARGE #1}

% pages is initially set to none, as I don't want the footer to
% appear on the title page.

\setdynamicframe*{footer}{oddx=\xoffset,y=\yoffset,width=2cm,height=2cm,
backcolor=blue,textcolor=white,style=footstyle,pages=none}

% now work out the x offset for the even pages

\computeleftedgeeven{\xoffset}
\setdynamicframe*{footer}{evenx=\xoffset}

% Now create some frames for the index, and modify
% theindex environment

% These are for the odd pages

\twocolumninarea[none]{0.6\textwidth}{\textheight}{0pt}{0pt}
% Give a label to the last 2 flow frames:
\newcounter{N}
\newcounter{I}
\setcounter{N}{\value{maxflow}}
\addtocounter{N}{-2}
\whiledo{\value{N}<\value{maxflow}}{%
\stepcounter{N}\stepcounter{I}
\setflowframe{\value{N}}{label=oddcol\theI}}

% These are for the even pages

\twocolumninarea[none]{0.6\textwidth}{\textheight}{0.4\textwidth}{0pt}
% Give a label to the last 2 flow frames:

\setcounter{I}{0}
\setcounter{N}{\value{maxflow}}
\addtocounter{N}{-2}
\whiledo{\value{N}<\value{maxflow}}{%
\stepcounter{N}\stepcounter{I}
\setflowframe{\value{N}}{label=evencol\theI}}

\makeatletter
\renewenvironment{theindex}{%
\setflowframe*{oddcol1,oddcol2}{pages=odd}
\setflowframe*{evencol1,evencol2}{pages=even}
\clearpage
\mbox{}\framebreak
\setflowframe*{main}{pages=none}
\setdynamiccontents*{chaphead}{\DFschapterstyle{\indexname}}
\addcontentsline{toc}{chapter}{\indexname}
\setlength{\parindent}{0pt}%
\setlength{\parskip}{0pt plus .3pt}%
\let\item\@idxitem}{%
\setflowframe*{main}{pages=all}%
\setflowframe{>1}{pages=none}%
\finishthispage}
\makeatother

\begin{document}
% swap frames around for title page
\ffswapoddeven*{main}
\dfswapoddeven*{chaphead}

\title{Creating Flow Frames for Posters, Brochures or 
Magazines using flowfram.sty v 1.02}
\author{Nicola L. C. Talbot}
\date{24th November 2005}
\pagenumbering{alph}
\maketitle

%suppress the background
\setstaticframe*{backleft}{pages=none}
\setstaticframe*{backright}{pages=none}

\noindent
Dr Nicola Talbot\\
School of Computing Sciences\\
University of East Anglia\\
Norwich. Norfolk. NR4 7TJ.\\
United Kingdom\\
\url{http://theoval.cmp.uea.ac.uk/~nlct/}

\frontmatter
% swap frames back again
\ffswapoddeven*{main}
\dfswapoddeven*{chaphead}

\tableofcontents
\thumbtabindex

%\tocandthumbtabindex % not enough room for this

% make the footers appear from this point on
\setdynamicframe*{footer}{pages=all}

\clearpage

\mainmatter
\chapter{Introduction}
\enablethumbtabs
\pagenumbering{arabic}

The \sty{flowfram} package is a \LaTeXe\ package designed to 
enable you to create text \gls{frame}s in a document such that 
the contents of the \env{document} environment flow from one 
\gls{frame} to the next in the order that they were defined.  
This is useful for creating posters
or magazines or any other form of document that does not 
conform to the standard one or two column layout.

The \sty{flowfram} package provides three types of \gls{frame}:
\gls{flow}s, \gls{static}s and \gls{dynamic}s, all rectangular
in shape with dimensions and positions specified by the 
user\footnote{Can I have arbitrary shaped frames? See 
\autoref{itm:parshape} on page~\pageref{itm:parshape}}.
The main contents of the document environment flow from
one \gls{flow} to the next in the order of definition,
whereas the contents of the static and dynamic frames
are set explicitly using commands described in 
\autoref{sec:modattr}. Note that
unless otherwise stated, all co-ordinates are relative to the
bottom left hand corner of the \gls{typeblock}. If you have
a two-sided document, the absolute position of the \gls{typeblock}
may vary depending on the values of \cmdname{oddsidemargin}
and \cmdname{evensidemargin}, and all the \gls{frame}s will shift
accordingly unless otherwise indicated.

Since floats (such as figures and tables) can only go in 
\gls{flow}s, this package provides
the additional environments:
\env{staticfigure} and 
\env{statictable} which can be used in \gls{static}s
and \gls{dynamic}s. Unlike their \env{figure} and
\env{table} counterparts, they are fixed in place, and
so do not take an optional placement specifier. The 
\cmdname{caption} and \cmdname{label} commands can 
be used within \env{staticfigure} and \env{statictable} as
usual, but it is recommended that you \cmdname{protect} the
\cmdname{label} command, otherwise the label may end up
being multiply defined.

The standard \env{figure} and \env{table} commands will 
behave as usual in the \gls{flow}s, but their starred versions,
\env{figure*} and \env{table*} behave no differently
from \env{figure} and \env{table}\footnote{This is because
of the arbitrary layout of the flow frames.}.

This package has only been tested with a limited number of
class files and packages. Since it modifies the output routine,
it is likely to conflict with any other package which also
does this.  The \sty{flowfram} package works fine with 
\appname{pdflatex}, \appname{latex} and \appname{dvips}, 
however \appname{ps2pdf} and \appname{dvipdf} both seem to 
cause a slight left shift in the page which may result in a 
vertical white band on the right side of the page. 
This problem does not seem to occur with Acrobat Distiller.

You should load \sty{flowfram} \emph{after} \sty{hyperref}
and any colour package (e.g.\ \sty{color}).

\section{Draft Option}

The \sty{flowfram} package has the package option \texttt{draft}
which will draw the \gls{bbox}es for
each \gls{frame} defined.  At the bottom right of each
\gls{bbox} (except for the \gls{bbox} denoting the 
\gls{typeblock}), a marker will be shown in the form:
[\meta{T}:\meta{idn};\meta{idl}], where \meta{T} is a single
letter denoting the \gls{frame} type, \meta{idn} is the \IDN\
for the \gls{frame} and \meta{idl} is the \IDL\ for that
\gls{frame}. Values of \meta{T} are: \texttt{F} (\gls{flow}),
\texttt{S} (\gls{static}) or \texttt{D} (\gls{dynamic}).
Markers of the form: [M:\meta{idn}] indicate that the
\gls{bbox} is the area taken up by the margin for \gls{flow}
with \IDN\ \meta{idn}. Note that even if a \gls{frame}
has been rotated, the \gls{bbox} will not be rotated.

If you want to show or hide specific types of bounding
boxes, you can use one of the following commands:
\begin{itemize}
\item 
\cmdname{showtypeblocktrue} Display the \gls{bbox}
for the \gls{typeblock}.

\item
\cmdname{showtypeblockfalse} Do not display the \gls{bbox}
for the \gls{typeblock}.

\item
\cmdname{showmarginstrue} Display the \gls{bbox}
for the margins.

\item
\cmdname{showmarginsfalse} Do not display the \gls{bbox}
for the margins.

\item
\cmdname{showframebboxtrue} Display the \gls{bbox}
for the \gls{frame}s.

\item
\cmdname{showframebboxfalse} Do not display the \gls{bbox}
for the \gls{frame}s.

\end{itemize}

You can see the layout for the current page (irrespective of
whether or not the \texttt{draft} option has been set) using
the command:
\cmdname{flowframeshowlayout}

The \sty{flowfram} package also has the options \pkgopt{nocolor}
and \pkgopt{norotate} for previewers that can not process
colour or rotating specials. (Otherwise you may end up with
large black rectangles obscuring your text, instead of
the pale background colour you were hoping for.)

\section{Frame Stacking Order}
\label{sec:stacking}

The material on each page is placed in the following order:
\begin{enumerate}
\item Each \gls{static} defined for that page in ascending
order of \IDN.

\item Each \gls{flow} defined for that page in ascending
order of \IDN.

\item Each \gls{dynamic} defined for that page in ascending
order of \IDN.

\item \useGlosentry{bbox}{Bounding boxes} if the \texttt{draft}
package option has been used.
\end{enumerate}

This ordering can be used to determine if you want something
to overlay or underlay everything else on the page.
Note that the \gls{frame}s do not interact with each other. If
you have two or more overlapping frames, the text in each 
\gls{frame} will not attempt to wrap around the other 
frames, but will simply overwrite 
them.\footnote{Can I have arbitrary
shaped frames? See \autoref{itm:parshape} on 
page~\pageref{itm:parshape}.}

\chapter{Defining New Frames}

\section{Flow Frames}

The \gls{flow} is the principle type of \gls{frame}.
The text of the \env{document} environment will flow from 
one \gls{frame} to the next in order of definition. Each 
\gls{flow} has an associated width, height, 
position on the page, and optionally a border. To define
a new \gls{flow} use:\\[10pt]
\cmdname{newflowframe}\verb/[/\meta{page list}\verb+]{+%
\meta{width}\verb/}{/%
\meta{height}\verb/}{/%
\meta{x}\verb/}{/%
\meta{y}\verb/}[/%
\meta{label}\verb/]/\\[10pt]
where \meta{width} is the width of the \gls{frame}, \meta{height}
is the height of the \gls{frame}, (\meta{x},\meta{y}) is the
position of the bottom left hand corner of the \gls{frame}
relative to the bottom left hand corner of the \gls{typeblock}.
The first optional argument, \meta{page list}, indicates the 
list of pages for which this \gls{frame} is defined. 

A \gls{pglist} can either be specified by the keywords: 
\texttt{all}, \texttt{odd}, \texttt{even} or \texttt{none}, or 
by a comma-separated list of either individual page numbers or 
\useGlosentry{pgrange}{page ranges}. If \meta{page list} is
omitted, \texttt{all} is assumed.  
A \gls{pgrange} can be a closed
range (e.g.\ \verb+2-8+) or an open range (e.g.\ 
\verb+<10+ or \verb/>5/). For example: \verb'<3,5,7-11,>15'
indicates pages 1, 2, 5, 7, 8, 9, 10, 11 and all pages 
greater than page 15. These page numbers refer to the value of
the page counter\footnote{why can't I use the page number format?
See item~\ref{itm:whynot} on page~\pageref{itm:whynot}}, 
so if you have a page~i and a page~1, they
will both have the same layout (unless you change the
page range setting somewhere between the two pages).

Each \gls{frame} has its own unique \IDN, 
corresponding to the order in which it was defined. So the first 
\gls{flow} to be defined has the \IDN~1, 
the second has \IDN~2, and so on. This number can then
be used to identify the \gls{frame} when you want to modify its
settings. Alternatively, you can assign a unique \IDL\ to the 
\gls{frame} using the final optional argument \meta{label}.

By default, the \gls{flow} will not have a border, but the 
starred form, \cmdname{newflowframe*}, 
will place a plain border around the \gls{flow}.

Note that if the document continues beyond the last
defined \gls{flow} (for example, the \gls{flow}s have only
been defined on pages~1 to~10, but the document contains 11 
pages) then a single \gls{flow} will be defined, 
emulating one column mode for all subsequent pages.

In this document, I have used the command
\begin{verbatim}
\newflowframe{0.6\textwidth}{\textheight}{0pt}{0pt}[main]
\end{verbatim}
to define the main \gls{flow}\footnote{the
position for the even pages is set using \cmdname{setflowframe}
defined in \autoref{sec:modattr}} (i.e.\ this one.)

\subsection{Prematurely Ending a Flow Frame}
\label{sec:framebreak}

You can force text to move immediately to the next defined
\gls{flow} using one of the commands: \cmdname{newpage},
\cmdname{clearpage}, \cmdname{pagebreak} or \cmdname{framebreak}.
The first three work in an analogous way to the way they
work in standard two column mode. The latter, 
\cmdname{framebreak}, is required
when a paragraph spans two \useGlosentry{flow}{flow frames}
of different widths, as \TeX's output routine does not 
adjust to the new value of \cmdname{hsize} until the last 
paragraph of the previous \gls{frame} has ended. As a 
result, the end of the paragraph at the beginning of the new
\gls{flow} retains the width of the previous \gls{flow}.

If you want to start a new page, rather than simply move to the 
next \gls{frame}, use the command \cmdname{finishthispage}, 
or for two-sided documents, to start on the next odd page
do \cmdname{cleardoublepage}.

\section{Static Frames}

A \gls{static} is a rectangular area in which text neither
flows into, nor flows out of\footnote{By ``neither flows into nor 
flows out of'' I mean you have to explicitly set the contents of 
this frame. Note that it may appear to contain text if another 
frame overlaps it, but this text belongs to the other frame.}.  
The contents must be set explicitly, and once set, the contents 
of the \gls{static} will remain the same on each page until it is 
explicitly changed.  Thus, a \gls{static} can be used, for 
example, to make a company logo appear in the same place on every 
page.

A new \gls{static} is defined using the command:\\[10pt]
\cmdname{newstaticframe}\verb/[/\meta{page list}\verb+]{+%
\meta{width}\verb/}{/%
\meta{height}\verb/}{/%
\meta{x}\verb/}{/%
\meta{y}\verb/}[/%
\meta{label}\verb/]/\\[10pt]
where, as with \cmdname{newflowframe}, \meta{width} is the width of 
the \gls{static}, \meta{height} is the height of the \gls{static},
(\meta{x},\meta{y}) is the position of the bottom left hand 
corner of the \gls{static} relative to the bottom left hand 
corner of the \gls{typeblock}. The first optional argument,
\meta{page list}, indicates the \gls{pglist} for which this
\gls{static} should appear, and the final optional argument,
\meta{label} is a unique textual \IDL\ which you can use to
identify this \gls{static}.  If no label is specified, you
can refer to this \gls{static} by its unique \IDN.
The first \gls{static} defined has \IDN~1, the second
has \IDN~2, and so on.

As with \cmdname{newflowframe}, there is a starred version 
\cmdname{newstaticframe*} which will place a border around that 
\gls{static}.

To set the contents of a particular \gls{static}, you can
either use the \env{staticcontents} environment:\\[10pt]
\verb/\begin{staticcontents}{/\meta{IDN}\verb+}+\\
\meta{contents}\\
\verb/\end{staticcontents}/\\[10pt]
where \meta{IDN} is the unique \IDN\ associated with
that \gls{static} and \meta{contents} is the contents of the
\gls{static}, or you can use the 
command:\\[10pt]
\cmdname{setstaticcontents}\verb/{/\meta{IDN}\verb+}{+\meta{contents}\verb!}!\\[10pt]
which will do the same thing.

There are starred versions available for both the environment
and the command to enable you to identify the \gls{static}
by its associated \IDL\ rather than its \IDN
:\\[10pt]
\verb/\begin{staticcontents*}{/\meta{IDL}\verb+}+\\
\meta{contents}\\
\verb/\end{staticcontents*}/\\[10pt]
or the equivalent:\\[10pt]
\cmdname{setstaticcontents*}\verb/{/\meta{IDL}\verb+}{+\meta{contents}\verb!}!

\section{Dynamic Frames}

A \gls{dynamic} is similar to a \gls{static}, but its contents
are re-typeset on each page. (A \gls{static} stores its 
contents in a savebox, whereas a \gls{dynamic} stores its
contents in a macro).

To create a new \gls{dynamic}, use the command:\\[10pt]
\cmdname{newdynamicframe}\verb/[/\meta{page list}\verb+]{+%
\meta{width}\verb/}{/%
\meta{height}\verb/}{/%
\meta{x}\verb/}{/%
\meta{y}\verb/}[/%
\meta{label}\verb/]/\\[10pt]
The parameters are exactly the same as for \cmdname{newflowframe}
and \cmdname{newstaticframe}.
Again, each \gls{dynamic} has an associated unique \IDN,
starting from 1 for the first \gls{dynamic} defined, and
a unique \IDL\ can also be set using the final optional
argument \meta{label}.

As with the other \gls{frame} types, there is also a starred 
version
\cmdname{newdynamicframe*}
which will place a plain border around the \gls{dynamic}.
For example, in this document I have used the command
\begin{verbatim}
\newdynamicframe{0.38\textwidth}{\textheight}{0.62\textwidth}{0pt}[chaphead]
\end{verbatim}
which has created the \gls{frame} on the 
right on odd pages, and on the left on even pages (the
position for the even pages is set using \cmdname{setdynamicframe}
defined in \autoref{sec:modattr}).

The contents of a \gls{dynamic} are set using the command:\\[10pt]
\cmdname{setdynamiccontents}\verb/{/\meta{id}\verb+}{+\meta{contents}\verb'}'\\[10pt]
where \meta{id} is the unique \IDN\ associated with that
\gls{dynamic}, and \meta{contents} is the contents of the
\gls{dynamic}. Alternatively, if you have assigned an \IDL, 
\meta{label}, to the \gls{dynamic}, you can use the starred 
version:\\[10pt]
\cmdname{setdynamiccontents*}\verb/{/\meta{label}\verb+}{+\meta{contents}\verb'}'\\[10pt]
There is no environment equivalent.

You can additionally append text to a \gls{dynamic} using
either:\\[10pt]
\cmdname{appenddynamiccontents}\verb/{/\meta{id}\verb+}{+\meta{contents}\verb'}'\\[10pt]
or:\\[10pt]
\cmdname{appenddynamiccontents*}\verb/{/\meta{label}\verb+}{+\meta{contents}\verb'}'\\[10pt]

You can make the chapter titles (if \cmdname{chapter} is defined)
appear in a dynamic frame using the command
\cmdname{dfchaphead}\verb"{"\meta{IDN}\verb"}"
where \meta{IDN} is the \IDN\ of the dynamic frame. There is
also a starred version of this command if you want to
use the \IDL\ instead of the \IDN. For example, in this 
document, I used the command:
\begin{verbatim}
\dfchaphead*{chaphead}
\end{verbatim}

The headers and footers can be turned into \gls{dynamic}s
using the command \cmdname{makedfheaderfooter}. This will
create two \gls{dynamic}s with \IDL{}s \texttt{header} and
\texttt{footer}.  The page style will be used as usual, but
you can then move or resize the header and footer using
\cmdname{setdynamicframe} (described next).

\chapter{Modifying Frame Attributes}
\label{sec:modattr}

Once you have defined the \useGlosentry{flow}{flow frames},
\useGlosentry{static}{static frames} and 
\useGlosentry{dynamic}{dynamic frames}, their attributes
can be changed. The three types of \gls{frame} mostly have the 
same set of attributes, but some are specific to a certain type.
\\[10pt]
\useGlosentry{flow}{Flow frame} attributes are modified using 
either the command:\\[10pt]
\cmdname{setflowframe}\verb/{/\meta{idn list}\verb+}{+\meta{key-val list}\verb!}!\\[10pt]
or the starred version:\\[10pt]
\cmdname{setflowframe*}\verb/{/\meta{label list}\verb+}{+\meta{key-val list}\verb!}!\\[10pt]
or the attributes for all \useGlosentry{flow}{flow frames} can be
set using:\\[10pt]
\cmdname{setallflowframes}\verb+{+\meta{key-val list}\verb!}!
\\[10pt]
\useGlosentry{static}{Static frame} attributes are modified 
using either the command:\\[10pt]
\cmdname{setstaticframe}\verb/{/\meta{idn list}\verb+}{+\meta{key-val list}\verb!}!\\[10pt]
or the starred version:\\[10pt]
\cmdname{setstaticframe*}\verb/{/\meta{label list}\verb+}{+\meta{key-val list}\verb!}!\\[10pt]
or the attributes for all \useGlosentry{static}{static frames} 
can be set using:\\[10pt]
\cmdname{setallstaticframes}\verb+{+\meta{key-val list}\verb!}!
\\[10pt]
\useGlosentry{dynamic}{Dynamic frame} attributes are modified 
using either the command:\\[10pt]
\cmdname{setdynamicframe}\verb/{/\meta{idn list}\verb+}{+\meta{key-val list}\verb!}!\\[10pt]
or the starred version:\\[10pt]
\cmdname{setdynamicframe*}\verb/{/\meta{label list}\verb+}{+\meta{key-val list}\verb!}!\\[10pt]
or the attributes for all \useGlosentry{dynamic}{dynamic frames} 
can be set using:\\[10pt]
\cmdname{setalldynamicframes}\verb+{+\meta{key-val list}\verb!}!
\\[10pt]
In each of the above, \meta{idn list} can either be one of the
keywords: \texttt{all}, \texttt{odd} or \texttt{even} (indicating
all \useGlosentry{frame}{frames} of that type, 
\useGlosentry{frame}{frames} of that type whose \IDN\
is odd or \useGlosentry{frame}{frames} of that type whose \IDN\ 
is even), or it can be a comma-separated list of ID numbers, or 
\IDN\ ranges.

For the starred versions, \meta{label list} should be
a comma-separated list of \IDL{}s. Note that you can 
not use the above keywords, or have ranges with the 
starred versions.

The \meta{key-val list} argument must be a comma-separated
list of \meta{key}=\meta{value} pairs, indicating which
attributes to modify. The available values are as follows:

\begin{description}
\item[\key{width}=\meta{length}]  The width of the \gls{frame}.

\item[\key{height}=\meta{length}] The height of the \gls{frame}.

\item[\key{x}=\meta{length}] The x-coordinate of the \gls{frame}
for all pages on which it is defined.

\item[\key{y}=\meta{length}] The y-coordinate of the \gls{frame}
for all pages on which it is defined.

\item[\key{evenx}=\meta{length}] The x-coordinate of the 
\gls{frame} for all even pages on which it is defined, but only if
the document is a two-sided document.

For example, in this document, I have used the commands
\begin{verbatim}
\setflowframe*{main}{evenx=0.4\textwidth}
\setdynamicframe*{chaphead}{evenx=0pt}
\end{verbatim}
to switch the positions of the \gls{flow} and \gls{dynamic}
containing the document text and chapter headings, respectively,
on even pages.

You can swap the odd and even values using the commands:
\cmdname{ffswapoddeven}\verb"{"\meta{IDN}\verb"}" (for 
\gls{flow}s) \cmdname{sfswapoddeven}\verb"{"\meta{IDN}\verb"}"
(for \gls{static}s) or 
\cmdname{dfswapoddeven}\verb"{"\meta{IDN}\verb"}". These
commands all have starred versions which take the frame's
\IDL\ instead of its \IDN.

\item[\key{eveny}=\meta{length}] The y-coordinate of the 
\gls{frame} for all even pages on which it is defined, but only if
the document is a two-sided document.

\item[\key{oddx}=\meta{length}] The x-coordinate of the 
\gls{frame} for all odd pages on which it is defined, if the
document is two-sided.

\item[\key{oddy}=\meta{length}] The y-coordinate of the 
\gls{frame} for all odd pages on which it is defined,
if the document is two-sided.

\item[\key{label}=\meta{text}] Assign an IDL to the \gls{frame}.
(If you do not specify a label when you first define a \gls{frame}
it will be given a label identical to its \IDN.) This key is
provided to allow the user to label frames that have been 
generated by certain predefined layout commands described
in \autoref{sec:layouts}.

\item[\key{border}=\meta{style}] The style of the border around the 
\gls{frame}, this can take the values: \texttt{none} (no border),
\texttt{plain} (plain border) or the name of a \LaTeX\ 
\gls{fcmd} without the preceding backslash. (I admit
the notation is a little confusing, a \gls{fcmd} 
is a command that places some kind of border around its 
argument, such as \cmdname{fbox}, or if you are using the
\sty{fancybox} package: \cmdname{doublebox}, \cmdname{ovalbox},
\cmdname{Ovalbox} and \cmdname{shadowbox}.) 
The value \texttt{fbox} is equivalent to \texttt{plain}.

For example, to make the first \gls{static} have an oval border:
\begin{verbatim}
\setstaticframe{1}{border=ovalbox}
\end{verbatim}
Or you can define your own border:
\begin{verbatim}
\newcommand{\greenyellowbox}[1]{\fcolorbox{green}{yellow}{#1}}
\setstaticframe{1}{border=greenyellowbox}
\end{verbatim}

\item[\key{offset}=\meta{offset}] The border offset, if it is a 
user-defined border.  This is the distance from the outer
edge of the left hand border to the left edge of the
\gls{bbox} of the text inside the border.  The \sty{flowfram}
package is able to compute the border for the following
known \gls{fcmd}s: \cmdname{fbox}, \cmdname{ovalbox},
\cmdname{Ovalbox}, \cmdname{doublebox} and \cmdname{shadowbox}. 
For all other borders, the offset is assumed to be
$-$\cmdname{flowframesep}$-$\cmdname{flowframerule}.
If you define your own \useGlosentry{fcmd}{frame} 
\useGlosentry{fcmd}{making} \useGlosentry{fcmd}{command}, 
you may need to 
specify the offset explicitly, or the flow/static/dynamic frames 
may end up shifted to the right or left.

The above two examples can compute their own offsets, however,
if you were to do, for example:
\begin{verbatim}
\newcommand{\thickgreenyellowbox}[1]{%
{\setlength{\fboxsep}{5pt}\setlength{\fboxrule}{6pt}%
\fcolorbox{green}{yellow}{#1}}}
\end{verbatim}
Then you would have to specify the offset.  In this example,
the offset is $-5\mathrm{pt}-\mathrm{6pt}=-11\mathrm{pt}$,
so you would need to do:
\begin{verbatim}
\setstaticframe{1}{border=thickgreenyellowbox,offset=-11pt}
\end{verbatim}

\item[\key{bordercolor}=\meta{colour}] The colour of the border
if you are using a standard \gls{fcmd}.
The colour can either be specified as, e.g.\ \texttt{green},
or including the colour model, e.g. \verb/[rgb]{0,1,0}/.
For example:
\begin{verbatim}
\setallflowframes{border=doublebox,bordercolor=[rgb]{1,0,0.5}}
\end{verbatim}

\item[\key{textcolor}=\meta{colour}] The text colour for that 
\gls{frame}. Again, the colour can either be specified as, 
e.g.\ \texttt{green}, or including the colour model, 
e.g. \verb/[rgb]{0,1,0}/.

\item[\key{backcolor}=\meta{colour}] The background colour for 
that \gls{frame}. Again, the colour can either be specified as, 
e.g.\ \texttt{green}, or including the colour model, 
e.g. \verb/[rgb]{0,1,0}/. Note that the background colour
only extends as far as the \gls{bbox}, not the border.
If you want it to extend as far as the border, you
will need to define your own border type (see above).

\item[\key{pages}=\meta{page list}] The \useGlosentry{pglist}{list of 
pages} for which the \gls{frame}
should appear. This can either have the values: \texttt{all},
\texttt{even}, \texttt{odd} or \texttt{none} (the latter 
removes the \gls{frame} from that point on---useful if you
have multiple pages with the same number), or it can be a 
comma-separated list of single pages, or 
\useGlosentry{pgrange}{page ranges}.
For example:
\begin{verbatim}
\setdynamicframe{1}{pages=1,5,8-10}
\end{verbatim}

\item[\key{margin}=\meta{side}] The side of
the \gls{flow} that its corresponding margin should go on. This
can take the values \texttt{left}, \texttt{right}, 
\texttt{inner} or \texttt{outer}. This setting is only available
for \gls{flow}s.

\item[\key{clear}=\meta{boolean}] 
If this value is set, the static or dynamic frame will be 
cleared at the start of the
next page, otherwise it will only be cleared on the next
occurrence of \cmdname{setstaticcontents} or the 
\env{staticcontents} environment, or the 
\cmdname{setdynamiccontents}, depending on the frame type.
This value is not set by default. This setting is not 
available for \gls{flow}s.

For example, to prevent the chapter heading reappearing on
every page, I have used the command:
\begin{verbatim}
\setdynamicframe*{chaphead}{clear}
\end{verbatim}

\item[\key{style}=\meta{cmd}] This should be
the name of a command \emph{without} the preceding backslash, 
to be applied to the contents of the specified \gls{dynamic}. 
The command may either be a declaration, for example:
\begin{verbatim}
\setalldynamicframes{style=large}
\end{verbatim}
which will set the contents of all the dynamic frames in a
large font, or it can be a command that takes a single argument,
for example:
\begin{verbatim}
\setalldynamicframes{style=textbf}
\end{verbatim}
which will make the text for all the dynamic frames come out in 
bold.  To unset a style, do \verb/style=none/.
This setting is only available for \gls{dynamic}s.

\item[\key{angle}=\meta{n}]  Rotate the contents of the
\gls{frame} by \meta{n} degrees (new to 
version 1.02). Note that the \gls{bbox}es will not 
appear rotated.
\end{description}

\chapter{Predefined Layouts}
\label{sec:layouts}

The \sty{flowfram} package has a number of commands which 
create \gls{frame}s in a predefined layout. These commands
may only be used in the preamble.

\section{Column Styles}

The standard \LaTeX\ commands \cmdname{onecolumn} and
\cmdname{twocolumn} are redefined to create one or two
\gls{flow}s that fill the entire \gls{typeblock} separated
from each other (in the case of \cmdname{twocolumn}) by a
gap of width \cmdname{columnsep}. The height of these
\gls{flow}s may not be exactly as high as the \gls{typeblock},
as their height is adjusted to make them an integer
multiple of \cmdname{baselineskip}. You can switch off this
automatic adjustment using the command: \cmdname{ffvadjustfalse}.

The \cmdname{onecolumn} and \cmdname{twocolumn} commands
also take an optional argument which is the \gls{pglist}
for which those \gls{flow}s are defined. In addition
to \cmdname{onecolumn} and \cmdname{twocolumn}, the 
following commands are also defined:

\begin{itemize}
\item \cmdname{Ncolumn}\verb"["\meta{pages}\verb"]{"\meta{n}\verb"}"

Create \meta{n} column \gls{flow}s each separated by a 
distance of \cmdname{columnsep}.

\item \cmdname{onecolumninarea}\verb"["\meta{pages}\verb"]{"\meta{width}\verb"}{"\meta{height}\verb"}{"\meta{x}\verb"}{"\meta{y}\verb"}"

This creates a single \gls{flow} to fill the given area,
adjusting the height so that it is an integer multiple of
\cmdname{baselineskip}.

\item \cmdname{twocolumninarea}\verb"["\meta{pages}\verb"]{"\meta{width}\verb"}{"\meta{height}\verb"}{"\meta{x}\verb"}{"\meta{y}\verb"}"

This creates two column \gls{flow}s separated by a distance
of \cmdname{columnsep} filling the entire area specified.

\item \cmdname{Ncolumninarea}\verb"["\meta{pages}\verb"]{"\meta{n}\verb"}{"\meta{width}\verb"}{"\meta{height}\verb"}{"\meta{x}\verb"}{"\meta{y}\verb"}"

A more general form of \cmdname{twocolumninarea} making 
\meta{n} frames instead of two.
\end{itemize}

\section{Column Styles with an Additional Frame}

As well as the column style of \gls{flow}s defined in 
the previous section, it is also possible to define
\meta{n} columns with an additional frame spanning either 
above or below them. There will be a vertical gap of 
approximately\footnote{It may not be exact, as the \gls{flow}s
are adjusted so that their height is an integer multiple
of \cmdname{baselineskip}, which may increase the gap.} 
\cmdname{vcolumnsep} between the columns
and the extra frame. In each of the following definitions,
the argument \meta{pages} is the \gls{pglist} for which
the \gls{frame}s are defined, \meta{n} is the number of
columns required, \meta{type} is the type of frame to go
above or below the columns (this may be one of: \texttt{flow},
\texttt{static} or \texttt{dynamic}). The area in which the
new frames should fill is defined by \meta{width}, \meta{height}
(the width and height of the area) and \meta{x}, \meta{y}
(the position of the bottom left hand corner of the area
relative to the bottom left hand corner of the \gls{typeblock}.)

The height of the additional frame at the top or bottom of
the columns is given by \meta{H}.

\begin{itemize}
\item \cmdname{onecolumntopinarea}\verb"["\meta{pages}\verb"]{"\meta{type}\verb"}{"\meta{H}\verb"}"\meta{width}\verb"}{"\meta{height}\verb"}{"\meta{x}\verb"}{"\meta{y}\verb"}"

One \gls{flow} with a \meta{type} frame above it, filling
the area specified.

\item \cmdname{twocolumntopinarea}\verb"["\meta{pages}\verb"]{"\meta{type}\verb"}{"\meta{H}\verb"}"\meta{width}\verb"}{"\meta{height}\verb"}{"\meta{x}\verb"}{"\meta{y}\verb"}"

Two column style \gls{flow}s with a \meta{type} frame above
them, filling the area specified.

\item \cmdname{Ncolumntopinarea}\verb"["\meta{pages}\verb"]{"%
\meta{type}\verb"}{"\meta{n}\verb"}{"\meta{H}\verb"}"%
\meta{width}\verb"}{"\meta{height}\verb"}{"\meta{x}\verb"}{"%
\meta{y}\verb"}"

\meta{n} column style \gls{flow}s with a \meta{type} frame 
above them, filling the area specified.

\item \cmdname{onecolumnbottominarea}\verb"["\meta{pages}\verb"]{"\meta{type}\verb"}{"\meta{H}\verb"}"\meta{width}\verb"}{"\meta{height}\verb"}{"\meta{x}\verb"}{"\meta{y}\verb"}"

One \gls{flow} with a \meta{type} frame underneath it, filling
the area specified.

\item \cmdname{twocolumnbottominarea}\verb"["\meta{pages}\verb"]{"\meta{type}\verb"}{"\meta{H}\verb"}"\meta{width}\verb"}{"\meta{height}\verb"}{"\meta{x}\verb"}{"\meta{y}\verb"}"

Two column style \gls{flow}s with a \meta{type} frame below 
them, filling the area specified.

\item \cmdname{Ncolumnbottominarea}\verb"["\meta{pages}\verb"]{"%
\meta{type}\verb"}{"\meta{n}\verb"}{"\meta{H}\verb"}"%
\meta{width}\verb"}{"\meta{height}\verb"}{"\meta{x}\verb"}{"%
\meta{y}\verb"}"

\meta{n} column style \gls{flow}s with a \meta{type} frame 
below them, filling the area specified.

\end{itemize}

The following commands are special cases of the above:

\begin{itemize}
\item \cmdname{onecolumntop}\verb"["\meta{pages}\verb"]{"%
\meta{type}\verb"}{"\meta{H}\verb"}"

As \cmdname{onecolumntopinarea} where the area is the entire
\gls{typeblock}.

\item \cmdname{twocolumntop}\verb"["\meta{pages}\verb"]{"%
\meta{type}\verb"}{"\meta{H}\verb"}"

As \cmdname{twocolumntopinarea} where the area is the entire
\gls{typeblock}.

\item \cmdname{Ncolumntop}\verb"["\meta{pages}\verb"]{"%
\meta{type}\verb"}{"\meta{n}\verb"}{"\meta{H}\verb"}"

As \cmdname{Ncolumntopinarea} where the area is the entire
\gls{typeblock}.

\item \cmdname{onecolumnbottom}\verb"["\meta{pages}\verb"]{"%
\meta{type}\verb"}{"\meta{H}\verb"}"

As \cmdname{onecolumnbottominarea} where the area is the entire
\gls{typeblock}.

\item \cmdname{twocolumnbottom}\verb"["\meta{pages}\verb"]{"%
\meta{type}\verb"}{"\meta{H}\verb"}"

As \cmdname{twocolumnbottominarea} where the area is the entire
\gls{typeblock}.

\item \cmdname{Ncolumnbottom}\verb"["\meta{pages}\verb"]{"%
\meta{type}\verb"}{"\meta{n}\verb"}{"\meta{H}\verb"}"

As \cmdname{Ncolumnbottominarea} where the area is the entire
\gls{typeblock}.

\end{itemize}

\section{Backdrop Effects}

\useGlosentry{static}{Static frames} can be used to produce a 
backdrop. There are a number of commands which create
\gls{static}s that can be used as a backdrop. In the
following definitions, \meta{pages} is the \gls{pglist} for
which those \gls{static}s are defined (all is the default). 
For the vertical strips:
\meta{xoffset} is the amount by which the frames should be
shifted horizontally (0pt by default), \meta{W1} is the width of the first frame,
with colour specified by \meta{C1} and \IDL\ \meta{L1},
and so on up to \meta{Wn} the width of the \meta{n}th frame
with colour specified by \meta{Cn} and \IDL\ \meta{Ln}.
For the vertical strips:
\meta{yoffset} is the amount by which the frames should be
shifted vertically (0pt by default), \meta{H1} is the height of the first frame,
with colour specified by \meta{C1} and \IDL\ \meta{L1},
and so on up to \meta{Hn} the height of the \meta{n}th frame
with colour specified by \meta{Cn} and \IDL\ \meta{Ln}.

\textbf{NOTE:} unlike the earlier commands, these commands
are all relative to the actual page, not the \gls{typeblock}.
So an $x$ offset of 0pt indicates the first vertical frame
is flush with the left hand edge of the page, and a $y$ offset
of 0pt indicates the first horizontal frame is flush with 
the bottom edge of the page. This is because backdrops tend
to span the entire page, not just the \gls{typeblock}.

The colour specification must be completely enclosed in braces,
e.g.\ \verb"{[rgb]{1,0,1}}" not \verb"[rgb]{1,0,1}".

\subsection{Vertical stripe effects}

\begin{itemize}
\item \cmdname{vtwotone}\verb"["\meta{pages}\verb"]["\meta{xoffset}\verb"]{"\meta{W1}\verb"}{"\meta{C1}\verb"}{"\meta{L1}%
\verb"}{"\meta{W2}\verb"}{"\meta{C2}\verb"}{"\meta{L2}\verb"}"

Create a two tone vertical strip effect. (This command was
used to create the coloured background on the title page
of this document.)

\item \cmdname{vNtone}\verb"["\meta{pages}\verb"]["\meta{xoffset}\verb"]{"\meta{n}\verb"}{"\meta{W1}\verb"}{"\meta{C1}\verb"}{"\meta{L1}\verb"}"%
\ldots\verb"{"\meta{Wn}\verb"}{"\meta{Cn}\verb"}{"\meta{Ln}\verb"}"

Similar to \cmdname{vtwotone} but with \meta{n} \gls{static}s 
instead of two.

\item \cmdname{vtwotonebottom}\verb"["\meta{pages}\verb"]["\meta{xoffset}\verb"]{"\meta{H}\verb"}{"\meta{W1}\verb"}{"\meta{C1}\verb"}{"\meta{L1}%
\verb"}{"\meta{W2}\verb"}{"\meta{C2}\verb"}{"\meta{L2}\verb"}"

Similar to \cmdname{vtwotone} but the \gls{static}s are only
\meta{H} high, instead of the entire height of the page.
The frames are aligned along the bottom edge of the page.

\item \cmdname{vtwotonetop}\verb"["\meta{pages}\verb"]["\meta{xoffset}\verb"]{"\meta{H}\verb"}{"\meta{W1}\verb"}{"\meta{C1}\verb"}{"\meta{L1}%
\verb"}{"\meta{W2}\verb"}{"\meta{C2}\verb"}{"\meta{L2}\verb"}"

Similar to \cmdname{vtwotone} but the \gls{static}s are only
\meta{H} high, instead of the entire height of the page.
The frames are aligned along the top edge of the page.
(This command was used to create the border effect along
the top of every page in this document. Two \cmdname{vtwotonetop}
commands were used, one for the even pages, and the other
for the odd pages.)

\item \cmdname{vNtonebottom}\verb"["\meta{pages}\verb"]["\meta{xoffset}\verb"]{"\meta{H}\verb"}{"\meta{n}\verb"}{"\meta{W1}\verb"}{"\meta{C1}\verb"}{"\meta{L1}\verb"}"%
\ldots\linebreak
\verb"{"\meta{Wn}\verb"}{"\meta{Cn}\verb"}{"\meta{Ln}\verb"}"

More general version of \cmdname{vtwotonebottom} but for
\meta{n} frames.

\item \cmdname{vNtonetop}\verb"["\meta{pages}\verb"]["\meta{xoffset}\verb"]{"\meta{H}\verb"}{"\meta{n}\verb"}{"\meta{W1}\verb"}{"\meta{C1}\verb"}{"\meta{L1}\verb"}"%
\ldots\linebreak
\verb"{"\meta{Wn}\verb"}{"\meta{Cn}\verb"}{"\meta{Ln}\verb"}"

More general version of \cmdname{vtwotonetop} but for
\meta{n} frames.

\end{itemize}

\subsection{Horizontal stripe effect}

\begin{itemize}
\item \cmdname{htwotone}\verb"["\meta{pages}\verb"]["\meta{yoffset}\verb"]{"\meta{H1}\verb"}{"\meta{C1}\verb"}{"\meta{L1}%
\verb"}{"\meta{H2}\verb"}{"\meta{C2}\verb"}{"\meta{L2}\verb"}"

Create a two tone horizontal strip effect.

\item \cmdname{hNtone}\verb"["\meta{pages}\verb"]["\meta{yoffset}\verb"]{"\meta{n}\verb"}{"\meta{H1}\verb"}{"\meta{C1}\verb"}{"\meta{L1}\verb"}"%
\ldots\linebreak
\verb"{"\meta{Hn}\verb"}{"\meta{Cn}\verb"}{"\meta{Ln}\verb"}"

Similar to \cmdname{htwotone} but with \meta{n} \gls{static}s 
instead of two.

\item \cmdname{htwotoneleft}\verb"["\meta{pages}\verb"]["\meta{yoffset}\verb"]{"\meta{W}\verb"}{"\meta{H1}\verb"}{"\meta{C1}\verb"}{"\meta{L1}%
\verb"}{"\meta{H2}\verb"}{"\meta{C2}\verb"}{"\meta{L2}\verb"}"

Similar to \cmdname{htwotone} but the \gls{static}s are only
\meta{W} wide, instead of the entire width of the page.
The frames are aligned along the left edge of the page.

\item \cmdname{htwotoneright}\verb"["\meta{pages}\verb"]["%
\meta{yoffset}\verb"]{"\meta{W}\verb"}{"\meta{H1}\verb"}{"%
\meta{C1}\verb"}{"\meta{L1}%
\verb"}{"\meta{H2}\verb"}{"\meta{C2}\verb"}{"\meta{L2}\verb"}"

Similar to \cmdname{htwotone} but the \gls{static}s are only
\meta{W} wide, instead of the entire width of the page.
The frames are aligned along the right edge of the page.

\item \cmdname{hNtoneleft}\verb"["\meta{pages}\verb"]["%
\meta{yoffset}\verb"]{"\meta{W}\verb"}{"\meta{n}\verb"}{"%
\meta{H1}\verb"}{"\meta{C1}\verb"}{"\meta{L1}\verb"}"%
\ldots\linebreak
\verb"{"\meta{Hn}\verb"}{"\meta{Cn}\verb"}{"\meta{Ln}\verb"}"

More general version of \cmdname{htwotoneleft} but for
\meta{n} frames.

\item \cmdname{hNtoneright}\verb"["\meta{pages}\verb"]["%
\meta{yoffset}\verb"]{"\meta{W}\verb"}{"\meta{n}\verb"}{"%
\meta{H1}\verb"}{"\meta{C1}\verb"}{"\meta{L1}\verb"}"%
\ldots\linebreak
\verb"{"\meta{Hn}\verb"}{"\meta{Cn}\verb"}{"\meta{Ln}\verb"}"

More general version of \cmdname{htwotoneright} but for
\meta{n} frames.

\end{itemize}

\subsection{Background Frame}

To make a single \gls{static} covering the entire page, use:
\cmdname{makebackgroundframe}\verb"["\meta{pages}\verb"]["\meta{IDL}\verb"]". Note that this \gls{static} should be created
before any other \gls{static} as it will obscure all other
\gls{static}s created before it if it is given a background
colour.

\subsection{Vertical and Horizontal Rules}

You can create vertical or horizontal rules between two
\gls{frame}s using the commands:

\begin{itemize}

\item \cmdname{insertvrule}\verb"["\meta{y top}\verb"]["%
\meta{y bottom}\verb"]{"\meta{T1}\verb"}{"\meta{IDN1}\verb"}{"%
\meta{T2}\verb"}{"\meta{IDN2}\verb"}"

This creates a new \gls{static} which fits between \meta{T1}
\gls{frame} with \IDN\ \meta{IDN1} and \meta{T2} \gls{frame}
with \IDN\ \meta{IDN2}, and places a vertical rule in it
extending from the highest point of the highest frame to
the lowest point of the lowest frame. The first optional
argument \meta{y top} (default 0pt) extends the rule by that 
much above the highest point, and the second optional argument
\meta{y bottom} (default 0pt) extends the rule by that much 
below the lowest point. If either of the optional arguments
are negative, the rule will be shortened instead of extended.
The width of the rule is given by \cmdname{columnseprule}.

\item \cmdname{inserthrule}\verb"["\meta{x left}\verb"]["%
\meta{x right}\verb"]{"\meta{T1}\verb"}{"\meta{IDN1}\verb"}{"%
\meta{T2}\verb"}{"\meta{IDN2}\verb"}"

This creates a new \gls{static} which fits between \meta{T1}
\gls{frame} with \IDN\ \meta{IDN1} and \meta{T2} \gls{frame}
with \IDN\ \meta{IDN2}, and places a horizontal rule in it
extending from the leftmost point of the left frame to
the rightmost point of the right frame. The first optional
argument \meta{x left} (default 0pt) extends the rule by that 
much before the leftmost point, and the second optional argument
\meta{x right} (default 0pt) extends the rule by that much 
beyond the rightmost point. If either of the optional arguments
are negative, the rule will be shortened instead of extended.
The height of the rule is given by \cmdname{columnseprule}.

\end{itemize}

The default value for \cmdname{columnseprule} is 2pt.
Both of the above commands have starred versions which allow
you to identify the \gls{frame} by \IDL\ instead of \IDN.
The \gls{frame} types, \meta{T1} and \meta{T2} can be one of
the following keywords: \texttt{flow}, \texttt{static}
or \texttt{dynamic}.


\chapter{Thumbtabs and Minitocs}

\section{Thumbtabs}

On the right hand side of this page, there is a blue rectangle
with the chapter number in it. This is a thumbtab, and it gives
you a rough idea whereabouts in the document you are when you
quickly flick through the pages. Each thumbtab is in fact
a dynamic frame, and you can control whether to make the
number and/or title appear in the thumbtab by using one or
more of the package options: \pkgopt{ttbtitle} (show 
title---default), \pkgopt{ttbnotitle} (don't show the title), 
\pkgopt{ttbnum} (show the number) and \pkgopt{ttbnonum}
(don't show the number---default).

If you want thumbtabs in your document, you need to use
the command
\begin{center}
\cmdname{makethumbtabs}\verb"["\meta{y 
offset}\verb"]{"\meta{height}\verb"}["\meta{section type}\verb"}"
\end{center} 
in the document
preamble. By default, the topmost thumbtab is level with the
top of the \gls{typeblock}, but can be shifted vertically
using the first optional argument \meta{y offset}. Each
thumbtab will be \meta{height} high, and will correspond to the
sectioning type \meta{section type}. If \meta{section type}
is omitted, chapters will be used if the \cmdname{chapter}
command is defined, otherwise sections will be used.
The width of the thumbtabs is given by the length 
\cmdname{thumbtabwidth}, which is 1cm by default.

The command \cmdname{thumbtabindex} will display the
thumbtab index (all thumbtabs) on the current page. You 
then need to use the command \cmdname{enablethumbtabs}
to start the individual thumbtabs and \cmdname{disablethumbtabs}
to make them go away.

You can align the table of contents with the thumbtabs\footnote{but
only do this if there is enough room on the page!} using
the command \cmdname{tocandthumbtabindex} instead of
the commands \cmdname{tableofcontents} and 
\cmdname{thumbtabindex}. If you are using the \sty{hyperref}
package, the text on the thumbtab index will be a hyperlink
to the corresponding part of the document. Note that you may
need to shift the thumbtabs vertically up or down to make
sure that they align correctly with the table of contents.

The format of the text on the thumbtabs is given by the command
\cmdname{thumbtabindexformat} for the thumbtab index entries,
and \cmdname{thumbtabformat} for the individual thumbtabs. By
default the text on the thumbtabs will be rotated, but as
rotating is not implemented by some previewers, the package option
\pkgopt{norotate} is provided, which will stack the letters 
vertically. This does not look as good as the rotated text.
In addition, some previewers do not put the hyperlink in the
correct place when the link has been rotated, so this may
also cause a problem.

The thumbtab attributes can be changed using 
\cmdname{setthumbtab}\verb"{"\meta{n}\verb"}{"\meta{key value
list}\verb"}", where \meta{n} is the thumbtab number starting
from 1 (for the top thumbtab) to \verb"\value{maxthumbtabs}"
(for the bottom thumbtab). Note that these numbers are not
related to the \gls{frame} \IDN{}s.

To just change the settings for the thumbtab index, use
\cmdname{setthumbtabindex}\verb"{"\meta{n}\verb"}{"\meta{key value list}\verb"}". The \meta{key value list} for both these
commands is the same as that for \cmdname{setdynamicframe}.

By default, the thumbtabs are given a grey background. In this
document, I have used:
\begin{verbatim}
\setthumbtab{1}{backcolor=[rgb]{0.15,0.15,1}}
\setthumbtab{2}{backcolor=[rgb]{0.2,0.2,1}}
\setthumbtab{3}{backcolor=[rgb]{0.25,0.25,1}}
\setthumbtab{4}{backcolor=[rgb]{0.3,0.3,1}}
\setthumbtab{5}{backcolor=[rgb]{0.35,0.35,1}}
\setthumbtab{6}{backcolor=[rgb]{0.4,0.4,1}}
\setthumbtab{7}{backcolor=[rgb]{0.45,0.45,1}}
\end{verbatim}
to change the thumbtab background colour to shades of blue.

I have also changed the style of the thumbtab text using:
\begin{verbatim}
\newcommand{\thumbtabstyle}[1]{\textsc{\sffamily #1}}
\setthumbtab{all}{style=thumbtabstyle,textcolor=white}
\end{verbatim}

Given that I have specified white text, why does the thumbtab
index have black text? This is because the text on the thumbtab
index is a hyperlink, and I have used \verb"linkcolor=black"
with the hyperref package.

\section{Minitocs}

In this document, after each chapter heading, there is a
mini table of contents for that chapter. To enable minitocs,
use the command 
\cmdname{enableminitoc}\verb"["\meta{section type}\verb"]".
The default \meta{section type} is the same as the thumbtabs.

If you want the minitocs to appear in a \gls{dynamic}, you
can use \cmdname{appenddfminitoc}\verb"{"\meta{IDN}\verb"}"
where \meta{IDN} is the \IDN\ of the appropriate \gls{dynamic}.
There is also a starred version available if you want to 
use the \IDL\ instead of the \IDN.

For example, in this document I have used the command:
\begin{verbatim}
\appenddfminitoc*{chaphead}
\end{verbatim}
in the preamble, which has appended the minitocs to the
\gls{dynamic} with \IDL\ \texttt{chaphead}.

The style of the minitoc text is given by the command
\cmdname{minitocstyle} which takes one argument, the contents
of the minitoc. This command may be redefined if you want
to change the minitoc style. The gap before the minitoc
is given by the length \cmdname{beforeminitocskip} and
the gap after the minitoc is given by the length 
\cmdname{afterminitocskip}. These lengths may be changed
using \cmdname{setlength}.

\chapter{Global Values}

The following macros can be changed using \cmdname{renewcommand}:

\begin{itemize}
\item \cmdname{setffdraftcolor} 
This sets the colour of the \gls{bbox}
when it is displayed in draft mode.  The default value is:
\verb/\color[gray]{0.8}/. For example, if you want a darker grey,
do:
\begin{verbatim}
\renewcommand{\setffdraftcolor}{\color[gray]{0.3}}
\end{verbatim}

\item 
\cmdname{setffdrafttypeblockcolor} This sets the colour of
the \gls{bbox} of the \gls{typeblock} when it is displayed
in draft mode.  The default value is: \verb/\color[gray]{0.9}/.
For example, if you want a medium grey, do:
\begin{verbatim}
\renewcommand{\setffdrafttypeblockcolor}{\color[gray]{0.5}}
\end{verbatim}

\item \cmdname{fflabelfont}
This sets the font size for the \gls{bbox} markers in 
draft mode. The default value is: \verb/\small\sffamily/.
For example, if you want a larger font, do:
\begin{verbatim}
\renewcommand{\fflabelfont}{\large\sffamily}
\end{verbatim}

\end{itemize}

The following are lengths, which can be changed using
\cmdname{setlength}:

\begin{itemize}
\item \cmdname{fflabelsep}
This is the distance from the right hand side of the
\gls{bbox} at which to place the \gls{bbox} marker. The
default value is: \texttt{1pt}

\item \cmdname{flowframesep}
This is the gap between the text of the \gls{frame} and
its border, for the standard border types. 

\item \cmdname{flowframerule}
This is the width of the \gls{frame}'s border, if using
a border given by a \useGlosentry{fcmd}{frame}
\useGlosentry{fcmd}{making command} that uses \cmdname{fboxsep}
to set its border width (e.g.\ \cmdname{fbox}).
\end{itemize}

\chapter{Troubleshooting}

\section{General Queries}

\begin{enumerate}
\item If all my \gls{flow}s are only defined on, say, 
pages 1-10, what happens if I then add some extra text so that
the document exceeds 10 pages?

The output routine will create a new \gls{flow} the size 
of the \gls{typeblock}, and use that.

\item Can I use the formatted page number in \gls{pglist}s?

No.

\item\label{itm:whynot} Why not?

When the output routine finishes with one \gls{flow}
it looks for the next \gls{flow} defined on that page,
if there are none left, it then searches through the \gls{pglist}
of all the defined \gls{flow}s to see if the next page 
lies in that range, if there are none defined on that page,
it ships out that page, and tries the next page. 
This gives rise to two problems:

\begin{enumerate}
\item \LaTeX\ is not clairvoyant. If it is currently
on page 14, and on the next page the page numbering changes
to A, it has no way of knowing this until it has reached
that point, which it hasn't yet. So it is looking for a 
\gls{flow} defined on page~15, not on page~A.

\item How does \LaTeX\ tell if page C lies between 
pages A and D? It would require an algorithm that can convert
from a formatted number back to an integer. Given that there
are many different ways of formatting the value of a counter
(besides the standard Roman and alphabetical formats) it
would be impossible to write an algorithm to do this
for some arbitrary format.
\end{enumerate}

\item Can I have an arbitrarily shaped \gls{frame}?
\label{itm:parshape}

No. All frames are rectangular, however within a frame 
you may use commands such as \TeX's \cmdname{parshape} to
create a non rectangular paragraph. (See the sample file
\texttt{news.tex} for an example.)

\item Why has the text from my \gls{flow} appeared in a
\gls{static} or \gls{dynamic}?

Assuming you haven't inadvertently set that text as the contents
of the static or dynamic frame, the frames are most likely 
overlapping (see \autoref{sec:stacking}).
In an attempt to clarify what's going on, suppose you have 
defined a \gls{static}, a \gls{dynamic} and two \gls{flow}s. The 
following is an approximate\footnote{The pedantic may point out 
that \TeX\ may make several attempts to fill in the flow frames 
depending on penalties and so on.} analogy: \TeX\ has a sheet of 
paper on the table, and has pencilled\footnote{actually it hasn't
drawn anything really, but it has in its mind's eye.} in a 
rectangle denoting the \gls{typeblock}.  The paper is put to one 
side for now.  \TeX\ also has four rectangular sheets
of transparent paper. The first (which I shall call sheet~1)
represents the \gls{static}, the next two (which I shall call
sheets~2 and~3) represent the \gls{flow}s, and the last one
(which I shall call sheet~4) represents the \gls{dynamic}.
\TeX\ starts work on filling sheet~2 with the document text.
Once it has put as much text on that sheet as it considers 
possible (according to its views on aesthetics), it puts sheet~2
into the ``in tray'', and then continues on sheet~3. While it's
filling in sheets~2 and~3, if it encounters a command or 
environment that tells it what to put in the \gls{static}, 
it fills in sheet~1 and then puts sheet~1 into the ``in tray'' and
resumes where it left off on sheet~2 or~3. Similarly, if
it encounters a command that tells it what to put in the
\gls{dynamic}, it stops what it's doing, fills in sheet~4, then
puts sheet~4 into the ``in tray'', and resumes where it left off.
Only when it has finished sheet~3 (the last \gls{flow} defined
on that page), will it gather together all
the transparent sheets, and fix them onto the page starting
with sheet~1 through to sheet~4, measuring the bottom left hand 
corner of each transparent sheet relative to the bottom left hand 
corner of the \gls{typeblock}.  \TeX\ will then put that page 
aside, and start work on the next page. If two or more of the
transparent sheets overlap, you will see through the top one into
the one below (unless of course the top one has been painted
over, either by setting a background colour, or by adding an
image that has a non-transparent background.)

Note that it's also possible that the overlap is caused by an 
overfull hbox that's causing the text to poke out the side of the 
\gls{flow} into a neighbouring \gls{frame}.

\item Why do I get lots of overfull hbox messages?

Probably because you have narrow \gls{frame}s.
\end{enumerate}

\section{Unexpected Output}

\begin{enumerate}
\item The lines at the beginning of my \gls{flow}s are the
wrong width.

This is a problem that will occur if you have \gls{flow}s
with different widths, as the change in \cmdname{hsize}
does not come into effect until a paragraph break. So if
you have a paragraph that spans two \gls{flow}s, the end
of the paragraph at the beginning of the second \gls{flow}
will retain the width it had at the start of the 
paragraph at the bottom of the previous \gls{flow}. You can
fix the problem by inserting \cmdname{framebreak} at the
point where the \gls{frame} break occurs 
(see \autoref{sec:framebreak}).

\item My \gls{frame}s shift to the right when I add a border.

This may occur if you use a border that is not recognised
by the \sty{flowfram} package. You will need to set the
offset using the \texttt{offset} key (see \autoref{sec:modattr}).

\item I have a vertical white strip along the right hand side
of every page.

This occurs if you use \appname{ps2pdf} of \appname{dvipdf}.
Try using \appname{pdflatex} instead.

\item I don't have any output.

All your \gls{flow}s are empty. \TeX\ doesn't put the
frames onto the page until it has finished putting text
into the \gls{flow}s. So if there is no text to go in the
\gls{flow}s it won't output the page. If you only want the
\gls{static}s or \gls{dynamic}s filled in, and nothing 
outside of them, just do \verb|\mbox{}|. This will put
an invisible something with zero area into your
\gls{flow}, but it's enough to convince \TeX\ that the
document contains some text.

\item The last page hasn't appeared.

See the previous answer.

\end{enumerate}

\section{Error Messages}

\begin{enumerate}
\item \verb/Illegal unit of measure (pt inserted)/

All lengths must have units. Remember to include the
units when defining new \gls{frame}s. The following
keys require lengths: \texttt{width}, \texttt{height},
\texttt{x}, \texttt{y} and \texttt{offset}\footnote{%
\texttt{offset} can also have the value \texttt{compute}}.

\item \verb/Missing number, treated as zero/

\LaTeX\ is expecting a number. There are a number of 
possible causes:

\begin{enumerate}
\item You have used an \IDL\ instead of an \IDN. If you
want to refer to a frame by its label, you need to remember
to use the starred versions of the 
\verb/\set/\meta{type}\verb/frame/ commands, or when setting
the contents of \gls{static}s or \gls{dynamic}s.

\item When specifying page lists, you have mixed keywords
with page ranges. For example: \texttt{1,even} is invalid.
\end{enumerate}

\item \verb/Flow frame IDL '/\meta{label}\verb+' already defined+

All \useacronym[s]{IDL} within each \gls{frame} type must be
unique. There are similar error messages for duplicate 
\useacronym[s]{IDL} for \gls{static}s and \gls{dynamic}s.

\item \verb/Can't find flow frame id/

You have specified a non-existent \gls{flow} \IDL. There are
similar error messages for \gls{static}s and \gls{dynamic}s.
Check to make sure you have spelt the label correctly, and
check you are using the correct \gls{frame} type command.
(For example, if a \gls{static} has the \IDL\ \texttt{mylabel},
and you attempt to do 
\verb/\setflowframe*{mylabel}{/\meta{options}\verb/}/, 
then you will get this error, because \texttt{mylabel} refers
to a \gls{static} not a \gls{flow}.)

\item \verb/Key 'clear' is boolean/

The \texttt{clear} key can only have the values \texttt{true}
or \texttt{false}.

\item \verb/Key 'clear' not available/

The \texttt{clear} key is only available for static
and dynamic \gls{frame}s.

\item \verb/Key 'style' not available/

The \texttt{style} key is only available for \gls{dynamic}s.

\item \verb/Key 'margin' not available/

The \texttt{margin} key is only available for \gls{flow}s.

\item \verb/Dynamic frame style '/\meta{style}\verb+' not defined+

The specified style \meta{style} must be the name of a command
without the preceding backslash.  It is possible that you have
mis-spelt the name, or you have forgotten to define the command.

\item \verb/Argument of \fbox has an extra }/

This error will occur if you do, say, \verb/border=\fbox/
instead of \verb/border=fbox/. Remember not to include
the initial backslash.

\item \verb/Not in outer par mode/

You can not have floats (such as figures, tables or marginal
notes) in \gls{static}s or \gls{dynamic}s. If you want
a figure or table within a \gls{static} or \gls{dynamic}
use \env{staticfigure} or \env{statictable}.

\item \verb/Somethings wrong---maybe missing \item/

Assuming that all your list type of environments start
with \cmdname{item}, this may be caused by something going
wrong with the toc (table of contents), ttb (thumbtab)
or aux (auxiliary) files in the previous run. Try deleting
them, and try again.

\item \verb/Undefined control sequence/
\begin{verbatim}
\color ...vevmode \csname fi\endcsname }\@ldc@l@r
\end{verbatim}

This seems to happen when the \texttt{draft} option is used without
a colour package, in conjunction with PDF\LaTeX. It 
doesn't seem to happen with \LaTeX. For some reason with
PDF\LaTeX\ the \cmdname{color} command is defined in the
begin document hook, but not the command 
\cmdname{@ldc@l@r} which is the cause of this error 
message. You can avoid this error message by including
the \sty{color} package. You can use the 
\texttt{monochrome} option if you don't want any colour
in your document.

\end{enumerate}
\disablethumbtabs

\printglossary

\printindex

% have a backcover:
\setdynamicframe*{footer}{pages=none}
\setstaticframe*{lastbackleft,lastbackright}{pages=even}
\mbox{}\finishthispage
\end{document}
