\documentclass{article}
\usepackage{array,ifthen}
\usepackage{zefonts}
\title{Testing the `ze' fonts
       (virtual T1 fonts)}
\author{Robert Fuster\\
        \footnotesize Departament de Matem\`atica Aplicada\\
        \footnotesize Universitat Polit\`ecnica de Val\`encia\\
        \footnotesize 46071 Val\`encia (Spain)\\
        \footnotesize e-mail: \texttt{rfuster\@ mat.upv.es}\\
        \footnotesize URL: \texttt{HTTP://www.upv.es/\~{}rfuster}}
\date{June 24, 1997\\
      September 22, 1998 (version 2.0)}
\setlength{\extrarowheight}{2pt}
\newcounter{n} \newcounter{p}
\newcommand{\taula}{%
    \setcounter{n}{0}
    \setcounter{p}{0}
    \begin{center}
    \setcounter{n}{0}
    \setcounter{p}{0}
    \begin{tabular}{|c|c|c|c|c|c|c|c|c|c|c|c|c|c|c|c|}
    \hline
    \whiledo{\value{n} < 255}{%
     \symbol{\then}\stepcounter{p}\stepcounter{n}\ifthenelse{\value{p} = 16}{%
                                  \setcounter{p}{0}\\\hline
                                 }{\ifthenelse{\value{n} < 256}{&}{}%
                                     }}
    \symbol{\then}\\\hline
    \end{tabular}
    \end{center}
}
\begin{document}
\maketitle
This is a test of the `ze' fonts. In the following figures you can compare
these fonts with the real `ec' fonts.
\begin{figure}
    \caption{Roman virtual font zerm1000 \label{fig:zerm1000}}
    \taula
    \caption{Roman font ecrm1000 \label{fig:ecrm1000}}
    {\renewcommand{\rmdefault}{cmr}\rmfamily\selectfont\taula
    }
\end{figure}
\begin{figure}
    \caption{Sans Serif virtual font zess1000 \label{fig:zess1000}}
    \sffamily\taula
    \caption{Sans Serif font ecss1000 \label{fig:ecss1000}}
    {\renewcommand{\sfdefault}{cmss}\sffamily\selectfont\taula
    }
\end{figure}
\begin{figure}
    \caption{Typewriter virtual font zett1000 \label{fig:zett1000}}
    \ttfamily\taula
    \caption{Typewriter font ectt1000 \label{fig:ectt1000}}
    {\renewcommand{\ttdefault}{cmtt}\ttfamily\selectfont\taula
    }
\end{figure}
\end{document}
