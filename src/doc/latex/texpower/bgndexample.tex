%%
%% This is file `bgndexample.tex',
%% generated with the docstrip utility.
%%
%% The original source files were:
%%
%% texpower-doc.dtx  (with options: `bgndexample,bgndexample-src,end')
%% 
%% --------------------------------------------------------------
%% TeXPower bundle - dynamic online presentations with LaTeX
%% Copyright (C) 1999-2004 Stephan Lehmke
%% 
%% This program is free software; you can redistribute it and/or
%% modify it under the terms of the GNU General Public License
%% as published by the Free Software Foundation; either version 2
%% of the License, or (at your option) any later version.
%% 
%% This program is distributed in the hope that it will be useful,
%% but WITHOUT ANY WARRANTY; without even the implied warranty of
%% MERCHANTABILITY or FITNESS FOR A PARTICULAR PURPOSE.  See the
%% GNU General Public License for more details.
%% --------------------------------------------------------------
%% 
%% The list of all files belonging to the TeXPower bundle is
%% given in the file `00readme.txt'.
%% 
\ProvidesFile{bgndexample.tex}%
      [2004/07/27 TeXPower example file]
%-----------------------------------------------------------------------------------------------------------------
% File: bgndexample.tex
%
% Background style example for the package texpower.sty.
%
% This file can be compiled with pdfLaTeX or (standard) LaTeX. When using standard LaTeX, the dvi file produced should
% be converted to pdf afterwards (using dvips+distill/ps2pdf or dvipdf, for instance).
%
% The resulting pdf file is meant for presenting `interactively' with Adobe Acrobat Reader.
%
%-----------------------------------------------------------------------------------------------------------------
% Autor: Stephan Lehmke <Stephan.Lehmke@cs.uni-dortmund.de>
%
% v0.0.1 Aug 10, 2000: First version for the pre-alpha release of TeXPower.
%
% v0.0.2 Apr 29, 2003: Adapted to TeXPower v0.0.9c.
%
%COMMENT
%-----------------------------------------------------------------------------------------------------------------
% Use slifonts and a dark background.

\PassOptionsToPackage{darkbackground,colorhighlight,verbose}{texpower}

\RequirePackage{tpslifonts}

% Input the generic preamble.

%%
%% This is file `__TPpreamble.tex',
%% generated with the docstrip utility.
%%
%% The original source files were:
%%
%% texpower-doc.dtx  (with options: `preamble')
%% 
%% --------------------------------------------------------------
%% TeXPower bundle - dynamic online presentations with LaTeX
%% Copyright (C) 1999-2004 Stephan Lehmke
%% 
%% This program is free software; you can redistribute it and/or
%% modify it under the terms of the GNU General Public License
%% as published by the Free Software Foundation; either version 2
%% of the License, or (at your option) any later version.
%% 
%% This program is distributed in the hope that it will be useful,
%% but WITHOUT ANY WARRANTY; without even the implied warranty of
%% MERCHANTABILITY or FITNESS FOR A PARTICULAR PURPOSE.  See the
%% GNU General Public License for more details.
%% --------------------------------------------------------------
%% 
%% The list of all files belonging to the TeXPower bundle is
%% given in the file `00readme.txt'.
%% 
%
\documentclass
[%
%-----------------------------------------------------------------------------------------------------------------
% Document class options:
% -----------------------
%
% Landscape slides formatted for letter paper fit most screen resolutions (more or less).
%
  letterpaper,%
  landscape,%
%
% The KOMA option makes powersem load scrartcl.cls instead of article.cls.
%
  KOMA,%
% KOMA document class options are accepted.
  smallheadings,%
%
% The calcdimensions option makes powersem calculate the slide dimensions automatically from paper size and margins.
  calcdimensions,%
%
% The display option sets everything up for producing slides to be displayed interactively.
% This option is also recognized by the texpower package.
%
  display%
%-----------------------------------------------------------------------------------------------------------------
]
%-----------------------------------------------------------------------------------------------------------------
% Document class powersem, based on seminar.cls for simulating ppower via latex+distiller (instead of pdflatex).
%
{powersem}
%-----------------------------------------------------------------------------------------------------------------
%
% First part of the preamble of TeXPower demos.
%
%-----------------------------------------------------------------------------------------------------------------
% Autor: Stephan Lehmke <Stephan.Lehmke@cs.uni-dortmund.de>
%
% v0.0.1 Mar 20, 2000: First version for the pre-alpha release of TeXPower.
% v0.0.2 Mar 21, 2000: Remedying an incompatibility between LaTeX releases concerning the implementation of
%                      \@iiiparbox (Apr 11: this code is now part of texpower.sty).
% v0.0.3 Apr 11, 2000: Color emphasis code moved into texpower.
%

%-----------------------------------------------------------------------------------------------------------------
% Set slide margins rather small for maximum use of space. This is a demo, remember.
%
\renewcommand{\slidetopmargin}{5mm}
\renewcommand{\slidebottommargin}{5mm}

\renewcommand{\slideleftmargin}{5mm}
\renewcommand{\sliderightmargin}{5mm}


%-----------------------------------------------------------------------------------------------------------------
% Some setup for more reasonable spacing.
%

\makeatletter

\renewcommand\section{\@startsection{section}{1}{\z@}%
  {-1.5ex\@plus -1ex \@minus -.5ex}%
  {.5ex \@plus .2ex}%
  {\raggedsection\normalfont\size@section\sectfont}}

\renewcommand\subsection{\@startsection{subsection}{2}{\z@}%
  {-1.25ex\@plus -1ex \@minus -.2ex}%
  {.5ex \@plus .2ex}%
  {\raggedsection\normalfont\size@subsection\sectfont}}

\renewcommand\subsubsection{\@startsection{subsubsection}{3}{\z@}%
  {-1.25ex\@plus -1ex \@minus -.2ex}%
  {.5ex \@plus .2ex}%
  {\raggedsection\normalfont\size@subsubsection\sectfont}}

\renewcommand\paragraph{\@startsection{paragraph}{4}{\z@}%
  {1.25ex \@plus1ex \@minus.2ex}%
  {-1em}%
  {\raggedsection\normalfont\size@paragraph\sectfont}}

\def\slideitemsep{.5ex plus .3ex minus .2ex}

\makeatother

%-----------------------------------------------------------------------------------------------------------------
% We need some more packages...
%

\usepackage{url}

\usepackage[latin1]{inputenc}

% One more Text emphasis command...

\let\name=\textsc

% Second part of the preamble of TeXPower demos.
%
%-----------------------------------------------------------------------------------------------------------------
% Autor: Stephan Lehmke <Stephan.Lehmke@cs.uni-dortmund.de>
%
% v0.0.1 Mar 20, 2000: First version for the pre-alpha release of TeXPower.
% v0.0.2 Mar 22, 2000: Now loading the config file.
% v0.0.3 Mar 29, 2000: texpower doesn't load hyperref any more; there's now a package fixseminar.
% v0.0.4 Apr 19, 2000: Added \slidetitle command.
% v0.0.5 Sep 11, 2000: Added plainpages=false to the hyperref options to get correct page anchors.
% v0.0.5 Sep 11, 2002: Slight changes to title page.
%


%-----------------------------------------------------------------------------------------------------------------
% We load hyperref and fixseminar which fixes some problems with seminar.
%
\usepackage[plainpages=false,bookmarksopen,colorlinks,urlcolor=red,pdfpagemode=FullScreen]{hyperref}
\usepackage{fixseminar}

%-----------------------------------------------------------------------------------------------------------------
% Finally, the texpower package is loaded.
%
\usepackage{texpower}

%% The configuration file allows user-specific settings.

%%
%% This is file `__TPcfg.tex',
%% generated with the docstrip utility.
%%
%% The original source files were:
%%
%% texpower-doc.dtx  (with options: `config')
%% 
%% --------------------------------------------------------------
%% TeXPower bundle - dynamic online presentations with LaTeX
%% Copyright (C) 1999-2004 Stephan Lehmke
%% 
%% This program is free software; you can redistribute it and/or
%% modify it under the terms of the GNU General Public License
%% as published by the Free Software Foundation; either version 2
%% of the License, or (at your option) any later version.
%% 
%% This program is distributed in the hope that it will be useful,
%% but WITHOUT ANY WARRANTY; without even the implied warranty of
%% MERCHANTABILITY or FITNESS FOR A PARTICULAR PURPOSE.  See the
%% GNU General Public License for more details.
%% --------------------------------------------------------------
%% 
%% The list of all files belonging to the TeXPower bundle is
%% given in the file `00readme.txt'.
%% 
%-----------------------------------------------------------------------------------------------------------------
% File: __TPcfg.tex
%
% Code for user-specific configuration of TeXPower documentation files.
%
% This file is input by others. Don't compile it separately.
%
%-----------------------------------------------------------------------------------------------------------------
% Autor: Stephan Lehmke <Stephan.Lehmke@cs.uni-dortmund.de>
%
% v0.0.1 Mar 22, 2000: First version for the pre-alpha release of TeXPower.
%
\hypersetup{baseurl={http://texpower.sourceforge.net/doc/}}
\hypersetup{pdfsubject={Documentation and Examples for the texpower package}}
\hypersetup{pdfauthor={Stephan Lehmke}}
\endinput
%%
%% End of file `__TPcfg.tex'.


%-----------------------------------------------------------------------------------------------------------------
% Some more parameters...
%
\slidesmag{5}
\slideframe{none}
\pagestyle{empty}
\setcounter{tocdepth}{2}
\renewcommand{\currentpagevalue}{\value{slide}}

%-----------------------------------------------------------------------------------------------------------------
% The following command produces a title page for every example and documentation file.

\newcommand{\makeslidetitle}[1]
{%
  \title{The \TeX Power bundle\\[2ex]{\normalfont #1}}
  \author
  {%
    Stephan Lehmke\\
    \mdseries
    University of Dortmund\\
    \mdseries
    Department of Computer Science I\\
    \url{mailto:Stephan.Lehmke@udo.edu}%
  }
  {\centerslidestrue
  \maketitle
  \newslide}
  \setcounter{firststep}{1}% This way, the first step of all examples is displayed.
}
\endinput
%%
%% End of file `__TPpreamble.tex'.

\hypersetup{pdftitle={texpower background style example}}

\newcommand{\skipTo}[1]{\hyperlink{#1}{\present{\textsf{\textbf{Skip animation}}}}}

\makeatletter
\newcommand{\histogram}[1]
{{%
    \renewcommand{\vstripe@TP}[4]
    {\rule{##2-2pt}{(##3)*\real{##1}}\hspace*{2pt}##4}%
    #1%
    }}
\makeatother

\newcommand{\totalbarwidth}{2cm}

\newcommand{\mkbar}[2][100]
{%
  \ifthenelse{#1<#2}{\def\percentval{#1}}{\def\percentval{#2}}%
  \mkfactor{\intensity}{\percentval/100}%
  \colorbetween[\intensity]{ecolor}{green}{red}%
  \hgradrule[\percentval]{red}{ecolor}{\totalbarwidth*\real{\intensity}}{1ex}
  \textbf{\boldmath$\mathsf{\ifthenelse{\percentval<10}{\phantom{0}}{}\percentval\%}$}%
  }

\renewcommand{\bgndstripes}{100}

\setlength{\fboxrule}{1pt}

%-----------------------------------------------------------------------------------------------------------------
% Finally, everything is set up. Here we go...
%
\begin{document}
\begin{slide}
%-----------------------------------------------------------------------------------------------------------------
% Autor: Stephan Lehmke <Stephan.Lehmke@cs.uni-dortmund.de>
%
% v0.0.1 Aug 12, 2000: First version for the pre-alpha release of TeXPower.
%
% v0.0.2 Apr 29, 2003: Adapted to TeXPower v0.0.9c.
%
%-----------------------------------------------------------------------------------------------------------------
%


\centerslidestrue
\title{The \TeX Power bundle\\[2ex]{\normalfont Structured
    rules, box and page backgrounds}}
\author{Stephan Lehmke\\\url{mailto:Stephan.Lehmke@cs.uni-dortmund.de}}
\maketitle

\begin{small}
  This example demonstrates \TeX Power's support for structured rules,
  box and page backgrounds. The usage and parameterization of the
  corresponding commands is documented in the manual. Here, we only
  demonstrate the effects achievable with the parameters.
\end{small}

\pageDuration{0.01}

\parstepwise*%
{%
  \multistep
  {50}
  {%
    \mkfactor{\intensity}{(\value{substep}-1)/49}%
    \colorbetween[\intensity]{stcolor}{pagecolor}{white}%
    \backgroundstyle[startcolor=stcolor,endcolor=white]{vgradient}%
  }%
  \multistep
  {3}
  {%
    \afterstep{\pageDuration{1}}%
    \backgroundstyle[startcolor=pagecolor,endcolor=white,firstgradprogression=\value{substep}]{vgradient}%
  }%
  \multistep
  {10}
  {%
    \afterstep{\pageDuration{0.01}}%
    \mkfactor{\intensity}{(\value{substep}-1)/40}%
    \colorbetween[\intensity]{ecolor}{pagecolor}{white}%
    \backgroundstyle[startcolor=pagecolor,endcolor=ecolor,firstgradprogression=3]{vgradient}%
  }%
  \skipTo{eotitle}
}

\hypertarget{eotitle}{}

\stopAdvancing

\newslide

\renewcommand{\rulestripes}{100}

\newcounter{mstep}

\section{Color Gradients}
\liststepwise[\let\hidestepcontents=\hidesmartignore]
{%
  \concept{Horizontal} \step{or \concept{vertical}; \concept{single}} \step{or \concept{double}.}

  \step
  {%
    Parameters:
    \begin{itemize}
    \item Gradient \concept{start} {\bstep[\value{step}=13]{(and
          \concept{middle})}} and \concept{end} color.

      \step[\value{step}=23]{\item Number of \concept{stripes}.}

      \step[\value{step}=33]{\item \concept{Midpoint} of a double
        gradient.}

      \step[\value{step}=43]{\item Gradient \concept{Progression}}
      \step[\value{step}=56]{\par(independent for double gradients).}
    \end{itemize}
  }
  \vfill
  \steponce[\value{step}=0]{\hgradrule{red}{green}{\linewidth}{5ex}}%
  \steponce[\value{step}=1]{\vgradrule{red}{green}{\linewidth}{5ex}}%
  \steponce[\value{step}=2]{\dblvgradrule{red}{yellow}{green}{\linewidth}{5ex}}%
  \steponce[\value{step}>2\and\value{step}<13]
  {%
    \skipTo{eostartend}\\
    \multistep(\setcounter{mstep}{\value{substep}+2}\ifthenelse{\value{step}=\value{mstep}}){10}
    {%
      \mkfactor{\intensity}{(\value{substep}-1)/9}%
      \colorbetween[\intensity]{scolor}{blue}{red}%
      \colorbetween[\intensity]{ecolor}{yellow}{green}%
      \hgradrule{scolor}{ecolor}{\linewidth}{5ex}%
      \ifthenelse{\value{substep}=10}
      {\hypertarget{eostartend}{}\afterstep{\stopAdvancing}}
      {\afterstep{\pageDuration{0.01}}}%
    }%
  }%
  \steponce[\value{step}>12\and\value{step}<23]
  {%
    \skipTo{eomidcolor}\\
    \multistep(\setcounter{mstep}{\value{substep}+12}\ifthenelse{\value{step}=\value{mstep}}){10}
    {%
      \colorbetween{scolor}{blue}{yellow}%
      \mkfactor{\intensity}{(\value{substep}-1)/9}%
      \colorbetween[\intensity]{ecolor}{red}{scolor}%
      \dblhgradrule{blue}{ecolor}{yellow}{\linewidth}{5ex}%
      \ifthenelse{\value{substep}=10}
      {\hypertarget{eomidcolor}{}\afterstep{\stopAdvancing}}
      {\afterstep{\pageDuration{0.01}}}%
    }%
  }%
  \steponce[\value{step}>22\and\value{step}<33]
  {%
    \skipTo{eostripes}\\
    \multistep(\setcounter{mstep}{\value{substep}+22}\ifthenelse{\value{step}=\value{mstep}}){10}
    {%
      \histogram{\dblhgradrule[][10*\value{substep}]{red}{yellow}{green}{\linewidth}{2.5ex}}\\%
      \dblhgradrule[][10*\value{substep}]{red}{yellow}{green}{\linewidth}{2.5ex}%
      \ifthenelse{\value{substep}=10}
      {\hypertarget{eostripes}{}\afterstep{\stopAdvancing}}
      {\afterstep{\pageDuration{0.5}}}%
    }%
  }%
  \steponce[\value{step}>32\and\value{step}<43]
  {%
    \skipTo{eomidpoint}\\
    \multistep(\setcounter{mstep}{\value{substep}+32}\ifthenelse{\value{step}=\value{mstep}}){10}
    {%
      \mkfactor{\midpoint}{(\value{substep}-1)/9}%
      \histogram{\dblhgradrule[\midpoint]{blue}{red}{yellow}{\linewidth}{2.5ex}}\\%
      \dblhgradrule[\midpoint]{red}{yellow}{green}{\linewidth}{2.5ex}%
      \ifthenelse{\value{substep}=10}
      {\hypertarget{eomidpoint}{}\afterstep{\stopAdvancing}}
      {\afterstep{\pageDuration{0.5}}}%
    }%
  }%
  \steponce[\value{step}>42\and\value{step}<56]
  {%
    \skipTo{eofirstprog}\\
    \multistep(\setcounter{mstep}{\value{substep}+42}\ifthenelse{\value{step}=\value{mstep}}){13}
    {%
      \renewcommand{\rulefirstgradprogression}{\value{substep}-7}%
      \histogram{\hgradrule{red}{green}{\linewidth}{2.5ex}}\\%
      \hgradrule{red}{green}{\linewidth}{2.5ex}%
      \ifthenelse{\value{substep}=13}
      {\hypertarget{eofirstprog}{}\afterstep{\stopAdvancing}}
      {\afterstep{\pageDuration{0.5}}}%
    }%
  }%
  \steponce[\value{step}>55\and\value{step}<69]
  {%
    \skipTo{eosecondprog}\\
    \multistep(\setcounter{mstep}{\value{substep}+55}\ifthenelse{\value{step}=\value{mstep}}){13}
    {%
      \renewcommand{\rulefirstgradprogression}{\value{substep}-7}%
      \renewcommand{\rulesecondgradprogression}{7-\value{substep}}%
      \histogram{\dblhgradrule{red}{green}{red}{\linewidth}{2.5ex}}\\%
      \dblhgradrule{red}{green}{red}{\linewidth}{2.5ex}%
      \ifthenelse{\value{substep}=13}
      {\hypertarget{eosecondprog}{}\afterstep{\stopAdvancing}}
      {\afterstep{\pageDuration{0.5}}}%
    }%
  }%
}

\newslide

Applications of gradients:
\centerslidesfalse
\liststepwise*
{%
  \begin{itemize}
  \item
    \begin{tabular}[t]{@{}l@{}}
      As rules:\\
      \skipTo{eoruledemo}
    \end{tabular}
    \multistep{72}
    {%
      \present
      {%
        \small\renewcommand{\arraystretch}{.9}%
        \begin{tabular}{rp{\totalbarwidth+2em}}
          \multicolumn{2}{c}{\textbf{Compression rates}}\\[2ex]
          \code{compress} & \mkbar[\thesubstep]{51}\\
          \code{gzip -1} & \mkbar[\thesubstep]{62}\\
          \code{gzip -9} & \mkbar[\thesubstep]{66}\\
          \code{bzip2 -1} & \mkbar[\thesubstep]{65}\\
          \code{bzip2 -9} & \mkbar[\thesubstep]{73}
        \end{tabular}%
        }%
      \afterstep{\pageDuration{0.01}}%
      }%
    \step
    {%
      \present
      {%
        \small\renewcommand{\arraystretch}{.9}%
        \begin{tabular}{rp{\totalbarwidth+2em}}
          \multicolumn{2}{c}{\textbf{Compression rates}}\\[2ex]
          \code{compress} & \mkbar{51}\\
          \code{gzip -1} & \mkbar{62}\\
          \code{gzip -9} & \mkbar{66}\\
          \code{bzip2 -1} & \mkbar{65}\\
          \code{bzip2 -9} & \mkbar{73}
        \end{tabular}%
        }%
      \ifthenelse{\boolean{firstactivation}}{\AtShipout{\hypertarget{eoruledemo}{}}\afterstep{\stopAdvancing}}{}%
      }%

  \step
  {%
  \item
    \begin{tabular}[t]{@{}l@{}}
      As box backgrounds:\\
      \skipTo{eoboxdemo}
    \end{tabular}
    \afterstep{\pageDuration{0.01}}%
    \multistep{10}
    {%
      \mkfactor{\intensity}{(\value{substep}-1)/9}%
      \colorbetween[\intensity]{ecolor}{green}{blue}%
      \colorbetween[\intensity]{scolor}{yellow}{green}%
      \colorbetween{mcolor}{scolor}{ecolor}%
      \complementcolor{tcolor}{mcolor}%
      \raisebox{-.5\height}{\hgradbox{scolor}{ecolor}{\Huge\textsf{\textbf{\mbox{{\textcolor{tcolor}{Groovy!}}}}}}}%
      }%
    \multistep{10}
    {%
      \mkfactor{\intensity}{(\value{substep}-1)/9}%
      \colorbetween[\intensity]{ecolor}{yellow}{green}%
      \colorbetween[\intensity]{scolor}{red}{yellow}%
      \colorbetween{mcolor}{scolor}{ecolor}%
      \complementcolor{tcolor}{mcolor}%
      \raisebox{-.5\height}{\hgradbox{scolor}{ecolor}{\Huge\textsf{\textbf{\textcolor{tcolor}{Groovy!}}}}}%
      }%
    \multistep{10}
    {%
      \mkfactor{\intensity}{(\value{substep}-1)/9}%
      \colorbetween[\intensity]{ecolor}{red}{yellow}%
      \colorbetween[\intensity]{scolor}{blue}{red}%
      \colorbetween{mcolor}{scolor}{ecolor}%
      \complementcolor{tcolor}{mcolor}%
      \raisebox{-.5\height}{\hgradbox{scolor}{ecolor}{\Huge\textsf{\textbf{\textcolor{tcolor}{Groovy!}}}}}%
      }%
    \multistep{10}
    {%
      \mkfactor{\intensity}{(\value{substep}-1)/9}%
      \colorbetween[\intensity]{ecolor}{blue}{red}%
      \colorbetween[\intensity]{scolor}{green}{blue}%
      \colorbetween{mcolor}{scolor}{ecolor}%
      \complementcolor{tcolor}{mcolor}%
      \raisebox{-.5\height}{\hgradbox{scolor}{ecolor}{\Huge\textsf{\textbf{\textcolor{tcolor}{Groovy!}}}}}%
      }%
    \step
    {%
      \colorbetween{mcolor}{green}{blue}%
      \complementcolor{tcolor}{mcolor}%
      \raisebox{-.5\height}{\hgradbox{green}{blue}{\Huge\textsf{\textbf{\textcolor{tcolor}{Groovy!}}}}}%
      \ifthenelse{\boolean{firstactivation}}{\AtShipout{\hypertarget{eoboxdemo}{}}\afterstep{\stopAdvancing}}{}%
      }%
    }

  \step
  {%
  \item As page backgrounds.  \skipTo{eobgnddemo}
    \colorbetween[.22]{ecolor}{pagecolor}{white}%
    \afterstep{\pageDuration{0.01}}%
    \multistep{20}
    {%
      \backgroundstyle
      [%
        startcolor=pagecolor,endcolor=ecolor,firstgradprogression=3,
        rightpanelwidth=\TPpagewidth*\real{.025}*\value{substep},rightpanelcolor=pagecolor,
        leftpanelwidth=\TPpagewidth*\real{.025}*\value{substep},leftpanelcolor=pagecolor,
        toppanelheight=\TPpageheight*\real{.025}*\value{substep},toppanelcolor=pagecolor,
        bottompanelheight=\TPpageheight*\real{.025}*\value{substep},bottompanelcolor=pagecolor%
      ]{vgradient}%
      }%
    \multistep{20}
    {%
      \backgroundstyle
      [%
        startcolordef={rgb}{0.4,0,0},endcolordef={rgb}{0,0.4,0},firstgradprogression=3,
        rightpanelwidth=\TPpagewidth*\real{.025}*(20-\value{substep}),rightpanelcolor=pagecolor,
        leftpanelwidth=\TPpagewidth*\real{.025}*(20-\value{substep}),leftpanelcolor=pagecolor,
        toppanelheight=\TPpageheight*\real{.025}*(20-\value{substep}),toppanelcolor=pagecolor,
        bottompanelheight=\TPpageheight*\real{.025}*(20-\value{substep}),bottompanelcolor=pagecolor%
      ]{vgradient}%
      }%
    }
  \end{itemize}
  }
\hypertarget{eobgnddemo}{}
\end{slide}



\begin{slide}[\slidewidth-40mm,\slideheight-40mm]
\renewcommand{\sliderightmargin}{45mm}%
\renewcommand{\slidetopmargin}{25mm}%
\renewcommand{\slidebottommargin}{25mm}%
\colorbetween[.22]{ecolor}{pagecolor}{white}%
\backgroundstyle[startcolor=pagecolor,endcolor=ecolor,firstgradprogression=3]{vgradient}%

\liststepwise
{%
  Special parameters for page backgrounds:
  \begin{itemize}
  \item Leave space for panels, headers and footers.\\
    \skipTo{eopaneldemo}
    \afterstep{\pageDuration{0.01}}%
    \multistep{20}
    {%
      \backgroundstyle
      [%
      startcolor=pagecolor,endcolor=ecolor,firstgradprogression=3,
      toppanelheight=.1\semcm*\value{substep},toppanelcolor=black,
      bottompanelheight=.1\semcm*\value{substep}%
      ]{vgradient}%
      }%
    \multistep{20}
    {%
      \backgroundstyle
      [%
      startcolor=pagecolor,endcolor=ecolor,firstgradprogression=3,
      rightpanelwidth=.2\semcm*\value{substep},rightpanelcolordef={rgb}{0,0.4,0.6},
      toppanelheight=2\semcm,toppanelcolor=black,
      bottompanelheight=2\semcm%
      ]{vgradient}%
      }%
  \end{itemize}
}

\hypertarget{eopaneldemo}{}
\end{slide}
\end{document}
\endinput
%%
%% End of file `bgndexample.tex'.
