% \iffalse meta-comment
%
% --------------------------------------------------------------
% Package tpslifonts - part of TeXPower bundle - configure 
% presentation fonts. Copyright (C) 1999-2004 Stephan Lehmke
%
% This program is free software; you can redistribute it and/or
% modify it under the terms of the GNU General Public License
% as published by the Free Software Foundation; either version 2
% of the License, or (at your option) any later version.
%
% This program is distributed in the hope that it will be useful,
% but WITHOUT ANY WARRANTY; without even the implied warranty of
% MERCHANTABILITY or FITNESS FOR A PARTICULAR PURPOSE.  See the
% GNU General Public License for more details.
% --------------------------------------------------------------
%
% tpslifonts.dtx,v 1.2 2004/07/27 11:30:39 hansfn Exp
%
% \fi
%
% \iffalse
%
%<*driver>
\ProvidesFile{tpslifonts.dtx}
%</driver>
%<tpslifonts>\NeedsTeXFormat{LaTeX2e}
%<tpslifonts>\ProvidesPackage{tpslifonts}
%<*tpslifonts>
    [2004/07/27 v0.6d Configure presentation fonts.]
%</tpslifonts>
%
%<*driver>
\documentclass{ltxdoc}
\EnableCrossrefs
\CodelineIndex
\RecordChanges
\begin{document}
  \DocInput{tpslifonts.dtx}
\end{document}
%</driver>
%
% \fi
%
% \CheckSum{1030}
%
% \CharacterTable
%  {Upper-case    \A\B\C\D\E\F\G\H\I\J\K\L\M\N\O\P\Q\R\S\T\U\V\W\X\Y\Z
%   Lower-case    \a\b\c\d\e\f\g\h\i\j\k\l\m\n\o\p\q\r\s\t\u\v\w\x\y\z
%   Digits        \0\1\2\3\4\5\6\7\8\9
%   Exclamation   \!     Double quote  \"     Hash (number) \#
%   Dollar        \$     Percent       \%     Ampersand     \&
%   Acute accent  \'     Left paren    \(     Right paren   \)
%   Asterisk      \*     Plus          \+     Comma         \,
%   Minus         \-     Point         \.     Solidus       \/
%   Colon         \:     Semicolon     \;     Less than     \<
%   Equals        \=     Greater than  \>     Question mark \?
%   Commercial at \@     Left bracket  \[     Backslash     \\
%   Right bracket \]     Circumflex    \^     Underscore    \_
%   Grave accent  \`     Left brace    \{     Vertical bar  \|
%   Right brace   \}     Tilde         \~}
%
%
% \GetFileInfo{tpslifonts.dtx}
%
% \DoNotIndex{\newcommand,\newenvironment}
%
% \title{The \textsf{tpslifonts} package\thanks{This document
%   corresponds to \textsf{tpslifonts}~\fileversion, dated \filedate.}}
% \author{ Stephan Lehmke \\ \texttt{Stephan.Lehmke@cs.uni-dortmund.de}}
%
% \maketitle
%
% \tableofcontents
%
% \section{Introduction}
%
% Presentations to be displayed `online' with a video beamer have special needs
% concerning font configuration owing to low `screen' resolution and bad
% contrast caused by possibly bad light conditions combined with color
% highlighting.
%
% This package tries to cater to these needs by offering a holistic
% configuration of all document fonts, including text, typewriter, and math
% fonts. Special features are `smooth scaling' of Type1 fonts and careful design
% size selection for optimal readability.
%
% The default is to use ``lcmss'', also called ``slifonts'' for text and the
% usual cmtt and cmm for typewriter and math. There's a large variety of
% options and parameters for choosing different text and math fonts, all from
% the large ``cm'' family of fonts, which has the advantage that readability can
% be enhanced by scaling small design sizes.
% A couple of additional math fonts like ams and doublestroke fonts are
% supported.
%
% For some of the options to yield satisfying results, it is neccessary to
% install additional (free) Type1 fonts on your system.
% There's no intention to support other font families like the typical
% ``psnfss'' PostScript fonts, as they usually don't come in different design
% sizes, making the effort of tuning them for viewing futile. If you wish to use
% such a font, load it with the usual packages.
%
% This package is part of the TeXPower bundle, to be found at\\ |http://texpower.sourceforge.net/|
%
% \section{User interface}
%
% The package doesn't define any user-level commands; the font
% configuration is done entirely by package options.
% 
% Furthermore, the accompanying example document slifontsexample gives a
% lot of information about the fonts configured and some examples of
% math typesetting. Please read the comments in slifontsexample.tex
% carefully and try out some of the configuration variants described
% therein. 
% 
% \section{Description of package options}
%   
% \subsection{Text font options}
% 
% The default is to use \SliTeX\ computer modern sans serif, an eminently
% readable font and my personal favourite.\\
% 
% \begin{tabular}{@{}l@{\hspace{1cm}}l}
%   Option   & Meaning\\ \hline
%   cmr	     & Set computer modern roman text fonts.\\
%   cmfib    & Set computer modern roman Fibonacci text fonts.\\
%   cmss     & Set computer modern sans serif text fonts.\\
%   cmbright & Set computer modern bright text fonts.\\
%   concrete & Set concrete text fonts.
% \end{tabular}
% 
% \subsection{Design size bounding}
% 
% Usually design sizes are `bounded' by 8pt, that is, if an 8pt (or
% % lower) design size of a font exists, it will be scaled to larger sizes
% even if larger design sizes exist, because large design sizes tend to
% be less readable at low resolutions. For sizes below 8pt, smaller
% design sizes are used if available.  
% 
% The limit of 8pt is adjustable (downwards) by options. \\
% 
% \begin{tabular}{@{}l@{\hspace{1cm}}l}
%   Option   & Meaning\\ \hline
%   scale7pt & Use 7pt as a bound.\\
%   scale6pt & Use 6pt as a bound.\\
%   scale5pt & Use 5pt as a bound.
% \end{tabular}
% 
% \subsection{Font scaling}
% 
% The default lcmss text fonts have very high `small letters', making
% normal-size fonts look small and difficult to read. There are options
% for scaling up math and typewriter fonts slightly so they harmonize a
% little more with text fonts.\\
%
% \begin{tabular}{@{}l@{\hspace{1cm}}l}
%   Option      & Meaning\\ \hline
%   scaleupmath & scale up math fonts\\
%   scaleuptt   & scale up typewriter fonts
% \end{tabular}
% 
% \subsection{Math fonts}
% 
% Because the availability of different fonts in different encodings is
% not really homogenous, we differentiate between the font used for
% operator names (plus upper case greek, digits and some symbols) and
% the font used for other symbols. In some cases (math extension font),
% the choices are quite limited anyway.
% 
% The default is to use cmr for operators and such and the standard cmm
% and relatives for symbols. The concrete option for concrete text fonts
% selects concrete math fonts by default, but as these don't seem to
% exist in Type1 format currently, it's recommended to combine concrete
% text with Euler math.\\
% 
% \begin{tabular}{@{}l@{\hspace{1cm}}p{0.6\textwidth}}
%   Option       & Meaning \\ \hline
%   eulermath    & Use Euler math fonts. This implies the option textops
% 	           (see below). \\
%   eulerdigits  & Take digits also from the Euler fonts (default: the
%  	           text font used). Can be used only in conjunction with
%	           the eulermath option. \\
%   cmbrightmath & Use cmbright math fonts. \\
%   textops      & Take operator names, upper case greek, digits and some
% 	           symbols from the text font rather than from some font
% 	           matching the math font (default for Euler math).
% \end{tabular}
%
% \StopEventually{\PrintChanges\PrintIndex}
%
% \section{Implementation}
%
% \changes{v0.1}{2001/08/10}{First version for the pre-alpha release of
% texpower.}
%
% \changes{v0.2}{2001/11/17}{Added support for stmaryrd fonts.}
%
% \changes{v0.3}{2002/10/18}{Rewritten to go without .fd files.  Euler package
% dependence changed to eulervm.  Added support for ams, lasy, wasysym, rsfs,
% dstroke fonts.}
%
% \changes{v0.4}{2003/03/25}{Added support for cmbright math.}
%
% \changes{v0.5}{2003/03/26}{Added support for T1 encoding.}
%
% \changes{v0.6}{2003/05/29}{Change of attitude: I found out how desolate the
% situation of most presentation packages is wrt fonts. Hence, instead of
% trying to be as minimally invasive as possible, tpslifonts will try to hijack
% every single font which might possibly be used in a presentation and replace
% it by something sensible.  No `private' ``lcmr'' to replace ``cmr'' any more
% - cmr is redefined no matter what.  Now configures (almost) all fonts type1cm
% and type1ec do, plus concrete, cmbright, and euler fonts.  Added options
% ``scale5pt'', ``scale6pt'', ``scale7pt'' to bound the design sizes of fonts
% used.  Added options ``cmr'', ``cmss'', ``cmbright'' and ``concrete'' to
% choose text fonts different from lcmss.  Improved handling of option
% incompatibilities.}
%
% \changes{v0.6a}{2003/06/02}{Removed some font warnings by selecting `silent'
% forms of declaration. }
%
% \changes{v0.6b}{2003/07/19}{A small change making `cmbright' AMS fonts
% useable before \cs{begin\{document\}}. }
%
% \changes{v0.6c}{2003/09/23}{Added the option ``cmfib''.}
%
% \changes{v0.6d}{2004/07/27}{Moved to dtx format. No other code changes.}
%
%    \begin{macrocode}
\RequirePackage{ifthen}
%    \end{macrocode}
%
% \subsection{Options and general setup}
%
% Text fonts. The default is to use ``slifonts'' (lcmss).
%    \begin{macrocode}
\newcommand{\TPSFTextfont}{lcmss}
%    \end{macrocode}
% The option cmr sets ``computer modern roman'' text fonts.
%    \begin{macrocode}
\DeclareOption{cmr}{\renewcommand{\TPSFTextfont}{cmr}}
%    \end{macrocode}
% The option cmfib sets ``computer modern roman Fibonacci'' text fonts.
%    \begin{macrocode}
\DeclareOption{cmfib}{\renewcommand{\TPSFTextfont}{cmfib}}
%    \end{macrocode}
% The option cmss sets ``computer modern roman sans serif'' text fonts.
%    \begin{macrocode}
\DeclareOption{cmss}{\renewcommand{\TPSFTextfont}{cmss}}
%    \end{macrocode}
% The option cmbright sets ``computer modern bright'' text fonts.
%    \begin{macrocode}
\DeclareOption{cmbright}{\renewcommand{\TPSFTextfont}{cmbr}}
%    \end{macrocode}
% The option concrete sets ``concrete'' text fonts.
%    \begin{macrocode}
\DeclareOption{concrete}
{%
  \renewcommand{\TPSFTextfont}{ccr}%
  \renewcommand{\TPSFMathfont}{ccm}%
  \renewcommand{\TPSFOperatorfont}{ccr}%
}
%    \end{macrocode}
%
% \subsubsection{Design size selection} 
%
% Usually design sizes are `bounded' by 8pt, that is, if
% an 8pt (or lower) design size of a font exists, it will be scaled to larger
% sizes even if larger design sizes exist, to enhance readability at low
% resolutions. For sizes below 8pt, smaller design sizes are used if available. 
%    \begin{macrocode}
\providecommand{\TPSFMaxDesignSize}{8}
%    \end{macrocode}
% The option scale5pt causes scaled 5pt design sizes to be used wherever
% possible. 
%    \begin{macrocode}
\DeclareOption{scale5pt}{\renewcommand{\TPSFMaxDesignSize}{5}}
%    \end{macrocode}
% The option scale6pt causes scaled 6pt design sizes to be used wherever
% possible (for font sizes above 6pt). 
%    \begin{macrocode}
\DeclareOption{scale6pt}{\renewcommand{\TPSFMaxDesignSize}{6}}
%    \end{macrocode}
% The option scale7pt causes scaled 7pt design sizes to be used wherever
% possible (for font sizes above 7pt). 
%    \begin{macrocode}
\DeclareOption{scale7pt}{\renewcommand{\TPSFMaxDesignSize}{7}}
%    \end{macrocode}
%
% lcmss text fonts have very high `small letters', making normal-size fonts look
% small and difficult to read.  
% The option scaleupmath (scaleuptt) will scale up math (typewriter) fonts
% slightly so they harmonize a little more with text fonts.
% These options do not make much sense with other text fonts, hence a warning
% (but no error message) is generated to remind the user they might have
% overlooked something.
%    \begin{macrocode}
\newboolean{TPSFscaleup}
\DeclareOption{scaleupmath}
{%
  \providecommand\TPSFmathscale{1.15}%
  \providecommand\TPSFeulerscale{1.1}%
  \providecommand\TPSFcmbrscale{1.1}%
  \setboolean{TPSFscaleup}{true}%
}

\DeclareOption{scaleuptt}
{\providecommand\TPSFttscale{1.2}\setboolean{TPSFscaleup}{true}}
%    \end{macrocode}
%
% \subsubsection{Math fonts}
%
% Because the availability of different fonts in different encodings is not
% really homogenous, we differentiate between the font used for Operators (plus
% upper case greek, digits and some symbols) and the font used for other
% symbols. In some cases (math extension font), the choices are quite limited
% anyway. 
%
% The default is to use cmr for operators and such and the standard cmm and
% relatives for symbols.
%    \begin{macrocode}
\newcommand{\TPSFMathfont}{cmm}

\newcommand{\TPSFOpF@cmm}{cmrm}

\newcommand{\TPSFOpF@ccm}{ccr}

\newcommand{\TPSFOperatorfont}{cmr}
%    \end{macrocode}
% The option textops makes the operator font follow the text font. 
%    \begin{macrocode}
\newboolean{TPSFtextops}
\DeclareOption{textops}{\setboolean{TPSFtextops}{true}}
%    \end{macrocode}
% The option eulermath sets euler math fonts (slightly adapted from the eulervm 
% package to match lcmss better).  
% When using this option, the operator font automatically follows the text
% font. 
%    \begin{macrocode}
\DeclareOption{eulermath}
{\renewcommand{\TPSFMathfont}{euler}\setboolean{TPSFtextops}{true}}
%    \end{macrocode}
% The option eulerdigits works only in conjunction with eulermath and sets the
% euler-digits option of the eulervm package.   
%    \begin{macrocode}
\newboolean{TPSFeulerdigits}
\DeclareOption{eulerdigits}
{%
  \setboolean{TPSFeulerdigits}{true}%
  \PassOptionsToPackage{euler-digits}{eulervm}%
}
%    \end{macrocode}
% The option cmbrightmath sets cmbright math fonts (slightly adapted from the
% cmbright package to match lcmss better).   
%    \begin{macrocode}
\DeclareOption{cmbrightmath}{\renewcommand{\TPSFMathfont}{cmbrm}}

\newcommand{\TPSFOpF@cmbrm}{cmbrmt}
%    \end{macrocode}

%
% \subsubsection{Process options}
%    \begin{macrocode}
\ProcessOptions
%    \end{macrocode}
%
% Option postprocessing.
%
% Parameters |\TPSFmathscale| and such can be (re)defined from outside the
% tpslifonts package to adjust scaling. If any of these scaling parameters is
% defined before the tpslifonts package is loaded, there is no need to give the
% respective option.
% Here, we test whether the parameters have been provided (either by defining
% them independently or by giving one of the ``scaleup...'' options), and define
% the `real' scaling parameter for font definition accordingly.
%    \begin{macrocode}
\@ifundefined{TPSFmathscale}
{\let\TPSFmathscale\@undefined\let\mathscale@TPSF\empty}
{\def\mathscale@TPSF{[\TPSFmathscale]}}

\@ifundefined{TPSFeulerscale}
{\let\TPSFeulerscale\@undefined\let\eulerscale@TPSF\empty}
{\def\eulerscale@TPSF{[\TPSFeulerscale]}}

\@ifundefined{TPSFcmbrscale}
{\let\TPSFcmbrscale\@undefined\let\cmbrscale@TPSF\empty}
{\def\cmbrscale@TPSF{[\TPSFcmbrscale]}}

\@ifundefined{TPSFttscale}
{\let\TPSFttscale\@undefined\let\ttscale@TPSF\empty}
{\def\ttscale@TPSF{[\TPSFttscale]}}
%    \end{macrocode}
%
% As mentioned above, the ``scaleup...'' options make sense really only if lcmss
% text fonts are used. Otherwise, we give a warning.
%    \begin{macrocode}
\ifthenelse{\boolean{TPSFscaleup}\and\not\equal{\TPSFTextfont}{lcmss}}
{%
  \PackageWarning{tpslifonts}
  {Using the ``scaleup'' Options makes sense only\MessageBreak 
    when lcmss text fonts are used.\MessageBreak 
    Watch closely whether the results of scaling math\MessageBreak 
    and/or typewriter fonts together with ``\TPSFTextfont''\MessageBreak 
    text fonts is what you intended to achieve.%
  }%
}
{}
%    \end{macrocode}
%
% We need to set the math operator font corresponing to the settings of the
% TPSFtextops switch, the text and math fonts.
%    \begin{macrocode}
\ifthenelse{\boolean{TPSFtextops}}
{\let\TPSFOperatorfont\TPSFTextfont}
{%
  \expandafter\let\expandafter\TPSFOperatorfont
  \csname TPSFOpF@\TPSFMathfont\endcsname
}
%    \end{macrocode}
%
% The cmbright package also sets text fonts, so if cmbright math is requested,
% we load the cmbright package first and later overwrite the text font settings
% (if no cmbright text fonts were requested).
%    \begin{macrocode}
\ifthenelse{\equal{\TPSFMathfont}{cmbrm}}
{%
  \RequirePackage[standard-baselineskips]{cmbright}%
}
%    \end{macrocode}
%
% \subsection{Font shape configuration}

% We need a couple of macros to help in configuring fonts.
% To increase readability in presentations, design size selection is
% restricted. Without any options, the largest design size used is 8pt. The
% options scale5pt ... scale7pt impose further constraints.
% Another restriction is caused by the availability of design sizes. Not all
% fonts exist in all design sizes. If no design size exists at all in the
% requested range (i.e. the font exists only in size 10), then the smallest
% available size has to be chosen no matter what.

% |\FilterSizes@TPSF<name>{<list>}| filters out all sizes (represented as
% nonnegative integer numbers) from the comma-separated <list> which are above
% the threshold in |\TPSFMaxDesignSize| and stores the result in the macro
% <name>. To avoid inconsistencies if all sizes exceed |\TPSFMaxDesignSize|, the
% first size is always passed.
%    \begin{macrocode}
\newcommand{\FilterSizes@TPSF}[2]{\expandafter\@FilterSizes@TPSF#2,\@nil#1}

\def\@FilterSizes@TPSF#1,#2\@nil#3%
{%
  \def#3{#1}%
  \ifx\empty#2\empty\else\@@FilterSizes@TPSF#2\@nil#3\fi
}

\def\@@FilterSizes@TPSF#1,#2\@nil#3%
{%
  \ifnum#1>\TPSFMaxDesignSize\relax
   \else
    \expandafter\def\expandafter#3\expandafter{#3,#1}%
    \ifx\empty#2\empty\else\@@FilterSizes@TPSF#2\@nil#3\fi
  \fi
}
%    \end{macrocode}
% \begin{macro}{\DeclareFontShape@TPSF}
% |\DeclareFontShape@TPSF[<scale>]|\\
% |    {<enc>}{<fam>}{<ser>}{<shp>}{<name>}{<sizes>}|
% declares a font with presentation-friendly design size selection (which
% means all design sizes above |\TPSFMaxDesignSize| are filtered out. The actually
% existing design sizes have to be provided in |<sizes>| as a comma-separated list
% of numbers. If |<s>| is one of the sizes, it is expected that the corresponding
% font name can be constructed as |<name><s>|.
% If the optional argument |<scale>| is given, it has to be of the form
% |[<factor>]|, where |<factor>| specifies the scaling of the font (see the
% options ``scaleupmath'' and ``scaleuptt'' above).
%    \begin{macrocode}
\newcommand{\DeclareFontShape@TPSF}[7][]
{%
  \FilterSizes@TPSF\sizes@TPSF{#7}%
  \let\sizedecl@TPSF\empty
  \def\fontname@TPSF{#6}%
  \def\scale@TPSF{#1}%
  \expandafter\parsesizes@TPSF\sizes@TPSF,,\@nil
  \DeclareFontShape{#2}{#3}{#4}{#5}{\sizedecl@TPSF}{}%
}
%    \end{macrocode}
% \end{macro}
%    \begin{macrocode}

\def\parsesizes@TPSF#1,#2\@nil
{%
  \let\prevsize@TPSF\empty
  \def\currsize@TPSF{#1}%
  \@parsesizes@TPSF#2\@nil
}

\def\@parsesizes@TPSF#1,#2\@nil
{%
  \edef\sizedecl@TPSF{\sizedecl@TPSF\space<\prevsize@TPSF-#1>s*\scale@TPSF \fontname@TPSF\currsize@TPSF}%
  \def\currsize@TPSF{#1}%
  \let\prevsize@TPSF\currsize@TPSF
  \ifx\currsize@TPSF\empty\else\@parsesizes@TPSF#2\@nil\fi
}
%    \end{macrocode}
%
% \subsection{Declaration of font shapes}
%
% The following is copied more or less verbatim from type1cm.sty,
% type1ec.sty, and other sources of font definitions for e.g. cmbright fonts. 
% For fonts following the usual naming conventions, we use
% |\DeclareFontShape@TPSF| instead of |\DeclareFontShape| to implement design size
% bounding.   
% For T1 ec fonts, the same is achieved by appropriate redefinition of
% |\EC@family|. 
%    \begin{macrocode}
\begingroup
\nfss@catcodes
%    \end{macrocode}
%
% \subsubsection{Text fonts for OT1 encoding}
%
% computer modern roman.
%    \begin{macrocode}
\DeclareFontFamily{OT1}{cmr}{\hyphenchar\font=45}

\DeclareFontShape@TPSF{OT1}{cmr}{m}{n}{cmr}{5,6,7,8,9,10,12,17}

\DeclareFontShape@TPSF{OT1}{cmr}{m}{sl}{cmsl}{8,9,10,12}

\DeclareFontShape@TPSF{OT1}{cmr}{m}{it}{cmti}{7,8,9,10,12}

\DeclareFontShape@TPSF{OT1}{cmr}{m}{sc}{cmcsc}{9,10}

\DeclareFontShape@TPSF{OT1}{cmr}{m}{ui}{cmu}{10}

\DeclareFontShape@TPSF{OT1}{cmr}{b}{n}{cmb}{10}

\DeclareFontShape@TPSF{OT1}{cmr}{bx}{n}{cmbx}{5,6,7,8,9,10,12}

\DeclareFontShape@TPSF{OT1}{cmr}{bx}{sl}{cmbxsl}{10}

\DeclareFontShape@TPSF{OT1}{cmr}{bx}{it}{cmbxti}{10}

\DeclareFontShape{OT1}{cmr}{bx}{ui}
      {<->ssub * cmr/m/ui}{}

\DeclareFontFamily{OT1}{cmfr}{\hyphenchar\font=45}

\DeclareFontShape{OT1}{cmfr}{m}{n}{%
        <->s*     cmff10
    }{}
\DeclareFontShape{OT1}{cmfr}{m}{it}{
        <->s*     cmfi10
    }{}

\DeclareFontFamily{OT1}{cmdh}{\hyphenchar\font=45}

\DeclareFontShape{OT1}{cmdh}{m}{n}{
        <->s*     cmdunh10
    }{}

\DeclareFontFamily{OT1}{cmfib}{\hyphenchar\font=45}

\DeclareFontShape{OT1}{cmfib}{m}{n}{
        <->s*     cmfib8
    }{}

\DeclareFontShape{OT1}{cmfib}{m}{sl}
      {<->ssub * cmr/m/sl}{}

\DeclareFontShape{OT1}{cmfib}{m}{it}
      {<->ssub * cmr/m/it}{}

\DeclareFontShape{OT1}{cmfib}{m}{sc}
      {<->ssub * cmr/m/sc}{}

\DeclareFontShape{OT1}{cmfib}{m}{ui}
      {<->ssub * cmr/m/ui}{}

\DeclareFontShape{OT1}{cmfib}{b}{n}
      {<->ssub * cmr/b/n}{}

\DeclareFontShape{OT1}{cmfib}{bx}{n}
      {<->ssub * cmr/bx/n}{}

\DeclareFontShape{OT1}{cmfib}{bx}{sl}
      {<->ssub * cmr/bx/sl}{}

\DeclareFontShape{OT1}{cmfib}{bx}{it}
      {<->ssub * cmr/bx/it}{}

\DeclareFontShape{OT1}{cmfib}{bx}{ui}
      {<->ssub * cmr/m/ui}{}
%    \end{macrocode}
% computer modern sans serif.
%    \begin{macrocode}
\DeclareFontFamily{OT1}{cmss}{\hyphenchar\font=45}

\DeclareFontShape@TPSF{OT1}{cmss}{m}{n}{cmss}{8,9,10,12,17}

\DeclareFontShape{OT1}{cmss}{m}{it}
      {<->sub*cmss/m/sl}{}

\DeclareFontShape@TPSF{OT1}{cmss}{m}{sl}{cmssi}{8,9,10,12,17}

\DeclareFontShape{OT1}{cmss}{m}{sc}
       {<->sub*cmr/m/sc}{}

\DeclareFontShape{OT1}{cmss}{m}{ui}
       {<->sub*cmr/m/ui}{}

\DeclareFontShape@TPSF{OT1}{cmss}{sbc}{n}{cmssdc}{10}

\DeclareFontShape@TPSF{OT1}{cmss}{bx}{n}{cmssbx}{10}

\DeclareFontShape{OT1}{cmss}{bx}{ui}
       {<->sub*cmr/bx/ui}{}
%    \end{macrocode}
% slide computer modern sans serif.
%    \begin{macrocode}
\DeclareFontFamily{OT1}{lcmss}{\hyphenchar\font45 }

\DeclareFontShape@TPSF{OT1}{lcmss}{m}{n}{lcmss}{8}

\DeclareFontShape@TPSF{OT1}{lcmss}{m}{In}{ilcmss}{8}

\DeclareFontShape@TPSF{OT1}{lcmss}{m}{sl}{lcmssi}{8}

\DeclareFontShape@TPSF{OT1}{lcmss}{m}{Isl}{ilcmssi}{8}

\DeclareFontShape{OT1}{lcmss}{m}{it}{%
  <->sub*lcmss/m/sl}{}

\DeclareFontShape{OT1}{lcmss}{m}{Iit}{%
  <->sub*lcmss/m/Isl}{}

\DeclareFontShape@TPSF{OT1}{lcmss}{b}{n}{lcmssb}{8}

\DeclareFontShape@TPSF{OT1}{lcmss}{b}{In}{ilcmssb}{8}

\DeclareFontShape{OT1}{lcmss}{bx}{n}{%
  <->sub*lcmss/b/n}{}

\DeclareFontShape{OT1}{lcmss}{bx}{In}{%
  <->sub*lcmss/b/In}{}

\DeclareFontShape{OT1}{lcmss}{m}{ui}{%
  <->sub*cmr/m/ui}{}

\DeclareFontShape{OT1}{lcmss}{bx}{ui}{%
  <->sub*cmr/m/ui}{}
%    \end{macrocode}
% computer modern bright
%    \begin{macrocode}
\DeclareFontFamily{OT1}{cmbr}{\hyphenchar\font45}

\DeclareFontShape@TPSF{OT1}{cmbr}{m}{n}{cmbr}{8,9,10,17}

\DeclareFontShape@TPSF{OT1}{cmbr}{m}{sl}{cmbrsl}{8,9,10,17}

\DeclareFontShape{OT1}{cmbr}{m}{it}{%
<->ssub*cmbr/m/sl%
}{}

\DeclareFontShape{OT1}{cmbr}{b}{n}{%
<->ssub*cmbr/bx/n%
}{}

\DeclareFontShape@TPSF{OT1}{cmbr}{bx}{n}{cmbrbx}{10}
%    \end{macrocode}
% concrete roman
%    \begin{macrocode}
\DeclareFontFamily{OT1}{ccr}{\hyphenchar\font45 }

\DeclareFontShape@TPSF{OT1}{ccr}{m}{n}{ccr}{5,6,7,8,9,10}

\DeclareFontShape@TPSF{OT1}{ccr}{m}{sl}{ccsl}{9,10}

\DeclareFontShape@TPSF{OT1}{ccr}{m}{it}{ccti}{10}

\DeclareFontShape@TPSF{OT1}{ccr}{m}{sc}{cccsc}{10}

\DeclareFontShape{OT1}{ccr}{bx}{n} {<-> ssub * cmss/bx/n}{}
\DeclareFontShape{OT1}{ccr}{bx}{sl}{<-> ssub * cmss/bx/sl}{}
\DeclareFontShape{OT1}{ccr}{bx}{it}{<-> ssub * cmss/bx/it}{}
% \DeclareFontShape{OT1}{ccr}{bx}{n} {<-> ssub * cmr/bx/n}{}
% \DeclareFontShape{OT1}{ccr}{bx}{sl}{<-> ssub * cmr/bx/sl}{}
% \DeclareFontShape{OT1}{ccr}{bx}{it}{<-> ssub * cmr/bx/it}{}
\DeclareFontShape{OT1}{ccr}{sbc}{n} {<-> ssub * cmss/sbc/n}{}
\DeclareFontShape{OT1}{ccr}{sbc}{sl}{<-> ssub * cmss/sbc/n}{}
\DeclareFontShape{OT1}{ccr}{sbc}{it}{<-> ssub * cmss/sbc/n}{}

\DeclareFontShape@TPSF{OT1}{ccr}{c}{sl}{ccslc}{9}
%    \end{macrocode}
%
% \subsubsection{Text fonts for T1 encoding}
%
%    \begin{macrocode}
\let\ecscale@TPSF\empty

\ifcase\TPSFMaxDesignSize
 \or\or\or\or\or
  \providecommand{\EC@family}[5]{%
    \DeclareFontShape{#1}{#2}{#3}{#4}{%
      <->s*\ecscale@TPSF#50500%
    }{}}
 \or
  \providecommand{\EC@family}[5]{%
    \DeclareFontShape{#1}{#2}{#3}{#4}{%
      <-6>s*\ecscale@TPSF#50500%
      <6->s*\ecscale@TPSF#50600%
    }{}}
 \or
  \providecommand{\EC@family}[5]{%
    \DeclareFontShape{#1}{#2}{#3}{#4}{%
      <-6>s*\ecscale@TPSF#50500%
      <6-7>s*\ecscale@TPSF#50600%
      <7->s*\ecscale@TPSF#50700%
    }{}}
 \or
  \providecommand{\EC@family}[5]{%
    \DeclareFontShape{#1}{#2}{#3}{#4}{%
      <-6>s*\ecscale@TPSF#50500%
      <6-7>s*\ecscale@TPSF#50600%
      <7-8>s*\ecscale@TPSF#50700%
      <8->s*\ecscale@TPSF#50800%
    }{}}
 \or
  \providecommand{\EC@family}[5]{%
    \DeclareFontShape{#1}{#2}{#3}{#4}{%
      <-6>s*\ecscale@TPSF#50500%
      <6-7>s*\ecscale@TPSF#50600%
      <7-8>s*\ecscale@TPSF#50700%
      <8-9>s*\ecscale@TPSF#50800%
      <9->s*\ecscale@TPSF#50900%
    }{}}
 \or
  \providecommand{\EC@family}[5]{%
    \DeclareFontShape{#1}{#2}{#3}{#4}{%
      <-6>s*\ecscale@TPSF#50500%
      <6-7>s*\ecscale@TPSF#50600%
      <7-8>s*\ecscale@TPSF#50700%
      <8-9>s*\ecscale@TPSF#50800%
      <9-10>s*\ecscale@TPSF#50900%
      <10->s*\ecscale@TPSF#51000%
    }{}}
 \or
  \providecommand{\EC@family}[5]{%
    \DeclareFontShape{#1}{#2}{#3}{#4}{%
      <-6>s*\ecscale@TPSF#50500%
      <6-7>s*\ecscale@TPSF#50600%
      <7-8>s*\ecscale@TPSF#50700%
      <8-9>s*\ecscale@TPSF#50800%
      <9-10>s*\ecscale@TPSF#50900%
      <10-10.95>s*\ecscale@TPSF#51000%
      <10.95->s*\ecscale@TPSF#51095%
    }{}}
 \or
  \providecommand{\EC@family}[5]{%
    \DeclareFontShape{#1}{#2}{#3}{#4}{%
      <-6>s*\ecscale@TPSF#50500%
      <6-7>s*\ecscale@TPSF#50600%
      <7-8>s*\ecscale@TPSF#50700%
      <8-9>s*\ecscale@TPSF#50800%
      <9-10>s*\ecscale@TPSF#50900%
      <10-10.95>s*\ecscale@TPSF#51000%
      <10.95-12>s*\ecscale@TPSF#51095%
      <12->s*\ecscale@TPSF#51200%
    }{}}
\fi
%    \end{macrocode}
% european computer modern roman.
%    \begin{macrocode}
%%
%% This is file `t1cmr.fd',
%% generated with the docstrip utility.
%%
%% The original source files were:
%%
%% cmfonts.fdd  (with options: `fd,T1cmr,ec')
%% 
%% This is a generated file.
%% 
%% Copyright 1993 1994 1995 1996 1997 1998 1999 2000 2001 2002 2003 2004 2005
%% The LaTeX3 Project and any individual authors listed elsewhere
%% in this file.
%% 
%% This file was generated from file(s) of the LaTeX base system.
%% --------------------------------------------------------------
%% 
%% It may be distributed and/or modified under the
%% conditions of the LaTeX Project Public License, either version 1.3c
%% of this license or (at your option) any later version.
%% The latest version of this license is in
%%    http://www.latex-project.org/lppl.txt
%% and version 1.3c or later is part of all distributions of LaTeX
%% version 2005/12/01 or later.
%% 
%% This file may only be distributed together with a copy of the LaTeX
%% base system. You may however distribute the LaTeX base system without
%% such generated files.
%% 
%% The list of all files belonging to the LaTeX base distribution is
%% given in the file `manifest.txt'. See also `legal.txt' for additional
%% information.
%% 
%% In particular, permission is granted to customize the declarations in
%% this file to serve the needs of your installation.
%% 
%% However, NO PERMISSION is granted to distribute a modified version
%% of this file under its original name.
%% 
\ProvidesFile{t1cmr.fd}
        [1999/05/25 v2.5h Standard LaTeX font definitions]
\providecommand{\EC@family}[5]{%
  \DeclareFontShape{#1}{#2}{#3}{#4}%
  {<5><6><7><8><9><10><10.95><12><14.4>%
   <17.28><20.74><24.88><29.86><35.83>genb*#5}{}}
\DeclareFontFamily{T1}{cmr}{}
\EC@family{T1}{cmr}{m}{n}{ecrm}
\EC@family{T1}{cmr}{m}{sl}{ecsl}
\EC@family{T1}{cmr}{m}{it}{ecti}
\EC@family{T1}{cmr}{m}{sc}{eccc}
\EC@family{T1}{cmr}{bx}{n}{ecbx}
\EC@family{T1}{cmr}{b}{n}{ecrb}
\EC@family{T1}{cmr}{bx}{it}{ecbi}
\EC@family{T1}{cmr}{bx}{sl}{ecbl}
\EC@family{T1}{cmr}{bx}{sc}{ecxc}
\EC@family{T1}{cmr}{m}{ui}{ecui}
\endinput
%%
%% End of file `t1cmr.fd'.


%%
%% This is file `t1cmfr.fd',
%% generated with the docstrip utility.
%%
%% The original source files were:
%%
%% cmfonts.fdd  (with options: `fd,T1cmfr,ec')
%% 
%% This is a generated file.
%% 
%% Copyright 1993 1994 1995 1996 1997 1998 1999 2000 2001 2002 2003 2004 2005
%% The LaTeX3 Project and any individual authors listed elsewhere
%% in this file.
%% 
%% This file was generated from file(s) of the LaTeX base system.
%% --------------------------------------------------------------
%% 
%% It may be distributed and/or modified under the
%% conditions of the LaTeX Project Public License, either version 1.3c
%% of this license or (at your option) any later version.
%% The latest version of this license is in
%%    http://www.latex-project.org/lppl.txt
%% and version 1.3c or later is part of all distributions of LaTeX
%% version 2005/12/01 or later.
%% 
%% This file may only be distributed together with a copy of the LaTeX
%% base system. You may however distribute the LaTeX base system without
%% such generated files.
%% 
%% The list of all files belonging to the LaTeX base distribution is
%% given in the file `manifest.txt'. See also `legal.txt' for additional
%% information.
%% 
%% In particular, permission is granted to customize the declarations in
%% this file to serve the needs of your installation.
%% 
%% However, NO PERMISSION is granted to distribute a modified version
%% of this file under its original name.
%% 
\ProvidesFile{t1cmfr.fd}
        [1999/05/25 v2.5h Standard LaTeX font definitions]
\providecommand{\EC@family}[5]{%
  \DeclareFontShape{#1}{#2}{#3}{#4}%
  {<5><6><7><8><9><10><10.95><12><14.4>%
   <17.28><20.74><24.88><29.86><35.83>genb*#5}{}}
\DeclareFontFamily{T1}{cmfr}{}
\EC@family{T1}{cmfr}{m}{n}{ecff}
\EC@family{T1}{cmfr}{m}{it}{ecfi}
\endinput
%%
%% End of file `t1cmfr.fd'.


%%
%% This is file `t1cmdh.fd',
%% generated with the docstrip utility.
%%
%% The original source files were:
%%
%% cmfonts.fdd  (with options: `fd,T1cmdh,ec')
%% 
%% This is a generated file.
%% 
%% Copyright 1993 1994 1995 1996 1997 1998 1999 2000 2001 2002 2003 2004 2005
%% The LaTeX3 Project and any individual authors listed elsewhere
%% in this file.
%% 
%% This file was generated from file(s) of the LaTeX base system.
%% --------------------------------------------------------------
%% 
%% It may be distributed and/or modified under the
%% conditions of the LaTeX Project Public License, either version 1.3c
%% of this license or (at your option) any later version.
%% The latest version of this license is in
%%    http://www.latex-project.org/lppl.txt
%% and version 1.3c or later is part of all distributions of LaTeX
%% version 2005/12/01 or later.
%% 
%% This file may only be distributed together with a copy of the LaTeX
%% base system. You may however distribute the LaTeX base system without
%% such generated files.
%% 
%% The list of all files belonging to the LaTeX base distribution is
%% given in the file `manifest.txt'. See also `legal.txt' for additional
%% information.
%% 
%% In particular, permission is granted to customize the declarations in
%% this file to serve the needs of your installation.
%% 
%% However, NO PERMISSION is granted to distribute a modified version
%% of this file under its original name.
%% 
\ProvidesFile{t1cmdh.fd}
        [1999/05/25 v2.5h Standard LaTeX font definitions]
\providecommand{\EC@family}[5]{%
  \DeclareFontShape{#1}{#2}{#3}{#4}%
  {<5><6><7><8><9><10><10.95><12><14.4>%
   <17.28><20.74><24.88><29.86><35.83>genb*#5}{}}
\DeclareFontFamily{T1}{cmdh}{}
\EC@family{T1}{cmdh}{m}{n}{ecdh}
\endinput
%%
%% End of file `t1cmdh.fd'.


%%
%% This is file `t1cmfib.fd',
%% generated with the docstrip utility.
%%
%% The original source files were:
%%
%% cmfonts.fdd  (with options: `fd,T1cmfib,ec')
%% 
%% This is a generated file.
%% 
%% Copyright 1993 1994 1995 1996 1997 1998 1999 2000 2001 2002 2003 2004 2005
%% The LaTeX3 Project and any individual authors listed elsewhere
%% in this file.
%% 
%% This file was generated from file(s) of the LaTeX base system.
%% --------------------------------------------------------------
%% 
%% It may be distributed and/or modified under the
%% conditions of the LaTeX Project Public License, either version 1.3c
%% of this license or (at your option) any later version.
%% The latest version of this license is in
%%    http://www.latex-project.org/lppl.txt
%% and version 1.3c or later is part of all distributions of LaTeX
%% version 2005/12/01 or later.
%% 
%% This file may only be distributed together with a copy of the LaTeX
%% base system. You may however distribute the LaTeX base system without
%% such generated files.
%% 
%% The list of all files belonging to the LaTeX base distribution is
%% given in the file `manifest.txt'. See also `legal.txt' for additional
%% information.
%% 
%% In particular, permission is granted to customize the declarations in
%% this file to serve the needs of your installation.
%% 
%% However, NO PERMISSION is granted to distribute a modified version
%% of this file under its original name.
%% 
\ProvidesFile{t1cmfib.fd}
        [1999/05/25 v2.5h Standard LaTeX font definitions]
\providecommand{\EC@family}[5]{%
  \DeclareFontShape{#1}{#2}{#3}{#4}%
  {<5><6><7><8><9><10><10.95><12><14.4>%
   <17.28><20.74><24.88><29.86><35.83>genb*#5}{}}
\DeclareFontFamily{T1}{cmfib}{}
\EC@family{T1}{cmfib}{m}{n}{ecfb}
\EC@family{T1}{cmfib}{m}{sl}{ecfs}
\endinput
%%
%% End of file `t1cmfib.fd'.


\DeclareFontShape{T1}{cmfib}{m}{it}
      {<->ssub * cmr/m/it}{}

\DeclareFontShape{T1}{cmfib}{m}{sc}
      {<->ssub * cmr/m/sc}{}

\DeclareFontShape{T1}{cmfib}{m}{ui}
      {<->ssub * cmr/m/ui}{}

\DeclareFontShape{T1}{cmfib}{b}{n}
      {<->ssub * cmr/b/n}{}

\DeclareFontShape{T1}{cmfib}{bx}{n}
      {<->ssub * cmr/bx/n}{}

\DeclareFontShape{T1}{cmfib}{bx}{sl}
      {<->ssub * cmr/bx/sl}{}

\DeclareFontShape{T1}{cmfib}{bx}{it}
      {<->ssub * cmr/bx/it}{}

\DeclareFontShape{T1}{cmfib}{bx}{ui}
      {<->ssub * cmr/m/ui}{}
%    \end{macrocode}
% european computer modern roman sans serif.
%    \begin{macrocode}
%%
%% This is file `t1cmss.fd',
%% generated with the docstrip utility.
%%
%% The original source files were:
%%
%% cmfonts.fdd  (with options: `fd,T1cmss,ec')
%% 
%% This is a generated file.
%% 
%% Copyright 1993 1994 1995 1996 1997 1998 1999 2000 2001 2002 2003 2004 2005
%% The LaTeX3 Project and any individual authors listed elsewhere
%% in this file.
%% 
%% This file was generated from file(s) of the LaTeX base system.
%% --------------------------------------------------------------
%% 
%% It may be distributed and/or modified under the
%% conditions of the LaTeX Project Public License, either version 1.3c
%% of this license or (at your option) any later version.
%% The latest version of this license is in
%%    http://www.latex-project.org/lppl.txt
%% and version 1.3c or later is part of all distributions of LaTeX
%% version 2005/12/01 or later.
%% 
%% This file may only be distributed together with a copy of the LaTeX
%% base system. You may however distribute the LaTeX base system without
%% such generated files.
%% 
%% The list of all files belonging to the LaTeX base distribution is
%% given in the file `manifest.txt'. See also `legal.txt' for additional
%% information.
%% 
%% In particular, permission is granted to customize the declarations in
%% this file to serve the needs of your installation.
%% 
%% However, NO PERMISSION is granted to distribute a modified version
%% of this file under its original name.
%% 
\ProvidesFile{t1cmss.fd}
        [1999/05/25 v2.5h Standard LaTeX font definitions]
\providecommand{\EC@family}[5]{%
  \DeclareFontShape{#1}{#2}{#3}{#4}%
  {<5><6><7><8><9><10><10.95><12><14.4>%
   <17.28><20.74><24.88><29.86><35.83>genb*#5}{}}
\DeclareFontFamily{T1}{cmss}{}
\EC@family{T1}{cmss}{m}{n}{ecss}
\EC@family{T1}{cmss}{m}{sl}{ecsi}
\EC@family{T1}{cmss}{m}{it}{ecsi}
\EC@family{T1}{cmss}{bx}{n}{ecsx}
\EC@family{T1}{cmss}{bx}{it}{ecso}
\EC@family{T1}{cmss}{bx}{sl}{ecso}
%%%%%% Font/shape undefined, therefore substituted
\DeclareFontShape{T1}{cmss}{m}{sc}%
       {<->sub*cmr/m/sc}{}
%%%%%%% semibold condensed series
\DeclareFontShape{T1}{cmss}{sbc}{n}{%
      <5><6><7><8><9><10><10.95><12><14.4><17.28>%
      <20.74><24.88>%
ecssdc10
       }{}
\endinput
%%
%% End of file `t1cmss.fd'.


\DeclareFontShape@TPSF{T1}{cmss}{sbc}{n}{ecssdc}{10}
%    \end{macrocode}
% european slide computer modern sans serif.
%    \begin{macrocode}
\DeclareFontFamily{T1}{lcmss}{}
\DeclareFontShape@TPSF{T1}{lcmss}{m}{n}{eclq}{8}

\DeclareFontShape@TPSF{T1}{lcmss}{m}{In}{ieclq}{8}

\DeclareFontShape@TPSF{T1}{lcmss}{m}{sl}{ecli}{8}

\DeclareFontShape@TPSF{T1}{lcmss}{m}{Isl}{iecli}{8} 

\DeclareFontShape{T1}{lcmss}{m}{it}{%
  <->sub*lcmss/m/sl}{}

\DeclareFontShape{T1}{lcmss}{m}{Iit}{%
  <->sub*lcmss/m/Isl}{}

\DeclareFontShape@TPSF{T1}{lcmss}{bx}{n}{eclb}{8}

\DeclareFontShape@TPSF{T1}{lcmss}{bx}{In}{ieclb}{8}

\DeclareFontShape@TPSF{T1}{lcmss}{bx}{sl}{eclo}{8}

\DeclareFontShape@TPSF{T1}{lcmss}{bx}{Isl}{ieclo}{8}

\DeclareFontShape{T1}{lcmss}{bx}{it}{%
      <->sub*lcmss/bx/sl}{}

\DeclareFontShape{T1}{lcmss}{bx}{Iit}{%
      <->sub*lcmss/bx/Isl}{}

\DeclareFontShape{T1}{lcmss}{b}{n}{%
  <->sub*lcmss/bx/n}{}

\DeclareFontShape{T1}{lcmss}{m}{ui}{%
  <->sub*cmr/m/ui}{}

\DeclareFontShape{T1}{lcmss}{bx}{ui}{%
  <->sub*cmr/m/ui}{}
%    \end{macrocode}
% european computer modern bright
%    \begin{macrocode}
\DeclareFontFamily{T1}{cmbr}{}

\DeclareFontShape@TPSF{T1}{cmbr}{m}{n}{ebmr}{8,9,10,17}

\DeclareFontShape@TPSF{T1}{cmbr}{m}{sl}{ebmo}{8,9,10,17}

\DeclareFontShape{T1}{cmbr}{m}{it}{%
<->ssub*cmbr/m/sl%
}{}

\DeclareFontShape@TPSF{T1}{cmbr}{sb}{n}{ebsr}{8,9,10,17}

\DeclareFontShape@TPSF{T1}{cmbr}{sb}{sl}{ebso}{8,9,10,17}

\DeclareFontShape{T1}{cmbr}{sb}{it}{%
<->ssub*cmbr/sb/sl%
}{}

\DeclareFontShape{T1}{cmbr}{b}{n}{%
<->ssub*cmbr/bx/n%
}{}

\DeclareFontShape{T1}{cmbr}{bx}{n}{%
<->sub*cmbr/sb/n%
}{}
%    \end{macrocode}
% european concrete roman
%    \begin{macrocode}
\DeclareFontFamily{T1}{ccr}{}

\DeclareFontShape@TPSF{T1}{ccr}{m}{n}{eorm}{5,6,7,8,9,10}

\DeclareFontShape@TPSF{T1}{ccr}{m}{sl}{eosl}{5,6,7,8,9,10}

\DeclareFontShape@TPSF{T1}{ccr}{m}{it}{eoti}{10}

\DeclareFontShape@TPSF{T1}{ccr}{m}{sc}{eocc}{10}

\DeclareFontShape{T1}{ccr}{bx}{n}{%
<->ssub*cmss/bx/n}{}

\DeclareFontShape{T1}{ccr}{bx}{sl}{%
<->ssub*cmss/bx/sl}{}

\DeclareFontShape{T1}{ccr}{bx}{it}{%
<->ssub*cmss/bx/it}{}

% \DeclareFontShape{T1}{ccr}{bx}{n}{%
% <->ssub*cmr/bx/n}{}

% \DeclareFontShape{T1}{ccr}{bx}{sl}{%
% <->ssub*cmr/bx/sl}{}

% \DeclareFontShape{T1}{ccr}{bx}{it}{%
% <->ssub*cmr/bx/it}{}

\DeclareFontShape{T1}{ccr}{sbc}{n}{%
 <->ssubf*ecssdc10}{}
%    \end{macrocode}
%
% \subsection{Typewriter fonts for OT1 encoding}
%
% computer modern typewriter.
%    \begin{macrocode}
\DeclareFontFamily{OT1}{cmtt}{\hyphenchar\font\m@ne}

\DeclareFontShape@TPSF[\ttscale@TPSF]{OT1}{cmtt}{m}{n}{cmtt}{8,9,10,12}

\DeclareFontShape@TPSF[\ttscale@TPSF]{OT1}{cmtt}{m}{it}{cmitt}{10}

\DeclareFontShape@TPSF[\ttscale@TPSF]{OT1}{cmtt}{m}{sl}{cmsltt}{10}

\DeclareFontShape@TPSF[\ttscale@TPSF]{OT1}{cmtt}{m}{sc}{cmtcsc}{10}

\DeclareFontShape{OT1}{cmtt}{m}{ui}
  {<->sub * cmtt/m/it}{}

\DeclareFontShape{OT1}{cmtt}{bx}{n}
  {<->sub * cmtt/m/n}{}

\DeclareFontShape{OT1}{cmtt}{bx}{it}
  {<->sub * cmtt/m/it}{}

\DeclareFontShape{OT1}{cmtt}{bx}{ui}
  {<->sub * cmtt/m/it}{}
%    \end{macrocode}
% `Slides' computer modern typewriter exists only for compatibility with T1
% encoding. 
%    \begin{macrocode}
\DeclareFontFamily{OT1}{lcmtt}{\hyphenchar\font\m@ne}

\DeclareFontShape@TPSF[\ttscale@TPSF]{OT1}{lcmtt}{m}{n}{cmtt}{8,9,10,12}

\DeclareFontShape@TPSF[\ttscale@TPSF]{OT1}{lcmtt}{m}{it}{cmitt}{10}

\DeclareFontShape@TPSF[\ttscale@TPSF]{OT1}{lcmtt}{m}{sl}{cmsltt}{10}

\DeclareFontShape@TPSF[\ttscale@TPSF]{OT1}{lcmtt}{m}{sc}{cmtcsc}{10}

\DeclareFontShape{OT1}{lcmtt}{m}{ui}
  {<->sub * lcmtt/m/it}{}

\DeclareFontShape{OT1}{lcmtt}{bx}{n}
  {<->sub * lcmtt/m/n}{}

\DeclareFontShape{OT1}{lcmtt}{bx}{it}
  {<->sub * lcmtt/m/it}{}

\DeclareFontShape{OT1}{lcmtt}{bx}{ui}
  {<->sub * lcmtt/m/it}{}
%    \end{macrocode}
% computer modern typewriter light.
%    \begin{macrocode}
\DeclareFontFamily{OT1}{cmtl}{\hyphenchar\font\m@ne}

\DeclareFontShape{OT1}{cmtl}{m}{n}{%
<->s*\ttscale@TPSF cmtl10%
}{}

\DeclareFontShape{OT1}{cmtl}{m}{sl}{%
<->s*\ttscale@TPSF cmsltl10%
}{}

\DeclareFontShape{OT1}{cmtl}{m}{it}{<->ssub*cmtl/m/sl}{}
%    \end{macrocode}
%
% \subsection{Typewriter fonts for T1 encoding}
%
%    \begin{macrocode}
\ifcase\TPSFMaxDesignSize
 \or\or\or\or\or
  \providecommand{\EC@ttfamily}[5]{%
    \DeclareFontShape{#1}{#2}{#3}{#4}{%
      <->s*\ttscale@TPSF#50800%
    }{}}
 \or
  \providecommand{\EC@ttfamily}[5]{%
    \DeclareFontShape{#1}{#2}{#3}{#4}{%
      <->s*\ttscale@TPSF#50800%
    }{}}
 \or
  \providecommand{\EC@ttfamily}[5]{%
    \DeclareFontShape{#1}{#2}{#3}{#4}{%
      <->s*\ttscale@TPSF#50800%
    }{}}
 \or
  \providecommand{\EC@ttfamily}[5]{%
    \DeclareFontShape{#1}{#2}{#3}{#4}{%
      <->s*\ttscale@TPSF#50800%
    }{}}
 \or
  \providecommand{\EC@ttfamily}[5]{%
    \DeclareFontShape{#1}{#2}{#3}{#4}{%
      <-9>s*\ttscale@TPSF#50800%
      <9->s*\ttscale@TPSF#50900%
    }{}}
 \or
  \providecommand{\EC@ttfamily}[5]{%
    \DeclareFontShape{#1}{#2}{#3}{#4}{%
      <-9>s*\ttscale@TPSF#50800%
      <9-10>s*\ttscale@TPSF#50900%
      <10->s*\ttscale@TPSF#51000%
    }{}}
 \or
  \providecommand{\EC@ttfamily}[5]{%
    \DeclareFontShape{#1}{#2}{#3}{#4}{%
      <-9>s*\ttscale@TPSF#50800%
      <9-10>s*\ttscale@TPSF#50900%
      <10-10.95>s*\ttscale@TPSF#51000%
      <10.95->s*\ttscale@TPSF#51095%
    }{}}
 \or
  \providecommand{\EC@ttfamily}[5]{%
    \DeclareFontShape{#1}{#2}{#3}{#4}{%
      <-9>s*\ttscale@TPSF#50800%
      <9-10>s*\ttscale@TPSF#50900%
      <10-10.95>s*\ttscale@TPSF#51000%
      <10.95-12>s*\ttscale@TPSF#51095%
      <12->s*\ttscale@TPSF#51200%
    }{}}
\fi
%    \end{macrocode}
% european computer modern typewriter.
%    \begin{macrocode}
%%
%% This is file `t1cmtt.fd',
%% generated with the docstrip utility.
%%
%% The original source files were:
%%
%% cmfonts.fdd  (with options: `fd,T1cmtt,tt,ec,nowarn')
%% 
%% This is a generated file.
%% 
%% Copyright 1993 1994 1995 1996 1997 1998 1999 2000 2001 2002 2003 2004 2005
%% The LaTeX3 Project and any individual authors listed elsewhere
%% in this file.
%% 
%% This file was generated from file(s) of the LaTeX base system.
%% --------------------------------------------------------------
%% 
%% It may be distributed and/or modified under the
%% conditions of the LaTeX Project Public License, either version 1.3c
%% of this license or (at your option) any later version.
%% The latest version of this license is in
%%    http://www.latex-project.org/lppl.txt
%% and version 1.3c or later is part of all distributions of LaTeX
%% version 2005/12/01 or later.
%% 
%% This file may only be distributed together with a copy of the LaTeX
%% base system. You may however distribute the LaTeX base system without
%% such generated files.
%% 
%% The list of all files belonging to the LaTeX base distribution is
%% given in the file `manifest.txt'. See also `legal.txt' for additional
%% information.
%% 
%% In particular, permission is granted to customize the declarations in
%% this file to serve the needs of your installation.
%% 
%% However, NO PERMISSION is granted to distribute a modified version
%% of this file under its original name.
%% 
\ProvidesFile{t1cmtt.fd}
        [1999/05/25 v2.5h Standard LaTeX font definitions]
\providecommand{\EC@ttfamily}[5]{%
  \DeclareFontShape{#1}{#2}{#3}{#4}%
  {<5><6><7><8>#50800%
  <9><10><10.95><12><14.4><17.28><20.74><24.88><29.86>%
  <35.83>genb*#5}{}}
\DeclareFontFamily{T1}{cmtt}{\hyphenchar \font\m@ne}
\EC@ttfamily{T1}{cmtt}{m}{n}{ectt}
\EC@ttfamily{T1}{cmtt}{m}{sl}{ecst}
\EC@ttfamily{T1}{cmtt}{m}{it}{ecit}
\EC@ttfamily{T1}{cmtt}{m}{sc}{ectc}
\DeclareFontShape{T1}{cmtt}{bx}{n}%
  {<->ssub*cmtt/m/n}{}
\DeclareFontShape{T1}{cmtt}{bx}{it}%
  {<->ssub*cmtt/m/it}{}
\endinput
%%
%% End of file `t1cmtt.fd'.

%    \end{macrocode}
% european `Slides' computer modern typewriter. There exists \textbf{one} dedicated
% font ``ecltt8'' for this setup.
%    \begin{macrocode}
\DeclareFontFamily{T1}{lcmtt}{\hyphenchar\font\m@ne}

\DeclareFontShape{T1}{lcmtt}{m}{n}{%
  <->s*\ttscale@TPSF ecltt8%
}{}

\DeclareFontShape{T1}{lcmtt}{m}{In}{%
  <->s*\ttscale@TPSF iecltt8%
}{}

\DeclareFontShape{T1}{lcmtt}{m}{it}{%
  <->s*\ttscale@TPSF ecit0800%
}{}

\DeclareFontShape{T1}{lcmtt}{m}{sl}{
  <->s*\ttscale@TPSF     ecst0800
}{}

\DeclareFontShape{T1}{lcmtt}{m}{sc}{
  <->s*\ttscale@TPSF     ectc0800
}{}

\DeclareFontShape{T1}{lcmtt}{m}{ui}
{<->sub * lcmtt/m/it}{}

\DeclareFontShape{T1}{lcmtt}{bx}{n}
{<->sub * lcmtt/m/n}{}

\DeclareFontShape{T1}{lcmtt}{bx}{it}
{<->sub * lcmtt/m/it}{}

\DeclareFontShape{T1}{lcmtt}{bx}{ui}
{<->sub * lcmtt/m/it}{}
%    \end{macrocode}
% european computer modern typewriter light.
%    \begin{macrocode}
\DeclareFontFamily{T1}{cmtl}{\hyphenchar\font\m@ne}

\DeclareFontShape{T1}{cmtl}{m}{n}{%
<->s*\ttscale@TPSF ebtl10%
}{}

\DeclareFontShape{T1}{cmtl}{m}{sl}{%
<->s*\ttscale@TPSF ebto10%
}{}

\DeclareFontShape{T1}{cmtl}{m}{it}{<->ssub*cmtl/m/sl}{}
%    \end{macrocode}
%
% \subsection{Typewriter fonts for mTT encoding} 
%
% Used by cmtt.sty.
%    \begin{macrocode}
\@ifundefined{T@mTT}{}
{%
  \DeclareFontFamily{mTT}{cmtt}{\hyphenchar\font\m@ne}
  \DeclareFontShape@TPSF[\ttscale@TPSF]{mTT}{cmtt}{m}{n}{cmtt}{8,9,10,12}
  
  \DeclareFontShape@TPSF[\ttscale@TPSF]{mTT}{cmtt}{m}{it}{cmitt}{10}
  
  \DeclareFontShape@TPSF[\ttscale@TPSF]{mTT}{cmtt}{m}{sl}{cmsltt}{10}
  
  \DeclareFontShape@TPSF[\ttscale@TPSF]{mTT}{cmtt}{m}{sc}{cmtcsc}{10}

  \DeclareFontShape{mTT}{cmtt}{m}{ui}
  {<->sub * cmtt/m/it}{}
  
  \DeclareFontShape{mTT}{cmtt}{bx}{n}
  {<->sub * cmtt/m/n}{}
  
  \DeclareFontShape{mTT}{cmtt}{bx}{it}
  {<->sub * cmtt/m/it}{}
  
  \DeclareFontShape{mTT}{cmtt}{bx}{ui}
  {<->sub * cmtt/m/it}{}
  
  \DeclareFontFamily{mTT}{lcmtt}{\hyphenchar\font\m@ne}

  \DeclareFontShape@TPSF[\ttscale@TPSF]{mTT}{lcmtt}{m}{n}{cmtt}{8,9,10,12}
  
  \DeclareFontShape@TPSF[\ttscale@TPSF]{mTT}{lcmtt}{m}{it}{cmitt}{10}
  
  \DeclareFontShape@TPSF[\ttscale@TPSF]{mTT}{lcmtt}{m}{sl}{cmsltt}{10}
  
  \DeclareFontShape@TPSF[\ttscale@TPSF]{mTT}{lcmtt}{m}{sc}{cmtcsc}{10}
  
  \DeclareFontShape{mTT}{lcmtt}{m}{ui}
  {<->sub * lcmtt/m/it}{}
  
  \DeclareFontShape{mTT}{lcmtt}{bx}{n}
  {<->sub * lcmtt/m/n}{}
  
  \DeclareFontShape{mTT}{lcmtt}{bx}{it}
  {<->sub * lcmtt/m/it}{}
  
  \DeclareFontShape{mTT}{lcmtt}{bx}{ui}
  {<->sub * lcmtt/m/it}{}

  \DeclareFontFamily{mTT}{cmtl}{\hyphenchar\font\m@ne}
  
  \DeclareFontShape{mTT}{cmtl}{m}{n}{%
    <->s*\ttscale@TPSF cmtl10%
  }{}
  
  \DeclareFontShape{mTT}{cmtl}{m}{sl}{%
    <->s*\ttscale@TPSF cmsltl10%
  }{}
  
  \DeclareFontShape{mTT}{cmtl}{m}{it}{<->ssub*cmtl/m/sl}{}
}
%    \end{macrocode}
%
% \subsection{Math fonts}
%
% computer modern math italic.
%    \begin{macrocode}
\DeclareFontFamily{OML}{cmm}{\skewchar\font127}

\DeclareFontShape@TPSF[\mathscale@TPSF]{OML}{cmm}{m}{it}{cmmi}{5,6,7,8,9,10,12}

\DeclareFontShape@TPSF[\mathscale@TPSF]{OML}{cmm}{m}{Iit}{icmmi}{8}

\DeclareFontShape@TPSF[\mathscale@TPSF]{OML}{cmm}{b}{it}{cmmib}{5,7,10}

\DeclareFontShape{OML}{cmm}{bx}{it}%
   {<->ssub*cmm/b/it}{}

\DeclareFontFamily{OML}{cmr}{\skewchar\font127 }
\DeclareFontShape{OML}{cmr}{m}{n}
   {<-> ssub * cmm/m/it}{}
\DeclareFontShape{OML}{cmr}{m}{it}
   {<-> ssub * cmm/m/it}{}
\DeclareFontShape{OML}{cmr}{m}{sl}
   {<-> ssub * cmm/m/it}{}
\DeclareFontShape{OML}{cmr}{m}{sc}
   {<-> ssub * cmm/m/it}{}
\DeclareFontShape{OML}{cmr}{bx}{n}
   {<-> ssub * cmm/b/it}{}
\DeclareFontShape{OML}{cmr}{bx}{it}
   {<-> ssub * cmm/b/it}{}
\DeclareFontShape{OML}{cmr}{bx}{sl}
   {<-> ssub * cmm/b/it}{}
\DeclareFontShape{OML}{cmr}{bx}{sc}
   {<-> ssub * cmm/b/it}{}

\DeclareFontFamily{OT1}{cmrm}{\hyphenchar\font=45}

\DeclareFontShape@TPSF[\mathscale@TPSF]{OT1}{cmrm}{m}{n}{cmr}{5,6,7,8,9,10,12,17}

\DeclareFontShape@TPSF[\mathscale@TPSF]{OT1}{cmrm}{m}{it}{cmti}{7,8,9,10,12}

\DeclareFontShape@TPSF[\mathscale@TPSF]{OT1}{cmrm}{bx}{n}{cmbx}{5,6,7,8,9,10,12}

\DeclareFontShape@TPSF[\mathscale@TPSF]{OT1}{cmrm}{bx}{it}{cmbxti}{10}

{%
\let\ecscale@TPSF\mathscale@TPSF

\DeclareFontFamily{T1}{cmrm}{}

\EC@family{T1}{cmrm}{m}{n}{ecrm}

\EC@family{T1}{cmrm}{m}{it}{ecti}

\EC@family{T1}{cmrm}{bx}{n}{ecbx}

\EC@family{T1}{cmrm}{bx}{it}{ecbi}
}

\DeclareFontFamily{OMS}{cmsy}{\skewchar\font48}

\DeclareFontShape@TPSF[\mathscale@TPSF]{OMS}{cmsy}{m}{n}{cmsy}{5,6,7,8,9,10}

\DeclareFontShape@TPSF[\mathscale@TPSF]{OMS}{cmsy}{b}{n}{cmbsy}{5,7,10}

\DeclareFontShape@TPSF[\mathscale@TPSF]{OMS}{cmsy}{m}{In}{icmsy}{8}

\DeclareFontFamily{OMS}{cmr}{\skewchar\font48 }
\DeclareFontShape{OMS}{cmr}{m}{n}
   {<-> ssub * cmsy/m/n}{}
\DeclareFontShape{OMS}{cmr}{m}{it}
   {<-> ssub * cmsy/m/n}{}
\DeclareFontShape{OMS}{cmr}{m}{sl}
   {<-> ssub * cmsy/m/n}{}
\DeclareFontShape{OMS}{cmr}{m}{sc}
   {<-> ssub * cmsy/m/n}{}
\DeclareFontShape{OMS}{cmr}{bx}{n}
   {<-> ssub * cmsy/b/n}{}
\DeclareFontShape{OMS}{cmr}{bx}{it}
   {<-> ssub * cmsy/b/n}{}
\DeclareFontShape{OMS}{cmr}{bx}{sl}
   {<-> ssub * cmsy/b/n}{}
\DeclareFontShape{OMS}{cmr}{bx}{sc}
   {<-> ssub * cmsy/b/n}{}

\DeclareFontFamily{OMX}{cmex}{}

\DeclareFontShape@TPSF[\mathscale@TPSF]{OMX}{cmex}{m}{n}{cmex}{7,10}
%    \end{macrocode}
% Euler fonts (eulervm style).
%    \begin{macrocode}
\DeclareFontFamily{U}{zeur}{\skewchar \font =127}

\DeclareFontShape@TPSF[\eulerscale@TPSF]{U}{zeur}{m}{n}{zeurm}{5,7,10}

\DeclareFontShape@TPSF[\eulerscale@TPSF]{U}{zeur}{b}{n}{zeurb}{5,7,10}

\DeclareFontShape{U}{zeur}{bx}{n}{<->ssub * zeur/b/n}{}

\DeclareFontFamily{U}{zeus}{\skewchar \font =176}

\DeclareFontShape@TPSF[\eulerscale@TPSF]{U}{zeus}{m}{n}{zeusm}{5,7,10}

\DeclareFontShape@TPSF[\eulerscale@TPSF]{U}{zeus}{b}{n}{zeusb}{5,7,10}

\DeclareFontShape{U}{zeus}{bx}{n}{<->ssub * zeus/b/n}{}

\DeclareFontFamily{U}{zeuex}{}

\DeclareFontShape{U}{zeuex}{m}{n}{
  <->s* \eulerscale@TPSF zeuex10
}{}
%    \end{macrocode}
% computer modern bright math slanted.
%    \begin{macrocode}
\DeclareFontFamily{OML}{cmbrm}{\skewchar\font 127}

\DeclareFontShape@TPSF[\cmbrscale@TPSF]{OML}{cmbrm}{m}{it}{cmbrmi}{8,9,10}

\DeclareFontShape@TPSF[\cmbrscale@TPSF]{OML}{cmbrm}{b}{it}{cmbrmb}{10}

\DeclareFontFamily{OML}{cmbr}{\skewchar\font 127}

\DeclareFontShape{OML}{cmbr}{m}{it}{<->ssub*cmbrm/m/it}{}

\DeclareFontShape{OML}{cmbr}{sb}{it}{<->ssub*cmbrm/b/it}{}

\DeclareFontShape{OML}{cmbr}{bx}{it}{<->ssub*cmbrm/b/it}{}

\DeclareFontFamily{OT1}{cmbrmt}{\hyphenchar\font45}

\DeclareFontShape@TPSF[\cmbrscale@TPSF]{OT1}{cmbrmt}{m}{n}{cmbr}{8,9,10,17}

\DeclareFontShape@TPSF[\cmbrscale@TPSF]{OT1}{cmbrmt}{m}{sl}{cmbrsl}{8,9,10,17}

\DeclareFontShape{OT1}{cmbrmt}{m}{it}{%
<->ssub*cmbrmt/m/sl%
}{}

\DeclareFontShape@TPSF[\cmbrscale@TPSF]{OT1}{cmbrmt}{bx}{n}{cmbrbx}{10}

\DeclareFontFamily{T1}{cmbrmt}{}

\DeclareFontShape@TPSF[\cmbrscale@TPSF]{T1}{cmbrmt}{m}{n}{ebmr}{8,9,10,17}

\DeclareFontShape@TPSF[\cmbrscale@TPSF]{T1}{cmbrmt}{m}{sl}{ebmo}{8,9,10,17}

\DeclareFontShape{T1}{cmbrmt}{m}{it}{%
<->ssub*cmbrmt/m/sl%
}{}

\DeclareFontShape@TPSF[\cmbrscale@TPSF]{T1}{cmbrmt}{sb}{n}{ebsr}{8,9,10,17}

\DeclareFontShape@TPSF[\cmbrscale@TPSF]{T1}{cmbrmt}{sb}{sl}{ebso}{8,9,10,17}

\DeclareFontShape{T1}{cmbrmt}{sb}{it}{%
<->ssub*cmbrmt/sb/sl%
}{}

\DeclareFontShape{T1}{cmbrmt}{bx}{n}{%
<->sub*cmbrmt/sb/n%
}{}

\DeclareFontShape{T1}{cmbrmt}{bx}{it}{%
<->sub*cmbrmt/sb/it%
}{}

\DeclareFontFamily{OMS}{cmbrs}{\skewchar\font 48}

\DeclareFontShape@TPSF[\cmbrscale@TPSF]{OMS}{cmbrs}{m}{n}{cmbrsy}{8,9,10}

\DeclareFontShape{OMS}{cmbrs}{m}{b}{<->ssub*cmbrs/m/n}{}

\DeclareFontFamily{OMS}{cmbr}{\skewchar\font 48}

\DeclareFontShape{OMS}{cmbr}{m}{n}{<->ssub*cmbrs/m/n}{}

\DeclareFontShape{OMS}{cmbr}{m}{b}{<->ssub*cmbrs/m/n}{}
%    \end{macrocode}
% Concrete Math Italic
%    \begin{macrocode}
\DeclareFontFamily{OML}{ccm}{\skewchar\font127 }

\DeclareFontShape@TPSF[\mathscale@TPSF]{OML}{ccm}{m}{it}{xccmi}{5,6,7,8,9,10}

\DeclareFontShape{OML}{ccm}{b}{it}{<-> ssub * cmm/b/it}{}

\DeclareFontShape{OML}{ccm}{c}{it}{<9>\mathscale@TPSF ccmic9}{}

\DeclareFontFamily{OML}{ccr}{\skewchar\font127 }

\DeclareFontShape{OML}{ccr}{m}{it}  {<->ssub * ccm/m/it}{}

\DeclareFontShape{OML}{ccr}{bx}{it} {<->ssub * ccm/b/it}{}

\DeclareFontShape{OML}{ccr}{sbc}{it}{<->ssub * ccm/m/it}{}

\DeclareFontFamily{OMS}{ccsy}{\skewchar\font48 }

\DeclareFontShape@TPSF[\mathscale@TPSF]{OMS}{ccsy}{m}{n}{xccsy}{5,6,7,8,9,10}

\DeclareFontShape{OMS}{ccsy}{b}{n}{<-> ssub * cmsy/b/n}{}

\DeclareFontFamily{OMS}{ccr}{\skewchar\font48 }

\DeclareFontShape{OMS}{ccr}{m}{n}  {<->ssub * ccsy/m/n}{}

\DeclareFontShape{OMS}{ccr}{bx}{n} {<->ssub * ccsy/b/n}{}

\DeclareFontShape{OMS}{ccr}{sbc}{n}{<->ssub * ccsy/m/n}{}

\DeclareFontFamily{OMX}{ccex}{}

\DeclareFontShape{OMX}{ccex}{m}{n}{<->s*\mathscale@TPSF  xccex10}{}
%    \end{macrocode}
% LaTeX symbols
%    \begin{macrocode}
\DeclareFontFamily{U}{lasy}{}

\DeclareFontShape@TPSF[\mathscale@TPSF]{U}{lasy}{m}{n}{lasy}{5,6,7,8,9,10}

\DeclareFontShape{U}{lasy}{b}{n}{
        <->s* \mathscale@TPSF    lasyb10
        }{}
%    \end{macrocode}
% AMS fonts
%    \begin{macrocode}
\DeclareFontFamily{U}{msa}{}

\DeclareFontFamily{U}{msb}{}

\ifthenelse{\equal{\TPSFMathfont}{cmbrm}}
{%
%    \end{macrocode}
% AMS fonts should be available immediately, so they can be used in say panels. 
%    \begin{macrocode}
  \DeclareFontShape@TPSF[\cmbrscale@TPSF]{U}{msa}{m}{n}{cmbras}{8,9,10}%
  \DeclareFontShape@TPSF[\cmbrscale@TPSF]{U}{msb}{m}{n}{cmbrbs}{8,9,10}%
%    \end{macrocode}
% On the other hand, the cmbright package redeclares them |\AtBeginDocument|, so
% we need to overwrite that change as well.
%    \begin{macrocode}
  \AtBeginDocument
  {%
    \DeclareFontShape@TPSF[\cmbrscale@TPSF]{U}{msa}{m}{n}{cmbras}{8,9,10}%
    \DeclareFontShape@TPSF[\cmbrscale@TPSF]{U}{msb}{m}{n}{cmbrbs}{8,9,10}%
  }%
}
{%
  \DeclareFontShape@TPSF[\mathscale@TPSF]{U}{msa}{m}{n}{msam}{5,7,10}

  \DeclareFontShape@TPSF[\mathscale@TPSF]{U}{msb}{m}{n}{msbm}{5,7,10}
}

\DeclareFontFamily{U}{euex}{}

\DeclareFontShape{U}{euex}{m}{n}{
        <->s* \eulerscale@TPSF    euex10 
        }{}

\DeclareFontFamily{U}{euf}{}

\DeclareFontShape@TPSF[\eulerscale@TPSF]{U}{euf}{m}{n}{eufm}{5,7,10}

\DeclareFontShape@TPSF[\eulerscale@TPSF]{U}{euf}{b}{n}{eufb}{5,7,10}

\DeclareFontFamily{U}{eur}{\skewchar\font'177}

\DeclareFontShape@TPSF[\eulerscale@TPSF]{U}{eur}{m}{n}{eurm}{5,7,10}

\DeclareFontShape@TPSF[\eulerscale@TPSF]{U}{eur}{b}{n}{eurb}{5,7,10}

\DeclareFontFamily{U}{eus}{\skewchar\font'60}

\DeclareFontShape@TPSF[\eulerscale@TPSF]{U}{eus}{m}{n}{eusm}{5,7,10}

\DeclareFontShape@TPSF[\eulerscale@TPSF]{U}{eus}{b}{n}{eusb}{5,7,10}
%    \end{macrocode}
% St Mary's Road symbol font.
%    \begin{macrocode}
\DeclareFontFamily{U}{stmry}{}

\DeclareFontShape@TPSF[\mathscale@TPSF]{U}{stmry}{m}{n}{stmary}{5,6,7,8,9,10}
%    \end{macrocode}
% Wasy symbol font.
%    \begin{macrocode}
\DeclareFontFamily{U}{wasy}{}

\DeclareFontShape@TPSF[\mathscale@TPSF]{U}{wasy}{m}{n}{wasy}{5,6,7,8,9,10}
      
\DeclareFontShape{U}{wasy}{b}{n}
{ <->s*\mathscale@TPSF wasyb10 
}{}
%    \end{macrocode}
% formal script math symbols
%    \begin{macrocode}
\DeclareFontFamily{U}{rsfs}{\skewchar\font127 }

\DeclareFontShape@TPSF[\mathscale@TPSF]{U}{rsfs}{m}{n}{rsfs}{5,7,10}
%    \end{macrocode}
% double stroke computer modern
%    \begin{macrocode}
\DeclareFontFamily{U}{dsrom}{}

\DeclareFontShape@TPSF[\mathscale@TPSF]{U}{dsrom}{m}{n}{dsrom}{8,10,12}

\DeclareFontFamily{U}{dsss}{}

\DeclareFontShape@TPSF[\mathscale@TPSF]{U}{dsss}{m}{n}{dsss}{8,10,12}

\endgroup
%    \end{macrocode}

% Just to make sure in case these have been defined before...
%    \begin{macrocode}
\expandafter\let\csname OT1/cmr/m/n/10\endcsname\relax
\expandafter\let\csname OT1/cmr/m/n/10.95\endcsname\relax
\expandafter\let\csname OT1/cmr/m/n/12\endcsname\relax
\expandafter\let\csname T1/cmr/m/n/10\endcsname\relax
\expandafter\let\csname T1/cmr/m/n/10.95\endcsname\relax
\expandafter\let\csname T1/cmr/m/n/12\endcsname\relax
%    \end{macrocode}
%
% \subsection{Font shape selection}
%
% \subsubsection{Set text fonts}
%    \begin{macrocode}
\newcommand{\setlcmssTextFonts@TPSF}
{%
  \renewcommand{\familydefault}{\sfdefault}% Main text font is sans serif.
  \renewcommand{\sfdefault}{lcmss}%          Choose lcmss as sans serif font.
  \renewcommand{\ttdefault}{lcmtt}%          Typewriter font lcmtt.
}

\newcommand{\setcmrTextFonts@TPSF}
{%
  \renewcommand{\familydefault}{\rmdefault}% Main text font is roman.
  \renewcommand{\rmdefault}{cmr}%            Choose cmr as roman font.
  \renewcommand{\ttdefault}{cmtt}%           Typewriter font.
}

\newcommand{\setcmfibTextFonts@TPSF}
{%
  \renewcommand{\familydefault}{\rmdefault}% Main text font is roman.
  \renewcommand{\rmdefault}{cmfib}%          Choose cmfib as roman font.
  \renewcommand{\ttdefault}{cmtt}%           Typewriter font.
}

\newcommand{\setcmssTextFonts@TPSF}
{%
  \renewcommand{\familydefault}{\sfdefault}% Main text font is sans serif.
  \renewcommand{\sfdefault}{cmss}%           Choose cmss as sans serif font.
  \renewcommand{\ttdefault}{cmtt}%           Typewriter font.
}

\newcommand{\setcmbrTextFonts@TPSF}
{%
  \renewcommand{\familydefault}{\sfdefault}% Main text font is sans serif.
  \renewcommand{\sfdefault}{cmbr}%           Choose cmss as sans serif font.
  \renewcommand{\ttdefault}{cmtl}%           Light Typewriter font.
}

\newcommand{\setccrTextFonts@TPSF}
{%
  \renewcommand{\familydefault}{\rmdefault}% Main text font is roman.
  \renewcommand{\rmdefault}{ccr}%            Choose ccr as roman font.
  \renewcommand{\ttdefault}{cmtt}%           Typewriter font.
}

\csname set\TPSFTextfont TextFonts@TPSF\endcsname
%    \end{macrocode}
%
% \subsubsection{Configure Math fonts}
%
%    \begin{macrocode}
\newcommand{\setcmmMathFonts@TPSF}
{%
  \SetSymbolFont{letters}     {normal}{OML}{cmm} {m}{it}
  \SetSymbolFont{symbols}     {normal}{OMS}{cmsy}{m}{n}
  \SetSymbolFont{largesymbols}{normal}{OMX}{cmex}{m}{n}
  \SetSymbolFont{letters}     {bold}  {OML}{cmm} {b}{it}
  \SetSymbolFont{symbols}     {bold}  {OMS}{cmsy}{b}{n}
}

\newcommand{\setccmMathFonts@TPSF}
{%
  \SetSymbolFont{letters}     {normal}{OML}{ccm} {m}{it}
  \SetSymbolFont{symbols}     {normal}{OMS}{ccsy}{m}{n}
  \SetSymbolFont{largesymbols}{normal}{OMX}{ccex}{m}{n}
  \SetSymbolFont{letters}     {bold}  {OML}{ccm} {b}{it}
  \SetSymbolFont{symbols}     {bold}  {OMS}{ccsy}{b}{n}
}

\newcommand{\seteulerMathFonts@TPSF}
{%
  \RequirePackage{eulervm}
  \def\tpc@ncel##1##2{\m@th\ooalign{$\hfil##1/\hfil$\crcr$##1##2$}}
  \def\neq{\mathrel{\m@th\mathpalette\tpc@ncel=}}
  \DeclareSymbolFont{auxletters}{OML}{cmm}{m}{it}
  \SetSymbolFont{auxletters}{bold}{OML}{cmm}{b}{it}
  \DeclareMathSymbol{<}{\mathrel}{auxletters}{"3C}
  \DeclareMathSymbol{>}{\mathrel}{auxletters}{"3E}
}

\newcommand{\setcmbrmMathFonts@TPSF}
{%
  \SetSymbolFont{letters}     {normal}{OML}{cmbrm} {m}{it}
  \SetSymbolFont{symbols}     {normal}{OMS}{cmbrs}{m}{n}
  \SetSymbolFont{largesymbols}{normal}{OMX}{cmex}{m}{n}
  \SetSymbolFont{letters}     {bold}  {OML}{cmbrm} {b}{it}
  \SetSymbolFont{symbols}     {bold}{OMS}{cmbrs}{b}{n}
}

\csname set\TPSFMathfont MathFonts@TPSF\endcsname
%    \end{macrocode}
%
% \subsubsection{Configure Operator fonts}
%
%    \begin{macrocode}
\ifthenelse{\equal{\TPSFMathfont}{euler}}
{%
  \SetSymbolFont{operators}{normal}{\encodingdefault}{\TPSFOperatorfont}{m}{n}%
  \SetSymbolFont{operators}{bold}{\encodingdefault}{\TPSFOperatorfont}{bx}{n}%
}
{%
  \SetSymbolFont{operators}{normal}{OT1}{\TPSFOperatorfont}{m}{n}%
  \SetSymbolFont{operators}{bold}{OT1}{\TPSFOperatorfont}{bx}{n}%
}
%    \end{macrocode}
%
% \subsubsection{Alphabets for ``text inside math''}
%
%    \begin{macrocode}
\DeclareMathAlphabet      {\mathbf}{\encodingdefault}{\TPSFOperatorfont}{bx}{n}
\DeclareMathAlphabet      {\mathsf}{\encodingdefault}{\sfdefault}{m}{n}
\DeclareMathAlphabet      {\mathit}{\encodingdefault}{\TPSFOperatorfont}{m}{it}
\DeclareMathAlphabet      {\mathtt}{\encodingdefault}{\ttdefault}{m}{n}
\SetMathAlphabet\mathsf{bold}{\encodingdefault}{\sfdefault}{bx}{n}
\SetMathAlphabet\mathit{bold}{\encodingdefault}{\TPSFOperatorfont}{bx}{it}
\SetMathAlphabet{\mathtt}{bold}{\encodingdefault}{\ttdefault}{bx}{n}
%    \end{macrocode}
%
% |\oldstylenums| follow in style the math font setting.
%    \begin{macrocode}
\def\oldstylenums#1{%
   \begingroup
    \spaceskip\fontdimen\tw@\font
    \usefont{OML}{\TPSFMathfont}{\f@series}{it}%
    \mathgroup\symletters #1%
   \endgroup
}
%    \end{macrocode}
%
% \subsubsection{Configure doublestroke fonts}
%
%    \begin{macrocode}
\@ifpackageloaded{dsfont}
{%
  \ifthenelse{\equal{\TPSFOperatorfont}{cmr}\or\equal{\TPSFOperatorfont}{cmrm}}
  {%
    \DeclareMathAlphabet{\mathds}{U}{dsrom}{m}{n}
  }
  {%
    \DeclareMathAlphabet{\mathds}{U}{dsss}{m}{n}
  }
}{}%
%    \end{macrocode}
%
% \Finale
\endinput
