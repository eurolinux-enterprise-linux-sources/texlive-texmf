%%
%% This is file `__TPpreamble.tex',
%% generated with the docstrip utility.
%%
%% The original source files were:
%%
%% texpower-doc.dtx  (with options: `preamble')
%% 
%% --------------------------------------------------------------
%% TeXPower bundle - dynamic online presentations with LaTeX
%% Copyright (C) 1999-2004 Stephan Lehmke
%% 
%% This program is free software; you can redistribute it and/or
%% modify it under the terms of the GNU General Public License
%% as published by the Free Software Foundation; either version 2
%% of the License, or (at your option) any later version.
%% 
%% This program is distributed in the hope that it will be useful,
%% but WITHOUT ANY WARRANTY; without even the implied warranty of
%% MERCHANTABILITY or FITNESS FOR A PARTICULAR PURPOSE.  See the
%% GNU General Public License for more details.
%% --------------------------------------------------------------
%% 
%% The list of all files belonging to the TeXPower bundle is
%% given in the file `00readme.txt'.
%% 
%
\documentclass
[%
%-----------------------------------------------------------------------------------------------------------------
% Document class options:
% -----------------------
%
% Landscape slides formatted for letter paper fit most screen resolutions (more or less).
%
  letterpaper,%
  landscape,%
%
% The KOMA option makes powersem load scrartcl.cls instead of article.cls.
%
  KOMA,%
% KOMA document class options are accepted.
  smallheadings,%
%
% The calcdimensions option makes powersem calculate the slide dimensions automatically from paper size and margins.
  calcdimensions,%
%
% The display option sets everything up for producing slides to be displayed interactively.
% This option is also recognized by the texpower package.
%
  display%
%-----------------------------------------------------------------------------------------------------------------
]
%-----------------------------------------------------------------------------------------------------------------
% Document class powersem, based on seminar.cls for simulating ppower via latex+distiller (instead of pdflatex).
%
{powersem}
%-----------------------------------------------------------------------------------------------------------------
%
% First part of the preamble of TeXPower demos.
%
%-----------------------------------------------------------------------------------------------------------------
% Autor: Stephan Lehmke <Stephan.Lehmke@cs.uni-dortmund.de>
%
% v0.0.1 Mar 20, 2000: First version for the pre-alpha release of TeXPower.
% v0.0.2 Mar 21, 2000: Remedying an incompatibility between LaTeX releases concerning the implementation of
%                      \@iiiparbox (Apr 11: this code is now part of texpower.sty).
% v0.0.3 Apr 11, 2000: Color emphasis code moved into texpower.
%

%-----------------------------------------------------------------------------------------------------------------
% Set slide margins rather small for maximum use of space. This is a demo, remember.
%
\renewcommand{\slidetopmargin}{5mm}
\renewcommand{\slidebottommargin}{5mm}

\renewcommand{\slideleftmargin}{5mm}
\renewcommand{\sliderightmargin}{5mm}


%-----------------------------------------------------------------------------------------------------------------
% Some setup for more reasonable spacing.
%

\makeatletter

\renewcommand\section{\@startsection{section}{1}{\z@}%
  {-1.5ex\@plus -1ex \@minus -.5ex}%
  {.5ex \@plus .2ex}%
  {\raggedsection\normalfont\size@section\sectfont}}

\renewcommand\subsection{\@startsection{subsection}{2}{\z@}%
  {-1.25ex\@plus -1ex \@minus -.2ex}%
  {.5ex \@plus .2ex}%
  {\raggedsection\normalfont\size@subsection\sectfont}}

\renewcommand\subsubsection{\@startsection{subsubsection}{3}{\z@}%
  {-1.25ex\@plus -1ex \@minus -.2ex}%
  {.5ex \@plus .2ex}%
  {\raggedsection\normalfont\size@subsubsection\sectfont}}

\renewcommand\paragraph{\@startsection{paragraph}{4}{\z@}%
  {1.25ex \@plus1ex \@minus.2ex}%
  {-1em}%
  {\raggedsection\normalfont\size@paragraph\sectfont}}

\def\slideitemsep{.5ex plus .3ex minus .2ex}

\makeatother

%-----------------------------------------------------------------------------------------------------------------
% We need some more packages...
%

\usepackage{url}

\usepackage[latin1]{inputenc}

% One more Text emphasis command...

\let\name=\textsc

% Second part of the preamble of TeXPower demos.
%
%-----------------------------------------------------------------------------------------------------------------
% Autor: Stephan Lehmke <Stephan.Lehmke@cs.uni-dortmund.de>
%
% v0.0.1 Mar 20, 2000: First version for the pre-alpha release of TeXPower.
% v0.0.2 Mar 22, 2000: Now loading the config file.
% v0.0.3 Mar 29, 2000: texpower doesn't load hyperref any more; there's now a package fixseminar.
% v0.0.4 Apr 19, 2000: Added \slidetitle command.
% v0.0.5 Sep 11, 2000: Added plainpages=false to the hyperref options to get correct page anchors.
% v0.0.5 Sep 11, 2002: Slight changes to title page.
%


%-----------------------------------------------------------------------------------------------------------------
% We load hyperref and fixseminar which fixes some problems with seminar.
%
\usepackage[plainpages=false,bookmarksopen,colorlinks,urlcolor=red,pdfpagemode=FullScreen]{hyperref}
\usepackage{fixseminar}

%-----------------------------------------------------------------------------------------------------------------
% Finally, the texpower package is loaded.
%
\usepackage{texpower}

%% The configuration file allows user-specific settings.

%%
%% This is file `__TPcfg.tex',
%% generated with the docstrip utility.
%%
%% The original source files were:
%%
%% texpower-doc.dtx  (with options: `config')
%% 
%% --------------------------------------------------------------
%% TeXPower bundle - dynamic online presentations with LaTeX
%% Copyright (C) 1999-2004 Stephan Lehmke
%% 
%% This program is free software; you can redistribute it and/or
%% modify it under the terms of the GNU General Public License
%% as published by the Free Software Foundation; either version 2
%% of the License, or (at your option) any later version.
%% 
%% This program is distributed in the hope that it will be useful,
%% but WITHOUT ANY WARRANTY; without even the implied warranty of
%% MERCHANTABILITY or FITNESS FOR A PARTICULAR PURPOSE.  See the
%% GNU General Public License for more details.
%% --------------------------------------------------------------
%% 
%% The list of all files belonging to the TeXPower bundle is
%% given in the file `00readme.txt'.
%% 
%-----------------------------------------------------------------------------------------------------------------
% File: __TPcfg.tex
%
% Code for user-specific configuration of TeXPower documentation files.
%
% This file is input by others. Don't compile it separately.
%
%-----------------------------------------------------------------------------------------------------------------
% Autor: Stephan Lehmke <Stephan.Lehmke@cs.uni-dortmund.de>
%
% v0.0.1 Mar 22, 2000: First version for the pre-alpha release of TeXPower.
%
\hypersetup{baseurl={http://texpower.sourceforge.net/doc/}}
\hypersetup{pdfsubject={Documentation and Examples for the texpower package}}
\hypersetup{pdfauthor={Stephan Lehmke}}
\endinput
%%
%% End of file `__TPcfg.tex'.


%-----------------------------------------------------------------------------------------------------------------
% Some more parameters...
%
\slidesmag{5}
\slideframe{none}
\pagestyle{empty}
\setcounter{tocdepth}{2}
\renewcommand{\currentpagevalue}{\value{slide}}

%-----------------------------------------------------------------------------------------------------------------
% The following command produces a title page for every example and documentation file.

\newcommand{\makeslidetitle}[1]
{%
  \title{The \TeX Power bundle\\[2ex]{\normalfont #1}}
  \author
  {%
    Stephan Lehmke\\
    \mdseries
    University of Dortmund\\
    \mdseries
    Department of Computer Science I\\
    \url{mailto:Stephan.Lehmke@udo.edu}%
  }
  {\centerslidestrue
  \maketitle
  \newslide}
  \setcounter{firststep}{1}% This way, the first step of all examples is displayed.
}
\endinput
%%
%% End of file `__TPpreamble.tex'.
