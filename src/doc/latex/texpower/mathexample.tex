%%
%% This is file `mathexample.tex',
%% generated with the docstrip utility.
%%
%% The original source files were:
%%
%% texpower-doc.dtx  (with options: `mathexample,mathexample-src,end')
%% 
%% --------------------------------------------------------------
%% TeXPower bundle - dynamic online presentations with LaTeX
%% Copyright (C) 1999-2004 Stephan Lehmke
%% 
%% This program is free software; you can redistribute it and/or
%% modify it under the terms of the GNU General Public License
%% as published by the Free Software Foundation; either version 2
%% of the License, or (at your option) any later version.
%% 
%% This program is distributed in the hope that it will be useful,
%% but WITHOUT ANY WARRANTY; without even the implied warranty of
%% MERCHANTABILITY or FITNESS FOR A PARTICULAR PURPOSE.  See the
%% GNU General Public License for more details.
%% --------------------------------------------------------------
%% 
%% The list of all files belonging to the TeXPower bundle is
%% given in the file `00readme.txt'.
%% 
\ProvidesFile{mathexample.tex}%
      [2004/07/27 TeXPower example file]
%-----------------------------------------------------------------------------------------------------------------
% File: mathexample.tex
%
% Math example for the package texpower.sty.
%
% This file can be compiled with pdfLaTeX or (standard) LaTeX. When using standard LaTeX, the dvi file produced should
% be processed with
%
% dvips -Ppdf -j0 mathexample
%
% afterwards processing the resulting ps file with
%
% distill mathexample.ps
%
% (The syntax is for a unix system with tetex 1.0 and distiller 3. Modify appropriately for other configurations.)
%
% The resulting pdf file is meant for presenting `interactively' with Adobe Acrobat Reader.
%
%-----------------------------------------------------------------------------------------------------------------
% Autor: Stephan Lehmke <Stephan.Lehmke@cs.uni-dortmund.de>
%
% v0.0.1 Mar 20, 2000: First version for the pre-alpha release of TeXPower.
%
% v0.0.2 Apr 19, 2000: Using \bstep instead of \boxedsteps.
%
% v0.0.3 Apr 27, 2000: Some small changes in preparation of the update to TeXpower v0.0.7.
%
% v0.0.4 May 24, 2000: texpower 0.0.8 now supports equation numbers in the argument of \stepwise, so align* was
%                      changed to align.
%



%-----------------------------------------------------------------------------------------------------------------
% Enable all color emphasis and highlighting options. Use white background and slifonts.

\PassOptionsToPackage{coloremph,colormath,colorhighlight,whitebackground}{texpower}

% Input the generic preamble.

%%
%% This is file `__TPpreamble.tex',
%% generated with the docstrip utility.
%%
%% The original source files were:
%%
%% texpower-doc.dtx  (with options: `preamble')
%% 
%% --------------------------------------------------------------
%% TeXPower bundle - dynamic online presentations with LaTeX
%% Copyright (C) 1999-2004 Stephan Lehmke
%% 
%% This program is free software; you can redistribute it and/or
%% modify it under the terms of the GNU General Public License
%% as published by the Free Software Foundation; either version 2
%% of the License, or (at your option) any later version.
%% 
%% This program is distributed in the hope that it will be useful,
%% but WITHOUT ANY WARRANTY; without even the implied warranty of
%% MERCHANTABILITY or FITNESS FOR A PARTICULAR PURPOSE.  See the
%% GNU General Public License for more details.
%% --------------------------------------------------------------
%% 
%% The list of all files belonging to the TeXPower bundle is
%% given in the file `00readme.txt'.
%% 
%
\documentclass
[%
%-----------------------------------------------------------------------------------------------------------------
% Document class options:
% -----------------------
%
% Landscape slides formatted for letter paper fit most screen resolutions (more or less).
%
  letterpaper,%
  landscape,%
%
% The KOMA option makes powersem load scrartcl.cls instead of article.cls.
%
  KOMA,%
% KOMA document class options are accepted.
  smallheadings,%
%
% The calcdimensions option makes powersem calculate the slide dimensions automatically from paper size and margins.
  calcdimensions,%
%
% The display option sets everything up for producing slides to be displayed interactively.
% This option is also recognized by the texpower package.
%
  display%
%-----------------------------------------------------------------------------------------------------------------
]
%-----------------------------------------------------------------------------------------------------------------
% Document class powersem, based on seminar.cls for simulating ppower via latex+distiller (instead of pdflatex).
%
{powersem}
%-----------------------------------------------------------------------------------------------------------------
%
% First part of the preamble of TeXPower demos.
%
%-----------------------------------------------------------------------------------------------------------------
% Autor: Stephan Lehmke <Stephan.Lehmke@cs.uni-dortmund.de>
%
% v0.0.1 Mar 20, 2000: First version for the pre-alpha release of TeXPower.
% v0.0.2 Mar 21, 2000: Remedying an incompatibility between LaTeX releases concerning the implementation of
%                      \@iiiparbox (Apr 11: this code is now part of texpower.sty).
% v0.0.3 Apr 11, 2000: Color emphasis code moved into texpower.
%

%-----------------------------------------------------------------------------------------------------------------
% Set slide margins rather small for maximum use of space. This is a demo, remember.
%
\renewcommand{\slidetopmargin}{5mm}
\renewcommand{\slidebottommargin}{5mm}

\renewcommand{\slideleftmargin}{5mm}
\renewcommand{\sliderightmargin}{5mm}


%-----------------------------------------------------------------------------------------------------------------
% Some setup for more reasonable spacing.
%

\makeatletter

\renewcommand\section{\@startsection{section}{1}{\z@}%
  {-1.5ex\@plus -1ex \@minus -.5ex}%
  {.5ex \@plus .2ex}%
  {\raggedsection\normalfont\size@section\sectfont}}

\renewcommand\subsection{\@startsection{subsection}{2}{\z@}%
  {-1.25ex\@plus -1ex \@minus -.2ex}%
  {.5ex \@plus .2ex}%
  {\raggedsection\normalfont\size@subsection\sectfont}}

\renewcommand\subsubsection{\@startsection{subsubsection}{3}{\z@}%
  {-1.25ex\@plus -1ex \@minus -.2ex}%
  {.5ex \@plus .2ex}%
  {\raggedsection\normalfont\size@subsubsection\sectfont}}

\renewcommand\paragraph{\@startsection{paragraph}{4}{\z@}%
  {1.25ex \@plus1ex \@minus.2ex}%
  {-1em}%
  {\raggedsection\normalfont\size@paragraph\sectfont}}

\def\slideitemsep{.5ex plus .3ex minus .2ex}

\makeatother

%-----------------------------------------------------------------------------------------------------------------
% We need some more packages...
%

\usepackage{url}

\usepackage[latin1]{inputenc}

% One more Text emphasis command...

\let\name=\textsc

% Second part of the preamble of TeXPower demos.
%
%-----------------------------------------------------------------------------------------------------------------
% Autor: Stephan Lehmke <Stephan.Lehmke@cs.uni-dortmund.de>
%
% v0.0.1 Mar 20, 2000: First version for the pre-alpha release of TeXPower.
% v0.0.2 Mar 22, 2000: Now loading the config file.
% v0.0.3 Mar 29, 2000: texpower doesn't load hyperref any more; there's now a package fixseminar.
% v0.0.4 Apr 19, 2000: Added \slidetitle command.
% v0.0.5 Sep 11, 2000: Added plainpages=false to the hyperref options to get correct page anchors.
% v0.0.5 Sep 11, 2002: Slight changes to title page.
%


%-----------------------------------------------------------------------------------------------------------------
% We load hyperref and fixseminar which fixes some problems with seminar.
%
\usepackage[plainpages=false,bookmarksopen,colorlinks,urlcolor=red,pdfpagemode=FullScreen]{hyperref}
\usepackage{fixseminar}

%-----------------------------------------------------------------------------------------------------------------
% Finally, the texpower package is loaded.
%
\usepackage{texpower}

%% The configuration file allows user-specific settings.

%%
%% This is file `__TPcfg.tex',
%% generated with the docstrip utility.
%%
%% The original source files were:
%%
%% texpower-doc.dtx  (with options: `config')
%% 
%% --------------------------------------------------------------
%% TeXPower bundle - dynamic online presentations with LaTeX
%% Copyright (C) 1999-2004 Stephan Lehmke
%% 
%% This program is free software; you can redistribute it and/or
%% modify it under the terms of the GNU General Public License
%% as published by the Free Software Foundation; either version 2
%% of the License, or (at your option) any later version.
%% 
%% This program is distributed in the hope that it will be useful,
%% but WITHOUT ANY WARRANTY; without even the implied warranty of
%% MERCHANTABILITY or FITNESS FOR A PARTICULAR PURPOSE.  See the
%% GNU General Public License for more details.
%% --------------------------------------------------------------
%% 
%% The list of all files belonging to the TeXPower bundle is
%% given in the file `00readme.txt'.
%% 
%-----------------------------------------------------------------------------------------------------------------
% File: __TPcfg.tex
%
% Code for user-specific configuration of TeXPower documentation files.
%
% This file is input by others. Don't compile it separately.
%
%-----------------------------------------------------------------------------------------------------------------
% Autor: Stephan Lehmke <Stephan.Lehmke@cs.uni-dortmund.de>
%
% v0.0.1 Mar 22, 2000: First version for the pre-alpha release of TeXPower.
%
\hypersetup{baseurl={http://texpower.sourceforge.net/doc/}}
\hypersetup{pdfsubject={Documentation and Examples for the texpower package}}
\hypersetup{pdfauthor={Stephan Lehmke}}
\endinput
%%
%% End of file `__TPcfg.tex'.


%-----------------------------------------------------------------------------------------------------------------
% Some more parameters...
%
\slidesmag{5}
\slideframe{none}
\pagestyle{empty}
\setcounter{tocdepth}{2}
\renewcommand{\currentpagevalue}{\value{slide}}

%-----------------------------------------------------------------------------------------------------------------
% The following command produces a title page for every example and documentation file.

\newcommand{\makeslidetitle}[1]
{%
  \title{The \TeX Power bundle\\[2ex]{\normalfont #1}}
  \author
  {%
    Stephan Lehmke\\
    \mdseries
    University of Dortmund\\
    \mdseries
    Department of Computer Science I\\
    \url{mailto:Stephan.Lehmke@udo.edu}%
  }
  {\centerslidestrue
  \maketitle
  \newslide}
  \setcounter{firststep}{1}% This way, the first step of all examples is displayed.
}
\endinput
%%
%% End of file `__TPpreamble.tex'.


\usepackage{tpslifonts}

\hypersetup{pdftitle={texpower math example}}


%-----------------------------------------------------------------------------------------------------------------
% Packages and Preamble settings individual for this example.

% We write some aligned equations.

\usepackage{amsmath}
% Make nested braces grow.
\delimitershortfall-1sp

%-----------------------------------------------------------------------------------------------------------------
% Finally, everything is set up. Here we go...
%
\begin{document}
\begin{slide}
%-----------------------------------------------------------------------------------------------------------------
%
\makeslidetitle{\macroname{stepwise} Example: An Aligned Equation}\label{Sec:ExEq}
% In the following, an aligned system of equations is built incrementally. We use the custom command \liststepwise to
% avoid glitches in vertical spacing.
%
\liststepwise%
{%
  %
  % This is just for compressing the equations so they can be squeezed on one slide.
  %
  \fontsize{7.8pt}{9pt}\selectfont
  \renewcommand{\arraystretch}{0}%
  \setlength{\arraycolsep}{0pt}%
  \setlength{\abovedisplayskip}{0pt}%
  \setlength{\belowdisplayskip}{0pt}%
  %
  % \highlightboxed will be used for underlaying some formulas with color. To minimize overlap, the width of the outer
  % frame is reduced.
  \setlength{\highlightboxsep}{1pt}%
  %
  \begin{align}
    \lefteqn
    {%
      \min
      \left(
        % The nested braces are filled `from outer to inner'. This means nesting a lot of steps inside each other...
        % The outermost brace is displayed from the outset.
        % The first step (which follows right here) displays the next inner brace (the first argument of \min), filled
        % with an almost `empty' array (apart from one comma and some dots).
        % \bstep is used to get appropriate white space when the step is not yet active.
        \bstep
        {\max
          \left(
            \begin{array}{l}
              % The next two steps fill in the lines of the array.
              \bstep{\min\left(F'(x),\min\left(F_1(x),G_1(y)\right)\right)},\\[-2ex]
              \vdots\\
              \bstep{\min\left(F'(x),\min\left(F_n(x),G_n(y)\right)\right)}
            \end{array}
          \right)
          },
        % After the first brace is filled, the next step provides the second argument of \min.
        \bstep{\min\left(G_i(y),H_i(z)\right)}
      \right)
      }
    &
    % The next couple of steps will create the remaining lines of the aligned equations. These need to be
    % insubstantial (as is the default for \liststepwise), because & can't go in a box.
    % As a consequence, the horizontal alignment cannot kick in until the last step is performed. This would make the
    % alignment `flicker' sidewise.
    % So we have to bite the bullet and duplicate the widest entry here (invisibly), so that the horizontal alignment
    % is constant during all steps. *sigh*
    \phantom
    {%
      {}=
      \min
      \left(
        F'(x),
        \min
        \left(
          \max
          \left(
            \begin{array}{l}
              \min\left(F_1(x),\min\left(G_1(y),G_i(y)\right)\right),\\[-1.5ex]
              \vdots\\[-.5ex]
              \min\left(F_n(x),\min\left(G_n(y),G_i(y)\right)\right)
            \end{array}
          \right),
          H_i(z)
        \right)
      \right)
      }
    % The next step displays two lines at a time, but incompletely, i.e. some parts are missing (which are inside
    % nested calls of \bstep).
    % This way, it is demonstrated how the arguments of the nested \min's are reordered.
    \step
    {%
      \\
      &=
      \max
      \left(
        % The macro \activatestep is used by \stepwise to `wrap' the argument of a \bstep command at the _first_ time
        % it appears.
        % Usually, it does nothing. Now, we redefine it to highlight its background, so it is easier to spot the
        % places where the additional arguments were inserted.
        \let\activatestep\highlightboxed
        \begin{array}{l}
          \min
          \left(
            % The inner \bstep's display the missing arguments, which are completely identical in both lines.
            % It is intended that all the missing arguments appear at the same time, so \rebstep is used for the
            % remaining arguments which have been left out.
            \min\left(\bstep{F'(x)},\min\left(\rebstep{F_1(x),G_1(y)}\right)\right),\min\left(G_i(y),H_i(z)\right)
          \right),\\[-2ex]
          \vdots\\[-1ex]
          \min
          \left(
            \min\left(\rebstep{F'(x)},\min\left(\rebstep{F_n(x),G_n(y)}\right)\right),\min\left(G_i(y),H_i(z)\right)
          \right)
        \end{array}
      \right)
      \\
      &=
      \max
      \left(
        \let\activatestep\highlightboxed
        \begin{array}{l}
          \min
          \left(
            \min\left(
              % Here are the remaining arguments of \min which are all to be displayed in one step (together with
              % those from the previous line).
              \rebstep{F'(x)},\min\left(\rebstep{F_1(x)},\min\left(\rebstep{G_1(y)},G_i(y)\right)\right)
            \right),
            H_i(z)
          \right),\\[-2.5ex]
          \vdots\\[-1.5ex]
          \min
          \left(
            \min\left(
              \rebstep{F'(x)},\min\left(\rebstep{F_n(x)},\min\left(\rebstep{G_n(y)},G_i(y)\right)\right)
            \right),
            H_i(z)
          \right)
        \end{array}
      \right)
      }
    \step
    {%
      \\
      &=
      \min
      \left(
        F'(x),
        \min
        \left(
          \max
          \left(
            \begin{array}{l}
              \min\left(F_1(x),\min\left(G_1(y),G_i(y)\right)\right),\\[-1.5ex]
              \vdots\\[-.5ex]
              \min\left(F_n(x),\min\left(G_n(y),G_i(y)\right)\right)
            \end{array}
          \right),
          H_i(z)
        \right)
      \right)
      }
  \end{align}
  }%
  \newslide
\end{slide}
\end{document}
\endinput
%%
%% End of file `mathexample.tex'.
