\documentclass[30pt,landscape]{foils}
\usepackage[english,german]{babel}  % language support
\usepackage[latin1]{inputenc}       % allow Latin1 characters
\usepackage[pdftex]{color}
\usepackage[pdftex]{geometry}
\geometry{headsep=3ex,hscale=0.9}
\usepackage{hyperref}
\hypersetup{pdftitle={PPower4Beispiel},
  pdfsubject={Ein Beispiel zur Anwendung von PPower4},
  pdfauthor={Klaus Guntermann, FG Systemprogrammierung,
   TU Darmstadt <guntermann@iti.informatik.tu-darmstadt.de>},
  pdfkeywords={pdftex, acrobat},
  pdfpagemode={FullScreen}
  }
\usepackage{pause}
\usepackage{background}
\usepackage{pp4slide}
\begin{document}
\definecolor{bgblue}{rgb}{0.04,0.39,0.53}
\vpagecolor{bgblue}

\foilhead{Ein Beispiel mit verlaufendem Hintergrund}
\begin{itemize}
\item Aufz�hlungen\pause
  \begin{itemize}
  \item werden geschachtelt\pause
  \item und mit  Symbolen markiert
    \hypertarget{Anfang}{} %Markierung irgendwo auf der Seite
    \begin{itemize}
    \item auch in dieser Tiefe\pause
    \end{itemize}
  \item auf verschiedenen Ebenen\pause
  \end{itemize}
\item mit Formeln wie $\sum_{i=0}^\infty a_i\cdot x^i$

\end{itemize}

\foilhead{Andere Seiten�berg�nge}
\hypersetup{pdfpagetransition=Dissolve}
\begin{itemize}
\item Anderen Seiten�bergang einschalten\pause
\item Aber auch explizit wieder ausschalten
\end{itemize}

\foilhead[2ex]{Das war's schon}
\hypersetup{pdfpagetransition=R}
\begin{center}
Das soll als Beispiel reichen.

Auf Wunsch geht es \hyperlink{Anfang}{zur�ck} zum Anfang.
\end{center}
\end{document}
