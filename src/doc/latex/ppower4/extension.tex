\documentclass[30pt,landscape]{foils}
\usepackage[english,german]{babel}  % language support for german/english
\usepackage[latin1]{inputenc}       % allow Latin1 characters
\usepackage{ifvtex}
\usepackage{ifpdf}
% Using vtex we most probably also do create pdf...
\ifvtexpdf\pdftrue\fi
\ifpdf
\usepackage{pause}               % loads also color.sty
\usepackage{background}
\usepackage{graphicx}            % for including graphics
\usepackage{geometry}
\usepackage{hyperref}
\else
\usepackage[dvipdfm]{pause}      % loads also color.sty
\usepackage[dvipdfm]{background}
\usepackage[dvips]{graphicx}
\usepackage[dvips]{geometry}
\usepackage[dvipdfm]{hyperref}
\fi
\usepackage{pp4slide}
\usepackage{pp4link}
\geometry{headsep=3ex,hscale=0.9}
\hypersetup{pdftitle={pp4extensions},
  pdfsubject={Extensions to PPower4},
  pdfauthor={Klaus Guntermann, FG Systemprogrammierung, TU Darmstadt
  <guntermann@iti.informatik.tu-darmstadt.de>},
  pdfkeywords={pdftex, acrobat, ppower4},
  pdfpagemode={FullScreen},
  colorlinks={true},
  linkcolor={red}
  }
\begin{document}

{\Large\normalcolor\bf
  Extensions to PPower4\\
  \null\hfill enhancing Presentations\break}

{\noindent\small
There was enough demand to expand the page transition facilities of 
PPower4. This requires additional features in the post processor,
because it has to add the transitions for the intermediate pages.\\
In this document we show which effects are provided and how they are
specified. Not all of them are useful in a partial build, though.\\
Finally we introduce an additional \toplink{infofirst}{hyperlink target}
for partial builds.\pause}

{\tiny
Hit Return/Enter/PageDown to go on...\hfill\pauselevel{=1}}

\foilhead{How can one use it?}
\raggedright
\begin{itemize}
\item One can request vertical blinds
   with \verb|\pauseVBlinds|.\pauseVBlinds
\item Use \verb|\pauseReplace| to switch back to ``normal''.\pauseReplace
\item Also horizontal blinds can be used. Use
  \verb|\pauseHBlinds|.\pauseHBlinds
\end{itemize}{\small
The transition selections made by the variants of \\
\verb|\pause| do not affect the ``real'' page transitions.\\
Use the features of \verb|hyperref| to set those.\endgraf}

\foilhead{More page transitions}
\begin{itemize}
\item Yes, there is also \verb|\pauseDissolve|.\pauseDissolve
\item The variants of Split are \verb|\pauseVOSplit|,\pauseVOSplit
\item \verb|\pauseVISplit|,\dotfill See it?\pauseVISplit\null
\item \verb|\pauseHOSplit|,\dotfill Hi!\pauseHOSplit\null
\item and \verb|\pauseHISplit|.\dotfill Wow!\pauseHISplit\null
\end{itemize}
Can \dotfill{} we \dotfill{} continue?\null
\foilhead{Still more transitions}
\begin{itemize}
\item Box modes are \verb|\pauseOBox|\pauseOBox
\item and \verb|\pauseIBox|.\dotfill Oops?\pauseIBox\null
\item \verb|\pauseWipe| must be used with an argument (e.\,g. \{90\}\pauseWipe{90},
\item 0, 180 and 270). We try also \{180\}\dotfill\pauseWipe{180}\null
\end{itemize}
Do \dotfill{} you \dotfill{} want \dotfill{} more?\null
\foilhead{But now we are almost done with transitions}
\begin{itemize}\hypertarget{first.5}
\item \verb|\pauseGlitter| also needs an argument, which may
  be~\{0\}\hfill  {\tiny In case you want to \Acrobatmenu{GoBack}{go back...}}
  \pauseGlitter{0}
\item or 270\dotfill Yeah!\pauseGlitter{270}\null
\item or 315\dotfill Done.\pauseGlitter{315}\null
\item There is still the normal \verb|\pause| command. \pause
  It will keep the last transition mode selected.
\end{itemize}
\foilhead{More news}\toptarget{infofirst}
{\small
Since version 0.8 of PPower4 you can also go to the first
partial build created for a slide/page. With the old version you could
only link to the completely built slide.
\\
We show this with the
\hyperlink{first.5}{start} (\verb|\hyperlink{first.5}{start}|)
of slide 5 and with its
\hyperlink{page.5}{end} (\verb|\hyperlink{page.5}{end}|).

Note that the link to the start of a slide will need to be defined
with a name \verb|first.|$n$ and that it will not work before
postprocessing.\\
It is recommended to use \verb|pp4link.sty| and the
corresponding commands \verb|\toptarget| and \verb|\toplink|.

Thank you for using/considering PPower4. Enjoy!\par}
\end{document}
