\documentclass{hitec}
\newcommand{\HT}{\textsc{\raisebox{0.1em}{h}\raisebox{-0.1em}{i}%
	\raisebox{0.1em}{t}\raisebox{-0.1em}{e}\raisebox{0.1em}{c} }}
\title{The \HT class}
\author{Eli Billauer}
\company{The Company, Ltd}
\confidential{\textbf{-- Unlimited Distribution --}}
\usepackage{hyperref} % This line is readily ommited of it makes trouble
\begin{document}
\maketitle
\section{The general idea}
This short paper is mainly a demonstration page to show what the papers written with
this class will look like.

The document class is a hack on the well-known \texttt{article} class, where pieces of \TeX
\hspace{0pt} code has been stolen from a couple of other classes.

This class completes, in my opinion, the set of tools needed to use \LaTeX \hspace{0pt}
in a hi-tec environment, where Microsoft rules too often. I'm delighted by the fact
that Xemacs and \LaTeX \hspace{0pt} run in a win32 environment, and that proper pdf's
can be produced using utilities such as \texttt{pdflatex} or \texttt{dvipdfm}.
These allow me to get a Linux-feeling even when running Windows. Even better: I can
write technical documents the way I like, and produce pdf's that noone will complain about.

But I discovered that there was no way to escape the academic look of the well-known
\texttt{article} document class. There are many other classes around, yes, but almost
all of them smell quite the same. Academy is not a four-letter word, but when a paper is
submitted to your boss, he better not get the wrong impression before even reading it.

So, in order to solve \emph{my} problem of giving my documents a nice outfit, I started
the adventure of modifying \texttt{article}. While doing that, I understood why noone
else has yet published such a class...

\section{hitec vs. article}
Papers that were compiled neatly as \texttt{article} are expected to give a fairly nice
result right away after changing to \HT. The only exception is that \HT doesn't work
with two-column documents. Such documents' compilation will be aborted with an error
message.

Aside from that, the \HT class behaves a bit differently regarding document information:
\begin{itemize}
\item The \verb+\author+ command \emph{can not} be repeated to
present	more than one author. This wasn't an attempt to reflect the loneliness of hi-tec
workers, as opposed to the academic world, where they always work in pairs. The reason
is technical: The author's name appears on all pages (hi-tec, right?) and there's no place
for a list of people. In the case of multiple authors (did you waste time working together?)
write them all in a single line. Remember that omitting an author can be a very painful
mistake, and this will happen with no warning when switching to \HT.
%
\item The \verb+\company+ command allows you not only to tell who you are,
but also who you're working for.
%
\item The \verb+\confidential+ command has been added, to allow companies
to mark their papers as confidential.
\end{itemize}
\section{Just a small tip}
If a pdf is your final target, going \verb+\usepackage{hyperref}+
in the beginning of your document is very recommended. The \texttt{hyperref} package
will not only create web-like links where there are references to equations or sections,
but it also creates the well-known bookmark list.
\section{Summary}
The \HT class was designed to allow \LaTeX \hspace{0pt} to produce papers that suite
the hi-tec world, in functionality and appearance. Together with other free software,
a Windows-running PC can become a comfortable platform for creating impressive pdf documents
using well-tested tools.
\end{document}
