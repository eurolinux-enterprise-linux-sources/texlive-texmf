


\documentclass[12pt]{article}
\usepackage{bussproofs}
\usepackage{amssymb}
\usepackage{latexsym}

% This is the "centered" symbol
\def\fCenter{{\mbox{\Large$\rightarrow$}}}

% Optional to turn on the short abbreviations
\EnableBpAbbreviations

% \alwaysRootAtTop  % makes proofs upside down
% \alwaysRootAtBottom % -- this is the default setting

\begin{document}
\thispagestyle{empty}

When the {\tt bussproofs.sty} code was first written, the
only documention for the {\tt bussproofs} style was in the
comments at the beginning of the style file {\tt bussproofs.sty}.
But recently (July 2004), Peter Smith has written an excellent
exposition of {\tt bussproofs.sty}, presently available
at
\begin{center}
\tt
http://www.phil.cam.ac.uk/teaching\_staff/Smith/LaTeX/nd.html
\end{center}

The present document is
a sample \LaTeX{} file that was created for testing
purposes while writing the {\tt bussproofs} code and you might
find that it useful as an example of how to
use special features of the style.

Author: Sam Buss \hspace*{1in} Email: {\tt sbuss@ucsd.edu}.
\vspace*{0.25in}

Here is some text.
\begin{center}
\Axiom$\Gamma^\prime\fCenter\Delta,A,A$
\LeftLabel{Weakening}\RightLabel{}
\UnaryInf$\lnot A,\Gamma^\prime \fCenter \Delta, A$
\LeftLabel{.}\RightLabel{\small $\lor$:right}
\UnaryInf$\lnot A,\lnot A,\Gamma^\prime \fCenter \Delta$
\LeftLabel{eigenvariable $x$}\RightLabel{$\forall$:right}
\UnaryInf$\Gamma \fCenter \Delta$
\DisplayProof
\end{center}
Here is more text.
\begin{prooftree}
\alwaysDashedLine
\alwaysDoubleLine
\Axiom$\Delta\fCenter\Pi$
\Axiom$\Gamma^\prime\fCenter\Delta,A$
\dottedLine
\singleLine
\UnaryInf$\lnot A,\Gamma^\prime \fCenter \Delta$
\UnaryInf$\lnot A,\lnot A,\Gamma^\prime \fCenter \Delta$
\singleLine
\UnaryInf$\Gamma \fCenter \Delta$
\BinaryInf$\Gamma,\Pi,A \fCenter \Delta, \Delta,B$
\Axiom$\fCenter\mbox{\rm Hi there}$
\singleLine
\BinaryInf$\Gamma\fCenter\Delta$
\end{prooftree}
\begin{center}
\alwaysDoubleLine
\AX$\Delta\fCenter\Pi$
\AX$\Gamma^\prime\fCenter\Delta,A$
\singleLine
\UI$\lnot A,\Gamma^\prime \fCenter \Delta$
\UI$\lnot A,\lnot A,\Gamma^\prime \fCenter \Delta$
\singleLine
\UI$\Gamma \fCenter \Delta$
\LL{.}\RightLabel{$\lor$:left}
\BI$\Gamma,\Pi,A \fCenter \Delta, \Delta,B$
\AXC{Hi there}
\singleLine
\BI$\Gamma\fCenter\Delta$
\DisplayProof
\end{center}
The above examples show `displayed' proofs.
On the other hand,
for putting proofs inline instead of displayed,
it is also permissable to put proofs into text rather than into
centered environments.  For example, one can write a proof right
here:
\centerAlignProof  %Which ever one of these is LAST sets the vertical alignment
\bottomAlignProof  %Try commenting out all but one of these three lines.
\normalAlignProof
\Axiom$\Gamma^\prime\fCenter\Delta,A$
\doubleLine
\UnaryInf$\lnot A,\Gamma^\prime \fCenter \Delta$
\UnaryInf$\lnot A,\lnot A,\Gamma^\prime \fCenter \Delta$
\doubleLine
\UnaryInf$\Gamma \fCenter \Delta$
\Axiom$\Delta\fCenter\Pi$
\kernHyps{-.5in}\insertBetweenHyps{\hskip-.25in}
\BinaryInf$\Gamma,\Pi,A \fCenter \Delta, \Delta,B$
\Axiom$\fCenter\mbox{\rm Hi there}$
\doubleLine
\BinaryInf$\Gamma\fCenter\Delta$
\DisplayProof{}         %% NOTE THE USE OF "{}"
although of course the proof is quite big compared to the text.
There is no reason you could not add \verb+\subscriptstyle+ or \verb+\small+
commands to the lines of the proofs to shrink things down.
The previous proof looks strange because it is illustrated the usage
of \verb+\kernHyps+ and \verb+\insertBetweenHyps+.  Finally
a 3-ary inference with a usage of \verb+\noLine+ is:
\begin{prooftree}
\AxiomC{$A\lor B$}
\AxiomC{$[A]$}
\noLine
\UnaryInfC{$C$}
\AxiomC{$[B]$}
\noLine
\UnaryInfC{$C$}
\TrinaryInfC{$C$}
\end{prooftree}


Two more examples:
\begin{center}
\AxiomC{A,B} \AxiomC{C} \BIC{A-B-C}
\AXC{good} \AXC{bad} \BIC{$\frac{good}{bad}$A}
\BIC{done}
\DP
\end{center}

\begin{center}
\Axiom$\fCenter A,B$ \Axiom$\fCenter C$ \BI$\fCenter A-B-C$
\AX$\fCenter good$ \AX$\fCenter bad$ \BI$\fCenter\frac{good}{bad}A$
\BI$\fCenter done$
\DP
\end{center}


Small labels can be created as in the third proof below:
\[
\AxiomC{A}
\RightLabel{1}
\UnaryInfC{$\bot$}
\DisplayProof
\quad
\AxiomC{A}
\RightLabel{(2)}
\UnaryInfC{$\bot$}
\DisplayProof
\quad
\AxiomC{A}
\RightLabel{\scriptsize(3)}
\UnaryInfC{$\bot$}
\DisplayProof
\quad
\AxiomC{A}
\LeftLabel{(4)}
\UnaryInfC{$\bot$}
\DisplayProof
\]
Arnold's example of inline proof: The figure
\AxiomC{$\dots\Gamma_\iota\dots(\iota\in I)$}
\RightLabel{$I$}
\UnaryInfC{$\Gamma$}
\DisplayProof{}
is called ...
%% NOTE THE USE OF "{}" after \DisplayProof to keep LaTeX from eating subsequent spaces!

\bigskip

\noindent{\bf Upside down proofs}
Proofs can be rendered upside down.  For instance the proof above with 
a 3-ary inference can be made upside down by giving the command
\verb+\rootAtTop+.   This is useful if you want your proof trees to have
their root at the top.

\alwaysRootAtTop    % Henceforth puts the root at the top

\begin{prooftree}
\AxiomC{$A\lor B$}
\AxiomC{$[A]$}
\noLine
\UnaryInfC{$C$}
\AxiomC{$[B]$}
\noLine
\UnaryInfC{$C$}
\TrinaryInfC{$C$}
\end{prooftree}

To make the change permanent for the rest of your document, use the
command  \verb+\alwaysRootAtTop+

\bigskip

\noindent
Another upside-down example, from Alex Hertel:
\bigskip

\hbox{\kern-2cm%
\tiny
\rootAtTop
\AxiomC{$= 1$}
\noLine
\UnaryInfC{$( ( (1 \wedge 1) \vee 0 ) \vee ( (0 \wedge 0) \vee 0 ) )$}
\AxiomC{$= 1$}
\noLine
\UnaryInfC{$( ( (1 \wedge 1) \vee 1 ) \vee ( (0 \wedge 0) \vee 1 ) )$}
\LeftLabel{$[0/z]$}
\RightLabel{$[1/z]$}
\BinaryInfC{$\forall z ( ( (1 \wedge 1) \vee z ) \vee ( (0 \wedge 0) \vee z ) )$}
\RightLabel{$[0/y]$}
\UnaryInfC{$\exists y \forall z ( ( (1 \wedge \neg y) \vee z ) \vee ( (0 \wedge y) \vee z ) )$}
\RightLabel{$[0/y]$}
\UnaryInfC{$\exists y \forall z ( ( (1 \wedge \neg y) \vee z ) \vee ( (0 \wedge y) \vee z ) )$}

\AxiomC{$= 1$}
\noLine
\UnaryInfC{$( ( (0 \wedge 0) \vee 0 ) \vee ( (1 \wedge 1) \vee 0 ) )$}
\AxiomC{$= 1$}
\noLine
\UnaryInfC{$( ( (0 \wedge 0) \vee 1 ) \vee ( (1 \wedge 1) \vee 1 ) )$}
\LeftLabel{$[0/z]$}
\RightLabel{$[1/z]$}
\BinaryInfC{$\forall z ( ( (0 \wedge 0) \vee z ) \vee ( (1 \wedge 1) \vee z ) )$}
\RightLabel{$[1/y]$}
\UnaryInfC{$\exists y \forall z ( ( (0 \wedge \neg y) \vee z ) \vee ( (1 \wedge y) \vee z ) )$}

\LeftLabel{$[0/x]$}
\RightLabel{$[1/x]$}
\BinaryInfC{$\forall x \exists y \forall z ( ( (\neg x \wedge \neg y) \vee z ) \vee ( (x \wedge y) \vee z ) )$}
\DisplayProof
}  % End of \hbox

\bigskip

\newbox\gnBoxA
\newdimen\gnCornerHgt
\setbox\gnBoxA=\hbox{$\ulcorner$}
\global\gnCornerHgt=\ht\gnBoxA
\newdimen\gnArgHgt

\def\Godelnum #1{%
    \setbox\gnBoxA=\hbox{$#1$}%
    \gnArgHgt=\ht\gnBoxA%
    \ifnum \gnArgHgt<\gnCornerHgt
        \gnArgHgt=0pt%
    \else
        \advance \gnArgHgt by -\gnCornerHgt%
    \fi
    \raise\gnArgHgt\hbox{$\ulcorner$} \box\gnBoxA %
        \raise\gnArgHgt\hbox{$\urcorner$}}

This last sentence has nothing to do with proof trees, but shows my
macros for G\"odel number delimeters:
$\Godelnum A, \Godelnum B$
$\Godelnum s$
\end{document}
