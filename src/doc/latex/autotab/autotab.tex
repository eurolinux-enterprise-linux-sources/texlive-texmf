%format latex
%documentstyle[german,11pt,supertab,autotab]{darticle}
\documentstyle[german,11pt,supertab,autotab]{artdaf} %02.02.91
 
\textheight 24cm
\textwidth 17cm
\topmargin -2cm
\oddsidemargin -1cm
\parskip 6pt
 
\begin{document}
\begin{center}
\Large\bf Generating tabulars using the \\
          {\tt autotab} Style           \\[.5cm]
\normalsize Gabriele Kruljac \\
            5. Oktober 1989
\end{center}
This is the description of the {\tt autotab} style. It offers some
new commands to generate tabulars nearly automatically.
 
The style itself is called as \verb|\documenstyle| option
\verb|\documentstyle[...,autotab,...]{darticle}|. The tabular
creating commands are called inside a {\tt tabular} or
{\tt supertabular} environment. Syntax:
\begin{center}
\begin{tabular}{l p{10cm}}
\verb|\readtabline{...}| & reads from data file and generates
                          the tabular entries; argument is
                          the number of columns = number of records
                          which bild one tabular line \\
\verb|\autotabline|      & is the built tabular entry, inserts it's
                           into the tabular  at the specified point
\end{tabular}
\end{center}
 
Every time \verb|\readtabline| is called it asks for the name of
the input data file and starts reading. The style reads as much records
as specified by the argument of the \verb|\readtabline| command,
puts them into \verb|\autotabline| connected with ampersands (\verb|&|)
and adds \verb|\\| to the last of these records.
This algorithm is repeated until end-of-file is found.
 
Using {\tt autotab} in conjunction with the {\tt supertabular}
environment offers the possibility to generate tabulars without regard
to necessary pagebreaks, because they will be inserted automatically
by {\tt supertabular} (see separate description).
 
\subsection*{Example}
The data is stored in file {\tt sonderdr.dat}. \verb|\readtabline|
results in an online dialog question:
\begin{center}
\begin{tabular}{l}
\tt Please type the name of the tabular input data file:\\
\verb|\inputfile=|
\end{tabular}
\end{center}
The user types:\hfill{\tt sonderdr.dat}.\hspace*{\fill}
 
\noindent The data look like:
\begin{verbatim}
    265/89
    Study of Plasma Potential Effects in a 40 MHz ...
    Spectrochim. Acta
    1989
    44B
    219-228
    ww
    264/89
    Atomic Emission and Atomic Absorption Spectrometric ...
    .
    .
    .
    .
    .
    263/89
    Comparative Surface and Bulk Analysis of Oxygen in Si3N4 ...
    Fresenius Z.Anal.Chem.
    1989
    333
    502-506
    WW
    262/89
\end{verbatim}
Seven records build one tabular line. There must be a blank record for each empty entry (marked by . in the above example).
 
\noindent Here is the definition of the tabular:
\begin{verbatim}
    \small\renewcommand{\arraystretch}{.8}
    \tablehead{\hline Publ-Nr. & \multicolumn{1}{c|}{Titel}
               & ersch. & Jahr & Bd. & Seite & Institut \\
               \hline}
    \tabletail{\hline}
    \begin{center}
    \begin{supertabular}{|l|p{5cm}|p{3cm}|l|l|l|l|}
    \noalign{\readtabline{7}}
    \autotabline
    \end{supertabular}
    \end{center}
\end{verbatim}
 
\noindent And here is the result:
\small\renewcommand{\arraystretch}{.8}
\tablehead{\hline
           Publ-Nr. & \multicolumn{1}{c|}{Titel}
           & ersch.  & Jahr & Bd. & Seite & Institut \\ \hline}
\tabletail{\hline}
\begin{center}
\begin{supertabular}{|l|p{5cm}|p{3cm}|l|l|l|l|}
\noalign{\readtabline{7}}
\autotabline
\end{supertabular}
\end{center}
\end{document}
