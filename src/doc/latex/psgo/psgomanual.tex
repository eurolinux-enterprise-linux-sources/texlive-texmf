%% psgomanual.tex
%% Copyright 2003,2001 V. Bos
%
% This program may be distributed and/or modified under the
% conditions of the LaTeX Project Public License, either version 1.2
% of this license or (at your option) any later version.
% The latest version of this license is in
%   http://www.latex-project.org/lppl.txt
% and version 1.2 or later is part of all distributions of LaTeX 
% version 1999/12/01 or later.
%
% This program consists of the files:
%
%   psgo.sty
%   psgomanual.tex
%   psgomanual.ps
%
% RCS-ID: $Id: psgomanual.tex,v 1.6 2004/02/24 10:17:23 vbos Exp $
%
\documentclass[a4paper]{article}

\usepackage{psgo,array,url}
\newcommand{\pstricks}{\textsf{PSTricks}}

\newcommand{\commandname}[1]{{\ttfamily\symbol{'134}#1}}
% The following code is taken from the doc package. It defines a global 
% macro \bslash that produces a bslash (if present in the current font). 
\makeatletter
{\catcode`\|=\z@ \catcode`\\=12 |gdef|bslash{\}}
\makeatother
\newcommand{\cmd}[1]{\texttt{\bslash #1}}

\title{\psgo\\{\normalfont\normalsize Typesetting Go Diagrams with \pstricks}}
\author{Victor Bos\\email: vbos@abo.fi}
\date{October 06, 2001\\(last updated: February 24, 2004)}


\setgounit{0.4cm}

\begin{document}
\maketitle


\section{Introduction}

The \psgo{} package provides functionality to typeset Go diagrams in
\LaTeXe. It is built on top of the \pstricks{} package, which is
nowadays available in many \LaTeX{} distributions. Although \psgo{}
does not understand the \emph{Smart Game Format} (SGF), it has support
for all graphical markup properties of SGF FF[4], see
\verb|http://www.red-bean.com/sgf/|.

\section{Download and installation}

The \psgo{} package can be found at CTAN (mirrors) in the directory
\url{graphics/pstricks/contrib/psgo/}. It is also available from

\medskip
\url{http://www.abo.fi/~vbos/psgo-frame.html}

\medskip\noindent To install the package, download the files
\texttt{psgo.sty} and \texttt{psgomanual.tex} and put them in a
directory where \LaTeX{} can find them. After that, test the
installation by running \LaTeX{} on \texttt{psgomanual.tex}. 

Note that \textsf{psgo} uses \textsf{pstricks} to draw
graphics. Therefore, the same things that apply to viewing and
printing documents with \textsf{pstricks} graphics also apply to
documents with \textsf{psgo} graphics. In particular, it is usually
better to use a PostScript viewer (e.g., \texttt{gv} or
\texttt{gsview}) instead of a DVI viewer (like \texttt{xdvi} or
\texttt{yap}).
 
\section{Go boards}

The interface of \psgo{} is based on how Go diagrams are usually
displayed in books. That is, the rows of a $19\times19$ board are
indexed by $1,2,\ldots,19$ and the columns are indexed by $A, B,
\ldots, T$ (skipping $I$). Further, the lower left corner has index
$(A, 1)$ and the upper right corner has index
$(T,19)$. Figure~\ref{fig:empty:boards} shows a $19\times 19$ board
and a $9\times9$ board. Note that the sizes of the boards are reduced
in order to fit in one figure. To re-size a Go board, the command
\verb|\setgounit| can be used. This command takes the desired
horizontal unit distance as an argument. The default horizontal unit
distance is~$0.6$cm. The vertical unit distance is computed by the
\psgo{} package. For the diagrams in this document, except for the
diagrams of Figure~\ref{fig:empty:boards}, we have set the horizontal
unit distance to~$0.4$cm (\verb|\setgounit{0.4cm}|).

\begin{figure}
\setgounit{0.3cm}
\begin{center}
\begin{tabular}{cc}
\begin{psgoboard}
\end{psgoboard} &
\begin{psgoboard}[9]
\end{psgoboard}
\end{tabular}
\caption{Different size Go boards}
\label{fig:empty:boards}
\end{center}
\end{figure}

Go boards are defined in the \verb|psboard| environment of \psgo. This
environment takes~1 optional parameter which indicates the size of the
board (default size is~19). For instance, the boards of
Figure~\ref{fig:empty:boards} were defined by:

\begin{verbatim}
\begin{psgoboard}
\end{psgoboard}
\begin{psgoboard}[9]
\end{psgoboard}
\end{verbatim}

If the indices are not desired, the starred version \verb|psgoboard*|
of the environment should be used. Figure \ref{fig:no:indices} shows
the boards of Figure~\ref{fig:empty:boards} without indices. The code
for these boards is given below.
\begin{verbatim}
\begin{psgoboard*}
\end{psgoboard*}
\begin{psgoboard*}[9]
\end{psgoboard*}
\end{verbatim}

\begin{figure}
\setgounit{0.3cm}
\begin{center}
\begin{tabular}{cc}
\begin{psgoboard*}
\end{psgoboard*} &
\begin{psgoboard*}[9]
\end{psgoboard*}
\end{tabular}
\caption{Go boards of Figure~\ref{fig:empty:boards} without indices}
\label{fig:no:indices}
\end{center}
\end{figure}


\section{Stones and moves}

There are two commands to put stones on the board and one command to
make pass moves. The first one is \verb|\stone| which takes three
parameters: the color, the column, and the row of the stone. For
example, \verb|\stone{black}{c}{4}| puts a black stone at
position~$(C,4)$ (note that in the \LaTeX{} code, the columns are
indicated by lower case characters). The \verb|\stone| command can be
used to setup a particular configuration. For instance, the
configuration of Figure~\ref{fig:ko:configuration} is defined as
follows.

\begin{verbatim}
\begin{psgoboard}[9]
\stone{white}{c}{3}
\stone{white}{e}{3}
\stone{white}{d}{2}
\stone{white}{d}{4}
\stone{black}{f}{3}
\stone{black}{e}{2}
\stone{black}{e}{4}
\end{psgoboard}
\end{verbatim}

\begin{figure}
\begin{center}
\begin{psgoboard}[9]
\stone{white}{c}{3}
\stone{white}{e}{3}
\stone{white}{d}{2}
\stone{white}{d}{4}
\stone{black}{f}{3}
\stone{black}{e}{2}
\stone{black}{e}{4}
\end{psgoboard}
\caption{Setting up a configuration}
\label{fig:ko:configuration}
\end{center}
\end{figure}


The second command is \verb|move| which takes two parameters: the
column and the row of the next move. Moves are usually numbered. The
counter that keeps track of the move number is
called~\verb|gomove|. This is a normal \LaTeX{} counter and can be
changed using ordinary \LaTeX-counter commands. The color of the
stones placed by the \verb|\move| command alternates between
successive moves. For example, \verb|\move{b}{3}| puts a stone on
position~$(B, 3)$. If this was a black move, the stone is black,
otherwise it is white. 

In addition, there is a command for pass moves: \verb|\pass|. It is
like \verb|\move| in that it increases the move number and changes the
color for the next move, but it does not draw a stone on the board.

It is time for an example. The following code generates a $9\times 9$
board with six moves. Note that the third move (move number~2) is a pass by
black. The result is depicted in Figure~\ref{fig:9x9:example}(a).

\begin{verbatim}
\begin{psgoboard}[9]
\move{c}{3}
\move{g}{7}
\pass
\move{f}{1}
\move{a}{9}
\end{psgoboard}
\end{verbatim}


As can be seen, the move numbers are displayed on the stones. The
\verb|gomove| counter is never reset by the \psgo{} package. So, the
move numbers just continue in subsequent diagrams. Of course, it is
possible to reset the value of~\verb|gomove| manually. In that case,
one should know that the \verb|\move| command increases the
\verb|gomove| counter \emph{before} it draws the stone. So, if a
diagram should start with move~0, one should issue the command
\verb|\setcounter{gomove}{-1}| just before the diagram.

If a diagram continues another diagram, the numbers on the stones
played so far are usually not desirable. Therefore, \psgo{} has
defined a starred version the move command: \verb|\move*|.  This
command does not decorate stones with move numbers and it does not
increase the \verb|gomove| counter.  For instance, if the game of
Figure~\ref{fig:9x9:example}(a) is continued, all we have to do is
copy\&paste the game played so far, replace the existing \verb|\move|
commands by \verb|\move*|, and add some new \verb|\move| (unstarred!)
commands to it. For convenience, there is also a \verb|\pass*| command; its
effect is to change the color for the next move.  The code is given
below and the result displayed in Figure~\ref{fig:9x9:example}(b).
\begin{verbatim}
\begin{psgoboard}[9]
\move*{c}{3}
\move*{g}{7}
\pass*
\move*{f}{1}
\move*{a}{9}
\move{f}{6}  % new \move commands
\move{e}{6}
\pass
\move{c}{5}
\end{psgoboard}
\end{verbatim}

\begin{figure}
\begin{center}
\begin{tabular}{cc}
\begin{psgoboard}[9]
\move{c}{3}
\move{g}{7}
\pass
\move{f}{1}
\move{a}{9}
\end{psgoboard}
&
\begin{psgoboard}[9]
\move*{c}{3}
\move*{g}{7}
\pass*%
\move*{f}{1}
\move*{a}{9}
\move{f}{6}
\move{e}{6}
\pass
\move{c}{5}%
\end{psgoboard}\\
(a) First 5 moves (2: black pass).
&
(b) Successive moves (7: white pass).
\end{tabular}
\caption{Moves on a board}
\label{fig:9x9:example}
\end{center}
\end{figure}


\section{Markers}

Empty positions on the board can be marked with the command
\verb|\markpos|. This command takes three parameters, being, the
marker, the column, and the row. Available markers and the commands to
generate them are listed in Table~\ref{tbl:markers}. Each marker is
illustrated at position~$(B,2)$ on a~$3\times3$ board. Note that the
label marker command, \verb|\marklb|, takes one argument, being the
label. It is possible, though not advisable, to add more than one
marker to an empty position.

\begin{table}
\begin{center}
\begin{tabular}{|c|l|l|l|}
\hline
Diagram & \psgo{} Command & Description & Example\\
\hline\hline 
\raisebox{-0.5\height}{\begin{psgoboard*}[3]\markpos{\markma}{b}{2}\end{psgoboard*}} &
\commandname{markma}& Cross &
{\ttfamily\commandname{markpos}\{\commandname{markma}\}\{b\}\{2\}}\\\hline
%
\raisebox{-0.5\height}{\begin{psgoboard*}[3]\markpos{\marktr}{b}{2}\end{psgoboard*}} &
\commandname{marktr}& Triangle &
{\ttfamily\commandname{markpos}\{\commandname{marktr}\}\{b\}\{2\}}\\\hline
%
\raisebox{-0.5\height}{\begin{psgoboard*}[3]\markpos{\markcr}{b}{2}\end{psgoboard*}} &
\commandname{markcr}& Circle &
{\ttfamily\commandname{markpos}\{\commandname{markcr}\}\{b\}\{2\}}\\\hline
%
\raisebox{-0.5\height}{\begin{psgoboard*}[3]\markpos{\marksq}{b}{2}\end{psgoboard*}} &
\commandname{marksq}& Open square &
{\ttfamily\commandname{markpos}\{\commandname{marksq}\}\{b\}\{2\}}\\\hline
%
\raisebox{-0.5\height}{\begin{psgoboard*}[3]\markpos{\marklb{A}}{b}{2}\end{psgoboard*}} &
\commandname{marklb\{\#1}\}& Label &
{\ttfamily\commandname{markpos}\{\commandname{marklb}\{A\}\}\{b\}\{2\}}\\\hline
%
\raisebox{-0.5\height}{\begin{psgoboard*}[3]\markpos{\marksl}{b}{2}\end{psgoboard*}} &
\commandname{marksl}& Filled square &
{\ttfamily\commandname{markpos}\{\commandname{marksl}\}\{b\}\{2\}}\\\hline
%
\raisebox{-0.5\height}{\begin{psgoboard*}[3]\markpos{\markdd}{b}{2}\end{psgoboard*}} &
\commandname{markdd}& Hatched lines &
{\ttfamily\commandname{markpos}\{\commandname{markdd}\}\{b\}\{2\}}\\\hline
\end{tabular}
\caption{Markers on empty positions}
\label{tbl:markers}
\end{center}
\end{table}

Stones can be marked too. To this end, the commands \verb|\stone| and
\verb|\move| take an optional parameter representing the marker of the
stone.  For example, \verb|\stone[\markma]{black}{b}{4}| puts a black
stone marked with a cross at
position~$(B,4)$. Table~\ref{tbl:stone:markers} shows how stones can
be marked. It is possible, though not advisable, to add more than one
marker to a stone or to add markers to numbered stones.

\begin{table}
\begin{center}
\begin{tabular}{|c|l|l|p{0.5\linewidth}|}
\hline
Diagram & \psgo{} Command & Description & Example\\
\hline\hline 
\raisebox{-0.5\height}{\begin{psgoboard*}[4]\stone[\markma]{black}{b}{2}\stone[\markma]{white}{c}{3}\end{psgoboard*}} &
\commandname{markma}& Cross &
{\ttfamily\commandname{stone}[\commandname{markma}]\{black\}\{b\}\{2\}
          \commandname{stone}[\commandname{markma}]\{white\}\{c\}\{3\}}\\\hline
%
\raisebox{-0.5\height}{\begin{psgoboard*}[4]\stone[\marktr]{black}{b}{2}\stone[\marktr]{white}{c}{3}\end{psgoboard*}} &
\commandname{marktr}& Triangle &
{\ttfamily\commandname{stone}[\commandname{marktr}]\{black\}\{b\}\{2\}
          \commandname{stone}[\commandname{marktr}]\{white\}\{c\}\{3\}}\\\hline
%
\raisebox{-0.5\height}{\begin{psgoboard*}[4]\stone[\markcr]{black}{b}{2}\stone[\markcr]{white}{c}{3}\end{psgoboard*}} &
\commandname{markcr}& Circle &
{\ttfamily\commandname{stone}[\commandname{markcr}]\{black\}\{b\}\{2\}
          \commandname{stone}[\commandname{markcr}]\{white\}\{c\}\{3\}}\\\hline
%
\raisebox{-0.5\height}{\begin{psgoboard*}[4]\stone[\marksq]{black}{b}{2}\stone[\marksq]{white}{c}{3}\end{psgoboard*}} &
\commandname{marksq}& Open square &
{\ttfamily\commandname{stone}[\commandname{marksq}]\{black\}\{b\}\{2\}
          \commandname{stone}[\commandname{marksq}]\{white\}\{c\}\{3\}}\\\hline
%
\raisebox{-0.5\height}{\begin{psgoboard*}[4]\stone[\marklb{A}]{black}{b}{2}\stone[\marklb{B}]{white}{c}{3}\end{psgoboard*}} &
\commandname{marklb\{\#1}\}& Label &
{\ttfamily\commandname{stone}[\commandname{marklb}\{A\}]\{black\}\{b\}\{2\}
          \commandname{stone}[\commandname{marklb}\{B\}]\{white\}\{c\}\{3\}}\\\hline
%
\raisebox{-0.5\height}{\begin{psgoboard*}[4]\stone[\marksl]{black}{b}{2}\stone[\marksl]{white}{c}{3}\end{psgoboard*}} &
\commandname{marksl}& Filled square &
{\ttfamily\commandname{stone}[\commandname{marksl}]\{black\}\{b\}\{2\}
          \commandname{stone}[\commandname{marksl}]\{white\}\{c\}\{3\}}\\\hline
%
\raisebox{-0.5\height}{\begin{psgoboard*}[4]\stone[\markdd]{black}{b}{2}\stone[\markdd]{white}{c}{3}\end{psgoboard*}} &
\commandname{markdd}& Hatched lines &
{\ttfamily\commandname{stone}[\commandname{markdd}]\{black\}\{b\}\{2\}
          \commandname{stone}[\commandname{markdd}]\{white\}\{c\}\{3\}}\\\hline
\end{tabular}
\caption{Markers on stones}
\label{tbl:stone:markers}
\end{center}
\end{table}

\section{Lines and arrows}

In addition to markers, it is possible to add lines and arrows to the
diagrams. The command \verb|\goline| draws a line and the command
\verb|\goarrow| draws an arrow. Both commands take four parameters
indicating the column and the row of the start position and the column
and the row of the end position. That is, \verb|\goline{a}{1}{c}{5}|
draws a line from position~$(A,1)$ to position~$(C,5)$ and
\verb|\goarrow{e}{3}{b}{2}| draws an arrow from position~$(E,3)$ to
position~$(B,2)$, as illustrated in Figure~\ref{fig:more:features}(a).


\section{Partial Boards}
\label{sec:partial-boards}

It is sometimes desirable to display \emph{partial} go boards. For
instance, when describing joseki's, displaying a whole board might be
a waste of space. To this end, there is a variant of the
\texttt{psgoboard} environment called \texttt{psgopartialboard}. In
addition to the board size, this environment takes two points on the
board as parameters. These points define the lower-left and the
upper-right of a frame on the board. Only the inside
the frame will be displayed. The coordinates of the points should be
given as one parameter with syntax
%
\begin{center}
\texttt{($llx$,$lly$)($urx$,$ury$)} 
\end{center}
%
Here, $llx$ is the $x$-coordinate of the lower-left point, $lly$ is
the $y$-coordinate of the lower-left point, $llx$ is the
$x$-coordinate of the upper-right point, and $ury$ is the
$y$-coordinate of the upper-right
point. Figure~\ref{fig:part:goboards} shows examples of partial go
boards of different sizes.

\begin{figure}
\setgounit{0.3cm}
\begin{center}
\begin{tabular}[b]{cc}
\begin{psgopartialboard*}{(1,1)(6,8)}
\stone{white}{a}{3}
\stone{black}{f}{4}
\stone{white}{d}{7}
\stone{black}{b}{1}
\end{psgopartialboard*} &
\begin{psgopartialboard}[9]{(4,2)(9,9)}
\stone{white}{d}{2}
\stone{black}{f}{5}
\stone{white}{j}{8}
\end{psgopartialboard}\\[2ex]
\parbox{5.5cm}{\raggedright\small
               \texttt{\cmd{begin}\{psgopartialboard*\}\%\\
                 \qquad\{(1,1)(6,8)\}\\
                 \cmd{stone}\{white\}\{a\}\{3\}\\
                 \cmd{stone}\{black\}\{f\}\{4\}\\
                 \cmd{stone}\{white\}\{d\}\{7\}\\
                 \cmd{end}\{psgopartialboard*\}
               }}
& 
\parbox{5.5cm}{\raggedright\small
               \texttt{\cmd{begin}\{psgopartialboard\}[9]\%\\
                 \qquad\{(4,2)(9,9)\}\\
                 \cmd{stone}\{white\}\{d\}\{2\}\\
                 \cmd{stone}\{black\}\{f\}\{5\}\\
                 \cmd{stone}\{white\}\{j\}\{8\}\\
                 \cmd{end}\{psgopartialboard\}
               }}\\
\end{tabular}
\caption{Partial go boards}
\label{fig:part:goboards}
\end{center}
\end{figure}

Note that if a partial board includes an edge of the board, the
indices for that edge are printed. On the other hand, if an edge is
not included, no indices are printed. The current implementation of
partial boards uses the clipping feature of \textsf{pstricks} to cut
out part of the board specified by the lower-left and upper-right
points. That is, the partial board is drawn as a normal board, but by
using clipping, only part of the board is visible. In particular,
coordinates of moves and stones should be given as for a normal
board. An example of a partial go board with stones is given in
Figure~\ref{fig:partial:ko:configuration}.

\begin{figure}

\begin{center}
\begin{tabular}{cb{5cm}}
\begin{psgopartialboard}[9]{(4,1)(9,6)}
\stone{white}{c}{3}
\stone{white}{e}{3}
\stone{white}{d}{2}
\stone{white}{d}{4}
\stone{black}{f}{3}
\stone{black}{e}{2}
\stone{black}{e}{4}
\end{psgopartialboard}
&
\small\ttfamily
\cmd{begin}\{psgopartialboard\}[9]\{(4,1)(9,6)\}
\cmd\{stone\}\{white\}\{c\}\{3\}
\cmd{stone}\{white\}\{e\}\{3\}
\cmd{stone}\{white\}\{d\}\{2\}
\cmd{stone}\{white\}\{d\}\{4\}
\cmd{stone}\{black\}\{f\}\{3\}
\cmd{stone}\{black\}\{e\}\{2\}
\cmd{stone}\{black\}\{e\}\{4\}
\cmd{end}\{psgopartialboard\}
\end{tabular}

\caption{Part of the configuration of Figure~\ref{fig:ko:configuration}}
\label{fig:partial:ko:configuration}
\end{center}

\end{figure}





\section{Stones in text}

In comments on a Go diagram, one often sees graphical representations
of the stones on the board. Therefore, \psgo{} has facilities to
include stones in running text. For instance, we can refer to stone
\stone[4]{black} of Figure~\ref{fig:9x9:example}(a) with the
command~\verb|\stone[4]{black}|. As can be seen, in running text
(i.e., outside the \verb|psgoboard| environment), \verb|\stone| takes
two parameters. The first one is optional and denotes a number or a
marker to be placed on the stone. The second parameter denotes the
color of the stone. As another example, here is a list of various stones:
\stone[\markma]{black}, \stone[\markma]{white},
\stone[\marktr]{black}, \stone[\marktr]{white},
\stone[\markcr]{black}, \stone[\markcr]{white},
\stone[\marksq]{black}, \stone[\marksq]{white},
\stone[\marklb{A}]{black}, \stone[\marklb{A}]{white},
\stone[\marksl]{black}, \stone[\marksl]{white},
\stone[\markdd]{black}, \stone[\markdd]{white}.
This list was generated by the following code:
\begin{verbatim}
\stone[\markma]{black}, \stone[\markma]{white},
\stone[\marktr]{black}, \stone[\marktr]{white},
\stone[\markcr]{black}, \stone[\markcr]{white},
\stone[\marksq]{black}, \stone[\marksq]{white},
\stone[\marklb{A}]{black}, \stone[\marklb{A}]{white},
\stone[\marksl]{black}, \stone[\marksl]{white},
\stone[\markdd]{black}, \stone[\markdd]{white}.
\end{verbatim}


It is possible to attach ordinary \LaTeX{} labels to \emph{moves}
on a board. For instance, the digram of Figure~\ref{fig:more:features}(b) was
generated by the following code:

\begin{verbatim}
\begin{psgoboard}[5]
\move{b}{2}
\move{d}{3}\label{go:move}
\move{d}{1}
\end{psgoboard}
\end{verbatim}
As can be seen, a label \verb|{funny:go:move}| is defined after the
second move. To refer in the text to this move, the ordinary \LaTeX{}
\verb|\ref| command can be used. In this case, we type
\verb|\stone[\ref{go:move}]{white}|, which results in the
stone~\stone[\ref{go:move}]{white}, as expected. Of course, the
color should still be defined manually.

\begin{figure}
\begin{center}

\begin{tabular}{cc}
\begin{psgoboard}[5]
\goline{a}{1}{c}{5}
\goarrow{e}{3}{b}{2}
\end{psgoboard}
&
\setcounter{gomove}{-1}
\begin{psgoboard}[5]
\move{b}{2}
\move{d}{3}\label{go:move}
\move{d}{1}
\end{psgoboard}\\
(a) Lines and arrows &
(b) Labeling stones
\end{tabular}

\caption{More \psgo{} features: lines, arrows, and labels}
\label{fig:more:features}
\end{center}
\end{figure}

\section{Remarks, limitations, and known bugs}

\begin{enumerate}

\item The \verb|\pass| moves cannot be labels with \verb|\label|.

\item Version 0.15 fixes the stones-on-the-edge problem introduced in
  versions 13 and~14.

\item The \verb|\pass| command was requested by Markus Enzenberger
  (16.09.2003). It was implemented at 01.10.2003. For convenience,
  there is also a \verb|\pass*| command. Like \verb|\move*|,
  \verb|\pass*| is useful for copy\&pasting moves and passes from
  previous boards. See Figure~\ref{fig:9x9:example} for an example.

\item Lukas van de Wiel requested a possibility to generare partial
go-boards (27.09.2003). This was implemented in the
\texttt{psgopartialboard(*)} environments (see
Section~\ref{sec:partial-boards}) using the clipping features of
\textsf{pstricks}.

\item Xiaozhen Niu discovered a bug in partial go-boards (20.2.2004):
stones on the edge were not displayed completely if there are no
indices at that edge. This was fixed by setting the \verb+xoffset+ and
\verb+yoffset+ to .5\verb+\goxunit+ and .5\verb+goyunit+,
respectively.


\end{enumerate}

\end{document}
