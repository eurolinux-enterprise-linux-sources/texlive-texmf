% local02.tex -- updated for LaTeX2e 16 May 1996
%                first released 1 Sept 1993
%
% Copyright (C) 1993, 1996 by Wenzel Matiaske, mati1831@perform.ww.tu-berlin.de
%
% As input to the local LaTeX-guide "`local.tex"'.
%
% For distribution of this document see the copyright notice in the
% original sources mentioned below.
%

\makeatletter
\newif\iflocalin
\newif\ifappendixin
\@ifundefined{localin}{\localintrue}{\localinfalse}
\@ifundefined{appendixin}{\appendixinfalse}{\appendixintrue}
\@ifundefined{docdir}{\def\docdir{\dots /emtex/doc/}}{}
\makeatother

\iflocalin
   \subsubsection{The style files \texttt{paper} and \texttt{journal}}
\fi
\ifappendixin
  \subsection{The style files \texttt{paper} and \texttt{journal}}
\fi

The style files \verb|paper| and \verb|journal| are derived from the
standard class \verb|article|. In difference to the standard document
class the layout can be changed via layout options (\verb|slanted|,
\verb|bold|, \verb|sfbold|) and font commands 
(\verb|\partfont{|{\em font\/}\verb|}|, 
\verb|\sectionfont{|{\em font\/}\verb|}| etc.).

The class \texttt{paper} defines a new environment called
\texttt{keywords} and the commands
\verb|\subtitle{|\emph{text}\verb|}| and
\verb|\institution{|\emph{text}\verb|}| for the title section. Three
commands allow a small table of contents
(\verb|\smalltableofcontents|), a small lists of figures
(\verb|\smalllistoffigures|) or a small lists of tables
(\verb|\smalllistoftables|). These commands are obsolete when using the
\texttt{journal} style file.

The format \texttt{journal} typically uses a master file which \verb|\include|
the articles. The command \verb|\journalofcontents| produces a table of
articles, revisions and parts of a journal. The new commands
\verb|\shortauthor{|\emph{text}\verb|}| and
\verb|\shorttitle{|\emph{text}\verb|}| are defined for head titles
containing authors and titles. Head titles for the whole journal may be
produced with the commands
\verb|\oddrunhead{|\emph{text}\verb|}| and
\verb|\evenrunhead{|\emph{text}\verb|}|.

If you want to declare parts between the papers, you may use the command
\verb|\journalpart[|\emph{option}\verb|]{|\emph{text}\verb|}| or
\verb|\journalpart*{|\emph{text}\verb|}|.

Two new commands are especially designed for revisions. The command
\verb|\revison[|\emph{option}\verb|]{|\emph{author}\verb|}{|\emph{title}\verb|}|
takes the author and the title of the revisited book.  It produces a
subsection like headline and an entry for the table of contents. The
optional argument is used to put also the author of the revision into
the table of contents. This command is also defined in the form
\verb|\revision*|. The command \verb|\revauthor{|\emph{text}\verb|}|
may be useful to sign a revision. It allows the commands \verb|\and|
and \verb|\thanks|.


\iflocalin
For more details see the German documentation
\file{\docdir thesis} and \file{\docdir paper}.
\fi

