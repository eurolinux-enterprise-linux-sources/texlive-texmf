\documentclass{article}
\usepackage{sparklines}
\renewcommand{\baselinestretch}{1.25}

\begin{document}
\title{Sparklines}
\author{Andreas L\"{o}ffler}
\date{21.~April 2005}\maketitle

\section{What are sparklines?}

 Sparklines are intense, simple, wordlike
graphics (so named by Edward Tufte). In lieu of a more detailed introduction,
Professor Tufte's site has an early release of a chapter on sparklines, see
www.edwardtufte.com. A PHP implementation can be found at
http://sparkline.sourceforge.net/.


A sparkline can be added using the \texttt{sparkline}--environment. Also, you
can add sparkling rectangles for the median and special sparkling dots in red
or blue. \textbf{Only use sparklines with pdflatex}, sparklines do not appear
within a dvi-file. \texttt{Sparklines} does not work with \texttt{pictex}.

If we want to add a sparkline, be careful not to leave an empty line between
the text left from the sparkline and the environment itself (starting with the
sparkline environment)
% This now starts the sparkline of length 10 times the width of the letter x
\begin{sparkline}{10}
% This environment has now an own coordinate system. x and y run from 0 to 1.

% This draws a sparkling-rectangle with lower level y=0.3 and upper level y=0.5
\sparkrectangle 0.3 0.5

% This draws a blue colored sparkling-dot at x=0.5, y=0.8
\sparkdot 0.5 0.62 blue

% This draws a red colored sparkling-dot at x=1, y=0.2
\sparkdot 1 0.2 red

% draws a sparkline with ten values starting at x=0.1 and y=0.95, then x=0.2 and y=0.8 etc.
% drawing finishes if / appears
 \spark 0.1 0.95 0.2 0.8 0.3 0.3 0.4 0.52 0.5 0.62 0.6 0.7 0.7 0.5 0.8 0.4 0.9 0.25 1 0.2 /
\end{sparkline}
 Daimler Chrysler, 12\%, {\color{red} 129}, since otherwise the sparkline starts
 at a new line.

\section*{Version history}
\begin{description}
\item[] Apr 21, 2005

 version 1.1: bug removed thanks to Mathias Hofmann <mathias.hofmann@web.de>

\item[] Dec 12, 2004

 version 1.0: first version of sparklines
\end{description}

\end{document}
