%%
%% This is file `guittest.tex',
%% generated with the docstrip utility.
%%
%% The original source files were:
%%
%% guit.dtx  (with options: `test')
%% 
%%  Copyright 2003, 2004, 2005, 2006 Gruppo Utilizzatori Italiani di TeX
%% 
%%  This work may be distributed and/or modified under the
%%  conditions of the LaTeX Project Public License, either
%%  version 1.3a of this license or (at your option) any
%%  later version.
%%  The latest version of the license is in
%%     http://www.latex-project.org/lppl.txt
%% 
%%  Author: Emanuele Vicentini
%%          (emanuelevicentini at yahoo dot it)
%% 
%%  This work has the LPPL maintenance status "author-maintained".
%% 
%%  This work consists of the files: README, guit.dtx, guit.ins and the
%%  derived files guit.sty, guit.cfg and guittest.tex
%% 
\documentclass[12pt, a4paper]{article}
\usepackage[margin=1in, noheadfoot]{geometry}
\usepackage{booktabs, guit, rotating}

\pagestyle{empty}
\setupguit[link]

\begin{document}
\section*{Piccolo test per \guit}
Vediamo un po' come si comporta in una footnote\footnote{\guit~\guit\ldots
direi che \`e passabile, no?}. Diamoci\footnote{Un'altra nota:
\fontfamily{pzc}\fontseries{mb}\fontshape{it}\selectfont\guittext} dentro
con qualche cosa\footnote{Un ultima nota: Copyright 2003, 2004, 2005, 2006
\guittext} di ``strambo'':

\begin{center}
\itshape\guit~agus \TeX~go br\'ach!\\
$==$\\
\rmfamily\bfseries\guit~and \TeX~forever!
\end{center}

Una piccola ``spirale'' colorata e divertente:

\begingroup
\setupGuIT[color=yes]
\GuITcolor[rgb]{1, 0, 0}
\newcount\wang
\newsavebox{\wangtext}
\newdimen\wangspace
\def\wheel#1{\savebox{\wangtext}{#1}%
  \wangspace\wd\wangtext
  \advance\wangspace by 1cm%
  \centerline{%
    \rule{0pt}{\wangspace}%
    \rule[-\wangspace]{0pt}{\wangspace}%
    \wang=-180
    \loop
      \ifnum\wang<180
        \rlap{\begin{rotate}{\the\wang}%
          \rule{0.75cm}{0pt}#1
          \end{rotate}}%
        \advance\wang by 20
        \space
        \guitcolor*{coloredelGuIT!90!green}%
      \repeat}}
\wheel{\guit}
\endgroup

Ed ora una tabella riepilogativa parziale dei font supportati dalla versione
attuale.

\begin{center}
\begin{tabular}{lc}
\toprule
Font & Logo \\
\midrule
Computer Modern Roman & \guit[family=cmr] \\
Times New Roman & \guit[family=ptm] \\
Palatino & \guit[family=ppl] \\
NewCentury Schoolbook & \guit[family=pnc] \\
Charter & \guit[family=bch] \\
Bookman & \guit[family=pbk] \\
\bottomrule
\end{tabular}
\end{center}

Per finire, un sfilza di \guit:

\begingroup
  \let\pippo\par
  \makeatletter
  \@tfor\famiglia:={cmr}{ptm}{ppl}{pnc}{bch}{pbk}\do{%
    \noindent
    \setupGuIT[family=\famiglia]
    \@tfor\dimensione:=\tiny\scriptsize\footnotesize\small\normalsize
      \large\Large\LARGE\huge\do{%
      \dimensione\guit~}%
    \Huge\guit\pippo}%
\endgroup

\begin{center}
\Huge Venite tutti al prossimo \guitmeeting[color]
\end{center}

\end{document}
