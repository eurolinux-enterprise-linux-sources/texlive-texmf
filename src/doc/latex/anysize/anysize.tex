% Documentation for the anysize-Styleoption
\NeedsTeXFormat{LaTeX2e}

\documentclass[a4paper]{article}
\usepackage{anysize}
\usepackage{ae}

% Options for anysize
\marginsize{3cm}{3cm}{3cm}{3cm}

\newcommand{\anysize}{{\large a}{\small\bf n}{\sl y}{\Large\tt s}{\it i}{\sf\footnotesize z}e}

\begin{document}

\title{{\tt anysize.sty} ---  
to set the margin sizes in  \LaTeX}

\author{
first version: Michael Salzenberg,\\
conversion to \LaTeX2e{}: Thomas Esser,\\
documentation: Dirk Hillbrecht\thanks{eMail--addresses:
{\tt te@informatik.uni-hannover.de}, {\tt dh@chitec.de}}
}
\date{v1.0, Aug 13, 1994}
\maketitle
\thispagestyle{empty}

\section{Overview}

\anysize{} is a \LaTeX{} package set the papersize and the margins of
a document. These settings are done by one simple command.

\anysize{} is \LaTeX2e--compilant and does not run with the
obsolete \LaTeX{} 2.09 format.

Many thanks go to Michael Salzenberg, who made the first implementation of
this style. He does not want to be bothered with the wide-area-publication
of the style, so Thomas --- who converted it to \LaTeX2e{} --- and me,
who documented it properly (I hope {\tt :-)}), maintain it now. Therefore,
are only our email adresses are given.

\section{Usage}

\anysize{} is usually loaded by the \verb|\usepackage|--directive. The
style knows options for defining the paper size ({\tt a4paper}, {\tt a5paper}
etc.) and one additional option {\tt germanpar}, that changes the margins of
the paragraphes in the document particulary.

After inclusion of \anysize{}, there are two new commands available for the document preamble:

\begin{itemize}
\item \verb|\papersize{|{\it width}\verb|}{|{\it height}\verb|}|: sets the paper size. This command
  should not be used anymore, as there are standardized style options for the standard paper sizes
  to be passed through the \verb|\documentclass|--command in the header of the document. 
\item \verb|\marginsize{|{\it left}\verb|}{|{\it right}\verb|}{|{\it 
top}\verb|}{|{\it bottom}\verb|}|:
        sets the margin sizes. This is the really interesting thing of the whole one. In two--sided
        documents, left and right margin are those ones for the rectos. For the backs, they are swapped.
\end{itemize}

These are the default settings, as long as none of the above commands is given:
$$\mbox{\begin{tabular}{lr}
\multicolumn{2}{l}{paper size\footnotemark{}:} \\
\hline
width          & 21,0 cm       \\
height           & 29,7 cm       \\
\multicolumn{2}{l}{(DIN A4)}    \\
\\
\end{tabular}} 
\hspace{5em}
\mbox{\begin{tabular}{lr}
\multicolumn{2}{l}{margin sizes:} \\ 
\hline
left           & 3,0 cm \\
right          & 2,0 cm \\
top            & 2,0 cm \\
bottom           & 2,0 cm \\
\end{tabular}}$$

\footnotetext{only if there is no style option for the paper size}

\end{document}
%% Local Variables: ***
%% mode: LaTeX
%% TeX-command-default: LaTeX2e
%% End: ***
