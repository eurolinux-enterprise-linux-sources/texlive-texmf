%--------------------------------------------------------------------
%--------------------------------------------------------------------
% examdoc.tex
%
% This is the documentation for the exam documentclass, 
% by Philip Hirschhorn.
%
% The exam documentclass itself is in the file exam.cls.


%%% Copyright (c) 1997, 2000, 2004 Philip S. Hirschhorn
%
% This program may be distributed and/or modified under the
% conditions of the LaTeX Project Public License, either version 1.2
% of this license or (at your option) any later version.
% The latest version of this license is in
%   http://www.latex-project.org/lppl.txt
% and version 1.2 or later is part of all distributions of LaTeX 
% version 1999/12/01 or later.
%
% This program consists of the files exam.cls and examdoc.tex

% The user documentation for exam.cls is in the file examdoc.tex.


%%% Philip Hirschhorn
%%% Department of Mathematics
%%% Wellesley College
%%% Wellesley, MA 02481
%%% psh@math.mit.edu

% The newest version of this documentclass should always be available
% from my web page: http://www-math.mit.edu/~psh/


%--------------------------------------------------------------------
%--------------------------------------------------------------------

\documentclass[12pt]{exam}
%\usepackage{hyperref}

\newcommand{\docversion}{2.2}
\newcommand{\docdate}{August 14, 2004}
%\newcommand{\docdate}{\today}

%--------------------------------------------------------------------
%
% Changes from version 2.01 onward are described in the comments
% near the beginning of the file exam.cls.
%
%--------------------------------------------------------------------
%
% The only change from version 2.0 to version 2.01 is that this
% documentclass (and its accompanying documentation) is now 
% explicitly distributed under the LaTeX Project Public License.
%
%--------------------------------------------------------------------
%---------------------------------------------------------------------
\newenvironment{example}%
   {\bigskip\filbreak
    \subsubsection{Example:}
   }%
   {}


\def\samplehead#1#2#3#4{%
  \begin{trivlist}
     \item[]
     \leavevmode
     \hbox to \textwidth{%
         \rlap{\parbox[b]{\textwidth}{\raggedright#1\strut}}%
         \hfil\parbox[b]{\textwidth}{\centering#2\strut}\hfil
         \llap{\parbox[b]{\textwidth}{\raggedleft#3\strut}}%
  }% hbox
  #4
  \end{trivlist}
}

\def\samplefoot#1#2#3#4{%
  \begin{trivlist}
     \item[]
     \leavevmode
     #1
     \vskip 3pt

     \hbox to \textwidth{%
         \rlap{\parbox[t]{\textwidth}{\raggedright#2}}%
         \hfil\parbox[t]{\textwidth}{\centering#3}\hfil
         \llap{\parbox[t]{\textwidth}{\raggedleft#4}}%
  }% hbox
  \end{trivlist}
}



\makeatletter
\@ifundefined{AmS}{\def\AmS{{\protect\the\textfont\tw@
  A\kern-.1667em\lower.5ex\hbox{M}\kern-.125emS}}}
  {}
\makeatother


%---------------------------------------------------------------------
%---------------------------------------------------------------------
%---------------------------------------------------------------------
%---------------------------------------------------------------------

\begin{document}

\title{Using the exam document class}

\author{Philip Hirschhorn\\
Department of Mathematics\\
Wellesley College\\
Wellesley, MA 02481\\
psh@math.mit.edu\\[\bigskipamount]
Copyright \copyright~1994, 1997, 2000, 2004 Philip Hirschhorn\\
   All rights reserved}

\date{\docdate}

\maketitle

\begin{center}
  \small
  This is the documentation for version~\docversion{} of the
  \verb"exam" document class. 
\end{center}

\tableofcontents

\section{Introduction}

The file \verb"exam.cls" provides the \verb"exam" document class,
which attempts to make it easy for even a \LaTeX{} novice to prepare
exams.  Most of what's done by the \verb"exam" document class can also
be accomplished by using \verb"fancyheadings.sty", adjusting \LaTeX's
page layout parameters (perhaps by using \verb"fullpage.sty"), and
making careful use of the \verb"list" environment, but the \verb"exam"
document class tries to make all this as simple as possible.

Specifically, \verb"exam.cls" sets the page layout so that there are
one inch margins all around (no matter what size paper you're using)
and provides commands that make it easy to format questions, create
very flexible headers and footers, and change the margins.  In
particular:
\begin{itemize}
\item The class will automatically format and number the questions,
  parts of questions, subparts of questions, and subsubparts of
  questions, while making it easy to refer to specific questions by
  number in any special directions you need to print on the exam.
\item You can include the point value of each question (or part, or
  subpart, or subsubpart), with your choice of having the point values
  printed at the beginning of the text of the question, opposite that
  in the left margin, opposite that in the right margin, or in the
  right margin opposite the end of the question.
\item The class will add up the total points for each question (and
  all of its parts, subparts, and subsubparts) and the total points on
  each page, and make those totals available in macros.  You can also
  have the class print a grading table indexed either by question
  number or by page number.
\item You can specify the header in three parts: The left head, the
  center head, and the right head.  The left head is left justified,
  the center head is centered, and the right head is right justified,
  and one or all of these can be omitted.
\item The footer is also specified in three parts: Left justified,
  centered, and right justified.
\item The header and footer for the first page can be different from
  the ones used on other pages.
\item Both headers and footers can contain more than one line.  To
  accommodate headers and footers with many lines, simple commands are
  provided to increase the parts of the page devoted to the header
  and/or footer, and these commands can give one amount of space on
  the first page and a different amount of space on all other pages.
\item Macros are defined to enable you to state in the header and/or
  footer the total number of pages in the exam, and to change the
  header and/or footer that appears on the \emph{last} page of the
  exam.
\item Macros are defined so that the headers and footers can vary
  depending on whether the current page begins a new question or
  continues a question that started on an earlier page (and, if one
  continues onto the current page, to say what the number of that
  question is).  Macros are also defined so that the headers and
  footers can vary depending on whether a question is complete on the
  current page or continues on to the next page (and, if one
  continues, to say what the number of that question is).
\item You can have a horizontal rule at the base of the header and/or
  at the top of the footer.
\end{itemize}


%--------------------------------------------------------------------
\section{License}

\begin{itemize}
\item This work may be distributed and/or modified under the conditions of
  the \LaTeX{} Project Public License, either version~1.3 of this
  license or (at your option) any later version.  The latest version
  of this license is in \verb"http://www.latex-project.org/lppl.txt"
  and version 1.3 or later is part of all distributions of \LaTeX{}
  version 2003/12/01 or later.
\item This work has the LPPL maintenance status
  ``author-maintained''.
\item This work consists of the files exam.cls and examdoc.tex.
\end{itemize}


%---------------------------------------------------------------------
\section{The \texttt{documentclass} command}

To use the \verb"exam" document class, you should specify \verb"exam"
as your main document class.  For example, if you want to use
12~point type, then your \verb"documentclass" command should be
%
\begin{center}
\verb"\documentclass[12pt]{exam}"
\end{center}
%
If you would also like to use some of the features of \AmS-\LaTeX,
then you should use the commands
%
\begin{center}
  \begin{tabular}{l}
    \verb"\documentclass[12pt]{exam}"\\
    \verb"\usepackage{amsmath}"
  \end{tabular}
\end{center}
%
If you will be typing solutions into the exam (see
section~\ref{sec:solutions}) and you'd like the solutions to be
printed on the next run of \LaTeX, you should include the document
class option \verb"answers", as in
\begin{center}
  \verb"\documentclass[answers]{exam}"
\end{center}
or
\begin{center}
  \verb"\documentclass[answers,12pt]{exam}"
\end{center}
This is optional, though; you can accomplish the same thing by giving
the command \verb"\printanswers" (see section~\ref{sec:solutions}).



%--------------------------------------------------------------------
\section{Asking for the student's name}

This isn't anything specific to the \verb"exam" document class, but
it's worth mentioning because it isn't obvious.  If you're leaving
space for the answers on the question pages (see
section~\ref{sec:LeaveSpace}), then you'll probably also want to leave
space for the student's name.  If you type
%
\begin{verbatim}
\begin{center}
  \fbox{\fbox{\parbox{5.5in}{\centering
        Answer the questions in the spaces provided on the
        question sheets.  If you run out of room for an answer,
        continue on the back of the page.}}}
\end{center}

\vspace{0.1in}

\hbox to \textwidth{Name and section:\enspace\hrulefill}

\vspace{0.2in}

\hbox to \textwidth{Instructor's name:\enspace\hrulefill}
\end{verbatim}
%
after the \verb"\begin{document}" command and before the
\verb"\begin{questions}" command (see section~\ref{sec:Questions}),
then you'll get\\
\noindent
\parbox{\textwidth}{
\begin{center}
  \fbox{\fbox{\parbox{5.5in}{\centering
        Answer the questions in the spaces provided on the
        question sheets.  If you run out of room for an answer,
        continue on the back of the page.}}}
\end{center}

\vspace{0.1in}

\hbox to \textwidth{Name and section:\enspace\hrulefill}

\vspace{0.2in}

\hbox to \textwidth{Instructor's name:\enspace\hrulefill}
}


%---------------------------------------------------------------------
\section{Questions}
\label{sec:Questions}

To type the questions in the exam, you use the \verb"questions"
environment.  Each question is then begun with the command
\verb"\question", and the questions will be numbered automatically.

For example, if you type
%
\begin{verbatim}
\begin{questions}

\question
Why is there air?

\question

How much wood would a woodchuck chuck if a woodchuck could chuck
wood?

\question
Compute $\displaystyle\int_0^1 x^2 \, dx$.
\end{questions}
\end{verbatim}
%
then you'll get
%
\begin{questions}

\question
Why is there air?

\question

How much wood would a woodchuck chuck if a woodchuck could chuck
wood?

\question
Compute $\displaystyle\int_0^1 x^2 \, dx$.
\end{questions}

\bigskip

As the above example illustrates, you can leave blank lines between
the \verb"\question" command and the actual beginning of the
question, or before the first \verb"\question" command in the
environment, and they will be ignored.


%---------------------------------------------------------------------
\subsection{Questions with parts and subparts (and subsubparts)}

If you want a question to have several parts, then you use the
\verb"parts" environment.  For example, if you type
%
\begin{verbatim}
\begin{questions}
\question
Why is there air?

\question
What if there were no air?

\begin{parts}
\part
Describe the effect on the balloon industry.

\part
Describe the effect on the aircraft industry.
\end{parts}

\question

\begin{parts}
\part
Define the universe.  Give three examples.

\part
If the universe were to end, how would you know?

\end{parts}


\end{questions}
\end{verbatim}
%
then you'll get
\begin{questions}
\question
Why is there air?

\question
What if there were no air?

\begin{parts}
\part
Describe the effect on the balloon industry.

\part
Describe the effect on the aircraft industry.
\end{parts}

\question

\begin{parts}
\part
Define the universe.  Give three examples.

\part
If the universe were to end, how would you know?

\end{parts}


\end{questions}

\bigskip

The above example illustrates several things:
%
\begin{itemize}
\item
Parts of a question should be put into a \verb"parts" environment.

\item
If a question begins with a \verb"parts" environment, then the first
part will appear on the same line with the question number.

\item
You can leave blank lines before and after the \verb"\part" command,
and they will be ignored.
\end{itemize}


\bigskip

There is also a \verb"subparts" environment, and even a
\verb"subsubparts" environment, and they work just as you would
expect.  For example, if you type
%
\begin{verbatim}
\begin{questions}
\question
\begin{parts}
\part
What do you do with a drunken sailor?

\part
Is your answer different if it is before noon?
  
\end{parts}

\question
This is the second question.

\begin{parts}
\part
This is a part.

\part
This is also a part.

\begin{subparts}
\subpart
This is a subpart.

\subpart
This is a periscope.

\subpart
This is a pair of diving planes.

\subpart
\begin{subsubparts}
\subsubpart
This is a subsubpart.

\subsubpart
The lower surface of a diving plane?

\subsubpart
The ocean floor, perhaps?

\end{subsubparts}
\end{subparts}

\part
It's sad to be apart.

\end{parts}

\question
\begin{parts}
\part
\begin{subparts}
\subpart
This is a subpart.

\subpart
This is another subpart.
\end{subparts}
\part
This is another part.

\end{parts}


\end{questions}
\end{verbatim}
%
then you'll get
\begin{questions}
\question
\begin{parts}
\part
What do you do with a drunken sailor?

\part
Is your answer different if it is before noon?
  
\end{parts}

\question
This is the second question.

\begin{parts}
\part
This is a part.

\part
This is also a part.

\begin{subparts}
\subpart
This is a subpart.

\subpart
This is a periscope.

\subpart
This is a pair of diving planes.

\subpart
\begin{subsubparts}
\subsubpart
This is a subsubpart.

\subsubpart
The lower surface of a diving plane?

\subsubpart
The ocean floor, perhaps?

\end{subsubparts}
\end{subparts}

\part
It's sad to be apart.

\end{parts}

\question
\begin{parts}
\part
\begin{subparts}
\subpart
This is a subpart.

\subpart
This is another subpart.
\end{subparts}
\part
This is another part.

\end{parts}


\end{questions}


%---------------------------------------------------------------------
\subsection{Point values for the questions}
\label{sec:points}

Each of the commands \verb"\question", \verb"\part", \verb"\subpart"
and \verb"\subsubpart" take an optional argument, which is the number
of points for that question, part, subpart, or subsubpart.  (By
default, the point values are enclosed in parentheses, but the
parentheses can be replaced with either square brackets or a box; for
this, see section~\ref{sec:Boxed}.)

The default is that the point value will be inserted at the beginning
of the question (or part, or subpart, or subsubpart) in parentheses,
but
\begin{itemize}
\item the command \verb"\pointsinmargin" will cause the point values
  to be set in the left margin,
\item the command \verb"\pointsinrightmargin" will cause the point
  values to be set in the right margin, and
\item the commands \verb"\nopointsinmargin" and
  \verb"\nopointsinrightmargin" are equivalent, and either of them
  will  revert to the default situation.
\end{itemize}
All of these print the point values on the first line of the question
(or part, or subpart, or subsubpart).  There is also a way to print
the point values on the \emph{last} line of the question (or part,
etc.); for this, see section~\ref{sec:DrpPts}.

\medbreak

For example, if you type
%
\begin{verbatim}
\begin{questions}
\question[20]
Why is there air?


\question
What if there were no air?

\begin{parts}
\part[10]
Describe the effect on the balloon industry.

\part[10]
Describe the effect on the aircraft industry.
\end{parts}

\end{questions}
\end{verbatim}
%
then, with the default setup, you'll get
\nopointsinmargin
\begin{questions}
\question[20]
Why is there air?


\question
What if there were no air?

\begin{parts}
\part[10]
Describe the effect on the balloon industry.

\part[10]
Describe the effect on the aircraft industry.
\end{parts}

\end{questions}
(To change the word ``points'', see the commands described in
section~\ref{sec:pointname}.)

\bigskip
If you give the command \verb"\pointsinmargin", then the above input
will produce instead
\pointsinmargin
\begin{questions}
\question[20]
Why is there air?


\question
What if there were no air?

\begin{parts}
\part[10]
Describe the effect on the balloon industry.

\part[10]
Describe the effect on the aircraft industry.
\end{parts}

\end{questions}
(To have a word (e.g., ``points'') inside the parentheses with the
point values, see the \verb"\marginpointname" command in
section~\ref{sec:pointname}.)


\bigskip 

If you give the command \verb"\pointsinrightmargin", then the above
input will produce instead
\pointsinrightmargin
\begin{questions}
\question[20]
Why is there air?


\question
What if there were no air?

\begin{parts}
\part[10]
Describe the effect on the balloon industry.

\part[10]
Describe the effect on the aircraft industry.
\end{parts}

\end{questions}

%--------------------------------------------------------------------
\subsubsection{Half points}
\label{sec:Half}

The point values of questions, parts, subparts, and subsubparts can
include half points.  To specify half points, you either type 
\begin{center}
  \verb"\half"
\end{center}
immediately following the integer part of the point value or just type
\verb"\half" as the entire point value. That is, the valid point
values and their (default) printed appearances are:
\begin{center}
  \begin{tabular}{@{\extracolsep{0.1in}}rccccccc}
    Typed:& \verb"0"& \verb"\half"& \verb"1"& \verb"1\half"& \verb"2"&
      \verb"2\half"& $\cdots$\\[1ex]
    Printed:& 0& \half& 1& 1\half& 2& 2\half& $\cdots$
  \end{tabular}
\end{center}
If you prefer, you can cause the fraction \half{} to be printed as
\makeatletter
\horiz@half.
\makeatother
To do this, you give the command
\begin{center}
  \verb"\usehorizontalhalf"
\end{center}
after which the printed appearance of half point values will be
\usehorizontalhalf
\begin{center}
  \begin{tabular}{@{\extracolsep{0.1in}}rccccccc}
    Typed:& \verb"0"& \verb"\half"& \verb"1"& \verb"1\half"& \verb"2"&
      \verb"2\half"& $\cdots$\\[1ex]
    Printed:& 0& \half& 1& 1\half& 2& 2\half& $\cdots$
  \end{tabular}
\end{center}
If you want to return to using the default appearance, you can do that
by giving the command
\begin{center}
  \verb"\useslantedhalf"
\end{center}
\useslantedhalf



%--------------------------------------------------------------------
\subsubsection{Surrounding the points: Parentheses, brackets, or a box}
\label{sec:Boxed}

If you prefer having the points enclosed in brackets, instead of in
parentheses, give the command
\begin{center}
  \verb"\bracketedpoints"
\end{center}
For example, if you give the command \verb"\bracketedpoints", then the
questions typed above will produce
%
\nopointsinmargin
\bracketedpoints
\begin{questions}
\question[20]
Why is there air?


\question
What if there were no air?

\begin{parts}
\part[10]
Describe the effect on the balloon industry.

\part[10]
Describe the effect on the aircraft industry.
\end{parts}

\end{questions}
%

If you prefer having the points enclosed in a box, instead of in
parentheses, give the command
\begin{center}
  \verb"\boxedpoints"
\end{center}
For example, if you give the command \verb"\boxedpoints", then the
questions typed above will produce
%
\nopointsinmargin
\boxedpoints
\begin{questions}
\question[20]
Why is there air?


\question
What if there were no air?

\begin{parts}
\part[10]
Describe the effect on the balloon industry.

\part[10]
Describe the effect on the aircraft industry.
\end{parts}

\end{questions}
%
If you give the commands \verb"\boxedpoints" and
\verb"\pointsinmargin", then the above questions will produce
\boxedpoints
\pointsinmargin
\begin{questions}
\question[20]
Why is there air?


\question
What if there were no air?

\begin{parts}
\part[10]
Describe the effect on the balloon industry.

\part[10]
Describe the effect on the aircraft industry.
\end{parts}

\end{questions}
%
If you give the commands \verb"\bracketedpoints" and
\verb"\pointsinrightmargin", then the above questions will produce
%
\bracketedpoints
\pointsinrightmargin
\begin{questions}
\question[20]
Why is there air?


\question
What if there were no air?

\begin{parts}
\part[10]
Describe the effect on the balloon industry.

\part[10]
Describe the effect on the aircraft industry.
\end{parts}

\end{questions}
Other combinations of these commands will produce similar effects.

If you want to switch back and forth between formats during the exam,
you can do so by giving one of the commands
\begin{center}
  \begin{tabular}{l}
    \verb"\boxedpoints"\\
    \verb"\bracketedpoints"\\
    \verb"\noboxedpoints"\\
    \verb"\nobracketedpoints"
  \end{tabular}
\end{center}
whenever you want to switch.  (The commands
\verb"\nobracketedpoints" and \verb"\noboxedpoints" are equivalent;
they both return to the default of putting parentheses around the
points.)

If you want a word automatically inserted following the point values
when using either \verb"\pointsinmargin" or
\verb"\pointsinrightmargin", see the
  \verb"\marginpointname" command
in section~\ref{sec:pointname}.



%---------------------------------------------------------------------
\subsubsection{Using a substitute for the word ``points''}
\label{sec:pointname}

With the default setup, the number of points is inserted before the
text of the question followed by the word ``point'' if the number of
points is 1 and by the word ``points'' otherwise.  That is, the
default is \verb"\nopointsinmargin" and \verb"\noboxedpoints" (see
sections \ref{sec:points} and \ref{sec:Boxed}), and if you type
\nopointsinmargin \noboxedpoints
\begin{verbatim}
\begin{questions}
\question[1]
This is a short question.

\question[5]
This is a much longer question, requiring much more thought.
\end{questions}
\end{verbatim}
then you'll get
\begin{questions}
\question[1]
This is a short question.

\question[5]
This is a much longer question, requiring much more thought.
\end{questions}
The way this is achieved is:
\begin{itemize}
\item There is a command 
  \begin{center}
    \verb"\points"
  \end{center}
  whose default definition expands to the word `point' if the number
  of points is 1 and to `points' otherwise.  (This behavior can be
  modified by the command \verb"\pointpoints"; see below.)
\item There is a command 
  \begin{center}
    \verb"\pointname{Text}"
  \end{center}
  that causes `Text' to be inserted following the number of points of
  a question, part, subpart, or subsubpart.  The default setup is the
  result of the command
  \begin{center}
    \verb"\pointname{ \points}"
  \end{center}
  (Note the single space before the command \verb"\points".)
\end{itemize}



You can modify this setup by using the command \verb"\pointname", but
there is also a command
\begin{center}
  \verb"\pointpoints{SingularText}{PluralText}"
\end{center}
that changes the behavior of the \verb"\points" command so that it
expands to `SingularText' if the number of points is 1 and to
`PluralText' otherwise.  (The default is the result of the command
\verb"\pointpoints{point}{points}".)  Thus, if you give the
command
%
\begin{center}
  \verb"\pointpoints{Punkt}{Punkte}"
\end{center}
%
then typing the questions above would result in
%
\pointpoints{Punkt}{Punkte}
\begin{questions}
\question[1]
This is a short question.

\question[5]
This is a much longer question, requiring much more thought.
\end{questions}
%
and if you give the command
%
\begin{center}
  \verb"\pointpoints{mark}{marks}"
\end{center}
%
then typing the questions above would result in
%
\pointpoints{mark}{marks}
\begin{questions}
\question[1]
This is a short question.

\question[5]
%
This is a much longer question, requiring much more thought.
\end{questions}
There is a shortcut for that; the command
\begin{center}
  \verb"\marksnotpoints"
\end{center}
expands to \verb"\pointpoints{mark}{marks}".


\bigskip

For example, if you give the command
%
\begin{center}
\verb"\pointname{\%}"
\end{center}
%
and then type
%
\begin{verbatim}
\question[25]
Where, oh where, has my little dog gone?
\end{verbatim}
%
then you'll get
\pointname{\%}
\nopointsinmargin
\noboxedpoints
\begin{questions}
\question[25]
Where, oh where, has my little dog gone?
\end{questions}
\pointname{ \points}
%

Similarly, the command \verb"\marginpointname" can be used to affect
the text set with the number of points when \verb"\pointsinmargin" or
\verb"\pointsinrightmargin" is in effect.  For example, if you give
the commands
\begin{center}
  \begin{tabular}{l}
    \verb"\pointsinmargin"\\
    \verb"\marginpointname{\%}"
  \end{tabular}
\end{center}
and then type
\begin{verbatim}
\question[25]
Where, oh where, has my little dog gone?
\end{verbatim}
then you'll get
\pointsinmargin
\noboxedpoints
\marginpointname{\%}
\begin{questions}
\question[25]
Where, oh where, has my little dog gone?
\end{questions}
unless, of course, you've also given the command \verb"\boxedpoints"
(see section~\ref{sec:Boxed}), in which case you'll get
%
\pointsinmargin
\boxedpoints
\marginpointname{\%}
\begin{questions}
\question[25]
Where, oh where, has my little dog gone?
\end{questions}
If you give the commands
\begin{center}
  \begin{tabular}{l}
    \verb"\marksnotpoints"\\
    \verb"\marginpointname{ \points}"
  \end{tabular}
\end{center}
then the above will print
\noboxedpoints
\marginpointname{ \points}
\marksnotpoints
\begin{questions}
\question[25]
Where, oh where, has my little dog gone?
\end{questions}
\nopointsinmargin
\marginpointname{}
\pointpoints{point}{points}





Using the default is equivalent to giving the command
\begin{center}
  \verb"\marginpointname{}"
\end{center}

%--------------------------------------------------------------------
\subsubsection{Using \texttt{marginpointname} and enlarging the
  margins}

The default margins are not very large, and so if you use
\verb"\marginpointname" (see section~\ref{sec:pointname}) to place any
words along with the point values in the margin, you may run out of
room.  The solution to this problem is to enlarge the margins by using
the \verb"\extrawidth" command with a negative argument to
\emph{decrease} the width of the text, and thus \emph{increase} the
width of the margins.  For example, the commands
\begin{center}
  \begin{tabular}{ll}
    \verb"\extrawidth{-1in}"\\
    \verb"\marginpointname{ \points}"
  \end{tabular}
\end{center}
will increase each of the left and right margins by one half inch,
which will easily leave room for the word ``points''.  (See
section~\ref{sec:pagesize} for a discussion of the \verb"\extrawidth"
command.)
%---------------------------------------------------------------------
\subsubsection{Questions that begin with a parts environment}

One thing to keep in mind is that \emph{only one point value can
  appear on a line, and it will be the last one to be placed there}.
This matters only if a question begins with a parts environment, or if
a part begins with a subparts environment, or if a subpart begins with
a subsubparts environment.  In any of these cases, the question number
and part number (or the part number and subpart number, etc.) will
appear on the same line, and if both of these commands include an
optional point value, \emph{only the last one given will be used}.
For example, if you type
%
\begin{verbatim}
\begin{questions}
\question[10]
\begin{parts}
\part[5]
Who put the ``bop'' in the ``bop, sh-bop sh-bop''?

\part[5]
Who put the ``ram'' in the ``rama, rama ding-dong''?
\end{parts}
\end{questions}
\end{verbatim}
%
then you'll get
%
\noboxedpoints
\marginpointname{}
\begin{questions}
\question[10]
\begin{parts}
\part[5]
Who put the ``bop'' in the ``bop, sh-bop sh-bop''?

\part[5]
Who put the ``ram'' in the ``rama, rama ding-dong''?
\end{parts}
\end{questions}
%
Notice that the 10~points for the entire question are never mentioned
anywhere, since they would have appeared on the same line with the
5~points for the first part, and the 5~points for the part were placed
later.  This is true whether \verb"\pointsinmargin",
\verb"\pointsinrightmargin", or \verb"\nopointsinmargin" is in effect.
In addition, if you've given the command \verb"\addpoints" (see
section~\ref{sec:Count}), then this question will have 20~points
assigned to it (10 for the question itself plus 5 for each of the two
parts) even though the 10 for the question itself will not be printed.


%--------------------------------------------------------------------
\subsubsection{Printing the points at the end of the question}
\label{sec:DrpPts}

Instead of having the point value of a question (or part, or subpart,
or subsubpart) printed on the first line of a question, you can have
it printed in the right margin opposite the \emph{last} line, or even
opposite a blank line following the paragraph.  This is done with the
\begin{center}
  \verb"\pointsdroppedatright"
  \quad and \quad
  \verb"\droppoints"
\end{center}
commands.

If you give the command
\begin{center}
  \verb"\pointsdroppedatright"
\end{center}
then point values are not printed until you give the command
\verb"\droppoints" (and they're not printed at all if you don't give
the command \verb"\droppoints").  The only exception to this rule is
that if you've given a \verb"\qformat" command (see
section~\ref{sec:qformat}), then question number lines are printed as
specified by the argument to the \verb"\qformat" command even if you
have given the command \verb"\pointsdroppedatright".

The command \verb"\droppoints" should be given only at the end of a
paragraph or between paragraphs; if you give it within a paragraph, it
causes the paragraph to end.  \verb"\droppoints" prints the point
value in the right margin, formatted as it is when you give the
command \verb"\pointsinrightmargin", except that the points appear
opposite the last line of the paragraph (or, if the command
\verb"\droppoints" is given between paragraphs, then additional
vertical space is left between the paragraphs and the points are
printed opposite the blank space).  Thus, the formatting can be
changed by giving the commands \verb"\bracketedpoints",
\verb"\boxedpoints", or \verb"\marginpointname" (see
section~\ref{sec:pointname}).  The command \verb"\droppoints" actually
works this way even if one of the commands
\begin{center}
  \verb"\nopointsinmargin", \verb"\pointsinmargin", or
  \verb"\pointsinrightmargin"
\end{center}
is in effect, but if you use it that way the points will appear twice
on the page, which is probably not what you want.


For example, if you give the command \verb"\pointsdroppedatright" and
then type
\begin{verbatim}
\begin{questions}
\question[10]
Describe the implications of the Michelson-Morley ink drop experiment
for the use of punctuation marks such as colons and semicolons that
require multiple drops of ink.
\droppoints

\question[10]
Prove that the second dual of a finite dimensional real vector space
is naturally isomorphic to the given vector space, except when the
weapon chosen is a single shot pistol.\droppoints

\end{questions}
\end{verbatim}
then you'll get
\pointsdroppedatright
\begin{questions}
\question[10]
Describe the implications of the Michelson-Morley ink drop experiment
for the use of punctuation marks such as colons and semicolons that
require multiple drops of ink.
\droppoints

\question[10]
Prove that the second dual of a finite dimensional real vector space
is naturally isomorphic to the given vector space, except when the
weapon chosen is a single shot pistol.\droppoints

\end{questions}

If you type
\begin{verbatim}
\begin{questions}
\question[10]
Describe the implications of the Michelson-Morley ink drop experiment
for the use of punctuation marks such as colons and semicolons that
require multiple drops of ink.

\droppoints

\question[10]
Prove that the second dual of a finite dimensional real vector space
is naturally isomorphic to the given vector space, except when the
weapon chosen is a single shot pistol.

\droppoints
\end{questions}
\end{verbatim}
then you'll get
\begin{questions}
\question[10]
Describe the implications of the Michelson-Morley ink drop experiment
for the use of punctuation marks such as colons and semicolons that
require multiple drops of ink.

\droppoints

\question[10]
Prove that the second dual of a finite dimensional real vector space
is naturally isomorphic to the given vector space, except when the
weapon chosen is a single shot pistol.

\droppoints
\end{questions}
\nopointsinmargin

%--------------------------------------------------------------------
\subsubsection{Adding up the points for each question}


If you give the command \verb"\addpoints" (see
section~\ref{sec:Count}), then you can use the command
\begin{center}
  \verb"\droptotalpoints"
\end{center}
to put into the right margin the total number of points for the
current question (including the points for all parts, subparts, and
subsubparts).  The command \verb"\droptotalpoints" should be given
only at the end of a paragraph or between paragraphs; if you give it
within a paragraph, it causes the paragraph to end.
\verb"\droptotalpoints" prints the total points for the current
question in the right margin, formatted by default as:
\par\nobreak\medskip
%  \leavevmode\unskip\nobreak\hfill
\makeatletter
  \leavevmode\nobreak\hfill
  \rlap{%\hskip\rightmargin  % Defined by the list environment
        \hskip\@rightmargin % Defined by exam.cls
        \hskip-\rightpointsmargin
        \llap{Total for Question 2: 25}%
  }% rlap
\makeatother
  \par
\medskip
\noindent
(Note: The number of points is followed by the value of
\verb"\marginpointname", which by default is empty.)

You can change the format used by the \verb"\droptotalpoints" command
with the
\begin{center}
  \verb"\totalformat"
\end{center}
command.  It takes one argument, and that argument becomes the command
to print the total points, right justified a distance of
\verb"\rightpointsmargin" from the right edge of the paper.  The
argument can contain the command
\begin{center}
  \begin{tabular}{ll}
    \verb"\totalpoints"& to print the number of points, and\\
    \verb"\thequestion"& to print the question number.
  \end{tabular}
\end{center}
(\verb"\totalpoints" expands to
\verb"\pointsofquestion{\arabic{question}}"; see
section~\ref{sec:pointsofq} for a discussion of the
\verb"\pointsofquestion" command.)  For example, if you give the
command
\begin{center}
  \verb"\totalformat{Question \thequestion: \totalpoints \points}"
\end{center}
then the total points will be printed in the format
\begin{center}
  Question 2: 25 points
\end{center}
and if you give the command
\begin{center}
\verb"\totalformat{\fbox{Total: \totalpoints}}"
\end{center}
then the total points will be printed as
\begin{center}
  \fbox{Total: 25}
\end{center}

Thus, if you've given the commands 
\begin{center}
  \begin{tabular}{l}
    \verb"\addpoints"\\
    \verb"\bracketedpoints"\\
    \verb"\pointdroppedatright"
  \end{tabular}
\end{center}
and you type
\begin{verbatim}
\question
\begin{parts}
\part[10]
In no more than one paragraph, explain why the earth is round.
\droppoints
\part[10]
Explain what changes to the van Allen radiation belt are needed to
make the earth into a regular icosahedron?
\droppoints
\end{parts}
\droptotalpoints
\end{verbatim}
then you'll get
\addpoints
\pointsdroppedatright
\bracketedpoints
\begin{questions}
\question
\begin{parts}
\part[10]
In no more than one paragraph, explain why the earth is round.
\droppoints
\part[10]
Explain what changes to the van Allen radiation belt are needed to
make the earth into a regular icosahedron?
\droppoints
\end{parts}
  \leavevmode\unskip\nobreak\hfill
\makeatletter
  \rlap{\hskip\rightmargin  % Defined by the list environment
        \hskip\@rightmargin % Defined by exam.cls
        \hskip-\rightpointsmargin
        \llap{Total for Question 1: 20}%
  }% rlap
\makeatother
  \par
%\droptotalpoints
\end{questions}
\noaddpoints
\nopointsinmargin
\nobracketedpoints
If you've also given the command
\begin{center}
  \verb"\totalformat{Total for Question \thequestion: [\totalpoints]}"
\end{center}
then you'll get
\addpoints
\pointsdroppedatright
\bracketedpoints
\begin{questions}
\question
\begin{parts}
\part[10]
In no more than one paragraph, explain why the earth is round.
\droppoints
\part[10]
Explain what changes to the van Allen radiation belt are needed to
make the earth into a regular icosahedron?
\droppoints
\end{parts}
%\droptotalpoints
  \leavevmode\unskip\nobreak\hfill
\makeatletter
  \rlap{\hskip\rightmargin  % Defined by the list environment
        \hskip\@rightmargin % Defined by exam.cls
        \hskip-\rightpointsmargin
        \llap{Total for Question 1: [20]}%
  }% rlap
\makeatother
  \par
\end{questions}
\noaddpoints
\nopointsinmargin
\nobracketedpoints



If you want to switch back and forth between these formats during the
exam, you can do so by giving one of the commands
\begin{center}
  \begin{tabular}{l}
    \verb"\pointsinmargin"\\
    \verb"\pointsinrightmargin"\\
    \verb"\nopointsinmargin"
  \end{tabular}
\end{center}
whenever you want to switch.

%--------------------------------------------------------------------
\subsubsection{Margin sizes when using \texttt{pointsinmargin} and
  \texttt{pointsinrightmargin}}


\begin{itemize}
\item If you give the command \verb"\pointsinmargin", then the points
  are printed right justified in the left margin, with the right edge
  a distance of \verb"\marginpointsep" from the left edge of the text
  area.  The default value of \verb"\marginpointsep" is the result of
  the command
  \begin{center}
    \verb"\setlength{\marginpointssep}{5pt}"
  \end{center}
  and you can change it by giving a new \verb"\setlength" command.
  
\item If you give the command \verb"\pointsinrightmargin", then the
  points are printed right justified in the right margin, with the
  right edge a distance of \verb"\rightpointsmargin" from the right
  edge of the paper.  The default value of \verb"\rightpointsmargin"
  is the result of the command
  \begin{center}
    \verb"\setlength{\rightpointsmargin}{1cm}"
  \end{center}
  and you can change it by giving a new \verb"\setlength" command.
\end{itemize}



%--------------------------------------------------------------------
\subsubsection{Custom question number lines}
\label{sec:qformat}

The default setup is for the question number to appear in the left
margin and for the text of the question to begin on that line.  It's
possible to change this so that the text of the question starts on the
line following the question number, and the format of the line
containing the question number is determined by the user.  This is
done using the \verb"\qformat" command.  (There is also a
\verb"\noqformat" command, which reverts to the default setup.)

To use the \verb"\qformat" command, you give the command
\begin{center}
  \verb"\qformat{Format specification}"
\end{center}
where `Format specification' must contain some stretchability (e.g.,
at least one \verb"\hfill" or \verb"\dotfill" or \verb"\hrulefill" or
\ldots) and can contain the commands
\begin{itemize}
\item \verb"\thequestion", which expands to the question number, and
\item \verb"\thepoints", which expands to the number of points
  followed by the value of the last \verb"\pointname" command (if
  points are specified for the question) or to nothing (if no points
  were specified).  (For a discussion of the \verb"\pointname"
  command, see section~\ref{sec:pointname})
\end{itemize}
For example, if you give the commands
\pointname{ \points}
\pointpoints{point}{points}
%
\begin{verbatim}
\qformat{Question \thequestion \dotfill \thepoints}
\begin{questions}
\question[10]
What did Billy Joe MacAllister throw off the Tallahatchie Bridge?
\end{questions}
\end{verbatim}
%
then you'll get
\qformat{Question \thequestion \dotfill \thepoints}
\begin{questions}
\question[10]
What did Billy Joe MacAllister throw off the Tallahatchie Bridge?
\end{questions}
%
\medskip
If you instead use the command
\begin{verbatim}
\qformat{\textbf{Question \thequestion}\quad (\thepoints)\hfill}
\end{verbatim}
then you'll get
\begin{questions}
\qformat{\textbf{Question \thequestion}\quad (\thepoints)\hfill}
\question[10]
What did Billy Joe MacAllister throw off the Tallahatchie Bridge?
\end{questions}
%
\medskip
If you instead use the command
\begin{verbatim}
\qformat{\hfill Question \thequestion\hfill (\thepoints)}
\end{verbatim}
then you'll get
\begin{questions}
\qformat{\hfill Question \thequestion\hfill (\thepoints)}
\question[10]
What did Billy Joe MacAllister throw off the Tallahatchie Bridge?
\end{questions}
\noqformat

%--------------------------------------------------------------------
\subsubsection{Adding up the points for each question}
\label{sec:adding}

You can also combine \verb"\qformat" with the \verb"\pointsofquestion"
command (see section~\ref{sec:pointsofq}): If you assign points only
to parts, subparts, and subsubparts of questions, but none to the
questions themselves, and you give the commands
\begin{verbatim}
\addpoints
\qformat{Question \thequestion\dotfill
         \emph{(\pointsofquestion{\arabic{question}} \points)}}
\end{verbatim}
then you can type
%
\begin{verbatim}
\question
\begin{parts}
\part[10]
In no more than one paragraph, explain why the earth is round.
\part[10]
What changes to the van Allen radiation belt are needed to make the
earth into a regular icosahedron?
\end{parts}
\end{verbatim}
and you'll get
%
\nopointsinmargin
\qformat{Question \thequestion\dotfill
         \emph{(20 \points)}}
\begin{questions}
\question
\begin{parts}
\part[10]
In no more than one paragraph, explain why the earth is round.
\part[10]
What changes to the van Allen radiation belt are needed to make the
earth into a regular icosahedron?
\end{parts}
\end{questions}
\noqformat


%--------------------------------------------------------------------
\subsection{Counting the questions and adding up the points}
\label{sec:Count}

The \verb"exam" document class automatically counts the numbers of
questions, parts, and subparts, and makes these numbers available as
the macros
\begin{center}
  \begin{tabular}{l}
    \verb"\numquestions"\\
    \verb"\numparts"\\
    \verb"\numsubparts"\\
    \verb"\numsubsubparts"
  \end{tabular}
\end{center}
These numbers are also printed on the screen when you run \LaTeX, and
they are placed into the \verb".log" file as well.  If you have more
than one \verb"questions" environment (for example, if your exam has
several parts, with the questions in each part numbered beginning with
``1''), then \verb"\numquestions" will hold the total number of
questions on the exam.

If you give the command
\begin{center}
  \verb"\addpoints"
\end{center}
then the class will add the total number of points that you've given
to all of the questions, parts, and subparts of the exam, and make
that total available in the macro
\begin{center}
  \verb"\numpoints"
\end{center}
(If you do give the command \verb"\addpoints", then the total number
of points will also be displayed on the screen when you run \LaTeX,
and placed into the \verb".log" file as well.)  Thus, if you give the
command \verb"\addpoints" (after the \verb"\documentclass" command and
before the \verb"\begin{document}" command), and then type
\begin{verbatim}
\begin{center}
  This exam has \numquestions\ questions, for a total of \numpoints\ 
  points.
\end{center}
\end{verbatim}
after the \verb"\begin{document}" command, then you'll get
\begin{center}
  This exam has 8 questions, for a total of 120 points.
\end{center}


\emph{Warning}: If you give the command \verb"\addpoints", your point
values for questions, parts, and subparts must not contain anything
other than digits.  For example, if you \emph{don't} give the command
\verb"\addpoints", then you can type
\begin{center}
  \verb"\question[10\%]"
\end{center}
with no problems, but this will cause errors if you've given the
command \verb"\addpoints".  The correct way to accomplish what you
want is to give the command \verb"\marginpointname{\%}" or
\verb"\pointname{\%}" and then type \verb"\question[10]" (see
section~\ref{sec:pointname}). 

If you want to temporarily turn off the adding of points (for example,
if you list both the total points for each question and the points for
each part, but you don't want to count the points twice), you can give
the command
\begin{center}
  \verb"\noaddpoints"
\end{center}
to turn off the adding of points, and the command \verb"\addpoints" to
turn it back on.

If you give the command \verb"\addpoints", then you can also use the
command \verb"\gradetable" to print a grading table (see
section~\ref{sec:Gradetable}) or the command \verb"\pointsofquestion"
to list the total points for individual questions of the exam (see
section~\ref{sec:pointsofq}).

%---------------------------------------------------------------------
\subsection{Referring to specific questions by number (cross
  references)}

You can use the standard \LaTeX{} commands \verb"\label" and
\verb"\ref" to refer to questions (or parts, or subparts, or
subsubparts) by number.  For example, if you type
%
\begin{verbatim}
The first question is question number~\ref{ques:first}.
Question number~\ref{ques:second} has both a good part
(part~\ref{part:good}) and a bad part (part~\ref{part:bad}).

\begin{questions}
\question
\label{ques:first}
This is the first question.

\question
\label{ques:second}
\begin{parts}
\part
\label{part:good}
This is the good part.

\part
\label{part:bad}
This is the \emph{bad} part.
\end{parts}

\question
Is there a question?
\end{questions}
\end{verbatim}
%
then you'll get:

\medskip

The first question is question number~\ref{ques:first}.
Question number~\ref{ques:second} has both a good part
(part~\ref{part:good}) and a bad part (part~\ref{part:bad}).

\begin{questions}
\question
\label{ques:first}
This is the first question.

\question
\label{ques:second}
\begin{parts}
\part
\label{part:good}
This is the good part.

\part
\label{part:bad}
This is the \emph{bad} part.
\end{parts}

\question
Is there a question?
\end{questions}



\bigskip

As with all other cross references in \LaTeX, you'll have to run your
file through \LaTeX{} \emph{twice} to be sure that all the cross
references are correct.


%---------------------------------------------------------------------
\subsection{Including special instructions for a group of questions}
\label{sec:uplevel}

There are two commands provided for including special instructions
for specific questions: \verb"\uplevel" and \verb"\fullwidth".  These
commands allow you to give instructions that will be set with the
left indentation appropriate for the scope of the instructions.

For example, if you are inside of a parts environment, and you want
to give directions for the next few parts, then those directions
should be indented to the level of the question of which they are
parts, i.e., up one level.  If you type
%
\begin{verbatim}
\begin{questions}
\question
Why did you come to Casablanca?

\question
\begin{parts}
\part
Why, Oh why, Oh why, Oh; why did I ever leave Ohio?

\uplevel{The following two parts should be answered in classical
Greek:}
\part
Why do birds sing?

\part
Why do fools fall in love?

\end{parts}
\end{questions}
\end{verbatim}
%
then you'll get
\begin{questions}
\question
Why did you come to Casablanca?


\question
\begin{parts}
\part
Why, Oh why, Oh why, Oh; why did I ever leave Ohio?

\uplevel{The following two parts should be answered in classical
Greek:}
\part
Why do birds sing?

\part
Why do fools fall in love?

\end{parts}
\end{questions}




\bigskip
If you want to give instructions for a group of questions, then the
indenting for those instructions should be to the outer left margin,
i.e., up one level.  For example, if you type
%
\begin{verbatim}
\begin{questions}
\question
Approximate $\displaystyle \int_0^1 \sin x^2 \, dx$ within $.001$ of
its true value.

\uplevel{Questions \ref{exact-start} through~\ref{exact-end} should
be evaluated completely, not just approximated.}

\question
\label{exact-start}
$\displaystyle \int_0^1 \frac{x^2 \, dx}{\sqrt{1-x^2}}$

\question
$\displaystyle \int_0^1 \frac{1}{1+x^2}\, dx$

\question
\label{exact-end}
$\displaystyle \int_0^{\frac{\pi}{2}} \sin^3 x \cos x \, dx$
\end{questions}
\end{verbatim}
%
you'll get
\begin{questions}
\question
Approximate $\displaystyle \int_0^1 \sin x^2 \, dx$ within $.001$ of
its true value.

\uplevel{Questions \ref{exact-start} through~\ref{exact-end} should
be evaluated completely, not just approximated.}

\question
\label{exact-start}
$\displaystyle \int_0^1 \frac{x^2 \, dx}{\sqrt{1-x^2}}$

\question
$\displaystyle \int_0^1 \frac{1}{1+x^2}\, dx$

\question
\label{exact-end}
$\displaystyle \int_0^{\frac{\pi}{2}} \sin^3 x \cos x \, dx$
\end{questions}






\bigskip
If you want to give instructions that use the full width of the page
(rather than just going up one level of indentation), then use the
\verb"\fullwidth" command.  For example, if you type
%
\begin{verbatim}
\begin{questions}
\question
This is the first question.

\question
\begin{parts}
\part
This is the first part.

\part
This is the second part.
\begin{subparts}
\subpart
This is a subpart.

\fullwidth{When you finish this exam, you should go back and
reexamine your work, both on the earlier part of this exam and in
your life up until the day of this exam, for any errors that you may
have made.}

\subpart
This is another subpart.
\end{subparts}
\end{parts}
\end{questions}
\end{verbatim}
%
then you'll get
\begin{questions}
\question
This is the first question.

\question
\begin{parts}
\part
This is the first part.

\part
This is the second part.
\begin{subparts}
\subpart
This is a subpart.

\fullwidth{When you finish this exam, you should go back and
reexamine your work, both on the earlier part of this exam and in
your life up until the day of this exam, for any errors that you may
have made.}

\subpart
This is another subpart.
\end{subparts}
\end{parts}
\end{questions}





%---------------------------------------------------------------------
\subsection{Naming the parts of a long exam}

There are two ways of naming the parts of a long exam.  The first way
uses the \verb"\fullwidth" and \verb"\uplevel" commands (see
section~\ref{sec:uplevel}), and the other way uses the standard
\verb"\part" and \verb"\section" commands.




\subsubsection{Using \texttt{fullwidth} and \texttt{uplevel}}

To place a section name in the exam, just use a \verb"\fullwidth"
command (see section~\ref{sec:uplevel}) and include whatever font
changing commands that you want to use.  For example, if you type
%
\begin{verbatim}
\begin{questions}
\question
Is there, is there balm in Gilead?

\fullwidth{\Large\bf Essay questions}

\question
Explain how the cooling of matter in the centuries following the big
bang has influenced the British parliamentary system of government.

\fullwidth{\Large\bf Laboratory questions}

\question
In the cabinet below your laboratory bench you will find a single
edged razor blade, several C-clamps, and a bottle of whiskey.  Remove
your appendix.  Do not suture until your work has been inspected.
\end{questions}
\end{verbatim}
%
then you'll get
%
\begin{questions}
\question
Is there, is there balm in Gilead?

\fullwidth{\Large\bf Essay questions}

\question
Explain how the cooling of matter in the centuries following the big
bang has influenced the British parliamentary system of government.

\fullwidth{\Large\bf Laboratory questions}

\question
In the cabinet below your laboratory bench you will find a single
edged razor blade, several C-clamps, and a bottle of whiskey.  Remove
your appendix.  Do not suture until your work has been inspected.
\end{questions}






\subsubsection{Using the standard sectioning commands}

The exam documentclass is built upon the standard article
documentclass, and so the sectioning commands used with the article
documentclass can be used here as well.  In particular, you can give
the commands \verb"\part", \verb"\part*", \verb"\section", and
\verb"\section*".  The definitions made in \verb"exam.cls" ensure that
if a \verb"\part" command appears \emph{outside of a parts
  environment} it will be interpreted as a sectioning command, while
if it appears \emph{inside} of a parts environment, it will be
interpreted as beginning a new part of a question.



You can give these commands in the middle of a questions environment
so as not to interrupt the numbering of the questions, or you can end
a questions environment, give a sectioning command, and then start a
new questions environment (which would reset the question counter
to start again with number~1).  If you give any of these commands
while inside of a questions environment, then the section titles will
be indented to the same extent that questions are indented, unless
they are given as the argument of a \verb"\fullwidth" or
\verb"\uplevel" command.  These
commands have the advantage, however, that the unstarred versions
provide automatic numbering of the parts (or sections).
%---------------------------------------------------------------------
\subsection{Leaving space for the answers}
\label{sec:LeaveSpace}

%--------------------------------------------------------------------
\subsubsection{Leaving blank space}
\label{sec:BlankSpace}

To leave a specific amount of blank space on the page for the answer
to a question, you should use the \verb"\vspace*" command.  For
example, the command \verb"\vspace*{1in}" inserts one inch of vertical
space after the line in which it appears.  (If it appears in between
paragraphs, then it inserts the space right there.)  You can also use
the \verb"\vspace" command, the difference being that any space
inserted by \verb"\vspace" will be deleted if it occurs at the top
of a new page, whereas space inserted by \verb"\vspace*" will never
be deleted.

If you want to equally distribute the blank space among the questions
on the page, then just put \verb"\vspace*{\fill}" after each question
on the page and use \verb"\newpage" to end the page.

You can also leave blank space (when solutions are not being printed)
by including an optional argument with the \verb"solution"
environment; see section~\ref{sec:SolSpace}.

%--------------------------------------------------------------------
\subsubsection{Printing lined space for answers}
\label{sec:FillLines}

You can print blank lines with the command
\begin{center}
  \verb"\fillwithlines{length}"
\end{center}
which fills vertical space of height \verb"length" with horizontal
lines that run the length of the current line.  That is, they extend
from the current left margin (which depends on whether we're in a
question, part, subpart, or subsubpart) to the right margin.

\medskip

For example, if you type
\begin{verbatim}
\begin{questions}
\question
In no more than one paragraph, explain why the earth is round.

\fillwithlines{1in}

\question
\begin{parts}
\part
What changes to the van Allen radiation belt are needed to make the
earth into a regular icosahedron?

\fillwithlines{1in}

\part
Where should the field generator be constructed if you want one of the
vertices to be located at the Royal Observatory at Greenwich?

\fillwithlines{1in}

\end{parts}
\end{questions}
\end{verbatim}
then you'll get
\begin{questions}
\question
In no more than one paragraph, explain why the earth is round.

\fillwithlines{1in}

\question
\begin{parts}
\part
What changes to the van Allen radiation belt are needed to make the
earth into a regular icosahedron?

\fillwithlines{1in}

\part
Where should the field generator be constructed if you want one of the
vertices to be located at the Royal Observatory at Greenwich?

\fillwithlines{1in}

\end{parts}
\end{questions}

\medskip
The distance between the lines is \verb"\linefillheight", the default
value of which is set with the command
\begin{center}
  \verb"\setlength\linefillheight{.25in}"
\end{center}
That value can be changed by giving a new \verb"\setlength" command.
The thickness of the lines is \verb"\linefillthickness", the default
value of which is set with the command
\begin{center}
  \verb"\setlength\linefillthickness{0.1pt}"
\end{center}
and that value can also be changed by giving a new \verb"\setlength"
command.

\medskip

If you want to fill the remaining space on the page with lines, you
should give the commands
\begin{center}
  \begin{tabular}{l}
    \verb"\fillwithlines{\fill}"\\
    \verb"\newpage"
  \end{tabular}
\end{center}
If you want to equally distribute the space for answers among the
questions on the page, then just put \verb"\fillwithlines{\fill}"
after each question on the page and use \verb"\newpage" to end the
page.

You can also insert lined space (when solutions are not being printed)
by including an optional argument with the \verb"solutionorlines"
environment; see section~\ref{sec:SolSpace}.

%--------------------------------------------------------------------
\subsubsection{Short answer questions}
\label{sec:ShortAns}


The exam class can print answer lines for short answer questions.  The
command
\begin{center}
  \verb"\answerline"
\end{center}
leaves blank vertical space of length \verb"\answerskip" and then
inserts a horizontal line of length \verb"\answerlinelength" at the
right margin, preceded by the number of the current question, part,
subpart, or subsubpart.  The default values of \verb"\answerskip" and
\verb"\answerlinelength" are set by the commands
\begin{center}
  \begin{tabular}{l}
    \verb"\setlength\answerskip{2ex}"\\
    \verb"\setlength\answerlinelength{1in}"
  \end{tabular}
\end{center}
and these can be changed by giving new \verb"\setlength" commands.
If you use the command \verb"\answerline" outside of a
\verb"questions" environment, then the number before the line will be
omitted.

For example, if you type
\begin{verbatim}
\begin{questions}
\question
Who's buried in Grant's tomb?
\answerline

\question
What was the color of George Washington's white horse?
\answerline

\question
Which is heavier: A pound of feathers, or a pound of lead?
\answerline
\end{questions}
\end{verbatim}
then you'll get
\begin{questions}
\question
Who's buried in Grant's tomb?
\answerline

\question
What was the color of George Washington's white horse?
\answerline

\question
Which is heavier: A pound of feathers, or a pound of lead?
\answerline
\end{questions}
and if you type
\begin{verbatim}
\begin{questions}
\question
Answer the following questions on the lines at the right:  
\begin{parts}
\part
Who's buried in Grant's tomb? \answerline
  
\part
\begin{subparts}
\subpart
What light through yonder window breaks? \answerline

\subpart
To be, or not to be; that is the question? \answerline
\end{subparts}
  
\part
What was the color of George Washington's white horse? \answerline
\end{parts}
\end{questions}
\end{verbatim}
then you'll get
\begin{questions}
\question
Answer the following questions on the lines at the right:  
\begin{parts}
\part
Who's buried in Grant's tomb? \answerline
  
\part
\begin{subparts}
\subpart
What light through yonder window breaks? \answerline

\subpart
To be, or not to be; that is the question? \answerline
\end{subparts}
  
\part
What was the color of George Washington's white horse? \answerline
\end{parts}
\end{questions}


%--------------------------------------------------------------------
\subsection{Multiple choice questions}
\label{sec:MulChc}

There are two environments for listing the possible answers to a
multiple choice question:
\begin{itemize}
\item The \verb"choices" environment creates a list with the choices
  as the items in the list.
\item The \verb"oneparchoices" environment lists all of the choices in
  a single paragraph (that is a continuation of the paragraph
  preceding the environment unless you leave a blank line before
  beginning the environment).
\end{itemize}
Both of these environments use upper case letters (i.e., ``A'', ``B'',
``C'', \ldots) to label the choices; to change this, see
section~\ref{sec:CustNumb}.

%--------------------------------------------------------------------
\subsubsection{The \texttt{choices} environment}
\label{sec:choices}

For example, if you type
\begin{verbatim}
\begin{questions}
\question
One of these things is not like the others; one of these things is not
the same.  Which one is different?
\begin{choices}
\choice
John

\choice
Paul

\choice
George

\choice
Ringo

\choice
Socrates

\end{choices}
    
\end{questions}
\end{verbatim}
Then you'll get
\begin{questions}
\question
One of these things is not like the others; one of these things is not
the same.  Which one is different?
\begin{choices}
\choice
John

\choice
Paul

\choice
George

\choice
Ringo

\choice
Socrates

\end{choices}
    
\end{questions}

%--------------------------------------------------------------------
\subsubsection{The \texttt{oneparchoices} environment}

If instead of the \verb"choices" environment in the example of
section~\ref{sec:choices} you use a \verb"oneparchoices" environment,
then you'll get
\begin{questions}
\question
One of these things is not like the others; one of these things is not
the same.  Which one is different?
\begin{oneparchoices}
\choice
John

\choice
Paul

\choice
George

\choice
Ringo

\choice
Socrates

\end{oneparchoices}
    
\end{questions}
If you insert a blank line before the \verb"\begin{oneparchoices}",
  then you'll get
\begin{questions}
\question
One of these things is not like the others; one of these things is not
the same.  Which one is different?

\begin{oneparchoices}
\choice
John

\choice
Paul

\choice
George

\choice
Ringo

\choice
Socrates

\end{oneparchoices}
    
\end{questions}



%--------------------------------------------------------------------
\subsection{Including solutions}
\label{sec:solutions}


 
There are two environments for printing solutions to the problems:
\begin{center}
  \begin{tabular}{rll}
    The& \verb"solution"& environment, and\\
    the& \verb"solutionorlines"& environment.
  \end{tabular}
\end{center}
If you use these environments without any optional arguments, then
they are identical: They either print the solution or nothing at all.
Whether or not solutions are printed is controlled by the two commands
\begin{center}
  \begin{tabular}{l}
    \verb"\printanswers"\\
    \verb"\noprintanswers"
  \end{tabular}
\end{center}
Using these commands, you can print the solutions for the entire exam
or for only certain parts of it.  The default is that solutions are
\emph{not} printed.  There is also an \verb"answers" option to the
exam documentclass, used as in
\begin{center}
  \verb"\documentclass[answers]{exam}"
\end{center}
that is equivalent to giving the command \verb"\printanswers" at the
beginning of the document.

Both of these environments can take an optional argument, which is an
amount of space to be inserted when solutions are not being printed.
If this optional argument is used and solutions are not being printed,
then:
\begin{itemize}
\item The \verb"solution" enviroment inserts that amount of blank
  space, and
\item the \verb"solutionorlines" environment inserts that amound of
  space with ruled lines, as if you had given a \verb"\fillwithlines"
  command (see section~\ref{sec:FillLines}).
\end{itemize}

By default, the solution is printed inside of a box (i.e., an
\verb"\fbox"), and if the solution is broken across pages, then each
piece is enclosed in a box.  There is also a \verb"\shadedsolutions"
command to instead have the solution printed in a \verb"\colorbox"
(i.e., printed on a shaded background); for this, you must load the
\verb"color" package and your printer must be capable of printing
color, or at least grayscale (see section~\ref{sec:shaded}).

\medskip

For example, if you type
\begin{verbatim}
\begin{questions}
\question Did you mutter something?

\begin{solution}
  Once upon a midnight dreary, while I pondered, weak and weary, Over
  many a quaint and curious volume of forgotten lore--- While I
  nodded, nearly napping, suddenly there came a tapping, As of some
  one gently rapping, rapping at my chamber door.  ``\,'Tis some
  visitor,'' I muttered, ``tapping at my chamber door--- Only this and
  nothing more.''
\end{solution}
\end{questions}
\end{verbatim}
and you've given the command \verb"\printanswers" or used the
documentclass option \verb"answers", then you'll get
\bigskip
\begin{questions}
\question Did you mutter something?

\begin{TheSolution}
  Once upon a midnight dreary, while I pondered, weak and weary, Over
  many a quaint and curious volume of forgotten lore--- While I
  nodded, nearly napping, suddenly there came a tapping, As of some
  one gently rapping, rapping at my chamber door.  ``\,'Tis some
  visitor,'' I muttered, ``tapping at my chamber door--- Only this and
  nothing more.''
\end{TheSolution}
\end{questions}
\bigskip

The result would be exactly the same if the above \verb"solution"
environment were replaced by a \verb"solutionorlines" environment.

By default, the solution is printed in a box whose width equals that
of the text of the current question (or part, or subpart, or
subsubpart).  That is, the indentation at the left of the solution
equals the current level of indentation.  You can change this by
enclosing the \verb"solution" or \verb"solutionorlines" environment in
the argument of a \verb"\fullwidth" or \verb"\uplevel" command.  For
example, if you type
\begin{verbatim}
\begin{questions}
\question Did you mutter something?


\fullwidth{\begin{solution}
  Once upon a midnight dreary, while I pondered, weak and weary, Over
  many a quaint and curious volume of forgotten lore--- While I
  nodded, nearly napping, suddenly there came a tapping, As of some
  one gently rapping, rapping at my chamber door.  ``\,'Tis some
  visitor,'' I muttered, ``tapping at my chamber door--- Only this and
  nothing more.''
\end{solution}}
\end{questions}
\end{verbatim}
and you've given the command \verb"\printanswers" or used the
documentclass option \verb"answers", then you'll get
\bigskip
\begin{questions}
\question Did you mutter something?


\fullwidth{\begin{TheSolution}
  Once upon a midnight dreary, while I pondered, weak and weary, Over
  many a quaint and curious volume of forgotten lore--- While I
  nodded, nearly napping, suddenly there came a tapping, As of some
  one gently rapping, rapping at my chamber door.  ``\,'Tis some
  visitor,'' I muttered, ``tapping at my chamber door--- Only this and
  nothing more.''
\end{TheSolution}}
\end{questions}



For both of these environments, if solutions are not being printed
then the contents of the environment are ignored.  If, however, you
use the optional argument to the \verb"solution" environment (see
section~\ref{sec:SolSpace}), then the requested amount of blank space
will be inserted when the solution isn't printed, and if you use the
optional argument to the \verb"solutionorlines" environment, then the
requested amount of lined space will be inserted when solutions are
not being printed.  For all of this, see section~\ref{sec:SolSpace}.


%--------------------------------------------------------------------
\subsubsection{Printing the solution on a shaded background}
\label{sec:shaded}

If you load the \verb"color" package with the command
\begin{center}
  \verb"\usepackage{color}"
\end{center}
in the preamble of your document, then you can give the command
\begin{center}
  \verb"\shadedsolutions"
\end{center}
This will cause solutions to be printed on a shaded background, which
by default is a light gray.  (Most laser printers can at least print
grayscale when printing with the correct driver.)  If you want to
change the color of the shading, you can do so by redefining the color
\verb"ShadeColor".  For example, if you give the command
\begin{center}
  \verb"\definecolor{SolutionColor}{rgb}{0.8,0.9,1}"
\end{center}
then the solutions will be printed on a light blue background
(assuming that your printer is capable of producing that color).

If you want to switch back to printing solutions inside of an
\verb"\fbox", you can do so by giving the command
\begin{center}
  \verb"\framedsolutions"
\end{center}
which returns you to the default situation.


%--------------------------------------------------------------------
\subsubsection{Customizing the heading of the solution}
\label{sec:SolCust}


The word ``\textbf{Solution:}'' that you see at the beginning of the
solutions printed by the \verb"solution" and \verb"solutionorlines"
environment cans be changed by redefining the command
\verb"\solutiontitle".  The default value was created by the command
\begin{center}
\verb"\newcommand{\solutiontitle}{\noindent\textbf{Solution:}\enspace}"
\end{center}
If, for example, you give the command
\begin{center}
\verb"\renewcommand{\solutiontitle}{\noindent\textbf{Solution:}\par\noindent}"
\end{center}
then the \verb"solution" environment above would print
\renewcommand{\solutiontitle}{\noindent\textbf{Solution:}\par\noindent}
\par\bigskip
\begin{questions}
\question Did you mutter something?

\begin{TheSolution}
  Once upon a midnight dreary, while I pondered, weak and weary, Over
  many a quaint and curious volume of forgotten lore--- While I
  nodded, nearly napping, suddenly there came a tapping, As of some
  one gently rapping, rapping at my chamber door.  ``\,'Tis some
  visitor,'' I muttered, ``tapping at my chamber door--- Only this and
  nothing more.''
\end{TheSolution}
\end{questions}
\renewcommand{\solutiontitle}{\noindent\textbf{Solution:}\enspace}


\bigskip

The appearance of the \verb"solution" and \verb"solutionorlines"
environments can actually be totally customized, if you're up to
defining a \LaTeX{} environment.  The appearance of the solution typed
into either a \verb"solution" or \verb"solutionorlines" environment is
defined by an environment called \verb"TheSolution", and you can
change the definition of \verb"TheSolution" with a
\verb"\renewenvironment" command.  That is, the \verb"solution"
environment decides whether the command \verb"\printanswers" has been
given and, if so, it feeds the contents of the \verb"solution"
environment into a \verb"TheSolution" environment.  (The behavior of a
\verb"solutionorlines" environment is similar.)  If you use
\verb"\renewenvironment" to change the definition of the
\verb"TheSolution" environment, you'll be changing the way the
solution is printed.  For the default definition of the
\verb"TheSolution" environment, see the file \verb"exam.cls".



%--------------------------------------------------------------------
\subsubsection{Leaving blank space for the answers when solutions aren't
  printed}
\label{sec:SolSpace}

Both the \verb"solution" and \verb"solutionorlines" environments take
an optional argument:
\begin{itemize}
\item In a \verb"solution" environment this is an amount of blank
  space to be left when the solutions are not being printed, and
\item in a \verb"solutionorlines" environment this is an amount of
  lined space to be left when the solutions are not being printed
\end{itemize}
For example, if you type
\begin{verbatim}
\begin{solution}[2in]
  Once upon a midnight dreary, while I pondered, weak and weary, Over
  many a quaint and curious volume of forgotten lore--- While I
  nodded, nearly napping, suddenly there came a tapping, As of some
  one gently rapping, rapping at my chamber door.  ``\,'Tis some
  visitor,'' I muttered, ``tapping at my chamber door--- Only this and
  nothing more.''
\end{solution}
\end{verbatim}
then there will be 2 inches of blank space inserted when solutions are
not being printed, and if you type
\begin{verbatim}
\begin{solutionorlines}[2in]
  Once upon a midnight dreary, while I pondered, weak and weary, Over
  many a quaint and curious volume of forgotten lore--- While I
  nodded, nearly napping, suddenly there came a tapping, As of some
  one gently rapping, rapping at my chamber door.  ``\,'Tis some
  visitor,'' I muttered, ``tapping at my chamber door--- Only this and
  nothing more.''
\end{solutionorlines}
\end{verbatim}
then there will be 2 inches of lined space inserted (as if you had
given a \verb"\fillwithlines" command; see
section~\ref{sec:FillLines}) when solutions are not being printed.




%--------------------------------------------------------------------
\subsection{Customizing the numbers}
\label{sec:CustNumb}

The default setup is that:
\begin{questions}
  \question Question numbers are arabic, and are followed by a period.
  \begin{parts}
    \part Part numbers are lower case letters, and are enclosed in
    parentheses.
    \begin{subparts}
      \subpart Subpart numbers are lower case roman, and are followed
      by a period.
      \begin{subsubparts}
        \subsubpart Subsubpart numbers are greek, and are followed by
        a right parenthesis.
        \begin{choices}
          \choice Choices are upper case letters, and are followed by
          a period.
        \end{choices}
      \end{subsubparts}
    \end{subparts}
  \end{parts}
\end{questions}
All of this can be changed.

\bigskip

To change the type of numbering, you would redefine the commands
\begin{center}
  \begin{tabular}{l}
    \verb"\thequestion"\\
    \verb"\thepartno"\\
    \verb"\thesubpart"\\
    \verb"\thesubsubpart"\\
    \verb"\thechoice"
  \end{tabular}
\end{center}
(Note: The second one listed there is \verb"\thepartno", \emph{not}
\verb"\thepart".  The command \verb"\thepart" refers to the counter
used in the article documentclass standard sectioning command
\verb"\part".)  The numbering commands available are
\begin{center}
  \begin{tabular}{ll}
    \verb"\arabic"& Regular arabic integers\\
    \verb"\alph"& Lower case letters\\
    \verb"\Alph"& Upper case letters\\
    \verb"\roman"& Lower case roman numbers\\
    \verb"\Roman"& Upper case roman numbers\\
    \verb"\greeknum"& Greek letters
  \end{tabular}
\end{center}
and any of these can be applied to the counters \verb"question",
\verb"partno", \verb"subpart", \verb"subsubpart", and \verb"choice".
(The \verb"\greeknum" command is defined by the exam documentclass,
but all of the others are standard \LaTeX{} commands.)

For example, to have questions numbered using upper case roman numbers
and parts numbered using upper case letters, you would give the
commands
\begin{center}
  \begin{tabular}{l}
    \verb"\renewcommand\thequestion{\Roman{question}}"\\
    \verb"\renewcommand\thepartno{\Alph{partno}}"
  \end{tabular}
\end{center}

\bigskip


The ``decorations'' around the numbers (i.e., the periods, or
parentheses, or \ldots) are determined by the commands
\begin{center}
  \begin{tabular}{l}
    \verb"\questionlabel"\\
    \verb"\partlabel"\\
    \verb"\subpartlabel"\\
    \verb"\subsubpartlabel"\\
    \verb"\choicelabel"
  \end{tabular}
\end{center}
the default definitions of which are:
\begin{center}
  \begin{tabular}{l}
    \verb"\newcommand\questionlabel{\thequestion.}"\\
    \verb"\newcommand\partlabel{(\thepartno)}"\\
    \verb"\newcommand\subpartlabel{\thesubpart.}"\\
    \verb"\newcommand\subsubpartlabel{\thesubsubpart)}"\\
    \verb"\newcommand\choicelabel{\thechoice.}"
  \end{tabular}
\end{center}
You can change any of these by giving \verb"\renewcommand" commands to
redefine them.


%--------------------------------------------------------------------
\section{Grading tables}
\label{sec:Gradetable}

\newlength\cwidth

The exam documentclass can print a ``grading table'', indexed either
by question number or by page numer.  That is, you can print
\begin{itemize}
\item a table listing the question numbers and the total points
  possible for each question (including all of its parts, subparts,
  and subsubparts) and leaving space for you to fill in (by hand) the
  points earned on each question, or
\item a table listing each page that has at least one question, part,
  subpart, or subsubpart with points assigned to it, the total number
  of points possible on that page, and leaving space for you to fill
  in (by hand) the points earned on that page.
\end{itemize}
In order to use this feature you must give the command
\verb"\addpoints" (see section~\ref{sec:Count}), and there must be
only one \verb"questions" environment in the entire exam.  In
addition, you must run \LaTeX{} \emph{twice} after making any changes
to the file in order to make sure that the point values are correct on
a grading table indexed by question number, and \emph{at least three
  times} for a grading table indexed by page number.  (Since the table
appears on the third run of \LaTeX{}, and the space it occupies can
change the page on which each question falls, it may take a fourth run
of \LaTeX{} for the table to have the points per page correct.)

\medbreak

The command to create a grading table is 
\begin{center}
  \verb"\gradetable",
\end{center}
and it takes two optional arguments:
\begin{enumerate}
\item The first optional argument should be either \verb"[v]" or
  \verb"[h]", to choose between a vertically oriented table or a
  horizontally oriented table, and
\item the second optional argument should be either \verb"[questions]"
  or \verb"[pages]", to choose between a table indexed by question
  number or a table indexed by page number.
\end{enumerate}
Thus,
\begin{itemize}
\item \verb"\gradetable[v][questions]" prints a vertically oriented
  table indexed by question number,
\item \verb"\gradetable[h][questions]" prints a horizontally oriented
  table indexed by question number,
\item \verb"\gradetable[v][pages]" prints a vertically oriented table
  indexed by page number, and
\item \verb"\gradetable[h][pages]" prints a horizontally oriented
  table indexed by page number.
\end{itemize}
If you leave out the optional arguments (i.e., if you give the command
\verb"\gradetable") you'll get a vertically oriented table indexed by
question number.

For example, if the exam has 8 questions, each worth a total of 15
points, and you type
\begin{verbatim}
\begin{center}
  \gradetable[v][questions]
\end{center}
\end{verbatim}
then you'll get
\begin{center}
  \cwidth=2em
  \renewcommand\arraystretch{1.5}
  \begin{tabular}{|c|c|c|}
    \hline
    Question& Points& Score\\
    \hline
    1& 15& \hbox to \cwidth{\hfill}\\
    \hline
    2& 15& \hbox to \cwidth{\hfill}\\
    \hline
    3& 15& \hbox to \cwidth{\hfill}\\
    \hline
    4& 15& \hbox to \cwidth{\hfill}\\
    \hline
    5& 15& \hbox to \cwidth{\hfill}\\
    \hline
    6& 15& \hbox to \cwidth{\hfill}\\
    \hline
    7& 15& \hbox to \cwidth{\hfill}\\
    \hline
    8& 15& \hbox to \cwidth{\hfill}\\
    \hline
    Total:& 120&\\
    \hline
  \end{tabular}
\end{center}
and if you type
\begin{verbatim}
\begin{center}
  \gradetable[h][questions]
\end{center}
\end{verbatim}
then you'll get
\begin{center}
  \cwidth=2em
  \renewcommand\arraystretch{1.5}
  \begin{tabular}{|l|c|c|c|c|c|c|c|c|c|}
    \hline
    Question:& 1& 2& 3& 4& 5& 6& 7& 8& Total\\
    \hline
    Points:&
    15&
    15&
    15&
    15&
    15&
    15&
    15&
    15&
    120\\
    \hline
    Score:&
    \hbox to \cwidth{\hfill}&
    \hbox to \cwidth{\hfill}&
    \hbox to \cwidth{\hfill}&
    \hbox to \cwidth{\hfill}&
    \hbox to \cwidth{\hfill}&
    \hbox to \cwidth{\hfill}&
    \hbox to \cwidth{\hfill}&
    \hbox to \cwidth{\hfill}&
    \\
    \hline
  \end{tabular}
\end{center}
The number of points listed for a question is the sum of the point
values for that question and all of its parts, subparts, and
subsubparts.

If those 8 questions are distributed two to a page on each of pages 2
through 5 and if you type
\begin{verbatim}
\begin{center}
  \gradetable[v][pages]
\end{center}
\end{verbatim}
then you'll get
\begin{center}
  \cwidth=2em
  \renewcommand\arraystretch{1.5}
  \begin{tabular}{|c|c|c|}
    \hline
    Page& Points& Score\\
    \hline
    2& 30& \hbox to \cwidth{\hfill}\\
    \hline
    3& 30& \hbox to \cwidth{\hfill}\\
    \hline
    4& 30& \hbox to \cwidth{\hfill}\\
    \hline
    5& 30& \hbox to \cwidth{\hfill}\\
    \hline
    Total:& 120&\\
    \hline
  \end{tabular}
\end{center}
and if you type
\begin{verbatim}
\begin{center}
  \gradetable[h][pages]
\end{center}
\end{verbatim}
then you'll get
\begin{center}
  \cwidth=2em
  \renewcommand\arraystretch{1.5}
  \begin{tabular}{|l|c|c|c|c|c|}
    \hline
    Page:& 2& 3& 4& 5& Total\\
    \hline
    Points:&
    30&
    30&
    30&
    30&
    120\\
    \hline
    Score:&
    \hbox to \cwidth{\hfill}&
    \hbox to \cwidth{\hfill}&
    \hbox to \cwidth{\hfill}&
    \hbox to \cwidth{\hfill}&
    \hbox to \cwidth{\hfill}
    \\
    \hline
  \end{tabular}
\end{center}

\medskip 

\noindent \emph{Warning:} If you have a large number of questions on
the exam, then these tables can easily become too large to fit on the
page!  If this becomes a problem, then you can use the
\verb"\pointsofquestion" command (see section~\ref{sec:pointsofq}) or
the \verb"\pointsonpage" command (see section~\ref{sec:pointsonp}) to
create a custom \verb"tabular" environment that has more rows (or
columns) than the tables produced by the \verb"gradingtable" command.

%--------------------------------------------------------------------
\subsection{Customizing the table}
\label{sec:CustTable}

There are three ways in which you can customize the default appearance
of the grading tables:
\begin{itemize}
\item You can change the words (and the fonts) that appear in the
  table.
\item You can change the width of the cells that are left blank for
  you to write in the scores.
\item You can change the value of \verb"\baselinestretch" used for the
  table.
\end{itemize}
For vertical grading tables:
\begin{center}
\begin{tabular}{l@{\qquad}l}
Command& Effect\\
\verb"\vqword{text}"& substitutes \verb"text" for ``Question''\\
\verb"\vpgword{text}"& substitutes \verb"text" for ``Page''\\
\verb"\vpword{text}"& substitutes \verb"text" for ``Points''\\
\verb"\vsword{text}"& substitutes \verb"text" for ``Score''\\
\verb"\vtword{text}"& substitutes \verb"text" for ``Total:''
\end{tabular}
\end{center}
For horizontal grading tables:
\begin{center}
\begin{tabular}{l@{\qquad}l}
Command& Effect\\
\verb"\hqword{text}"& substitutes \verb"text" for ``Question:''\\
\verb"\hpgword{text}"& substitutes \verb"text" for ``Page:''\\
\verb"\hpword{text}"& substitutes \verb"text" for ``Points:''\\
\verb"\hsword{text}"& substitutes \verb"text" for ``Score:''\\
\verb"\htword{text}"& substitutes \verb"text" for ``Total''
\end{tabular}
\end{center}
For both vertical and horizontal grading tables:
\begin{center}
  \begin{tabular}{l@{\qquad}l}
    \verb"\cellwidth{length}"& changes the width of the blank cells to
    \verb"length"\\
    \verb"\gradetablestretch{number}"& uses \verb"number" as the
    \verb"\baselinestretch"
  \end{tabular}
\end{center}
If you don't use any of these commands then you get the default
values, which are defined by the commands
\begin{center}
\begin{tabular}{l@{\qquad}l@{\qquad}l}
  \verb"\hqword{Question:}"& \verb"\vqword{Question}"&
                             \verb"\cellwidth{2em}"\\
  \verb"\hpgword{Page:}"&    \verb"\vpgword{Page}"&
                           \verb"\gradetablestretch{1.5}"\\
  \verb"\hpword{Points:}"& \verb"\vpword{Points}"\\
  \verb"\hsword{Score:}"& \verb"\vsword{Score}"\\
  \verb"\htword{Total}"& \verb"\vtword{Total:}"
\end{tabular}
\end{center}

\medskip
For example, if you type
\begin{verbatim}
\begin{center}
  \hqword{Aufgabe Nr.:}
  \hpword{Punktzahl:}
  \htword{\textbf{Summe}}
  \hsword{Davon erreicht:}
  \cellwidth{2.2em}
  \gradetable[h]
\end{center}
\end{verbatim}
then you'll get
\begin{center}
  \cwidth=2.2em
  \renewcommand\arraystretch{1.5}
  \begin{tabular}{|l|c|c|c|c|c|c|c|c|c|}
    \hline
    Aufgabe Nr.:& 1& 2& 3& 4& 5& 6& 7& 8& \textbf{Summe}\\
    \hline
    Punktzahl:&
    15&
    15&
    15&
    15&
    15&
    15&
    15&
    15&
    120\\
    \hline
    Davon erreicht:&
    \hbox to \cwidth{\hfill}&
    \hbox to \cwidth{\hfill}&
    \hbox to \cwidth{\hfill}&
    \hbox to \cwidth{\hfill}&
    \hbox to \cwidth{\hfill}&
    \hbox to \cwidth{\hfill}&
    \hbox to \cwidth{\hfill}&
    \hbox to \cwidth{\hfill}&
    \\
    \hline
  \end{tabular}
\end{center}



%--------------------------------------------------------------------
\subsection{\texttt{pointsofquestion}}
\label{sec:pointsofq}

If you give the command \verb"\addpoints" (see
section~\ref{sec:Count}), then you can use the
\begin{center}
  \verb"\pointsofquestion"
\end{center}
command.  This command takes one argument, which must be the number of
a question on the exam, and it prints the total number of points for
that question.  That is:
\begin{quote}
  \verb"\pointsofquestion{1}" prints the sum of the point values for
  question 1 and all of its parts, subparts, and subsubparts.

  \verb"\pointsofquestion{2}" prints the sum of the point values for
  question 2 and all of its parts, subparts, and subsubparts.

  Etc.
\end{quote}
The \verb"\pointsofquestion" command is used by the \verb"\gradetable"
command (see section~\ref{sec:Gradetable}), and it can be used to
create a grading table using \LaTeX's \verb"tabular" environment when
the tables produced using the \verb"\gradetable" command either don't
fit on the page or are unsuitable for some other reason.  It can also
be used in a \verb"\qformat" command to list the total number of
points of all parts, subparts, and subsubparts of a question on the
line with the question number (see section~\ref{sec:adding}).


%--------------------------------------------------------------------

\subsection{\texttt{pointsonpage}}
\label{sec:pointsonp}

If you give the command \verb"\addpoints" (see
section~\ref{sec:Count}), then you can use the \verb"\pointsonpage"
command.  This command takes one argument, which must be the number of
a page of the exam, and it prints the total number of points for all
the questions, parts, subparts, and subsubparts on that page.  That
is:
\begin{quote}
  \verb"\pointsonpage{1}" prints the sum of the point values for
  all questions, etc., on page 1,

  \verb"\pointsonpage{2}" prints the sum of the point values for
  all questions, etc., on page 2,

  Etc.
\end{quote}
This command can be used, e.g., inside of a \verb"tabular" environment
to print a custom grading table (if the grading tables that can be
printed using the \verb"\gradetable" command (see
section~ref{sec:Gradetable}) are somehow unsuitable).  It can also be
used in headers and footers to print on each page the total number of
points available on that page; for example, the command
\begin{verbatim}
  \runningfooter{}
                {}
                {Points earned: \hbox to 1in{\hrulefill}
                 out of a possible \pointsonpage{\thepage} points}
\end{verbatim}
(see section~\ref{sec:runningfooter}) will produce the footer
\samplefoot{}{}{}{Points earned: \hbox to 1in{\hrulefill} out of a
  possible 20 points} on all pages after the first.  (For another
example, see section~\ref{sec:PtsPgEx}) 

\textbf{Caution:} The command \verb"\pointsonpage{\thepage}" will only
work reliably in headers and footers, since \verb"\thepage" will only
work reliably in headers and footers.


%--------------------------------------------------------------------
%\section{Changing the page size}
\section{Changing the page margins}
\label{sec:pagesize}

The exam documentclass arranges things so that you get one inch
margins at the top, bottom, and sides no matter what size paper you
use, as long as you use the corresponding documentclass option
(\verb"a4paper", \verb"a5paper", \verb"b5paper", \verb"letterpaper",
\verb"legalpaper", \verb"executivepaper", or \verb"landscape").  If
you want to change the size of these margins, commands are provided to
change the size of the printed area.

To change the width of the printed area, you would use the
\verb"\extrawidth" command.  The \verb"\extrawidth" command takes one
argument and enlarges the width of the printed area by the amount of
the argument.  It keeps the printed area centered as it changes its
width.  If the argument is negative, then the width of the printed
area is decreased.

For example, to enlarge the left and right margins by one half inch
each, you would use the command
\begin{center}
  \verb"\extrawidth{-1in}"
\end{center}
since the printed region must shrink by one inch to allow an
additional one half inch on both sides.  To decrease the left and
right margins to three quarters of an inch each, you would use the
command 
\begin{center}
  \verb"\extrawidth{.5in}"
\end{center}
since the printed region must grow by one half inch to decrease both
margins by one quarter of an inch.

To change the height of the printed area, you must choose whether the
top or the bottom of the printed area (or possibly both) should move.
The commands for this are principally intended to allow additional
room for large headers and footers, and so they are called
\verb"\extraheadheight" and \verb"\extrafootheight".  For a full
description of these commands, see section~\ref{sec:extra-room}.

To move the top of the printed region (and any header that's present)
downwards, you use the command \verb"\extraheadheight".  This command
takes one argument and moves the top of the text and the header down
by this amount (keeping the distance between header and text
constant).  Thus, to increase the top margin by three quarters of an
inch, you would give the command
\begin{center}
  \verb"\extraheadheight{.75in}"
\end{center}
To decrease the top margin by one half inch, you would give the
command 
\begin{center}
  \verb"\extraheadheight{-.5in}"
\end{center}
The \verb"\extraheadheight" command takes an optional argument to
provide a top margin on the first page that's different from that on
all other pages.  For an explanation of this, see
section~\ref{sec:extra-room}.


To move the bottom of the printed region (and any footer that's
present) upwards, you use the command \verb"\extrafootheight".  This
command takes one argument, and moves the bottom of the text and the
footer up by this amount (keeping the distance between footer and text
constant).  Thus, to increase the bottom margin by three quarters of an
inch, you would give the command
\begin{center}
  \verb"\extrafootheight{.75in}"
\end{center}
To decrease the bottom margin by one half inch, you would give the
command
\begin{center}
  \verb"\extrafootheight{-.5in}"
\end{center}
The \verb"\extrafootheight" command takes an optional argument to
provide a bottom margin on the first page that's different from that
on all other pages.  For an explanation of this, see
section~\ref{sec:extra-room}.


%--------------------------------------------------------------------
%---------------------------------------------------------------------
\section{Headers and footers}
\label{sec:headfoot}

The following sections explain all of the technicalities of the
commands that deal with headers and footers.  There are a number of
things to explain here, and so you may find it easier to skip this
section and instead look at the examples in
sections~\ref{sec:beginexamples} through~\ref{sec:endexamples} (on
pages~\pageref{sec:beginexamples} through~\pageref{sec:endexamples}).
You can then refer back to the technical sections for the full story
on whatever isn't clear from the examples.  All of the commands
described in this section should be given after the
\verb"\documentclass" command and before the \verb"\begin{document}"
command.
  
It's also important to remember that if you use a \verb"coverpages"
environment (described in section~\ref{sec:coverpages}), then the
commands described here affect only the pages in the main section of
the exam, and not the pages of the \verb"coverpages" environment.
There are commands for headers and footers in cover pages that are
analogous to the commands described in this section; for that, see
section~\ref{sec:Coverhdft}.



%---------------------------------------------------------------------
\subsection{Page styles: Headers and/or footers}

It's the \verb"\pagestyle" command that determines whether the exam
will have headers, footers, both, or neither.  The contents of the
header and footer are specified using the commands described in
sections~\ref{sec:header} through~\ref{sec:rules}, but it's the
\verb"\pagestyle" command that determines whether the header and
footer that you construct will actually be placed onto the page.  The
\verb"\pagestyle" command should be given after the
\verb"\documentclass" command and before the \verb"\begin{document}"
  command.

To have both a header and a footer, give the command
%
\begin{center}
\verb"\pagestyle{headandfoot}"
\end{center}
%
If you want every page to have a header but no footer, give the
command
%
\begin{center}
\verb"\pagestyle{head}"
\end{center}
%
To give every page a foot but no head, give the command
%
\begin{center}
\verb"\pagestyle{foot}"
\end{center}
%
Finally, to omit both the header and the footer from the page, give
the command
%
\begin{center}
\verb"\pagestyle{empty}"
\end{center}


\bigskip


As is true in all \LaTeX{} document classes, you can change the page
style used on a single page by giving the command
%
\begin{center}
\verb"\thispagestyle{somestyle}"
\end{center}
%
somewhere on that page (where \verb"somestyle" is the style that you
want to use on that page).  This is most often needed if you use the
\verb"\maketitle" command, since that command inserts a
\verb"\thispagestyle{plain}" immediately following the title.  If you
use the \verb"\maketitle" command and you want the entire document to
use \verb"\pagestyle{headandfoot}", then you'll need to put the
command \verb"\thispagestyle{headandfoot}" immediately after the
\verb"\maketitle" command to override the \verb"\thispagestyle{plain}"
that is inserted by \verb"\maketitle".
 

%---------------------------------------------------------------------
\subsection{The three parts of the header} 
\label{sec:header}
The header is specified in three parts:
%
\begin{itemize}
\item
One part to be left justified.

\item
One part to be centered.

\item
One part to be right justified.
\end{itemize}
%
There are two different ways in which you can specify the three parts
of the header.  The first uses the single command \verb"\header" to
specify all three parts of the header, or the commands
\verb"\firstpageheader" and \verb"\runningheader" to specify a
different header for the first page (see
section~\ref{sec:runningheader}). The second uses the commands
\verb"\lhead", \verb"\chead", and \verb"\rhead", each of which takes
an optional argument to specify a different header for the first page
(see section~\ref{sec:lhead}).  All of these commands should be given
after the \verb"\documentclass" command and before the
\verb"\begin{document}" command.



%--------------------------------------------------------------------
\subsubsection{Using \texttt{header}, \texttt{firstpageheader} and
    \texttt{runningheader}}
\label{sec:runningheader}


The command \verb"\header{Text 1}{Text 2}{Text 3}" puts ``Text~1''
into the left justified header, ``Text~2'' into the centered header
and ``Text~3'' into the right justified header on every page.  If you
want the header on the first page to be different from that on the
other pages, then you should use the commands \verb"\firstpageheader"
and \verb"\runningheader", which also take three arguments and affect
either the first page or all pages except the first.


For example, to put the header
\samplehead{Math 115}{Second Exam}{July 4, 1776}{}
on every page of the exam, you would give the command
%
\begin{verbatim}
\header{Math 115}{Second Exam}{July 4, 1776}
\end{verbatim}

If you want to have different header on the first page from the header
on all other pages, you would use the commands \verb"\firstpageheader"
and \verb"\runningheader" instead of \verb"\header".
For example, if you want the above header for the first
page, but on all pages after the first you want to have the header
\samplehead{Math 115}{Second Exam (Continued)}{July 4, 1776}{}
then you would give the commands
%
\begin{verbatim}
\firstpageheader{Math 115}{Second Exam}{July 4, 1776}
\runningheader{Math 115}{Second Exam (Continued)}{July 4, 1776}
\end{verbatim}


You can leave one or more of the three parts empty.  To have the
header
\samplehead{Math 115}{Second Exam}{July 4, 1776}{}
on the first page, with the header
\samplehead{Math 115}{}{Second Exam (Continued)}{}
on all other pages, you would give the commands
%
\begin{verbatim}
\firstpageheader{Math 115}{Second Exam}{July 4, 1776}
\runningheader{Math 115}{}{Second Exam (Continued)}
\end{verbatim}
%

Any of the three parts of the header can have multiple lines.  To
specify where the line breaks should go, you just type \verb"\\". 
Thus, to have the header
\samplehead{\bf\large Math 115\\Professor Hilbert}{}
       {\bf\large First Exam\\July 4, 1776}{}
appear on every page, you would give the command
%
\begin{verbatim}
\header{\bf\large Math 115\\Professor Hilbert}%
       {}%
       {\bf\large First Exam\\July 4, 1776}
\end{verbatim}


%--------------------------------------------------------------------
\subsubsection{Using \texttt{lhead}, \texttt{chead} and \texttt{rhead}}
\label{sec:lhead}

The command \verb"\lhead{Text}" puts ``Text'' into the left justified
part of the header on every page.  The command
\verb"\lhead[Text 1]{Text 2}" puts ``Text~1'' into the left justified
header on the first page and ``Text~2'' into the left justified header
on all other pages.  The commands \verb"\chead" and \verb"\rhead" have
similar effects on the centered and right justified parts of the
header.


For example, to put the header 
\samplehead{Math 115}{Second Exam}{July 4, 1776}{}
on every page of the exam, you would give the commands
%
\begin{verbatim}
\lhead{Math 115}
\chead{Second Exam}
\rhead{July 4, 1776}
\end{verbatim}

If you want any of the three parts to have a special version to be
used only on the first page, then you just include that special
version as an optional argument (enclosed in square brackets) to the
command.  For example, if you want the above header for the first
page, but on all pages after the first you want to have the header
\samplehead{Math 115}{Second Exam (Continued)}{July 4, 1776}{}
then you would give the commands
%
\begin{verbatim}
\lhead{Math 115}
\chead[Second Exam]{Second Exam (Continued)}
\rhead{July 4, 1776}
\end{verbatim}


You can leave one or more of the three parts empty.  To have the
header
\samplehead{Math 115}{Second Exam}{July 4, 1776}{}
on the first page, with the header
\samplehead{Math 115}{}{Second Exam (Continued)}{}
on all other pages, you would give the commands
%
\begin{verbatim}
\pagestyle{headandfoot}
\lhead{Math 115}
\chead[Second Exam]{}
\rhead[July 4, 1776]{Second Exam Continued)}
\end{verbatim}
%

Any of the three parts of the header can have multiple lines.  To
specify where the line breaks should go, you just type \verb"\\". 
Thus, to have the header
\samplehead{\bf\large Math 115\\Professor Hilbert}{}
       {\bf\large First Exam\\July 4, 1776}{}
appear on every page, you would give the commands
%
\begin{verbatim}
\lhead{\bf\large Math 115\\Professor Hilbert}
\chead{}
\rhead{\bf\large First Exam\\July 4, 1776}
\end{verbatim}


%---------------------------------------------------------------------
\subsubsection*{Leaving extra room for multiple line headers}

See section~\ref{sec:extra-room} for a description of the
\verb"\extraheadheight" command.


%---------------------------------------------------------------------

\subsection{The three parts of the footer}
\label{sec:footer}

The footer is composed of three parts, the whole setup being similar
to that for the header (see section~\ref{sec:header}).  There are two
different ways in which you can specify the three parts of the footer.
The first uses the single command \verb"\footer" to specify all three
parts of the footer, or the commands \verb"\firstpagefooter" and
\verb"\runningfooter" to specify a different footer for the first page
(see section~\ref{sec:runningfooter}).  The second uses the commands
\verb"\lfoot", \verb"\cfoot", and \verb"\rfoot", each of which takes
an optional argument to specify a different footer for the first page
(see section~\ref{sec:lfoot}).  All of these commands should be given
after the \verb"\documentclass" command and before the
\verb"\begin{document}" command.

%--------------------------------------------------------------------
\subsubsection{Using \texttt{footer}, \texttt{firstpagefooter}
   and \texttt{runningfooter}}
\label{sec:runningfooter}

The command \verb"\footer{Text 1}{Text 2}{Text 3}" puts ``Text~1''
into the left justified footer, ``Text~2'' into the centered footer
and ``Text~3'' into the right justified footer on every page.  If you
want the footer on the first page to be different from that on the
other pages, then you should use the commands \verb"\firstpagefooter"
and \verb"\runningfooter", which also take three arguments and affect
either the first page or all pages except the first.


For example, to have an empty footer on the first page and the footer
\samplefoot{}{}{Page 3 of 5}{}
on all pages after the first, you would give the commands
%
\begin{verbatim}
\firstpagefooter{}{}{}
\runningfooter{}{Page \thepage\ of \numpages}{}
\end{verbatim}
%
(For an explanation of the \verb"\numpages" command, see
section~\ref{sec:numpages}.)  



%--------------------------------------------------------------------
\subsubsection{Using \texttt{lfoot}, \texttt{cfoot} and \texttt{rfoot}}
\label{sec:lfoot}

The command \verb"\lfoot{Text}" puts ``Text'' into the left justified
part of the footer on every page.  The command
\verb"\lfoot[Text 1]{Text 2}" puts ``Text~1'' into the left justified
footer on the first page and ``Text~2'' into the left justified footer
on all other pages.  The commands \verb"\cfoot" and \verb"\rfoot" have
similar effects on the centered and right justified parts of the
footer. 


For example, to have an empty footer on the first page and the footer
\samplefoot{}{}{Page 3 of 5}{}
on all pages after the first, you would give the commands
%
\begin{verbatim}
\lfoot{}
\cfoot{}
\rfoot[]{Page \thepage\ of \numpages}
\end{verbatim}
%
(For an explanation of the \verb"\numpages" command, see
section~\ref{sec:numpages}.)  



%---------------------------------------------------------------------
\subsubsection*{Leaving extra room for multiple line footers}

See section~\ref{sec:extra-room} for a description of the
\verb"\extrafootheight" command.




%---------------------------------------------------------------------
\subsection{Leaving extra room for multiple line headers and footers}
\label{sec:extra-room}

\subsubsection*{Headers}
If you specify more than one or two lines for any part of the header,
then you may want to move the header down slightly so that it doesn't
run off of the top of the paper.  The command for this is
\verb"\extraheadheight".  You can also use the \verb"\extraheadheight"
command to adjust the size of the text area.  The
\verb"\extraheadheight" command never changes the distance between the
header and the text.

For example, to move the header and the text a half inch
down from the top of the page, you would give the command
%
\begin{center}
\verb"\extraheadheight{.5in}"
\end{center}
%
You can also specify a negative distance to \verb"extraheadheight" to
move the header up closer to the top of the paper.  For example, the
command
%
\begin{center}
\verb"\extraheadheight{-.25in}"
\end{center}
%
moves the header one quarter inch closer to the top of the paper.

If you want to have a different value for \verb"\extraheadheight" on
the first page from that on the pages after the first, then use the
same syntax as in the \verb"\lhead", \verb"\chead", and \verb"\rhead"
commands: Include an optional argument giving the extra head height
for the first page, and the required argument will apply only to
those pages after the first.  For example, the command
%
\begin{center}
\verb"\extraheadheight[.5in]{.25in}"
\end{center}
%
gives a half inch of extra head height on the first page and a
quarter inch of extra head height on all pages after the first. 
If you say
%
\begin{center}
\verb"\extraheadheight[.5in]{}"
\end{center}
%
then this will be interpreted as if it was
%
\begin{center}
\verb"\extraheadheight[.5in]{0in}"
\end{center}
%
Note that the braces \emph{must} appear.


If you give an \verb"\extraheadheight" command, it should be after the
\verb"\documentclass" command but before the \verb"\begin{document}"
command.  The \verb"\extraheadheight" command can also be used to
change the size of the text region (see section~\ref{sec:pagesize}).




\subsubsection*{Footers}
To leave extra room for multiple line footers, you use the command
\verb"\extrafootheight".  Thus, to move the footer one half inch
higher up on the paper, you would give the command
%
\begin{center}
\verb"\extrafootheight{.5in}"
\end{center}
%
If you wanted to move the footer an eighth of an inch lower down on
the paper, you would give the command
%
\begin{center}
\verb"\extrafootheight{-.125in}"
\end{center}

If you want to have a different value for \verb"\extrafootheight" on
the first page from that on the pages after the first, then use the
same syntax as in the \verb"\lfoot", \verb"\cfoot", and \verb"\rfoot"
commands: Include an optional argument giving the extra foot height
for the first page, and the required argument will apply only to
those pages after the first.  For example, the command
%
\begin{center}
\verb"\extrafootheight[.5in]{.25in}"
\end{center}
%
gives a half inch of extra foot height on the first page and a
quarter inch of extra foot height on all pages after the first. 
If you say
%
\begin{center}
\verb"\extrafootheight[.5in]{}"
\end{center}
%
then this will be interpreted as if it was
%
\begin{center}
\verb"\extrafootheight[.5in]{0in}"
\end{center}
%
Note that the braces \emph{must} appear.


If you give an \verb"\extrafootheight" command, it should be after the
\verb"\documentstyle" command but before the \verb"\begin{document}"
command.  The \verb"\extrafootheight" command can also be used to
change the size of the text region (see section~\ref{sec:pagesize}).






%---------------------------------------------------------------------
\subsection{Horizontal rules}
\label{sec:rules}

The \verb"exam" document class make it easy to put a horizontal rule
under the header and one above the footer.  It is also easy to do
this for the pages after the first page without affecting the first
page.

\begin{itemize}
\item
The command \verb"\runningheadrule" puts a horizontal rule below the
header on all pages after the first.

\item
The command
\verb"\firstpageheadrule" puts a rule under the header of only the
first page.

\item
The command \verb"\headrule" is equivalent to the two
commands \verb"\firstpageheadrule" and \verb"\runningheadrule".

\item
The command \verb"\runningfootrule" puts a horizontal rule above the
footer on all pages after the first.

\item
The command
\verb"\firstpagefootrule" puts a rule above the footer of only the
first page.

\item
The command \verb"\footrule" is equivalent to the two
commands \verb"\firstpagefootrule" and \verb"\runningfootrule".

\end{itemize}
%
For example, to have the header
\samplehead{Math 115}{First Exam}{July 4, 1776}{}
on the first page, with the header
\samplehead{Math 115}{First Exam}{July 4, 1776}{\hrule}
on all pages after the first, give the commands
%
\begin{verbatim}
\runningheadrule
\lhead{Math 115}
\chead{First Exam}
\rhead{July 4, 1776}
\end{verbatim}


\bigskip

To have no footer on the first page, and the footer
\samplefoot{\hrule}{}{Page 3 of 5}{}
on all pages after the first, you would give the commands
%
\begin{verbatim}
\runningfootrule
\lfoot{}
\cfoot[]{Page \thepage\ of \numpages}
\rfoot{}
\end{verbatim}






%---------------------------------------------------------------------
\subsection{Listing the number of pages in the exam}
\label{sec:numpages}

The \verb"exam" document class defines the command \verb"\numpages"
so that it will expand to the number of pages in the exam.  Thus, to
have the footer
\samplefoot{}{}{Page \thepage\ of \numpages}{}
you should give the commands
%
\begin{verbatim}
\lfoot{}
\cfoot{Page \thepage\ of \numpages}
\rfoot{}
\end{verbatim}
%
For a description of the commands \verb"\lfoot", \verb"\cfoot", and
\verb"\rfoot", see section~\ref{sec:footer}.

As with all other cross referencing commands in \LaTeX, you'll have to
run the file through \LaTeX{} \emph{twice} to be sure that
\verb"\numpages" is correct.

%---------------------------------------------------------------------
\subsection{Treating the last page differently}
\label{sec:lastpage}

If you want to vary the text that appears in the header or footer on
the last page of the exam, you  should use the command
\verb"\iflastpage".  The command
%
\begin{center}
\verb"\iflastpage{Text 1}{Text 2}"
\end{center}
%
expands to `Text~1' on the last page, and to `Text~2' on all
pages before the last.  Thus, to have the footer
\samplefoot{}{}{Please go on to the next page\ldots}{}
on all pages before the last page, and the footer
\samplefoot{}{}{End of exam}{}
on the last page, you would give the commands
%
\begin{verbatim}
\lfoot{}
\cfoot{\iflastpage{End of exam}{Please go on to the next page\ldots}}
\rfoot{}
\end{verbatim}
%
For a description of the commands \verb"\lfoot", \verb"\cfoot", and
\verb"\rfoot", see section~\ref{sec:footer}.

As with all other cross referencing commands in \LaTeX, you'll have to
run the file through \LaTeX{} \emph{twice} to be sure that
\verb"\iflastpage" correctly detects the last page.

%--------------------------------------------------------------------
\subsection{Treating odd and even numbered pages differently}

If you'd like odd and even numbered pages to be treated differently
(perhaps because you'll be printing onto both sides of the paper), you
should use the \verb"\oddeven" command.

The \verb"\oddeven" command takes two arguments.  If the current page
number is odd it expands to the first argument; otherwise, it expands
to the second argument.

For example, to have the page number printed in the right head on odd
numbered pages and in the left head on even numbered pages, you would
use the commands
\begin{center}
\begin{verbatim}
\rhead{\oddeven{\thepage}{}}
\lhead{\oddeven{}{\thepage}}
\chead{}
\end{verbatim}
\end{center}
If you wanted the footer of the even numbered pages to be empty and
the footer of the odd numbered pages to contain the message ``Please
continue\dots'', except that the last page of the exam should have an empty
footer whether its page number is even or odd, then you would use the
commands 
\begin{center}
\begin{verbatim}
\lfoot{}
\rfoot{}
\cfoot{\oddeven{\iflastpage{}{Please continue\dots}}{}}
\end{verbatim}
\end{center}
(see section~\ref{sec:lastpage} for an explanation of
\verb"\iflastpage"). 

Although the \verb"\oddeven" command can be used anywhere in the
document (i.e., not just in headers and footers), it is reliable only
in headers and footers.  This is because \LaTeX{} generally processes
more text than can fit on the current page before it outputs a page.
When the \verb"\oddeven" command is encountered it will act as though
it will appear on the current page whether it appears on that page or
on the following page.

%--------------------------------------------------------------------
\subsection{Questions that span multiple pages}
\label{sec:QuesSpan}

The exam document class provides the commands \verb"\ifcontinuation",
\verb"\ContinuedQuestion", \verb"\ifincomplete" and
\verb"\IncompleteQuestion" to enable headers and footers that announce
whether this page is continuing a question begun on an earlier page
(and, if so, the number of that question) and whether the question at
the bottom of this page continues onto the next page (and, if so, the
number of that question).  

\emph{These commands are only guaranteed to work in headers and
  footers.}  If you use any of them elsewhere, they may be fooled by
\LaTeX's practice of typesetting a bit too much material to fit on a
page before it chooses the pagebreak and ships out the page.  They
also assume that there is only one \verb"questions" environment in the
entire exam.

There is also a command \verb"\nomorequestions" to mark the point at
which the last question ends, which can be useful if you want to
include supplementary material (e.g., tables or figures) at the end of
the exam but you don't want that material to be described as
continuing the last question.

\medskip

In more detail:
\begin{itemize}
\item \verb"\ifcontinuation{Text 1}{Text 2}"\\
  Expands to `Text 1' if this page begins with a part, subpart or
  subsubpart of a question begun on an earlier page, and expands to
  `Text 2' if this page begins with a new question.
  
  More specifically, it expands to `Text 2' if either
  \begin{itemize}
  \item a question begins on this page before any part, subpart, or
    subsubpart begins, or
  \item The current page is later than a page with the
    \verb"\nomorequestions" command
  \end{itemize}
  and otherwise it expands to `Text 1'.

\item \verb"\ContinuedQuestion"\\
  If this page does begin with a continuation of a question begun on
  an earlier page, then \verb"\ContinuedQuestion" expands to the
  number of that question.
  
  More specifically, \verb"\ContinuedQuestion" expands to a positive
  number if either
  \begin{itemize}
  \item this page doesn't contain the beginning of any question, part,
    subpart, or subsubpart, or
  \item this page has a part, subpart, or subsubpart that appears
    before any question appears,
  \end{itemize}
  in which case \verb"\ContinuedQuestion" expands to the number of the
  last question begun before this page.
  
\item \verb"\ifincomplete{Text 1}{Text 2}"\\
  Expands to \verb"Text 1" if we have not yet encountered a
  \verb"\nomorequestions" command and if the last question begun on or
  before this page has a part, subpart, or subsubpart that begins on a
  later page. Otherwise, it expands to \verb"Text 2".
  
\item \verb"\IncompleteQuestion"\\
  If the last question begun on or before this page has a part,
  subpart, or subsubpart that begins on a later page, then this
  expands to the number of that question.
  
\item \verb"\nomorequestions"\\
  No page following the page on which this command was given will be
  considered to continue a question from an earlier page.  Thus, if
  you give this command after the last question is complete, then you
  can include extra material (e.g., tables for use on the exam)
  without having those pages labelled as continuing the last question
  on the exam.
\end{itemize}

\medskip

Thus, to have the header
\samplehead{}{First Exam}{Page 4}{}
%
on all pages that begin a new question and the header
\samplehead{Question 6 continues\ldots}{First Exam}{Page 4}{}
%
on all pages that continue a question begun on a previous page, you
would give the commands
\begin{verbatim}
\lhead{\ifcontinuation{Question \ContinuedQuestion\ continues\ldots}{}}
\chead{First Exam}
\rhead{Page \thepage}
\end{verbatim}
(See section~\ref{sec:lhead} for an explanation of \verb"\lhead",
\verb"\chead" and \verb"\rhead".)

\medskip

To have the footer
\samplefoot{\hrule}{}{Question 3 continues\ldots}{}
%
on all pages that end with a question that will be continued onto the
next page, the footer
\samplefoot{\hrule}{}{End of Exam}{}
%
on the last page, and the footer
\samplefoot{\hrule}{}{Exam continues\ldots}{}
%
on all other pages, you would give the commands
\begin{verbatim}
\footrule
\lfoot{}
\newcommand\continues{\ifincomplete{\incompletemessage}{Exam continues\ldots}}
\newcommand\incompletemessage{Question \IncompleteQuestion\ continues\ldots}
\cfoot{\iflastpage{End of Exam}{\continues}}
\rfoot{}
\end{verbatim}
(See section~\ref{sec:lfoot} for an explanation of \verb"\lfoot",
\verb"\cfoot" and \verb"\rfoot", and section~\ref{sec:lastpage} for an
explanation of \verb"\iflastpage").






%--------------------------------------------------------------------
\subsection{Examples}

\begin{example}
\label{sec:beginexamples}
To have the header
\samplehead{Math 115}{First Exam}{July 4, 1776}{}
on the first page, the header
\samplehead{Math 115}{First Exam, Page 2 of 5}{July 4, 1776}{\hrule}
on all pages after the first, and no footer on any page, give the
commands
%
\begin{verbatim}
\pagestyle{head}
\runningheadrule
\firstpageheader{Math 115}{First Exam}{July 4, 1776}
\runningheader{Math 115}%
              {First Exam, Page \thepage\ of \numpages}%
              {July 4, 1776}
\end{verbatim}
%
Alternatively, you could give the commands
%
\begin{verbatim}
\pagestyle{head}
\runningheadrule
\lhead{Math 115}
\chead[First Exam]{First Exam, Page \thepage\ of \numpages}
\rhead{July 4, 1776}
\end{verbatim}
%
\end{example}

%--------------------------------------------------------------------

\begin{example}
To have the header
\samplehead{Math 115}{First Exam}{July 4, 1776}{}
on the first page, no header on the pages after the first, no footer
on the first page, and the footer
\samplefoot{\hrule}{Math 115}{First Exam}{Page 2 of 5}
on all pages after the first, give the commands
%
\begin{verbatim}
\pagestyle{headandfoot}
\runningfootrule
\firstpageheader{Math 115}{First Exam}{July 4, 1776}
\runningheader{}{}{}
\firstpagefooter{}{}{}
\runningfooter{Math 115}{First Exam}{Page \thepage\ of \numpages}
\end{verbatim}
%
Alternatively, you could give the commands
\begin{verbatim}
\pagestyle{headandfoot}
\runningfootrule
\lhead[Math 115]{}
\chead[First Exam]{}
\rhead[July 4, 1776]{}
\lfoot[]{Math 115}
\cfoot[]{First Exam}
\rfoot[]{Page \thepage\ of \numpages}
\end{verbatim}
%
\end{example}

%--------------------------------------------------------------------

\begin{example}
To have the header
\samplehead{\large\bf Mathematics 115\\
            First Exam, July 4, 1776}%
           {}{\large\bf Name:\enspace\hbox to 2in{\hrulefill}}{}
on the first page, the header
\samplehead{\large\bf Mathematics 115\\
            First Exam, July 4, 1776}%
           {}{}{}
on all pages after the first, an empty foot on the first page, and
the footer
\samplefoot{}{}{Page 2}{}
on all pages after the first, give the commands
%
\begin{verbatim}
\pagestyle{headandfoot}
\firstpageheader{\large\bf Mathematics 115\\
                 First Exam, July 4, 1776}%
                {}{\large\bf Name:\enspace\hbox to 2in{\hrulefill}}
\runningheader{\large\bf Mathematics 115\\
               First Exam, July 4, 1776}{}{}
\firstpagefooter{}{}{}
\runningfooter{}{Page \thepage}{}
\end{verbatim}
%
Alternatively, you could give the commands
\begin{verbatim}
\pagestyle{headandfoot}
\lhead{\large\bf Mathematics 115\\ First Exam, July 4, 1776}
\chead{}
\rhead[\large\bf Name:\enspace\hbox to 2in{\hrulefill}]{}
\lfoot{}
\cfoot[]{Page \thepage}
\rfoot{}
\end{verbatim}
\end{example}

%--------------------------------------------------------------------

\begin{example}
To have the header
\samplehead{}{Wellesley College\\
              Second Semester Final Examination, Spring 1993\\
              Mathematics 115}{}{}
on the first page, the header
\samplehead{}{Wellesley College\\
              Second Semester Final Examination, Spring 1993\\
              Mathematics 115 (Continued)}{}{}
on all pages after the first, the footer
\samplefoot{}{}{Page 3 of 10}{Please go on to the next page\ldots}
on all pages \emph{except the last} page, and the footer
\samplefoot{}{}{Page 10 of 10}{End of exam.}
on the last page, give the commands
%
\begin{verbatim}
\pagestyle{headandfoot}
\extraheadheight{.25in}
\firstpageheader{}{Wellesley College\\
                   Second Semester Final Examination, Spring 1993\\
                   Mathematics 115}{}
\runningheader{}{Wellesley College\\
                 Second Semester Final Examination, Spring 1993\\
                 Mathematics 115 (Continued)}{}
\footer{}{Page \thepage\ of \numpages}%
   {\iflastpage{End of exam.}{Please go on to the next page\ldots}}
\end{verbatim}
%
Alternatively, you could give the commands
\begin{verbatim}
\pagestyle{headandfoot}
\extraheadheight{.25in}
\lhead{}
\chead[Wellesley College\\
       Second Semester Final Examination, Spring 1993\\
       Mathematics 115]
      {Wellesley College\\
       Second Semester Final Examination, Spring 1993\\
       Mathematics 115 (Continued)}
\rhead{}
\lfoot{}
\cfoot{Page \thepage\ of \numpages}
\rfoot{\iflastpage{End of exam.}{Please go on to the next page\ldots}}
\end{verbatim}

\end{example}

%--------------------------------------------------------------------

\begin{example}
To have the header
\samplehead{}{Wellesley College\\
              Second Semester Final Examination, Spring 1993\\
              Mathematics 115}{}{}
on the first page, the header
\samplehead{Mathematics 115 (Continued)}{}{Spring, 1993}{}
on all pages after the first, the footer
\samplefoot{}{}{Page 3 of 10}{Please go on to the next page\ldots}
on all pages \emph{except the last} page, and the footer
\samplefoot{}{}{Page 10 of 10}{End of exam.}
on the last page, give the commands
%
\begin{verbatim}
\pagestyle{headandfoot}
\extraheadheight[.25in]{}
\firstpageheader{}{Wellesley College\\
                   Second Semester Final Examination, Spring 1993\\
                   Mathematics 115}{}
\runningheader{Mathematics 115}{}{Spring, 1993}
\footer{}{Page \thepage\ of \numpages}%
   {\iflastpage{End of exam.}{Please go on to the next page\ldots}}
\end{verbatim}
%
Alternatively, you could give the commands
\begin{verbatim}
\pagestyle{headandfoot}
\extraheadheight[.25in]{}
\lhead[]{Mathematics 115}
\chead[Wellesley College\\
       Second Semester Final Examination, Spring 1993\\
       Mathematics 115]{}
\rhead[]{Spring, 1993}
\lfoot{}
\cfoot{Page \thepage\ of \numpages}
\rfoot{\iflastpage{End of exam.}{Please go on to the next page\ldots}}
\end{verbatim}
\end{example}

%--------------------------------------------------------------------

\begin{example}
To have the header
%
\samplehead{Math 115}{First Exam}{July 4, 1776}{}
on the first page, the header
%
\samplehead{Question 3 continues\ldots}{First Exam}{Page 3 of 10}{\hrule}
on all pages after the first that continue a question begun on an
earlier page, the header
%
\samplehead{}{First Exam}{Page 3 of 10}{\hrule}
on all pages after the first that don't continue a question begun on
an earlier page, the footer
%
\samplefoot{}{}{}{Question 6 continues on the next page\ldots}
%
on all pages whose last question continues onto the following page,
and no footer on pages that don't have a question that continues onto
the following page, give the commands
%
\begin{verbatim}
\pagestyle{headandfoot}
\runningheadrule
\newcommand\continuedmessage{%
  \ifcontinuation{Question \ContinuedQuestion\ continues\ldots}{}%
}
\firstpageheader{Math 115}{First Exam}{July 4, 1776}
\runningheader{\continuedmessage}%
              {First Exam}%
              {Page \thepage\ of \numpages}%
\footer{}%
       {}%
       {\ifincomplete{Question \IncompleteQuestion\ continues
                      on the next page\ldots}{}}
\end{verbatim}
%
Alternatively, you could give the commands
%
\begin{verbatim}
\pagestyle{headandfoot}
\runningheadrule
\newcommand\continuedmessage{%
  \ifcontinuation{Question \ContinuedQuestion\ continues\ldots}{}%
}
\lhead[Math 115]{\continuedmessage}
\chead[First Exam]{First Exam}
\rhead[July 4, 1776]{Page \thepage\ of \numpages}
\lfoot{}
\cfoot{}
\rfoot{\ifincomplete{Question \IncompleteQuestion\ continues
                     on the next page\ldots}{}}
\end{verbatim}
%
\end{example}

%--------------------------------------------------------------------
\begin{example}
\label{sec:PtsPgEx}
  
To have the header
%
\samplehead{Math 115}{Second Exam}{July 4, 1776}{}
%
on the first page, the header
%
\samplehead{Second Exam}{}{July 4, 1776}{}
%
on all pages after the first, no footer on the first page, and the
footer
%
\samplefoot{}{Page 3 of 5}{}{Points earned: \hbox to .5in{\hrulefill}\\
               out of a possible 20 points}
%
on all pages after the first, give the commands
\begin{verbatim}
\pagestyle{headandfoot}
\firstpageheader{Math 115}{Second Exam}{July 4, 1776}
\runningheader{Second Exam}{}{July 4, 1776}
\runningfooter{Page \thepage\ of \numpages}
              {}
              {Points earned: \hbox to .5in{\hrulefill}\\
               out of a possible \pointsonpage{\thepage} points}
\end{verbatim}
%
Alternatively, you could give the commands
%
\begin{verbatim}
\pagestyle{headandfoot}
\lhead[Math 115]{Second Exam}
\chead[Second Exam]{}
\rhead{July 4, 1776}
\lfoot[]{Page \thepage\ of \numpages}
\cfoot{}
\rfoot[]{Points earned: \hbox to .5in{\hrulefill}\\
               out of a possible \pointsonpage{\thepage} points}
\end{verbatim}
%
\end{example}

%--------------------------------------------------------------------

\begin{example}
\label{sec:endexamples}

To have the header
%
\samplehead{Math 115}{Second Exam}{July 4, 1776}{}
%
on the first page, the header
%
\samplehead{}{Second Exam}{July 4, 1776}{}
%
on all odd numbered pages after the first, the header
%
\samplehead{Question 3 continues\ldots}{Math 115}{}{}
%
on all even numbered pages that continue a question begun on an
earlier page, the header
%
\samplehead{}{Math 115}{}{}
%
on all even numbered pages that don't continue a question begun on an
earlier page, the footer
%
\samplefoot{}{}{}{Page 5 of 10\\Question 6 continues\ldots}
%
on all odd numbered pages that have a question that continues onto the
following page, the footer
%
\samplefoot{}{}{}{Page 5 of 10}
%
on all odd numbered pages that don't have a question that continues
onto the following page, and the footer
%
\samplefoot{}{Page 4 of 10}{}{}
%
on all even numbered pages, give the commands
\begin{verbatim}
\pagestyle{headandfoot}
\newcommand\continuedmessage{%
  \ifcontinuation{Question \ContinuedQuestion\ continues\ldots}{}%
}
\newcommand\oddrightfoot{%
  \ifincomplete{Page \thepage\ of \numpages\\
                Question \IncompleteQuestion\ continues\ldots}%
                {Page \thepage\ of \numpages}
}
\firstpageheader{Math 115}{Second Exam}{July 4, 1776}
\runningheader{\oddeven{}{\continuedmessage}}%
              {\oddeven{Second Exam}{Math 115}}%
              {\oddeven{July 4, 1776}{}}
\footer{\oddeven{}{Page \thepage\ of \numpages}}%
       {}%
       {\oddeven{\oddrightfoot}{}}
\end{verbatim}
%
Alternatively, you could give the commands
%
\begin{verbatim}
\pagestyle{headandfoot}
\newcommand\continuedmessage{%
  \ifcontinuation{Question \ContinuedQuestion\ continues\ldots}{}%
}
\newcommand\oddrightfoot{%
  \ifincomplete{Page \thepage\ of \numpages\\
                Question \IncompleteQuestion\ continues\ldots}%
                {Page \thepage\ of \numpages}
}
\lhead[Math 115]{\oddeven{}{\continuedmessage}}
\chead{\oddeven{Second Exam}{Math 115}}
\rhead{\oddeven{July 4, 1776}{}}
\lfoot{\oddeven{}{Page \thepage\ of \numpages}}
\cfoot{}
\rfoot{\oddeven{\oddrightfoot}{}}
\end{verbatim}

\end{example}


%--------------------------------------------------------------------
%--------------------------------------------------------------------
\section{Cover pages}
\label{sec:coverpages}


There is a \verb"coverpages" environment that allows you to have one
or more pages before page~1 of the exam.  By default, there are no
headers or footers printed on the coverpages, but if you put the page
number into the headers and/or footers (using the commands described
in section~\ref{sec:Coverhdft}), the page numbers of cover pages are
printed in roman numerals.

The \verb"coverpages" environment must begin and end \emph{before} the
beginning of the \verb"questions" environment.  You begin the
environment with the command
\begin{center}
  \verb"\begin{coverpages}"
\end{center}
and you end it with the command
\begin{center}
  \verb"\end{coverpages}"
\end{center}
In between those two commands you can put whatever you want, except
that you are not allowed to begin the \verb"questions" environment
until you've ended the \verb"coverpages" environment.  The command
\verb"\begin{coverpages}" sets the page number to 1 and causes any
page numbers printed (using the commands described in
section~\ref{sec:Coverhdft}) to appear as roman numerals.  The command
\verb"\end{coverpages}" causes the current page to end, sets the page
number of the following page to 1, and changes page numbering back to
arabic numerals.

%--------------------------------------------------------------------
\subsection{Headers and footers}
\label{sec:Coverhdft}

By default, there are no headers or footers on the cover pages.  If
you'd like to have headers and/or footers on the cover pages, there
are commands to create them that are completely analogous to the
commands described in section~\ref{sec:headfoot} for headers and/or
footers in the main part of the document.  The commands for headers
and footers in coverpages and the corresponding commands for the main
exam pages are:

\begin{center}
  \begin{tabular}{l@{\qquad\qquad}l}
\multicolumn{1}{l}{Cover Pages Commands}& Main Pages Commands\\[\medskipamount]
\verb"\coverheader"&\verb"\header"\\
\verb"\coverrunningheader"&\verb"\runningheader"\\
\verb"\coverfirstpageheader"&\verb"\firstpageheader"\\[\medskipamount]

\verb"\coverlhead"&\verb"\lhead"\\
\verb"\coverchead"&\verb"\chead"\\
\verb"\coverrhead"&\verb"\rhead"\\[\medskipamount]

\verb"\coverfooter"&\verb"\footer"\\
\verb"\coverrunningfooter"&\verb"\runningheader"\\
\verb"\coverfirstpagefooter"&\verb"\firstpageheader"\\[\medskipamount]

\verb"\coverlfoot"&\verb"\lfoot"\\
\verb"\covercfoot"&\verb"\cfoot"\\
\verb"\coverrfoot"&\verb"\rfoot"\\[\medskipamount]

\verb"\coverextraheadheight"&\verb"\extraheadheight"\\
\verb"\coverextrafootheight"&\verb"\extrafootheight"
  \end{tabular}
\end{center}
The commands \verb"\coverlhead", \verb"\coverchead",
\verb"\coverrhead", \verb"\coverlfoot", \verb"\covercfoot",
\verb"\coverrfoot", \verb"\coverextraheadheight", and
\verb"\coverextrafootheight" all all take the same optional arguments
(for special treatment of page number 1) as the corresponding commands
for the main pages.  For an explanation of these commands, see
section~\ref{sec:headfoot}.




% The following is necessary to avoid warnings, since we have multiple
% questions environments in this documentation file:
\makeatletter
\@pointschangedfalse
\makeatother




\end{document}
%--------------------------------------------------------------------
%--------------------------------------------------------------------
%--------------------------------------------------------------------
