%%%%%%%%%%%%%%%%%%%%%%%%%%%%%%%%%%%%%%%%%%%%%%%%%%%%%%%%%%%%%%%%%
% Contents: Main Input File of the LaTeX2e Introduction
% $Id: lshort2e.tex,v 1.4 1998/03/26 15:36:54 oetiker Exp oetiker $
%%%%%%%%%%%%%%%%%%%%%%%%%%%%%%%%%%%%%%%%%%%%%%%%%%%%%%%%%%%%%%%%%
% lshort2e.tex - The not so short introduction to LaTeX2e
%                                                      by Tobias Oetiker
%                                                     oetiker@ee.ethz.ch
%
%                           based on LKURTZ.TEX Uni Graz & TU Wien, 1987
%-----------------------------------------------------------------------
%
% To compile lshort, you need TeX 3.x, LaTeX2e and makeindex
%
% The sources files of the Intro are:
%      lshort.tex (this file),
%      titel.tex, contrib.tex, biblio.tex
%      things.tes, typeset.tex, math.tex, lssym.tex, spec.tex,
%      lshort.sty, fancyheadings.sty
%
% Further the  verbatim.sty and the layout.sty
% from the LaTeX Tools distribution is
% required.
%
%
% To print the AMS symbols you need the AMS fonts and the packages
% amsfonts, eufrak and eucal from (AMS LaTeX 1.2)
%
% ---------------------------------------------------------------------


%****************** FIXME!s:
% �������� ������ �� ������

\ifx\pdfoutput\undefined % We're not running pdftex
\documentclass[dvips,11pt,a4paper,twoside]{scrbook}
\else
\documentclass[pdftex,11pt,a4paper,twoside]{book}
\usepackage{thumbpdf}
\pdfcompresslevel=9
\fi

\lccode`\-=`\-
\defaulthyphenchar=127
%\usepackage{mathtext}
\usepackage[T2A]{fontenc}
\usepackage[koi8-r]{inputenc}
\usepackage[russian,french,german,english]{babel}
%\usepackage{geometry}

%\usepackage{pslatex,pscyr}
\usepackage{lshort}
\usepackage{makeidx,latexsym}
\usepackage[russian]{mylayout}
% are we in pdftex ????
\ifx\pdfoutput\undefined % We're not running pdftex
\usepackage[plainpages=true]{hyperref}
\usepackage{color}
\else
\usepackage[colorlinks,hyperindex,plainpages=false,unicode]{hyperref}
%\usepackage{pslatex,freepsfonts}
\usepackage{keyval}
%\usepackage{hyppre}
%\usepackage{hypuc}
\hypersetup{pdfauthor={Boris Tobotras}}
\def\pdfBorderAttrs{/Border [0 0 0] } % No border arround Links
\makeatletter
\newcounter{chapter.H}
\setcounter{chapter.H}{0}
\let\Hmat@chapter\@chapter
\def\@chapter{\if@mainmatter\else\refstepcounter{chapter.H}\fi\Hmat@chapter}
\def\theHchapter{\if@mainmatter\arabic{chapter}\else H.\arabic{chapter.H}\fi}
\makeatother
\fi
\usepackage{footnpag}
% This document is ``public domain''. It may be printed and
% distributed free of charge in its original form (including the
% list of authors). If it is changed or if parts of it are used
% within another document, then the author list must include
% all the original authors AND that author (those authors) who
% has (have) made the changes.
%
% Original Copyright H.Partl, E.Schlegl, and I.Hyna (1987).
% English Version Copyright by Tobias Oetiker (1994,1995),
%
% ---------------------------------------------------------------------
%
%
% Formats also with\textt{letterpaper} option, but the pagebreaks might not
% fall as nicely.
%
% To produce a A5 booklet, use a tool like  pstops or dvitodvi
% to  past them together in the right order. Most dvi printer drivers
% can shrink the resulting output to fit on a A4 sheet.
%
\makeindex
\typeout{Copyright T.Oetiker, H.Partl, E.Schlegl, I.Hyna, B.Tobotras}

\begin{document}
\selectlanguage{russian}
\frontmatter
%%%%%%%%%%%%%%%%%%%%%%%%%%%%%%%%%%%%%%%%%%%%%%%%%%%%%%%%%%%%%%%%%
% Contents: The title page
% $Id: title.tex,v 1.2 2003/03/19 20:57:47 oetiker Exp $
%%%%%%%%%%%%%%%%%%%%%%%%%%%%%%%%%%%%%%%%%%%%%%%%%%%%%%%%%%%%%%%%%

\ifx\pdfoutput\undefined % We're not running pdftex
\else
\pdfbookmark{Title Page}{title}
\fi
\newlength{\centeroffset}
\setlength{\centeroffset}{-0.5\oddsidemargin}
\addtolength{\centeroffset}{0.5\evensidemargin}
%\addtolength{\textwidth}{-\centeroffset}
\thispagestyle{empty}
\vspace*{\stretch{1}}
\noindent\hspace*{\centeroffset}\makebox[0pt][l]{\begin{minipage}{\textwidth}
\flushright
{\Huge\bfseries �� ����� �������\\ 
�������� � \LaTeXe

}
\noindent\rule[-1ex]{\textwidth}{5pt}\\[2.5ex]
\hfill\emph{\Large ��� \LaTeXe{} �� \pageref{verylast} �����}
\end{minipage}}

\vspace{\stretch{1}}
\noindent\hspace*{\centeroffset}\makebox[0pt][l]{\begin{minipage}{\textwidth}
\flushright
{\bfseries 
Tobias Oetiker\\[1.5ex]
Hubert Partl, Irene Hyna �  Elisabeth Schlegl\\[3ex]} 
������~4.12, 13 April, 2003\\[3cm]
�������: �.~��������, 22 ��� 2003~�.
\end{minipage}}

%\addtolength{\textwidth}{\centeroffset}
\vspace{\stretch{2}}


\pagebreak
\begin{small} 

%  \selectlanguage{english}

  Copyright \copyright 1995-2002 Tobias Oetiker and all the Contributers to
  LShort.  All rights reserved.
 
  This document is free; you can redistribute it and/or modify it
  under the terms of the GNU General Public License as published by
  the Free Software Foundation; either version 2 of the License, or
  (at your option) any later version.
  
  This document is distributed in the hope that it will be useful, but
  WITHOUT ANY WARRANTY; without even the implied warranty of
  MERCHANTABILITY or FITNESS FOR A PARTICULAR PURPOSE\@.  See the GNU
  General Public License for more details.
  
  You should have received a copy of the GNU General Public License
  along with this document; if not, write to the Free Software
  Foundation, Inc., 675 Mass Ave, Cambridge, MA 02139, USA.

%  \selectlanguage{russian}

\end{small}


\endinput

%

% Local Variables:
% TeX-master: "lshort2e"
% mode: latex
% mode: flyspell
% End:

%%%%%%%%%%%%%%%%%%%%%%%%%%%%%%%%%%%%%%%%%%%%%%%%%%%%%%%%%%%%%%%%%
% Contents: Who contributed to this Document
% $Id: contrib.tex,v 1.1.1.1 2002/02/26 10:04:20 oetiker Exp $
%%%%%%%%%%%%%%%%%%%%%%%%%%%%%%%%%%%%%%%%%%%%%%%%%%%%%%%%%%%%%%%%%
\chapter{�������������}

\noindent ������ �� ����������� � ��� �������� ��������� ������� ��
������������ �������� � \LaTeX\ 2.09, ����������� �� ��������:

\begin{verse}
\contrib{Hubert Partl}{partl@mail.boku.ac.at}%
{Zentraler Informatikdienst der Universit\"at f\"ur Bodenkultur Wien}
\contrib{Irene Hyna}{Irene.Hyna@bmwf.ac.at}%
   {Bundesministerium f\"ur Wissenschaft und Forschung Wien}
\contrib{Elisabeth Schlegl}{no email}%
   {in Graz}
\end{verse}

���� �� ������������� �������� ��������� ���������, �� ������ �����
��� ������, ������� J\"org Knappen ������� ��� \LaTeXe{}, �� ������\\
\texttt{\CTAN|info/lshort/german|}.

\newpage
\noindent � �������� ���������� ����� ��������� � ������ � ��������� �
\texttt{comp.text.tex}. � ������� ��������� ��������. �������� �������
�������� ��� ������ �������������, ������������� � �����������. ���
��������� ������ ��� ������� �������� �� ��� ��������� ���������. �
����� �� �������� �� ���� �������������.  �����������, ��� ������,
������� �� ������� � ���� ������,~--- ���.  �������� �������� ����
��������� ���������� ����� ��������� ������� ����� ���������� ��������
�� ������ �� �����������������.


{\flushleft\small\sloppy
Rosemary~Bailey,        %r.a.bailey@qmw.ac.uk 0.2
Marc~Bevand,            % <bevand_m@epita.fr>
Friedemann~Brauer,      %fbrauer@is.dal.ca 3.4
Jan~Busa,               % <busaj@ccsun.tuke.sk>
Markus~Br\"uhwiler,\\     % <m.br@switzerland.org>
Pietro~Braione,         % <braione@elet.polimi.it>
David~Carlisle,         %GONE carlisle@cs.man.ac.uk 1.0
Jos\'e~Carlos~Santos,   % <jcsantos@fc.up.pt>
Mike~Chapman,           %chapman@eeh.ee.ethz.ch 3.16
Pierre~Chardaire,\\       % <pc@sys.uea.ac.uk
Christopher~Chin,       %chris.chin@rmit.edu.au 3.1
Carl~Cerecke,           %cdc@cosc.canterbury.ac.nz>
Chris~McCormack,        %GONE chrismc@eecs.umich.edu 0.1
Wim~van~Dam,            %GONE wimvdam@cs.kun.nl 2.2
Jan~Dittberner,\\         %jan@jan-dittberner.de 3.15
Michael~John~Downes,    %<mjd@ams.org> 14 Oct 1999
Matthias~Dreier,        %dreier@ostium.ch
David~Dureisseix,       %dureisse@lmt.ens-cachan.fr 1.1
Elliot,                 %GONE enh-a@minster.york.ac.uk 1.1
Hans~Ehrbar,            %ehrbar@econ.utah.edu
Daniel~Flipo,           %Daniel.Flipo@univ-lille1.fr
David~Frey,             %david@eos.lugs.ch 2.2
Hans~Fugal,             %hans@fugal.net
Robin~Fairbairns,       %Robin.Fairbairns@cl.cam.ac.uk 0.2 1.0
J\"org~Fischer,        %j.fischer@xpoint.at 3.16
Erik~Frisk,             %frisk@isy.liu.se 3.4
Mic~Milic~Frederickx,   % <mic.milic@web.de>
Frank,                  %frank@freezone.co.uk 11 Feb 2000
Kasper~B.~Graversen,    % <kbg@dkik.dk>
Arlo~Griffiths,         % <A.Griffiths@let.leidenuniv.nl>
Alexandre~Guimond,      %guimond@IRO.UMontreal.CA 0.9
Andy~Goth,              % <unununium@openverse.com>
Cyril~Goutte,           %goutte@ei.dtu.dk 2.1 2.2
Greg~Gamble,            %gregg@maths.uwa.edu.au 2.2
Neil~Hammond,           %nfh@dmu.ac.uk 0.3
Rasmus~Borup~Hansen,    %GONE rbhfamos@math.ku.dk 0.2 0.9 0.91 0.92 1.9.9
Joseph~Hilferty,        % <hilferty@fil.ub.es>
Bj\"orn Hvittfeldt,\\     %bjorn@hvittfeldt.com 3.13
Martien~Hulsen,         %M.A.Hulsen@WbMt.TUDelft.NL 1.0 1.1
Werner~Icking,          %<Werner.Icking@gmd.de> 3.1
Jakob,                  %diness@get2net.dk
Eric~Jacoboni,          %GONE jacoboni@enseeiht.fr 0.1 0.9
Alan~Jeffrey,           %alanje@cogs.sussex.ac.uk 0.2
Byron~Jones,            %bj@dmu.ac.uk 1.1
David~Jones,            %GONE djones@CA.McMaster.dcss.insight 1.1
Johannes-Maria~Kaltenbach, %<kaltenbach@zeiss.de> 3.01
Michael~Koundouros,     % <mkoundouros@hotmail.com>
Andrzej~Kawalec,        %GONE akawalec@prz.rzeszow.pl 1.9.9
Alain~Kessi,\\            %ALAIN.KESSI@HOTMAIL.COM 2.2
Christian~Kern,         %ck@unixen.hrz.uni-oldenburg.de 2.1
J\"org~Knappen,         %knappen@vkpmzd.kph.uni-mainz.de 0.1
Kjetil~Kjernsmo,        %<kjetil.kjernsmo@astro.uio.no> 3.2
Maik~Lehradt,           %greek@uni-paderborn.de 0.1
R\'emi~Letot,           % <r_letot@yahoo.com>
Johan~Lundberg,         %p99jlu@physto.se
Alexander~Mai,          %Alexander.Mai@physik.tu-darmstadt.de 3.8
Martin~Maechler,        %<maechler@stat.math.ethz.ch> 2.2
Aleksandar~S~Milosevic, % <aleksandar.milosevic@yale.edu>
Henrik~Mitsch,          % <Henrik.Mitsch@gmx.at>
Claus~Malten,\\           %GONE <ASI138%BITNET.DJUKFA11@BITNET.CEARN> 1.1
Kevin~Van~Maren,        % <vanmaren@fast.cs.utah.edu>  24 Nov 1999
Philipp~Nagele,         % Philipp.Nagele@t-systems.com
Lenimar~Nunes~de~Andrade, % <lenimar@mat.ufpb.br> Fri, 12 Nov 1999
Urs~Oswald,\\             % osurs@bluewin.ch
Demerson~Andre~Polli,   % polli@linux.ime.usp.br
Maksym~Polyakov         % <polyama@myrealbox.com>
Hubert~Partl,           %partl@mail.boku.ac.at 0.2 1.1
John~Refling,           %refling@sierra.lbl.gov 0.1 0.9
Mike~Ressler,\\           %ressler@cougar.jpl.nasa.gov 0.1 0.2 0.9 1.0 1.9.9
Brian~Ripley,           %ripley@stats.ox.ac.uk 2.1
Young~U.~Ryu,           %ryoung@utdallas.edu 2.1
Bernd~Rosenlecher,      %9rosenle@informatik.uni-hamburg.de 10 Feb 2000
Chris~Rowley,           %C.A.Rowley@open.ac.uk 0.91
Risto~Saarelma,\\         %risto.saarelma@cs.helsinki.fi
Hanspeter~Schmid,       %schmid@isi.ee.ethz.ch
Craig~Schlenter,        %cschle@lucy.ee.und.ac.za 0.1 0.2 0.9
Baron~Schwartz,         % <bps7j@cs.virginia.edu>      
Christopher~Sawtell,    %<csawtell@xtra.co.nz> 1 Sep 1999
Geoffrey~Swindale,      % <geofftswin@ntlworld.com>
Laszlo~Szathmary,       % <szathml@delfin.klte.hu>
Boris~Tobotras,         % <tobotras@jet.msk.su>
Josef~Tkadlec,      %tkadlec@math.feld.cvut.cz 2.0 2.2
Scott~Veirs,            %scottv@ocean.washington.edu
Didier~Verna,        %verna@inf.enst.fr 2.2
Fabian~Wernli,       %wernli@iap.fr 3.2
Carl-Gustav~Werner,     % <Carl-Gustav.Werner@math.lu.se> 11 Oct 1999,3.16
David~Woodhouse,        % <dwmw2@infradead.org> 3.16
Chris~York,             % <c.s.york@Cummins.com>  21 Nov 1999
Fritz~Zaucker,       %zaucker@ee.ethz.ch 3.0
Rick~Zaccone,               %zaccone@bucknell.edu 2.2
� Mikhail~Zotov.      %zotov@eas.npi.msu.su 3.1
}

\vspace*{\stretch{1}}



\pagebreak
\endinput
%%% Local Variables:
%%% mode: latex
%%% TeX-master: "lshort"
%%% End:
% Local Variables:
% mode: flyspell
% End:

%%%%%%%%%%%%%%%%%%%%%%%%%%%%%%%%%%%%%%%%%%%%%%%%%%%%%%%%%%%%%%%%%
% Contents: Who contributed to this Document
% $Id: overview.tex,v 1.1.1.1 2002/02/26 10:04:21 oetiker Exp $
%%%%%%%%%%%%%%%%%%%%%%%%%%%%%%%%%%%%%%%%%%%%%%%%%%%%%%%%%%%%%%%%%

% Because this introduction is the reader's first impression, I have
% edited very heavily to try to clarify and economize the language.
% I hope you do not mind! I always try to ask "is this word needed?"
% in my own writing but I don't want to impose my style on you... 
% but here I think it may be more important than the rest of the book.
% --baron

\chapter{�����������}

\LaTeX{}~\cite{manual}~--- ������� �������, ��������������� ��
������������ ������� �������������� ���������� �������� �������������
��������. ������� ����� ������ �������� ��� ������������ ������ �����
����������, �� ������� ����� �� ��������� ����������� ����.  \LaTeX{}
���������� \TeX{} \cite{texbook} � �������� ������ ��������� �������.

��� ������� �������� ��������� \LaTeXe{} � ������ ���� ���������� ���
����������� ���������� \LaTeX. ��� ������� �������� ������� \LaTeX{}
�������~\cite{manual,companion}.

\bigskip
\noindent ��� �������� ������� �� ���� ����:
\begin{description}
\item[����� 1] ������������ � ������� ��������� ���������� \LaTeX{}.
  �� �������� ��������� ������������� �� ������� \LaTeX{}. �����
  ������ ���� ����� �� ������ �������� ������ ������������� � p�����
  \LaTeX{}. 
\item[����� 2] ����������� � ������ ������� ����� ����������. ���
  ��������� ����������� ������ ������ � ��������� \LaTeX{}. �����
  ������ ���� ����� �� ������ ������ ������ ���������.
\item[����� 3] ���������, ��� �������� ������� � \LaTeX{}. ���������
  �������� ������� ��� ������, ��� ������������ ���, ���� �� ����� �������,
  ������� \LaTeX{}. � ����� ���� ����� �� ������� �������,
  ������������� ����������� ��������� � \LaTeX{} ��������������
  ��������.
\item[����� 4] ������������ ��� ��������� ����������� ��������� �
  ������������ � ��������� EPS �������. ��� ���������, ��� �p� ������
  pdf\LaTeX{} ��������� ��������� � ��p���� PDF, � ����� �p����������
  ��������� �p���� �������� ����������, �����, ��� ����� XY-pic.\sloppypar
\item[����� 5] ����������, ��� ������������ \LaTeX{} ��� ��������
�������. ������ ��������� ����������� � �����-������ ����������� ���������,
������ �� � ���� � ��������� ��� � \LaTeX{}, �� ���������� �����������, �
\LaTeX{} ������ ��.
\item[����� 6] �������� ������������ ������� �������� � ���, ���
  ������ ����������� ����� ���������. ��� ��������� ���, ���
  ������� �������� ����� \LaTeX{} ������� ��� ���p�������, �
  ����������� �� ����� ������������.
\end{description}

\bigskip
\noindent ����� ������ ����� ���������������. � ����� ������, ������ �� ����� ��
�������. �� ���� ������ ��� ���������� ������� ���������� ��������.
������� �� �����������, ��������� ������ � ���� ������ �����������
������ � ������ �p���p��.

\bigskip
\noindent \LaTeX{} �������� �� ����������� �����������, ������� � IBM PC ���
Mac, � ������ �������� ��������� UNIX ��� VMS. � ������
��������������� ����� ������� ��� ����������� � ������ � ������.
���������� � ���, ��� ������������ ��������� ��������� \LaTeX{},
������ ���� ������������� � \guide. ���� � ��� ����� �������� �
������� ������, ��������� � ������ ����, ��� ����������� ��� ���
������. ���� �� ����������� \emph{��} � ���, ����� ������� ���
��������� � ��������� ������� \LaTeX{}, � � ���, ����� �������, ���
������ ���� ��������� ���, ����� ��� ����� ���� ���������� \LaTeX{}.

\medskip
\noindent ���� ��� ����������� �����, ����������� � \LaTeX{},
��������, ������� ��� �� ����� �� ftp ������� Comprehensive \TeX{}
Archive Network (\texttt{CTAN}). ��� ��������
����c~---\texttt{http://www.ctan.org}. ��� ������ ����� ����� ��������
� ftp-������ \texttt{ftp://www.ctan.org} � ��� ������ �� ����� ����.
��������, ��� ��� ���~--- \texttt{ftp://ctan.tug.org}, ���
��������~--- \texttt{ftp://ftp.dante.de}, ��� ��������������~---
\texttt{ftp://ftp.tex.ac.uk}, ��� ������~---
\texttt{ftp://ftp.radio-msu.net}.  ���� �� �� � ����� �� ���� �����,
�������� ��������� � ��� �����.\sloppypar

����� � ����� �� ������ ��������� ��������� ������ �� CTAN, ������
����������� �� �������� ����������� ������ � ���������. ������ ������
������ ���������� ������� ����� �������� ������ \texttt{CTAN:} � �����
����� � ������ ������ CTAN.

\noindent ���� �� ������ ����� \LaTeX{} �� ����� ����������� ����������,
���������� �� ��, ��� �������� �� ������ \CTANref|systems|.

\vspace{\stretch{1}}
\noindent ���� � ��� ���� ����� �� ������ ����, ��� ��\'��� ��������,
������� ��� �������� � ���� ���������, ����������, ����� ��� �����. �
�������� ������������� � �������� �� �������� � \LaTeX{} �� ���� ����,
����� ����� �������� ����� ���������, � ��� ����� ��������� �����.

\medskip
\begin{verse}
\contrib{Tobias Oetiker}{oetiker@ee.ethz.ch}%
\noindent{Department of Information Technology and\\Electrical Engineering,
Swiss Federal Institute of Technology}
\end{verse}
\vspace{\stretch{1}}
\noindent ������� ������ ����� ��������� �������� �� ������\\
\CTANref|info/lshort|\trfootnote{������� ������ �������
�������� �������� �� ������ \texttt{http://xtalk.msk.su/tex}, � ����� ��
\texttt{CTAN}.}.  \endinput


%%% Local Variables:
%%% mode: latex
%%% TeX-master: "lshort2e"
%%% End:

\tableofcontents
\listoffigures
\listoftables
\mainmatter
%%%%%%%%%%%%%%%%%%%%%%%%%%%%%%%%%%%%%%%%%%%%%%%%%%%%%%%%%%%%%%%%%
% Contents: Things you need to know
% $Id: things.tex,v 1.1.1.1 2002/02/26 10:04:20 oetiker Exp $
%%%%%%%%%%%%%%%%%%%%%%%%%%%%%%%%%%%%%%%%%%%%%%%%%%%%%%%%%%%%%%%%%

\chapter{��� ����� �����}
\begin{intro}
  ������ ����� ���� ����� �������� ������� ����� ��������� � �������
  \LaTeX{}. ������ ����� ����� ������������ �� �������� ����������
  ���������� \LaTeX{}. ����� ������ ���� ����� �� ������ ����� �����
  ������������� � ���, ��� �������� \LaTeX{}. � ���������� ��� �������
  ��� ���������� ��� ����� ���������� � ������ �������.
\end{intro}


\section{��������}
\subsection{\TeX}

\sloppy{}\TeX{}~--- ��� ������������ ���������, ��������� ��������� ������
(\index{Knuth, Donald E.}Donald E.~Knuth)~\cite{texbook}. ���
������������� ��� ������� ������ � �������������� ������. ���� �����
������ \TeX{} � 1977 ���� ��� �������� ���������� ���������
����������� ������������, ������� ������ ���������� � ��� �����,
�������, � �����������, �������� ��������� ��������� �������������
��������, ������� �� ����� �� ������� ��� ����������� ���� � ������.
\TeX{}, � ��� ����, � ����� �� ��� ������� ����������, ��� ������� �
1982 ���� � ���������� ������������ � 1989 (������ ��������� 8-������
�������� � ��������� ������). \TeX{} �������� ����� ������������
�������������, ������� �� ��������� ����� ����������� � �����������
������ ����������� ������. ����� ������ \TeX{} �������� � $\pi$ �
������ ����� $3.14159$.

\TeX{} ������������ ��� <<���>>. � ����� \texttt{ASCII} \TeX{}
����� ������ ��� \texttt{TeX}.

\subsection{\LaTeX}

\LaTeX{}~--- ����������, ����������� ������� �������� � �������� ��
������ � ������� ������������ ���������, ��� ������ �������
������������, ���������������� �������. \LaTeX{} ��� �������
\index{Lamport, Leslie}{\lat Leslie Lamport}~\cite{manual}. � ��������
��������� ��� ������� �� ���������� \TeX{}. ������ \LaTeX{}
������������ \index{Mittelbach, Frank}Frank Mittelbach. 

%� 1994 ���� ����� \LaTeX{} ��� �������� ��������
%\index{LaTeX3@\LaTeX3}\LaTeX3 �� ����� � \index{Mittelbach,
%  Frank}Frank Mittelbach. � ��� ���� ������� ��������� �����
%����������� ���������, � ����� ���������� ��� �������� \LaTeX{},
%������������ � ������� ����� ��� ����� \index{LaTeX 2.09@\LaTeX{}
%  2.09}������ \LaTeX{}~2.09.  ����� �� ������ ��� ����� ������ ��
%������, ��� ���������� \index{LaTeX 2e@\LaTeXe}\LaTeXe. ���
%������������ ��������� ������ \LaTeXe{}.

\LaTeX{} ������������ ��� <<������>> ��� ��� <<�����>>. ���� ��
���������� �� \LaTeX{} � \texttt{ASCII} ���������, ������
\texttt{LaTeX}. \LaTeXe{} ������� ��� \texttt{LaTeX2e}.

%������� \ref{components} �� ��������~\pageref{components} ����������,
%��� �������� ������ \TeX{} � \LaTeXe{}. �� ���� �� \texttt{wots.tex}
%(����� Kees van der Laan).

%\begin{figure}[btp]
%\begin{lined}{0.8\textwidth}
%\begin{center}
%\input{kees.fig}
%\end{center}
%\end{lined}
%\caption{������������ \TeX{} �������}\label{components}
%\end{figure}

\section{������}

\subsection{�����, �������� � �����������}

��� ����, ����� ��������������, ������ ������ ���� �������� �
������������. ����� ���� �� ���������� ������������ ���������� �����
��������� (������ ��������, ������, ��������� ���� � ���� ����������
�~�.�.). �������� ���������� ���� ���������� � �������� � ������ ��
������������, ������� �������� ����� � ������������ � �����
������������.

��������--������� �������� ������, ��� ����� ���� � ����, ����� �����
���� ��������. �� ���������� ��������� ����, ������, �������,
������� � ������, ������ �� ������ ����������������� ����� � ��
���������� ��������.

� ����� \LaTeX, \LaTeX{} ����� �� ���� ���� ��������� �����, ���������
\TeX{} � �������� ������������. �� \LaTeX~--- ��� ����� ����
���������, �, �������������, ��������� � ����� ������ �����������.
����� ������ ������������ �������������� ����������, �����������
���������� ��������� ����� ������. ��� ���������� ������������ � �����
� ���� <<������ \LaTeX{}>>.

��� � ����� ���������� �� \wi{WYSIWYG}\footnote{What you see is what
  you get.} �������, ��������� � ����������� ����������� ���������
�����������, ����� ��� \emph{MS Word} ��� \emph{Corel WordPerfect}.  �
���� ����������� ������ ����������� �������� ������������ � ��������
������ ������ �� ����������. � �������� ������ ��� ����� ������ ��
������ ��� ����� ��������� �� ������, �����, � ����� ������, ��� �����
����������.

��� ������������� \LaTeX{} ������ ���������� ������� �������� ������� ��
����� ��������� ������. ��, ������, ����� ���������� �� ������ �����
��������� ����� \LaTeX. ����� ����� ������ ����������� �����
���������� �������.

\subsection{������ ������}

������������ ������~--- ��� ���������. ��������� ������ �����
��������� ��������� ������ ��������������, �����������, ��� ������
�����~--- ��� ������� ������ ������ ��������: <<���� �������� ��������
�������������, ������, �� ������ ����������>>. ��, ��� ��� ��������
������������ ��� ������, � �� ��� ����������� � ��������� �������,
�������� ��� ������ � ��������� ������� ����� �����, ������ �������.
��������:
\begin{itemize}
\item ������ ������ � ��������� ���������� ������ ���������� � ���,
  ����� ������� ��������� ���� � �������� ����� ��� ��������.
\item ������ ������ ���� ���������� ��������, ����� �� ��������� �����
  ��������, � ���������� ������� ��� ��������� ���������� ��������.
\end{itemize}

� \wi{WYSIWYG} ��������� ������ ����� ���������� ����������� ��������
��������� �� ����� ���������� ��� ������������� ����������. \LaTeX{}
������������� ����� ������ ��������������, ��������� ������ ���������
\emph{����������} ��������� ��� ���������. ����� ��� \LaTeX{} ��������
�������� ���������� ����� (���������) ���������.

\subsection{������������ � ����������}

����, ����� �����������, ����� ���� �� ���� \wi{WYSIWYG} ����������� �
�������������� \LaTeX{},~--- <<\wi{������������ \LaTeX} �����
����������� ���������� ������������>>, ��� ��������. ������, ��� ��
������ �������, ����� ���������� ����� ���������,~--- ��� ����������,
��� ��� ��� ����� ������� ��-��� ��������. ������, ������ �� �� ������
����������~\ldots


\medskip\noindent ��� ��� ��������� ������. �������� ������������
\LaTeX{} ����� �������� ���������� ������������:
\begin{itemize}

\item ������� ��������������� ����������� ������, �������� ���������
  ������������� ����������� <<��� ��������>>.
\item ������ ���������� ������� �������������� ������.
\item ������������ ����� ������� ���� ��������� �������� ������,
  �������� ���������� ��������� ���������. ��� ����������� ������� ��
  ����� �������� ���������� � ������� ���������.
\item ����� ��������������� ���� ������� ���������, ���� ����������,
  ����������, ������������ � ������.
\item ��� ������� ������ ������������ �����, �� ��������������
  �������� ������� \LaTeX{}, ���� �������� ����������������
  �������������� ������. ��������, ���������� ������ ��� ���������
  \PSi{}-������� ��� ��� ������� ������������ � ������
  ������������ � ����������� �����������. ������ �� ����
  �������������� ��������� ������� � \companion.
\item \LaTeX{} �������� ������� ������ ������ �����������������
  ���������, ��� ��� ������ ��� \LaTeX{} � ��������~--- �����
  ������������ ���������.
\item \TeX{}, ������������� ������ \LaTeXe{}, ����������� ������� �
  �������� ��������. ������� ������� �������� ����������� �� ����
  ������������ ����������.

%
% Add examples ...
%
\end{itemize}

\medskip

\noindent\LaTeX{} ����� ����� � ��������� ����������, ��, �������, ���
������ ����� ����� ��� ��������, ����, � ������, ������ ��� ������ ��
����� \texttt{;-)}

\begin{itemize}
% ��� �� �����?
% \item \LaTeX{} ����� �������� � ��������� ���� ����~\ldots
\item ���� ���������������� ������ ����� ��������� �������������
  ����������, �������� ��������� ������ ������ ��������� �� �����
  ������ � �������� ����� �������.\footnote{�������, ��� ���~--- ����
    �� �������� ����� ������� ������� \LaTeX3.}\index{LaTeX3@\LaTeX3}
\item ����� ������ ������ ������������������� � ����������������
  ���������.
\item ���� ������� ������ ����� ��� �� ����� � �� ������ ���������
  ���������� ��������, �������� �� ������� ������ ������.
\end{itemize}

\section{�������� ����� \LaTeX{}}

��������� ������� ��� \LaTeX{} �������� ������� ��������� ���� �
\texttt{ASCII}. ��� ����� ������� � ����� ��������� ���������. ��
�������� ����� ��������� ������ � ���������, ������������ \LaTeX{},
��� �������� �����.

\subsection{�������}

<<������>> �������, �����, ��� ������ ��� ���������, ����������
\LaTeX{} ���������, ��� <<\wi{������}>>. \emph{���������
  ����������������} ������ ��������\index{������ �������} ����������
��� \emph{����} <<������>>. ������ ������� � ������ ������ ������
������������, � ��������� ������� ������ �������������� ���
<<������>>.  \index{������!� ������ ������}

������ ������ ����� ���� ����� ������ ���������� �����
������. \emph{���������} ������ ����� ���������� ��� ��, ���
\emph{����} ������ ������. ���� �������� ������. ������~--- ����� ��
�������� �����, �����~--- ��������������� �����.

\begin{example}
�������, ���������� �� ��
����   ���    ���������
�������� ����� �������.

������ ������� ��������
����� �����.
\end{example}

\subsection{�����������}

��������� ������� �������� \index{�������!�����������������}
������������������ ���������, ������� ���� ����� � \LaTeX{}
����������� ��������, ���� ������� �� �� ���� �������. ���� �� �������
�� � ����� ��������, �� ��� ������ �� ������������, � ��������
\LaTeX{} ������� ���--������, ���� ����� �� ���������������.

\begin{code}
%\verb.$ & % # _ { }  ~  ^  \ . %$
\verb.#  $  %  ^  &  _  {  }  ~  \ . %$
\end{code}

��� �� ����� �������, ��� ������� ����� ������������ � �����
����������, �������� � ��� ������� <<\verb|\|>>:

\begin{example}
\# \$ \% \^{} \& \_ \{ \} \~{}
\end{example}

������ �������, ��� � ������, ������ ������, ����� �������
������������ ��������� � �������������� �������� ��� ��� �������. ����~<<\verb|\|>> \emph{������} �������, �������� ����� ��� ��� ����, ���
��� ��� ������� (\verb|\\|) ������������ ��� �������
������.\footnote{������ ����� ����������� ��������
  \texttt{\$}\ci{backslash}\texttt{\$}. ��� ���� `$\backslash$'.}

\subsection{������� \LaTeX{}}

�������\index{�������} \LaTeX{} ������������� � �������� � ���������
���� �� ��������� ���� ����:

\begin{itemize}
\item ��� ���������� � ������� \wi{backslash} <<\verb|\|>> �
  ������������ ������, ��������� ������ �� ����. ����� ������
  ����������� ��������, ������ ��� ����� ������ <<��-������>>.
\item ��� ������� �� <<\verb|\|>> � ����� ������ ������������ �������.
\end{itemize}

%
% \\* doesn't comply !
%

%
% Can \3 be a valid command ? (jacoboni)
%
\label{whitespace}


\LaTeX{} ���������� ������� ����� ������. ���� �� ������ ��������
\index{������!����� �������}������ ����� �������, �� ������ ���������
��� <<\verb|{}|>> � ������, ��� ����������� ������� ������� �����
����� �������. <<\verb|{}|>> �� ���� \LaTeX{} ������������ ��� �������
����� ����� �������.

\begin{example}
� ������, ��� ���� ���������
�����, ���������� � \TeX{}
�� \TeX{}����� � \TeX ������.\\
�������~--- \today
\end{example}

��������� ������� ��������� � \index{��������}���������, �������
������ ���� ����� ����� \index{�������� ������}���������
��������~<<\verb|{ }|>> ����� ����� �������.  ��������� �������
������������ \wi{�������������� ���������}, ������� ����������� �����
����� ������� � \index{���������� ������}����������
�������~<<\verb|[ ]|>>.  ��������� ������ ���������� ���������
������� \LaTeX. �� ������������� ��� ����, ��� ����� ���������� �����.

\begin{example}
�� ������ \textsl{����������}
�� ����!
\end{example}

\begin{example}
����������, ������� �����
������� ����� ���!\newline
�������!
\end{example}

\subsection{�����������}
\index{�����������}

����� � �������� ��������� �������� ����� \LaTeX{} ��������� ������
\verb|%|, �� ���������� ������� ������� ������, ������� ������� � ���
������� � ������ ��������� ������.

���� ����� ������������ ��� ���������� � �������� ���� ���������,
������� �� ����� ���������� �� ������.

\begin{example}
��� Spercal%
         ifragilist%
     icexpialidocious
\end{example}

������ \texttt{\%} ����� ����� ������������, ����� ������� ������� ������� �
��� ������, ��� �� ����������� ������� ��� �������� �����.

\sloppy{}��� ����� ������� ������������ ����� ����� ������������ ����������
\ei{comment}, ��������������� ������� \pai{verbatim}. ��� ��������,
���, ��� ������������� ��������� \ei{comment}, �� ������ � ���������
������ ��������� �������� ������� \verb|\usepackage{verbatim}|:

\begin{example}
���~--- ��� ����
\begin{comment}
�������� ������,
�� ��������
\end{comment}
������ ������� ������������
� ��� ��������.
\end{example}

��������, ��� ��� �� ����� �������� ������ ������� ���������,
��������, ����������.


\section{��������� �������� �����}

����� \LaTeXe{} ������������ ������� ����, �� ������� �� ����
���������� ������������ \index{��������� �����}���������. ���, ������
������� ���� ������ ���������� � �������
\begin{code}
\verb|\documentclass{...}|
\end{code}
��� ���������, �������� ������ ���� �� ����������� ������. �����
�����, �� ������ �������� �������, �������� �� ����� ��������� �
�����, ��� ��������� \index{�����}������, ����������� �����
����������� � ������� \LaTeX. ��� �������� ������ ������ ������������
�������
\begin{code}
\verb|\usepackage{...}|
\end{code}

����� ��� ��������� ���������,\index{���������}\footnote{������� �����
  \texttt{\bs documentclass} � \texttt{\bs
    begin$\mathtt{\{}$document$\mathtt{\}}$} ����������
  \emph{����������}.} �� ��������� ���� ������ ��������

\begin{code}
\verb|\begin{document}|
\end{code}

������ �� ������� ����� � ��������� \LaTeX. � ����� ��������� ��
���������� �������
\begin{code}
\verb|\end{document}|
\end{code}
���, ��� ������� ����� ���, \LaTeX{} ����������.

���.~\ref{mini} ���������� ���������� ������������ ����� ��� \LaTeXe.
��������� ����� ������� \wi{������� ����} ��� ��
���.~\ref{document}.\trfootnote{� ������ ������ �������� �����
����������� ����� ������, ����������� ��� ��������� ��������
�����. ����������� ��������� ��������� ������� ����������� \LaTeX,
����� ������������ ����� \pai{babel}, ���������� ����������� ���������
����������� \TeX{}. ���� �� ��������� �� ����� ������ �������,
��������, ��� ����� ����� ������������ ������� ��������������
����������.  ������������������� � ����� \guide{} ��� �
���������������.}

\begin{figure}[!bp]
\begin{lined}{6cm}
\begin{Verbatim}
\documentclass{article}
\usepackage[koi8-r]{inputenc}
\usepackage[russian]{babel}
\begin{document}
���������~--- ������ �������.
\end{document}
\end{Verbatim}
\end{lined}
\caption{����������� ���� \LaTeX{}} \label{mini}
\end{figure}

\begin{figure}[!bp]
\begin{lined}{10cm}
\begin{Verbatim}
\documentclass[a4paper,11pt]{article}
\usepackage[russian]{babel}
\begin{document}
% ���������� �����
\author{�.~��������}
\title{����������}
\frenchspacing
\begin{document}
% ���������� �����
\maketitle
% ��������� ����������
\tableofcontents
\section{������}
��� ��� � ���������� ��� ������������� ������.
\section{�����}
\ldots{} � ��� ��� ���������.
\end{document}
\end{Verbatim}
\end{lined}
\caption{������ ������������ ���������� ������} \label{document}
\end{figure}

\section{�������� ������ ������ � \LaTeX{}}

����� ��������, ��� ��� ��� �� �������� ����������� ��������� ������
\LaTeX{}-�����, ����������� �� ��������~\pageref{mini}. �������
���������. ��� �� ����, \LaTeX{} �� �������� ������������
����������. ��~--- ������ ���������, �������������� ��������� �������
����. ��������� ������������ \LaTeX{} �������� ����������� ��������,
��� �� ������ ������� ������ ���������� ���������� �����. � ������ ��������
��� ����� �������� ������� � ���������� ������.��������������, ��� ��
����� ������ \LaTeX{} ���������� � ��������\footnote{��� ������ ��� �
  ����� ����������� Unix-�������.}.

\begin{enumerate}
\item 

  �������������� ��� �������� ������� ���� \LaTeX{}. ���� ���� ������
  ���� ������� ASCII �������. � Unix ����� �������� ������� ��� ������
  ���. � Windows �� ������ ���������, ��� �� ���������� ���� � �������
  ASCII, ��� \emph{Plain text}. ������� ��� ��� �����, ��������
  ���������� \eei{.tex}.

\item 

  ����������� ��� ������� ���� \LaTeX{}. ���� ��� ������� ��� ������,
  �� �������� ���� \texttt{.dvi}. ����� �������� ���������� ����������
  � ������������ ������, ��� ����� ������������ ��������� \LaTeX{}
  ��������� ���. ���� �� ������� ����� ���� ������, \LaTeX{} ������
  ��� �� ��� � ��������� ��������� �������� �����. ����� ��������� �
  ��������� ������, ������� \texttt{ctrl-D}.

\begin{lscommand}
\verb+latex foo.tex+
\end{lscommand}

\item 
  ������ �� ������ ����������� ���� \texttt{.dvi}. ��� ����� �������
  ��-�������. ����� ���������� ��� �� ������ ��������
\begin{lscommand}
\verb+xdvi foo.dvi &+
\end{lscommand}
��� �������� ������ � ����� Unix � X11. ���� ��  ��������� ���
  Windows, ���������� \texttt{yap} (``yet another previewer'').

  �� ���� ������ ������������� ���� DVI � \PSi{} ��� ������ ���
  ��������� ��� ������ Ghostscript:

\begin{lscommand}
\verb+dvips -Pcmz foo.dvi -o foo.ps+
\end{lscommand}

���� ��� �������, �� � ������� ����� ������� ����� ���������
\texttt{dvipdf}, ������� ����� ������������� ���� \texttt{.dvi}-�����
����� � PDF.
\begin{lscommand}
\verb+dvipdf foo.dvi+
\end{lscommand}


\end{enumerate}


\section{������ ����������}

\subsection{������ ����������}\label{sec:documentclass}

������, ��� \LaTeX{} ������ ����� ��� ��������� �������� �����, ���
��� ������������ ������� ���������. �� �������� ��������
\ci{documentclass}.
\begin{lscommand}
\ci{documentclass}\verb|[|\emph{�����}\verb|]{|\emph{�����}\verb|}|
\end{lscommand}
\noindent ����� \emph{�����} ���������� ��� ������������ ���������.
�������~\ref{documentclasses} ����������� ������ ����������,
��������������� � ���� ��������. � ������ \LaTeXe{} ������
�������������� ������ ��� ������ ����������, ������� ������ � ������.
�������� \emph{\wi{�����}} �������� ��������� ������ ���������. �����
������ ����������� ��������. � �������~\ref{options} ����������� �����
��������������� ����� ����������� ������� ����������.

\begin{table}[!bp]
\caption{������ ����������} \label{documentclasses}
\begin{lined}{12cm}
\begin{description}

\item [\normalfont\texttt{article}] ��� ������ � ������� ��������,
  �����������, �������� �������, ����������� ������������, �����������\ldots
  \index{�����!article}
\item [\normalfont\texttt{report}] ��� ����� ������� �������,
  ���������� ��������� ����, ��������� ������, �����������\ldots
  \index{�����!report}
\item [\normalfont\texttt{book}] ��� ��������� ���� \index{�����!book}
\item [\normalfont\texttt{slides}] ��� �������. ���������� �������
  ����� ��� �������. ������ ����� ����� ������������
  Foil\TeX.\footnote{\CTANref|macros/latex/contrib/supported/foiltex|}
  \index{�����!slides}\index{foiltex}
\end{description}
\end{lined}
\end{table}

\begin{table}[!bp]
\caption{����� ������� ����������} \label{options}
\begin{lined}{12cm}
\begin{flushleft}
\begin{description}
\item[\normalfont\texttt{10pt}, \texttt{11pt}, \texttt{12pt}] \quad
  ������������� ������ ��������� ������ ���������. ���� �� ���� ��
  ���� ����� �� �������, ��������������� \texttt{10pt}.
  \index{�����!���������, ������}\index{������ ��������� ������}
\item[\normalfont\texttt{a4paper}, \texttt{letterpaper}\ldots] \quad
  ���������� ������ �����. �� ��������� ���������������
  \texttt{letterpaper}. ��� �� ����� ���� ������� \texttt{a5paper},
  \texttt{b5paper}, \texttt{executivepaper} � \texttt{legalpaper}.
  \index{����!legal}\index{������ �����}\index{����!A4}\index{����!letter}
  \index{����!A5}\index{����!B5}\index{����!executive}

\item[\normalfont\texttt{fleqn}] \quad ��������� ������� �����
  ��������� �����, � �� ��������������.

\item[\normalfont\texttt{leqno}] \quad ������� ���������� �����, � ��
  ������.

\item[\normalfont\texttt{titlepage}, \texttt{notitlepage}] \quad
  ���������, ������ ���������� ����� �������� ����� ���������
  ���������\index{��������� ���������} ��� ���. �� ��������� �����
  \texttt{article} �� �������� ����� ��������, � \texttt{report} �
  \texttt{book}~--- ��������. \index{����!���������}

\item[\normalfont\texttt{onecolumn}, \texttt{twocolumn}] \quad ���������� \LaTeX{}
  �������� �������� � \wi{���� �������} ��� � \wi{��� �������}.

\item[\normalfont\texttt{twoside, oneside}] \quad �������� ����- ���
  ������������ �����. �� ��������� ������ \texttt{article} �
  \texttt{report} ���������� \wi{������������� �����}, �����
  \texttt{book}~--- \wi{������������ �����}. ��������, ��� �����
  \texttt{twoside} \emph{��} ���������� ��� �������  �� ����� ����
  �������� � ���� ������.

\item[\normalfont\texttt{landscape}] \quad ������ ��������� �������� ��
  �����������. 

\item[\normalfont\texttt{openright, openany}] \quad ������ �����
  ������������� ��� ������ �� ������ ��������, ��� �� ������
  ���������. ��� �� �������� � ������� \texttt{article}, ��� ��� ��
  ������ �� ����� � ������. ����� \texttt{report} �� ���������
  �������� ����� �� ��������� ��������, � ����� \texttt{book}~--- ��
  ������.
\end{description}
\end{flushleft}
\end{lined}
\end{table}

������: ������� ���� ��� ��������� \LaTeX{} ����� ���������� �������
\begin{code}
\ci{documentclass}\verb|[11pt,twoside,a4paper]{article}|
\end{code}
��� ���������� \LaTeX{} �������� �������� ��� \emph{������}, � �������
�������� ������ \emph{����������� �������} � ������������� ��������
��� \emph{������������} ������ �� ������ \emph{������� A4}.

\subsection{������}
\index{�����} � �������� ��������� ������ ���������, ��, ��������,
����������, ��� � ��������� �������� ������� \LaTeX{} �� ������ ������
���� ��������. ���� �� �������� �������� � ��������
�������\index{�������}, \wi{������� �����} ��� �������� ��� ���������
�� �������� �����, ��� ����� ����� ��������� ����������� \LaTeX{}.
����� ���������� ���������� ��������. ������ �������������� ��������
\begin{lscommand}
\ci{usepackage}\verb|[|\emph{�����}\verb|]{|\emph{�����}\verb|}|
\end{lscommand}
\noindent ��� \emph{�����}~--- ��� ��� ������, � \emph{�����}~---
������ �������� ����, ���������� ����������� �������� ������.
��������� ������ �������� � �������� �������� \LaTeXe{} (��.
�������~\ref{packages}). ������ ��������������� ��������.
�������������� ���������� �� ������������� � ��� ������� �����
����������� � \guide{}. �������� �������� ���������� � �������
\LaTeX{}~--- ��� \companion{}. ��� �������� �������� ����� �������
������ � ����������� � ���, ��� ������ ���� ����������� ���������� ���
\LaTeXe{}.


\begin{table}[btp]
\caption{��������� �� ���������������� � \LaTeX{} �������} \label{packages}
\begin{lined}{11cm}
\begin{description}
\item[\normalfont\pai{doc}] ��������� ��������������� ��������� �� \LaTeX{}.\\
  ������ � \texttt{doc.dtx}\footnote{���� ���� ������ ���� ����������
    �� ����� �������, � �� ������ �������� \texttt{dvi} ����,
    ��������� \texttt{latex doc.dtx} � ����� ��������, ��� �� ������
    ����� �� ������. �� �� ����� ��������� �� ���� ������ ������,
    ���������� � ���� �������.} � � \companion.

\item[\normalfont\pai{exscale}] ������������� ���������������� ������
  ����������� �������������� �������.\\
  ������ � \texttt{ltexscale.dtx}.

\item[\normalfont\pai{fontenc}] ���������, ����� ���������
  ������\index{��������� ������}
  ������ ������������   \LaTeX{}.\\
  ������ � \texttt{ltoutenc.dtx}.

\item[\normalfont\pai{ifthen}] ������������� ������� ���� `����
  \ldots, �� ��������� \ldots, ����� ��������� \ldots'.\\ ������ �
  \texttt{ifthen.dtx} � � \companion.

\item[\normalfont\pai{latexsym}] ����� ���������� ����� �����������
  �������� \LaTeX{}, ����� ������������ ����� \texttt{latexsym}.
  ������ � \texttt{latexsym.dtx} � � \companion.

\item[\normalfont\pai{makeidx}] ������������� ������� ��� ���������
  ����������. ������ � �������~\ref{sec:indexing} � � \companion.

\item[\normalfont\pai{syntonly}] ������������ ��������, �� �������
  ���.

\item[\normalfont\pai{inputenc}] ��������� ������� ������� ���������,
  ����� ��� ASCII, ISO Latin-1, ISO Latin-2, 437/850 IBM
  code pages,  Apple Macintosh, Next, ANSI-Windows ��� ������������
  �������������. ������ � \texttt{inputenc.dtx}.
\end{description}
\end{lined}
\end{table}

\clearpage

\subsection{����� ��������}

\LaTeX{} ������������ ��� ���������������� ���������� ��������
����������� � �������
�����������\index{����������!�������}\index{����������!������}~--- ���
���������� \wi{����� ��������}. �������� \emph{�����} �������
\index{����� ��������!plain@\texttt{plain}}\index{plain@\texttt{plain}}%
\index{����� ��������!headings@\texttt{headings}}\index{headings@\texttt{headings}}%
\index{����� ��������!empty@\texttt{empty}}\index{empty@\texttt{empty}}%
\begin{lscommand}
\ci{pagestyle}\verb|{|\emph{�����}\verb|}|
\end{lscommand}
\noindent ����������, ����� �� ��� ������������.
���������������� ����� �������� ����������� � �������~\ref{pagestyle}.

\begin{table}[!hbp]
\caption{���������������� ����� �������� \LaTeX} \label{pagestyle}
\begin{lined}{12cm}
\begin{description}

\item[\normalfont\texttt{plain}] �������� ������ ������� �����
  �������� � �������� ������� �����������. ���� ����� ���������� ��
  ���������.

\item[\normalfont\texttt{headings}] �������� �������� ������� ����� �
  ����� �������� � ������� ����������� ������ ��������, � ������
  ���������� �������� ������. (���� ����� ����������� � ������
  ���������.)

\item[\normalfont\texttt{empty}] ������ � �������, � ������ �����������
  �������.

\end{description}
\end{lined}
\end{table}

�������� ������� ����� ������� �������� ��������
\begin{lscommand}
\ci{thispagestyle}\verb|{|\emph{�����}\verb|}|
\end{lscommand}
�������� ����, ��� ��������� ���� ����������� ����� ������������,
�������� � \companion{}, � ����� � �������~\ref{sec:fancy} ��
��������~\pageref{sec:fancy}.
%
% Pointer to the Fancy headings Package description !
%

\section{������������� ���� ������}

������� � \LaTeX{}, �� ������ ������� �������� � ���� ������ �
���������� ������������\index{����������}. ���� ����������� ���������
\wi{���� ������}, ������������ ��� �������� � \TeX{}. ��������, ���
��� �� ������ ������ ����������, ��, ���� �� ������� �� ����������
����������, ������� �������� ������,~--- ���������, ����������,
������. 

%% \begin{description}
%% \item[\eei{.tex}]  ������� ���� \LaTeX{} ��� \TeX{}. �������������
%%   �������� \texttt{latex}.
%% \item[\eei{.sty}] ���������� \LaTeX{}. ���� ���� ����������� �
%%   �������� �������� \ci{usepackage}.
%% \item[\eei{.dtx}] ����������������� \TeX{}. ���~--- �������� ������
%%   ��������������� �������� ������ \LaTeX{}. ���� �� ����������� ����
%%   \texttt{.dtx}, �� �������� ����������������� �������� ������,
%%   ������������� � ���� �����.
%% \item[\eei{.ins}] ����������� ��� ������, ������������ �
%%   ��������������� ����� \texttt{.dtx}. ����� ��������� ������ ��
%%   ���������, �� ������ ��������� ����� \texttt{.dtx} �
%%   \texttt{.ins}. ��������� \LaTeX{} � ������ \texttt{.ins}, �����
%%   ����������� ���� \texttt{.dtx}.
%% \item{\eei{.fd}] ���� �������� ������, ����������� ����� ��� \LaTeX{}.
%% \item[\eei{.cls}] ���� ������, ������������ ��� ������
%%   ���������. ���������� �������� \ci{documentclass}.
%% \end{description}

��������� ����� ������������, ����� \LaTeX{} ������������ �������
����:

\begin{description}
\item[\eei{.dvi}] Device Independent file (����, �� ��������� ��
  ����������). ���~--- �������� ��������� ������� \LaTeX{}. ����������
  ��� ����� ������� ��� ������ ��������� ����������� DVI, ���
  ����������� ���������� \texttt{dvips} ��� �����������.
\item[\eei{.log}] �������� ��������� ����� � ���, ��� ����������� �
  ��������� ������ ����������.
\item[\eei{.toc}] ������ ��������� ���� ��������. �������� � ���������
  ������ ���������� � ������������ ��� ��������� ����������.
\item[\eei{.lof}] ������ \texttt{.toc} ��� ������ �����������.
\item[\eei{.lot}] �� ��, ��� ������ ������.
\item[\eei{.aux}] ��� ���� ����, ���������� ���������� ����� ���������
  ����������. ����� ��� �������, ������������ ��� ���������
  ������������ ������.
\item[\eei{.idx}] ���� ��� �������� �������� ���������� ���������,
  \LaTeX{} �������� ��� ����� ��� ��������� � ���� ����. �����������
  ��� ���������� \texttt{makeindex}. ��������� �������� �
  �������~\ref{sec:indexing} �� ��������~\pageref{sec:indexing}.
\item[\eei{.ind}] ������������ ���� \texttt{.idx}, ������� ��
  ��������� � ��� �������� ��� ��������� ������� ����������.
\item[\eei{.ilg}] ������ ������ \texttt{makeindex}.
\end{description}


% Package Info pointer
%
%




%
% Add Info on page-numbering, ...
% \pagenumbering

\section{������� �������}
��� ������ � �������� ����������� ��� ����� ���� ������ ���������
������� ���� �� ��������� ������. \LaTeX{} �������� ��� �������,
������� �������� ��� ������.

\begin{lscommand}
\ci{include}\verb|{|\emph{����}\verb|}|
\end{lscommand}
\noindent ��� ������� ����� ������������ � ���� ���������, �����
�������� � ���� ���������� ������� �����. ��������, ��� \LaTeX{}
������ ����� ��������, ������, ��� ������������ ��������, ������������
� \emph{�����}.

������ ������� ����� �������������� ������ � ���������. ��� ���������
 \LaTeX{} ������ ������ ��������� �� ���������� (\verb|\include|)
 ������.

\begin{lscommand}
\ci{includeonly}\verb|{|\emph{filename}\verb|,|\emph{filename}%
\verb|,|\ldots\verb|}|
\end{lscommand}
����� ���������� � ��������� ��������� ���� ������� ����� �����������
������ �� ������� \ci{include}, ��� ����� ������ ����������� �
��������� ������� \ci{includeonly}.

������� \ci{include} �������� ������� ����������� ������ �� �����
��������. ��� ������ ��� ������������� \ci{includeonly}, ������ ���
������� ������� �� ����� ���������, ���� ����� ��������� ����������
����� ������������. ������ ��� ��������, � � ���� ������ �� ������
������������ �������

\begin{lscommand}
\ci{input}\verb|{|\emph{filename}\verb|}|
\end{lscommand}
\noindent ��� ������ �������� ���������� ���������� �����.

��� ������� �������� ��������� ����������� ����� \pai{syntonly}. ��
���������� \LaTeX{} ��������� �� ���������, �������� ��������� �
������������� ������, �� �� ��������� ��������� ����� DVI. \LaTeX{} �
���� ������ �������� �������, ������� ���� �����. ��������� �����
�����: 

\begin{verbatim}
\usepackage{syntonly}
\syntaxonly
\end{verbatim}

����� �� �������� �������� �������� ����, ������ ���������������
������ �������, ������� ���� ��������.

%%% Local Variables:
%%% mode: latex
%%% TeX-master: "lshort"
%%% End:

%%%%%%%%%%%%%%%%%%%%%%%%%%%%%%%%%%%%%%%%%%%%%%%%%%%%%%%%%%%%%%%%%
% Contents: Typesetting Part of LaTeX2e Introduction
% $Id: typeset.tex,v 1.3 1998/03/26 15:36:54 oetiker Exp oetiker $
%%%%%%%%%%%%%%%%%%%%%%%%%%%%%%%%%%%%%%%%%%%%%%%%%%%%%%%%%%%%%%%%%
\chapter{������� ������}

\begin{intro}
����� ��������� ���������� ����� �� ������ ������������, �� ����
������� �������� \LaTeX{}. ������ ����� ��������� ��� ��������� ����
����������� ��� ������������ �������� ����������.
\end{intro}

\section{��������� ������ � �����}
\secby{Hanspeter Schmid}{hanspi@schmid-werren.ch}

�������� ���� ��������� ������~--- �������� �������� ����, ����������
��� ������. �������� ����� ������ �����, ���� ��� ����
���������������, � ������� ����� ������ � ����������� ��� ���������,
���� ������������ ����� �������� ���������� � ��������� ���������
����������.

\LaTeX{} ���������� �� ������ ������ ������� � ���, ��� ��� ����� ����
�������� ��� ���������� � ��������� ��������� ������. �� �����
�������� ������������ ����� ������ � ������������ � <<���������>>,
��������� � ����� ������ ��������� � � ��������� �������� ������.

����� ������ ������� ������ � \LaTeX{} (� � ������ ������)~---
�����. �� ���������� <<��������� ������>>, ������ ��� ����� ��������
��� ������������ ������, � ������� ���������� ���� ������� ����� ���
����. � ��������� �������� �� �������, ��� �� ������ ������� �������
������, ��������, ��� ������ \texttt{\bs\bs} � ������ ������,
��������, ������� ������ ������. �������, ���� ���������� ����� �����,
������ ���������� ����� �����, �, ���� ���,~--- ����������� ��������
������. ���� �� ������������ � ����� ��������� �� ������, ��������� �
����� ������ ��� � �������� ���� � ������. ���� �� ������ ����� �����,
��������� ��������� ��� �� �����,~--- �������� ������ ������
������. ���� �� ���������� ������ ���������� ���������� �����
�����~--- ���������� ��� �� ���.

����������� ����� ���������� �� �������� �������� ��������� ���������
������ �� ������. ������ �� ������������ � �������� ������� �������
���, �������� � \LaTeX{}, �������� ������, �� ���� �����. ���������
������ �������� ����� �������, ����� � ����� ����������
���������. ���������� �� ��������� ������� � �����������, ������
������ � ��� ������������ ������ ������� (������� ����������), �
������~--- ���. (���� �� ��� �� ��������� ���� ������ ����������,
����� ������ �������, ��������, ����������, ��� � ��������� ����� �
����� ��������� � �������� �����.)

% Example 1
\begin{example}
\ldots ����� ��������
���� ���� �������
\begin{equation}
  e = m \cdot c^2 \; ,
\end{equation}
������� �������� �����
������ ��������� � �����
���� �������� ����������
��������.
\end{example}

% Example 2
\begin{example}
\ldots ������ �������
����� ���� ��������:
\begin{equation}
  \sum_{k=1}^{n} I_k = 0 \; .
\end{equation}

����� ���������� ��������
����� ���� ������� \ldots
\end{example}

% Example 3
\begin{example}
\ldots ������� �����
��������� �����������.

\begin{equation}
  I_D = I_F - I_R
\end{equation}
�������� ����� ����������
������ ������ �����������.
\ldots
\end{example}

���������, ����� ���������, ������� ������~--- �����������. �
���������� ������ ����� �����, ����������� �����������, ��������
�\'������ ������, ��� ����� �����, ������� ����� ����������. \LaTeX{}
��������� ����������, ����� �� ��� �� ����� � ����. ���� �� ���������,
�� ������ ��� ����������. ��� ��������� ����� � ���� �����.

��������� ������ ���������� ���� ��������� �����
�����������. ����������� ������ ���������� ����� ������� �������
����������, �� �� ������ ������ (������� �������� � ����������) ��
���������� ����� ��� ������� �� �����, ������ �����, ��� ��� ��������:
�������� ��������� � ������ ���������. ���� �� �� ������� � �����
�������, �������� ����������� �����, �������������� �� ������
�������. ���� ���-�� ��� ��������� �������, ������� ��� �������, ����
�� ������������ ������������� ��������� � ������ �����, ��������
�������.\trfootnote{��� ��� ���� ��������� ��� ��������, �� � ����� ��
  ����������� ���� ������� �� ��������, ����� ��� ����-������ ������� ������
  ��-���������.}

�������, ������ ������ ������ ���� ����� ������� ��������������� ��
����� ������� ������, ����������� � �����, �������, � ���
�����. ������, ������������ ������ ������, ��������,
\verb|\section{|\texttt{��������� ������ � �����}\verb|}|
������ ���������� ��������, ����� ���� ����� ����, ��� ������������
��� ��������������� ���������.

\section{��������� �� ������ � ��������}

\subsection{����������� ������}

����� ����� ���������� ���, ����� ��� ������� ����� ���������� �����.
\LaTeX{} ��������� ����������� ������� �����\index{������� �����} �
������� ����� �������, ����������� �������������� ������ ��� ������.
��� ������������� �� ����� ��������� �����, ������� �� ���������� ��
������. �� ������ ��������� ������� ��, ��� ���������� ������. ������
������ ����� ���������� � ������� ������, � ��������������� ���������
����� ���� ���. ����������� �������� � �������~\ref{parsp}.

� ��������� ������� ����� ���� ���������� ������� \LaTeX{} ���������
������:
\begin{lscommand}
\ci{\bs} ��� \ci{newline}
\end{lscommand}
\noindent �������� ����� ������, �� ������� ������ ������.

\begin{lscommand}
\ci{\bs*}
\end{lscommand}
\noindent ���������, ����� ����, ������ �������� ����� ������������
������� ������.

\begin{lscommand}
\ci{newpage}
\end{lscommand}
\noindent �������� ����� ��������.

\begin{lscommand}
\ci{linebreak}\verb|[|\emph{n}\verb|]|,
\ci{nolinebreak}\verb|[|\emph{n}\verb|]|,
\ci{pagebreak}\verb|[|\emph{n}\verb|]| �
\ci{nopagebreak}\verb|[|\emph{n}\verb|]|
\end{lscommand}
\noindent ��������� ������, ��������� ������ ������, ���������
�������� � ��������� ������ ��������, ��������������. ��������������
�������� \emph{n} ��������� ������ ������ �� �� ��������. �� �����
���� ����� ����� �� ���� �� �������. ������������ \emph{n} � ��������,
�������~4, �� ���������� \LaTeX{} ����������� ������������ ����
�������, ���� ��������� ����� ����� ����. �� ������� ��� �������
<<������� (break)>> � ��������� <<������ (new)>>. ���� ���� ��
�������� ������� <<�������>>, \LaTeX{} ���������� ��������� ������
������� �������� � ����� ������ ��������, ��� ������� � ���������
������. ���� �� ������������� ������ ������ <<����� �������>>,
����������� ��������������� �������.

\LaTeX{} ������ �������� ����������� ��������� �� ��������� ���������
�����. ���� �� �� ����� ����� ������ ������� ������ � ������������ ��
������ �����������, �� ��������� ����� ������ ��������� �� ������
������. \LaTeX{} ����� ������� ����������� (<<\wi{overfull hbox}>>) ��
����� ��������� �������� �����. ���� ����� ��� ���������, �����
\LaTeX{} �� ����� ����� ����� ��� �������� �����.\footnote{����
  \LaTeX{} � ���� ��� ��������������, ����� ������ �� ������ �����
  �����. ���� �� � ������� \ci{documentclass} ����������� �����
  \texttt{draft}, ����� ������ ����� �������� ������� ������ ������ ��
  ������ �����.} ����� ������� \ci{sloppy}, �� ������ �������, �����
\LaTeX{} ��������� ������� ���� ���������. ����� �� ������
������������� ����� ������� ������� ������, ���������� ��������� �����
�������~--- ���� ���� �������� ����� ����� �� ���������. � ���� ������
������������ ������� �������������� (<<\wi{underfull hbox}>>). �
����������� ������� ��������� �������� �� ����� ������. �������
\ci{fussy} ��������� � �������� �������.


\subsection{��������} \label{hyph}

\LaTeX{} ��������� �����, ����� ��� ����������. ���� ��������
��������� �� ������� ���������� ����� ��������, �� ������ ���������
���������, ������ \TeX{} �� ���������� ��� ������ ��������� ������.

�������
\begin{lscommand}
\ci{hyphenation}\verb|{|\emph{������ ����}\verb|}|
\end{lscommand}
\noindent �������� ������� ����, ������������� � �� ���������, ������
� ������, ���������� <<\verb|-|>>. ��� ������� ������ �������� �
��������� �������� �����, � ������ ��������� ������ �����, ���������
�� ������� ����. ������� �������� ������������ ��� �����, ��������� �
������ ������� ������� \ci{hyphenation}. ��� ������, ���, ���� ��
��������� �� � ��������� ���������, ��� ����� ������ �� ����������
��������. ���� �� ��������� �� ����� ������� \verb|\begin{document}|,
  ��������� ��� ���� ��������� ������������ ������, ��������,
  \pai{babel}, �� �������� ����� ��������� ��� �����,
  ����������������� ��� ������ \pai{babel}.
  
  ��������� ������ ��������� �������� � ����� <<hyphenation>>, ��� ��,
 ��� � � ����� <<Hyphenation>>, � ��������� �������� � ������
  <<FORTRAN>>, <<Fortran>> � <<fortran>>. � ��������� �� �����������
  ����������� �������.

������:
\begin{code}
\verb|\hyphenation{FORTRAN Hy-phen-a-tion}|
\end{code}

������� \ci{-} ��������� � ����� ���������� �������. �� �����
���������� ������������ ����������� ������ �������� � ���� �����. ���
������� � ����������� ������� ��� ����, ���������� ����������� �������
(��������, ������� � ���������), ������ ��� \LaTeX{} �� ���������
����� ����� �������������.
%\footnote{Unless you are using the new
%\wi{DC fonts}.}

\begin{example}
I think this is: su\-per\-cal\-%
i\-frag\-i\-lis\-tic\-ex\-pi\-%
al\-i\-do\-cious
\end{example}

��������� ���� ����� �������� ������ �� ����� ������ ��������
\begin{lscommand}
\ci{mbox}\verb|{|\emph{�����}\verb|}|
\end{lscommand}
\noindent ��� ����� � ����� ������ ��������� ���� �������� ������.

\begin{example}
����� ����� �������� ����� ��������.
�� ����� \mbox{0116 291 2319}.

��������
\mbox{\emph{��� �����}} ������
��������� ��� �����.
\end{example}

�������~\ci{fbox} ���������� ������� \ci{mbox}, �� ������ ��
����������� ����� ���������� ������� �����.

\section{����������� ������}

� ��������� ���������� �������� �� ��� ��������� ������� �������
\LaTeX{} ��� ������ ����������� ��������� �����:

\vspace{2ex}

\noindent
\begin{tabular}{@{}lll@{}}
������� & ������ & ��������\\
\hline
\ci{today} & \today   & ������� ���� �� ������� �����\\
\ci{TeX} & \TeX       & �������� ����� ������� ������� �������\\
\ci{LaTeX} & \LaTeX   & �������� \LaTeX \\
\ci{LaTeXe} & \LaTeXe & ������� ���������� \LaTeX\\
\end{tabular}


\section{����������� ����� � �������}

\subsection{����� �������}

��� ������ �������\index{�������} �� \emph{�� ������} ������������
����~\verb|"|\index{""@\texttt{""}}, ��� �� ������� �������. ���
������� ���������� ����������� ����� ����������� � �����������
�������. � \LaTeX{} ����������� ��� �����~\verb|`| � ��������
����������� ������� � ��� �����~\verb|'| � ��������
�����������.\trfootnote{��� ������� ������ ��������� ��� ���� �������,
  �� ������ <<\glqq �������\grqq{} � ��������>>. ��� ������� ��
  ������� ����� ������ ������ ``�����'' � `�����' ������� ����������
  <<���>> � \glqq ���\grqq. ������ ��������� ��� ��������~--- �������
  �� ������������ �����������, �� ������ ��� \ci{flqq}, \ci{frqq}
  (������ ���� ���� ����� ���������� �������� \texttt{<{}<} �
  \texttt{>{}>}\index{<<}\index{>>}) � \ci{glqq}, \ci{grqq}.}

\begin{example}
``����������, ������� ������� `x' ''
\end{example}

\subsection{���� � ������}

\LaTeX{} ����� ��� ������ ���� \wi{����}. ��� �� ��� �� ������
�������� ��������� ������ ���������������� ������~\verb|-|. ���������
�� ����� ���� �� ���� �����, � �������������� ���� �����:

\index{-} \index{--} \index{---} \index{-@$-$}
\index{��������������!�����}

\begin{example}
�������-�-�������, X-����\\
�������� 13--67\\
��~--- ��� ���?\\
$0$, $1$ � $-1$
\end{example}

��� ���� ���������� ���: \texttt{-} \wi{�����}, \texttt{-{}-}
\wi{�������� ����}, \texttt{-{}-{}-} \wi{������� ����} � \verb|$-$|
\wi{���� ������}.\index{����!��������}\index{����!�������}


\subsection{������ ($\sim$)}
\index{www}\index{URL}\index{������}

������ ����� ����������� � WWW-�������. ��� ��������� �� � \LaTeX{}
����� ����������� \verb|\~|, �� ��������� (\~) ����� �� ���, ��� ���
�����. ������ ���������� ��� ���:

\begin{example}
http://www.rich.edu/\~{}bush \\
http://www.clever.edu/$\sim$demo
\end{example}  


\subsection{���� ������� ($\circ$)}

��� ���������� \wi{���� �������} � \LaTeX{}?

\begin{example}
����������� ��� 
$-30\,^{\circ}\mathrm{C}$.
����� �������� �����������������.
\end{example}

%\subsection{������ ���� \texorpdfstring{(\EUR)}{}}
\subsection{������ ���� \texorpdfstring{(\texteuro)}{}}
� ���� �����, ������ � ������, ��� ����������� ������
����������. ������ ����������� ������ �������� ������ ����. �����
�������� ������ \pai{textcomp} � ��������� ���������
\begin{lscommand}
\ci{usepackage}\verb|{textcomp}| 
\end{lscommand}
\noindent �� ������ ������������ ��������
\begin{lscommand}
\ci{texteuro}
\end{lscommand}
\noindent ��� ������� � ����� �������.

���� ��� ����� �� ����� ������������ ������� ����, ��� ��� ��
��������, ��� �� ��������, � ��� ���� ��� ��� ������.

������~--- ������������� ������ \pai{eurosym}. �� �������������
����������� ������ ����:
\begin{lscommand}
\ci{usepackage}\verb|[|\emph{official}\verb|]{eurosym}| 
\end{lscommand}

���� �� ������������� ������������ ������ ����, ��������������� ������
������, ������ ����� \texttt{official} ����������� ����� \texttt{gen}.

����� \pai{marvosym} �������� ��������� ������ ��������, � ��� ����� �
����, ��������� ����� �������:
\begin{lscommand}
\ci{EUR}
\end{lscommand}

\begin{example}
������� \texteuro{}, 
\euro{} � \EUR{} ��� 
�������� ��-�������.
\end{example} 

\subsection{���������� ( \ldots )}

�� ������� ������� \wi{�����} ��� \wi{�������} �������� �� ��
������������, ��� � ����� ������ �����. ������������ � �����, ���
������� �������� ����� ���� ����� � ���������� ����� ������ �
�������������� �����. ������� �� �� ������ ������ \wi{����������}
������ ��������� ��� �����, ��� ��� ���������� ����� ���� �����
������������. ����� ����, ��� ���������� ���� ����������� �������,
������������

\begin{lscommand}
\ci{ldots}
\end{lscommand}
\index{...@\ldots}

\begin{example}
�� ��� ... � ��� ���:\\
���-����, �����, ��������, \ldots
\end{example}

\subsection{��������}

��������� ���������� ���� ���������� �� ������ ������� ������ ����
���� �� ������, � � �������������� �����������
��������.\trfootnote{�������� ������ ������������ ��� ������� ������� ��
  ���������� �����. ��������� ��� ��������, ��� ������� �� �������
  ����� ������� �� ������������. ������, �������� ������� \TeX{}
  ������������ ��� ���������, ��������, ������� � ����.}
\begin{code}
{\large ff fi fl ffi\ldots}\quad
������\quad {\large f{}f f{}i f{}l f{}f{}i \ldots}
\end{code}
��� ��� ���������� \index{��������}�������� ����� ���� ���������
�������� \ci{mbox}\verb|{}| ����� ����� ���������������� �������. ���
����� ���� ���������� ��� ����, ������������ �� ���� ������.
\begin{example}
  \Large �� ``shelfful''\\
  � ``shelf\mbox{}ful''
\end{example}

\subsection{������� � ����������� �������}

\LaTeX{} ������������ ������������� \index{�������}�������� �
����������� ��������\index{����������� �������} �� ������ ������.
�������~\ref{accents} ���������� ������������ ������� � ���������� �
����� �. �������, ��� �� ����� ����� ���� � ������ �����.

����� ��������� ���� ������� ��� ������� i ��� j, ����� ��� ����
������ ���� �������. ��� ����������� ������� \verb|\i| � \verb|\j|.
\begin{example}
  H\^otel, na\"\i ve, \'el\`eve,\\
  sm\o rrebr\o d, !`Se\~norita!,\\
  Sch\"onbrunner Schlo\ss{}
  Stra\ss e
\end{example}

\begin{table}[!hbp]
\caption{������� � ����������� �������} \label{accents}
\begin{lined}{10cm}
\begin{tabular}{*4{cl}}
\A{\`o} & \A{\'o} & \A{\^o} & \A{\~o} \\
\A{\=o} & \A{\.o} & \A{\"o} & \B{\c}{c}\\[6pt]
\B{\u}{o} & \B{\v}{o} & \B{\H}{o} & \B{\c}{o} \\
\B{\d}{o} & \B{\b}{o} & \B{\t}{oo} \\[6pt]
\A{\oe}  &  \A{\OE} & \A{\ae} & \A{\AE} \\
\A{\aa} &  \A{\AA} \\[6pt]
\A{\o}  & \A{\O} & \A{\l} & \A{\L} \\
\A{\i}  & \cm{\j} & \verb|\j| & \cm{!`} & \verb|!`| & \cm{?`} & \verb|?`|
\end{tabular}
\index{i � j ��� �����@\cm{\i} � \cm{\j} ��� �����}\index{�����!�����������}
\index{ae@\ae}\index{�������!umlaut}\index{�������!grave}\index{�������!acute}
\index{oe@\oe}

\bigskip
\end{lined}
\end{table}

\section{��������� ����������� ������}
\index{������ �����}���� ��� ����� ������ ��������� �� �������� ��
����������� ������, �� \LaTeX{} ������ ���� ��������������� �������
��������������� � ���� �������:

\begin{enumerate}
\item ��� ������������ ������������� ���������
  ������\footnote{����������, ������ �����������, ������������ \ldots}
  ������ ���� ���������� �� ������ ����. ��� ������ ������ ���
  ��������� ����������� �������������� ������ \pai{babel} (�����
  Johannes Braams).

\item \LaTeX{} ������ ����� \wi{������� ��������} ��� ������
  �����. ����������� ������ �������� � \LaTeX{} ����� ������. ���
  �������� ������������ ���������� ����� � ������� ������������
  ��������� ��������. ��� \guide{} ������ ��������� ������ ����������
  �� ����.

\item ����������� ��� ����� ������������ �������. ������, ���
  ������������ ����� ������ �������� ��������� (:) ������ ������
  ������. 
\end{enumerate}

���� ���� ������� ��� ��������������� ������� ����������������, ��
������ �������������� ����� \pai{babel} ����������� �������
\begin{lscommand}
\ci{usepackage}\verb|[|\emph{����}\verb|]{babel}|
\end{lscommand}
\noindent ����� ������� \verb|\documentclass|. ����� \emph{����}�
������������ ���� �������, ����� ����� ���� ������� � \guide{}.
\texttt{Babel} ������������� ������������ ���������� ������� ��������
��� ���������� ���� ������. ���� ��������� ���� ������ \LaTeX{} ��
������������ ��������� ��� ���������� �����, \texttt{babel} �����
��������, �� �������� ��������, ��� ��������� �������� �� ������� ����
���������.

��� ��������� ������ \pai{babel} ������ ����� �������, ���������� ����
����������� ��������. ��������, \wi{�������� ����} �������� ���������
�������������� ������ (\"a\"o\"u). � �������������� \pai{babel} ��
������ ������� \"o, ������� \verb|"o| ������~\verb|\"o|.

���� �� ��������� \pai{babel}, �������� ��������� ������, 
\begin{lscommand}
\ci{usepackage}\verb|[|\emph{����1}\verb|,|\emph{����2}\verb|]{babel}| 
\end{lscommand}

\noindent �� ��� ������ �������� ����� ����������� �������

\begin{lscommand}
\ci{selectlanguage}\verb|{|\emph{����1}\verb|}|
\end{lscommand}

%Input Encoding
\newcommand{\ieih}[1]{%
\index{���������!�������!#1@\texttt{#1}}%
\index{������� ���������!#1@\texttt{#1}}%
\index{#1@\texttt{#1}}}
\newcommand{\iei}[1]{%
\ieih{#1}\texttt{#1}}
%Font Encoding
\newcommand{\feih}[1]{%
\index{���������!������!#1@\texttt{#1}}%
\index{��������� ������!#1@\texttt{#1}}%
\index{#1@\texttt{#1}}}
\newcommand{\fei}[1]{%
\feih{#1}\texttt{#1}}

����������� ����������� ������������ ������ ��������� ��� ������� �����������
������� ����� � ����������. \LaTeX{} ����� ������������ �����
������� ��� ������ ������ \pai{inputenc}:

\begin{lscommand}
\ci{usepackage}\verb|[|\emph{���������}\verb|]{inputenc}| 
\end{lscommand}

��� ������������� ����� ������ �� ������ ��������, ��� ������ �����
����������� ������ ���� ������� ����� �� ����� ���������� ��-��
������������� ������ ���������. ��������, �������� ������ \"a �
������� IBM~OS/2 ���������� ��� 132, � �� ��������� Unix ��������,
������������ ��������� ISO-LATIN~1,~--- ��� 228. ������� �����������
��� ����������� � �������������. � ����������� �� ����� �������, �����
��������������, ��������, ��������� ���������:

\begin{center}
\begin{tabular}{l | r}
������������ ������� & ���������\\
\hline
Mac     &  \texttt{applemac} \\
Unix    &  \texttt{latin1} \\ 
Windows &  \texttt{ansinew} \\
DOS, OS/2 & \texttt{cp850}
\end{tabular}
\end{center}

���� �� ������ ������������ �������� � �������������� ��������
�����������, ����� ������������� �� Unicode ��� ������ ������
\pai{ucs}:
\begin{lscommand}
\ci{usepackage}\verb|{ucs}|\\ 
\ci{usepackage}\verb|[|\iei{utf8}\verb|]{inputenc}| 
\end{lscommand}
��� �������� ��� ������������ ����� \LaTeX{} � \iei{utf8},
������������� ���������, ��� ������ ������ �������� �� ������ ��
������� ����.

���� ������~--- � ���������� ������. ��� ����������, � ����� ������� �
�������� ������ \TeX{} ��������� ������ �����. ��������� �������
��������� ����� ���������� �� ���� ��������� ������, ��� ���������
����� ����������� ������� �������. ��������� ������� ��������������
������� \pai{fontent}:\label{fontenc}
\begin{lscommand}
\ci{usepackage}\verb|[|\emph{���������}\verb|]{fontenc}|
\index{��������� ������}
\end{lscommand}
\noindent ��� \emph{���������}~--- ��������� ��������� ������. �����
������������ ��������� ��������� ���������.

��������� \LaTeX{} �� ���������~--- \fei{OT1}, ��������� �������������
������ \TeX{} Computer Modern. ��� �������� ������ 128 ��������
7-������� ������ �������� ASCII. ����� ��������� ���������������
�������, \TeX{} ������� ��, ���������� ���������� ������ � ��������.
�������� �� ��, ��� �������������� ������� �������� ���������, ����
������ �� ���� �������������� ��������� �������� ������ ����,
������������ ��������������� �������. ����� ����, ��������� �����
��������� ������ �� ���������� ������� ����� �������, �� ������ ��� �
������ ��-��������� ���������, ������, ���������� ��� ���������.

��� ������ � ����� ������������� ���� ������� ��������� 8-������
������� �������, ������� �� CM. \emph{Extended Cork} (EC) ������ �
��������� \fei{T1} �������� ����������� ������� ��� �����������
��������������� ����, ������������ � ����������� ������, ���������� ��
��������. ����� ������� LH �������� ������� ������, ���������� ��
���������. ����� �������� ����� ������������� ������, ��� ������������
� ������ ��������� �������: \fei{T2A}, \fei{T2B}, \fei{T2C} �
\fei{X2}.\footnote{�������� ������, �������������� ������ �� ����
  ��������� ����� ����� � \cite{cyrguide}.} ����� CB �������� ������ �
��������� \fei{LGR} ��� ������ ��������� �������.

��������� ��� ������, �� ������ �������� ��� ������ ������� ����������
�������� � ��-���������� ����������. ��� ���� ������������
������������� ����� CM-�������� �������~--- � ���, ��� ���
������������� ������ ��������� CM �� ���� ��������, ������ � ���������
���������������� �������� ������.

\subsection{��������� ��������������}
\secby{Demerson Andre Polli}{polli@linux.ime.usp.br}

��� ��������� ��������� � ����� ������������� ������������� ������ ��
�������������\index{�������������}, ����������� ��������
\begin{lscommand}
\verb|\usepackage[portuguese]{babel}|
\end{lscommand}
���, ��� ������ � ������������ ��������, ������� � �������� �����
\texttt{brazilian}\index{�����������}.

��������� � ������������� ��������� ��������������� ����, ��� �����
���� ������ ������������
\begin{lscommand}
\verb|\usepackage[latin1]{inputenc}|
\end{lscommand}
��� ����, ����� ��������� �� �������. ��� ������ ��������� �������
\begin{lscommand}
\verb|\usepackage[T1]{fontenc}|
\end{lscommand}

���������, ����������� ��� ������ � ������������� ������, �������� �
�������~\ref{portuguese}. ��������, ��� �� ����� ���������� ���������
latin1, ��� ��� ��� ��������� �� ����� �������� �� Mac ��� �
DOS. ������ ���������� ���������, ���������� ��� ����� �������.

\begin{table}[btp]
\caption{��������� ��� ������������� ����������} \label{portuguese}
\begin{lined}{5cm}
\begin{verbatim}
\usepackage[portugese]{babel}
\usepackage[latin1]{inputenc}
\usepackage[T1]{fontenc}
\end{verbatim}
\bigskip
\end{lined}
\end{table}

\subsection{��������� ������������}
\secby{Daniel Flipo}{daniel.flipo@univ-lille1.fr}

��������� ������� ��� �������� � \LaTeX{}
�����������\index{�����������} ����������. ��������� ���������
������������ ����� ��������

\begin{lscommand}
\verb|\usepackage[frenchb]{babel}|
\end{lscommand}

��������, ��� �� ������������ �������� �������� ����� \pai{babel} ���
������������ �����~--- \emph{frenchb} ��� \emph{francais}, � ��
\emph{french}. 

��� ������� ����������� ��������, ���� ���� \LaTeX{}-������� ���������
���������. ����� ����, ���� �������������� ����� ����� ����
�����������: \verb+\chapter+ �������� Chapitre, \verb+\today+ ��������
������� ���� ��-����������, � ��� �����. ���������� ��������� �����
����� ������, ����������� ����� �������� ����������� ������. ���
��������� �������� �������~\ref{cmd-french}.

\begin{table}[!htbp]
\caption{����������� ������� ��� ������������ �����} \label{cmd-french}
\begin{lined}{9cm}
%%\selectlanguage{french}
\begin{tabular}{ll}
\verb+\og guillemets \fg{}+         \quad &\og guillemets \fg \\[1ex]
\verb+M\up{me}, D\up{r}+            \quad &M\up{me}, D\up{r}  \\[1ex]
\verb+1\ier{}, 1\iere{}, 1\ieres{}+ \quad &1\ier{}, 1\iere{}, 1\ieres{}\\[1ex]
\verb+2\ieme{} 4\iemes{}+           \quad &2\ieme{} 4\iemes{}\\[1ex]
\verb+\No 1, \no 2+                 \quad &\No 1, \no 2   \\[1ex]
\verb+20~\degres C, 45\degres+      \quad &20~\degres C, 45\degres \\[1ex]
\verb+\bsc{M. Durand}+              \quad &\bsc{M.~Durand} \\[1ex]
\verb+\nombre{1234,56789}+          \quad &\nombre{1234,56789}
\end{tabular}
%%\selectlanguage{english}
\bigskip
\end{lined}
\end{table}

�� ����� �������� ��������� ���� �������. ����� ��������� �������
���������, �������� \pai{babel} ��� ������ � �����������, ���������
����� \LaTeX{} ���� \texttt{frenchb.dtx} � �������� ����������
\texttt{frenchb.dvi}. 

\subsection{��������� ���������}

��������� ������� ���, ���� ����� ��������� � \LaTeX{}
��������\index{��������} ���������. ��������� ��������� ���������
����� ��������

\begin{lscommand}
\verb|\usepackage[german]{babel}|
\end{lscommand}

��� ������� �������� ��������, ���� ���� \LaTeX{}-������� ���������
���������. ����� ����, ������������� ������������ ����� ������
��������, ��������, ``Chapter'' ������ ``Kapitel''. ���������� �����
����� ������, ���������� ����� ��������� ������, ���� ���� ��
������������ ����� \pai{inputenc}. ��� ��������� ��������
�������~\ref{german}. � �������������� inputenc ��� ��� ���������� ��
������, �� ��� ����� ������ ������������ ������ ���� ���������.

\begin{table}[!htbp]
\caption{����������� ������� ��������� �����} \label{german}
\begin{lined}{8cm}
%%\selectlanguage{german}
\begin{tabular}{*2{ll}}
\verb|"a| & "a \hspace*{1ex} & \verb|"s| & "s \\[1ex]
\verb|"`| & "` & \verb|"'| & "' \\[1ex]
\verb|"<| or \ci{flqq} & "<  & \verb|">| or \ci{frqq} & "> \\[1ex]
\ci{flq} & \flq & \ci{frq} & \frq \\[1ex]
\ci{dq} & " \\
\end{tabular}
%%\selectlanguage{english}
\bigskip
\end{lined}
\end{table}

� �������� ������ ����� ����������� ����������� ����� �������
(\flqq{}guillemets\frqq). �������� ���������, ������, ���������� ��
��-�������. ������ � �������� ����� ������ �������� \frqq{}���\flqq. �
���������������� ����� ��������� ��������� ����������
\flqq{}guillemets\frqq ��� ��, ��� � �� �������.

������� �������� ������� � ��������� ���� \verb+\flq+: � ������� OT1
(� ��� ���������� �� ���������) ��� ������� �������� ���
�������������� ������ ``$\ll$'', �� ���� � ���������� �����
��������. ������ � ��������� T1, � ������ �������, ����� ���������
�������. ��� ���, ���� ��� ����� ����� �������, ���������, ���
�������� ��������� T1 (\verb|\usepackage[T1]{fontenc}|).

\subsection[��������� ����������]{���������
  ����������\footnotemark}\label{support_korean}%
\footnotetext{%
��������������� ��������� ��������, ���������� ��������� �������������
\LaTeX{}. ���� ������ ������� Karnes Kim �� ����� ��������� �������
������������ lshort. ������ ��������� �� ���������� Shun Jungshik �
�������� Tobi Oetiker.}

��������� \LaTeX{} ��� ������ ���������� �����\index{���������}, ���
����� ��������� ��� ��������:

\begin{enumerate}
\item ����� ��������� ������������� ������� ����� �� ���������. ���
  ������ ���� � ������� �������� ������, ��, ��������� �������
  ���������� ����� �� ������ � ����� US-ASCII, ��� ����� �������
  ��������� � ����������� ���������� ASCII. ��� ����� ����������
  ��������� ��������� ��������� ������~--- EUK-KR � �� �����������
  ����������, ������������ � ��������� ������� MS Windows,
  CP949/Windows-949/UHC. ��� ��������� �������� ASCII ���
  ������������, ���������� ���������� ISO-8859-\textit{x}, EUC-JP,
  Shift\_JIS � Big5. � ������ �������, ����� Hangul, ����� Hanjas
  (������������ � ����� ��������� �������), Hangul Jamos, ��������,
  ��������, ���������, ��������� � ������ ������� �� KS~X~1001
  �������������� ����� �������. ������ ���� ����� �������������
  ������� ���. �� �������� 90-� ����� ��������� ���������� ��������� �
  ���������������� �� ��������� ������������ ������ � �������. ��
  ������ ����������� ������ ���������� \url{http://jshin.net/faq},
  ����� �������� ������������� �� ���� �������. � ���� ����� ��� ���
  �������� ������������ ������� (Mac OS, Unix, Windows) ��������
  ������ ��������� ������������� ��������� � �����������
  �������������������, ��� ��� �������������� ��������� ������� ��� ��
  �������� ���������, ���� � ���������������� �������.
  
\item \TeX{} � \LaTeX{} ���� ���������� ������� ��� �������������,
  ������� � �������� �� ����� 256 ��������. ��� ����, ����� ���
  �������� � �������, ��� ������� ������ ��������, ��������,
  ���������\footnote{��������� ������� Hangul~--- ���������
  ������������ � 14 ��������� ���������� � 10 ��������� ��������
  (Jamos). � ������� �� �������� ��� ���������, ��������� �������
  ������ ���� ��������� � ������������� ������ �������� ������ ��
  �������, ��� � ���������. ������ ���� ������������ ����. �� �����
  ���������� ���������� ������� � ��������� ����� ������������
  ����������� ����� ������. ������, ����������� ���������������
  ��������� (��� � ����� �����, ��� � � ��������) �������� ���������
  ����������� �� ������������ ���� ������. �������������, ����������
  �������� ����� �������������� ���������� ������. ��������� ���������
  �������� ���������� �������������� ���� ��� ������� �� ���� ������
  (KS~X~1001:1998 � KS~X~1002:1992). ����� �������, Hangul, ������
  ���������, ����������, ��� �� �����, ��� ��, ��� ������������ �����
  � ������, � �������� ��������������� ������. ��������
  ISO~10646/Unicode ���������� ��� ������� ������������� Hangul,
  ����������� � \emph{�����������} �����, �������, � ���������� ��
  ���� �������������� ����������� ������ Hangul
  (\url{http://www.unicode.org/charts/PDF/UAC00.pdf}), ��� �
  Conjoining Hangul Jamos (�������:
  \url{http://www.unicode.org/charts/PDF/U1100.pdf}). ���� �� �����
  ����������� ����� � ������ ���������� ������ � \LaTeX{} �
  ����������� ��������~--- ��������� ������������� ��������� (�,
  ��������, � ������� ���������) ������, ������� ����� �����������
  ������ ����������� ������ Jamos � Unicode. ���� �������, ��� �������
  �������� \TeX, ��������, $\Omega$ � $\Lambda$ � ����� ������
  ����������� ������� ��� ���� ������, ��� ��� ��������� ��������� �
  �������� ������ ������� �� MS Word, ��� ��� ���� �������� ��������
  ��������� �������������� ����������.} � ���������, ��� ����������
  �������� ����������. �� ����� ������ ����� CJK � �������� ���
  ��������� ����� ������ �� ����� ���������� � 256 ������� �
  ������. ��� ���������� ����� ���� ��� ������ ������������ ������:
  \wi{H\LaTeX} (����� Un~Koanghi), \wi{h\LaTeX{}p} (�����
  Cha~Jaechoon) � \wi{����� CJK} (����� Werner~Lemberg).\footnote{%
    �� ����� �������� �� ������� \CTANref|language/korean/HLaTeX/|\\
   \CTANref|language/korean/CJK/| �
   \texttt{http://knot.kaist.ac.kr/htex/}}
H\LaTeX{} � h\LaTeX{}p ���������� ��� ���������� ����� � ����������
  ��������� �����������, ������ ��������� ���������. ��� ���
  ������������ ������� ����� � ��������� EUK-KR. H\LaTeX{} �����
  ������������ ����� � ���������� CP949/Windows-949/UHC � UTF-8, ����
  ������������ ������ � $\Lambda$, $\Omega$.

����� CJK �� ���������� ��� ���������� �����. �� ������������ �������
����� � UTF-8, � ����� � ��������� ���������� CJK, ������� EUK-KR �
CP949/Windows-949/UHC. ��� ����� ������������ ��� ������ ���������� �
������������ ����������� (�������� ���������, �������� �
���������). ����� CJK �� �������� ��������� �����������, �����������
���, ��� ���� � H\LaTeX{}, � ����� �� �������� ������ ���������
����������� ��������� �������.

\item �������� ���� ������������� �������� ������, ������� \TeX{},~---
  �������� ���������, ����������� <<�����������>>
  �����������������. ����� ����������, ��� �������� ������ �������
  �������~--- ��� ����� ������ ���������������� �������. �����������
  H\LaTeX{} �������� \index{��������� �����!UHC �����}UHC \PSi{}
  ������ ������ ��������� �������� � TrueType ������
  Munhwabu\footnote{������������ �������� �����} ���� ���������
  ��������. ����� CJK �������� � ������� �������, ���������������� �
  ������ ������� H\LaTeX{}, � ����� ����� ������������ TrueType �����
  cyberbit �������� Bitstream.
\end{enumerate}

��� �������� ������ H\LaTeX{} ��������� � ��������� ���������:
\begin{lscommand}
\verb+\usepackage{hangul}+
\end{lscommand}

��� ������� �������� ��������� �����������. ��������� ����, ��������,
�����������, ���������� � ������� ����������� ����� ���������� ��
���������, � �������������� ��������� ��������� ���, ����� ���������
�������� � ����� ��������. ����� ����� ������������� ��������������
<<����� ������>>. � ��������� ����� ���������� ���� �����������
������, ������������� �������������, �� �������� �� �����. �����
������� �� ���� ���������, ������� �� ����, ������������� ��
���������� ���� �� ������� ��� ���������. (�� ����� ����, ������� ����
�������, �� ����� ���� ������.) �������� �� ���������� �������� �����
�������� ���������� �������, ������ ���������� ������� ����������,
����� ������� ������������ � ������� � ������ ������ � �������������
������������ �������, ���������� ��� ��������������
���������. ��������� ��� ������ ��������� ������ ���������� �������
����� �������~--- ������� ���������� �������. H\LaTeX{} ��������������
���� �������.

���� ��� �� ����� ��������� �����������, � �� ������ ������ �������
��������� �����, ��������� ������ � ��������� ���������:
\begin{lscommand}
\verb+\usepackage{hfont}+
\end{lscommand}

������ � ��������� ���������� � ������� ���������� ������ ��� ������
H\LaTeX{} ��������� � \emph{����������� �� H\LaTeX}. �������� ���-����
������ ��������� ������������� \TeX{} (KTUG) �� ������
\url{http://www.ktug.or.kr/}. ���������� ����� ��������� �������
���������� ���������.


\subsection{��������� ���������}
\secby{Maksym Polyakov}{polyama@myrealbox.com}

����� \pai{babel}, ������� � ������~3.7, �������� ���������
���������~\fei{T2*} � ��������� ������� ����������, ������� �
���������� ������� � �������������� �������������
��������.\index{����������}\index{�������}\index{����������} 

��������� ��������� �������� �� ����������� ���������� \LaTeX{}:
������� \pai{fontenc} � \pai{inputenc}. ��, ���� ��� �����
������������ ��������� � �������������� ������, ��������� �� ������
\pai{fontenc} ����� \pai{mathtext}:\footnote{���� �� �����������
  �������� \AmS-\LaTeX{}, ���������� �� ����� �� ������� \pai{fontenc}
  � \pai{babel}.}
\begin{lscommand}
\verb+\usepackage{mathtext}+\\
\verb+\usepackage[+\fei{T1}\verb+,+\fei{T2A}\verb+]{fontenc}+\\
\verb+\usepackage[+\iei{koi8-r}\verb+]{inputenc}+\\
\verb+\usepackage[english,bulgarian,russian,ukranian]{babel}+
\end{lscommand}

������ ������, \pai{babel} ������������� ������� ��������� ������� ��
���������. ��� ���� ������������� ������ ��� ����� \fei{T2A}. ������,
��������� �� ���������� ������ ����� ���������� ������. ���
������������� ����������, ������������ �����, ������������ ��
��������� � ��������, ����� ����� ���� ���������� ��������� ���������
�������. \pai{babel} ����������� � ������������ �� ����������
���������, ����� �� ������� ���� � ���������.

� ���������� �� ��������� ���������, ���������� �������������
������������ ����� � ��������� ��������� ����������� ��� �����
������������ ������ (��������, \ci{frenchspacing}), \pai{babel}
������������� ��������� ������ ��� ������� � ������������ ��
����������� �����������, �������� ��� ����������� ������.

��� ���� ���� ������ ��������������� ������������� ����������:
������������� ���� ��� ����� (��� ������ \'��� ���������� ���� �
�������� ���������� ���������), ���� ��� ������ ����, ������� �
������� ���������� ����������. �������� �������~\ref{Cyrillic}.

% Table borrowed from Ukrainian.dtx
\begin{table}[htb]
  \begin{center}
  \index{""-@\texttt{""}\texttt{-}} 
  \index{""---@\texttt{""}\texttt{-}\texttt{-}\texttt{-}} 
  \index{""=@\texttt{""}\texttt{=}} 
  \index{""`@\texttt{""}\texttt{`}} 
  \index{""'@\texttt{""}\texttt{'}} 
  \index{"">@\texttt{""}\texttt{>}} 
  \index{""<@\texttt{""}\texttt{<}} 
  \caption[����������, ������� � ����������]{��������������
           �����������, �������� ������� \pai{babel} ��� �����������,
           �������� � ����������� ������}\label{Cyrillic}
  \begin{tabular}{@{}p{.1\hsize}@{}p{.9\hsize}@{}}
   \hline
   \verb="|= & ��������� �������� � ���� �������.               \\
   \verb|"-| & �����, ����������� ���������� ������� �����. \\
   \verb|"---| & ������������� ���� � ������� ������.                      \\
   \verb|"--~| & ������������� ���� � ��������� ��������.       \\
   \verb|"--*| & ������������� ���� ��� ����������� ������ ����.         \\
   \verb|""| & ��� |"-|, �� �� ���������� ����� ������
               (��� ��������� ���� � �������, ��������, |x-""y|.\\
   \verb|"~| & �������� ��������� ����� ��� �������.        \\
   \verb|"=| & �������� ��������� ����� � ��������, �������� ������� �
   ������-�����������.                   \\
   \verb|",| & ������ ��� ��������� � ����������� �������� � ���������
   �� ���� �������.                                \\
   \verb|"`| & �������� ����� ������� ������� 
               (�������� ���: ,\kern-0.08em,).                     \\
   \verb|"'| & �������� ������ ������� ������� (�������� ���: ``).       \\%''
   \verb|"<| & ����������� ����� ������� ������� (�������� ���: $<\!\!<$).  \\
   \verb|">| & ����������� ������ ������� ������� (�������� ���: $>\!\!>$). \\
   \hline
  \end{tabular}
  \end{center}
\end{table}

������� � ���������� ����� ������ \pai{babel} ������ �������
\ci{Asbuk} � \ci{asbuk}, ������� �������� ���������� ��������
\ci{Alph} � \ci{alph}, �� ���������� ��������� � �������� �����
�������� ��� ����������� ��������� (� ����������� �� �������� �����
���������). ���������� ����� \pai{babel} ������ ������� \ci{enumBul} �
\ci{enumLat} (\ci{enumEng}), ������� ���������� \ci{Alph} � \ci{alph}
������������ ����� ���� �����������, ���� ���������� (�����������)
���������. �� ��������� ������������ ���������� �����.

%Finally, math alphabets are redefined and  as well as the commands for math
%operators according to Cyrillic typesetting traditions. 

\section{������� ����� �������}

��� ��������� ������� ������� ���� ������ \LaTeX{} ��������� ���������
��������� ����� �������. � ����� ����������� �� ��������� ������
������� ��������, ����� ����� ����� �����������.\trfootnote{�
  ������������ � ���������� �������, ��������� � ����������
  �����.} \LaTeX{} ������������, ���
����������� ������������� �������, ��������������� ���
���������������� �������. ���� ����� ������� �� ������ � �������
��������, ��� �� ��������� ������ �����������, ��� ��� ����� �����
���� �������� �������� ������ ������������ ��� ����������.

����� ���������� �� ���� ������������� ������ ���� ���� ���������
�������. ����~<<\verb|\|>> ����� �������� ���� � ���������� ������,
������� �� ����� ��������. ����~<<\verb|~|>> ���� ������, ������� ��
����� ����������� � �������, ����� ����, ��������� ������ ������.
������� \verb|\@| ����� ������ ���������, ��� ��� ����� �����������
�����������, �������� �� ��, ��� ����� �� ������ �������� ��������.
\cih{"@} \index{~@ \verb.~.} \index{������@������ (\verb.~.)}
\index{�����, ������ �����}

\begin{example}
Mr.~Smith was happy to see her\\
cf.~Fig.~5\\
I like BASIC\@. What about you?
\end{example}

�������������� ������ ����� ����� ����� ��������� ��������
\begin{lscommand}
\ci{frenchspacing}
\end{lscommand}
\noindent ������� ��������� \LaTeX{} \emph{��} ��������� ������� �����
����� �����, ��� ����� ������� ��������. ��� ������ ��� ������,
�������� �� �����������, �� ����������� ������������. ���� ��
����������� \ci{frenchspacing}, ������� \verb|\@| �� �����.

\section{���������, ����� � �������}

����� ������ �������� ��������������� � ����� ������, �� ������
��������� �� �� �����, ������� � ����������. \LaTeX{} ������������ ���
������������ ���������, ������������ � �������� ��������� ���������
�������. ���� ����~--- ������������ �� � ���������� �������.

����� \texttt{article} �������� ��������� ������� ���������������:

\begin{code}
\ci{section}\verb|{...}           |\ci{paragraph}\verb|{...}|\\
\ci{subsection}\verb|{...}        |\ci{subparagraph}\verb|{...} |\\
\ci{subsubsection}\verb|{...}|
\end{code}

� ������� \texttt{report} � \texttt{book} �� ������ ������������
�������������� ������� \ci{chapter}\verb|{...}|.

���� �� ������ ������� ��� �������� �� ����� ��� ��������� ���������
��������/����, ����������� �������  \ci{part}\verb|{...}|.

��� ��� ���� (chapters) � ������ \texttt{article} ���, ��
������ �������� ����� ��������� � ����� � �������� ����. ���������
����� ���������, ��������� � ������ ������ ���������� ���������������
\LaTeX{} �������������.

��� �� ������ ���������������~--- ���������:
\begin{itemize}
\item ������� \ci{part} �� ������ �� ������������������ �����������
  ����.
\item ������� \ci{appendix} ��������� �� �����. ��� ������ ��������
  ���������� ����� ������� ������ ����.\footnote{� ������
    \texttt{article} �������� ��������� ��������.}
\end{itemize}



\LaTeX{} ������� ����������, ���� ��������� �������� � ������ �������
�� ����������� ����� ���������� ���������. �������
\begin{lscommand}
\ci{tableofcontents}
\end{lscommand}
\noindent ��������� ���������� � �� �����, ��� ��� �������. �����
�������� ���������� \wi{����������}, ����� �������� ������ ����
��������� \LaTeX{} ������. � ������ ������� ����� ���� ��������� �
������ ������. ����� ��� �����������, \LaTeX{} ��� �����������.

��� ����������������� ������� ��������������� ���������� ����� �
��������� �� ����������. ����� ������� ���������� ����������� \verb|*|
� ����� �������. ��� ���������� ��������� ��������, ������� ��
���������� � �� ���������� � ����������. ��������, �������
\verb|\section{�������}| ���������� \verb|\section*{�������}|.\sloppypar

������ ��������� �������� ���������� � ���������� ����� � ��� �� ����,
� ����� ��� �������� � ������. ������ ��� ���������� ��-�� ����, ���
��������� ������� ������ ��� ����������. ������� ���������� ����� �
���� ������ ����������� �������������� ���������� ����� ����������
����������.
\begin{code}
\verb|\chapter[��������� ��� ����������]{���~--- �������,|\\
\verb| ������� � ����� ������ ���������, ������������ � ������}|
\end{code}

��������� ����
\index{����!���������}\index{��������� ����}
��������� � ����� ������������ ��� ������ �������
\begin{lscommand}
\ci{maketitle}
\end{lscommand}
\noindent ��� ���������� ������ ���� ���������� ���������
\begin{lscommand}
  \ci{title}\verb|{...}|, \ci{author}\verb|{...}| �
  \ci{date}\verb|{...}|
\end{lscommand}
\noindent �� ������� ������ \verb|\maketitle|. �������� �������
\ci{author} ����� ��������� ��������� ����, ����������� ���������
\ci{and}.

������ ��������� �� ���������� ������ ����� ���� ������ ��
�����������~\ref{document} �� ��������~\pageref{document}.

������ ��������� ���� ������ ���������������, \LaTeXe{} ������ ���
�������������� ������� ��� ������������� � ������� \texttt{book}.
��� ������� ��� ������� ����� ����������. ������� �������� ���������
���� � ��������� ������� ���, ��� ��� ��������� ��
�����:

\begin{description}
\item[\ci{frontmatter}] ������ ���� ����� ������ �������� �����
  \verb|\begin{document}|. ��� ����������� ��������� ������� ��
    ������������� ������� ����. ��� ������� ����� ����� ����������
    ������� ��������������� �� ����������� (��������,
    \verb|\chapter*{�����������}|), ����� \LaTeX{} �� ��������� ���
    �������. 
  \item[\ci{mainmatter}] ����������� ����� ����� ������ ������
    �����. ��� �������� ��������� ������� ������� � ���������� �������
    �������.
  \item[\ci{appendix}] �������� ������ ��������������� ��������� �
    ����� �����. ����� ���� ������� ����� ����� ������������ �������.
  \item[\ci{backmatter}] ����������� ����� ������ ���������� �������
    �����, ��������, ����� ������������� � ���������� ����������. �
    ����������� ������� ���������� �������� ������� ��� �� �����.
\end{description}

\section{������������ ������}

� ������, ������� � ������� ����� ����������� \wi{������������ ������}
�� �����������, ������� � ��������� ����� ������. ��� ����� \LaTeX{}
������������� ��������� �������:
\begin{lscommand}
\ci{label}\verb|{|\emph{�����}\verb|}|, \ci{ref}\verb|{|\emph{�����}\verb|}|
� \ci{pageref}\verb|{|\emph{�����}\verb|}|
\end{lscommand}
\noindent ��� \emph{�����}~--- ��������� �������������
�������������. \LaTeX{} �������� \verb|\ref| ������� �������,
����������, �����������, ������� ��� ���������, ��� ���� ������������
��������������� ������� \verb|\label|. \verb|\pageref| �������� �����
��������, ��������������� ������� \verb|\label|.\footnote{��������, ���
  ��� ������� �� �����, �� ��� ������ ��� ���������. \ci{label} ������
  ��������� ��������� ������������� ������������ �����.} ��� ��, ���
� � ������ � ����������� ��������, ����� ����� ������������ ������ ��
����������� �������.

\begin{example}
������ �� ����
������~\label{sec:this}
�������� ���: <<��.
������~\ref{sec:this} ��
���.~\pageref{sec:this}.>>
\end{example}

\section{������}

�������
\begin{lscommand}
\ci{footnote}\verb|{|\emph{����� ������}\verb|}|
\end{lscommand}
\noindent �������� ������ ����� ������� ��������. ������ ������ ������
���������� ����� ����� ��� �����������, � ������� ��� ���������. �
������� ����� ������, ����������� � �����������, ������ ���������
����� ����� ������ ��� �������\footnote{��������, ��� ������ ���������
  �������� �� ��������� ������ ���������. ��� ������ ������ ������,
  ������ ��� ��~--- ���������� ��������. ������� ���������� ��������
  ���, ��� �� ������ �������, � �������� ����� ���������.}.
%�� ����� �����. BT.

\begin{example}
  ������������ \LaTeX{}
  ����� �����������
  ������.\footnote{%
    ���~--- ������.}
\end{example}


\section{���������� �����}

� ��������, ������������ �� �������, \texttt{������ ����� ����������
  \underline{��������������}}:

\begin{lscommand}
\ci{underline}\verb|{|\emph{�����}\verb|}|
\end{lscommand}

� �������� ��������, ������, ��� ����� ���������� \emph{��������}. ������� ���
������������ �� ����� \emph{���������} ����������
\begin{lscommand}
\ci{emph}\verb|{|\emph{�����}\verb|}|
\end{lscommand}
\noindent �� ���������� �������� ����� ���
���������\index{���������}. ��� �� ����� ���� ������ ��� �������,
������� �� ���������:

\begin{example}
���� �� �����������
\emph{��������� � ���
���������� ������, ��
\LaTeX{} ����������
\emph{������} �����.}
\end{example}

�������� ������� ����� ��������� \emph{���������} � �����
\emph{������}:

\begin{example}
\textit{�� ������ �����
  \emph{��������} �����,
  ������ ��� ��������,}
\textsf{������� ���
  \emph{�������}}
\texttt{��� � �����
  \emph{������� �������}.}
\end{example}

\section{���������} \label{env}

��� ������� ����������� ����� ������ \LaTeX{} ���������� ���������
\index{���������}��������� ��� ������ ����� ��������������:
\begin{lscommand}
\ci{begin}\verb|{|\emph{��������}\verb|}|\quad
   \emph{�����}\quad
\ci{end}\verb|{|\emph{��������}\verb|}|
\end{lscommand}
\noindent ��� \emph{��������} ���������� ���������. ��������� �����
�������� ������ ���������, �������� ������� ������ � ��������:
\begin{code}
\verb|\begin{aaa}...\begin{bbb}...\end{bbb}...\end{aaa}|
\end{code}

\noindent � ��������� �������� �������������� ��� ���� ������
����������.

\subsection{������, ������������ � ��������}

��������� \ei{itemize} �������� ��� ������� �������, ���������
\ei{enumerate}~--- ��� ������������ �������, � ���������
\ei{description}~--- ��� ��������.
\cih{item}

\begin{example}
\flushleft
\begin{enumerate}
\item �� ������ ��� ������
��������� ��������� �������:
\begin{itemize}
\item �� ��� ����� ����������
�����.
\item[-] � �������.
\end{itemize}
\item ������� �������:
\begin{description}
\item[������] ���� �� ������
����� �� ��������� � ������.
\item[�����] ����, ������,
������ ����� �����������
�������.
\end{description}
\end{enumerate}
\end{example}

\subsection{������������ �����, ������ � �� ������}

��������� \ei{flushleft} � \ei{flushright} ����������� ������,
����������� ����� ��� ������.\index{������������!������ ��� �����}
��������� \ei{center} ���� �������������� �����. ���� �� ��
����������� \ci{\bs} ��� �������� �������� �����, \LaTeX{} ���������
�� �������������.

\begin{example}
\begin{flushleft}
  ���� �����\\ �������� �����.
  \LaTeX{} �� ��������� �������
  ��� ������ ���������� �����.
\end{flushleft}
\end{example}

\begin{example}
\begin{flushright}
  ���� �����\\ �������� ������.
  \LaTeX{} �� ��������� �������
  ��� ������ ���������� �����.
\end{flushright}
\end{example}

\begin{example}
\begin{center}
  � ������\\�����
\end{center}
\end{example}

\subsection{������ � �����}

��������� \ei{quote} ������� ��� �����, ������ ���� � ��������.

\begin{example}
������������ ������� ���
����� ������:
\begin{quote}
  ������ ������ ������ 
  ��������� �� ������ 
  66~��������.
\end{quote}

������� \LaTeX{} ������ ������
�������� ���� �������. ������� 
� ������� ����� ��������� ����� 
� ��������� �������.
\end{example}

���������� ��� ��� ������� ���������: \ei{quotation} � \ei{verse}.
��������� \texttt{quotation} ������� ��� ����� ������� �����,
������������ ��������� �������, ������ ��� ��� �������� ������ �
������� ������. ��������� \texttt{verse} ���������� ��� ������, ���
����� ������� �����. ������ ����������� ��� ������ \ci{\bs} � �����
������ � ������ ������ ����� ������ ������.

\begin{example}
� ���� ������ ���� ����������
������������� ��������: ���
������-������:
\begin{flushleft}
\begin{verse}
Humpty Dumpty sat on a wall:\\
Humpty Dumpty had a great fall.\\
All the King's horses and all
the King's men\\
Couldn't put Humpty together
again.
\end{verse}
\end{flushleft}
\end{example}

\subsection{���������� ���������������}

�����, ����������� ����� \verb|\begin{|\ei{verbatim}\verb|}| �
\verb|\end{verbatim}| ����� �������� ���������, ��� ��������� ��
������� �������, �� ����� ��������� � ���������� �������, ���
���������� ����� �� �� �� ���� ������ \LaTeX{}.

������ ������ ����������� ������� ��������� �������
\begin{lscommand}
\ci{verb}\verb|+|\emph{�����}\verb|+|
\end{lscommand}
\noindent ����� <<\verb|+|>>~--- ��� ������ ������
�������-������������. �� ������ ������������ ����� ������, ����� ����,
<<\verb|*|>> ��� �������. ������ ������� �� \LaTeX{} � ���� �������
������� ���� ��������.

\begin{example}
������� \verb|\ldots| \ldots

\begin{verbatim}
10 PRINT "HELLO WORLD ";
20 GOTO 10
\end{verbatim}
\end{example}

\begin{example}
\begin{verbatim*}
�������   ���������
verbatim         ��
���������� ��������
�������   �  ������
\end{verbatim*}
\end{example}

�������  \ci{verb} ���� ����� ������������ ����������� ������� ��
����������:

\begin{example}
\verb*|���   ��� :-) |
\end{example}

��������� \texttt{verbatim} � ������� \verb|\verb| ������ ������������
������ ���������� ������ ������.


\subsection{�������}

��������� \ei{tabular} ���������� ��� ������� ������, ��������, �
��������������� � ������������� �������. \LaTeX{} �������������
���������� ������ ��������.

�������� \emph{������������} �������
\begin{lscommand}
\verb|\begin{tabular}[|\emph{�������}\verb|]{|\emph{������������}\verb|}|
\end{lscommand}
\noindent ���������� ������ �������. ����������� \texttt{l} ���
������� ������, ������������ �����, \texttt{r} ��� ������,
������������ ������ � \texttt{c} ��� ��������������� ������,
\verb|p{|\emph{������}\verb|}| ��� �������, ����������� �����������
����� � ��������� �����, � \verb.|. ��� ������������ �����.

\emph{�������} ���������� ������������ ��������� ����� ����������
���������: \texttt{t}, \texttt{b} � \texttt{c} �������� ������������
�� �������� ����, ������� ���� ��� �� ������ ���������.

������ ��������� \texttt{tabular} ����~<<\verb|&|>> ��������� �
���������� �������, �������~\ci{\bs} �������� ����� ������, �
\ci{hline} ��������� �������������� �����.\index{"|@\verb."|.} ��
������ ��������� �������� ����� ��� ������ �������
\ci{cline}\verb|{j-i}|, ��� \texttt{j} � \texttt{i}~--- ������
��������, ��� �������� ������ ��������� �����.

\begin{example}
\begin{tabular}{|r|l|}
\hline
7C0 & ����������������� \\
3700 & ������������ \\ \cline{2-2}
11111000000 & �������� \\
\hline \hline
1984 & ���������� \\
\hline
\end{tabular}
\end{example}

\begin{example}
\begin{tabular}{|p{4.7cm}|}
\hline
����� ���������� � ����� �
�������. ��������, ��� ����
��� ����������.\\
\hline
\end{tabular}
\end{example}

����������� �������� ����� ������ ������������ \verb|@{...}|. ���
������� ������� ������ ����� ��������� � �������� ��� �� ��, ���
�������� � �������� ������. ���� �� ������ ������������� ���� �������
�������� ����, ��� �������� � �������� ������������ �� ����������
�����. ������ �������� �������������~--- ��� ���������� ��������
������� � ������� ��� ������ \verb|@{}|:

\begin{example}
\begin{tabular}{@{} l @{}}
\hline
��� �������� �������\\
\hline
\end{tabular}
\end{example}

\begin{example}
\begin{tabular}{l}
\hline
������� ������ ����� � ������\\
\hline
\end{tabular}
\end{example}

%
% This part by Mike Ressler
%

\index{������������!�� ���������� �����} ��������� ���������� ������
��������� �������� ������� �� ���������� �����
�����������,\footnote{���� �� ����� ������� ���������� ��������
  `tools', �������� �������� �� ����� \pai{dcolumn}.} �� �����
<<��������>> \TeX{} � �������� ����� ��� ������ ���� ��������:
����������� ������ ����� ����� � ����������� ����� �������. �������
\verb|@{.}| � ������ \verb|\begin{tabular}| �������� ���������� ������
  ����� ��������� ������ �� <<.>>, ����� ������ ������ �������,
  ������������ �� ���������� �����. �� �������� �������� � �����
  ������ ����� �� ����������� �������� (\verb|&|)! ����� ������� �����
  ��������� ��� ����� �������� <<��������>> �������� \ci{multicolumn}:

\begin{example}
\begin{tabular}{c r @{.} l}
��������� � $\pi$       &
\multicolumn{2}{c}{��������} \\
\hline
$\pi$               & 3&1416  \\
$\pi^{\pi}$         & 36&46   \\
$(\pi^{\pi})^{\pi}$ & 80662&7 \\
\end{tabular}
\end{example}

\begin{example}
\begin{tabular}{|c|c|}
\hline
\multicolumn{2}{|c|}{Ene} \\
\hline
Mene & Muh! \\
\hline
\end{tabular}
\end{example}

��������, ���������� � ��������� \texttt{tabular}, ������ ����������
�� ����� ��������. ���� ��� ����� �������� ������� �������,
����������� ��������� \pai{supertabular} ��� \pai{longtable}.

\section{��������� �������}
����������� ���������� � ���� ��� �������� ��������� ����������� �
������. ��� �������� ��������� � ����������� ��������� � ����, ��� ���
��� �� ����� ���� ������� ����� ����������. ����� �� ������� ���� ��
�������� ����� �������� ������ ���, ����� ����������� ����������� ���
�������, ������� �������, ����� ����������� �� ������� ��������. ����
������ ������ �� � ����, ��� �������� ���������� �� �������� �������,
��� ��������� ����� �����.

��� ������� ���� �������� ����� ����������� ��� �������, �� �����������
�� ������� ��������, ����� `�������', ����������� �� ���������
�������� � �������� ���������� ������� �������. \LaTeX{} ����������
��� ��������� ��������\index{��������� �������} ��� ���������, ���� ��� ������ � ���� ���
�����������. ����� ��������� ������������ �� ������������, �����
�������� ������������, ��� \LaTeX{} ������������ ���������
�������. ����� ��� ����� ����� ���������� ������������� ��-�� ����,
��� \LaTeX{} �������� �� �� ����, ���� �� ������.

������� ������� ���������� �������, ��������������� \LaTeX{} ���
��������� ��������.

����� ��������, ���������� � ��������� \ei{figure} ��� \ei{table},
���������� ��� ���������. ��� ��������� ����� ��������������
��������
\begin{lscommand}
\verb|\begin{figure}[|\emph{������������ ����������}\verb|]| ���\\
\verb|\begin{table}[|\emph{������������ ����������}\verb|]|
\end{lscommand}
\noindent ���������� \emph{������������� ����������}. ���� ��������
������������ ��� �������� \LaTeX{}, ���� ����� ���������� ���������
������. \emph{������������ ����������} �������������� ����� ���������
� ������� \emph{������ ���������� ���������� �������}. ��.
�������~\ref{tab:permiss}.

\begin{table}[!bp]
\caption{����� ���������� ���������� �������}\label{tab:permiss}
\noindent \begin{minipage}{\textwidth}
\medskip
\begin{center}
\begin{tabular}{@{}cp{10cm}@{}}
����&��������� �������� ������ \ldots\\
\hline
\rule{0pt}{1.05em}\texttt{h} & \emph{����� ��}, � ��� ����� �����
������, ��� �� ��������. ������ ������������ ��� ��������� ��������.\\[0.3ex]
\texttt{t} & \emph{�������} ��������\\[0.3ex]
\texttt{b} & \emph{�����} ��������\\[0.3ex]
\texttt{p} & �� \emph{����������� ��������}, ���������� ������
��������� �������.\\[0.3ex]
\texttt{!} & �� ������������� ����������� ����������
����������,\footnote{�����, ��� ������������ ����� ��������� ��������,
  ����������� �� ����� ��������} ������� ����� �������������
���������� ����� �������.
\end{tabular}
\end{center}
\end{minipage}
\end{table}

���������: \texttt{0pt} � \texttt{1.05em}~--- ������� ��������� ����
\TeX{}. ����������� � �������� � ������������ �������� �
�������~\ref{units} �� ��������~\pageref{units}.


��������, ������� ����� ������ ��������� �������:
\begin{code}
\verb|\begin{table}[!hbp]|
\end{code}
\noindent ������������ ���������� \index{������������ ����������} \verb|[!hbp]| ���������
\LaTeX{} ���������� ������� ����� �� ����� (\texttt{h}), ��� ����� ���
�� �������� (\texttt{b}), ��� �� ���������� �������� (\texttt{p}), �
��� ���~--- ���� ���� ��� ����� ���������� �� ��� �� ������
(\texttt{!}). ���� ������� ������������ ���������� �� ������,
����������� ������ ������������ \verb|[tbp]|.

\LaTeX{} ��������� ������ ����������� ��������� ������ � ������������
� �������� ������� �������������. ���� ������ ������ ��������� ��
������� ��������, �� �������������, ��������� � �������
�����������\index{�����������} ��� � �������
������.\index{�������}\footnote{��� ������� ����������� ����������
  \emph{fifo}: `������ �����~--- ������ �����'.} ����� ����������
����� ��������, \LaTeX{} ���������, ����� �� ��������� �����������
�������� ���������� ��������� �� ��������. ���� ���, �� ������ ������
�� ������ ������� ��������� ������ ��� ������������� � ������:
\LaTeX{} ����� �������� ���������� �� � ������������ � ��
�������������� (�� ����������� `h', ��� ��� ����������). �����
����������� � ������ ��������� ������� ���������� � ���������������
�������. \LaTeX{} ��������� �������, � ������� ����������� ���������
������� ���������������� ����. ������� �����������, ������� �� �������
����������, ����������� ��� ���������� ����������� � ����� ���������.
�������������:

\begin{quote}
  ���� \LaTeX{} �� ��������� ��������� �������, ��� �� ����� ��������,
  �� ����� ��� ������ ���� ������ ������� ����� � ����� �� ��������.
\end{quote}

���� � �������� �������� � \LaTeX{} ���������� ������������ ����������
���������� �������, ��� ����� ������� ��������. ���� ������ ��
���������� � ��������� �����, �� <<����������>>, �������� �����������
��������� �������. � ���������, ������� �� ����������� ����
\texttt{[h]}; ��� ��������� �����, ��� � ����������� ������� \LaTeX{},
�� ������������� ���������� \texttt{[ht]}.

\bigskip
\noindent ����� ���������� ���� ���������� �������� ��� ���������
��������� ��� ��������� \ei{table} � \ei{figure}. ��������

\begin{lscommand}
\ci{caption}\verb|{|\emph{����� ���������}\verb|}|
\end{lscommand}

\noindent �� ������ ������ ��������� ��� �������. ���������������
����� � ������ <<�������>> ��� <<�������>> ����������� \LaTeX{}.

��� �������

\begin{lscommand}
\ci{listoffigures} � \ci{listoftables}
\end{lscommand}

\noindent �������� ���������� ������� \verb|\tableofcontents|, �������
������ ����������� ��� ������, ��������������. � ���� �������
��������� ����������� �������. ���� �� ����������� ������� ���������,
�� �� ������ ������������ �� ������� ������� ��� ��������� � ������.
��� �������� ���������� �������� �������� � ���������� ������ �����
������� \verb|\caption|.
\begin{code}
\verb|\caption[��������]{��������������������������������������}|
\end{code}

��� ������ \verb|\label| � \verb|\ref| ����� ������ ������ �� ������
������ �� ��������� ������.

��������� ������ ������ ������� � ��������� ��� � ��������. ��������
������� ����� ������������, ����� �������� � ��������� ����� ���
�����������, ������� �� �������� �����.

\begin{code}
\begin{Verbatim}
�������~\ref{white} �������� �������� ���-����.
\begin{figure}[!hbp]
\makebox[\textwidth]{\framebox[5cm]{\rule{0pt}{5cm}}}
\caption{���� �� ���� �����������} \label{white}
\end{figure}
\\\end{Verbatim}
\end{code}

\noindent � ���� ������� \LaTeX{} ����� \emph{����� ������}~(\texttt{!}) ���������
���������� ����������� ����� \emph{��
  �����}~(\texttt{h}).\footnote{�����������, ��� ������� �����������
  �����} ���� ��� ����������, �� ���������� ���������� �� \emph{�����
  ��������}~(\texttt{b}). ���� ��� �� ������� ��������� ����������� ��
������� ��������, �� �������, ����� �� ������� �������� ���������
��������, ���������� ��� ����������� �, ��������, ��������� ������� ��
������� ������. ���� ��� ��������� �������� ��������� ��� ��
����������, \LaTeX{} �������� ����� �������� � ����� �������������
�����������, ��� ���� �� ��� ������ ��� ��������� � ������.

� ������������ ������� ����� ���� ���������� ������������ �������
\begin{lscommand}
\ci{clearpage} ��� ���� \ci{cleardoublepage}
\end{lscommand}

\noindent ��� ���������  \LaTeX{} ���������� ���������� ��� ���������
�������, ������������ � ��������, � ����� ������ ����� ��������.
\ci{cleardoublepage}, ������ �����, �������� ����� ��������������
��������.

����� �� �������, ��� �������� � ���� ��������� \LaTeX{} ������� �
������� \textsc{PostScript}.


\section{������ ������� ������}

�����, �������� � ���������� ������ ��������� \ci{caption} ���
\ci{section}, ���������� � ��������� ������ ������ ���� (��������, �
����������, � ������������ � � ���� ���������). ��������� ������� ��
��������, ������ ������������ � ���������� ������ ���� \ci{section}.
�� �������� �������� ���������\index{�������!�������}. � ���������,
�������� �������� ������� \ci{footnote} ��� \ci{phantom}. ��� ����,
����� ��� �������� ��� ��������������, ���������� ����� ���� ���������
������� \ci{protect}.

������� \ci{protect} ��������� ������ � �������, ���������������
��������� �� ���; ���� �� � �� ����������. � ����������� �������
������ ������� \ci{protect} �� ��������.

\begin{code}
\verb|\section{� ����������|\\
\verb|      \protect\footnote{� ������� ���� ������}}|
\end{code}

%%% Local Variables:
%%% mode: latex
%%% TeX-master: "lshort"
%%% End:

%%%%%%%%%%%%%%%%%%%%%%%%%%%%%%%%%%%%%%%%%%%%%%%%%%%%%%%%%%%%%%%%
% Contents: Math typesetting with LaTeX
% $Id: math.tex,v 1.1.1.1 2002/02/26 10:04:21 oetiker Exp $
%%%%%%%%%%%%%%%%%%%%%%%%%%%%%%%%%%%%%%%%%%%%%%%%%%%%%%%%%%%%%%%%%
\chapter{����� �������������� ������}

\begin{intro}
  ��� ������ �� ������! � ���� ����� �� ���������� � �������� �����
  \TeX{}: �������������� ��������. �� ������ � ����, ��� ��� �����
  ���� ������ ������������� �����. ���� ��� ������ �� ��� ����������
  ����� ����� ����� ����������, �� ������������, ���� �� �� �������
  ����� �������, ���������� ������ ������� ����� ����������. ������
  ��������, ��� ���� �������� �������� �
  \AmS-\LaTeX{}\footnote{\emph{������������ �������������� ��������}
    ��p������� p������� p����p���� � \LaTeX{}. ������ �p���p� ����
    ����� ���������� ��� p����p����, �������� �� ��� ���p�������
    ����p������� \TeX{}. ���� � ����� ��� �����������, �� ������
    �������� ��� �� ��p��� \CTANref|macros/latex/required/amslatex|.}.
\end{intro}


\section{����� ��������}

\LaTeX{} �������� � ���� ����������� ����� ��� ������� ����������.
\index{����������} ���������� ����� ���� ���p��� ����p� ������, ��
����� � p�������� ����� ���������� ��p�����. �������������� �����
������ ������ �������� ����� \ci{(}
� \ci{)}, \index{$@\texttt{\$}} %$
����� \texttt{\$} � \texttt{\$} ��� �����
\verb|\begin{|\ei{math}\verb|}| ��� \verb|\end{math}|.\index{�������}

\begin{example}
  ��������� $a$ � �������� �
  $b$ � ��������, ��������
  $c$ � ��������. ���
  ������� ������ ����������:
  $c^{2}=a^{2}+b^{2}$
\end{example}

\begin{example}
  \TeX{} ������������ ���
  \(\tau\epsilon\chi\).\\[6pt]
  100~�$^{3}$ ����.\\[6pt]
  ��� ������� �� �����
  \begin{math}\heartsuit\end{math}
\end{example}

������� �������������� ��������� ��� ������� ����������������
<<���������>>, �� ���� �������� �� �� ��������� ��������. ��� �����
���������� �� ����� \ci{[} � \ci{]} ��� �����
\verb|\begin{|\ei{displaymath}\verb|}| � \verb|\end{displaymath}|. 

\begin{example}
  ��������� $a$ � �������� �
  $b$ � ��������, ��������
  $c$ � ��������. ���
  ������� ������ ����������:
  \begin{displaymath}
    c^{2}=a^{2}+b^{2}
  \end{displaymath}
  ��� �� ������ ��p����� ���
  ��p���: \[a+b=c\]
\end{example}

���� �� ������, ����� \LaTeX{} ����p���� ���� �p�������, �����������
��p������ \ei{equation}. �� ������ �p� ���� �������� �p������� ������
\ci{label} � ��������� �� ���� � ����� ����� ������ ��������� \ci{ref}
��� \ci{eqref}:

\begin{example}
  \begin{equation}
    \label{eq:eps}
    \epsilon > 0
  \end{equation}
  �� (\ref{eq:eps})
  ������� \ldots{}�� 
  \eqref{eq:eps} �� 
  ������ �� ��.
\end{example}

�������� p������ � ����� ��p���� ��p������ � ������� � ���������:

\begin{example}
  $\lim_{n \to \infty}
  \sum_{k=1}^n \frac{1}{k^2}
  = \frac{\pi^2}{6}$
\end{example}

\begin{example}
  \begin{displaymath}
    \lim_{n \to \infty}
    \sum_{k=1}^n \frac{1}{k^2}
    = \frac{\pi^2}{6}
  \end{displaymath}
\end{example}

\emph{�������������� �����} ���������� �� \emph{����������
  ������}. ��������, � \emph{�������������� ������}:

\begin{enumerate}

\item ����������� �������� � ��������� ������� �� ����������� ��
  ��������, ��� ��� ��� ������� ���� ��������� �� ������
  �������������� ���������, ��� ������ � ����� ���� ����������
  ��������� ����� \ci{,}, \ci{quad} ��� \ci{qquad}.
\item ������ ������� �����������. ������ ������� �������� ������ ����
  �����.
\item ������ ����� ��������� ������ ����������, � ���������� � ����
  ��������. ���� �� ������ � ������� ������ ���������� �����
  (���������� ������ ����� � ����������� ���������), �� ��� �����
  ������� ��� ��������� \verb|\textrm{...}| (��. �����
  p�����~\ref{sec:fontsz} �� ��p.~\pageref{sec:fontsz}).

\end{enumerate}


\begin{example}
  \begin{equation}
    \forall x \in \mathbf{R}:
    \qquad x^{2} \geq 0
  \end{equation}
\end{example}

\begin{example}
  \begin{equation}
    x^{2} \geq 0\qquad
    \textrm{��� ���� }x\in
    \mathbf{R}
  \end{equation}
\end{example}

%
% Add AMSSYB Package ... Blackboard bold .... R for realnumbers
%

���������� ������ ����� ������ � ������������ ��������: ����� �����
������ ������������ `\wi{������� ���������� �������}',\index{���������� �������}
������� ���������� �������� \ci{mathbb} �� �������
\pai{amsfonts} ��� \pai{amssymb}.  \ifx\mathbb\undefined\else
��������� ������ ������ �������� ���:
\begin{example}
  \begin{displaymath}
    x^{2} \geq 0\qquad
    \textrm{��� ���� }x\in
    \mathbb{R}
  \end{displaymath}
\end{example}
\fi


\section{����������� � �������������� ������}

����������� ������ ��������������� ������ ��������� ������ ��
��������� ������. ��� ���, ���� �� ������, ����� ������� ������ ��
��������� ��������, ��� ����� ������������� �� ������ ��� ������
�������� ������: \verb|{...}|.

\begin{example}
  \begin{equation}
    a^x+y \neq a^{x+y}
  \end{equation}
\end{example}


\section{������������ �������������� �������}

� ���� ������� ����� ������� �������� ������ �������, ������������ �
�������������� �������. ��������� �������� ������ ��� ������
�������������� �������� �������� � �������~\ref{symbols} ��
��������~\pageref{symbols}.

\textbf{�������� \wi{��������� �����}} �������� ��� \verb|\alpha|,
 \verb|\beta|, \verb|\gamma|,~\ldots, ��������� ����� �������� ���
 \verb|\Gamma|, \verb|\Delta|,~\ldots\footnote{� \LaTeXe{} ��
   ������������ ��������� <<�����>>, ������ ��� ��� �������� ��� ��,
   ��� ��������� <<A>>. ��� ����� ��������� ���������� ��� �����
   ��������.}

\begin{example}
  $\lambda,\xi,\pi,\mu,%
  \Phi,\Omega$
\end{example}

\textbf{������� � ������ �������} �������� ��� ������
��������~<<\index{������� �������}\index{������
  �������}\verb|^|\index{^@\verb+^+}>>
�~<<\verb|_|\index{_@\verb+_+}>>.

\begin{example}
  $a_{1}$ \qquad $x^{2}$ \qquad
  $e^{-\alpha t}$ \qquad
  $a^{3}_{ij}$\\
  $e^{x^2} \neq {e^x}^2$
\end{example}

\textbf{���������� ������}\index{���������� ������} �������� ���
\ci{sqrt}, ������ $n$-��� ������� ���������� ��� ������
\verb|\sqrt[|$n$\verb|]|. ������ ����� ����� ���������� \LaTeX{}
�������������. ���� ����� ���� ������ ����, ����������� \verb|\surd|.

\begin{example}
  $\sqrt{x}$ \qquad
  $\sqrt{ x^{2}+\sqrt{y} }$
  \qquad $\sqrt[3]{2}$\\[3pt]
  $\surd[x^2 + y^2]$
\end{example}

������� \ci{overline} � \ci{underline} ������� \textbf{��������������
  �����} ����� ��� ��� ��� ����������.
\index{�����!��������������}

\begin{example}
  $\overline{m+n}$
\end{example}

������� \ci{overbrace} � \ci{underbrace} �������
������� \textbf{�������������� �������� ������} ����� ��� ��� ���
����������.
\index{�������� ������!��������������}
\begin{example}
  $\underbrace{ a+b+\cdots%
    +z }_{26}$
\end{example}

\index{������!��������������} ��� ���������� � ���������� ������
�������������� ��������, �����, ��� ��������� ������� ��� �����
\wi{������}, �� ������ ������������ ���������, ������������� �
�������~\ref{mathacc}. ������� <<������>> � ������, ������������
��������� ��������, ������������ ��������� \ci{widetilde} �
\ci{widehat}. ������~<<\verb|'|\index{'@\verb+'+}>> ���� ����
�����������.\index{�����������}.

% a dash is --

\begin{example}
  \begin{displaymath}
    y=x^{2}\qquad y'=2x
    \qquad y''=2
  \end{displaymath}
\end{example}

\textbf{�������}\index{�������} ����� ����������� �����������
��������� ������� \wi{�������} ��� ����������. ��� �������� ��������
\ci{vec}. ��� ����������� ������� �� $A$ �� $B$ ������� ��� �������
\ci{overrightarrow} � \ci{overleftarrow}.

\begin{example}
  \begin{displaymath}
    \vec a\quad
    \overrightarrow{AB}
  \end{displaymath}
\end{example}

������ ���� �����, ������������ ���������, ���� �� ����������. ������,
������ �� �������, ����� ������ �������� �������������
�������. ����������� ��� ����� \ci{cdot}:

\begin{example}
\begin{displaymath}
v = {\sigma}_1 \cdot {\sigma}_2
    {\tau}_1 \cdot {\tau}_2
\end{displaymath}
\end{example}


����� ������� ���� $\lg$ ����� ���������� ������ �������, � ��
��������, ��� ����������. ������� \LaTeX{} �������� ��������� �������
��� ������ ���� �������� ������ �������:
\index{��������������!�������}

\begin{tabular}{lllllll}
\ci{arccos} &   \ci{cos} &    \ci{csc} &   \ci{exp} &   \ci{ker}
&     \ci{limsup} &  \ci{min} \\
\ci{arcsin} &   \ci{cosh} &   \ci{deg} &   \ci{gcd} &   \ci{lg} &
\ci{ln} &      \ci{Pr} \\
\ci{arctan} &   \ci{cot} &    \ci{det} &   \ci{hom} &   \ci{lim}
&     \ci{log} &     \ci{sec} \\
\ci{arg} &      \ci{coth} &   \ci{dim} &   \ci{inf} &
\ci{liminf} &  \ci{max} &     \ci{sin} \\
\ci{sinh} & \ci{sup} & \ci{tan} & \ci{tanh}\\
\end{tabular}

\begin{example}
\[\lim_{x \rightarrow 0}
\frac{\sin x}{x}=1\]
\end{example}

��� ������� ������ \wi{������� ������} ���� ��� �������: \ci{bmod} ���
��������� ��������� <<$a \bmod b$>> � \ci{pmod} ��� ��������� �����
<<$x\equiv a \pmod{b}$>>.

\begin{example}
$a\bmod b$\\
$x\equiv a \pmod{b}$
\end{example}

������������ \textbf{\wi{�����}} ���������� ��������
\ci{frac}\verb|{...}{...}|. ����� ���������������� �� ����� � �����
������ $1/2$, ������ ��� ��� ��������� ����� ��� ��������� ����������
`�������� ���������'.

\begin{example}
  $1\frac{1}{2}$~����
  \begin{displaymath}
    \frac{ x^{2} }{ k+1 }\qquad
    x^{ \frac{2}{k+1} }\qquad
    x^{ 1/2 }
  \end{displaymath}
\end{example}

��� ������� ������������ ������������� ��� ����������� �������� �����
������������ �������� \ci{binom} �� ������ \pai{amsmath}.

\begin{example}
\begin{displaymath}
\binom{n}{k}\qquad\mathrm{C}_n^k
\end{displaymath}
\end{example}

��� �������� ��������� ������ ������ ��������� ������� ���� ���
������. ������� \ci{stackrel} �������� ������, �������� ������
����������, ������� ������� �������� � ��������� ��� ��� ������
����������, ���������� � ������� �������:

\begin{example}
\begin{displaymath}
\int f_N(x) \stackrel{!}{=} 1
\end{displaymath}
\end{example}


\textbf{�������� ���������}\index{��������!���������} ��������
�������~\ci{int}, \textbf{�������� �����}\index{��������!�����}~---
�������~\ci{sum}, \textbf{��������
  ������������}\index{��������!������������}~--- �������~\ci{prod}.
������� � ������ ������� ����������� ��� ������ ������~<<\verb|^|>>
�~<<\verb|_|>>, ��� ��, ��� ������� � ������
�������\footnote{\AmS-\LaTeX{}, ����� ����, ������������ �������������
������� � ������ �������.}.

\begin{example}
\begin{displaymath}
\sum_{i=1}^{n} \qquad
\int_{0}^{\frac{\pi}{2}} \qquad
\prod_\epsilon
\end{displaymath}
\end{example}

����� �������� ������� ����p��� ��� p���������� �������� � �������
��p�������, \pai{amsmath} �p����������� ��� ��� ����p������: �������
\ci{substack} � ��p������ \ei{subarray}:

\begin{example}
\begin{displaymath}
\sum_{\substack{0<i<n \\ 1<j<m}}
   P(i,j) =
\sum_{\begin{subarray}{l} i\in I\\
         1<j<m
      \end{subarray}}     Q(i,j)
\end{displaymath}
\end{example}

\medskip

��� \index{������}\textbf{������} � ������
�������������\index{������������} � \TeX{} ���������� ���������
�������� (������, $[\;\langle\;\|\;\updownarrow$). ������� �
���������� ������ ����� ������� ���������������� ���������, ��������
������~---\verb|\{|, ������ ������������~--- ������������ ���������
(��������, \verb|\updownarrow|). ������ ��������� �������������
�������� � �������~\ref{tab:delimiters} ��
��������~\pageref{tab:delimiters}.

\begin{example}
\begin{displaymath}
{a,b,c}\neq\{a,b,c\}
\end{displaymath}
\end{example}

���� �� ��������� ����� ����������� ������������� ������� \ci{left},
��� ����� �����������~--- \ci{right}, �� \TeX{} ������������� �������
���������� ������ ������������. ��������, ��� �� ������ ������
\ci{left} ��������� ��������������� \ci{right}, � ��� ������
������������ ��������� ������ ���� ��� ��� ������� �� ����� ������.
���� �� �� ������ ����� ������� ������������, ����������� ���������
������������ `\ci{right.}'!

\begin{example}
\begin{displaymath}
1 + \left( \frac{1}{ 1-x^{2} }
    \right) ^3
\end{displaymath}
\end{example}

� ��������� ������� ���������� ������� ���������� ������
��������������� ������������\index{��������������!������������}
�������, ��� ����� ���� ������� \ci{big}, \ci{Big}, \ci{bigg} �
\ci{Bigg}, �������� ���������� � ����������� ������
�������������.\footnote{��� ������� �� �������� ��� ���������, ����
  ������������ ������� ����� ������� ������, ��� ���� ������� �����
  \texttt{11pt} ��� \texttt{12pt}. ��� ��������������� ����� ���������
  ����������� �������� \pai{exscale} ��� \pai{amsmath}.}

\begin{example}
$\Big( (x+1) (x-1) \Big) ^{2}$\\
$\big(\Big(\bigg(\Bigg($\quad
$\big\}\Big\}\bigg\}\Bigg\}$\quad
$\big\|\Big\|\bigg\|\Bigg\|$
\end{example}

����� ������ � ������� \textbf{\wi{��� �����}}, ���� ��������� ������.
\ci{ldots} �������� ����� �� ������� �����, \ci{cdots}~---
��������������. ����� ����, ���������� ������� \ci{vdots} ���
������������ � \ci{ddots} ��� ������������
�����.\index{�����!������������}\index{�����!��������������}\index{�����!������������}
� �������~\ref{sec:vert} �� ������� ������ ������.

\begin{example}
\begin{displaymath}
x_{1},\ldots,x_{n} \qquad
x_{1}+\cdots+x_{n}
\end{displaymath}
\end{example}


\section{�������������� �������}

\index{��������������!������} ���� ��������� \TeX{} ������� ������
������ �������������������, �� ������ �� ������������ � ��������������
������ ���������� ���������. ������� ��� ��������� ��������: \ci{,}
��� $\frac{3}{18}\:\textrm{quad}$ (\demowidth{0.166em}), \ci{:} ���
$\frac{4}{18}\: \textrm{quad}$ (\demowidth{0.222em}) � \ci{;} ���
$\frac{5}{18}\: \textrm{quad}$ (\demowidth{0.277em}). ��������������
������ ������� \verb*.\ . ���� ������� �������� ������, � \ci{quad}
(\demowidth{1em}) � \ci{qquad} (\demowidth{2em}) ���� ������� �������.
������ \ci{quad} �������� ������������� ������ ����� `M' � �������
������.  �������~\verb|\!|\cih{"!} ���������� ������������� ������
�������� $-\frac{3}{18}\:\textrm{quad}$ (\demowidth{0.166em}).

\begin{example}
\newcommand{\ud}{\mathrm{d}}
\begin{displaymath}
\int\!\!\!\int_{D} g(x,y)
  \, \ud x\, \ud y
\end{displaymath}
������
\begin{displaymath}
\int\int_{D} g(x,y)\ud x \ud y
\end{displaymath}
\end{example}

��������, ��� `d' � ������������� ������ ���������� ������ �������.

\AmS-\LaTeX{} �������� ������ ������ ������ ���������� �������� �����
����������� ������� ����������: ������� \ci{iint}, \ci{iiint},
\ci{iiiint} � \ci{idotsint}. � ����������� ������� \pai{amsmath}
���������� ������ ����� �������� ���:

\begin{example}
\newcommand{\ud}{\mathrm{d}}
\begin{displaymath}
\iint_{D} \, \ud x \, \ud y
\end{displaymath}
\end{example}

������ �������� � ����������� ��������� textmath.tex (����������������
� \AmS-\LaTeX{}) ��� � ����� 8 \companion{}.

\section{����������� ������������� ��������}
\label{sec:vert}

��� ������� \textbf{������} ����������� ���������� \ei{array}. ���
������ ���������� ��������� \ei{tabular}. ��� ������� ������
������������ ������� \verb|\\|.

\begin{example}
\begin{displaymath}
\mathbf{X} =
\left( \begin{array}{ccc}
x_{11} & x_{12} & \ldots \\
x_{21} & x_{22} & \ldots \\
\vdots & \vdots & \ddots
\end{array} \right)
\end{displaymath}
\end{example}

��������� \ei{array} ����� ����� ������������ ��� ������� ���������,
������� ���� ������� ������������, ���������� <<\verb|.|>> � ��������
���������� ������� ������������:

\begin{example}
\begin{displaymath}
y = \left\{ \begin{array}{ll}
 a & \textrm{���� $d>c$}\\
 b+x & \textrm{�� �����}\\
 l & \textrm{��������� ����� ���}
  \end{array} \right.
\end{displaymath}
\end{example}

��� ��, ��� � ��������� \verb|tabular|, ����� �������� ������� �
��������� \ei{array}, ��������, �������� �������� �������:

\begin{example}
\begin{displaymath}
\left(\begin{array}{c|c}
 1 & 2 \\
\hline
3 & 4
\end{array}\right)
\end{displaymath}
\end{example}


��� ������, ���������� ��������� ����� ��� ��� ������
���������\index{������� ���������}
������ \ei{equation} ����������� ����������� \ei{eqnarray} �
\verb|eqnarray*|. � \ei{eqnarray} ������ ������ �������� ���������
����� ���������. � \verb|eqnarray*| ������ �� ��������.

��������� \ei{eqnarray} � \verb|eqnarray*| �������� ��������� �������
�� ���� �������� ������� \verb|{rcl}|, ��� ������� �������
������������ ��� ����� ���������, ��� ����� �����������, ��� �������
����������� �����. ������� \verb|\\| ��������� ������.

\begin{example}
\begin{eqnarray}
f(x) & = & \cos x     \\
f'(x) & = & -\sin x   \\
\int_{0}^{x} f(y)dy &
 = & \sin x
\end{eqnarray}
\end{example}

��������, ��� �� ����� �������� ������� �������, ������ ���������,
������� ����� ���������� �����. ��� ����� ���� ��������� ����������
\verb|\setlength\arraycolsep{2pt}|, ��� � ��������� �������.\sloppypar

\index{������� ���������} \textbf{������� ���������} �� �����
������������� ����������� �� ���������� �����. ����� ������ �������,
��� �� ��������� � ��������� �����������. ���� ����� ��� �����
���������� ��������� ������:

\begin{example}
{\setlength\arraycolsep{2pt}
\begin{eqnarray}
\sin x & = & x -\frac{x^{3}}{3!}
     +\frac{x^{5}}{5!}-{}
                    \nonumber\\
 & & {}-\frac{x^{7}}{7!}+{}\cdots
\end{eqnarray}}
\end{example}

\begin{example}
\begin{eqnarray}
\lefteqn{ \cos x = 1
     -\frac{x^{2}}{2!} +{} }
                    \nonumber\\
 & & {}+\frac{x^{4}}{4!}
     -\frac{x^{6}}{6!}+{}\cdots
\end{eqnarray}
\end{example}

\noindent ������� \ci{nonumber} ���������� \LaTeX{} �� ������������
����� ��� ����� ���������.

������ �������� ����� ���� ������ �������� ��������� ����������
����������� ����������� ���������; ����� ������ ������������
������������� ����� \pai{amsmath} (��. ��������� \verb|align|,
\verb|flalign|, \verb|gather|, \verb|multiline| � \verb|split|).

\section{�������}

�� �� ����� ������� �p��������, �� ���, ��� �� �����, ��� p����
�������� ���� ����� � ���� ������p�� �����. ��� � \LaTeX{} ���������
������������ �������, ��������� �������, ��� ���������� ����������
������ � ����������� ������� ��������.

����� \LaTeX{} ��������� ����� �� ��������� ��� ������ ������ \verb|^|
� \verb|_|, �� ������ ��������� ��������� ����������. ��������
\ci{phantom} �� ������ ��������������� ������������ ��� ��������,
������� �� ����� ���� ��������� �� �����. ����� ����� ��� ������ ��
��������� ��������.

\begin{example}
\begin{displaymath}
{}^{12}_{\phantom{1}6}\textrm{C}
\qquad \textrm{versus} \qquad
{}^{12}_{6}\textrm{C}
\end{displaymath}
\end{example}

\begin{example}
\begin{displaymath} 
\Gamma_{ij}^{\phantom{ij}k}
\qquad \textrm{versus} \qquad
\Gamma_{ij}^{k}
\end{displaymath}  
\end{example}

\section{������ ��������������� ������}\label{sec:fontsz}

\index{�����!��������������, ������} � �������������� ������ \TeX{}
�������� ������ ������ � ����������� �� ���������. �������, ��������,
���������� ������� �������. ���� �� ������ �������� � ���������
������� �����, �� ����������� �������� \verb|\textrm|, ��� ���
�������� ������������ ������� �������� �� �����, ������ ���
\verb|\textrm| �������� ������� � ��������� �����. ����� �������� ���
����������, ����������� ������� \verb|\mathrm|.\trfootnote{�
  ����������� �� ������������ �����������, � ��� ����� �� ��������
  ������� ����� ������ ������� \ci{mathrm}, ������ ������� �����
  �������������� ������� \ci{cyrmathrm}.} �� ������ � ����,
\ci{mathrm} ����� ������ �������� ������ � ��������� ����������.
������� ��-�������� �� ������� � ��������������� ������� ��
��������.\footnote{��� ����������� \AmS-\LaTeX{} (����� \pai{amsmath})
  ������� \ci{textrm} �������� �������� � ���������� �������.}

\begin{example}
\begin{equation}
2^{\textrm{nd}} \quad
2^{\mathrm{nd}}
\end{equation}
\end{example}

��� �� �����, ������ ��� ����� ���� ����� ������� \LaTeX{} ������
������ ������. � �������������� ������ ������ ��������������� ��������
���������:

\begin{flushleft}
  \ci{displaystyle}~($\displaystyle 123$),
  \ci{textstyle}~($\textstyle 123$),
  \ci{scriptstyle}~($\scriptstyle 123$) �
  \ci{scriptscriptstyle}~($\scriptscriptstyle 123$).
\end{flushleft}

����� ������ ������ ����� �� ������ ����������� ��������.

\fvset{xrightmargin=0.52\textwidth}

\begin{example}
\begin{displaymath}
\mathop{\mathrm{corr}}(X,Y)=
 \frac{\displaystyle
   \sum_{i=1}^n(x_i-\overline x)
   (y_i-\overline y)}
  {\displaystyle\biggl[
 \sum_{i=1}^n(x_i-\overline x)^2
\sum_{i=1}^n(y_i-\overline y)^2
\biggr]^{1/2}}
\end{displaymath}
\end{example}
% This is not a math accent, and no maths book would be set this way.
% mathop gets the spacing right.

\fvset{xrightmargin=0.5\textwidth}

\noindent ��� ���� �� ��������, ����� ��� ����� ������ �\'������, ���
��������������� ������������  \verb|\left[  \right]|.


\section{�������, ������, \ldots{}}

��� ��������� �������������� ����������, ���, ��������, ����� ������
������� <<����>>, <<�����������>>, <<������>> � ����������� ��������.
\LaTeX{} ������������ ��� ���������
\begin{lscommand}
\ci{newtheorem}\verb|{|\emph{��������}\verb|}[|\emph{�������}\verb|]{|%
         \emph{�����}\verb|}[|\emph{������}\verb|]|
\end{lscommand}
�������� \emph{��������}~--- ��� ������� �������� �����, ������������
��� ������������� <<�������>>. ���������� \emph{�����} �� �����������
��������� �������� <<�������>>, ��� ������� ��� ����� ���������� �
���������.

��������� � ���������� ������� �������������. ��� ��� ������������ ���
����������� ����, ��� ���������� <<�������>>. ����������
\emph{�������} �� ������ ������� \emph{��������} ��������������
����������� <<�������>>. ����� <<�������>> ����� ����� ������������ �
��� �� ������������������. �������� \emph{������} ��������� ���
������� ������, ������ �������� �� ������ ���������� ����
<<�������>>.

����� ������������� � ��������� ��������� ������� \ci{newtheorem}, ��
������ ������������ ���������� ���������:

\begin{code}
\verb|\begin{|\emph{��������}\verb|}[|\emph{�����}\verb|]|\\
��� ���������� �������.\\
\verb|\end{|\emph{��������}\verb|}|
\end{code}

�� ���� ������ ������ ���� ����������. ���������� ������� ������
�������� ��������� ���� ��������, � ������������ ������� ���, ���
��������� \verb|\newtheorem| ������� ������, ����� ��� ����� ����
������:\sloppypar

\begin{example}
% ����������� ���
% ��������� ���������
\newtheorem{law}{Law}
\newtheorem{jury}[law]{Jury}
% � ���� ���������
\begin{law} \label{law:box}
Don't hide in the witness box
\end{law}
\begin{jury}[The Twelve]
It could be you! So beware and
see law~\ref{law:box}\end{jury}
\begin{law}No, No, No\end{law}
\end{example}

������� <<Jury>> ���������� ��� �� �������, ��� � �������
<<Law>>. �������������, ��� ������� ����� � ������������������ �
������� ��������� <<Law>>. �������� � ���������� ������� ���������
��������� �������, ��� ����� �����������.

\begin{example}
\flushleft
\newtheorem{mur}{Murphy}[section]
\begin{mur}
���� ���������� ��� ���
����� ������� �������
�����, � ���� �� ����
�������� ����� ��������
� ����������, �� ���-��
����������� ��� �������.
\end{mur}
\end{example}

������� <<Murphy>> �������� �����, ��������� � ������� ��������
�������. �� ������ ����� ������������ ������ ����������� �������,
��������, ����� ��� ���������.


\section{���������� �������}
\index{���������� �������}

� \LaTeX{} �������� �������� �������� ������ �������; ���, ��������,
������� �������������, ������ ��� ��������������� ������� �����
�������������� ���. ������� ����� ������ \verb|\mathbf| ����
���������� �������, �� ��� ������� (������), ����� ��� ��������������
������� ������ ���������. ���������� ������� \ci{boldmath}, ��
\emph{��� ����� �������������� ������ ��� ��������������� ������}. ��
�� ��������� � � ��������.

\begin{example}
\begin{displaymath}
\mu, M \qquad \mathbf{M} \qquad
\mbox{\boldmath $\mu, M$}
\end{displaymath}
\end{example}

\noindent ��������, ��� ������� ���� ����������, ��� ����� ����
�������������.

����� \pai{amsbsy} (���������� ������� \pai{amsmath}), ����� ��� �
����� \pai{bm} (�� ������ \texttt{tools}), �������� �������
\ci{boldsymbol}.

\ifx\boldsymbol\undefined\else
\begin{example}
\begin{displaymath}
\mu, M \qquad
\boldsymbol{\mu}, \boldsymbol{M}
\end{displaymath}
\end{example}
\fi




%%% Local Variables:
%%% mode: latex
%%% TeX-master: "lshort"
%%% End:



%%%%%%%%%%%%%%%%%%%%%%%%%%%%%%%%%%%%%%%%%%%%%%%%%%%%%%%%%%%%%%%%%
% Contents: TeX and LaTeX and AMS symbols for Maths
% $Id: lssym.tex,v 1.1.1.1 2002/02/26 10:04:21 oetiker Exp $
%%%%%%%%%%%%%%%%%%%%%%%%%%%%%%%%%%%%%%%%%%%%%%%%%%%%%%%%%%%%%%%%%

\section{������ �������������� ��������}
\label{symbols}

� ��������� �������� �� ������� ��� �������, ��������� ������ �
\emph{�������������� ������}.


%
% Conditional Text in case the AMS Fonts are installed
%
��� ������� � ��������, ������������� �
��������~\ref{AMSD}--\ref{AMSNBR}\footnote{��� ������� ���� ��������
  �� \texttt{symbols.tex} (����� David~Carlisle) � ����� ������
  �������� �� ������ Josef~Tkadlec} � ��������� ��������� ������ ����
�������� ����� \pai{amssymb}, � � ������� ������ ���� �����������
�������������� ������ AMS. ���� ������ � ������ AMS � ����� ������� ��
�����������, ���������� ��
\texttt{\CTAN|macros/latex/required/amslatex|}. ��� �����
������ ��p����� �������� ����� ����� �� ��p���
\texttt{\CTANref|info/symbols/comprehensive|}.

\begin{table}[!h]
\caption{������� ��������������� ������}  \label{mathacc}
\begin{symbols}{*4{cl}}
\W{\hat}{a}     & \W{\check}{a} & \W{\tilde}{a} & \W{\acute}{a} \\
\W{\grave}{a} & \W{\dot}{a} & \W{\ddot}{a}    & \W{\breve}{a} \\
\W{\bar}{a} &\W{\vec}{a} &\W{\widehat}{A}&\W{\widetilde}{A}\\
\end{symbols}
\end{table}

\begin{table}[!h]
\caption{�������� ��������� �����}
\begin{symbols}{*4{cl}}
 \X{\alpha}     & \X{\theta}     & \X{o}          & \X{\upsilon}  \\
 \X{\beta}      & \X{\vartheta}  & \X{\pi}        & \X{\phi}      \\
 \X{\gamma}     & \X{\iota}      & \X{\varpi}     & \X{\varphi}   \\
 \X{\delta}     & \X{\kappa}     & \X{\rho}       & \X{\chi}      \\
 \X{\epsilon}   & \X{\lambda}    & \X{\varrho}    & \X{\psi}      \\
 \X{\varepsilon}& \X{\mu}        & \X{\sigma}     & \X{\omega}    \\
 \X{\zeta}      & \X{\nu}        & \X{\varsigma}  & &             \\
 \X{\eta}       & \X{\xi}        & \X{\tau}
\end{symbols}
\end{table}

\begin{table}[!h]
\caption{��������� ��������� �����}
\begin{symbols}{*4{cl}}
 \X{\Gamma}     & \X{\Lambda}    & \X{\Sigma}     & \X{\Psi}      \\
 \X{\Delta}     & \X{\Xi}        & \X{\Upsilon}   & \X{\Omega}    \\
 \X{\Theta}     & \X{\Pi}        & \X{\Phi}
\end{symbols}
\end{table}
\clearpage

\begin{table}[!tbp]
\caption{�������� ���������}
\bigskip
�� ������ �������� ��������������� ��������� ����������� �����
���������� ��������� ������� \ci{not}.
\begin{symbols}{*3{cl}}
 \X{<}           & \X{>}           & \X{=}          \\
 \X{\leq}���  \verb|\le|   & \X{\geq}��� \verb|\ge|   & \X{\equiv}     \\
 \X{\ll}         & \X{\gg}         & \X{\doteq}     \\
 \X{\prec}       & \X{\succ}       & \X{\sim}       \\
 \X{\preceq}     & \X{\succeq}     & \X{\simeq}     \\
 \X{\subset}     & \X{\supset}     & \X{\approx}    \\
 \X{\subseteq}   & \X{\supseteq}   & \X{\cong}      \\
 \X{\sqsubset}$^1$ & \X{\sqsupset}$^1$ & \X{\Join}$^1$    \\
 \X{\sqsubseteq} & \X{\sqsupseteq} & \X{\bowtie}    \\
 \X{\in}         & \X{\ni}, \verb|\owns|  & \X{\propto}    \\
 \X{\vdash}      & \X{\dashv}      & \X{\models}    \\
 \X{\mid}        & \X{\parallel}   & \X{\perp}      \\
 \X{\smile}      & \X{\frown}      & \X{\asymp}     \\
 \X{:}           & \X{\notin}      & \X{\neq}��� \verb|\ne|
\end{symbols}
\centerline{\footnotesize $^1$��� ������� � ����� ������� �����������
  ������� \textsf{latexsym}.}
\end{table}

\begin{table}[!tbp]
\caption{�������� ���������}
\begin{symbols}{*3{cl}}
 \X{+}              & \X{-}              & &                 \\
 \X{\pm}            & \X{\mp}            & \X{\triangleleft} \\
 \X{\cdot}          & \X{\div}           & \X{\triangleright}\\
 \X{\times}         & \X{\setminus}      & \X{\star}         \\
 \X{\cup}           & \X{\cap}           & \X{\ast}          \\
 \X{\sqcup}         & \X{\sqcap}         & \X{\circ}         \\
 \X{\vee}, \verb|\lor|     & \X{\wedge}, \verb|\land|  & \X{\bullet}       \\
 \X{\oplus}         & \X{\ominus}        & \X{\diamond}      \\
 \X{\odot}          & \X{\oslash}        & \X{\uplus}        \\
 \X{\otimes}        & \X{\bigcirc}       & \X{\amalg}        \\
 \X{\bigtriangleup} &\X{\bigtriangledown}& \X{\dagger}       \\
 \X{\lhd}$^1$         & \X{\rhd}$^1$         & \X{\ddagger}      \\
 \X{\unlhd}$^1$       & \X{\unrhd}$^1$       & \X{\wr}
\end{symbols}

\end{table}

\begin{table}[!tbp]
\caption{������� ���������}
\begin{symbols}{*4{cl}}
 \X{\sum}      & \X{\bigcup}   & \X{\bigvee}   & \X{\bigoplus}\\
 \X{\prod}     & \X{\bigcap}   & \X{\bigwedge} &\X{\bigotimes}\\
 \X{\coprod}   & \X{\bigsqcup} & &             & \X{\bigodot} \\
 \X{\int}      & \X{\oint}     & &             & \X{\biguplus}
\end{symbols}

\end{table}

\begin{table}[!tbp]
\caption{�������}
\begin{symbols}{*3{cl}}
 \X{\leftarrow}��� \verb|\gets|& \X{\longleftarrow}     & \X{\uparrow}          \\
 \X{\rightarrow}��� \verb|\to|& \X{\longrightarrow}    & \X{\downarrow}        \\
 \X{\leftrightarrow}    & \X{\longleftrightarrow}& \X{\updownarrow}      \\
 \X{\Leftarrow}         & \X{\Longleftarrow}     & \X{\Uparrow}          \\
 \X{\Rightarrow}        & \X{\Longrightarrow}    & \X{\Downarrow}        \\
 \X{\Leftrightarrow}    & \X{\Longleftrightarrow}& \X{\Updownarrow}      \\
 \X{\mapsto}            & \X{\longmapsto}        & \X{\nearrow}          \\
 \X{\hookleftarrow}     & \X{\hookrightarrow}    & \X{\searrow}          \\
 \X{\leftharpoonup}     & \X{\rightharpoonup}    & \X{\swarrow}          \\
 \X{\leftharpoondown}   & \X{\rightharpoondown}  & \X{\nwarrow}          \\
 \X{\rightleftharpoons} & \X{\iff}(�\'������ ������)& \X{\leadsto}$^1$

\end{symbols}
\centerline{\footnotesize $^1$��� ������� � ����� ������� �����������
  ������� \textsf{latexsym}.}
\end{table}

\begin{table}[!tbp]
\caption{������������}\label{tab:delimiters}
\begin{symbols}{*4{cl}}
 \X{(}            & \X{)}            & \X{\uparrow} & \X{\Uparrow}    \\
 \X{[}��� \verb|\lbrack|   & \X{]}��� \verb|\rbrack|  & \X{\downarrow}   & \X{\Downarrow}  \\
 \X{\{}��� \verb|\lbrace|  & \X{\}}��� \verb|\rbrace|  & \X{\updownarrow} & \X{\Updownarrow}\\
 \X{\langle}      & \X{\rangle}  & \X{|}��� \verb|\vert| &\X{\|}��� \verb|\Vert|\\
 \X{\lfloor}      & \X{\rfloor}      & \X{\lceil}       & \X{\rceil}      \\
 \X{/}            & \X{\backslash}   & &% (dual. empty)

\end{symbols}
\end{table}

\begin{table}[!tbp]
\caption{������� ������������}
\begin{symbols}{*4{cl}}
 \Y{\lgroup}      & \Y{\rgroup}      & \Y{\lmoustache}  & \Y{\rmoustache} \\
 \Y{\arrowvert}   & \Y{\Arrowvert}   & \Y{\bracevert}
\end{symbols}
\end{table}


\begin{table}[!tbp]
\caption{������ �������}
\begin{symbols}{*4{cl}}
 \X{\dots}       & \X{\cdots}      & \X{\vdots}      & \X{\ddots}     \\
 \X{\hbar}       & \X{\imath}      & \X{\jmath}      & \X{\ell}       \\
 \X{\Re}         & \X{\Im}         & \X{\aleph}      & \X{\wp}        \\
 \X{\forall}     & \X{\exists}     & \X{\mho}$^1$      & \X{\partial}   \\
 \X{'}           & \X{\prime}      & \X{\emptyset}   & \X{\infty}     \\
 \X{\nabla}      & \X{\triangle}   & \X{\Box}$^1$     & \X{\Diamond}$^1$ \\
 \X{\bot}        & \X{\top}        & \X{\angle}      & \X{\surd}      \\
\X{\diamondsuit} & \X{\heartsuit}  & \X{\clubsuit}   & \X{\spadesuit} \\
 \X{\neg}��� \verb|\lnot| & \X{\flat}       & \X{\natural}    & \X{\sharp}

\end{symbols}
\centerline{\footnotesize $^1$��� ������� � ����� ������� �����������
  ������� \textsf{latexsym}.}
\end{table}

\begin{table}[!tbp]
\caption{��-�������������� �������}
\bigskip
��� ������� ����� ������������ � � ��������� ������.
\begin{symbols}{*4{cl}}
\SC{\dag}  & \SC{\S} & \SC{\copyright} & \SC{\textregistered} \\
\SC{\ddag} & \SC{\P} & \SC{\pounds}    & \SC{\%}              \\
\end{symbols}
\end{table}

%
%
% If the AMS Stuff is not available, we drop out right here :-)
%

\begin{table}[!tbp]
\caption{������������ AMS}\label{AMSD}
\bigskip
\begin{symbols}{*4{cl}}
\X{\ulcorner}&\X{\urcorner}&\X{\llcorner}&\X{\lrcorner}\\
\X{\lvert}&\X{\rvert}&\X{\lVert}&\X{\rVert}
\end{symbols}
\end{table}

\begin{table}[!tbp]
\caption{����� ���������� � ������ AMS}
\begin{symbols}{*5{cl}}
\X{\digamma}     &\X{\varkappa} & \X{\beth} &\X{\gimel} & \X{\daleth}   
\end{symbols}
\end{table}

\begin{table}[!tbp]
\caption{�������� ��������� AMS}
\begin{symbols}{*3{cl}}
 \X{\lessdot}           & \X{\gtrdot}            & \X{\doteqdot}��� \verb|\Doteq| \\
 \X{\leqslant}          & \X{\geqslant}          & \X{\risingdotseq}     \\
 \X{\eqslantless}       & \X{\eqslantgtr}        & \X{\fallingdotseq}    \\
 \X{\leqq}              & \X{\geqq}              & \X{\eqcirc}           \\
 \X{\lll}��� \verb|\llless|      & \X{\ggg}��� \verb|\gggtr| & \X{\circeq}  \\
 \X{\lesssim}           & \X{\gtrsim}            & \X{\triangleq}        \\
 \X{\lessapprox}        & \X{\gtrapprox}         & \X{\bumpeq}           \\
 \X{\lessgtr}           & \X{\gtrless}           & \X{\Bumpeq}           \\
 \X{\lesseqgtr}         & \X{\gtreqless}         & \X{\thicksim}         \\
 \X{\lesseqqgtr}        & \X{\gtreqqless}        & \X{\thickapprox}      \\
 \X{\preccurlyeq}       & \X{\succcurlyeq}       & \X{\approxeq}         \\
 \X{\curlyeqprec}       & \X{\curlyeqsucc}       & \X{\backsim}          \\
 \X{\precsim}           & \X{\succsim}           & \X{\backsimeq}        \\
 \X{\precapprox}        & \X{\succapprox}        & \X{\vDash}            \\
 \X{\subseteqq}         & \X{\supseteqq}         & \X{\Vdash}            \\
 \X{\Subset}            & \X{\Supset}            & \X{\Vvdash}           \\
 \X{\sqsubset}          & \X{\sqsupset}          & \X{\backepsilon}      \\
 \X{\therefore}         & \X{\because}           & \X{\varpropto}        \\
 \X{\shortmid}          & \X{\shortparallel}     & \X{\between}          \\
 \X{\smallsmile}        & \X{\smallfrown}        & \X{\pitchfork}        \\
 \X{\vartriangleleft}   & \X{\vartriangleright}  & \X{\blacktriangleleft}\\
 \X{\trianglelefteq}    & \X{\trianglerighteq}   &\X{\blacktriangleright}
\end{symbols}
\end{table}

\begin{table}[!tbp]
\caption{������� AMS}
\begin{symbols}{*3{cl}}
 \X{\dashleftarrow}      & \X{\dashrightarrow}     & \X{\multimap}          \\
 \X{\leftleftarrows}     & \X{\rightrightarrows}   & \X{\upuparrows}        \\
 \X{\leftrightarrows}    & \X{\rightleftarrows}    & \X{\downdownarrows}    \\
 \X{\Lleftarrow}         & \X{\Rrightarrow}        & \X{\upharpoonleft}     \\
 \X{\twoheadleftarrow}   & \X{\twoheadrightarrow}  & \X{\upharpoonright}    \\
 \X{\leftarrowtail}      & \X{\rightarrowtail}     & \X{\downharpoonleft}   \\
 \X{\leftrightharpoons}  & \X{\rightleftharpoons}  & \X{\downharpoonright}  \\
 \X{\Lsh}                & \X{\Rsh}                & \X{\rightsquigarrow}   \\
 \X{\looparrowleft}      & \X{\looparrowright}     &\X{\leftrightsquigarrow}\\
 \X{\curvearrowleft}     & \X{\curvearrowright}    & &                      \\
 \X{\circlearrowleft}    & \X{\circlearrowright}   & &
\end{symbols}
\end{table}

\begin{table}[!tbp]
\caption{������������� �������� ��������� � ������� AMS}\label{AMSNBR}
\begin{symbols}{*3{cl}}
 \X{\nless}           & \X{\ngtr}            & \X{\varsubsetneqq}  \\
 \X{\lneq}            & \X{\gneq}            & \X{\varsupsetneqq}  \\
 \X{\nleq}            & \X{\ngeq}            & \X{\nsubseteqq}     \\
 \X{\nleqslant}       & \X{\ngeqslant}       & \X{\nsupseteqq}     \\
 \X{\lneqq}           & \X{\gneqq}           & \X{\nmid}           \\
 \X{\lvertneqq}       & \X{\gvertneqq}       & \X{\nparallel}      \\
 \X{\nleqq}           & \X{\ngeqq}           & \X{\nshortmid}      \\
 \X{\lnsim}           & \X{\gnsim}           & \X{\nshortparallel} \\
 \X{\lnapprox}        & \X{\gnapprox}        & \X{\nsim}           \\
 \X{\nprec}           & \X{\nsucc}           & \X{\ncong}          \\
 \X{\npreceq}         & \X{\nsucceq}         & \X{\nvdash}         \\
 \X{\precneqq}        & \X{\succneqq}        & \X{\nvDash}         \\
 \X{\precnsim}        & \X{\succnsim}        & \X{\nVdash}         \\
 \X{\precnapprox}     & \X{\succnapprox}     & \X{\nVDash}         \\
 \X{\subsetneq}       & \X{\supsetneq}       & \X{\ntriangleleft}  \\
 \X{\varsubsetneq}    & \X{\varsupsetneq}    & \X{\ntriangleright} \\
 \X{\nsubseteq}       & \X{\nsupseteq}       & \X{\ntrianglelefteq}\\
 \X{\subsetneqq}      & \X{\supsetneqq}      &\X{\ntrianglerighteq}\\[0.5ex]
 \X{\nleftarrow}      & \X{\nrightarrow}     & \X{\nleftrightarrow}\\
 \X{\nLeftarrow}      & \X{\nRightarrow}     & \X{\nLeftrightarrow}

\end{symbols}
\end{table}

\begin{table}[!tbp]
\caption{�������� ��������� AMS}
\begin{symbols}{*3{cl}}
 \X{\dotplus}        & \X{\centerdot}      & \X{\intercal}      \\
 \X{\ltimes}         & \X{\rtimes}         & \X{\divideontimes} \\
 \X{\Cup}��� \verb|\doublecup|& \X{\Cap}��� \verb|\doublecap|& \X{\smallsetminus} \\
 \X{\veebar}         & \X{\barwedge}       & \X{\doublebarwedge}\\
 \X{\boxplus}        & \X{\boxminus}       & \X{\circleddash}   \\
 \X{\boxtimes}       & \X{\boxdot}         & \X{\circledcirc}   \\
 \X{\leftthreetimes} & \X{\rightthreetimes}& \X{\circledast}    \\
 \X{\curlyvee}       & \X{\curlywedge}
\end{symbols}
\end{table}

\begin{table}[!tbp]
\caption{������ ������� AMS}
\begin{symbols}{*3{cl}}
 \X{\hbar}             & \X{\hslash}           & \X{\Bbbk}            \\
 \X{\square}           & \X{\blacksquare}      & \X{\circledS}        \\
 \X{\vartriangle}      & \X{\blacktriangle}    & \X{\complement}      \\
 \X{\triangledown}     &\X{\blacktriangledown} & \X{\Game}            \\
 \X{\lozenge}          & \X{\blacklozenge}     & \X{\bigstar}         \\
 \X{\angle}            & \X{\measuredangle}    & \X{\sphericalangle}  \\
 \X{\diagup}           & \X{\diagdown}         & \X{\backprime}       \\
 \X{\nexists}          & \X{\Finv}             & \X{\varnothing}      \\
 \X{\eth}              & \X{\mho}
\end{symbols}
\end{table}



\begin{table}[!tbp]
\caption{�������������� ��������}
\begin{symbols}{@{}*3l@{}}
������ & ������� & ��������� �����\\
\hline
\rule{0pt}{1.05em}$\mathrm{ABCdef}$
        & \verb|\mathrm{ABCdef}|
        &       \\
$\mathit{ABCdef}$
        & \verb|\mathit{ABCdef}|
        &       \\
$\mathnormal{ABCdef}$
        & \verb|\mathnormal{ABCdef}|
        &       \\
$\mathcal{ABC}$
        & \verb|\mathcal{ABC}|
        &\pai{euscript} � ������: \textsf{mathcal}\\
$\mathscr{ABC}$
        & \verb|\mathscr{ABC}|
        &\pai{mathrsfs}\\
$\mathfrak{ABCdef}$
        & \verb|\mathfrak{ABCdef}|
        &\pai{eufrak}                \\
$\mathbb{ABC}$
        & \verb|\mathbb{ABC}|
        &\pai{amsfonts} ��� \textsf{amssymb}        \\
\end{symbols}
\end{table}


%%% Local Variables:
%%% mode: latex
%%% TeX-master: "lshort2e"
%%% End:

\documentclass[russian,koi8-r]{eskdtab}
\usepackage{eskdspec}
\usepackage[T2A]{fontenc}
\usepackage{pscyr}

\ESKDtitle{�������� ��������� eskdx. �������� ������������}
\usepackage[unicode]{hyperref}

\begin{document}
\begin{ESKDspecification}
&&&&&&\\
&&&&\hfil\underline{������������}\hfil&&\\
&&&&&&\\
*&&&��.6.104.00 ��&��������� ������&1&*�1,�3\\
�4&&&��.6.104.00 ��&����������� �������&1&2 ���.\\
&&&&&&\\
&&&&\hfil\underline{������}\hfil&&\\
&&&&&&\\
�2&&1&��.6.104.01&������&1&\\
�4&&2&��.6.104.02&�����&1&\\
�4&&3&��.6.104.03&������&1&\\
�3&&4&��.6.104.04&������&1&\\
�4&&5&��.6.104.05&������&1&\\
�4&&6&��.6.104.06&������&1&\\
�4&&7&��.6.104.07&����� ��������&1&\\
�4&&8&��.6.104.08&��������&1&\\
�4&&9&��.6.104.09&���������&1&\\
�4&&10&��.6.104.10&���������&1&\\
&&&&������� ���� 481-71&&0{,}150 ��\\
&&&&&&\\
&&&&\hfil\underline{����������� �������}\hfil&&\\
&&&&&&\\
&&11&&���� �8$\times$40.58 ���� 78 08-79&4&\\
&&&&����� ���� 59 15-70&&\\
&&12&&�8.5&4&\\
&&13&&�10.5&1&\\
&&&&&&\\
&&&&&&\\
&&&&&&\\
&&&&&&\\
&&&&\hfil\underline{���������}\hfil&&\\
&&&&&&\\
&&14&&������&&0{,}010 ��\\
&&15&&��������� �3$\times$45&&\\
&&&&���� 792-67&&0{,}150 ��\\
\end{ESKDspecification}
\end{document}

%%%%%%%%%%%%%%%%%%%%%%%%%%%%%%%%%%%%%%%%%%%%%%%%%%%%%%%%%%%%%%%%%
%%%%%%%%%%%%%%%%%%%%%%%%%%%%%%%%%%%%%%%%%%%%%%%%%%%%%%%%%%%%%%%%%
\setcounter{chapter}{4}
\newcommand{\graphicscompanion}{\emph{The \LaTeX{} Graphics Companion}~\cite{graphicscompanion}} 
\newcommand{\hobby}{\emph{����������� ������������ MetaPost}~\cite{metapost}}
\newcommand{\hoenig}{\emph{\TeX{} Unbound}~\cite{unbound}}
\newcommand{\graphicsinlatex}{\emph{������� �  \LaTeXe{}}~\cite{ursoswald}}

\chapter{��������� �������������� �������}

\begin{intro}
����������� ����� ���������� \LaTeX\ ��� ������� ������. �� \LaTeX{} �����
����������, ���� � ������������, ����������� ��������� ������� ��
���������� ��������. ����� ����, ���������� ��������� ���������� \LaTeX{},
��������� ��� ����������. ��� ����� ��������� � ���������� �� ���.
\end{intro}

\section{�����}

��������� \ei{picture} ��������� ��������������� �������� ����� �
����� \LaTeX. ��������� �������� ���������� � \manual. � �����
�������, ��� ���� ���� �������� ��������� �����������, ��������� ���
������� ��������, ��� � ������� ��� ������ ���������� � ���������
���������. � ������ �������, � ��������� \ei{picture} �����������
������� \ci{qbezier}, ��� <<\texttt{q}>> ��������
<<������������>>. ��������� ����� ������������ ������, ����� ���
����������, ������� ��� ������ �����, ����� �����������������
���������������� ������������� ������� �����, ���� ��� ����� �
��������� ��������� �������������� ������. ����� ����, ����� �����
������ \ci{qbezier} ������������ �� ����� ����������������, ������,
Java, ��������� \ei{picture} ���������� �������� ������.

���� ���������������� �������� ����� �� \LaTeX{} �������� ���������� �
����� �������� �����������, ��� ��� �� ������ ��������. ����������
����� ������� ��������� ���������� �����������, � ��� �������������
��������� � ��������� ����������� �����.

������ ��������� \pai{epic} � \pai{eepic} (���������, ��������, �
\companion) ��� \pai{pstricks} �������� ����� �����������
������������� ��������� \ei{picture} � ������ ��������� �����������
����������� \LaTeX{}.

����� ��� ������ ��� ������ ������ �������� ��������� \ei{picture},
����� \pai{pstricks} ����� ����������� ��������� ��� ���������,
\ei{pspicture}. ����������� \pai{pstricks} ����������� �� ����, ��� ��
������ ���������� ���������������� \PSi{}. ����� ����� ������,
���������� � ��������� ������, ���������� ��� ���������� �����. �����
�� ��� �������� ����� \texorpdfstring{\Xy}{Xy}-pic, ��������� � �����
���� �����. ��������� ����� ������� �������� ������� �
\graphicscompanion{} (�� ������ � \companion).

��������, ����� ������ ����������� ������������, ��������� � \LaTeX{},
�������� \MP, ���������-������� ���������� Donald E. Knuth ���������
\texttt{METAFONT}. \MP{} ���������� ����� ������ � �������������
���������� ���� \texttt{METAFONT}. � ������� �� �� \texttt{METAFONT},
������������� ������, \MP{} ���������� ����� \EPSi{}, ������� �����
������������� � \LaTeX{}. ����� �������� \hobby, ��� �������� �
\cite{ursoswald}. 

����� �������� ������ � �������� (� ��������) � \LaTeX{} � \TeX{}
������� � \hoenig.

\section{��������� \texttt{picture}}
\secby{Urs Oswald}{osurs@bluewin.ch}

\subsection{�������� �������}

��������� \ei{picture} ��������� ����� �� ���� ������:
\begin{lscommand}
\ci{begin}\verb|{picture}(|$x,y$\verb|)|\ldots\ci{end}\verb|{picture}|
\end{lscommand}
\noindent ����
\begin{lscommand}
\ci{begin}\verb|{picture}(|$x,y$\verb|)(|$x_0,y_0$\verb|)|\ldots\ci{end}\verb|{picture}|
\end{lscommand}

����� $x,\,y,\,x_0,\,y_0$ ���������� � ����������� \ci{unitlength},
������� ����� ������ � ����� ������ (�� �� ������ ���������
\ei{picture}) ��������� ���������
\begin{lscommand}
\ci{setlength}\verb|{|\ci{unitlength}\verb|}{1.2cm}|
\end{lscommand}

�������� \ci{unitlength} �� ��������� ���������� 1pt. ������ ����,
$(x,y)$, ������� �������������� ��� �������� ��������������
������������ ������ ���������. �������������� ������ ����,
$(x_0,y_0)$, ����������� ������������ ���������� ������� ������ ����
������������������ ��������������.

����������� ������ ��������� ����� ���� �� ���� ����:
\begin{lscommand}
\ci{put}\verb|(|$x,y$\verb|){|\emph{������}\verb|}|
\end{lscommand}
\noindent ���
\begin{lscommand}
\ci{multiput}\verb|(|$x,y$\verb|)(|$\Delta x,\Delta y$\verb|){|$n$\verb|}{|\emph{������}\verb|}|\end{lscommand}

������ ����� �������� �����������. �� ������ ��������
\begin{lscommand}
\ci{qbezier}\verb|(|$x_1,y_1$\verb|)(|$x_2,y_2$\verb|)(|$x_3,y_3$\verb|)|
\end{lscommand}

\subsection{�������}

\begin{example}
\setlength{\unitlength}{5cm}
\begin{picture}(1,1)
  \put(0,0){\line(0,1){1}}
  \put(0,0){\line(1,0){1}}  
  \put(0,0){\line(1,1){1}}  
  \put(0,0){\line(1,2){.5}}
  \put(0,0){\line(1,3){.3333}}
  \put(0,0){\line(1,4){.25}}  
  \put(0,0){\line(1,5){.2}}
  \put(0,0){\line(1,6){.1667}}
  \put(0,0){\line(2,1){1}}
  \put(0,0){\line(2,3){.6667}}
  \put(0,0){\line(2,5){.4}}
  \put(0,0){\line(3,1){1}}  
  \put(0,0){\line(3,2){1}}
  \put(0,0){\line(3,4){.75}}
  \put(0,0){\line(3,5){.6}}
  \put(0,0){\line(4,1){1}}
  \put(0,0){\line(4,3){1}}  
  \put(0,0){\line(4,5){.8}}
  \put(0,0){\line(5,1){1}}
  \put(0,0){\line(5,2){1}}
  \put(0,0){\line(5,3){1}}
  \put(0,0){\line(5,4){1}}
  \put(0,0){\line(5,6){.8333}}
  \put(0,0){\line(6,1){1}}
  \put(0,0){\line(6,5){1}}
\end{picture}
\end{example}
������� �������� ��������
\begin{lscommand}
\ci{put}\verb|(|$x,y$\verb|){|\ci{line}\verb|(|$x_1,y_1$\verb|){|$length$\verb|}}|
\end{lscommand}
������� \ci{line} ����� ��� ���������:
\begin{enumerate}
 \item ������ �����������,
 \item �����.
\end{enumerate}
���������� ������� ���������� � ����� �������� ������� ����� �����
\[
 -6,\,-5\,\ldots,\,5,\,6,
\]
� ��� ������� ���� ������� �������� (�� ����� ������ ��������, �����
1). ����������� ���������� ��� 25 ��������� �������� ������� � ������
���������. ����� ���������� � �������� \ci{unitlength}. �������� �����
������ ������������ ���������� � ������ ������������� ������� �
��������������~--- �� ���� ��������� �������.

\subsection{�������}

\begin{example}
\setlength{\unitlength}{1mm}
\begin{picture}(60,40)
  \put(30,20){\vector(1,0){30}}
  \put(30,20){\vector(4,1){20}}
  \put(30,20){\vector(3,1){25}}
  \put(30,20){\vector(2,1){30}}
  \put(30,20){\vector(1,2){10}}
  \thicklines
  \put(30,20){\vector(-4,1){30}}
  \put(30,20){\vector(-1,4){5}}
  \thinlines
  \put(30,20){\vector(-1,-1){5}}
  \put(30,20){\vector(-1,-4){5}}
\end{picture}
\end{example}
������� �������� ��������
\begin{lscommand}
\ci{put}\verb|(|$x,y$\verb|){|\ci{vector}\verb|(|$x_1,y_1$\verb|){|$length$\verb|}}|
\end{lscommand}
��� �������� �������� ������� ����������� ��� ����� ���������� �
���������, ��� ��� ��������, � ������~--- �������
\[
  -4,\,-3,\,\ldots,\,3,\,4.
\]
���������� ����� ������� ���� ������� �������� (�� ����� ������
��������, ����� 1). �������� ������ ������� \ci{thicklines} �� ����
��������, ����������� � ������� ����� ����.

\subsection{����������}

\begin{example}
\setlength{\unitlength}{1mm}
\begin{picture}(60, 40)
  \put(20,30){\circle{1}}
  \put(20,30){\circle{2}}
  \put(20,30){\circle{4}}
  \put(20,30){\circle{8}}
  \put(20,30){\circle{16}}
  \put(20,30){\circle{32}}
  
  \put(40,30){\circle{1}}
  \put(40,30){\circle{2}}
  \put(40,30){\circle{3}}
  \put(40,30){\circle{4}}
  \put(40,30){\circle{5}}
  \put(40,30){\circle{6}}
  \put(40,30){\circle{7}}
  \put(40,30){\circle{8}}
  \put(40,30){\circle{9}}
  \put(40,30){\circle{10}}
  \put(40,30){\circle{11}}
  \put(40,30){\circle{12}}
  \put(40,30){\circle{13}}
  \put(40,30){\circle{14}}
  
  \put(15,10){\circle*{1}}
  \put(20,10){\circle*{2}}
  \put(25,10){\circle*{3}}
  \put(30,10){\circle*{4}}
  \put(35,10){\circle*{5}}
\end{picture}
\end{example}
�������
\begin{lscommand}
  \ci{put}\verb|(|$x,y$\verb|){|\ci{circle}\verb|{|\emph{�������}\verb|}}|
\end{lscommand}
\noindent ������ ���������� � ������� � ����� $(x,y)$ � ��������� (��
��������!) \emph{�������}. ��������� \ei{picture} ��������� ��������
���������� ��������� �� ����� �������� 14\,��, � ���� � ���� ��������
��������� �� ��� ��������. ������� \ci{circle*} ������ ����
(����������� ����������).

��� � � ������ ��������, �� ������ ��������� � ������ ��������������
�������, �����, ��� \pai{eepic} ��� \pai{pstricks}.
��������� �������� ���� ������� ��������� � \graphicscompanion.

���������� ����� � ����� � ������ ��������� \ei{picture}. ���� �� ��
������� ���������� ����������� ���������� (��� ���������� �� ��
���������), �� ����� ���������� ������������ ���������� � ������� ���
������ ������ �����. ������� � �������� ������ �� Java ��������� �
\graphicsinlatex.

\subsection{����� � �������}

\begin{example}
\setlength{\unitlength}{1cm}
\begin{picture}(6,5)
  \thicklines
  \put(1,0.5){\line(2,1){3}}
  \put(4,2){\line(-2,1){2}}
  \put(2,3){\line(-2,-5){1}}
  \put(0.7,0.3){$A$}
  \put(4.05,1.9){$B$}
  \put(1.7,2.95){$C$}
  \put(3.1,2.5){$a$}
  \put(1.3,1.7){$b$}
  \put(2.5,1.05){$c$}
  \put(0.3,4){$F=
    \sqrt{s(s-a)(s-b)(s-c)}$}  
  \put(3.5,0.4){$\displaystyle
    s:=\frac{a+b+c}{2}$}
\end{picture}
\end{example}
��� ���������� ���� ������, ����� � ������� ����� ����������� �
��������� \ei{picture} ������� ��������~--- �������� \ci{put}.

\subsection{������� \ci{multiput} � \ci{linethickness}}

\begin{example}
\setlength{\unitlength}{2mm}
\begin{picture}(30,20)
  \linethickness{0.075mm}
  \multiput(0,0)(1,0){31}%
    {\line(0,1){20}}
  \multiput(0,0)(0,1){21}%
    {\line(1,0){30}}
  \linethickness{0.15mm}    
  \multiput(0,0)(5,0){7}%
    {\line(0,1){20}}
  \multiput(0,0)(0,5){5}%
    {\line(1,0){30}}
  \linethickness{0.3mm}    
  \multiput(5,0)(10,0){3}%
    {\line(0,1){20}}
  \multiput(0,5)(0,10){2}%
    {\line(1,0){30}}
\end{picture}
\end{example}
�������
\begin{lscommand}
  \ci{multiput}\verb|(|$x,y$\verb|)(|$\Delta x,\Delta y$\verb|){|$n$\verb|}{|\emph{������}\verb|}|
\end{lscommand}
\noindent ����� 4 ���������: ��������� �����, ������ �������� ��
������ ������� � ����������, ����� �������� � ���������� ������ ���
���������. ������� \ci{linethickness} ��������� � �������������� �
������������ ��������, �� �������~--- � ��������� ���
�����������. ���, ������ ��������� ����� � � ������������ ������
�����! 

\subsection{�����. ������� \ci{thinlines} � \ci{thicklines}}

\begin{example}
\setlength{\unitlength}{1cm}
\begin{picture}(6,4)
  \linethickness{0.075mm}
  \multiput(0,0)(1,0){7}%
    {\line(0,1){4}}
  \multiput(0,0)(0,1){5}%
    {\line(1,0){6}}
  \thicklines
  \put(2,3){\oval(3,1.8)} 
  \thinlines
  \put(3,2){\oval(3,1.8)} 
  \thicklines
  \put(2,1){\oval(3,1.8)[tl]} 
  \put(4,1){\oval(3,1.8)[b]} 
  \put(4,3){\oval(3,1.8)[r]} 
  \put(3,1.5){\oval(1.8,0.4)}     
\end{picture}
\end{example}
�������
\begin{lscommand}
  \ci{put}\verb|(|$x,y$\verb|){|\ci{oval}\verb|(|$w,h$\verb|)}|
\end{lscommand}
\noindent ���
\begin{lscommand}
  \ci{put}\verb|(|$x,y$\verb|){|\ci{oval}\verb|(|$w,h$\verb|)[|\emph{�������}\verb|]}|
\end{lscommand}
\noindent ������� ���� � ������� � $(x,y)$, ������� ����� $w$ � ������
$h$. �������������� �������� \emph{�������} ����� ��������� ��������
\texttt{b}, \texttt{t}, \texttt{l} � \texttt{r} (����/�����/�����/������)
� ����� �������� ��� ��������, ��� �� ����������� �������.

������� ����� ����� ����������� ����� �� ���� ������:\\
\ci{linethickness}\verb|{|\emph{�����}\verb|}|
� ����� �������, � \ci{thinlines} � \ci{thicklines}~--- � ������. � ��
�����, ��� \ci{linethickness}\verb|{|\emph{�����}\verb|}| ������
������ �� �������������� � ������������ ����� (� ������������ ������
�����), \ci{thinlines} � \ci{thicklines} ������ �� ��������� �������,
���������� � �����.

\subsection{��������� ������������� ������ ��������}

\begin{example}
\setlength{\unitlength}{0.5mm}
\begin{picture}(120,168)
\newsavebox{\foldera}% ����������
\savebox{\foldera}
  (40,32)[bl]{% ����������� 
  \multiput(0,0)(0,28){2}
    {\line(1,0){40}}
  \multiput(0,0)(40,0){2}
    {\line(0,1){28}}
  \put(1,28){\oval(2,2)[tl]}
  \put(1,29){\line(1,0){5}}
  \put(9,29){\oval(6,6)[tl]}
  \put(9,32){\line(1,0){8}}
  \put(17,29){\oval(6,6)[tr]}
  \put(20,29){\line(1,0){19}}
  \put(39,28){\oval(2,2)[tr]}  
}
\newsavebox{\folderb}% ����������
\savebox{\folderb}
  (40,32)[l]{%        ����������� 
  \put(0,14){\line(1,0){8}}
  \put(8,0){\usebox{\foldera}}
}
\put(34,26){\line(0,1){102}} 
\put(14,128){\usebox{\foldera}}
\multiput(34,86)(0,-37){3}
  {\usebox{\folderb}} 
\end{picture}
\end{example}
���� ������� ����� ���� \emph{��������} ��������
\begin{lscommand}
  \ci{newsavebox}\verb|{|\emph{��������}\verb|}|
\end{lscommand}
\noindent � ����� \emph{���������} ��������
\begin{lscommand}
  \ci{savebox}\verb|{|\emph{��������}\verb|}(|\emph{������,������}\verb|)[|\emph{�������}\verb|]{|\emph{����������}\verb|}|
\end{lscommand}
\noindent �, �������, ������� ������ ��� \emph{���������} ��������
\begin{lscommand}
  \ci{put}\verb|(|$x,y$\verb|)|\ci{usebox}\verb|{|\emph{��������}\verb|}|
\end{lscommand}

�������������� �������� \emph{�������} ���������� ����� ��������
�����. � ����������� ������� �� ���������� � �������� \texttt{bl}, ���
�������� ����� �������� � ������ ����� ���� �����. ������ ��������
��������~--- \texttt{t} (�����) � \texttt{r} (������).

�������� \emph{��������} ���������� �������� �������� \LaTeX{} (������
� �������� ����� ����� ����� ��� � ��������������� �������). �����
����� ���� ����������: � ���� ������� ������ ����������� \ci{folderb}
������������ \ci{foldera}.

������� \ci{oval} �������� ��������� ������ ��� ������� \ci{line} ��
��������, ���� ����� ������� ������ �������� 3\,��.

\subsection{������������ ������ �����}

\begin{example}
\setlength{\unitlength}{1cm}
\begin{picture}(6,4)
  \linethickness{0.075mm}
  \multiput(0,0)(1,0){7}
    {\line(0,1){4}}
  \multiput(0,0)(0,1){5}
    {\line(1,0){6}}
  \thicklines
  \put(0.5,0.5){\line(1,5){0.5}}    
  \put(1,3){\line(4,1){2}} 
  \qbezier(0.5,0.5)(1,3)(3,3.5)
  \thinlines   
  \put(2.5,2){\line(2,-1){3}}
  \put(5.5,0.5){\line(-1,5){0.5}}
  \linethickness{1mm}
  \qbezier(2.5,2)(5.5,0.5)(5,3)
  \thinlines
  \qbezier(4,2)(4,3)(3,3)
  \qbezier(3,3)(2,3)(2,2)
  \qbezier(2,2)(2,1)(3,1)
  \qbezier(3,1)(4,1)(4,2)
\end{picture}
\end{example}
��� ���������� ���� ������, ��������� ���������� �� ������ ������������
������ ����� ���� �������������������� ���������. ��������� ���
������� ������. ����������� ����� ���������� ������� �������
\ci{linethickness} �� �������������� � ������������ �����, � ������
\ci{thinlines} � \ci{thicklines}~--- �� ��������� �������. ��� �����
����������, ��� ��� ������� ������ �� ������������ ������ �����, �
������ ��������� ������� �������� ���������� ����������.

����� $P_1=(x_1,\,y_1),\,P_2=(x_2,\,y_2)$ ������ �������� �����, �
$m_1,\,m_2$~--- ��������������� ������� % �����������?
������������ ������ �����. ����� ������������� ����������� �����
$S=(x,\,y)$ �������� ����������
\begin{equation} \label{zwischenpunkt}
  \left\{
    \begin{array}{rcl}
      x & = & \displaystyle \frac{m_2 x_2-m_1x_1-(y_2-y_1}{m_2-m_1}, \\
      y & = & y_i+m_i(x-x_i)\qquad (i=1,\,2).
    \end{array}
  \right.
\end{equation}
\noindent � \graphicsinlatex\ ��������� Java-���������, ����������
    ����������� �������  \ci{qbezier}.

\subsection{������ �����}

\begin{example}
\setlength{\unitlength}{1.3cm}
\begin{picture}(4.3,3.6)(-2.5,-0.25)
  \put(-2,0){\vector(1,0){4.4}}
  \put(2.45,-.05){$x$}
  \put(0,0){\vector(0,1){3.2}}
  \put(0,3.35){\makebox(0,0){$y$}}
  \qbezier(0.0,0.0)(1.2384,0.0)
    (2.0,2.7622) 
  \qbezier(0.0,0.0)(-1.2384,0.0)
    (-2.0,2.7622)
  \linethickness{.075mm}
  \multiput(-2,0)(1,0){5}
    {\line(0,1){3}}
  \multiput(-2,0)(0,1){4}
    {\line(1,0){4}}
  \linethickness{.2mm}
  \put( .3,.12763){\line(1,0){.4}}
  \put(.5,-.07237){\line(0,1){.4}}
  \put(-.7,.12763){\line(1,0){.4}}
  \put(-.5,-.07237){\line(0,1){.4}}
  \put(.8,.54308){\line(1,0){.4}}
  \put(1,.34308){\line(0,1){.4}}
  \put(-1.2,.54308){\line(1,0){.4}}
  \put(-1,.34308){\line(0,1){.4}}
  \put(1.3,1.35241){\line(1,0){.4}}
  \put(1.5,1.15241){\line(0,1){.4}}
  \put(-1.7,1.35241){\line(1,0){.4}}
  \put(-1.5,1.15241){\line(0,1){.4}}
  \put(-2.5,-0.25){\circle*{0.2}}
\end{picture}
\end{example}

�� ���� ����������� ������ ������������ �������� ������ ����� $y=\cosh
x -1$ ���������������� ������������ ������ �����. ������ ��������
������ ������������� � ����� \((2,\,2.7622)\), ������ � ������� �����
�������� \(m=3.6269\). ����� ��������� ���������
(\ref{zwischenpunkt}), �� ����� ��������� ���������� �����������
�����. ��� ��������� $(1.2384,\,0)$ � $(-1.2384,\,0)$. ��������
�������� ����� \emph{���������} ������ �����. ������ ���� �������,
������ ������ ������ ��������.

���� ������ ���������� ������������� ��������������� ��������� �������
\verb|\begin{picture}|. �������� ���������� � �������
<<��������������>> �����������, ����� ��� �������
\begin{lscommand} 
  \ci{begin}\verb|{picture}(4.3,3.6)(-2.5,-0.25)|
\end{lscommand}
\noindent ����������� �� ������� ������ ���� (����������� ������
�������) ���������� $(-2.5,-0.25)$. 

\subsection{�������� � ����������� ������ ���������������}

\begin{example}
\setlength{\unitlength}{1cm}
\begin{picture}(6,4)(-3,-2)
  \put(-2.5,0){\vector(1,0){5}}
  \put(2.7,-0.1){$\chi$}
  \put(0,-1.5){\vector(0,1){3}}
  \multiput(-2.5,1)(0.4,0){13}
    {\line(1,0){0.2}}
  \multiput(-2.5,-1)(0.4,0){13}
    {\line(1,0){0.2}}
  \put(0.2,1.4)
    {$\beta=v/c=\tanh\chi$}
  \qbezier(0,0)(0.8853,0.8853)
    (2,0.9640)
  \qbezier(0,0)(-0.8853,-0.8853)
    (-2,-0.9640)
  \put(-3,-2){\circle*{0.2}}
\end{picture}
\end{example}

����������� ����� ���� ������ ����� ���� ��������� �� ��������
(\ref{zwischenpunkt}). ������������� ����� ������������
$P_1=(0,\,0),\,m_1=1$ � $P_2=(2,\,\tanh 2),\,m_2=1/\cosh^2
2$. �������� ����� ������������ � ������������� ������� �����������, �
������ ����� ���� �������� �������������� ���������� $(-3,-2)$ (������
������). 

\section{\texorpdfstring{\Xy}{Xy}-pic}
\secby{Alberto Manuel Brand\~ao Sim\~oes}{albie@alfarrabio.di.uminho.pt}
\pai{xy}~--- ��� ����������� ����� ��� ��������� ��������. ��� ���
������������� ������ �������� � ��������� ��������� ��������� �������:
\begin{lscommand}
\verb|\usepackage[|\emph{�����}\verb|]{xy}|
\end{lscommand}
\noindent ��� \emph{�����}~--- ������ ����������� ������� \Xy-pic. ���
����� �������, � ������ �������, ��� ������� ������. �������������
������������ ����� \verb!all!, ������������ \LaTeX{} ��������� ���
������� \Xy{}. 

��������� \Xy-pic �������� � ��������� �����, ��� ������ �������
��������� ���������� � ������������ ������ �������:
\begin{example}
\begin{displaymath}
\xymatrix{A & B \\
          C & D }
\end{displaymath}
\end{example}
������� \ci{xymatrix} ������ �������������� � ��������������
������. ����� �� ������ ��� ������ � ��� �������. ����� �� ����
������� ������� ���������, ������� ������� �������� �������� \ci{ar}. 
\begin{example}
\begin{displaymath}
\xymatrix{ A \ar[r] & B \ar[d] \\
           D \ar[u] & C \ar[l] }
\end{displaymath}
\end{example}
������� ��������� ������� ���������� � ������, ������ �������
������. ���������� �������� �����������, ���� ���������� ������
(\texttt{u}p, \texttt{d}own, \texttt{r}ight � \texttt{l}eft).
\begin{example}
\begin{displaymath}
\xymatrix{
  A \ar[d] \ar[dr] \ar[r] & B \\
  D                       & C }
\end{displaymath}
\end{example}
��� ��������� ���������� ������� ������ ������ �����������. �� �����
������ ��������� ���� ����������� ��� ��������� �\'������ ��������.
\begin{example}
\begin{displaymath}
\xymatrix{
  A \ar[d] \ar[dr] \ar[drr] & & \\
  B                      & C & D }
\end{displaymath}
\end{example}

����� �������� ��� ����� ���������� ���������, �������� � ��������
�����. ��� ����� ������������  ������� ��������� ������ � �������
��������. 
\begin{example}
\begin{displaymath}
\xymatrix{
  A \ar[r]^f \ar[d]_g &
             B \ar[d]^{g'} \\
  D \ar[r]_{f'}       & C }
\end{displaymath}
\end{example}

��� �������� ����, ��� ��������� ������������ ��� � ��������������
������. ������������ ������� ����������� � ���, ��� ������� ������
�������� <<��� �������� �������>>, � ������~--- <<��� ��������>>.
���� ��� ������ ��������, ������������ �����: \verb+|+. �� ��������
����� \emph{�} �������.
\begin{example}
\begin{displaymath}
\xymatrix{
  A \ar[r]|f \ar[d]|g &
             B \ar[d]|{g'} \\
  D \ar[r]|{f'}       & C }
\end{displaymath}
\end{example}

����� ���������� ������� � �������� � ���, ����������� ��������
\verb!\ar[...]|\hole!. � ��������� ������� ����� ��������� ���������
����� �������. ����� ����� ��������, ������� �� ��� ����� ��� ����� ��
���: 
\begin{example}
\shorthandoff{"}
\begin{displaymath}
\xymatrix{
 \bullet\ar@{->}[rr] && \bullet\\
 \bullet\ar@{.<}[rr] && \bullet\\
 \bullet\ar@{~)}[rr] && \bullet\\
 \bullet\ar@{=(}[rr] && \bullet\\
 \bullet\ar@{~/}[rr]  && \bullet\\
 \bullet\ar@{^{(}->}[rr]  && \bullet\\
 \bullet\ar@2{->}[rr]  && \bullet\\
 \bullet\ar@3{->}[rr]  && \bullet\\
 \bullet\ar@{=+}[rr]   && \bullet
}
\end{displaymath}
\shorthandon{"}
\end{example}

�������� ������� ����� ���������� ����� �����������:
\begin{example}
\begin{displaymath}
\xymatrix{
 \bullet \ar[r] 
         \ar@{.>}[r] & 
 \bullet
}
\end{displaymath}
\end{example}

\begin{example}
\begin{displaymath}
\xymatrix{
 \bullet \ar@/^/[r] 
         \ar@/_/@{.>}[r] &
 \bullet
}
\end{displaymath}
\end{example}

������������ ����� ������ ������� ����������, ��� ����� ����������
������. \Xy-pic ���������� ��������� �������� �������� ������
��������� ������; ����������� �������� � ������������ �� \Xy-pic.

% \begin{example}
% \begin{lscommand}
% \ci{dum}
% \end{lscommand}
% \end{example}


%%%%%%%%%%%%%%%%%%%%%%%%%%%%%%%%%%%%%%%%%%%%%%%%%%%%%%%%%%%%%%%%%
% Contents: Customising LaTeX output
% $Id: custom.tex,v 1.1.1.1 2002/02/26 10:04:20 oetiker Exp $
%%%%%%%%%%%%%%%%%%%%%%%%%%%%%%%%%%%%%%%%%%%%%%%%%%%%%%%%%%%%%%%%%
\chapter{��������� \LaTeX}

\begin{intro}
  ������������� � �������������� ��������� �� ��� ��� ������ ���������
  ����� ��������� ������ ��������� ��� ������� ���������. ��� ��
  �������� ����� �����, ���������� ���� ���� �������� �������� ���� �
  ������, ������� �� ����� ������ � �� ��� ������� ��������.

  �� ������ ��������, � ������� \LaTeX{} �� ������������� ������� ���
  ���������, ��������������� ����� ���������, ��� ������������
  ��������� ������������ �������� ����� �� �������� ����� �����������.

  � ���� ����� ������ ��������� ������ �� �������� \LaTeX{} �����
  �����, � ����, ��� ������� ��� ����� ������������ �� ����, ���
  ������������ �� ���������.
\end{intro}


\section{����� �������, ��������� � ������}

��� �� ��������, ��� �������� � ���� ����� ������� ���������� � �����
� ���������� � ��������� � ����� �����. ������ ����, ����� ��������
������������ ����������� ��� ����� ������� \LaTeX{}, ����� ������
\wi{�����}, � ������� ��������� ����� ������� � ��������� ��� ����
����. ������ ����� ������ ������:

\begin{example}
\begin{lscommand}
\ci{dum}
\end{lscommand}
\end{example}

� ���� ������� ������������ ��� ����� ���������, ������������
\ei{lscommand} � ���������� �� ��������� ����� ������ �������, ��� �
����� �������, ������������ \ci{ci} � ���������� �������� ������� �
��������� ��������������� ������� � ���������. �� ������ � ����
���������, ������� ������� \ci{dum} � ��������� � ����� �����, ��� ��
������� ������ ��� \ci{dum}, ����������� �� ��� ��������.

���� ����� �����-������ �����, ��� ��� �� �������� ������ �������,
����������� � �����, �� ������ ������� ����������� ���������
\texttt{lscommand}. ��� ������� �����, ��� ������ �� ����� ���������,
��������� ��� �����, ��� ������������ ����� ������� \LaTeX{} ���
��������� ����� ������ ����.


\subsection{����� �������}

����� �������� ���� ����������� �������, ����������� ��������

\begin{lscommand}
\ci{newcommand}\verb|{|%
       \emph{��������}\verb|}[|\emph{�����}\verb|]{|\emph{�����������}\verb|}|
\end{lscommand}

������ ��� ������� ������� ���� ����������. \emph{��������} �������,
������� �� ��������, � \emph{�����������} �������. ��������
\emph{�����} � ���������� ������� �� ����������. �� ����������� ���
�������� ����� �������, �������, � ���� �������, ��������� �� 9
����������.

��������� ��� ������� ������ ��� ������ �������� ������������� �
�������. ������ ������ ���������� ����� �������, ������������
\ci{tnss}, ��� �������� ����������� �� ``The Not So Short Introduction
to \LaTeXe''. ����� ������� ����������, ���� ��� ����� ��� ����������
������ �������� ���� �����.

\begin{example}
\newcommand{\tnss}{The not
   so Short Introduction to
   \LaTeXe}
% � ���� ���������:
``\tnss'' \ldots{} ``\tnss''
\end{example}

��������� ������ ����������, ��� ���������� ����� �������, �����������
���� ��������. ����� \verb|#1| ���������� �� �������� ��������. ���� ��
������ ������������ ����� ������ ���������, ����������� \verb|#2|, �
��� �����.

\begin{example}
\newcommand{\txsit}[1]
    {\emph{#1} �������
      �������� � \LaTeXe}
% � ���� ���������
\begin{itemize}
\item \txsit{�� �����}
\item \txsit{�����}
\end{itemize}
\end{example}

\LaTeX{} �� �������� ��� ������� ����� �������, ������� �� ��������
��� ������������. �� ��� ������, ����� �� ���� ������ ��������
������������ �������, ���� ����������� �������: \ci{renewcommand}. ���
����� ��� �� ���������, ��� � ������� \verb|\newcommand|.

� ��������� ������� ����� ����������� ������� \ci{providecommand}. ���
�������� ��� ��, ��� \ci{newcommand}, ��, ���� ������� ��� ����������,
�� \LaTeXe{} �� ����� �������������.

���������� ������������ �����������, ��������� � ��������� �����
������ \LaTeX{}. ����������� �������� ��
��������~\pageref{whitespace}. 


\subsection{����� ���������}

����������� ������� \verb|\newcommand|, ���������� ������� ��� ��������
������ ������������ ���������, \ci{newenvironment}, ������� ���������
���������:

\begin{lscommand}
\ci{newenvironment}\verb|{|%
       \emph{��������}\verb|}[|\emph{�����}\verb|]{|%
       \emph{������}\verb|}{|\emph{�����}\verb|}|
\end{lscommand}

������� ������� \verb|\newcommand|, \ci{newenvironment} �����
������������ � �������������� ����������, ��� ��� ����. ��������,
����������� � �������� \emph{������}, �������������� �� ���������
������ ������ ���������. ��������, ����������� � ��������
\emph{�����}, ��������������, ����� ����������� �������
\verb|\end{|\emph{��������}\verb|}|. ��������� ������ ������������
������������� ������� \ci{newenvironment}.

\begin{example}
\newenvironment{king}
 {\rule{1ex}{1ex}%
      \hspace{\stretch{1}}}
 {\hspace{\stretch{1}}%
      \rule{1ex}{1ex}}

\begin{king}
��� ��������� ���������\ldots
\end{king}
\end{example}

�������� \emph{�����} ���������� ��� ��, ��� � ��� �������
\verb|\newcommand|. \LaTeX{} ������������, ����� �� �� ���������� ���
������������ ���������. ���� �� �������� ��� �� ��� �������,
����������� �������� \ci{renewenvironment}. ��� ����� ��� ��
���������, ��� �� \ci{newenvironment}.\sloppypar

�������, �������������� � ���� �������, ����� ���������� �����:
�������� ������� \ci{rule} ��. �� ���.~\pageref{sec:rule}, �������
\ci{stretch} ������� �� ���.~\pageref{cmd:stretch}, � �������� �������
\ci{hspace} ��������� �� ���.~\pageref{sec:hspace}.

\subsection{��� ����������� �����}

����� �� ����������� ��������� ����� ��������� � ������, ���������
����� ���������� ���������� ����� ��������. � ���� ��������
�������������� �������� ������� ����� \LaTeX{}, ���������� �����������
���� ����� ������ � ���������. ����� ����� �������� \ci{usepackage}
������������ ����� � ����� ����������.\sloppypar

\begin{figure}[!htbp]
\begin{lined}{\textwidth}
\begin{verbatim}
% ����� ��� ������������. Tobias Oetiker.
\ProvidesPackage{demopack}
\newcommand{\tnss}{�� ����� ������� �������� � \LaTeXe}
\newcommand{\txsit}[1]{\emph{#1} �������
                       �������� � \LaTeXe}
\newenvironment{king}{\begin{quote}}{\end{quote}}
\end{verbatim}
\end{lined}
\caption{������ ������} \label{package}
\end{figure}

�������� ������ � �������� ������� �� �������� ����������� �����
��������� � ��������� ���� � ������, ��������������� ��
\texttt{.sty}. ���� ������ ���� ����������� �������, ������� �� ������
������������
\begin{lscommand}
\ci{ProvidesPackage}\verb|{|\emph{�������� ������}\verb|}|
\end{lscommand}
\noindent � ����� ������ ����� � �����
�������. \verb|\ProvidesPackage| ��������� \LaTeX{} �������� ������,
��� ��������� ��� �������� ����������� ��������� �� ������, ����� ��
��������� �������� ����� ������. �����������~\ref{package} ����������
��������� ������ ������, ����������� ������������ � ���������������
�������� �������.


\section{������ � �� �������}


\subsection{������� ����� ������}

\index{�����}\index{�����!������} \LaTeX{} �������� ����������
���������� � ������ ������, ����������� �� ���������� ���������
��������� (�������, ������,~\ldots). ������ ����� ���� ����������
������� ����� �������. ��� ����� �� ������ ������������ ���������,
�������������� � ��������~\ref{fonts} �~\ref{sizes}. ��������������
������ ������� ������ ������������ �������� � ������� �� ������ �
����� ���������. �������~\ref{tab:pointsizes} ���������� ����������
�������, ��������������� ���� �������� � ����������� �������
����������.

% ��, �� ��������� ���� ���� :(
%\begin{example}
%{\small The small and
%\textbf{bold} Romans ruled}
%{\Large all of great big
%\textit{Italy}.}
%\end{example}

\begin{example}
  {\small ���������,
    \textbf{����������},
    \Large �������,
    \textit{������}.}
\end{example}

������ ����������� \LaTeXe{} ����������� � ���, ��� �������� ������
����������. ��� ������, ��� �� ������ ������ ������� ����� ������� ���
���� ��������� ������, �������� ��� ���� ��������� ��������� �������
��� ������������.

� \emph{�������������� ������} �� ������ ������������ \emph{�������}
����� ������, ����� �������� ����� �� \emph{��������������� ������} �
������ ���������� �����. ���� �� ������ ������������� �� ������ �����
��� ������� ����������, �� ��� ����� ���������� ��������� �����
������. �������� �������~\ref{mathfonts}.

\begin{table}[!bp]
\caption{������} \label{fonts}
\begin{lined}{12cm}
%
% Alan suggested not to tell about the other form of the command
% eg \verb|\sffamily| or \verb|\bfseries|. This seems a good thing to me.
%
\begin{tabular}{@{}rl@{\qquad}rl@{}}
\fni{textrm}\verb|{...}|        &      \textrm{\wi{������ �����}}&
\fni{textsf}\verb|{...}|        &      \textsf{\wi{��� �������}}\\
\fni{texttt}\verb|{...}|        &      \texttt{������� �������}\\[6pt]
\fni{textmd}\verb|{...}|        &      \textmd{����������}&
\fni{textbf}\verb|{...}|        &      \textbf{\wi{����������}}\\[6pt]
\fni{textup}\verb|{...}|        &       \textup{\wi{������ �����}}&
\fni{textit}\verb|{...}|        &       \textit{\wi{������}}\\
\fni{textsl}\verb|{...}|        &       \textsl{\wi{��������� �����}}&
\fni{textsc}\verb|{...}|        &       \textsc{\wi{��������}}\\[6pt]
\ci{emph}\verb|{...}|          &            \emph{���������� �����} &
\fni{textnormal}\verb|{...}|    &    \textnormal{�������}
\end{tabular}

\bigskip
\end{lined}
\end{table}


\begin{table}[!bp]
\index{�����!������}
\caption{������� ������} \label{sizes}
\begin{lined}{12cm}
\begin{tabular}{@{}ll}
\fni{tiny}      & \tiny        ��������� \\
\fni{scriptsize}   & \scriptsize  ����� ���������\\
\fni{footnotesize} & \footnotesize  �������� ��������� \\
\fni{small}        &  \small        ��������� \\
\fni{normalsize}   &  \normalsize  ���������� \\
\fni{large}        &  \large       �������
\end{tabular}%
\qquad\begin{tabular}{ll@{}}
\fni{Large}        &  \Large       ��� ������ \\[5pt]
\fni{LARGE}        &  \LARGE       ����� ������� \\[5pt]
\fni{huge}         &  \huge        �������� \\[5pt]
\fni{Huge}         &  \Huge        ���������
\end{tabular}

\bigskip
\end{lined}
\end{table}

\begin{table}[!tbp]
\caption{���������� ������� ������� � ����������� �������}
\label{tab:pointsizes}
\label{tab:sizes}
\begin{lined}{12cm}
\begin{tabular}{lrrr}
\multicolumn{1}{c}{\textit{������}} &
\multicolumn{1}{c}{\textit{10pt (�� ���������)}} &
           \multicolumn{1}{c}{\textit{����� 11pt}}  &
           \multicolumn{1}{c}{\textit{����� 12pt}}\\[6pt]
\verb|\tiny|       & 5pt  & 6pt & 6pt\\
\verb|\scriptsize| & 7pt  & 8pt & 8pt\\
\verb|\footnotesize| & 8pt & 9pt & 10pt \\
\verb|\small|        & 9pt & 10pt & 11pt \\
\verb|\normalsize| & 10pt & 11pt & 12pt \\
\verb|\large|      & 12pt & 12pt & 14pt \\
\verb|\Large|      & 14pt & 14pt & 17pt \\
\verb|\LARGE|      & 17pt & 17pt & 20pt\\
\verb|\huge|       & 20pt & 20pt & 25pt\\
\verb|\Huge|       & 25pt & 25pt & 25pt\\
\end{tabular}

\bigskip
\end{lined}
\end{table}

\begin{table}[!bp]
\caption{�������������� ������} \label{mathfonts}
\begin{lined}{\textwidth}
\begin{tabular}{@{}lll@{}}
\textit{�������}&\textit{������}&    \textit{�����}\\[6pt]
\fni{mathcal}\verb|{...}|&    \verb|$\mathcal{B}=c$|&     $\mathcal{B}=c$\\
\fni{mathrm}\verb|{...}|&     \verb|$\mathrm{K}_2$|&      $\mathrm{K}_2$\\
\fni{mathbf}\verb|{...}|&     \verb|$\sum x=\mathbf{v}$|& $\sum x=\mathbf{v}$\\
\fni{mathsf}\verb|{...}|&     \verb|$\mathsf{G\times R}$|&        $\mathsf{G\times R}$\\
\fni{mathtt}\verb|{...}|&     \verb|$\mathtt{L}(b,c)$|&   $\mathtt{L}(b,c)$\\
\fni{mathnormal}\verb|{...}|& \verb|$\mathnormal{R_{19}}\neq R_{19}$|&
$\mathnormal{R_{19}}\neq R_{19}$\\
\fni{mathit}\verb|{...}|&     \verb|$\mathit{ffi}\neq ffi$|& $\mathit{ffi}\neq ffi$
\end{tabular}

\bigskip
\end{lined}
\end{table}

� ����� � ��������� ����� ������� ������ �������� ���� ������
\wi{�������� ������}. ��� ������������ ��� ����������
\emph{�����}. ������ ������������ ������� �������� ����������� ������
\LaTeX{}.\index{�������������}

\begin{example}
  ��� �������� {\LARGE
    ������� � {\small
      ���������} �����}.
\end{example}

�������, �������� �� ������ ������, ������ ����� �� ���������� �����
��������, �� ������ ���� ��������������� ����� ������������� ������
������� �������� �������. ������� ����������� �������� ������ \verb|}|
�� ������ ������ ������� ����. �������� ��������� ������� \verb|\par|
� ��������� ���� ��������\footnote{\texttt{\bs{}par} ������������
  ������ ������.}:
\begin{example}
{\Large �� ������� ���! ���
��������. ������ ���!\par}
\end{example}

\begin{example}
{\Large ��� ���� ��������.
�� �������, ��� � ���.}\par
\end{example}

���� �� ������ ��������� ������� ��������� ������� � ������ ������
������ ��� ������ ����, �� ��� ����� ����� ������������ ���������
���������.

\begin{example}
  \begin{Large}
    ��� ��������. ��
    ��� � ���� ���\ldots
  \end{Large}
\end{example}

\noindent ��� ������� ��� �� �������� ��������� �������� ������.


\subsection{���������!}

��� �������� � ������ ���� �����, ������ �������� ���� ���������
������ ���������, ����� ������ ��� ���������, ������ ��� ���
������������ �������� ���� \LaTeX{}: ���������� ���������� �
���������� �������� ������ ���������. ��� ������, ���, ���� ��
����������� ������ � ���� �� ��������� ����� ������ � ������ ������
��� ������� ������������ ���� ����������, �� ������ ������������
\verb|\newcommand| � ���������� �������, <<�������������>> � ����
������� ����� ������.

\begin{example}
% � ��������� ��� ������
\newcommand{\danger}[1]{\textbf{#1}}
% � ���������
�� \danger{�������} � ��� �������.
��� ������ \danger{�������}
������������ ����������.
\end{example}

���� ������ ����� �� ������������, ��� �� ����� ������ ������, ���
������ ������������ ������ ���������� �������������
���������,\trfootnote{danger} ������
\verb|\textbf|, ��� ������������� ����������� ����� ���� ��������,
��������� ��� ��������� \verb|\textbf| � ���������, �������� �� ������
�� ��� ��������� ��� ���-������ ������.


\subsection{�����}

��� ���������� ������ ����������� � ��� ������� � �� ��������,
��������� ���� ��� ���� �����:

\begin{quote}
  \underline{\textbf{�������\Huge!}} \textit{���}
  \textsf{�\textbf{\LARGE �}\texttt{��}\textsl{��}} ������� \Huge ��
  \tiny ����������� \footnotesize \textbf{�} ����� \small
  \texttt{���������}, \large \textit{���} \normalsize �����
  \textsc{��� ������} � ��� \textsl{\textsf{��������}} ��
  �\large{}�\Large{}�\LARGE{}�\huge{}�\normalsize.
\end{quote}


\section{���������}


\subsection{��������� ����� ��������}

\index{��������!�������������} ���� ��� ����� �\'������ ��������� �����
��������, �� �� �������� ����� �������� ���������� � ��������� �������
\begin{lscommand}
\ci{linespread}\verb|{|\emph{�����������}\verb|}|
\end{lscommand}
\noindent ��� ������ <<����� ������� ���������>> �����������
\verb|\linespread{1.3}|, ��� ������ <<����� ��� ���������>>~---
\verb|\linespread{1.6}|. �� ��������� ���� �����������
�����~1.\index{��������!�������}

�������, ��� ������ �� ������� \ci{linespread} �������� ���������, �
������� ��� �� �������� ��� ����������� �����. �������, ���� � ��� ����
������� ����������� ��� ��������� ������������ ���������, ����� �����������
��������� ��������:

\begin{lscommand}
\verb|\setlength{\baselineskip}{1.5\baselineskip}|
\end{lscommand}

\begin{example}
{\setlength{\baselineskip}%
           {1.5\baselineskip}
���� ����� ������ � ���������� � 
1.5 ���� ������ �����������. ��������
������� \par{} � ����� ������.\par}

����, ����� ������ ���� �����: ��
����������, ��� �� ������������� 
�������� ������� ��� ��������� �
���������� ����������.
\end{example}


\subsection{�������������� �������}
\label{parsp}

��� ��������� � \LaTeX{} ������ �� ������� �������. �������� �
��������� ����������� ����
\begin{code}
\ci{setlength}\verb|{|\ci{parindent}\verb|}{0pt}| \\
\verb|\setlength{|\ci{parskip}\verb|}{1ex plus 0.5ex minus 0.2ex}|
\end{code}
�� �������� ������� ��� �������. ��� ��� ������� �����������
���������� ����� �������� � ������������� �������� ������ ������
����. 

����� \texttt{plus} � \texttt{minus} ������� \TeX{}, ��� �� �����
����������� � ��������� �������� ����� �������� �� ��������� ��������,
���� ��� ���������� ��� ����������� ���������� ������� �� ��������.

� ������ ������ ����� �������� ��������� � �� ������ � ���
�������. ������, ������ � ����, ��� ��� ������ ����� � �� ����������:
��� ������ ���� ���������� ������ ����� ������������. ����� �����
��������, ��� ������� ����� ��������� �� ��������� ���������
����-������ ����� \verb|\tableofcontents|, ��� �� ������������ ��
������, ������ ��� � ���������������� ������� ������� ������������
��������� ������� ������� �������, � �� ���������.

���� �� ������ ������� �������� ������ � �� ������� ��� ������, ��
�������� � ������ ������ �������\footnote{��� ���������� ������� �
  ������� ������ ����� ������� ��������� ������� ����������� �������
  \pai{indentfirst} �� ��������� `tools'.}
\begin{lscommand}
\ci{indent}
\end{lscommand}
\noindent �������, ��� ������ �� ��� ����� ������ ����
\verb|\parindent| �� ���������� ������ ����.

��� �������� ������ ��� ������� ������ �������� ������ ����� �������
\begin{lscommand}
\ci{noindent}
\end{lscommand}
\noindent ��� ����� ���� ������, ����� �� ��������� �������� � ������,
� �� � ������� ���������������.


\subsection{�������������� ���������}

\label{sec:hspace}
\LaTeX{} ������������� ���������� ������� ����� ������� �
�������������. ����� �������� ��������������
������\index{������!��������������}, �����������
\begin{lscommand}
\ci{hspace}\verb|{|\emph{�����}\verb|}|
\end{lscommand}
���� ����� �������� ������ ���� ��������, ���� ���� �� ���������� ��
������ ��� ����� ������, ����������� \verb|\hspace*|, � ��
\verb|\hspace|. � ���������� ������ \emph{�����}~--- ��� ������ �����
� ������� ���������. �������� ������ ������� ����������� �
�������~\ref{units}. \index{�������}\index{�����}

\begin{example}
���\hspace{1.5cm}������
� 1,5��.
\end{example}

%\suppressfloats
\begin{table}[tbp]
\caption{������� ����������� � \TeX{}} \label{units}\index{�������}
\begin{lined}{9.5cm}
\begin{tabular}{@{}ll@{}}
\texttt{mm} & ��������� $\approx 1/25$~����� \quad \demowidth{1mm} \\
\texttt{cm} & ��������� $=$ 10~mm  \quad \demowidth{1cm}                     \\
\texttt{in} & inch $=$ 25.4~mm \quad \demowidth{1in}                    \\
\texttt{pt} & ����� $\approx 1/72$~����� $\approx \frac{1}{3}$~mm  \quad\demowidth{1pt}\\
\texttt{em} & ��������� ������ ����� `M' �������� ������ \quad \demowidth{1em}\\
\texttt{ex} & ��������� ������ ����� `x' �������� ������ \quad \demowidth{1ex}
\end{tabular}

\bigskip
\end{lined}
\end{table}

\label{cmd:stretch}
�������
\begin{lscommand}
\ci{stretch}\verb|{|\emph{n}\verb|}|
\end{lscommand}
\noindent ���������� ����������� <<���������>> ������. ��
�������������, �������� ��� ���������� ����� �� ������. ���� �� �����
������ ����������� ��� �������
\verb|\hspace{\stretch{|\emph{n}\verb|}}|, �� ��� �������������
��������������� ����� �������������.\sloppypar

\begin{example}
x\hspace{\stretch{1}}
x\hspace{\stretch{3}}x
\end{example}

�p� ������������� ��p����������� ����p����� ������ � ������� �����
����� ����� ����p�p����� ����p���, p����p ����p��� ����������� �
p����p�� �������� �p����. ����� ����� �������� �p� ������
������������� ������ p����p����� \texttt{em} � \texttt{en}:

\begin{example}
{\Large{}big\hspace{1em}y}\\
{\tiny{}tin\hspace{1em}y}
\end{example}
 

\subsection{������������ ���������}

��������� ����� ��������, ���������, ������������,~\ldots\
������������ \LaTeX{} �������������. ���  ������������� ��������������
������ \emph{����� ����� ��������} ����� �������� ��������
\begin{lscommand}
\ci{vspace}\verb|{|\emph{�����}\verb|}|
\end{lscommand}

������ ��� ������� ����������� ����� ����� ������� ���������. ���� ���
������������ ������ ����������� ������ ��� ����� ��������, �����������
������� ������� �� ����������: \verb|\vspace*|.\index{������!������������}

������� \verb|\stretch| ������ � \verb|\pagebreak| ����� ��������� ���
������� ������ �� ��������� ������ �������� ��� ��� �������������
������������� ������ �� ��������.
\begin{code}
\begin{verbatim}
����� �����\ldots

\vspace{\stretch{1}}
� ��� �������� �� ��������� ������ ��������.\pagebreak
\end{verbatim}
\end{code}

�������������� ������ ����� ����� �������� \emph{������} ������ ���
������ ������� ����������� ��������
\begin{lscommand}
\ci{\bs}\verb|[|\emph{�����}\verb|]|
\end{lscommand}

��� ������ \ci{bigskip} � \ci{smallskip} �� ������ ���������� �������
������������ ������������ ���������, �� ����������� � ����������
������. 

\section{���������� ��������}

\begin{figure}[!hp]
\begin{center}
\makeatletter\@layout\makeatother
\end{center}
\vspace*{1.8cm}
\caption{��������� ���������� ��������}
\label{fig:layout}
\end{figure}

\index{���������� ��������}
\LaTeXe{} ��������� ������� \wi{������ ������} � �������
\verb|\documentclass|. ����� �� ������������� �������� ������
\wi{����}. �� ������ ���������������� �������� ����� ��� ��
��������. ����������, �� �� ������ ��������.
%no idea why this is needed here ...
\thispagestyle{fancyplain} �����������~\ref{fig:layout} ���������� ���
���������, ������� ����� ��������. ��� ���� ������������� �������
\pai{layout} �� ���������
`tools'.\footnote{\texttt{\CTAN|macros/latex/required/tools|}}

\textbf{���������!} \ldots ������, ��� ���������� ��������� ������ ���
������� ����� �������� ������ ������, ��������� ��������� ������ ��
�����������. ������� ������ �����, ����� ���������� �������� �
\LaTeX{} ������ ��������.

����������, ���� �������� �� ���������, �������� ������������������ MS
Word, �� �������� \LaTeX{} �������� ������ ������. ������, ���������
�� ���� ������� �����\footnote{���� � ���� ��������� �������� �����,
  ���������� ��������� �������������.} � ���������� ���������� ���� ��
����� �������. �� ����������, ��� �� ������ ������� �� ������ 66
����. ������ ��������� ��� �� ��������� \LaTeX{}. �� �������, ��� �
����� ���� ����� 66 ���� � ������. ���� ����������, ��� ��� �������
���������� ���� ������ ������������, ������, ��� ������ ����������
������� ���������� �� ����� ����� ������ � ������ ���������. ������
������� ������ ����� ���������� � ��������� �������.

��� ���, ���������� ������ ������ ������, ������ � ����, ��� ��
����������� ����� ��� ���������. ������, ���������� ��������������,
��� ��� ������ ������� � ���, ��� �� ��� �������\ldots

\LaTeX{} ������������� ��� ������� ��� ��������� ���� ����������. ��
������ ���������� � ��������� ���������.

������ ������� ����������� ������������� �������� ������ ���������:
\begin{lscommand}
\ci{setlength}\verb|{|\emph{��������}\verb|}{|\emph{�����}\verb|}|
\end{lscommand}

������ ������� ���������� ����� � ������ ���������:
\begin{lscommand}
\ci{addtolength}\verb|{|\emph{��������}\verb|}{|\emph{�����}\verb|}|
\end{lscommand}

��� ���� ����� �������, ��� \ci{setlength}, ������ ��� ��������� ���
������ ��������� ������������ ������������ ���������. ����� ��������
��������� � ����� ������ ������, ��������, � ��������� ����� ���������
���������:
\begin{code}
\verb|\addtolength{\hoffset}{-0.5cm}|\\
\verb|\addtolength{\textwidth}{1cm}|
\end{code}

����� ��� ����� ���� ��������� ����� \pai{calc}, ������� ���������
������������ �������������� �������� � ��������� \verb|\setlength| � �
������ ������, ��� ����������� �������� ��������� ��������.


\section{��� � ������}

������, ����� ��� ��������, ��������� ������������ � ����������
���������� �������. ����� ������������� �� ������ ��� ������ ������
��������� ��������. ��� ������ ����������� ���� ����� �������
\verb|\textwidth|, ����� ��� ��������� �������� �������.

��������� ��� ������� ��������� ���������� ������, ������ � �������
��������� ������.

\begin{lscommand}
\ci{settoheight}\verb|{|\emph{��p�������}\verb|}{|\emph{�����}\verb|}|\\
\ci{settodepth}\verb|{|\emph{��p�������}\verb|}{|\emph{�����}\verb|}|\\
\ci{settowidth}\verb|{|\emph{��p�������}\verb|}{|\emph{�����}\verb|}|
\end{lscommand}

\noindent ������������� ������ ���������� ��������� ���������� ����
������.

\begin{example}
\flushleft
\newenvironment{vardesc}[1]{%
  \settowidth{\parindent}{#1:\ }
  \makebox[0pt][r]{#1:\ }}{}

\begin{displaymath}
a^2+b^2=c^2
\end{displaymath}

\begin{vardesc}{���}$a$,
$b$ -- ��������� � ������� ����
�������������� ������������.

$c$ -- �������� ����������
����� ������������.

$d$ -- ������ ��� �� ���������.
��� �������\ldots
\end{vardesc}
\end{example}


\section{�����}

\LaTeX{} ����������� ��������, ���������� �����. ������� ������ �����
�������� ��������� ������, ������� ������������� � ������ ������,
�������� �����. ����� ����������� � ������� �������, �� �����������
���������� �����, ������� ����� ������������� ��� ���������, ���,
����� � �������� ��������� ������.

���� ��������, ��� ��� �������� ���������� ������ ����, ��� ����������
�� ����� ����, �� ���� � ���, ��� \TeX{} ������ �������� � ������� �
�����. �� ������ ����� ����� ���� ������. �� ������ ��������� � ����
����������� ���, ��� ������, �� �������� � ������ �����. ������ ����
����� �������������� \LaTeX{}, ��� ���� �� ��� ���� ��������� �����.

� ���������� ������ �� ��� ��������� ��������� �����, ���� �� ���� �
�� ����������. ��������� ����� ���� ��������� \ei{tabular} ���
\ci{includegraphics}, ��� ������������ ����. ��� ������, ��� �� �����
������ ��������� ����� ��� ������� ��� �����������. ������ ���������,
��� �� ����� ������ �� ��������� \verb|\textwidth|.

�� ����� ������ ��������� ����� ����� � ���� ��� ��������

\begin{lscommand}
\ci{parbox}\verb|[|\emph{���}\verb|]{|\emph{������}\verb|}{|\emph{�����}\verb|}|
\end{lscommand}

\noindent ��� ����������

\begin{lscommand}
\verb|\begin{|\ei{minipage}\verb|}[|\emph{���}\verb|]{|\emph{������}\verb|}| �����
\verb|\end{|\ei{minipage}\verb|}|
\end{lscommand}

�������� \emph{���} ����� ��������� ���� �� ���� \texttt{c, t} ���
\texttt{b}, ����������� ������������ ������������ ����� �� ��������� �
������� ����� ����������� ������. \emph{������} ��������� ����������
�����, ������������ ������ �����. �������� ������� ����� \ci{minipage}
� \ci{parbox}~--- � ���, ��� ������ \ci{parbox} ����� ������������ ��
��� ������� � ���������, ����� ��� ������ \ci{minipage} �����
����������� ���.

� �� �����, ��� \ci{parbox} ����������� ����� �����, �������� �������
� ������, ���������� ����� �������� ������, ���������� ������ ��
������������� ������������� ���������. ���� �� ��� �� ��� �����. ���
���������� \ci{mbox} � ������ ����������� ������������������ ������,
��� ����� ������������ ��� �������������� �������� \LaTeX{} ����
����. ��� ��� �� ������ �������� ���� ����� � ������, ��� ����������
�������������� ������ ����������� �����.

\begin{lscommand}
\ci{makebox}\verb|[|\emph{������}\verb|][|\emph{���}\verb|]{|\emph{�����}\verb|}|
\end{lscommand}

\noindent \emph{������} ���������� ������ ��������������� ����� ���,
��� ��� ����� �������.\footnote{��� ��������, ��� ��� ����� ����
  ������, ��� �������� ������ �����. � ���������� ������ �� ������
  ���� ���������� �� � 0pt, ��� ��� ����� ������ ����� ����������,
  ������ �� �������� ������� �� ���������� �����.} ����� ���������
�����, �� ��� ������ ������������ \ci{width}, \ci{height}, \ci{depth}
� \ci{totalheight}. ��� ��������������� ������� ���������, ����������
���������� ���������� \emph{������}.\trfootnote{������, ������, �������
  � ����� ������ (������ ���� �������) ������,
  ��������������.} �������� \emph{���}
��������� ������������� ��������: \textbf{c}: ������������,
\textbf{l}: ������ �����, \textbf{r}: ������ ������ ��� \textbf{s}:
���������� ��������� ���� �������.

������� \ci{framebox} �������� � �������� ��� ��, ��� \ci{makebox}, ��
������ ����� ������ ������.

��������� ������ ���������� ��������� ����������� ������������� ������
\ci{makebox} � \ci{framebox}.

\begin{example}
\makebox[\textwidth]{%
    � � � � �}\par
\makebox[\textwidth][s]{%
    � � � � � � � � � �}\par
\framebox[1.1\width]{� ������
  � �����!} \par
\framebox[0.8\width][r]{��,
    � ������� �������} \par
\framebox[1cm][l]{������,
  � ����}
������ ��� ���������?
\end{example}

������, ����� �� ��������� ������������, ��������� ��������� ���~---
���������. ������� �������. �������

\begin{lscommand}
\ci{raisebox}\verb|{|\emph{�����}\verb|}[|\emph{�������}\verb|][|\emph{������}\verb|]{|\emph{�����}\verb|}|
\end{lscommand}

\noindent ��������� ��� ���������� ������������ ��������������
�����. � ������ ���� ���������� ����� ������������ \ci{width},
\ci{height}, \ci{depth} � \ci{totalwidth}, ����� �������� �������
��������� \emph{�����}.

\begin{example}
\raisebox{0pt}[0pt][0pt]{\Large%
\textbf{Aaaa\raisebox{-0.3ex}{a}%
\raisebox{-0.7ex}{aa}%
\raisebox{-1.2ex}{a}%
\raisebox{-2.2ex}{a}%
\raisebox{-4.5ex}{a}}}
������ ��, �� ���� ������� �����
�� �������, ��� � ��� ���������
���-�� �������.
\end{example}


\section{������� � ��������}
\label{sec:rule}

��������� ������� ����� �� ����� �������� �������

\begin{lscommand}
\ci{rule}\verb|[|\emph{�����}\verb|]{|\emph{������}\verb|}{|\emph{������}\verb|}|
\end{lscommand}

\noindent ��� ������� ������������� ��� ���������� ������� ������
����.

\newpage
\begin{example}
\rule{3mm}{.1pt}%
\rule[-1mm]{5mm}{1cm}%
\rule{3mm}{.1pt}%
\rule[1mm]{1cm}{5mm}%
\rule{3mm}{.1pt}
\end{example}

\noindent ��� ����� ������������ ��� ��������� ������������ �
�������������� �����. ��������, ����� �� ��������� ����� ����������
�������� \ci{rule}.

����������� ������� �������� �������, � ������� ��� ������, �� ����
������������ ������. � ���������������� ������� �� ��������
\emph{���������}\index{��������}. �� ����������, ����� ����������
������������ ����������� ������ �������� ��������. �� ������
������������ ��, ����� ������� ������ ��������� \texttt{tabular}
������� ������������ ����������� ������.

\begin{example}
\begin{tabular}{|c|}
\hline
\rule{0pt}{4ex}Pitprop \ldots\\
\hline
\rule{0pt}{4ex}Strut\\
\hline
\end{tabular}
\end{example}

%%% Local Variables:
%%% mode: latex
%%% TeX-master: "lshort2e"
%%% End:

\bigskip
{\flushright �����.\par}

%

\backmatter
%%%%%%%%%%%%%%%%%%%%%%%%%%%%%%%%%%%%%%%%%%%%%%%%%%%%%%%%%%%%%%%%%
% Contents: The Bibliography
% File: biblio.tex (lshort2e.tex)
% $Id: biblio.tex,v 1.1.1.1 2002/02/26 10:04:20 oetiker Exp  $
%%%%%%%%%%%%%%%%%%%%%%%%%%%%%%%%%%%%%%%%%%%%%%%%%%%%%%%%%%%%%%%%%
\begin{thebibliography}{99}
\addcontentsline{toc}{chapter}{\bibname}
\bibitem{manual} Leslie Lamport.  \newblock \emph{{\LaTeX:} A Document
    Preparation System}.  \newblock Addison-Wesley, Reading,
  Massachusetts, ������ �������, 1994, ISBN~0-201-52983-1.

\bibitem{texbook} Donald~E. Knuth.  \newblock \textit{The \TeX{}book,}
  Volume~A of \textit{Computers and Typesetting}, Addison-Wesley,
  Reading, Massachusetts, second edition, 1984, ISBN~0-201-13448-9.

\bibitem{companion} Michel Goossens, Frank Mittelbach and Alexander
  Samarin.  \newblock \emph{The {\LaTeX} Companion}.  \newblock
  Addison-Wesley, Reading, Massachusetts, 1994,
  ISBN~0-201-54199-8.\footnote{����� ������� �������: �.�������,
  �.����������, �.�������. \emph{������������ �� ������ \LaTeX{} � ���
  ���������� \LaTeXe{}}. ���, 1999, ISBN~5-03-003325-4.}

\bibitem{graphicscompanion} Michel Goossens, Sebastian Rahtz and Frank
  Mittelbach.  \newblock \emph{The {\LaTeX} Graphics Companion}.  \newblock
  Addison-Wesley, Reading, Massachusetts, 1997,
  ISBN~0-201-85469-4.\footnote{����� ������� �������: �.�������,
  �.���� � �.����������. \emph{������������ �� ������ \LaTeX{} � ���
  ����������� �����������}. ���, 2002, ISBN~5-03-003388-2.}

\bibitem{local} ������ ��������� \LaTeX{} ������ ��������� ���
  ���������� \emph{\LaTeX{} Local Guide}, ����������� �����������
  ��������� �������. �� ������ ���������� � �����, ������������
  \texttt{local.tex}. � ���������, ��������� ������� ��������������
  ������ ��������� �� �������������. � ����� ������ ������� � ������
  �������� \LaTeX{} ����.

\bibitem{usrguide} \LaTeX3 Project Team.  \newblock \emph{\LaTeXe~for
    authors}.  \newblock ������� � �������� \LaTeXe{} ���
  \texttt{usrguide.tex}.

\bibitem{clsguide} \LaTeX3 Project Team.  \newblock \emph{\LaTeXe~for
    Class and Package writers}.  \newblock ������� � �������� \LaTeXe{}
  ��� \texttt{clsguide.tex}.

\bibitem{fntguide} \LaTeX3 Project Team.  \newblock \emph{\LaTeXe~Font
    selection}.  \newblock ������� � �������� \LaTeXe{} ���
  \texttt{fntguide.tex}.

\bibitem{graphics} D.~P.~Carlisle.  \newblock \emph{Packages in the
    `graphics' bundle}.  \newblock ������ � ������ ��������� `graphics' ���
  \texttt{grfguide.tex}, �������� ������ ��, ������ ���� �������� \LaTeX{}.

\bibitem{verbatim} Rainer~Sch\"opf, Bernd~Raichle, Chris~Rowley.
\newblock \emph{A New Implementation of \LaTeX's verbatim
  Environments}.
 \newblock ������ � ������ ��������� `tools' ���
  \texttt{verbatim.dtx}, �������� ������ ��, ������ ���� �������� \LaTeX{}.

\bibitem{cyrguide} Vladimir Volovich, Werner Lemberg and \LaTeX3 Project Team.
    \newblock \emph{Cyrillic languages support in \LaTeX}.  \newblock ������� �
    �������� \LaTeXe{} ��� \texttt{cyrguide.tex}.

\bibitem{catalogue} Graham~Williams.  \newblock \emph{The TeX
    Catalogue} ������ ������ ��������� �������, ������� ��������� �
  \TeX{} � \LaTeX{} \newblock �������� � �������� �� ������
  \texttt{\CTAN|help/Catalogue/catalogue.html|}

\bibitem{eps} Keith~Reckdahl.  \newblock \emph{Using EPS Graphics in
    \LaTeXe{} Documents} ��������� ���, ��� �� ����� �� �� �� ����
  ������ ����� ��� EPS ����� � �� ������������� � ���������� \LaTeX{}.
  \newblock �������� � �������� �� ������
  \texttt{\CTAN|info/epslatex.ps|}

\bibitem{xy-pic} Kristoffer H. Rose.
  \newblock \emph{\Xy-pic User's Guide}.  \newblock
  �������� � ����p��� � CTAN ������ � ����p�������� \Xy-pic.

\bibitem{metapost} John D. Hobby.
  \newblock \emph{A User's Manual for MetaPost}. \newblock
  �������� �� ������ \url{http://cm.bell-labs.com/who/hobby/} 
  
\bibitem{unbound} Alan Hoenig.
  \newblock \emph{\TeX{} Unbound}. \newblock Oxford University Press, 1998,
    ISBN 0-19-509685-1; 0-19-509686-X (pbk.) 
  
\bibitem{ursoswald} Urs Oswald.  
    \newblock \emph{Graphics in \LaTeXe{}}, �������� ����� �������� ������� ��
    Java ��� ��������� ������������ ����������� � �������� �� ��������� \texttt{picture},
    � \emph{MetaPost - A Tutorial}.
  \newblock ��� �������� �� ������ \url{http://www.ursoswald.ch}

\end{thebibliography}


%%% Local Variables:
%%% mode: latex
%%% TeX-master: "lshort2e"
%%% End:

\refstepcounter{chapter}
\addcontentsline{toc}{chapter}{���������� ���������} 
\printindex
\refstepcounter{chapter}
\label{verylast}
\mbox{}
\end{document}
