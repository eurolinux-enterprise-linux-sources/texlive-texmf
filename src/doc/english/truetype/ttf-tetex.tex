\documentclass[a4paper,fleqn,11pt]{article}
\usepackage[T1]{fontenc} % use T1 fonts
\usepackage{graphicx}    % graphic inclusion commands
\usepackage{url}         % internet addresses
\newcommand*{\orgurl}{}
\let\orgurl=\url  % save original url command in orgurl
\usepackage{html}        % make compilable for latex2html
                         % hint by Heiko Oberdiek, oberdiek@ruf.uni-freiburg.de
\usepackage{hyperref}    % hyper references
\usepackage{thumbpdf}    % pdf thumbnails

% setup for pdf file format
\hypersetup{%
  pdftitle={Using TrueType fonts with \TeX\ via Postscript Type\,1 format},
  pdfsubject={Using TrueType fonts with \TeX\ via Postscript Type\,1 format},
  pdfkeywords={truetype, pdftex, pdflatex, dvips, type1, postscript,
    font, tex, latex},
  pdfauthor={Harald Harders (mailto:h.harders@tu-bs.de)},
  pdfcreator={\LaTeX},
  pdfproducer={pdf\LaTeX},
  bookmarksopen=true,
  bookmarksnumbered=true,
  pdfstartview={FitH 1000},
%  pdfview={Fit H 1000}
}
% use color links only in pdf file (because ps file does not have
% active links)
\makeatletter
\ifx\pdftexversion\undefined
\else
  \hypersetup{colorlinks=true}
\fi
%\@ifundefined{pdfoutput}{}{\hypersetup{colorlinks=true}}
\makeatother

% one command for references in both pdf and html output
\latexhtml{%
  \newcommand{\hypref}[2]{\href{#2}{\texttt{#1}}}
}{%
  \newcommand{\hypref}[2]{\htmladdnormallink{\texttt{#1}}{#2}}
}
%\newcommand{\hypref}[2]{\latexhtml{\href{#2}{\texttt{#1}}}
%  {\htmladdnormallink{\texttt{#1}}{#2}}}

% data for the title page
\title{Using TrueType fonts with \TeX\ via Postscript Type\,1 format}
\author{Harald Harders 
  (\hypref{\orgurl{h.harders@tu-bs.de}}{mailto:h.harders@tu-bs.de})}
\date{2003-10-31}

% -----------------------------------------------------------------------
\begin{document}
% -----------------------------------------------------------------------

% titlepage
\maketitle

\begin{abstract}
\noindent This document describes how I have managed to use TrueType
fonts with te\TeX\ 1.0 under SuSE Linux 6.2.\footnote{The updmap
  section has been updated to te\TeX\ 2.0.} The way described in
``Using TrueType fonts with \TeX\ (\LaTeX) and pdf\TeX\ (pdf\LaTeX)''
\cite{rakityansky} did work with MiK\TeX, but I did not manage to use
the fonts with te\TeX. 

Strictly speaking this document doesn't describe how to use TrueType
fonts with te\TeX\ but how to convert TrueType fonts to PostScript
Type\,1 format which can be used with \TeX.

The shown shell commands are unix style. 
Similar commands should also work under Windows.
Please try yourself to find the according commands.

If I have made any errors or if you have a suggestion please mail it
to me.

\latexhtml{%
  This document can be downloaded from
  \url{ftp://ftp.dante.de/tex-archive/info/TrueType/}
  (PDF and html available).%
}{%
  A \hypref{PDF version%
    }{ftp://ftp.dante.de/tex-archive/info/TrueType/ttf-tetex.pdf}
  is also available.%
}
\end{abstract}

\tableofcontents

% the document itself
\section{Disclaimer}

This document is provides as is.
I will not guarantee that the described conversion works and that it
doesn't course any damage.
I also will not give any support for problems doing the conversion.
Please refer to the newsgroups, e.g., \orgurl{comp.text.tex} or
\orgurl{de.comp.text.tex}.
My email address is given to give feedback (suggestions, error
reports), not for support.

I don't know if the shown conversion violates any copyrights.



\section{Preparing the conversion}

As an example I will show how to convert the font ``VAG Rounded BT'' which
is part of Microsoft Windows 98 (\verb|tt0756m_.TTF|).

First copy the fonts you want to convert into a temporary directory
(e.g., a Windows disk is mounted on /dos/c):
\begin{quote}
\verb|$ mkdir ~/ttf|\\ %$
% The comment in the line above is a workaround for the corrupted 
% xemacs syntax highlightning which does not recognize that the $ sign 
% is within the argument of the \verb command.
\verb|$ cp /dos/c/windows/fonts/tt0756m_.TTF ~/ttf|  %$
\end{quote}
Then rename the files to a name conforming the fontname scheme by K.~Berry
\cite{berry1999a}. In this case the supplier is ``Bitstream'' (Filename
\verb|b*******.ttf|)\footnote{You can find this out by viewing the
  file with less.}. The Shortcut for the typeface is ``vr'' (Filename
\verb|*vr*****.ttf|) taken from \cite{berry1999a}. The weight is
``regular'' (Filename \verb|***r****.ttf|). The variant is omitted
because itself and the width are normal. The encoding is set to ``8a''
which means Adobe standard encoding\footnote{Maybe the TrueType font
  is in Windows encoding, but the approach using ``8a'' worked fine,
  so I won't change it.} (Filename \verb|****8a**.ttf|). Because the
width is standard and the font is linearly scaled, these parts of the
filename are omitted. Finally the filename results in \verb|bvrr8a.ttf|. Move
the original file to this filename: 
\begin{quote}
\verb|$ mv tt0756m_.TTF bvrr8a.ttf| %$
\end{quote}
A more detailed description on the naming conventions can be found in
\cite{berry1999a}. 

\section{Generating the Postscript Type\,1 fonts}

To convert the TrueType font to Postscript Type\,1 format I used the program
\verb|ttf2pt1| by Andrew Weeks et~al.
(\url{http://ttf2pt1.sourceforge.net/}). Generate the font files
\verb|bvrr8a.afm|, \verb|bvrr8a.pfa|, and \verb|bvrr8a.pfb| by using
these commands: 
\begin{quote}
\verb|$ ttf2pt1 -a -e bvrr8a.ttf bvrr8a|\\
\verb|$ ttf2pt1 -a -b bvrr8a.ttf bvrr8a|
\end{quote}
The switch \verb|-a| switches the conversion of some ligatures
(e.g., fi) on (thanks to Hume Smith for that hint).
In one of the last lines of the output the fontname is noted:
\begin{quote}
\verb|FontName VAGRoundedBT_Regular|
\end{quote}
Note the name on a sheet of paper---You will need it later again.

The script
\hypref{ttf2type1}{ftp://ftp.dante.de/tex-archive/info/TrueType/ttf2type1}
does these conversion automaticaly for all files with 
the extension \verb|ttf| in the present working directory. To get the
font names you should start it as follows:
\begin{quote}
\verb_$ ./ttf2type1 2>&1 | grep FontName_ %$
\end{quote}

\section{Generating the \TeX\ related font files}

Use ``fontinst'' by Alan Jeffrey and Rowland McDonnell
(\url{ftp://ftp.tex.ac.uk/tex-archive/fonts/utilities/fontinst}) to
generate the files that \TeX\ needs to use the fonts:
\begin{quote}
\verb|$ tex fontinst.sty|\\
\verb|* \latinfamily{bvr}{} \bye| %$
\end{quote}
If you use fonts with different variants you have to append the letter of
the variant to the family name of the font (e.g., VAGRoundedBT\_Condensed
would be bvrc).
This automatic conversion only works if you are using a text font and
if you have used file names according to the fontname scheme by
K.~Berry~\cite{berry1999a}.
Otherwise, you will get some nearly empty \verb|.fd| files and nothing
more.

Now use pltotf on every file with the extension \verb|.pl| and
vptovf on all files with the extension \verb|.vpl|:
\begin{quote}
\verb|$ for a in *.pl; do pltotf $a; done|\\
\verb|$ for a in *.vpl; do vptovf $a; done| 
\end{quote}
Now you may delete all files that are not used anymore:
\begin{quote}
\verb|$ rm *.pl *.vpl *.mtx| %$
\end{quote}
The manual of the fontinst package includes a better description.

\section{Move the files to the right places}

Now all files have to be moved to a position where \TeX\ can find them. I
suggest to put them in the \verb|TEXMFLOCAL| tree. 
One possibility to get its location is to view the file \verb|texmf.cnf|.
You can locate it by using \verb|kpsewhich|:
\begin{quote}
\verb|$ kpsewhich texmf.cnf| %$
\end{quote}
e.g., on SuSE~6.2 and te\TeX~1.0 \verb|texmf.cnf| is located in the directory
\verb|/etc/texmf/|.
Another possibility to get \verb|TEXMFLOCAL| is to use \verb|kpsexpand|:
\begin{quote}
\verb|$ kpsexpand '$TEXMFLOCAL'|
\end{quote}
On my computer the \verb|TEXMFLOCAL| tree starts at
\verb|/usr/local/teTeX/share/texmf.local|. The \verb|TEXMFMAIN| tree starts at
\verb|/usr/local/teTeX/share/texmf|.
In order to have less work I set the shell variable \verb|TMF| to the
local \TeX\ tree by typing:
\begin{quote}
\verb|$ export TMF=`kpsexpand '$TEXMFLOCAL'`|%$
\end{quote}


The files of each file type are installed in an own directory tree
which has this structure:
\begin{quote}
\verb|$TMF/fonts/|<extension>\verb|/|<%
supplier>\verb|/|<fontname>\verb|/|
\end{quote}
In this case:
\begin{quote}
\verb|$TMF/fonts/|<extension>\verb|/bitstream/vagrounded/|
\end{quote}
The extensions are: \verb|tfm|, \verb|vf|, \verb|pfa|, 
\verb|afm|, and \verb|ttf|. 
The extension \verb|pfb| is an exception, its files have to be copied
to the subdirectory \dots\verb|/type1/|\dots.
Copy the files by typing:
\begin{quote}
\verb|$ for a in tfm vf pfa afm ttf; do|\\ %$
\verb|> mkdirhier $TMF/fonts/$a/bitstream/vagrounded;|\\
\verb|> mv *.$a $TMF/fonts/$a/bitstream/vagrounded;|\\ %$
\verb|> done|\\
\verb|$ mkdirhier $TMF/fonts/type1/bitstream/vagrounded;|\\ %$
\verb|$ mv *.pfb $TMF/fonts/type1/bitstream/vagrounded;|\\ %$
\end{quote}
You do not really need to copy the \verb|ttf| and \verb|pfa| files
into the directory because \TeX\ does not use them. 
I just did it to save them at a special place where I surely find
them, if I need them for other purposes later.

Move the \verb|*.fd| files to the directory
\verb|$TMF/tex/latex/psnfss/|: %$
\begin{quote}
\verb|$ mkdirhier $TMF/tex/latex/psnfss|\\
\verb|$ mv *.fd $TMF/tex/latex/psnfss|\\
\end{quote}

\section{Make dvips find the new font}\label{sec:mapping}

There are at least two possibilities to make dvips find the new font. 
The first has a simple installation but it' usage is a little bit more
complicated and it does not enable xdvi to use the font. The second
possibility has a more complicated installation
and may leed to problems when updating \LaTeX\ later. But it enables
xdvi to use the new fonts.

\subsection{Use an additional map file}

Create the file
\hypref{\$TMF/dvips/vagrounded/config.vagrounded}
{ftp://ftp.dante.de/tex-archive/info/TrueType/config.vagrounded} 
with these contents:
\begin{quote}
\verb|o|\\
\verb|p +vagrounded.map|
\end{quote}
Create the file \hypref{\$TMF/dvips/vagrounded/vagrounded.map}
{ftp://ftp.dante.de/tex-archive/info/TrueType/vagrounded.map} 
with these contents (\emph{Each font definition is in one single
  line.} So in this example each line starts with \verb|bvr| and ends
with \verb|<bvrr8a.pfb|. \latex{The $\backslash$ just shows that the
  line is continued in the next one; it does not appear in the file itself.}):
\begin{quote}
\verb|bvrr8r VAGRoundedBT_Regular|\latex{ $\backslash$\\}%
\verb|     "TeXBase1Encoding ReEncodeFont" <8r.enc <bvrr8a.pfb|\\
\verb|bvrro8r VAGRoundedBT_Regular|\latex{ $\backslash$\\}%
\verb|     "0.167 SlantFont TeXBase1Encoding ReEncodeFont"|\latex{ $\backslash$\\}%
\verb|     <8r.enc <bvrr8a.pfb|
\end{quote}%

The first item is the filename of the TrueType font with ``8r'' instead of
``8a''. The second item is the font name you held in mind,
hopefully. The next items are the same all times. The last one is the
filename of the TrueType font with the extension \verb|.pfb|.
Don't use the tabulator character in the mapping file because this
causes trouble with the \verb|updmap| tool generating the map file for 
pdf\LaTeX. Use one single space instead.

In the second line the slanted shape of the font is defined. 
The fontinst package generates slanted, italic and small capital
shapes of the font automatically if no special font file is available.
To use the generated slanted shape the second line is necessary.

Additional font effects can be reached by using afm2tfm. Type 
\begin{quote}
\verb|$ info afm2tfm| %$
\end{quote}
and go to the section ``Special font effects'' (This was a hint of
Thomas Henlich (\hypref{\orgurl{henlich@mmers1.mw.tu-dresden.de}}
{mailto:henlich@mmers1.mw.tu-dresden.de})).

Finally type
\begin{quote}
\verb|$ texhash| %$
\end{quote}
to update the \TeX\ file database.


\subsection{Append data to the global map file}

This technique was suggested by Nguy{\^e}n-Dai Qu�
(\hypref{\orgurl{daiquy.nguyen@ulg.ac.be}}
{mailto:daiquy.nguyen@ulg.ac.be}) \cite{quy2000a}.
There are different possibilities to append the map lines to the
global map file.
Here, only the simplest (and best) possibility is described.
It works with all relativley recent te\TeX\ versions.

Before you can do it the \TeX\ file database has to be updated if that
has not been done, yet:
\begin{quote}
\verb|$ texhash| %$
\end{quote}

Then, just type
\begin{quote}
\verb|$ updmap --enable Map vagrounded.map| %$
\end{quote}
If you have generated more than one map file you have to repeat that
line for each map file.
It is not possible to enable multiple map files in one call of
\texttt{updmap}.

When you install a new version of te\TeX\ probably the added map
entries get lost.
Then you have to repeat the \texttt{updmap} calls again.


\section{Usage of the new font}

To use the new font you simple have to insert
\begin{quote}
\verb|\renewcommand{\rmdefault}{bvr}\rmfamily|
\end{quote}
into you \TeX\ sourcecode. For example \hypref{sample.tex}
{ftp://ftp.dante.de/tex-archive/info/TrueType/sample.tex}
\begin{quote}
\verb|\documentclass{article}|\\
\verb|\begin{document}|\\
\verb|\renewcommand{\rmdefault}{bvr}\rmfamily|\\
\verb|\noindent Hello, I am VAG Rounded BT|\\
\verb|{\slshape Hello, I am VAG Rounded BT slanted}\\|\\
\verb|{\scshape Hello, I am VAG Rounded BT small capitals}\\|\\
\verb|\end{document}|
\end{quote}
It is more elegant to create an new style file that switches to your
new font. 
The style file \hypref{vagrounded.sty}
{ftp://ftp.dante.de/tex-archive/info/TrueType/vagrounded.sty} is an
example how this can be done.
\begin{quote}
\verb|\ProvidesPackage{vagrounded}|\\
\verb|[2000/05/12 VAG-Rounded font as default sf font]|\\
\verb|%%|\\
\verb|\renewcommand{\sfdefault}{bvr}|\\
\verb|%%|\\
\verb|\AtEndDocument{\PackageWarningNoLine{vagrounded.sty}%|\\
\verb|  {Ensure to use dvips with the option -Pvagrounded}}|\\
\verb|%%|\\
\verb|\endinput|
\end{quote}
%The \verb|\AtEndDocument| command is only necessary if you don't
%include the map entries in the file \verb|psfonts.map|.

If you have not included the mapping entries to the file
\verb|psfonts.map| (section~\ref{sec:mapping}) you also have to tell
dvips that it should use the font: 
\begin{quote}
\verb|$ latex sample|\\
\verb|$ dvips -Pvagrounded sample|
\end{quote}
This should produce the PostScript file
\hypref{sample.ps}{ftp://ftp.dante.de/tex-archive/info/TrueType/sample.ps} 
which looks like figure~\ref{fig:beispiel}.
%
\begin{figure}[h!tbp]
\centering
\includegraphics{sample_eps}
\caption{Sample of the font VAGRounded~BT}
\label{fig:beispiel}
\end{figure}

\appendix

\section{To do}

Nguy{\^e}n-Dai Qu� has complained that the fontnames contain the
underscore (\verb|_|) instead of the minus (\verb|-|).
I have not been able to find out whether this causes problems using
the fonts. 
But I also know that nobody uses fonts with an underscore in the name.
So I should find out whether the underscores may cause problems e.g., 
when including eps files which use these fonts (e.g., from Adobe
Illustrator).


\section{Links}

Nguy{\^e}n-Dai Qu\'y has written a script that does all or most of the
conversion automatically \cite{quy2000a}. 
It is available from
\url{http://iris.ltas.ulg.ac.be/viettug/contrib/q/}. 
This script also replaces the underscores by minuses in the
fontnames.

Hume Smith has developed a method that does not include the Type\,1
fonts into the postscript file but tells ghostscript to use the
TrueType fonts directly \cite{smith2001a}.
He says that this approach avoids some problems with some fonts. 
But these files are not portable anymore because they do not contain
the used fonts.
His description is available from
\url{http://geocities.com/kwantus/ttf.html}.

\section{\refname}
%\addcontentsline{toc}{section}{\protect\numberline{\space}\refname}
\def\section*#1{}
\bibliographystyle{alpha}
\bibliography{fonts}

%\listoffigures
%\listoftables
% -----------------------------------------------------------------------
\end{document}
% -----------------------------------------------------------------------
