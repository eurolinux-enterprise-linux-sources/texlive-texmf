% macropackage lplaing
%format latexg
% This is for tests with `GERXAMPL.BIB' or `XAMPL.BIB'
% and the different style-files
\def\BibTeX{{\rm B\kern-.05em{\sc i\kern-.025em b}\kern-.08em
    T\kern-.1667em\lower.7ex\hbox{E}\kern-.125emX}}

\documentstyle[bibgerm,mynormal%
%,apalike%
]{article}
%\nonfrenchspacing
%\originalTeX
\begin{document}

Siehe \cite[Kapitel 3.4, S. 105\,ff.]{article-full}.


Aaamport schreibt in \cite[S. 77--78]{article-minimal} wie sich 
die Gnats mit den Gnus vertragen.


Ein Artikel in \cite{article-crossref}.

B�cher, B�cher, B�cher \cite{book-crossref}.


Ein volles Buch kann schwer sein, ein vielb�ndiges Werk wie die Reihe 
\cite{book-full} erst recht.

Etwas f�r Leseratten zum Thema \TeX\ findet sich in \cite{book-minimal}.

F�r weniger Leseinteressierte ein kleines Buch \cite{booklet-full}.

Booklet hei�en solche Dinger auf "`USenglish"' \cite{booklet-minimal}.

Ein Kapitel aus einem Buch ist in \cite{inbook-crossref} angegeben, 
dasselbe Kapitel noch einmal unter \cite{inbook-full}. Der n�chste 
Eintrag ist glaube ich ein neuer \cite{inbook-minimal} --- oder auch nicht.

Einzelne B�nde einer Reihe wie \cite{incollection-crossref} k�nnen so oder
auch anders wie hier \cite{incollection-full} zitiert werden. Auch die
folgende Variante kann ich ihnen anbieten \cite{incollection-minimal}.

Konferenzen, Konferenzen \cite{inproceedings-crossref}. Und wenn auf der 
Konferenz etwas ver�ffentlicht wurde wie \cite{inproceedings-full} dann kann
man es unter \cite{inproceedings-minimal} nachlesen.

Hnadb�cher gibts auch \cite{manual-full}. Auch wenn \cite{manual-minimal} ein
bl�des Beispiel daf�r ist.

Dieser Text und diese \cite{mastersthesis-full} Diplomarbeit sind wohl 
ebensowenig wie die folgende \cite{mastersthesis-minimal} ganz ernst zu 
nehmen.

Vermixtes will hei�en \cite{misc-full}. Gemischtes \cite{misc-minimal} ist
f�r alles gut.

Anspruschsvolelr ist da schon \cite{phdthesis-full}, eine Doktorarbeit, mit der
sich auch \cite{phdthesis-minimal} besch�ftigt.

Schon wieder Konferenzen \cite{proceedings-full} �ber Konferenzen \cite{proceedings-minimal}.


Diese ist ein Band mit einem zuf�lligen Querverweis \cite{random-note-crossref} --- oder so.

Hei�t es nun "`Technischer Bericht"' \cite{techreport-full} oder sind einfach
Bedienungsanleitungen damit gemeint \cite{techreport-minimal}.

Selbst Sachen, die noch nicht publiziert sind \cite{unpublished-full} kann man 
so \cite{unpublished-minimal} zitieren.

Ein Beispiel f�r eine Gesamtausgabe \cite{whole-collection}.

Ein Beispiel f�r eine Zeitschrift \cite{whole-journal}.

Und noch mehr vollst�ndiges: Konferenzen \cite{whole-proceedings} und
ein "`Set"' \cite{whole-set}.

\nocite{*}

\bibliography{gerxampl}
\bibliographystyle{gerabbrv}
%\bibliographystyle{gerplain}
%\bibliographystyle{gerunsrt}
%\bibliographystyle{geralpha}
%%\bibliographystyle{gerapali}


\end{document}
