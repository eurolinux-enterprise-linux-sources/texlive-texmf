%% @texfile{
%%   filename  = "joyerr2.tex",
%%   version   = "1.01",
%%   date      = "2001/08/05",
%%   time      = "17:03:46 EDT",
%%   checksum  = "04000 319 1132 9628",
%%   author    = "American Mathematical Society",
%%   filetype  = "AMS-TeX: user documentation",
%%   copyright = "Copyright 2000 American Mathematical Society,
%%                all rights reserved.  Copying of this file is
%%                authorized only if either:
%%                (1) you make absolutely no changes to your copy
%%                    including name; OR
%%                (2) if you do make changes, you first rename it to some
%%                    other name.",
%%   address   = "American Mathematical Society,
%%                Technical Support,
%%                Electronic Products and Services,
%%                P. O. Box 6248,
%%                Providence, RI 02940,
%%                USA",
%%   telephone = "401-455-4080 or (in the USA and Canada)
%%                800-321-4AMS (321-4267)",
%%   FAX       = "401-331-3842",
%%   email     = "tech-support@ams.org (Internet)",
%%   codetable = "ISO/ASCII",
%%   keywords  = "amstex, ams-tex, tex",
%%   abstract  = "This file contains errata to The Joy of TeX,
%%                second edition, 1990. It must be run with AMSTEX
%%                and AMSPPT 2.0+; it is incompatible with previous
%%                versions. It also requires the file amssym.tex and
%%                the fonts msam10 and msbm10.",
%%   docstring = "The checksum field above contains a CRC-16 checksum
%%                as the first value, followed by the equivalent of
%%                the standard UNIX wc (word count) utility output of
%%                lines, words, and characters.  This is produced by
%%                Robert Solovay's checksum utility.",
%% }
%%%%%%%%%%%%%%%%%%%%%%%%%%%%%%%%%%%%%%%%%%%%%%%%%%%%%%%%%%%%%%%%%%%%%%%%
\input amstex
\documentstyle{amsppt}

\define\lastupdate{5 August 2001}

\pagewidth{29pc}
\raggedbottom
\tenpoint

%  \ninepoint required for diversions; sidetrip symbol not available.
%  Substitute 7pt for \scriptfont, usually 6pt; not really needed
\catcode`\@=11
\font@\ninerm=cmr10 at 9pt
\font@\ninei=cmmi10 at 9pt
\font@\ninesy=cmsy10 at 9pt
\font@\nineit=cmti10 at 9pt
\font@\ninesl=cmsl10 at 9pt
\font@\ninebf=cmbx10 at 9pt
\font@\ninett=cmtt10 at 9pt
\def\ninepoint{\normalbaselineskip12pt
 \def\rm{\fam0\ninerm}%
 \def\it{\fam\itfam\nineit}%
 \def\sl{\fam\slfam\ninesl}%
 \def\bf{\fam\bffam\ninebf}%
 \def\tt{\ninett}%
 \def\smc{\tensmc}%
 \textfont0=\ninerm   \scriptfont0=\sevenrm   \scriptscriptfont0=\fiverm
 \textfont1=\ninei    \scriptfont1=\seveni    \scriptscriptfont1=\fivei
 \textfont2=\ninesy   \scriptfont2=\sevensy   \scriptscriptfont2=\fivesy
 \textfont3=\tenex   \scriptfont3=\tenex     \scriptscriptfont3=\tenex
 \textfont\itfam=\nineit
 \textfont\slfam=\ninesl
 \textfont\bffam=\ninebf \scriptfont\bffam=\sevenbf
   \scriptscriptfont\bffam=\fivebf
 \tt\ttglue=.45em  minus .135em
 \setbox\strutbox=\hbox{\vrule height8pt depth3pt width0pt}%
 \normalbaselines\rm}

\def\JoT{{\sl The Joy of \TeX}}

%  Support verbatim listing of TeX source, as defined in TeXbook, p. 421;
%  lifted from MANMAC.TEX, and modified slightly for narrower columns.
\catcode`\@=11

\chardef\other=12
\def\ttverbatim{\begingroup \catcode`\\=\other
  \catcode`\{=\other \catcode`\}=\other \catcode`\$=\other
  \catcode`\&=\other \catcode`\#=\other \catcode`\%=\other
  \catcode`\~=\other \catcode`\_=\other \catcode`\^=\other
  \catcode`\"=\other
  \obeyspaces \obeylines \hyphenpenalty=10000 \tt}

\newskip\ttglue
{\tenpoint\tt \global\ttglue=.5em plus .25em minus .15em}
% this should be installed in each font

%  From David Eppstein's ``Trees'' paper (TUGboat 6#1), preserve initial
%  spaces.
{\obeyspaces\gdef {\ifvmode\indent\fi\space}}

%  Permissible overhang beyond right margin.
\newdimen\ttrightskip
\ttrightskip=5pc

%  Although | is ordinarily an escape character within verbatim mode,
%  provide a method for letting it instead be the character itself
%  within a display verbatim listing, as needed; this is based on
%  a technique developed by Michael Ferguson.  Note that within one
%  \begintt...\endtt block, | can be only one of:
%       the printing | character, or
%       active (the escape character)
%  It cannot perform both functions at the same time.
\newif\ifttVertChar     \ttVertCharfalse
{\catcode`\|=\active \gdef\VertChar{\def|{\char"7C }}}

%  Other non-tt elements that may be embedded within \begintt...\endtt .
\def\MTH{$}
\def\sb{_}
\def\sp{^}
\def\SP{{\tt\char"20 }}         % "visible" space
\chardef\bs=`\\
\def\vrt{{\tt\char`\|}}

\catcode`\|=\active
{\obeylines \gdef\activatettbar{\global\catcode`\|=\active %
  \gdef|{\ttverbatim \spaceskip\ttglue \xspaceskip\ttglue %
         \let^^M=\  \let|=\endgroup}}}
\activatettbar

\catcode`\@=13

\def\ttindent{\noindent\kern3\parindent\hangindent3\parindent}

%  This definition is stolen from the file of TeXbook errata.
\def\bugonpage#1(#2) \par{\bigbreak\tenpoint
  \hrule width\hsize
  \line{\lower3.5pt\vbox to13pt{}Page #1\hfil(#2)}\hrule width\hsize
  \nobreak\medskip}

%  Some definitions for setting particular Joy notation.
\def\CR{$\langle$carriage-return$\rangle$}
\def\tab{{\smc tab}}

\NoBlackBoxes

%%%%%%%%%%%%%%%%%%%%%%%%%%%%%%%%%%%%%%%%%%%%%%%%%%%%%%%%%%%%%%%%%%%%%%%%

\topmatter
\title Errata to \JoT, second edition,\linebreak
  for \AmSTeX{} 2.0\endtitle
\leftheadtext{Errata to \JoT, second edition, for \AmSTeX{} 2.0}
\rightheadtext{Errata to \JoT, second edition, for \AmSTeX{} 2.0}
\endtopmatter

\document

\noindent
This list of corrections to \JoT, second edition, 1990, includes all known
corrections to that edition.  Corrections for the previous edition, which
preceded the release of \AmSTeX{} Version 2.0, can be found in the file
|joterr.tex|.

The printing date of each copy of \JoT\ is identified on the reverse
of the title page.  No corrections have been made in any reprint of
\JoT, second edition, since the initial printing in 1990.

\smallskip
(This errata list was last updated \lastupdate.)


\bugonpage xviii, line 16 (1997/06/07)

\noindent
[{\sl change $d0$ at end of line to $d\theta$.}]

\bugonpage xx, line 14 (1997/06/07)

\noindent
[{\sl change $d0$ at end of line to $d\theta$.}]

\bugonpage 19, lines $-9$ to $-7$ (1997/06/07)

\begingroup
\hsize27.5pc
\leftskip1.5pc
\noindent
sensation, particularly the material that he has expounded on pages\linebreak
22--23. Including this material cost an extra \$1000, but it did make
\P\P\linebreak
1 and 2 quite popular
\endgroup

\bugonpage 23, lines 24--26 (1997/06/07)

\begingroup
\hsize=30.15pc
\BlackBoxes
\indent The head of the department, our university's own I. M. Stable,
  attributes Treemu\-nch's recent aberrant behavior to the much-publicized
  ``research'' for his paper; notwithstanding, others say that Treemunch's
  name isn't on the computers' databases,
\par
\endgroup

\bugonpage 24, line 17 (1997/06/07)

\noindent\kern1.5pc
\line{notwithstanding, others say that Treemunch's name isn't on
  the computers'}

\bugonpage 24, line $-13$ (1997/06/07)

\ttindent
|the computers'\linebreak databases ...|

\bugonpage 33, line 9 (1997/06/07)

\line{\indent Ties are often preferable to |\.|\ after abbreviations.
For example, it is best}

\bugonpage 97, lines 7--8 (1997/06/07)

\noindent
some other kind of tag. Some journals place these tags to the left of
the formula:

\bugonpage 129, line 13 (1997/06/07)

\noindent\kern-1.5pc
{\bf ACCENTS IN MATH MODE}

\bugonpage 137, line 12 (1997/06/07)

\noindent\kern-1.5pc
{\bf ARROWS}

\bugonpage 163, line 16 (1997/06/07)

\noindent\kern-1.5pc
{\bf GREEK LETTERS}

\bugonpage 163, line 21 (1997/06/07)

\noindent\kern-1.5pc
{\bf HORIZONTAL BRACES}

\bugonpage 175, line $-13$ (1997/06/07)

\noindent
|amsppt| style, you still must type |\loadmsam| and |\loadmsbm| but you can

\bugonpage 182, line 20 (1997/06/07)

\noindent\kern-1.5pc
{\bf RUNNING HEADS}

\bugonpage 183, lines $-3$ to $-2$ (1997/06/07)

\noindent
the |\bigl| and |\bigr| for the delimiters (see {\bf big and bigg})
should be accompanied by a |\bigm|{\tt\char`\|} for the $\vert$:

\bugonpage 194, lines $-4$ to $-3$ (1997/06/07)

\begingroup
\NoBlackBoxes
\ninepoint
\leftskip18pt
\noindent
\AmSTeX\ also takes care to disregard extraneous spaces in constructions
like \hbox{|\author|} and |\title|.  Although we typed\par
\endgroup

\bugonpage 201, line $-11$ (1997/06/07)

\noindent
\line{tuation, the |amsppt| preprint style provides a command |\rom| to
be applied}

\bugonpage 244, lines $-3$ to $-1$ (1997/06/07)

\noindent
When you leave out the |&|, \TeX\ simply inserts a blank for the right
hand part of the second formula, so the whole second formula was treated
as the left hand part, with the right hand part being blank.

\bugonpage 291, column 2 (1999/01/07)

Add entry\newline
empty delimiter (|\left.|, |\right.|), 85, 144

\bugonpage 293, column 2, index entry ``formulas'' (1996/11/23)

Add second level entry\newline
\indent{\it see also\/} tags

\bugonpage 298, column 1 (1997/06/07)

\noindent
margin gap in |\multline|, 175

\bugonpage 298, column 1, index entry ``math formulas'' (1997/06/07)

Add second level entry\newline
\indent|\tag|, 97

\bugonpage 293, column 2, index entry ``numbers'' (1996/11/23)

Add second level entry\newline
\indent{\it see also\/} tags

\bugonpage 300, column 1, index entry ``operators'' (1997/06/07)

\indent space around, 43--44, 164-165, 224

\bugonpage 303, column 1, index entry ``relations'' (1997/06/07)

\indent space around, 43--44, 137, 165

\enddocument
