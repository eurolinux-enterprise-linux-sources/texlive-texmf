%% \section{Possible Interferences}
\section{\texorpdfstring{Ảnh hưởng đến các gói khác}{Anh huong den goi khac}}

%% Since |ntheorem| reimplements the handling of theorem-environments
%% completely, it is incompatible with every package also concerning
%% those macros.
Vì |ntheorem| xây dựng lại hoàn toàn cách xử lý |THM|, nên nó không tương thích
với mọi gói có liên quan đến cách xử lý |THM|.

%% Additionally, the |thmmarks| algorithm for placing endmarks 
%% requires modifications of several environments (cf.\ Section 
%% \ref{sec:code}).
\medskip
Ngoài ra, thuật toán đặt dấu kết thúc |thmmarks| đòi hỏi phải thay
đổi vài môi trường (xem tài liệu về mã nguồn của |ntheorem|).
%% Thus, environments which are reimplemented or additionally defined
%% by document options or styles are not covered by the endmark 
%% algorithm of |ntheorem.sty|.
Vì vậy, những môi trường được thay đổi, xây dựng lại bởi các lớp tài liệu
hay các gói sẽ không chịu ảnh hưởng bởi thuật toán cung cấp bởi gói |ntheorem|.

%% The |[thref]| option changes the |\label| command and the treatment
%% of labels when reading the |.aux| file. Thus it is potentially
%% incompatible with all packages also changing |\label| (or
%% |\newlabel|). Compatibility with babel's |\newlabel| isa
%% achieved if babel is loaded before ntheorem.
\medskip
Tùy chọn |thref| sẽ định nghĩa lại lệnh |\label| và sẽ xử lý các nhãn
trong khi đọc tập tin |.aux|. Chính vì thế, việc dùng tùy chọn này sẽ
gây ra sự không tương thích với mọi gói có thay đổi lệnh |\label|
hoặc |\newlabel|. Với gói |babel|, sự không tương thích được giải
quyết chỉ khi gói |ntheorem| được nạp sau gói |babel|.

%% \subsection{Interfering Document Options.}
\subsection{\texorpdfstring{Ảnh hưởng đến tuỳ chọn lớp tài liệu}{Anh huong den tuy chon lop}}

%% |ntheorem.sty| also copes with the usual document options 
%% |leqno| and |fleqn|\footnote{although for \texttt{fleqn} and 
%%   long formulas
%%   reaching to the right margin, equation numbers and endmarks can
%%   be smashed over the formula since \texttt{fleqn} does not use
%%   \texttt{\bslash eqno} for controlling the setting of the equation
%%   number.}.
%% If one of those options is used in the |\documentclass|
%% declaration, it is automatically recognized by the |thmmarks| part
%% of |ntheorem.sty|.
Gói |ntheorem| cũng đụng độ với các các tuỳ chọn |leqno| và |fleqn| cho
lớp tài liệu. Nếu một trong hai tùy chọn đó được chỉ ra khi nạp lớp,
gói |ntheorem| (với tùy chọn |thmmarks|) sẽ phát hiện được.

\medskip
%% If one of those options is not used in |\documentclass|, but
%% with |amsmath| (see next section), it must not be specified 
%% for |ntheorem|, since all |amsmath| environments detect this option 
%% by themselves.
Nếu các tuỳ chọn đó không được chỉ ra khi gọi lớp, mà khi nạp gói |amsmath|
(xem mục tiếp theo), thì tùy đó phải không được chỉ ra khi nạp gói |ntheorem|,
bởi vì chính mọi môi trường của gói |amsmath| sẽ nhận ra tuỳ chọn ấy.

%% \subsection{Combination with amslatex.}
\subsection{\texorpdfstring{Với gói}{Voi goi} amslatex}
\label{sec:amslatex}

%% |ntheorem.sty| interferes with |amsmath.sty| and |amsthm.sty|.
Gói |ntheorem| ảnh hưởng đến gói |amsmath| và |amsthm|.

%% Note, that the LaTeX amstex package |amstex.sty| (\LaTeX2.09) is
%% obsolete and you should use |amsmath| and |amstext| for
%% \LaTeXe\ instead.  Up to |ntheorem-1.18|, it is compatible with 
%% |amsmath-1.x|. Since |ntheorem-1.19|, it is (hopefully) compatible 
%% with |amsmath-2.x|.
\medskip
Chú ý rằng, gói |amstex.sty| của \LaTeX{}2.09 đã cũ và bạn nên thay thế
bởi hai gói |amsmath| và |amstext| của \LaTeXe{}. Gói |ntheorem| các phiên
bản từ 1.18 về trước tương thích với |amsmath-1.x|, và các phiên bản từ 1.19
tương thích với |amsmath-2.x| (hy vọng vậy ;)

%% We would be happy if someone knowing and using |amsmath| would
%% join the development and maintenance of this style.
\medskip
Tác giả |ntheorem| hy vọng có ai đó dùng và hiểu gói |amsmath| có thể
tham gia đội phát triển và bảo dưỡng |ntheorem| để bảm đảm sự tương thích này.

\subsubsection{\texorpdfstring{Với gói}{Voi goi} amsmath}

%% Compatibility with amsmath (end marks for math environments, and 
%% handling of labels in math environments) is provided in the option
%% |[amsmath]|, (i.e., if |\usepackage{amsmath}| is used then
Sự tương thích với gói |amsmath| (ở các điểm: đặt dấu kết thúc,
xử lý nhãn trong môi trường toán) được bảo đảm nhờ tuỳ chọn |amsmath|
khi nạp lớp |ntheorem|.
\begin{itemize}
\item |\usepackage[thmmarks]{ntheorem}|
%% must be completed to \\
cẩn phải được thay bởi\\
|\usepackage[amsmath,thmmarks]{ntheorem}|), và tương tư%%and also
\item |\usepackage[thref]{ntheorem}|
%% must be completed to \\
cần thay bởi \\
|\usepackage[amsmath,thref]{ntheorem}|).
\end{itemize}
%% Note, that |amsmath| has to be loaded \emph{before} |ntheorem| 
%% since the definitions have to be overwritten.
Cũng cần lưu ý rằng, gói |amsmath| cần phải nạp trước gói |ntheorem|,
để đảm bảo rằng các định nghĩa của |amsmath| sẽ được |ntheorem| xử lý lại.

\subsubsection{\texorpdfstring{Với gói}{Voi goi} amsthm}

%% |amsthm.sty| conflicts with the definition of theorem
%% layouts in |theorem.sty|, some features of |amsthm.sty|
%% have been incorporated into option |[amsthm]| which has
%% to be used \emph{instead of} |\usepackage{amsthm}|.
Gói |amsthm| xung đột với gói |theorem| (về kiểu |THM|).
Thay vì dùng gói |amsthm|,
bạn hãy dùng tùy chọn |amsthm| khi nạp gói |ntheorem|.

%% The Option provides theoremstyles |plain|, |definition|, and 
%% |remark|, and a |proof| environment as in |amsthm.sty|. 
\medskip
Tùy chọn đó cung cấp các kiểu |THM| là |plain|, |definition|,
|remark| và |proof| đúng như của gói |amsthm|.

%% The |\newtheorem*| command is defined even without this
%% option. Note that |\newtheorem*| always switches to the
%% nonumbered version of the current theoremstyle which
%% thus must be defined.
\medskip
Lệnh |\newtheorem*| được cung cấp bởi |ntheorem| ngay cả khi
bạn không dùng tùy chọn |amsthm|. Chú ý rằng, |\newtheorem*|
luôn tương ứng với bản không đánh số của kiểu |THM| hiện tại;
do đó, khi dùng |\newtheorem*| thì kiểu |THM| là kiểu đã có.

%% The command |\newtheoremstyle| is not taken over from 
%% |amsthm.sty|. Also, |\swapnumbers| is not implemented.
%% Here, the user has to express his definitions by the 
%% |\newtheoremstyle| command provided by |ntheorem.sty|,
%% including the use of |\theoremheaderfont| and |\theorembodyfont|.
%% The options |[amsthm]| and |[standard]| are in conflict
%% since they both define an environment |proof|.
\medskip
Lệnh |\newtheoremstyle| và |\swapnumbers| của gói |amsthm| không
được |ntheorem| xây dựng lại. Vì vậy,
bạn phải định nghĩa các lệnh này (như gói |amsthm| định nghĩa),
cộng thêm các thay đổi nhờ dùng các lệnh |\theoremheaderfont|
và |\theorembodyfont|.

\medskip
%% Thus, we recommend not to use
%% |amsthm|, since the features for defining theorem-like
%% environments in |ntheorem.sty|---following 
%% |theorem.sty|---seem to be more intuitive and user-friendly.
Tóm lại, bạn không nên dùng gói |amsthm|, vì các tính năng
cung cấp bởi |ntheorem| trực quan và thân thiện hơn.

\subsection{\texorpdfstring{Với gói babel}{Voi goi babel}}
\label{sec:babel}

%% The |[thref]| option interferes with the |babel| package, thus in 
%% case that |babel| is used, |ntheorem| has to be loaded \emph{after} 
%% |babel|.
Khi dùng tùy chọn |thref|, gói |babel| phải được nạp trước gói |babel|.

\subsection{\texorpdfstring{Với gói}{Voi goi} hyperref}
\label{sec:hyperref}

%% Since |hyperref| redefines the \LaTeX\ |\contentsline|-command, it breaks
%% with |ntheorem| below version 1.17. Since version 1.17, the option 
%% |[hyperref]| makes |ntheorem| work with |hyperref|.
%% Theoremlists will then get linked list.
Vì gói |hyperref| định nghĩa lại lệnh |\contentsline| của \LaTeX{},
nên gói này sẽ trục trặc với |ntheorem| phiên bản 1.17 về trước.
Từ bản 1.17 của |nthereom|, có thêm tùy chọn |hyperref| bảo đảm
sự tương thích: trong danh sách |THM| bạn sẽ có liên kết đến các |THM| tương ứng.

%% WARNING: The definition and redefinition of Theorem List Layouts
%% (see Section~\ref{sec:listtypes}) isn't yet working with
%% the |hyperref|-package. 
\medskip
Chú ý rằng,
nếu bạn định nghĩa (lại) kiểu danh sách |THM| như ở Mục~\vref{sec:listtypes}),
kiểu mới đó sẽ không làm việc tốt với |hyperref|.

\endinput
