%%% ====================================================================
%%%  @LaTeX-file{
%%%     author          = "Alan Jeffrey and Rowland McDonnell and
%%%                        Ulrik Vieth and Lars Hellstr{\"o}m",
%%%     version         = "1.927",
%%%     date            = "6 August 2004",
%%%     time            = "22:10:00 CEST",
%%%     filename        = "fontinst.tex",
%%%     email           = "fontinst@tug.org",
%%%     checksum        = "",
%%%     codetable       = "ISO/ASCII",
%%%     keywords        = "fontinst, TeX, PostScript, documentation",
%%%     supported       = "yes",
%%%     abstract        = "This is the documentation for the fontinst
%%%                        font installation package.",
%%%     package         = "fontinst",
%%%     dependencies    = "ltxguide.cls, url.sty, xspace.sty"
%%%  }
%%% ====================================================================

% Copyright 1993, 1994, 1995, 1996 Alan Jeffrey

% Modified by Rowland McDonnell June 1998
% Modified and revised by Ulrik Vieth June 1998
% Modified by Lars Hellstr\"om July 2004


\documentclass[a4paper]{ltxguide}
\usepackage[OT1]{fontenc}
\usepackage{url,xspace}
\usepackage{graphicx}
% \renewcommand\rmdefault{ppl}
%\renewcommand\rmdefault{padx}
%\renewcommand\rmdefault{pmnx}

% \MakeShortVerb{\|}


% Title page and running heads.

\makeatletter
\newcommand{\subtitle}[1]{\renewcommand{\@subtitle}{#1}}
\newcommand{\version}[1]{\renewcommand{\@version}{#1}}
\newcommand{\illustration}[1]{\renewcommand{\@illustration}{#1}}

\newcommand{\@subtitle}{}
\newcommand{\@version}{???}
\newcommand{\@illustration}{cover.eps}

% \pagestyle{myheadings}
% \AtBeginDocument{\markright{\small\itshape
%   \@author\hfill\@title: \@subtitle\quad}}


\renewcommand{\maketitle}{{%
   \thispagestyle{empty}%
   \normalfont\centering
   \null
   {\fontsize{100}{100}\textit{\@title}}%
   \par\bigskip
   {\Large\textbf{\@subtitle}}%
   \vfill
   \includegraphics[width=\textwidth]{\@illustration}%
   \vfill
   {\Large\textbf{%
      \def\and{\egroup\qquad\hbox\bgroup}
      \leavevmode \hbox{\@author}\\[\smallskipamount] 
      \@title~v\@version
      \quad\textperiodcentered\quad \@date}}%
   \clearpage
}}
\makeatother


% Set lists tighter (assuming non-zero \parskip).

\makeatletter
\renewcommand{\@listI}{%
   \leftmargin\leftmargini
   \parsep\medskipamount
   \itemsep\z@ % + parsep
   \topsep\z@ % + parskip
   \partopsep\z@
}
\newenvironment{isyntax}{%
   \let\\\@centercr
   \list{}{%
      \itemsep \z@
      \itemindent -1.5em%
      \listparindent \itemindent
      \advance \leftmargin 1.5em%
   }%
   \advance \rightskip \z@\@plus0.7\linewidth \relax
   \linepenalty=100\relax
   \item\relax
}{\endlist}
\makeatother

\newenvironment{smalldes}{%
   \list{}{%
      \setlength\labelwidth{0pt}%
      \setlength\itemindent{-\leftmargin}%
      \setlength\listparindent{1em}%
      \setlength\parsep{0pt}%
      \setlength\itemsep{0pt plus 1pt}%
      \setlength\topsep{\itemsep}%
      \let\makelabel\descriptionlabel
   }%
}{\endlist}

\newenvironment{hackernote}{%
   \list{}{
     \setlength{\leftmargin}{0pt}%
     \setlength\labelwidth{0pt}%
     \setlength{\listparindent}{1.4em}%
     \setlength{\parsep}{0pt plus 1pt}%
     \setlength{\itemsep}{\medskipamount}%
   }\item[]%
   \small
   \textit{Note for hackers.}\hspace{0.5em}%
}{\endlist}

% % I don't use the <...> feature for verbatim, 
% % so I undo that ltxguide feature.
% \makeatletter
% \renewcommand{\verbatim@font}{%
%   \normalfont \ttfamily
% %   \catcode`\<=\active
% %   \catcode`\>=\active
% }
\makeatother

% Set spacing around captions.

\setlength{\abovecaptionskip}{\medskipamount}
\setlength{\belowcaptionskip}{\medskipamount}

% Markup for logos, file types, programs, etc.

\newcommand*{\meta}{\m}
\newcommand*{\marg}{\arg}
\newcommand*{\parg}[1]{\texttt{(}\m{#1}\texttt{)}}

\newcommand*{\cs}[1]{\texttt{\char`\\ #1}\xspace}

\newcommand*{\OzTeX}{O\kern-.03em z\kern-.15em\TeX}
\newcommand*{\OzMF}{O\kern-.03em zMF}
\newcommand*{\OzTools}{O\kern-.03em z\kern-.15em Tools}

\newcommand{\PS}{Post\-Script\xspace}
\newcommand{\TT}{True\-Type\xspace}

\newcommand*{\setfilename}[1]{\texttt{#1}}
\newcommand*{\setdotfilename}[1]{\setfilename{.#1}}
\newcommand*{\setpackagename}[1]{\textsf{#1}}

\newcommand{\dvips}   {\setpackagename{dvips}\xspace}
\newcommand{\Dvips}   {\setpackagename{Dvips}\xspace}
\newcommand{\fontinst}{\setpackagename{font\-inst}\xspace}
\newcommand{\Fontinst}{\setpackagename{Font\-inst}\xspace}

% \show\fontname
\newcommand{\fontnamekb}{fontname\xspace}
\newcommand{\Fontnamekb}{Fontname\xspace}

\newcommand{\mf} {\setfilename{mf}\xspace}
\newcommand{\Mf} {\setfilename{Mf}\xspace}
\newcommand{\vf} {\setfilename{vf}\xspace}
\newcommand{\Vf} {\setfilename{Vf}\xspace}
\newcommand{\pl} {\setfilename{pl}\xspace}
\newcommand{\Pl} {\setfilename{Pl}\xspace}
\newcommand{\fd} {\setfilename{fd}\xspace}
\newcommand{\Fd} {\setfilename{Fd}\xspace}
\newcommand{\pk} {\setfilename{pk}\xspace}
\newcommand{\Pk} {\setfilename{Pk}\xspace}
\newcommand{\afm}{\setfilename{afm}\xspace}
\newcommand{\Afm}{\setfilename{Afm}\xspace}
\newcommand{\vpl}{\setfilename{vpl}\xspace}
\newcommand{\Vpl}{\setfilename{Vpl}\xspace}
\newcommand{\tfm}{\setfilename{tfm}\xspace}
\newcommand{\Tfm}{\setfilename{Tfm}\xspace}
\newcommand{\mtx}{\setfilename{mtx}\xspace}
\newcommand{\Mtx}{\setfilename{Mtx}\xspace}
\newcommand{\etx}{\setfilename{etx}\xspace}
\newcommand{\Etx}{\setfilename{Etx}\xspace}
\newcommand{\pfa}{\setfilename{pfa}\xspace}
\newcommand{\Pfa}{\setfilename{Pfa}\xspace}
\newcommand{\pfb}{\setfilename{pfb}\xspace}
\newcommand{\Pfb}{\setfilename{Pfb}\xspace}
\newcommand{\dvi}{\setfilename{dvi}\xspace}
\newcommand{\Dvi}{\setfilename{Dvi}\xspace}
\newcommand{\pdf}{\setfilename{pdf}\xspace}
\newcommand{\Pdf}{\setfilename{Pdf}\xspace}
\newcommand{\ttf}{\setfilename{ttf}\xspace}
\newcommand{\Ttf}{\setfilename{Ttf}\xspace}

\newcommand{\vftovp}{\setpackagename{vftovp}\xspace}
\newcommand{\vptovf}{\setpackagename{vptovf}\xspace}
\newcommand{\pltotf}{\setpackagename{pltotf}\xspace}
\newcommand{\tftopl}{\setpackagename{tftopl}\xspace}

\newcommand{\BibTeX}{Bib\TeX}


% % Stolen from Dr Knuth
% \makeatletter % borrow the private macros of PLAIN (with care)
% \def\oct#1{\hbox{\rm\'{}\kern-.2em\it#1\/\kern.05em}} % octal constant
% \def\hex#1{\hbox{\rm\H{}\tt#1}} % hexadecimal constant
% % macros for font tables
% \def\oddline#1{\cr
%   \noalign{\nointerlineskip}
%   \multispan{19}\hrulefill&
%   \setbox0=\hbox{\lower 2.3pt\hbox{\hex{#1x}}}\smash{\box0}\cr
%   \noalign{\nointerlineskip}}
% \def\evenline{\cr\noalign{\hrule}}
% \def\chartstrut{\lower4.5pt\vbox to14pt{}}
% \def\beginchart#1{$$\global\count@=0 #1
%   \halign to\hsize\bgroup
%     \chartstrut##\tabskip0pt plus10pt&
%     &\hfil##\hfil&\vrule##\cr
%     \lower6.5pt\null
%     &&&\oct0&&\oct1&&\oct2&&\oct3&&\oct4&&\oct5&&\oct6&&\oct7&\evenline}
% \def\endchart{\raise11.5pt\null&&&\hex 8&&\hex 9&&\hex A&&\hex B&
%   &\hex C&&\hex D&&\hex E&&\hex F&\cr\egroup$$}
% \def\*{\global\advance\count@ by1 }
% \def\:{\setbox0=\hbox{\char\count@}%
%   \ifdim\ht0>7.5pt\reposition
%   \else\ifdim\dp0>2.5pt\reposition\fi\fi
%   \box0\global\advance\count@ by1 }
% \def\reposition{\setbox0=\hbox{$\vcenter{\kern2pt\box0\kern2pt}$}}
% \def\smallchart{%
% %\global\advance\count@ by16
%   &\oct{00x}&&\:&&\:&&\:&&\:&&\:&&\:&&\:&&\:&&\oddline0
%   &\oct{01x}&&\:&&\:&&\:&&\:&&\:&&\:&&\:&&\:&\evenline
%   &\oct{02x}&&\:&&\:&&\:&&\:&&\:&&\:&&\:&&\:&&\oddline1
%   &\oct{03x}&&\:&&\:&&\:&&\:&&\:&&\:&&\:&&\:&\evenline
%   &\oct{04x}&&\:&&\:&&\:&&\:&&\:&&\:&&\:&&\:&&\oddline2
%   &\oct{05x}&&\:&&\:&&\:&&\:&&\:&&\:&&\:&&\:&\evenline
%   &\oct{06x}&&\:&&\:&&\:&&\:&&\:&&\:&&\:&&\:&&\oddline3
%   &\oct{07x}&&\:&&\:&&\:&&\:&&\:&&\:&&\:&&\:&\evenline
%   &\oct{10x}&&\:&&\:&&\:&&\:&&\:&&\:&&\:&&\:&&\oddline4
%   &\oct{11x}&&\:&&\:&&\:&&\:&&\:&&\:&&\:&&\:&\evenline
%   &\oct{12x}&&\:&&\:&&\:&&\:&&\:&&\:&&\:&&\:&&\oddline5
%   &\oct{13x}&&\:&&\:&&\:&&\:&&\:&&\:&&\:&&\:&\evenline
%   &\oct{14x}&&\:&&\:&&\:&&\:&&\:&&\:&&\:&&\:&&\oddline6
%   &\oct{15x}&&\:&&\:&&\:&&\:&&\:&&\:&&\:&&\:&\evenline
%   &\oct{16x}&&\:&&\:&&\:&&\:&&\:&&\:&&\:&&\:&&\oddline7
%   &\oct{17x}&&\:&&\:&&\:&&\:&&\:&&\:&&\:&&\:&\evenline}
% 
% \def\bigchart{%
% %\global\advance\count@ by16
%   &\oct{00x}&&\:&&\:&&\:&&\:&&\:&&\:&&\:&&\:&&\oddline0
%   &\oct{01x}&&\:&&\:&&\:&&\:&&\:&&\:&&\:&&\:&\evenline
%   &\oct{02x}&&\:&&\:&&\:&&\:&&\:&&\:&&\:&&\:&&\oddline1
%   &\oct{03x}&&\:&&\:&&\:&&\:&&\:&&\:&&\:&&\:&\evenline
%   &\oct{04x}&&\:&&\:&&\:&&\:&&\:&&\:&&\:&&\:&&\oddline2
%   &\oct{05x}&&\:&&\:&&\:&&\:&&\:&&\:&&\:&&\:&\evenline
%   &\oct{06x}&&\:&&\:&&\:&&\:&&\:&&\:&&\:&&\:&&\oddline3
%   &\oct{07x}&&\:&&\:&&\:&&\:&&\:&&\:&&\:&&\:&\evenline
%   &\oct{10x}&&\:&&\:&&\:&&\:&&\:&&\:&&\:&&\:&&\oddline4
%   &\oct{11x}&&\:&&\:&&\:&&\:&&\:&&\:&&\:&&\:&\evenline
%   &\oct{12x}&&\:&&\:&&\:&&\:&&\:&&\:&&\:&&\:&&\oddline5
%   &\oct{13x}&&\:&&\:&&\:&&\:&&\:&&\:&&\:&&\:&\evenline
%   &\oct{14x}&&\:&&\:&&\:&&\:&&\:&&\:&&\:&&\:&&\oddline6
%   &\oct{15x}&&\:&&\:&&\:&&\:&&\:&&\:&&\:&&\:&\evenline
%   &\oct{16x}&&\:&&\:&&\:&&\:&&\:&&\:&&\:&&\:&&\oddline7
%   &\oct{17x}&&\:&&\:&&\:&&\:&&\:&&\:&&\:&&\:&\evenline
%   &\oct{20x}&&\:&&\:&&\:&&\:&&\:&&\:&&\:&&\:&&\oddline8
%   &\oct{21x}&&\:&&\:&&\:&&\:&&\:&&\:&&\:&&\:&\evenline
%   &\oct{22x}&&\:&&\:&&\:&&\:&&\:&&\:&&\:&&\:&&\oddline9
%   &\oct{23x}&&\:&&\:&&\:&&\:&&\:&&\:&&\:&&\:&\evenline
%   &\oct{24x}&&\:&&\:&&\:&&\:&&\:&&\:&&\:&&\:&&\oddline{A}
%   &\oct{25x}&&\:&&\:&&\:&&\:&&\:&&\:&&\:&&\:&\evenline
%   &\oct{26x}&&\:&&\:&&\:&&\:&&\:&&\:&&\:&&\:&&\oddline{B}
%   &\oct{27x}&&\:&&\:&&\:&&\:&&\:&&\:&&\:&&\:&\evenline
%   &\oct{30x}&&\:&&\:&&\:&&\:&&\:&&\:&&\:&&\:&&\oddline{C}
%   &\oct{31x}&&\:&&\:&&\:&&\:&&\:&&\:&&\:&&\:&\evenline
%   &\oct{32x}&&\:&&\:&&\:&&\:&&\:&&\:&&\:&&\:&&\oddline{D}
%   &\oct{33x}&&\:&&\:&&\:&&\:&&\:&&\:&&\:&&\:&\evenline
%   &\oct{34x}&&\:&&\:&&\:&&\:&&\:&&\:&&\:&&\:&&\oddline{E}
%   &\oct{35x}&&\:&&\:&&\:&&\:&&\:&&\:&&\:&&\:&\evenline
%   &\oct{36x}&&\:&&\:&&\:&&\:&&\:&&\:&&\:&&\:&&\oddline{F}
%   &\oct{37x}&&\:&&\:&&\:&&\:&&\:&&\:&&\:&&\:&\evenline}
% 
% % \font\xrm = cmr10 at 10pt % change this line to test font
% % \beginchart\xrm \normalchart \endchart
% 
% \newcommand{\smallfontchart}[1]{%
%   \font\xrm = #1 \beginchart\xrm \smallchart \endchart}
% \newcommand{\bigfontchart}[1]{%
%   \font\xrm = #1 \beginchart\xrm \bigchart \endchart}
% 
% % \smallfontchart{cmr10 at 10pt}
% % \bigfontchart{ecrm1000 at 10pt}
% \makeatother
% % End stolen from Dr Knuth



\title{fontinst}
\subtitle{Font installation software for \TeX}
\author{Alan Jeffrey\and Rowland McDonnell\and Lars Hellstr\"om}
\illustration{roadmap.eps}
\version{1.9}
\date{July 2004}

\errorcontextlines=999

\begin{document}

\maketitle

\begin{itshape}
  This manual is currently being rewritten, and may therefore be a 
  bit disorganised. For authoritative information on command syntaxes 
  and the like, see the Literate Programming sources for 
  \fontinst. The main file there is \textsl{\texttt{fisource.tex}}.
\end{itshape}

\bigskip

\begin{footnotesize}
  This manual describes the \texttt{fontinst} software for converting
  fonts from Adobe Font Metric format to forms readable by \TeX.  This
  manual should be distributed with the \texttt{fontinst} software,
  which is available by anonymous FTP from
  \url{ftp://ftp.tex.ac.uk/tex-archive/fonts/utilities/fontinst}, and
  on the various CD-ROMs containing material from the CTAN archives.
  Please do not contact the author directly for copies.

  If you would like to report a bug with \texttt{fontinst}, please
  mail \url{fontinst@tug.org}.  The mail will be sent to the
  \texttt{fontinst} mailing list.  If you would like to be on the
  \texttt{fontinst} mailing list, see 
  \url{http://tug.org/mailman/listinfo/fontinst}.
\end{footnotesize}

\tableofcontents

% \vfill

\clearpage


% \section{Introduction}
% 
% Digital fonts, as they may be bought from a foundry or other supplier, 
% are usually not directly usable with \LaTeX. An obvious problem can be 
% that the font information is not available in a format that \LaTeX\ 
% understands. A more subtle problem is that the fonts are often 
% organised in a way that is unsuitable for automatic typesetting. 
% % \Fontinst was written to overcome these problems.
% 
% % The main thing \fontinst does is creating virtual fonts (\vf's).
% 
% 
% % The \fontinst package is a set of \TeX\ macros written to create
% % virtual fonts for use with \TeX. 


\section{Defining terms}

This is rather a large and perhaps tedious section.  You might be
tempted to skip it so you can get to some more direct information on
how to use \fontinst.  That's fine if you understand everything about
how \TeX\ handles fonts.  If not, I suggest you at least skim though
this section.

\subsection{What's a font?}

Once upon a time, this question was easily answered: a font is a
set of type in one size, style, etc.  There used to be no
ambiguity, because a font was a collection of chunks of type metal
kept in a drawer, one drawer for each font.

These days, with digital typesetting, things are more complicated.
What a font `is' isn't easy to pin down.  A typical use of a \PS
font with \LaTeX\ might use these elements:
\begin{itemize}
\item Type 1 printer font file
\item Bitmap screen font file
\item Adobe font metric file (\afm file)
\item \TeX\ font metric file (\tfm file)
\item Virtual font file (\vf file)
\item font definition file (\fd file)
\end{itemize}
Looked at from a particular point of view, each of these files
`is' the font.  So what's going on?


\subsubsection{Type 1 printer font files}

These files contain the information needed by your printer to draw
the shapes of all the characters in a font.  They're typically
files with a \pfa or \pfb extension; on Macs they're usually in
files of type `LWFN' which usually have icons that look like a
laser printer.  The information in all these files is basically
the same: the only difference is in its representation.  \pfa
stands for `printer font ASCII', while \pfb stands for `printer
font binary'.  That is, \pfa files contain plain text information,
while \pfb files contain the same information encoded as
machine-readable binary data.

If you have Adobe Type Manager (ATM) installed on your computer,
ATM will use these files to draw an accurate representation of the
letters on the screen of your computer when you are previewing a
\TeX\ document.

Printer font files are not used directly by \TeX\ at all -- \TeX\
just prepares a \dvi file that refers to the fonts by name and the
characters by number: \TeX\ knows nothing about the shapes
involved.  The DVI driver uses the printer font files when you ask
it to print the \dvi file.  This means that you can produce a \dvi
file which uses, say, Palatino, even if you do not have the
Type~1 printer font file for this font on your computer.  You will
need to find a computer that does have Palatino before you can
print it or preview it, though.

Pdf\TeX\ is different from \TeX\ in this respect; since pdf\TeX\ 
integrates most of the functionality of a DVI driver, it may be unable 
to generate working \pdf output if the some Type~1 printer font file 
is not available.


\subsubsection{Bitmap screen font files}

These files contain a low-resolution bitmap for drawing a
representation of the font on the screen of your computer if ATM
is not installed.  In the \TeX\ world, these files are only used
for screen previews by the DVI driver.  They are kept in font
suitcase files on Macintoshes.
%\marginnote{And where on other computers?}


\subsubsection{Adobe font metric files (\afm files)}

These files are text files which contain information about the
size of each character in a font, kerning and ligature
information, and so on.  They can't be used by \TeX\ directly, but
the information they contain is essential if you want to use a
font with \TeX. \Fontinst can create from an \afm file the
necessary \tfm and \vf files so you can use a font with \TeX.
Once you have created all the files you need to use a font with
\TeX, you can remove the corresponding \afm files from your
computer unless you have other software that needs them.

The job of turning an \afm file into a set of \tfm and \vf files
is one of the main uses for \fontinst.  Most of this document is
concerned with this process, so don't worry if it seems a bit
vague at the moment.


\subsubsection{\TeX\ font metric files (\tfm files)}

These are binary data files in a format designed for use by \TeX\
which contain (more-or-less) the same information as \afm files:
the size of each character in a font (font metric data), kerning,
and ligature information.

When you select a font in \TeX, you are telling \TeX\ to typeset
using a particular \tfm file; from \TeX's point of view, a \tfm
file (and nothing else) \emph{is} a font.  \TeX\ itself doesn't
see printer font files, screen bitmaps, \pk files, \vf files, or
anything else to do with fonts: only \tfm files.

\TeX\ uses these \tfm files to decide where to put characters when
typesetting.  From \TeX's point of view, \tfm files are fonts,
even though they contain no information about the shape of
letters, and are not used by anything except \TeX\ -- once you
have produced a \dvi file, you don't need the \tfm files to print
it out.  (This is a slight lie: \dvips can read \tfm files
corresponding to \PS and \TT fonts so it can modify the metrics
slightly to improve the letterspacing at your chosen output
resolution.  This is an optional minor tweak and not an essential
part of the output process.)


\subsubsection{Property list files (\pl files)}

\pl files are human-readable text files which contain all the font
metric, kerning, ligature, and other information needed to create
a \tfm file.  You can convert between the two file formats using
\tftopl and \pltotf.


\subsubsection{Virtual font files (\vf files)}

These are binary data files in a format designed for use by \TeX\
\dvi drivers.  They're main purpose in life is to let you use
fonts in different encodings to the standard \TeX\ encodings.
These files are used by \dvi driver software only.

They are used only by \dvi drivers to work out what it should
\emph{really} print when you ask for a particular character. 
Technically they are like subroutine libraries for \dvi drivers, 
with one subroutine for each character in the virtual font: when 
the \dvi driver sees a \dvi command to set a character from a virtual 
font, it will execute a sequence of \dvi commands (the ``mapcommands 
property'' of this character) that it reads in the \vf file. You need 
not worry about the details of this, as \fontinst deals with them for 
you. Creating and using virtual fonts is what this document is about,
so don't worry if this doesn't make sense yet. (After all, how much 
do you need to know about the inner workings of \dvi files to typeset 
and print \TeX\ documents?)

Each \vf file has a \tfm file with the same name.  To use a
virtual font, you select the \tfm file as the font to use in your
document.  When the \dvi driver comes across this \tfm file in the
\dvi file, it looks up the \vf file and uses that to decide what
to do.


\subsubsection{Virtual property list files (\vpl files)}

\vpl files are human-readable text files which contain all the
font metric, kerning, mapping, and other information needed to
create a \vf and \tfm pair.

\vptovf will create a \vf/\tfm pair from a \vpl file.  \vftovp
will create a \vpl from a \vf/\tfm pair.  \vftovp also needs to be
able to read all the \tfm files that are referred to by a \vf to
recreate the \vpl\ -- it looks at the checksums to verify that
everything's okay.


\subsubsection{Font definition files (\fd files)}

These are files containing commands to tell \LaTeX\ which \tfm
files to associate with a request for a font using \LaTeX's font
selection commands.

For example, here is a small and edited part of the \fd file
supplied with \setpackagename{PSNFSS} to allow you to use the
Adobe Times font in T1 encoding:

\begin{verbatim}
  \ProvidesFile{t1ptm.fd}
      [1997/02/11 Fontinst v1.6 font definitions for T1/ptm.]

  \DeclareFontFamily{T1}{ptm}{}

  \DeclareFontShape{T1}{ptm}{m}{n} {<<->> ptmr8t}{}
  \DeclareFontShape{T1}{ptm}{m}{it}{<<->> ptmri8t}{}
  ...
  \DeclareFontShape{T1}{ptm}{b}{n} {<<->> ptmb8t}{}
  \DeclareFontShape{T1}{ptm}{b}{it}{<<->> ptmbi8t}{}
  ...
\end{verbatim}
What this means is: when you use \LaTeX\ to select the font family
|ptm| in T1 encoding in the medium series (|m|) and normal shape
(|n|), \TeX\ uses the font \texttt{ptmr8t.tfm}.  Similarly, if you
select bold italic, \TeX\ uses \texttt{ptmbi8t.tfm}.

\LaTeX\ works out which \fd file to load based on the current
encoding and font family selected.  If you've selected T1 encoded
|ptm| like this:
\begin{verbatim}
  \fontencoding{T1}\fontfamily{ptm}\selectfont
\end{verbatim}
\LaTeX\ loads the file \url{t1ptm.fd} (if it doesn't exist, you're
in trouble).  As you can see above, this file contains information
so that \LaTeX\ knows which \tfm file to use.  So if you ask for,
say, |T1/ptm/b/it| (T1 encoded Times-Roman, bold series, italic
shape), you get the font \texttt{ptmbi8t}.

You can find more about \fd files and \LaTeX's font selection commands
at CTAN:
\url{ftp://ftp.tex.ac.uk/tex-archive/macros/latex/base/fntguide.tex}
and \url{ftp://ftp.tex.ac.uk/tex-archive/info/simple-nfss.tex} are
both useful.


\subsection{What does \fontinst do?}

\fontinst creates \vpl and \pl files from \afm or \pl files to map
any glyph or combination of glyphs in the original font files to
any slot in the output font file.  There, isn't that better?  Off
you go now\ldots

If you're still confused, I'll explain a few things.
\begin{description}
\item[Glyph] 
  A glyph is an image, often associated with one or several characters. 
  Some examples of glyphs are: 
  `A', `\textsf{A}', `$\mathcal{A}$', `B', `F', 
  `{\usefont{OT1}{cmr}{m}{n}f}', 
  `{\usefont{OT1}{cmr}{m}{n}fi}', `\~{}'.
  Fonts are collections of glyphs. \Fontinst refers to glyphs by name.
\item[Slot] 
  This is jargon for `a numbered position in a font'. (What is 
  important is the number, and that this number refers to a position 
  in a font, but which font is usually irrelevant.)
\item[Character]
  The modern definition is that a character is the smallest component 
  of written language that has semantic value. Speaking of a 
  character, one refers to the abstract meaning and/or shape, rather 
  than a specific shape.
  
  Since fonts have often contained a unique glyph for each character 
  and each usable glyph has been assigned a particular slot, 
  it is not uncommon (in particular in older terminology) to see the 
  three mixed up, usually so that one says `character' where one of 
  the other two would have been more correct. The \TeX-related font 
  file formats is no exception, as you may see examples of elsewhere 
  in this document.
\item[Encoding] 
  There are really two different encoding concepts that one encounters 
  when using \fontinst. The differences are not great, and an 
  encoding of one kind often corresponds to an encoding of the other 
  kind, but it is not merely a matter of translation.
  
  A \emph{\LaTeX\ encoding} is a mapping from characters (or more 
  formally \LaTeX\ Internal Character Representations) to slots. 
  In the \texttt{OT1} encoding, `\o' (or more technically `|\o|') maps 
  to slot~28, whereas in the \texttt{T1} encoding it maps to slot~248. 
  This kind of encoding affects what \TeX\ is doing; \dvi drivers are 
  not involved.
  
  A \emph{font encoding} (or \emph{encoding vector}) is a mapping from 
  slots to glyph names. This is the kind of encoding that \fontinst 
  primarily deals with, and also the kind of encoding that \dvi 
  drivers make use of. \texttt{ot1.etx} associates slot~28 with 
  `\texttt{oslash}', whereas \texttt{t1.etx} and \texttt{EC.enc} (one 
  of several to \texttt{T1} corresponding encoding vectors that come 
  with \dvips) associates slot~28 with `\texttt{fi}'.
  
  \LaTeX\ encodings occur in \fontinst only as names and only in 
  relation to \fd files. It is unlikely that you will need to create 
  one of your own. The mappings defined by font encodings are on the 
  other hand of great importance and \etx files are used to direct 
  the generation of virtual fonts. Advanced \fontinst users may well 
  find that they need to create new font encodings to achieve their 
  goals.
\end{description}

% You're probably familiar with ASCII
% encoding, which has the letter `A' in slot~65, `B' in slot~66,
% and so on.  That's it, really.  \TeX\ uses several different
% encodings.  The most common ones are OT1 (the original \TeX\
% 7~bit encoding) and T1 (the newer \TeX\ 8~bit encoding).
% % -- much more on this later.

The thing is that the average \PS font comes in Adobe standard
encoding, which, for example, has the glyph dotless~i `\i' in
slot~245.  But \TeX\ \texttt{T1} encoding expects the glyph o~dieresis
`{\"o}' in that slot, and wants dotless~i in slot~25.  So if you
tried to use a raw \PS font with \TeX, any time you tried to get
a `{\"o}', you'd get a `\i'; and every time you tried to get a `\i',
you'd get a blank, because Adobe standard encoding says that
slot~25 is empty.  The process of dealing with this problem is
called `re-encoding', and is what \fontinst helps with.

This might not make much sense yet; the best thing to do is relax.
There's a lot of things that need to be dealt with when you're
setting up \LaTeX\ to use a new font, so you can expect to be a
bit confused until you've done it a few times.


\subsection{What do you do with \fontinst?}

If you're using \fontinst, the usual steps you need to take to use
an ordinary \PS text font with \LaTeX\ are these:

\begin{enumerate}
\item Give the \afm files an appropriate name.
\item Use \fontinst to produce an |8r| encoded \pl files from
  these \afm files.
\item Use \fontinst to create T1 and OT1 encoded \pl and \vpl
  files from the |8r| encoded \pl files (this procedure will also
  create suitable \fd files).
\item Use \pltotf to turn each \pl file into a \tfm file.
\item Use \vptovf to turn each \vpl file into a pair of \vf and
  \tfm files.
\item Move the \tfm, \vf, and \fd files into the appropriate
  directories so \LaTeX\ can see them.
\item Tell your DVI driver about the new font (edit \dvips's
  \texttt{psfonts.map} file, or \OzTeX's \texttt{Default}
  configuration file.
\item Perhaps write a package file to make selecting the new font
  a little easier.
\item Test it.
\end{enumerate}


\section{Fontmaking commands}

There are three main types of files that you may write to control what 
\fontinst does: \emph{command files} (usually with suffix 
\setdotfilename{tex}), \emph{encoding definition files} (suffix 
\setdotfilename{etx}), and \emph{metric files} (suffix 
\setdotfilename{mtx}). Command files directly tell \fontinst to do 
things, whereas the purpose of an encoding or metric file is more to 
store data, but all three file types are technically sequences of 
\TeX\ commands that \fontinst execute when reading the file. Normal 
\TeX\ syntax rules apply in all three file types, although a few 
commands may behave in unfamiliar ways.

Within the command file category, it is possible to discern certain 
subcategories. Most command files are written for one particular 
task, but some are common pieces that have been factored out from 
larger command files and are merely meant to be |\input| where 
appropriate. (\setfilename{csc2x.tex} in the \fontinst distribution is 
an example of this latter kind.) One may also distinguish between 
command files that are made for use with \setfilename{fontinst.sty} 
command definitions and command files that are made for use with 
\setfilename{finstmsc.sty} command definitions. This section documents 
the commands that are particular to the former category, whereas the 
next section documents commands that are particular to the latter.


\subsection{Install commands}

The core fontmaking takes place within a block of ``install 
commands''. (This name is a bit unfortunate since nothing is actually 
installed; rather some files that need to be installed are generated.) 
Such blocks have the structure
\begin{quote}
  |\installfonts|\\
  \m{install commands}\\
  |\endinstallfonts|
\end{quote}
The \m{install commands} describe the fonts, glyphs and encodings used
to build fonts, whereas the purpose of the delimiting |\installfonts| 
and |\endinstallfonts| are rather to organise the writing of \fd files.
\begin{decl}
  |\installfonts|\\
  |\endinstallfonts|
\end{decl}
At |\installfonts|, \fontinst's internal list of \fd files to generate 
are cleared. At |\endinstallfonts|, \fd files are written for those 
combinations of encoding and font family that appeared in the 
\meta{install commands}.

\begin{hackernote}
  |\installfonts|, |\endinstallfonts|, and the individual 
  install commands between them also cooperate in a rather complicated 
  grouping scheme to cache glyphbases. This may interfere with 
  non-\fontinst commands in the \meta{install commands}. If for 
  example an assignment to some |\tracing|\dots\ parameter here does 
  not seem to have any effect, try making the assignment |\global|.
\end{hackernote}

The most important \meta{install command} is
\begin{decl}
  |\installfont|\arg{font-name}\arg{metrics-list}\arg{etx-list}\\
  \hspace*{1.5em}\arg{encoding}\arg{family}\arg{series}%
  \arg{shape}\arg{size}
\end{decl}
This produces a \TeX\ virtual font called \m{font-name}. The 
\meta{metrics-list} and the \meta{etx-list} determine this font, 
whereas the other arguments specify how the \fd file will declare it 
for \LaTeX. The \m{encoding}, \m{family}, \m{series}, and \m{shape} 
are precisely the NFSS parameters. The \m{size} is either a shorthand 
declared by \verb|\declaresize| (see below), or is an \fd size 
specification.

Like most \fontinst lists, the elements in the \meta{metrics-list} and 
\meta{etx-list} are separated by commas (so-called comma-separated 
lists). In their simplest form, the elements of these lists are file 
names (minus suffixes): \mtx files in the \meta{metrics-list} and 
\etx files in the \meta{etx-list}. First the \mtx files are processed 
to build up a glyphbase, i.e., store definitions of glyphs and their 
metric properties in memory, and then the \etx files are processed 
(several times) to select a set of glyphs and write the corresponding 
information to a \vpl file.

For example, to install the T1-encoded Times Roman font
(using \texttt{t1.etx} and \texttt{latin.mtx}), you say:
\begin{verbatim}
  \installfont{ptmr8t}{ptmr8r,latin}{t1}
    {T1}{ptm}{m}{n}{}
\end{verbatim}
To install a OT1-encoded Times Roman font with a scaled version of
Symbol for the Greek letters, you say:
\begin{verbatim}
  \installfont{zptmrsy}{ptmr8r,psyr scaled 1100,latin}{ot1}
    {OT1}{ptm}{m}{n}{}
\end{verbatim}

As the second example indicates, there is more to the list items than 
just file names. In the case of an metrics list item, the syntax 
permits the two forms
\begin{decl}
  \meta{filename}\meta{optional modifiers}\\
  |\metrics| \meta{metric commands}
\end{decl}
\NEWfeature{v1.923}
where an \meta{optional modifier} is one of
\begin{decl}
  \verb*| scaled |\meta{rawscale factor}\\
  \verb*| suffix |\meta{glyph name suffix}\\
  \verb*| encoding |\meta{etx}\\
  \verb*| option |\meta{string}
\end{decl}
A list item may contain several such modifiers, but most commonly it 
does not contain any. The 
\meta{metric commands} are explicit metric commands, as described in 
Section~\ref{Sec:Metric}; this latter feature is meant for minor 
adjustments that you don't want to bother creating a separate \mtx 
file for.

The \meta{filename} above primarily refers to a file 
\meta{filename}\texttt{.mtx}, but that need not always exist before 
executing the above command. If there exists a \pl, \afm, or \vpl file 
with the right name then that is first converted to a corresponding 
\mtx file. However, a special case occurs if there is an |encoding| 
modifier: this forces conversion of a \pl or \vpl file even if an 
\mtx file exists, and also forces using the specified \etx file when 
assigning glyph names to the slots of that file. Normally the choice 
of \etx file for such conversions to \mtx is based on 
|\declareencoding| declarations.

The |scaled| modifier sets the \texttt{rawscale} variable for the 
processing of that file. This has the effect of scaling all raw 
glyphs from that file to \meta{rawscale factor} per milles of their 
previous size. The |suffix| modified causes the \meta{glyph name 
suffix} to be implicitly appended to all glyphs defined by this file. 
The |option| modifier adds the \meta{string} to the list of ``options'' 
for this file. The |\ifoption| command can be used in the file to test 
whether a particular string has been supplied as an option.

\begin{hackernote}
  In general, \fontinst commands process comma-separated list arguments 
  by first splitting at commas and then fully expanding each item, but 
  this \meta{metrics-list} argument is an exception. This is first 
  fully expanded (|\edef|) and then split into items. The difference 
  is that a macro used in this \meta{metrics-list} argument can expand 
  to several list items, whereas a macro used in an ordinary 
  comma-separated list argument can only expand to (part of) a single 
  list item.
  
  The |\metrics| list items do however constitute an exception within 
  this exception. These list items are in their entirety protected 
  from the initial full expansion, so you don't have to worry about 
  peculiar fragility errors there.
\end{hackernote}

The elements in the \meta{etx-list} have fewer variants, but there is 
still a general syntax
\begin{decl}
  \meta{filename}\meta{optional modifiers}
\end{decl}
The only \meta{optional modifier} permitted is however
\begin{decl}
  \verb*| mtxasetx|
\end{decl}
and that is probably only relevant for use with |\installrawfont| (see 
below).


\begin{decl}
  |\installfontas|\arg{font-name}\\%
  \hspace*{1.5em}\arg{encoding}\arg{family}\arg{series}%
  \arg{shape}\arg{size}
\end{decl}
\NEWfeature{v1.912}
This install command adds an \fd entry for the \meta{font-name}, but 
it doesn't actually generate that font. Usually that font was 
generated by a previous |\installfont|, and this is used to create 
additional entries for the font.


\begin{decl}
  |\installrawfont|\arg{font-name}\arg{metrics-list}\arg{etx-list}\\
  \hspace*{1.5em}\arg{encoding}\arg{family}\arg{series}%
  \arg{shape}\arg{size}
\end{decl}
This is similar to |\installfont| except that it produces
a \TeX\ raw font as \pl file rather than a virtual font. Often a \pl 
font with the specified name will already exist when this command is 
called, and that will then be overwritten. These two \pl files will 
typically be somewhat different.
The normal reason for using this command is that one wishes to 
``refine'' the metrics of a font that was generated by transformation 
commands.

For example, to install an 8r-encoded Times Roman raw font
(using \texttt{8r.etx} and \texttt{8r.mtx}), you say:
\begin{verbatim}
  \installrawfont{ptmr8r}{ptmr8r,8r}{8r}
    {8r}{ptm}{m}{n}{}
\end{verbatim}

The effect of a
\begin{decl}
  \meta{filename}\verb*| mtxasetx|
\end{decl}
\NEWfeature{v1.923}
in the \meta{etx-list} is not that \meta{filename}\texttt{.etx} is 
read, but that \meta{filename}\texttt{.mtx} is read. The 
interpretation of the commands in this file is however not the 
customary, and almost the only thing paid attention to is the 
correspondence between glyph names and slot numbers that is provided 
by the |\setrawglyph| and |\setscaledrawglyph| commands; this 
correspondence is treated as if it was given by |\setslot| \dots\ 
|\endsetslot| commands in an \etx file. This is however only 
guaranteed to work with transformable metric files.

The purpose of this feature is to simplify installation of fonts with 
very special encodings, such as ``Dingbat'' or ``Pi'' fonts. Instead 
of creating an \etx file, which would probably only be useful with 
that particular font, one can make use of the fact that the 
interesting information is anyway available in the \mtx file. To 
install Zapf Dingbats in their default encoding, one can thus say
\begin{verbatim}
  \installrawfont{pzdr}{pzdr}{pzdr mtxasetx}
    {U}{pzd}{m}{n}{}
\end{verbatim}

Unlike the case with |\installfont|, which actually creates a real 
(although virtual) font, |\installrawfont| can only create the 
metrics for a font. The \dvi driver will require some other kind of 
implementation of this font, usually an entry in some map file (e.g. 
\texttt{psfonts.map}, in the case of \dvips) that links the \TeX\ font 
name to e.g.\ a \PS\ font name and file. (Many \dvi drivers are 
configured in such a way that they, without such a map file entry, 
will call Metafont with the font name and thereby raise a sequence of 
error messages about a \setdotfilename{mf} that doesn't exist. These 
results are often rather confusing.)



\begin{decl}
  |\installfamily|\arg{encoding}\arg{family}\arg{fd-commands}
\end{decl}
This tells \fontinst to write an \fd file for the given combination 
of encoding and family, and clears the internal list of entries to 
put in that file. |\installfamily| commands usually come first in each 
block of \meta{install commands}.

For example, to produces a \LaTeX\ family with the Cork-encoded Times 
family, you say:
\begin{verbatim}
  \installfamily{T1}{ptm}{}
\end{verbatim}
The \m{fd-commands} are executed every time a font in that family is
loaded, for example to stop the Courier font from being hyphenated you
say:
\begin{verbatim}
  \installfamily{T1}{pcr}{\hyphenchar\font=-1}
\end{verbatim}
In more recent versions of \fontinst, the |\installfamily| command is 
only necessary if you want the \m{fd-commands} argument to be 
nonempty, but it doesn't hurt to make it explicit.




\subsection{Transformation commands}


\begin{decl}
  |\transformfont|\arg{font-name}\arg{transformed font}
\end{decl}
This makes a raw transformed font, for example expanded, slanted,
condensed or re-encoded.  \emph{It is the responsibility of the device
  driver to implement this transform.}  Each \verb|\transformfont|
command writes out an \mtx file and a raw \pl file for \m{font-name}.

A \m{transformed font} is given by the following commands:
\begin{decl}
  |\fromafm|\arg{afm}\\
  |\fromany|\arg{whatever}\\
  |\frompl|\arg{pl}\\
  |\fromplgivenetx|\marg{pl}\marg{etx}\\
  |\frommtx|\arg{mtx}
\end{decl}
This reads the metrics of a font which is about to be transformed from
an external file. |\fromafm|, |\frompl|, and |\fromplgivenetx| write out 
an \mtx file corresponding to the \afm or \pl file.  In addition, 
|\formafm| also writes out a raw \pl file, containing just the glyph 
metrics but no kerning information. |\fromplgivenetx| permits 
specifying which encoding file to use when associating glyph names to 
slots, whereas |\frompl| tries to guess this from the 
\texttt{CODINGSCHEME} property of the \pl file. |\fromany| looks for 
a file in any of the formats (in the order \mtx, \pl, \afm) and 
behaves as the first |\from|\dots\ for which it found a file.

A \m{transformed font} may also be one of the following:
\begin{decl}
  |\scalefont|\arg{integer expression}\arg{transformed font}\\
  |\xscalefont|\arg{integer expression}\arg{transformed font}\\
  |\yscalefont|\arg{integer expression}\arg{transformed font}\\
  |\slantfont|\arg{integer expression}\arg{transformed font}
\end{decl}
This applies a geometric transformation to the font metrics of
\m{transformed font}.  The scale factor or slant factor are given in
1000 units to the design size.  Typical examples are 167 for slanted
fonts or 850 for condensed fonts.

The final case of a \m{transformed font} is:
\begin{decl}
  |\reencodefont|\arg{etx}\arg{transformed font}
\end{decl}
This rearranges the encoding vector of \m{transformed font} to match 
the encoding given by the \etx file.


For example, to create an oblique, |8r|-encoded version of Adobe Times
called \texttt{ptmro8r} you say:
\begin{verbatim}
  \transformfont{ptmro8r}{\reencodefont{8r}{\slantfont{167}{\fromafm{ptmr8a}}}}
\end{verbatim}
This will create \texttt{ptmr8a.mtx}, \texttt{ptmr8a.pl},
\texttt{ptmro8r.mtx} and \texttt{ptmro8r.pl}, which can then be used
as raw fonts in \verb|\installfont| commands.  The same transformation
can also be achieved in two steps:
\begin{verbatim}
  \transformfont{ptmr8r}{\reencodefont{8r}{\fromafm{ptmr8a}}}
  \transformfont{ptmro8r}{\slantfont{167}{\frommtx{ptmr8r}}}
\end{verbatim}
This will create \texttt{ptmr8a.mtx}, \texttt{ptmr8a.pl},
\texttt{ptmr8r.mtx}, \texttt{ptmr8r.pl}, \texttt{ptmro8r.mtx} and
\texttt{ptmro8r.pl}.

You will have to inform your device driver about the transformed font,
using the syntax appropriate for that driver.  For example, in \dvips
you add a line to \texttt{psfonts.map}:
\begin{verbatim}
  ptmro8r Times-Roman ".167 SlantFont TeXBase1Encoding ReEncodeFont" <<8r.enc
\end{verbatim}


\subsection{Miscellaneous settings}

\begin{decl}
  |\substitutesilent|\arg{to}\arg{from}\\
  |\substitutenoisy|\arg{to}\arg{from}
\end{decl}
This declares a \LaTeX\ font substitution, that the series or shape
\m{to} should be substituted if necessary by the series or shape
\m{from}.  \verb|\substitutesilent| means that when the font
substitution is made, no warning will given.

\verb|\substitutenoisy| is the same as \verb|\substitutesilent|, but
gives a warning when the substitution is made by \LaTeX.

For example, to say that the series |bx| can be replaced by the
series |b|, you say:
\begin{verbatim}
  \substitutesilent{bx}{b}
\end{verbatim}
To say that the shape |ui| can be replaced by the shape |i|, you say:
\begin{verbatim}
  \substitutenoisy{ui}{it}
\end{verbatim}

The following weight substitutions are standard:
\begin{verbatim}
  \substitutesilent{bx}{b}
  \substitutesilent{b}{bx}
  \substitutesilent{b}{sb}
  \substitutesilent{b}{db}
  \substitutesilent{m}{mb}
  \substitutesilent{m}{l}
\end{verbatim}
The following shape substitutions are standard:
\begin{verbatim}
  \substitutenoisy{ui}{it}
  \substitutesilent{it}{sl}
  \substitutesilent{sl}{it}
\end{verbatim}

The |\installfontas| command should be considered as an 
alternative to using font substitution, as it gives much finer 
control over what \fd entries will be made.


\begin{decl}
  |\declaresize|\arg{size}\arg{fd-size-range}
\end{decl}
This declares a new size, and gives the \fd commands for it.  For
example, \url{fontinst.sty} declares the following sizes:
\begin{verbatim}
  \declaresize{}{<<->>}
  \declaresize{5}{<<5>>}
  \declaresize{6}{<<6>>}
  \declaresize{7}{<<7>>}
  \declaresize{8}{<<8>>}
  \declaresize{9}{<<9>>}
  \declaresize{10}{<<10>>}
  \declaresize{11}{<<10.95>>}
  \declaresize{12}{<<12>>}
  \declaresize{14}{<<14.4>>}
  \declaresize{17}{<<17.28>>}
  \declaresize{20}{<<20.74>>}
  \declaresize{25}{<<24.88>>}
\end{verbatim}

\begin{decl}
  |\declareencoding|\arg{string}\arg{etx}
\end{decl}
This declares which \etx file corresponds to which encoding
string.  For example, \url{fontinst.sty} declares the following
encoding strings:
\begin{verbatim}
  \declareencoding{TEX TEXT}{ot1}
  \declareencoding{TEX TEXT WITHOUT F-LIGATURES}{ot1}
  \declareencoding{TEX TYPEWRITER TEXT}{ot1tt}
  \declareencoding{TEX MATH ITALIC}{oml}
  \declareencoding{TEX MATH SYMBOLS}{oms}
  \declareencoding{TEX MATH EXTENSION}{omx}
  \declareencoding{EXTENDED TEX FONT ENCODING - LATIN}{t1}
  \declareencoding{TEX TEXT COMPANION SYMBOLS 1---TS1}{ts1}
  \declareencoding{TEXBASE1ENCODING}{8r}
  \declareencoding{TEX TYPEWRITER AND WINDOWS ANSI}{8y}
\end{verbatim}





\subsection{Other}

The following commands belong to this section, but there is currently 
no description of them here.


|\afmtomtx|\marg{afmfile}\marg{mtxfile}

|\endrecordtransforms|

|\endreglyphfonts|

|\etxtopl|\marg{encoding list}\marg{plfile}

|\etxtovpl|\marg{encoding list}\marg{vplfile}

|\fakenarrow|\marg{width factor}

|\generalpltomtx|\marg{plfile}\marg{mtxfile}\marg{plsuffix}\marg{opt-enc}

|\killglyph|\marg{glyph}

|\killglyphweighted|\marg{glyph}\marg{weight}

|\latinfamily|\marg{family}\marg{commands}

|\mtxtomtx|\marg{source MTX}\marg{destination MTX}

|\mtxtopl|\marg{mtxfile}\marg{plfile}

|\NOFILES|

|\offmtxcommand|\marg{command}

|\onmtxcommand|\marg{command}

|\recordtransforms|\marg{filename}

|\reglyphfont|\marg{destination font}\marg{source font}

|\reglyphfonts|

|\renameglyph|\marg{to}\marg{from}

|\renameglyphweighted|\marg{to}\marg{from}\marg{weight}



\section{Mapmaking commands}

The following commands belong to this section, but there is currently 
no description of them here. Note that several of them have a 
different syntax and meaning than they do in fontmaking command files.

|\adddriver|\marg{driver name}\marg{fragment file name}

|\AssumeAMSBSYY|

|\AssumeBaKoMa|

|\AssumeLWFN|

|\AssumeMetafont|

|\debugvalue|\arg{name}

|\declarepsencoding|\marg{etx}\marg{postscript name}\marg{action}

|\donedrivers|

|\download|\marg{file}

|\enctoetx|\marg{encfile}\marg{etxfile}

|\etxtoenc|\marg{etxfile}\marg{encfile}

|\fromafm|\marg{AFM name}\marg{PS name}

|\frompl|\marg{PL name}

|\frommtx|\marg{MTX name}

|\fromvpl|

|\makemapentry|\marg{\TeX\ font name}

|\reencodefont|\marg{etx}

|\reglyphfont|

|\specifypsfont|\marg{PS font name}\marg{actions}

|\storemapdata|\marg{\TeX\ font name}\marg{source}\marg{transforms}

|\transformfont|\marg{x-scale}\marg{slant-scale}



\section{General commands}



This section describes commands and mechanisms that are the same in 
all file types. Commands that are particular for one type of file are 
described in subsequent sections.


\subsection{Variables}

Many (but not all) of the activities \fontinst perform can be 
understood as either ``setting variables'' or ``formatting and writing 
to file data stored in some variable''. The accessing of variables is 
an important aspect of how \fontinst works.

Variables come in different types and variables of different types 
live in different namespaces; |\int{foo}|, |\str{foo}|, and 
|\dim{foo}| refer to three different variables which are all named 
|foo|. Variables are either set or not set. Unless the contrary is 
stated explicitly, variables default to not being set. It is an error 
to access the value of a variable that has not been set. \Fontinst 
variable assignments are as a rule local, i.e., will be undone when 
the enclosing \TeX\ group is ended. Most command file commands that 
cause files to be read will begin a group before reading the file(s) 
and end the group at some point after having read them.

Taking string variables as an example, there are three commands for 
changing a string variable:
\begin{decl}
  |\setstr|\arg{name}\arg{string expression}\\
  |\resetstr|\arg{name}\arg{string expression}\\
  |\unsetstr|\arg{name}
\end{decl}
The |\resetstr| command unconditionally sets the string variable 
\meta{name} to the full expansion of the \meta{string expression}. 
The |\unsetstr| command unconditionally renders the string variable 
\meta{name} unset. If the the string variable \meta{name} is 
currently unset then the |\setstr| command will set it to the full 
expansion of the \meta{string expression}, but if it already is set 
then |\setstr| does nothing.

This pattern with three commands, one |\set|\dots\ which only sets 
unset variables, one |\reset|\dots\ which sets variables regardless of 
whether they have been set or not, and one |\unset|\dots\ which unsets 
variables is recurring in \fontinst. Variables are most commonly set 
using some |\set|\dots\ command; this has the effect that the first 
command to try to set a variable is the one which actually sets it. 



% \subsection{Argument types}
% 
% There are roughly five types of arguments that \fontinst 
% commands can take. These are
% \begin{itemize}
%   \item integer expressions,
%   \item string expressions,
%   \item dimensions,
%   \item commands (i.e., \TeX\ control sequences), and
%   \item other (pretty much ``none of the above'').
% \end{itemize}
% The most common form of an integer expression is simply a \TeX\ 
% \meta{number} and the most common form of a string expression is 
% simply a sequence of character tokens, but there are more complicated 
% forms. Dimensions are simply \TeX\ \meta{dimen}s; their use is rather 
% limited. Common to integer expressions, string expressions, and 
% dimensions is that these argument types get expanded during 
% evaluation (in the case of string expressions, this expansion 
% \emph{is} the evaluation), which means one can use macros in 
% arguments of these types.
% 
% Command arguments do not get expanded---they are mainly used with 
% commands that modify the definitions of other commands. As for the 
% ``other'' arguments one cannot give any rules: they might get 
% expanded, but it could also happen that they won't.





% There are three types of files used by the \fontinst package:
% \begin{itemize}
% \item \emph{fontinst files} contain commands to process fonts
%   metrics so you can use a font with \TeX. For example,
%   \texttt{fontptcm.tex} is a \emph{fontinst file}.
% 
% \item \emph{encoding files} contain information about an encoding,
%   including the code table, ligatures, and font dimensions.  For
%   example, \texttt{8r.etx} is an \emph{encoding file}.
% 
% \item \emph{metric files} contain information about glyphs,
%   including glyph dimensions, composite characters, and kerning.
%   For example, \texttt{latin.mtx} is a \emph{metric file}.
% \end{itemize}


\subsection{General commands}

The following \emph{general commands} can be used anywhere:

\begin{decl}
  |\needsfontinstversion|\arg{version}
\end{decl}
This issues a warning if the current version of the \fontinst
package is less than \m{version}.

\begin{decl}
  |\setdim|\arg{dim}\arg{dimension}\\
  |\setint|\arg{int}\arg{integer expression}\\
  |\setstr|\arg{str}\arg{string}
\end{decl}
If the dimension variable \m{dim} is currently undefined, it is
defined to be the current value of \m{dimension}.

If the integer variable \m{int} is currently undefined, it is
defined to be the current value of \m{integer expression}.

If the string variable \m{str} is currently undefined, it is
defined to be the current value of \m{string}.

\begin{decl}
  |\setcommand|\arg{command}\meta{parameter text}\arg{replacement text}
\end{decl}
If the command \m{command} is currently undefined, it is defined
to be the \m{definition}.  This uses the same syntax for
parameters as the \TeX\ \verb|\def| command.
\begin{center}
  \begin{tabular}{r l}
    No. parameters& \meta{parameter text}\\
    0& (empty)\\
    1& |#1|\\
    2& |#1#2|\\
    3& |#1#2#3|\\
    \multicolumn{2}{c}{and so on.}
  \end{tabular}
\end{center}

\begin{decl}
  |\resetdim|\arg{dim}\arg{dimension}\\
  |\resetint|\arg{int}\arg{integer expression}\\
  |\resetstr|\arg{str}\arg{string}
\end{decl}
The dimension variable \m{dim} is defined to be the current value
of \m{dimension}.

The integer variable \m{int} is defined to be the current value of
\m{integer expression}.

The string variable \m{str} is defined to be the current value of
\m{string}.

\begin{decl}
  |\resetcommand|\arg{command}\meta{parameter text}\arg{replacement text}
\end{decl}
The command \m{command} is defined to be the \m{definition},
regardless of whether it was already defined or not.  This is a
synonym for the \TeX\ \verb|\def| command.

\begin{decl}
  |\ifisint|\arg{int}|\then|\\
  |\ifisdim|\arg{dim}|\then|\\
  |\ifisstr|\arg{str}|\then|\\
  |\ifisglyph|\arg{glyph}|\then|\\
  |\ifiscommand|\arg{command}|\then|
\end{decl}
Expands out to \verb|\iftrue| if the integer variable
\m{int} is defined, and \verb|\iffalse| otherwise.

Expands out to \verb|\iftrue| if the dimension variable
\m{dim} is defined, and \verb|\iffalse| otherwise.

Expands out to \verb|\iftrue| if the string variable
\m{str} is defined, and \verb|\iffalse| otherwise.

Expands out to \verb|\iftrue| if the glyph variable
\m{glyph} is defined, and \verb|\iffalse| otherwise.

Expands out to \verb|\iftrue| if the command
\m{command} is defined, and \verb|\iffalse| otherwise.

\begin{decl}
  |\unsetdim|\arg{dim}\\
  |\unsetint|\arg{int}\\
  |\unsetstr|\arg{str}\\
  |\unsetcommand|\arg{command}
\end{decl}
Makes \m{dim}, \m{int}, \m{str}, or \m{command} an undefined
dimension, integer, string or command.

\subsection{Integer expressions}
\label{Sec:integer}

The \emph{integer expressions} provide a user-friendly syntax for
\TeX\ arithmetic.  They are used to manipulate any integers,
including glyph dimensions, which are given in \afm units, that is
1000 to the design size.  \TeX\ \pl fonts have their dimensions
converted to \afm units automatically.

The \emph{integer expressions} are:

\begin{decl}
  \m{number}
\end{decl}
Returns the value of a \TeX\ \m{number} (as explained in \emph{The \TeX book}).

\begin{decl}
  |\int|\arg{int}
\end{decl}
Returns the value of the integer variable \m{int}.

\begin{decl}
  |\width|\arg{glyph}\\
  |\height|\arg{glyph}\\
  |\depth|\arg{glyph}\\
  |\italic|\arg{glyph}
\end{decl}
Returns the width, height, depth, or italic correction of the glyph
variable \m{glyph}.

\begin{decl}
  |\kerning|\arg{left}\arg{right}
\end{decl}
Returns the kerning between the \m{left} and \m{right} glyph
variables.

\begin{decl}
  |\neg|\arg{integer expression}\\
  |\add|\arg{integer expression}\arg{integer expression}\\
  |\sub|\arg{integer expression}\arg{integer expression}\\
  |\max|\arg{integer expression}\arg{integer expression}\\
  |\min|\arg{integer expression}\arg{integer expression}\\
  |\mul|\arg{integer expression}\arg{integer expression}\\
  |\div|\arg{integer expression}\arg{integer expression}\\
  |\scale|\arg{integer expression}\arg{integer expression}\\
  |\half|\arg{integer expression}\\
  |\otherhalf|\arg{integer expression}
\end{decl}
|\neg| returns the negation of the \m{integer expression}.

|\ad| returns the sum of the two \m{integer expression}s.

|\sub| returns the first \m{integer expression} minus the second.

|\mul| returns the product of the two \m{integer expression}s.

|\div| returns the first \m{integer expression} divided by the second.

|\scale| returns the first \m{integer expression} times the second,
divided by 1000.


\subsection{Other}

The following commands belong to this section, but there is currently 
no description of them here.

|\begincomment|

|\bye|

|\Else|

|\endcomment|

|\endfor|\parg{name}

|\eTeX|\marg{version number}

|\Fi|

|\fontinstcc|

|\fontinsterror|\marg{subsystem}\marg{error}\marg{help}

|\fontinstinfo|\marg{subsystem}\marg{info}

|\fontinstwarning|\marg{subsystem}\marg{warning}

|\fontinstwarningnoline|\marg{subsystem}\marg{warning}

|\for|\parg{name}\marg{start}\marg{stop}\marg{step}

|\foreach|\parg{name}\marg{csep-list}

|\ifareglyphs|\marg{glyph list}|\then|

|\ifiskern|\marg{glyph1}\marg{glyph2}|\then|

|\ifnumber|\marg{integer expression}\meta{rel}%
\marg{integer expression}|\then|

|\ifoption|\marg{string}|\then|

|\input| \meta{file name}

|\messagebreak|

|\needsTeXextension|\marg{extension tests}\marg{who}

|\normalcc|

|\offcommand|\marg{command}

|\oncommand|\marg{command}

|\pdfTeX|\marg{version number}\marg{revision}

|\strint|\arg{int}



\section{Encoding files}

An \emph{encoding file} (or \texttt{.etx} file) is a \TeX\ document
consisting of:

\begin{decl}
  |\relax|\\
  \emph{ignored material}\\
  |\encoding|\\
  \m{encoding commands}\\
  |\endencoding|\\
  \emph{ignored material}
\end{decl}
This describes the encoding of a font, using the \m{encoding
  commands}.

Since the encoding file ignores any material between \verb|\relax| and
\verb|\encoding|, an \emph{encoding file} can also be a \LaTeX\
document.

See also the descriptions in \texttt{encspecs.tex}.


\subsection{Encoding commands}

The \m{encoding commands} are:

\begin{decl}
  |\nextslot|\arg{integer expression}
\end{decl}
Sets the number of the next slot.  If there is no \verb|\nextslot|
command, the number is the successor of the previous slot.

\begin{decl}
  |\skipslots|\arg{integer expressions}
\end{decl}
\NEWfeature{v1.8}
Advances the number of the next slot.

\begin{decl}
  |\setslot|\arg{glyph}\\
  \m{slot commands}\\
  |\endsetslot|
\end{decl}
Sets the slot of the \m{glyph}.  The \m{slot commands} describe
the glyph, and give its usage in \TeX.

\begin{decl}
  |\inputetx|\arg{file}
\end{decl}
Inputs the \m{encoding commands} of \m{file}\texttt{.etx}.



\subsection{Slot commands}

The \m{slot commands} are:

\begin{decl}
  |\comment|\arg{text}
\end{decl}
A comment, which is ignored by \fontinst.

\begin{decl}
  |\label|\arg{text}
\end{decl}
A reference label. Ignored by \fontinst.

\begin{decl}
  |\ligature|\arg{ligtype}\arg{glyph}\arg{glyph}\\
  |\Ligature|\arg{ligtype}\arg{glyph}\arg{glyph}\\
  |\oddligature|\marg{note}\arg{ligtype}\arg{glyph}\arg{glyph}
\end{decl}
Specifies a ligature of type \m{ligtype} from the current glyph
followed by the first glyph to the second glyph.  The \m{ligtype}s are
as in \vpl files (see the \texttt{vptovf} Web source for more
details).  For example:
\begin{verbatim}
  \setslot{ff}
    \ligature{LIG}{i}{ffi}
    \ligature{LIG}{l}{ffl}
    \comment{The `ff' ligature.}
  \endsetslot
\end{verbatim}

\begin{decl}
  |\makerightboundary|\arg{glyph}
\end{decl}

\begin{decl}
  |\Unicode|\marg{code point}\marg{name}
\end{decl}


\begin{decl}
  |\usedas|\arg{type}\arg{control sequence}
\end{decl}
\NEWfeature{Obsolete?!}
Sets the \TeX\ control sequence for this slot, with the \emph{type}
taken from:
\begin{verbatim}
  char       accent     mathord
  mathbin    mathrel    mathopen
  mathclose  mathpunct  mathvariable
  mathaccent mathdelim
\end{verbatim}

\begin{decl}
  |\nextlarger|\arg{glyph}
\end{decl}
Sets a \textsc{nextlarger} entry from the current slot to the
\m{glyph}.

\begin{decl}
  |\varchar|\\
  \m{varchar commands}\\
  |\endvarchar|
\end{decl}
Sets a \textsc{varchar} entry for the current slot, using the
\m{varchar commands}.  The \m{varchar commands} are:
\begin{decl}
  |\vartop|\arg{glyph}\\
  |\varmid|\arg{glyph}\\
  |\varbot|\arg{glyph}\\
  |\varrep|\arg{glyph}
\end{decl}
Sets the top, middle, bottom, or repeated \m{glyph} of the
\textsc{varchar}.


\subsection{Other}

The following commands belong to this section, but there is currently 
no description of them here.

|\endsetleftboundary|

|\ifdirect|

|\ifisinslot|\marg{glyph}\marg{slot}|\then|

|\resetslotcomment|\marg{text}

|\setfontdimen|\marg{fontdimen no.}\marg{integer variable}

|\setleftboundary|\marg{glyph}

|\setslotcomment|\marg{text}

|\useexamplefont|\marg{font}

|\unsetslotcomment|



\section{Metric files}
\label{Sec:Metric}


A \emph{metric file} (or \texttt{.mtx} file) is a \TeX\ document
consisting of:

\begin{decl}
  |\relax|\\
  \emph{ignored material}\\
  |\metrics|\\
  \m{metric commands}\\
  |\endmetrics|\\
  \emph{ignored material}
\end{decl}
This describes the glyphs in a font, using the \m{metric commands}.

Metric files are usually either \emph{hand-crafted} or 
\emph{transformable}. The transformable metric files typically encode 
the metrics of one particular font and are automatically generated. 
Hand-crafted metric files (such as \setfilename{latin.mtx}) typically 
do not contain much explicit metric data, instead the code there makes 
use of metrics previously specified by other files to construct new 
glyphs or adjust metrics to meet special conditions. Whereas 
transformable metric files tend to be mere lists of metric data, the 
hand-crafted metric files are more like programs.


\subsection{Metric commands}

The \m{metric commands} are:

\begin{decl}
  |\setglyph|\arg{name}\\
  \m{glyph commands} \\
  |\endsetglyph|
\end{decl}
If the glyph called \m{name} is undefined, it is built using the
\m{glyph commands} given below, for example:
\begin{verbatim}
  \setglyph{IJ}
     \glyph{I}{1000}
     \glyph{J}{1000}
  \endsetglyph
  \setglyph{Asmall}
     \glyph{A}{850}
  \endsetglyph
\end{verbatim}

\begin{decl}
  |\resetglyph|\arg{name}\\
  \m{glyph commands}\\
  |\endsetglyph|
\end{decl}
Gives the definition of the glyph called \m{name} using the
\m{glyph commands}.

\begin{decl}
  |\unsetglyph|\arg{name}
\end{decl}
Makes the glyph called \m{name} undefined.

\begin{decl}
  |\setrawglyph|\arg{name}\arg{font}\arg{dimen}\arg{integer}\\
    \arg{integer}\arg{integer}\arg{integer}\arg{integer}\\
  |\setscaledrawglyph|\arg{name}\arg{font}\arg{dimen}\arg{integer}\arg{integer}\\
    \arg{integer}\arg{integer}\arg{integer}\arg{integer}
\end{decl}
This sets a glyph called \m{name} from the \m{font}, which has the
given design size, slot, width, height, depth and italic correction.
If the integer variable \texttt{rawscale} is set, the glyph will be
scaled by that amount.  This command will usually be generated
automatically from an \afm or \pl file.

\begin{decl}
  |\setnotglyph|\arg{name}\arg{font}\arg{dimen}\\
    \arg{integer}\arg{integer}\arg{integer}\arg{integer}\\
  |\setscalednotglyph|\arg{name}\arg{font}\arg{dimen}\arg{integer}\arg{integer}\\
    \arg{integer}\arg{integer}\arg{integer}\arg{integer}
\end{decl}
This sets a glyph called \m{name}\texttt{-not}, which is present in
the \m{font}, but is not in the default encoding.  It takes the same
arguments as \verb|\setrawglyph|, although the slot will normally be
$-1$.  This command will usually be generated automatically from an
\afm file.

\begin{decl}
  |\setkern|\arg{glyph}\arg{glyph}\arg{integer expression}\\
  |\resetkern|\arg{glyph}\arg{glyph}\arg{integer expression}
\end{decl}
|\setkern| sets a kern between the two glyphs, scaled by the current 
value of \texttt{rawscale}, unless such a kern already has been set.


\begin{decl}
  |\setleftkerning|\arg{glyph}\arg{glyph}\arg{integer expression}\\
  |\setrightkerning|\arg{glyph}\arg{glyph}\arg{integer expression}
\end{decl}
Sets the amount by which the first glyph should mimic how the second
glyph kerns on the left or right, for example:
\begin{verbatim}
  \setleftkerning{Asmall}{A}{850}
  \setrightkerning{Asmall}{A}{850}
  \setleftkerning{IJ}{I}{1000}
  \setrightkerning{IJ}{J}{1000}
\end{verbatim}
Sets the amount by which the first glyph should mimic how the second
glyph kerns on the right, for example:

\begin{decl}
  |\setleftrightkerning|\arg{glyph}\arg{glyph}\arg{integer expression}
\end{decl}
\NEWfeature{v1.8}
Sets the amount by which the first glyph should mimic how the second
glyph kerns on both sides, for example:
\begin{verbatim}
  \setleftrightkerning{Asmall}{A}{850}
\end{verbatim}

\begin{decl}
  |\noleftkerning|\arg{glyph}\\
  |\norightkerning|\arg{glyph}\\
  |\noleftrightkerning|\arg{glyph}
\end{decl}
\NEWfeature{v1.9}
Removes all kerning on the specified side(s) of the \meta{glyph}.

\begin{decl}
  |\inputmtx|\arg{file}
\end{decl}
Inputs the \m{metric commands} of \m{file}\texttt{.mtx}.



\subsection{Glyph commands}

The \m{glyph commands} are:

\begin{decl}
  |\glyph|\arg{glyph}\arg{integer expression}
\end{decl}
Sets the named glyph \m{glyph} at the given scale, with 1000 as the
natural size.  This:
\begin{itemize}
\item Advances the current glyph width.
\item Sets the current glyph height to be at least the height of the
  named glyph, adjusted for the current vertical offset.
\item Sets the current glyph depth to be at least the depth of the
  named glyph, adjusted for the current vertical offset.
\item Sets the current glyph italic correction to be the same as the
  set glyph.
\end{itemize}
The named glyph must have already been defined, otherwise an error
will occur.  For example:
\begin{verbatim}
  \setglyph{fi}
    \glyph{f}{1000}
    \glyph{i}{1000}
  \endsetglyph
\end{verbatim}

\begin{decl}
  |\glyphrule|\arg{integer expression}\arg{integer expression}
\end{decl}
Sets a rule of the given width and height, for example:
\begin{verbatim}
  \setglyph{underline}
    \glyphrule{333}{40}
  \endsetglyph
\end{verbatim}

\begin{decl}
  |\glyphspecial|\arg{text}
\end{decl}
Sets a driver-dependent \verb|\special|, for example:
\begin{verbatim}
  \setglyph{crest}
    \glyphspecial{Filename: crest.eps}
  \endsetglyph
\end{verbatim}

\begin{decl}
  |\glyphwarning|\arg{text}
\end{decl}
Sets a warning \verb|\special|, and produces a warning message each
time the glyph is used, for example:
\begin{verbatim}
  \setglyph{missingglyph}
    \glyphrule{500}{500}
    \glyphwarning{Missing glyph `missingglyph'}
  \endsetglyph
\end{verbatim}

\begin{decl}
  |\movert|\arg{integer expression}
\end{decl}
Moves right by the given amount, and advances the current glyph width,
for example:
\begin{verbatim}
  \setglyph{Asmall}
    \movert{50}
    \glyph{A}{700}
    \movert{50}
  \endsetglyph
\end{verbatim}

\begin{decl}
  |\moveup|\arg{integer expression}
\end{decl}
Moves up by the given amount, and advances the current vertical
offset.  Each glyph should always end at vertical offset zero, for
example:
\begin{verbatim}
  \setglyph{onehalf}
    \moveup{500}
    \glyph{one}{700}
    \moveup{-500}
    \glyph{slash}{1000}
    \moveup{-200}
    \glyph{two}{700}
    \moveup{200}
  \endsetglyph
\end{verbatim}

\begin{decl}
  |\push|\\
  \m{glyph commands}\\
  |\pop|
\end{decl}
Performs the \m{glyph commands} without adjusting the current position
or glyph width, for example:
\begin{verbatim}
  \setglyph{aacute}
    \push
      \movert{\div{\sub{\width{a}}{\width{acute}}}{2}}
      \glyph{acute}{1000}
    \pop
    \glyph{a}{1000}
  \endsetglyph
\end{verbatim}

\begin{decl}
  |\glyphpcc|\arg{glyph}\arg{integer expression}\arg{integer expression}
\end{decl}
This is generated from \texttt{PCC} instructions in an \afm file, and
is syntactic sugar for:
\begin{quote}
\begin{small}
|\push|\\
|\movert|\arg{first integer expression}\\
|\moveup|\arg{second integer expression}\\
|\glyph|\arg{glyph}|{1000}|\\
|\pop|
\end{small}
\end{quote}

\begin{decl}
  |\resetwidth|\arg{integer expression}\\
  |\resetheight|\arg{integer expression}\\
  |\resetdepth|\arg{integer expression}\\
  |\resetitalic|\arg{integer expression}
\end{decl}
Sets the width, height, depth, or italic correction of the current
glyph.

\begin{decl}
  |\samesize|\arg{glyph}
\end{decl}
Sets the dimensions of the current glyph to be the same as \m{glyph}.

Inside the definition of \m{glyph}, you can use expressions such as
|\width|\arg{glyph}, which will refer to the glyph defined so far.
For example, a display summation sign can be defined to be a text
summation $\sum$ scaled 120\% with 0.5\,pt extra height and depth
using:
\begin{verbatim}
  \setglyph{summationdisplay}
    \glyph{summationtext}{1200}
    \resetheight{\add{\height{summationdisplay}}{50}}
    \resetdepth{\add{\depth{summationdisplay}}{50}}
  \endsetglyph
\end{verbatim}
Within a |\resetglyph|, these expressions will refer to the previous
definition of the glyph.  For example, you can add sidebearings to the
letter `A' with:
\begin{verbatim}
  \resetglyph{A}
    \movert{25}
    \glyph{A}{1000}
    \movert{25}
  \endresetglyph
\end{verbatim}


\subsection{Other}

The following commands belong to this section, but there is currently 
no description of them here.

|\aliased|\marg{font's name}\marg{alias name}

|\ProvidesMtxPackage|\marg{package~name}

|\unsetkerns|\marg{left~glyph~list}\marg{right~glyph~list}

|\usemtxpackage|\marg{package~list}



\section{\textsf{fontdoc} commands}

|\macroparameter|\marg{digit}

|\plaindiv|

|\plainint|

|\plainmax|

|\plainmin|

|\showbranches|

|\slotexample|

|\textunicode|\marg{code point}\marg{name}



\section{\fontinst variables}

The following is a list of the \fontinst variables that are 
accessible for the user through the |\set|\textellipsis, 
|\reset|\textellipsis, |\unset|\textellipsis, etc.\ commands. You may 
of course set or use other variables in the MTX and ETX files you 
write yourself, as does for example the standard MTX file 
\texttt{latin.mtx}, but all variables that \fontinst commands 
implicitly use or set are listed below.

\begin{list}{}{%
   \setlength\labelwidth{0pt}%
   \setlength\itemindent{-\leftmargin}%
   \setlength\parsep{0pt}
   \def\makelabel#1{\hspace\labelsep \normalfont\ttfamily #1}%
}
  \item[acccapheight] (integer denoting length)
    \begin{smalldes}
      \item[Description] The height of accented full capitals.
      \item[Set by] MTX files.
      \item[Used by] Some ETX and MTX files.
    \end{smalldes}
  \item[address] (string)
    \begin{smalldes}
      \item[Description] Snailmail address put in \BibTeX-style file 
        header of automatically generated ENC files. No 
        \texttt{address} field is written unless the \texttt{address} 
        string is set. Quotes are not automatically inserted around 
        the \texttt{address} string.
      \item[Set by] ETX files.
      \item[Used by] The ETX-to-ENC converter.
    \end{smalldes}
  \item[afm-name] (string)
    \begin{smalldes}
      \item[Description] Name of source font. Internal variable.
      \item[Set by] |\from|\dots\ commands.
      \item[Used by] The |\transform|\-|font|, |\install|\-|font|, 
        |\install|\-|raw|\-|font|, and |\reglyph|\-|font| commands.
    \end{smalldes}
  \item[ascender] (integer denoting length)
    \begin{smalldes}
      \item[Description] The ascender height of the font.
      \item[Set by] MTX files. The AFM-to-MTX converter usually writes 
        |\setint| commands for this integer.
      \item[Used by] Some MTX and ETX files.
    \end{smalldes}
  \item[author] (string)
    \begin{smalldes}
      \item[Description] Author name(s) put in \BibTeX-style file 
        header of automatically generated ENC files. See the macro 
        |\ref_to_sourcefile| for more details.
      \item[Set by] ETX files.
      \item[Used by] The ETX-to-ENC converter. When not set, the 
        value \texttt{"See file }\meta{etx name}\texttt{"} is used 
        instead.
    \end{smalldes}
  \item[\cs{autoinstallfamily}] (command)
    \begin{smalldes}
      \item[Description] Command called by the font installation 
        commands, as
        \begin{quote}
          |\autoinstallfamily|\marg{encoding}\marg{family}
        \end{quote}
        when they are asked to install a font with a combination of 
        \meta{encoding} and \meta{family} that has not been seen 
        before (there was no explicit |\installfamily|).
      \item[Set by] Explicit commands. Defaults to calling 
        |\installfamily|.
      \item[Used by] Font installation commands.
    \end{smalldes}
  \item[axisheight] (integer denoting length)
    \begin{smalldes}
      \item[Description] Math formula parameter $\sigma\sb{22}$.
      \item[Set by] MTX files.
      \item[Used by] Some ETX and MTX files.
    \end{smalldes}
  \item[baselineskip] (integer denoting length)
    \begin{smalldes}
      \item[Description]
        The font designer's recommendation for natural length of the 
        \TeX\ parameter |\baselineskip|.
      \item[Set by] MTX files.
      \item[Used by] Some ETX and MTX files.
    \end{smalldes}
  \item[bigopspacing1] (integer denoting length)
    \begin{smalldes}
      \item[Description] Math formula parameter $\xi\sb{9}$.
      \item[Set by] MTX files.
      \item[Used by] Some ETX and MTX files.
    \end{smalldes}
  \item[bigopspacing2] (integer denoting length)
    \begin{smalldes}
      \item[Description] Math formula parameter $\xi\sb{10}$.
      \item[Set by] MTX files.
      \item[Used by] Some ETX and MTX files.
    \end{smalldes}
  \item[bigopspacing3] (integer denoting length)
    \begin{smalldes}
      \item[Description] Math formula parameter $\xi\sb{11}$.
      \item[Set by] MTX files.
      \item[Used by] Some ETX and MTX files.
    \end{smalldes}
  \item[bigopspacing4] (integer denoting length)
    \begin{smalldes}
      \item[Description] Math formula parameter $\xi\sb{12}$.
      \item[Set by] MTX files.
      \item[Used by] Some ETX and MTX files.
    \end{smalldes}
  \item[bigopspacing5] (integer denoting length)
    \begin{smalldes}
      \item[Description] Math formula parameter $\xi\sb{13}$.
      \item[Set by] MTX files.
      \item[Used by] Some ETX and MTX files.
    \end{smalldes}
  \item[capheight] (integer denoting length)
    \begin{smalldes}
      \item[Description] The height of the font's full capitals.
      \item[Set by] MTX files. The AFM-to-MTX converter usually writes 
        |\setint| commands for this variable.
      \item[Used by] Some MTX and ETX files.
    \end{smalldes}
  \item[codingscheme] (string)
    \begin{smalldes}
      \item[Description] The codingscheme name.
      \item[Set by] ETX files.
      \item[Used by] The (V)PL writer. When not set, the 
        value \texttt{UNKNOWN} is used instead.
    \end{smalldes}
  \item[defaultrulethickness] (integer denoting length)
    \begin{smalldes}
      \item[Description] Math formula parameter $\xi\sb{8}$.
      \item[Set by] MTX files.
      \item[Used by] Some ETX and MTX files.
    \end{smalldes}
  \item[delim1] (integer denoting length)
    \begin{smalldes}
      \item[Description] Math formula parameter $\sigma\sb{20}$.
      \item[Set by] MTX files.
      \item[Used by] Some ETX and MTX files.
    \end{smalldes}
  \item[delim2] (integer denoting length)
    \begin{smalldes}
      \item[Description] Math formula parameter $\sigma\sb{21}$.
      \item[Set by] MTX files.
      \item[Used by] Some ETX and MTX files.
    \end{smalldes}
  \item[denom1] (integer denoting length)
    \begin{smalldes}
      \item[Description] Math formula parameter $\sigma\sb{11}$.
      \item[Set by] MTX files.
      \item[Used by] Some ETX and MTX files.
    \end{smalldes}
  \item[denom2] (integer denoting length)
    \begin{smalldes}
      \item[Description] Math formula parameter $\sigma\sb{12}$.
      \item[Set by] MTX files.
      \item[Used by] Some ETX and MTX files.
    \end{smalldes}
  \item[descender] (integer denoting length)
    \begin{smalldes}
      \item[Description] The depth of lower case letters with descenders.
      \item[Set by] MTX files.
      \item[Used by] Some ETX and MTX files.
    \end{smalldes}
  \item[descender\_neg] (integer denoting length)
    \begin{smalldes}
      \item[Description] The vertical position of the descender line 
        of the font, i.e., the negative of the font's descender depth.
      \item[Set by] MTX files. The AFM-to-MTX converter usually writes 
        |\setint| commands for this variable.
      \item[Used by] Some MTX and ETX files.
    \end{smalldes}
  \item[designsize] (dimension)
    \begin{smalldes}
      \item[Description] The design size of the font.
      \item[Set by] MTX files. The (V)PL-to-MTX converter usually writes 
        |\setdim| commands for this variable.
      \item[Used by] The (V)PL writer. The design size defaults to 
        $10\,\mathrm{pt}$ if this variable is not set.
      \item[Note] The value of this variable has no effect on how 
        the font is declared to \LaTeX.
    \end{smalldes}
  \item[designunits] (dimension denoting a real number)
    \begin{smalldes}
      \item[Description] The design size of a font expressed in the 
        design unit used in a (V)PL file. 
      \item[Set by] MTX files. The (V)PL-to-MTX converter usually writes 
        |\setdim| commands for this variable.
      \item[Used by] Nothing. If this variable is set, but to any 
        value other than $1\,\mathrm{pt}$, then some metrics are most 
        likely wrong.
    \end{smalldes}
  \item[digitwidth] (integer denoting length)
    \begin{smalldes}
      \item[Description] The median width of the digits in the font.
      \item[Set by] MTX files.
      \item[Used by] Some ETX and MTX files.
    \end{smalldes}
  \item[email] (string)
    \begin{smalldes}
      \item[Description] Email address put in \BibTeX-style file 
        header of automatically generated ENC files. See the macro 
        |\ref_to_sourcefile| for more details.
      \item[Set by] ETX files.
      \item[Used by] The ETX-to-ENC converter. When not set, the 
        value \texttt{"See file }\meta{etx name}\texttt{"} is used 
        instead.
    \end{smalldes}
  \item[encodingname] (string)
    \begin{smalldes}
      \item[Description] The name by which the encoding in question is 
        made known to a Postscript interpreter.
      \item[Set by] ETX files.
      \item[Used by] The ETX-to-ENC converter. When not set, the 
        value |fontinst-|\nolinebreak[1]|autoenc-|\nolinebreak[1]%
        \meta{etx name} is used instead.
    \end{smalldes}
  \item[etx-name] (string)
    \begin{smalldes}
      \item[Description] Name of ETX file. Internal variable in 
        |\transform|\-|font|.
      \item[Set by] The |\reencodefont| command.
      \item[Used by] The |\mtxtomtx| command.
    \end{smalldes}
  \item[extraspace] (integer denoting length)
    \begin{smalldes}
      \item[Description]
        The natural width of extra interword glue at the end of a 
        sentence.
      \item[Set by] MTX files.
      \item[Used by] Some ETX and MTX files.
    \end{smalldes}
  \item[fontdimen($n$)] (integer)
    \begin{smalldes}
      \item[Description] Family of semi-internal variables that store 
         the values to use for font dimension $n$. It is preferred 
         that the newer |\set|\-|font|\-|dimen| interface is used for 
         setting these values.
      \item[Set by] ETX files.
      \item[Used by] The (V)PL writer.
    \end{smalldes}
  \item[\cs{iftokeep}] (macro)
    \begin{smalldes}
      \item[Description] |\iftokeep|\,\#1\,|\then|, where \#1 will 
        be a \meta{number}, behaves like a switch and decides whether 
        a glyph is kept or not while reglyphing.
      \item[Set by] Explicit commands. Defaults to
        $$
          \mbox{\cs{iftokeep}\,\#1\,\cs{then}} \mapsto
          \mbox{\cs{ifnum}\,\texttt{-1<}\#1}
        $$
      \item[Used by] The |\reglyphfont| command.
    \end{smalldes}
  \item[interword] (integer denoting length)
    \begin{smalldes}
      \item[Description] The natural width of interword glue (spaces).
      \item[Set by] MTX files.
      \item[Used by] Some ETX and MTX files.
    \end{smalldes}
  \item[italicslant] (integer denoting factor)
    \begin{smalldes}
      \item[Description] The italic slant of a font.
      \item[Set by] MTX files generated from AFM or (V)PL files. MTX 
        files generated by |\transformfont|. Locally in the AFM-to-MTX 
        converter for possible use in |\uprightitalcorr| or 
        |\slanteditalcorr|.
      \item[Used by] MTX files (\texttt{latin.mtx} and the like). ETX 
        files (for determining \texttt{fontdimen(1)}).
    \end{smalldes}
  \item[killweight] (integer)
    \begin{smalldes}
      \item[Description] Weight for glyphs that are killed.
      \item[Set by] Explicit commands. Defaults to $-10$ if not set.
      \item[Used by] The |\kill|\-|glyph| command; indirectly 
        the |\reglyphfont| command.
    \end{smalldes}
  \item[letterspacing] (integer denoting length)
    \begin{smalldes}
      \item[Description] Extra width added to all glyphs of a font.
      \item[Set by] ETX or MTX files.
      \item[Used by] The (V)PL writer. Defaults to $0$ if not set.
    \end{smalldes}
  \item[maxdepth] (integer denoting length)
    \begin{smalldes}
      \item[Description] The maximal depth over all slots in the font.
      \item[Set by] MTX files.
      \item[Used by] Some ETX and MTX files.
    \end{smalldes}
  \item[maxdepth\_neg] (integer denoting length)
    \begin{smalldes}
      \item[Description] The negative of the maximal depth of a glyph in 
        the font.
      \item[Set by] MTX files. The AFM-to-MTX converter usually writes 
        |\setint| commands for this variable.
      \item[Used by] Some ETX and MTX files.
    \end{smalldes}
  \item[maxheight] (integer denoting length)
    \begin{smalldes}
      \item[Description] The maximal height of a glyph in the font.
      \item[Set by] MTX files. The AFM-to-MTX converter usually writes 
        |\setint| commands for this variable.
      \item[Used by] Some ETX and MTX files.
    \end{smalldes}
  \item[minimumkern] (integer denoting length)
    \begin{smalldes}
      \item[Description] Kerns whose size in absolute value is less 
        than or equal to this variable are ignored.
      \item[Set by] Command files or MTX files.
      \item[Used by] The AFM-to-MTX converter and the (V)PL file 
        generator. When not set, the value $0$ is used instead.
    \end{smalldes}
  \item[monowidth] (flag integer)
    \begin{smalldes}
      \item[Description] Set if this font is monowidth, unset otherwise.
      \item[Set by] MTX files. The AFM-to-MTX converter writes a 
        |\setint| command for this variable if the AFM specifies 
        \texttt{IsFixedPitch true}.
      \item[Used by] Some MTX files (\texttt{latin.mtx} and the like), 
         ETX files.
    \end{smalldes}
  \item[num1] (integer denoting length)
    \begin{smalldes}
      \item[Description] Math formula parameter $\sigma\sb{8}$.
      \item[Set by] MTX files.
      \item[Used by] Some ETX and MTX files.
    \end{smalldes}
  \item[num2] (integer denoting length)
    \begin{smalldes}
      \item[Description] Math formula parameter $\sigma\sb{9}$.
      \item[Set by] MTX files.
      \item[Used by] Some ETX and MTX files.
    \end{smalldes}
  \item[num3] (integer denoting length)
    \begin{smalldes}
      \item[Description] Math formula parameter $\sigma\sb{10}$.
      \item[Set by] MTX files.
      \item[Used by] Some ETX and MTX files.
    \end{smalldes}
  \item[quad] (integer denoting length)
    \begin{smalldes}
      \item[Description]
        The quad width of the font, normally approximately equal to 
        the font size and\slash or the width of an `M'.
      \item[Set by] MTX files.
      \item[Used by] Some ETX and MTX files.
    \end{smalldes}
  \item[PSfontsuffix] (string)
    \begin{smalldes}
      \item[Description] Suffix added to font names to form name of 
        file to download to include font.
      \item[Set by] Explicit commands in mapmaking command 
        files. Defaults to `\texttt{.pfa}'.
      \item[Used by] The map file fragments writer.
    \end{smalldes}
  \item[rawscale] (integer denoting factor)
    \begin{smalldes}
      \item[Description] Scaling factor applied to raw glyphs.
      \item[Set by] The |\installfont| command (\texttt{scaled} 
        clauses in argument \#2). Unset for metric files listed 
        without a \texttt{scaled} clause.
      \item[Used by] The |\set|\-|raw|\-|glyph|, 
        |\set|\-|not|\-|glyph|, |\set|\-|scaled|\-|raw|\-|glyph|, 
        |\set|\-|scaled|\-|not|\-|glyph|, |\set|\-|kern|, and 
        |\reset|\-|kern| commands.
    \end{smalldes}
  \item[renameweight] (integer)
    \begin{smalldes}
      \item[Description] Weight for glyphs that are renamed.
      \item[Set by] Explicit commands. Defaults to $1$ if not set.
      \item[Used by] The |\rename|\-|glyph| command; indirectly 
        the |\reglyphfont| command.
    \end{smalldes}
  \item[requireglyphs] (flag integer)
    \begin{smalldes}
      \item[Description] Set if warnings are to be generated for 
        glyphs listed in ETX files but not present in the glyph 
        base.
      \item[Set by] Explicit commands. By default not set.
      \item[Used by] The (V)PL file generator.
    \end{smalldes}
  \item[rightboundary] (string)
    \begin{smalldes}
      \item[Description] The name of a glyph with the property that 
        kerns on the left may be intended as right word boundary kerns.
      \item[Set by] MTX files. The (V)PL-to-MTX converter can write 
        |\setstr| commands for this variable.
      \item[Used by] Some MTX files.
    \end{smalldes}
  \item[shrinkword] (integer denoting length)
    \begin{smalldes}
      \item[Description] 
        The (finite) shrink component of interword glue.
      \item[Set by] MTX files.
      \item[Used by] Some ETX and MTX files.
    \end{smalldes}
  \item[slant-scale] (integer denoting factor)
    \begin{smalldes}
      \item[Description] Factor to slant by. 
        Internal variable in |\transform|\-|font|.
      \item[Set by] The |\slant|\-|font|, |\xscale|\-|font|, and
        |\scale|\-|font| commands.
      \item[Used by] The |\mtxtomtx| command.
    \end{smalldes}
  \item[\cs{SlantAmount}] (macro expanding to an integer expression)
    \begin{smalldes}
      \item[Description] Slant factor used for faking oblique shape.
      \item[Set by] Explicit commands. Defaults to \texttt{167}.
      \item[Used by] The |\latinfamily| command.
    \end{smalldes}
  \item[\cs{slanteditalcorr}]
    (macro expanding to an integer expression)
    \begin{smalldes}
      \item[Description] The integer expression used to calculate a 
        guess for the italic correction of glyphs in a font with 
        positive slant. It has the syntax
        \begin{quote}
          \cs{slanteditalcorr}\marg{width}\marg{left}\marg{right}%
          \marg{bottom}\marg{top}
        \end{quote}
        where \meta{width} is the glyph's advance width, and the 
        remaining arguments are coordinates of sides of the glyph's 
        bounding box. The \texttt{italicslant} integer provides the 
        italic slant of the font.
      \item[Set by] Explicit commands in \textsf{fontinst} command 
        files. Defaults to 
        $$
          \max\{0, \mathit{right}-\mathit{width}\}.
        $$
      \item[Used by] The AFM-to-MTX converter.
    \end{smalldes}
  \item[stretchword] (integer denoting length)
    \begin{smalldes}
      \item[Description]
        The (finite) stretch component of interword glue.
      \item[Set by] MTX files.
      \item[Used by] Some ETX and MTX files.
    \end{smalldes}
  \item[sub1] (integer denoting length)
    \begin{smalldes}
      \item[Description] Math formula parameter $\sigma\sb{16}$.
      \item[Set by] MTX files.
      \item[Used by] Some ETX and MTX files.
    \end{smalldes}
  \item[sub2] (integer denoting length)
    \begin{smalldes}
      \item[Description] Math formula parameter $\sigma\sb{17}$.
      \item[Set by] MTX files.
      \item[Used by] Some ETX and MTX files.
    \end{smalldes}
  \item[subdrop] (integer denoting length)
    \begin{smalldes}
      \item[Description] Math formula parameter $\sigma\sb{19}$.
      \item[Set by] MTX files.
      \item[Used by] Some ETX and MTX files.
    \end{smalldes}
  \item[sup1] (integer denoting length)
    \begin{smalldes}
      \item[Description] Math formula parameter $\sigma\sb{13}$.
      \item[Set by] MTX files.
      \item[Used by] Some ETX and MTX files.
    \end{smalldes}
  \item[sup2] (integer denoting length)
    \begin{smalldes}
      \item[Description] Math formula parameter $\sigma\sb{14}$.
      \item[Set by] MTX files.
      \item[Used by] Some ETX and MTX files.
    \end{smalldes}
  \item[sup3] (integer denoting length)
    \begin{smalldes}
      \item[Description] Math formula parameter $\sigma\sb{15}$.
      \item[Set by] MTX files.
      \item[Used by] Some ETX and MTX files.
    \end{smalldes}
  \item[supdrop] (integer denoting length)
    \begin{smalldes}
      \item[Description] Math formula parameter $\sigma\sb{18}$.
      \item[Set by] MTX files.
      \item[Used by] Some ETX and MTX files.
    \end{smalldes}
  \item[TFMfileprefix] (string)
    \begin{smalldes}
      \item[Description] Prefix (typically a path) added to names of 
        TFM files.
      \item[Set by] Explicit commands in mapmaking command 
        files. By default not set, which is equivalent to being empty.
      \item[Used by] The \textsf{PLtoTF} ``map file fragments writer''.
    \end{smalldes}
  \item[underlinethickness] (integer denoting length)
    \begin{smalldes}
      \item[Description] The recommended thickness of an underlining 
        rule.
      \item[Set by] MTX files. The AFM-to-MTX converter usually writes 
        |\setint| commands for this variable.
      \item[Used by] Some MTX files (\texttt{latin.mtx} and the like).
    \end{smalldes}
  \item[\cs{uprightitalcorr}]
    (macro expanding to an integer expression)
    \begin{smalldes}
      \item[Description] The integer expression used to calculate a 
        guess for the italic correction of glyphs in a font with 
        non-positive slant. It has the syntax
        \begin{quote}
          \cs{uprightitalcorr}\marg{width}\marg{left}\marg{right}%
          \marg{bottom}\marg{top}
        \end{quote}
        where \meta{width} is the glyph's advance width, and the 
        remaining arguments are coordinates of sides of the glyph's 
        bounding box. The \texttt{italicslant} integer provides the 
        italic slant of the font.
      \item[Set by] Explicit commands in \textsf{fontinst} command 
        files. Defaults to $0$.
      \item[Used by] The AFM-to-MTX converter.
    \end{smalldes}
  \item[version] (string)
    \begin{smalldes}
      \item[Description] Version number put in \BibTeX-style file 
        header of automatically generated ENC files. See the macro 
        |\ref_to_sourcefile| for more details.
      \item[Set by] ETX files.
      \item[Used by] The ETX-to-ENC converter. When not set, the 
        value \texttt{"See file }\meta{etx name}\texttt{"} is used 
        instead.
    \end{smalldes}
  \item[verticalstem] (integer denoting length)
    \begin{smalldes}
      \item[Description] The dominant width of vertical stems 
        (usually the width of stems of lower case letters).
      \item[Set by] MTX files. The AFM-to-MTX converter writes 
        |\setint| commands for this variable if the AFM file specifies 
        \texttt{StdVW}.
      \item[Used by] Currently nothing.
    \end{smalldes}
  \item[\texttt{warningspecials}] (switch)
    \begin{smalldes}
      \item[Description] Controls whether |\glyphwarning| commands 
        will generate VPL \texttt{SPECIAL}s. Defaults to `true'.
      \item[Set by] Explicit commands (|\warningspecialstrue| and 
        |\warningspecialsfalse|).
      \item[Used by] The (V)PL file generator.
    \end{smalldes}
  \item[x-scale] (integer denoting factor)
    \begin{smalldes}
      \item[Description] Horizontal scaling factor. 
        Internal variable in |\transform|\-|font|.
      \item[Set by] The |\xscale|\-|font| and |\scale|\-|font| commands.
      \item[Used by] The |\mtxtomtx| command.
    \end{smalldes}
  \item[xheight] (integer denoting length)
    \begin{smalldes}
      \item[Description] The x-height of the font.
      \item[Set by] MTX files. The AFM-to-MTX and (V)PL-to-MTX 
        converters usually write |\setint| commands for this variable.
      \item[Used by] MTX files, and ETX files (for determining 
        \texttt{fontdimen(5)}).
    \end{smalldes}
  \item[y-scale] (integer denoting factor)
    \begin{smalldes}
      \item[Description] Vertical scaling factor. 
        Internal variable in |\transform|\-|font|.
      \item[Set by] The |\yscale|\-|font| and |\scale|\-|font| commands.
      \item[Used by] The |\mtxtomtx| command.
    \end{smalldes}
  \item[\meta{\rmfamily glyph}-spacing] (integer denoting length)
    \begin{smalldes}
      \item[Description] Glyph-specific override for 
        \texttt{letterspacing}; extra width added to the glyph 
        \meta{glyph} as part of the process of writing a VPL 
        file.
      \item[Set by] ETX or MTX files.
      \item[Used by] The (V)PL writer. Defaults to $0$ if not set.
    \end{smalldes}
\end{list}
Besides these, the |\latinfamily| command provides a whole range of 
parameters (|\latin_weights|, |\latin_widths|, |\latin_shapes|, etc.) 
that are often used somewhat like variables. That subject 
does however deserve to be treated separately.



\section{Customisation}

The \fontinst package reads a file \texttt{fontinst.rc} if it
exists.  This can contain your own customisations.

You can create a \texttt{fontinst} format by running ini\TeX{} on
\texttt{fontinst.sty} then saying \verb|\dump|.


\section{Notes on features new with v\,1.9}

The following notes are copied from \texttt{fisource.tex}; they were 
written to explain new \fontinst features to old \fontinst users.


\subsection{Metric packages}

\Fontinst has traditionally come with a collection of MTX 
files that complement the MTX files generated from base font metrics, 
in that they build glyphs that may be missing from the base fonts or 
in some other way needs to be improved. The most well-known of these 
is the \texttt{latin.mtx} file; other examples include 
\texttt{textcomp.mtx}, \texttt{mathit.mtx}, and \texttt{latinsc.mtx}. 
A problem with these is however that they cannot produce optimal 
results for all fonts simply because there are irregular differences 
in how fonts are set up by the foundries. Most glyphs come out all right, 
but there are usually a few for which the parameters used are more or 
less wrong. Therefore most high quality font installations are made 
with modified versions of these files, where the parameters have been 
tuned to the specific font design.

Modifying in particular \texttt{latin.mtx} is however not an entirely 
easy task, because this is a rather large file (with plenty of 
archaic pieces of code in curious places). Doing it once is no big 
problem, but if one has to do it several times (maybe because some 
errors are discovered in the original \texttt{latin.mtx}) then it is 
probably no fun anymore. Furthermore, if one has two or three 
modified copies of this file because one has made high quality 
installations of that many different fonts then even a trivial bugfix 
might start to feel like entirely too much work.

If one has to make modifications then it is usually easier to deal 
with several small files (many of which can be used unchanged) than 
one big file. Thus it would be better if these big files were split up 
into several smaller ones.
The main problem with splitting up something like \texttt{latin.mtx}
is that there are some commands which are defined at the top and 
which are then used in almost all sections of the file. One must make 
certain that these commands are always loaded, which makes the metric 
files somewhat harder to use (especially if the one who tries to use 
them is not the one who wrote them).

One strategy is to include all definitions needed for a metric file in 
it. This has the slight disadvantage that the commands will have to be 
defined several times. What is worse however, is that the command 
definitions will appear in several files, so if one finds a bug in one 
of them, one cannot simply correct this bug in one place. As the number 
of files can soon become quite large, correcting such bugs can become 
a boring procedure indeed.

Another strategy is to put all the command definitions in one file 
and then explicitly include it in the \meta{file-list} argument of 
|\installfont|. This eliminates the repeated bug fixing problem, but 
requires the user to do something that the computer can actually do 
just as well.

A third strategy is to put the command definitions in one or several 
files and then in each metric file the user explicitly mentions load 
the command definitions needed for that particular file. Metric 
packages uses an improved version of this strategy, since they also 
make it possible for \fontinst to remember which packages (i.e., sets 
of command definitions) that have already been loaded, so that they 
are not unnecessarily loaded again. The \texttt{newlatin.mtx} file is 
an alternative to \texttt{latin.mtx} that implements this strategy. 
Most of the actual code is located in the following metric packages:
\begin{center}
\begin{tabular}{l p{0.7\linewidth}}
  \texttt{ltcmds.mtx}& Defines some common commands used by the other 
    files.\\
  \texttt{llbuild.mtx}& Builds the latin lower case alphabet 
    (unaccented letters are `unfakable', the rest are constructed if 
    not present in the base fonts).\\
  \texttt{lubuild.mtx}& Builds the latin upper case alphabet.\\
  \texttt{lsbuild.mtx}& Builds accented letters in the latin 
    smallcaps alphabet, but only if there are unaccented letters to 
    build them from in the base fonts.\\
  \texttt{lsfake.mtx}& Fakes a latin smallcaps alphabet by shrinking 
    the upper case alphabet, but only if the glyph had not already 
    been manufactured.\\
  \texttt{lsmisc.mtx}& Make some miscellaneous smallcaps glyphs 
    (mostly ``smallcaps f-ligatures'').\\
  \texttt{ltpunct.mtx}& Makes digits, punctuation marks, and other 
    symbols (mostly by marking as ``unfakable'').
\end{tabular}
\end{center}
\noindent All of these are easy to use as components of equivalents 
of a modified \texttt{latin.mtx} files, and all dependencies of one 
package upon another are handled via explicit |\usemtxpackage| 
commands.

% For information on the syntax etcetera of commands related to metric 
% packages, see Section~\ref{Sec:Metric files}.


\subsection{Word boundary ligatures and kerns}

One of the new features added in \TeX~3 was that of ligatures and 
kerns with word boundaries. \Fontinst has had an interface 
for making such ligatures and kerns, but it has been completely 
redesigned in v\,1.9 and the old interface (setting the integer 
|boundarychar|) is no longer recognized by \fontinst. Files 
which use the old interface can still be processed with 
\texttt{cfntinst.sty}, though.

Before considering the new commands, it is suitable to make a 
distinction between proper glyphs and pseudoglyphs. A proper glyph has 
been set using one of the commands |\setrawglyph|, |\setglyph|, and 
|\resetglyph|. A pseudoglyph is any name used in the context of a 
glyph name which does not denote a proper glyph. If a pseudoglyph 
|g-not| was set using the |\setnotglyph| command, then 
|\ifisglyph{g-not}\then| will evaluate to true, but something can be 
a pseudoglyph even if an |\ifisglyph| test evaluates to false. The 
interesting point about pseudoglyphs when considering word boundaries 
however, is that a pseudoglyph can have ligatures and kerns.

Kerns and ligatures at the left word boundary (beginning of word) are 
specified using the commands |\setleftboundary| and 
|\endsetleftboundary|, which are syntactically identical to 
|\setslot| and |\endsetslot| respectively. One important difference is 
however that the argument to |\setslot| must be a proper glyph, while 
the argument to |\setleftboundary| may be any glyph, hence any 
pseudoglyph will do just fine.

|\ligature| commands between |\setleftboundary| and 
|\endsetleftboundary| will generate beginning of word ligatures. Kerns 
on the right of the glyph specified in |\setleftboundary| will become 
beginning of word kerns.

Kerns and ligatures at the right word boundary (end of word) are 
trickier, due to the asymmetrical nature of the ligkern table in a PL 
file. What a font can do is to specify that the right word boundary, 
for purposes of kerning and ligatures, should be interpreted as 
character $n$. By including a kern or ligature with character $n$ on 
the right, that kern or ligature will be used at the end of a word, 
but it will also be used each time the next character is character 
$n$. Because of this, one usually wants the slot $n$, which the right 
word boundary is interpreted as being, to be empty whenever the 
encoding allows this.

The command
\begin{quote}
  |\setrightboundary|\marg{glyph}
\end{quote}
will mark the current slot as used to denote the right word boundary, 
and leave the slot empty, increasing the current slot number by one 
just like a |\setslot| \textellipsis\ |\endsetslot| block does. Kerns on 
the left of \meta{glyph} will be end of word kerns and |\ligature| 
commands with \meta{glyph} as the second argument will be for the end 
of a word.

The command
\begin{quote}
  |\makerightboundary|\marg{glyph}
\end{quote}
is similar to |\setrightboundary|, but it is a slot command which may 
only be used between a |\setslot| and the matching |\endsetslot|. Like 
|\setrightboundary|, it marks the current slot as used to denote the 
right word boundary, but the glyph specified in the enclosing |\setslot| 
will be written to that slot. Ligatures for the glyph specified by the 
|\setslot| and ligatures for the glyph specified by the 
|\makerightboundary| will both be for this single slot. Kerns on the 
right of the |\setslot| glyph and the |\makerightboundary| glyph will 
similarly both be for this single slot. The idea is that the |\setslot| 
glyph should be used when making a kern or ligature for that glyph, 
while the |\makerightboundary| glyph should be used when making a kern 
or ligature for the end of a word. \Fontinst will warn you if 
these two uses of the slot directly contradict each other.


\subsection{Changing the names of glyphs}
\label{Ssec:Des:Reglyph}

Sometimes, primarily when making a virtual font from more than one raw 
font and two of the raw fonts contain different glyphs with the same 
name, it becomes necessary to change the names of some glyphs to make 
some sense out of it. The main source of this kind of trouble is the 
``caps and small caps'' (SC) and ``oldstyle figures'' (OsF) fonts 
within many commercial font families. The typical problem is that what 
is typographically different glyphs---such as the lowercase `a' 
(\texttt{a}, for \fontinst) and the smallcaps `\textsc{a}' 
(\texttt{Asmall}, for \fontinst)---are given the same name by 
the foundry. 

One way to get round this is to say for example
\begin{quote}
  |\setglyph{Asmall} \glyph{a}{1000} \endsetglyph|\\
  |\setleftrightkerning{Asmall}{a}{1000}|\\
  |\unsetglyph{a}|\\
  |\noleftrightkerning{a}|
\end{quote}
and continuing like that for all the duplicate glyph names. This is 
however a rather prolix method and if the number of glyphs is large 
then it is usually simpler to use the |\reglyphfont| command.

To reglyph one or several fonts, one writes
\begin{quote}
  |\reglyphfonts|\\
  \vadjust{}\quad \meta{reglyphing commands}\\
  |\endreglyphfonts|
\end{quote}
There are two types of reglyphing commands: the |\reglyphfont| 
command, and the commands that modify what |\reglyphfont| will do to 
the fonts it operates on. The syntax of |\reglyphfont| is
\begin{quote}
  |\reglyphfont|\marg{destination font}\marg{source font}
\end{quote}
The \meta{source font} font here is the name (suffix not included, of 
course) of the font metric file one wants to change the glyph names in. 
This font metric file can be in any of the formats MTX, PL, AFM, and 
VPL, and it will be converted to MTX format if it isn't already in 
that format (this happens just as for files listed in the second 
argument of |\installfont|). \meta{destination font} (which must be 
different from \meta{source font}) will be taken as the name for a 
new \texttt{.mtx} file that will be generated. The destination font 
can differ from the source font only in two ways: the names of some 
glyphs in the source font might be changed, and some of the commands 
from the source font might not have been copied to the destination 
font. To what extent the fonts are different is determined by what 
modifying commands have been executed; when no modifying commands 
have been executed, the source and destination font are equal.

The modifying reglyphing commands are
\begin{quote}
  |\renameglyph|\marg{to}\marg{from}\\
  |\renameglyphweighted|\marg{to}\marg{from}\marg{weight}\\
  |\killglyph|\marg{glyph}\\
  |\killglyphweighted|\marg{glyph}\marg{weight}\\
  |\offmtxcommand|\marg{command}\\
  |\onmtxcommand|\marg{command}
\end{quote}
|\renameglyph| simply declares that occurrences of the glyph name 
\meta{from} should be replaced by the glyph name \meta{to}. To each 
glyph name is also assigned a \emph{weight}, which is used by a 
mechanism which conditions copying of commands from the source font to 
the destination font by the set of glyphs that command mentions. The 
details of this mechanism are however somewhat tricky, so those 
interested in the full generality should read the comments in the 
source of \fontinst. Here it needs only be noted that if one applies 
|\killglyph| to a glyph name, then (under most circumstances) commands 
that refer to that glyph name will not be copied to the destination 
font.

|\offmtxcommand| and |\onmtxcommand| also control whether commands are 
copied to the destination font, but they look at the actual command 
rather than the glyphs it refers to. For example, after the command
\begin{quote}
  |\offmtxcommand{\setkern}|
\end{quote}
no |\setkern| commands will be copied. By using |\offmtxcommand|, it 
is possible to achieve effects similar to those of the files 
\texttt{kernoff.mtx} and \texttt{glyphoff.mtx}---the difference is 
that with |\offmtxcommand|, it happens at an earlier stage of the font 
generation. As expected, |\onmtxcommand| undoes the effect of 
|\offmtxcommand|.

A special rule pertains to the |\set|\-|raw|\-|glyph|, 
|\set|\-|not|\-|glyph|, |\set|\-|scaled|\-|raw|\-|glyph|, and 
|\set|\-|scaled|\-|not|\-|glyph| commands, since |\transformfont| 
doesn't care what something was in the source font when it generates 
the transformed font. To turn these commands off while reglyphing, 
you use |\offmtx|\-|command| on |\set|\-|scaled|\-|raw|\-|glyph|.

The effects of modifying reglyphing commands are delimited by 
|\reglyphfonts| and |\endreglyphfonts|, which starts and ends a group 
respectively.

As we expect the most common reglyphing operation will be to go from SC 
glyph names to expert glyph names, there is a file \texttt{csc2x.tex} 
in the \fontinst distribution which contains the modifying 
reglyphing commands needed for setting up that conversion. Thus you 
can write for example
\begin{quote}
  |\reglyphfonts|\\
  |  \input csc2x|\\
  |  \reglyphfont{padrcx8r}{padrc8r}|\\
  |  \reglyphfont{padscx8r}{padsc8r}|\\
  |\endreglyphfonts|
\end{quote}
to alter the glyph names in the SC fonts in the Adobe Garamond 
(\texttt{pad}) family. 
Note that the names of the destination fonts here really are rather 
arbitrary, since they will only exist as \texttt{.mtx} files, and 
thus only need to work within your local file system. In particular, 
all the |\setrawglyph| commands in the destination font files still 
refer to the source font, so it is that font which the drivers need 
to know about.


\subsection{Making map file fragments}

A \emph{map file fragment} is the lines\footnote{Not in general an 
entire map file, hence the word \emph{fragment}.} of a map file that 
the corresponding driver would need for handling some set of fonts. 
When told to, \fontinst can (in a fairly automatic way) create 
the map file fragment which is needed for the set of raw fonts 
\fontinst has (i) installed directly (using |\installrawfont|) 
or (ii) used as a base font for some installed virtual font (generated 
by |\installfont|). \Fontinst does not support the map file 
syntaxes of every existing driver, but the system is designed to be 
extendable and contributions that extend its capabilities are welcome. 
Nor can \fontinst examine your \TeX\ system and 
determine every piece of information needed to make the correct map 
file fragments, but you can tell it roughly how your installation 
looks, it can make guesses which work most of the time, and you can 
specify most things explicitly if the guesses turn out to be wrong. 
Should the available options for configuring the process turn out to 
be inadequate for your needs, then please write to the 
\fontinst mailing list about this---there is probably a way 
to improve the system so that your needs can be met.

Now what does one have to do to use this map file fragment writer, 
then? First you need to tell \fontinst to record the 
information the map file fragment writer needs. You do this by giving 
the command
\begin{quote}
  |\recordtransforms{whatever.tex}|
\end{quote}
at the beginning of the run. Here \texttt{whatever.tex} is the name of 
a file that will be created, so you can use some other name if you 
like. After that you do all the calls to |\transform|\-|font|, 
|\install|\-|font|, |\install|\-|raw|\-|font|, |\latin|\-|family|, 
etc.\ you need to make the fonts you want. When you're done, you give 
the command
\begin{quote}
  |\endrecordtransforms|
\end{quote}
and end the run (say |\bye|). The file \texttt{whatever.tex} will now 
contain the information about which fonts were used and what needs to 
be done with them.

The second step is to actually run the map file fragment writer. 
Observe that it is located in the file \texttt{finstmsc.sty}, not 
\texttt{fontinst.sty}! The commands you need to give it can be so few 
that you can type them in at \TeX's \texttt{*} prompt, but if you are 
writing a command file then it should typically have the following 
structure (comments not necessary, of course):
\begin{quote}
  \begin{tabular}{ll}
    |\input finstmsc.sty|& |%| Input command definitions\\
    \meta{general settings} & |%| See below\\
    |\adddriver|\marg{driver name}\marg{output file}& 
      |%| Open output file\\
    |\input whatever.tex|& |%| Writes to output file(s)\\
    |\donedrivers|& |%| Close output file(s), tidy up\\
    |\bye|& |%| Quit
  \end{tabular}
\end{quote}
The |\adddriver| command gives the order ``write map file entries for 
the \meta{driver name} DVI driver to the file \meta{output file}.'' The 
plan is that it should be possible to use the name of just about any 
major driver (\texttt{dvips}, \texttt{xdvi},\footnote{Or 
does that use the same map file as \texttt{dvips}? I heard somewhere 
that it did. /LH} \texttt{pdftex},\footnote{pdf\TeX\ can read 
the map files generated for \texttt{dvips}, but a separate driver is 
desirable because the formats are not completely identical.} 
\texttt{OzTeX}, etc.) here and get suitable map file entries for that 
driver as output, but for the moment only the \texttt{dvips} and 
\texttt{dvipdfm}\footnote{Whose support I made very much to illustrate 
that you \emph{don't} have to be a big and ancient driver like 
\texttt{dvips} to have supporting code put into \fontinst. 
(The fact that I just happened to have printed out the documentation and 
that is was easy to read also helped, of course.) Note, however, that 
there won't be any support for a driver unless someone sits down and 
writes the code for it! Don't assume I will. /LH} drivers are supported. 

You may also use \texttt{debug} or \texttt{pltotf} for \meta{driver name}. 
The \texttt{debug} ``DVI driver'' file simply contains all the available 
information about each font (hence it should come handy for debugging 
code writing entries for real drivers) in a format that should be easy 
to interpret for a human. It could be the right choice if you're going 
to write the map file manually, as the combined effects of several 
font transformations are not always easy to compute manually. The 
file generated for the \texttt{pltotf} ``driver'' is actually a shell 
script consisting of a sequence of \pltotf commands. These commands 
perform the \pl to \tfm conversion for precisely those fonts that are 
actually needed (\fontinst usually generates \pl files also for a 
number of fonts at intermediate stages of transformation, and many of 
these need not be converted to \tfm files). The \texttt{TFMfileprefix} 
string can be used to add a directory path to the \tfm file names, 
perhaps saving the step of moving them to their proper location later.

The file \texttt{whatever.tex} in the above example contains the 
commands (|\makemapentry| commands) that actually cause entries to be 
written to the output file. It also contains a number of |\storemapdata| 
commands---these describe how some given font was made. If some 
metric file you have used contains |\setrawglyph| commands that were 
not automatically generated by \fontinst, then there might 
not be a |\storemapdata| for the font they refer to in 
\texttt{whatever.tex}, so you will have to include such a command 
yourself somewhere. This can for example be done in the \meta{general 
settings} part of the above example file.

Another class of things that will typically appear in the 
\meta{general settings} part above is commands that will inform the 
routines actually writing output about your \TeX\ system, about the set 
of fonts you are using on this run, or about something else that might 
be useful. Some such commands are of a general nature and affect what 
assumptions \fontinst will make in certain conditions when no 
specific information is available. For the moment there commands are:
\begin{description}
  \item[\cs{AssumeMetafont}] Assume all fonts with PL metrics are 
    bitmaps generated by Metafont, and therefore make no entries for 
    them.
  \item[\cs{AssumeAMSBSYY}] Assume all fonts with PL metrics have their 
    \TeX\ names in all upper case as postscript names---just like the 
    Computer Modern fonts in the AMS\slash Blue~Sky\slash Y\&Y 
    distribution.
  \item[\cs{AssumeBaKoMa}] Assume all fonts with PL metrics have their 
    \TeX\ names in all lower case as postscript names---just like the 
    Computer Modern fonts in the BaKoMa distribution.
\end{description}
Otherwise the default action of the routine for finding out the 
postscript name of a font simply is to observe that it hasn't got a clue 
about what the right value is when the metrics were taken from a PL 
file, and therefore it writes `\texttt{??????}' for the postscript name.
\begin{description}
  \item[\cs{AssumeLWFN}] Assume postscript fonts for which nothing 
    else has been specified are stored in files which are named 
    according to the \mbox{MacOS} scheme for 
    \texttt{LWFN}s.%\footnote{LaserWriter FoNt}
\end{description}
Otherwise the default action is to use the name of the AFM or PL from 
which the metrics were originally taken, and add the file suffix stored 
in the string \texttt{PSfontsuffix}. The default value of this string 
is \texttt{.pfa}, but it can be changed using |\resetstr|.

If neither the default nor the LWFN scheme produce correct results 
then you may use the more specific |\specifypsfont| command, which 
describes exactly which file (or files, if any) a given font is stored 
in. The syntax of this command is
\begin{quote}
  |\specifypsfont|\marg{PS font name}\marg{actions}
\end{quote}
where the \meta{actions} is a sequence of ``action commands''. 
Currently the only such command is
\begin{quote}
  |\download|\marg{file}
\end{quote}
which instructs the map file writer to include in any entry using 
that PS font and ``instruction'' that the specified file should be 
downloaded. Some examples are
\begin{verbatim}
   \specifypsfont{Times-Roman}{}
   \specifypsfont{Shareware-Cyrillic-Regular}{\download{fcyr.gsf}}
   \specifypsfont{zmnl8ac6}{%
      \download{MinionMM.pfb}\download{zmnl8ac6.pro}%
   }
\end{verbatim}
Many \dvi drivers (for example \dvips) have more than one style of 
font downloading (e.g., partial and full downloading). This interface 
could be extended to control also such finer details (for example by 
adding a |\fulldownload| command to force full download of a font), 
but requests for this has so far been scarce.

Finally, there is the |\declarepsencoding| command which is used to 
link ETX files to postscript encodings. If no postscript encoding has 
been linked to a given ETX file then \fontinst will 
automatically create a postscript encoding (\texttt{.enc}) file for 
that encoding, and use this file for all reencoding commands. 
The \texttt{8r} encoding is predeclared, and it doesn't 
matter if an encoding is undeclared if you never use it to reencode 
fonts, but there is potentially a problem with not having declared 
encodings you have installed and use for reencoding, as you may then 
find yourself having two files with identical names that define 
encodings that do not have the same name (as far as postscript is 
concerned).



\subsection{Tuning accent positions---an application of loops}

The accent placements made by \texttt{latin.mtx} certainly aren't 
perfect for all fonts, and the only way to find out where they should 
be put is through trying in text the accented letters you get for a 
couple of values for the position parameter and deciding which one 
works best. Since to try one parameter value you need to (i) edit it 
into an MTX file, (ii) run \fontinst, (iii) run 
\vptovf, (iv) run \TeX\ on some test text, and (v) print that 
text, trying one parameter value can take annoyingly much time. 
Repeating the same procedure ten times to test ten values is not 
something one does without being bored (unless one scripts it, of 
course), but it is possible to try ten parameter values in a single 
virtual font, and without doing very much typing.

Say you're not too happy with how \texttt{latin.mtx} positions the 
accent in the \texttt{ohungarumlaut} glyph:
\begin{quote}
  |\setglyph{ohungarumlaut}|\\
  |   \topaccent{o}{hungarumlaut}{500}|\\
  |\endsetglyph|
\end{quote}
The |500| is the horizontal position (in thousandths of the width of 
the \texttt{o}) that the centre of \texttt{hungarumlaut} in the glyph 
constructed will have, so that is the position parameter value that 
you want to change. Create an MTX file containing the code
\begin{quote}
  |\for(pos){250}{750}{50}|\\
  |   \setglyph{ohungarumlaut\strint{pos}}|\\
  |      \topaccent{o}{hungarumlaut}{\int{pos}}|\\
  |   \endsetglyph|\\
  |   \setleftrightkerning{ohungarumlaut\strint{pos}}|\\
  |      {ohungarumlaut}{1000}|\\
  |\endfor(pos)|
\end{quote}
This will set eleven glyphs \texttt{ohungarumlaut250}, 
\texttt{ohungarumlaut300}, \texttt{ohungarumlaut350}, \textellipsis\,, 
\texttt{ohungarumlaut750}, each being an Hungarianly umlauted `o' 
(i.e., an `\H{o}') but all having that umlaut in slightly different 
positions. In order to put them in a font, you also need to make an 
encoding that contains them. Therefore create an ETX file which 
contains the code
\begin{quote}
  |\relax\encoding|\\
  |\nextslot{"C0}|\\
  |\for(pos){250}{750}{50}|\\
  |   \setslot{ohungarumlaut\strint{pos}}|\\
  |   \endsetslot|\\
  |\endfor(pos)|\\
  |\endencoding|
\end{quote}
The command for installing this experiment font would be something like
\begin{quote}
  |\installfont|\marg{some name}|{|\meta{the normal list of metrics}%
    |,|\penalty0\meta{the new MTX}|}|\penalty0
    |{ot1,|\meta{the new ETX}|}|\penalty0|{OT1}|\textellipsis
\end{quote}
The reason for including \texttt{ot1} in the third argument above is 
that you'll need letters other than `\H{o}' against which you can 
compare the experimental glyphs. It would not have been possible to 
use \texttt{t1} instead of \texttt{ot1} (even though that has more 
Hungarian letters) since that would set all slots in the font and 
leave none for these experimental \texttt{ohungarumlaut}s. 

It is even possible to use a loop for making the test text. The 
\LaTeX\ macros
\begin{verbatim}
\newcount\slotcount
\newcommand\testtext[3]{%
  \slotcount=#1\relax
  \begin{description}%
  \loop\item[\the\slotcount]#3%
  \ifnum #2>>\slotcount \advance \slotcount 1 \repeat
  \end{description}%
}
\DeclareTextCompositeCommand{\H}{OT1}{o}{\char\slotcount}
\end{verbatim}
will let you write
\begin{quote}
  |\testtext|\marg{first}\marg{last}\marg{text}
\end{quote}
to get the text \meta{text} typeset once for each slot from 
\meta{first} to \meta{last} inclusive, with |\H{o}| ranging through the 
glyphs in this interval. Thus in this case 
|\testtext|\penalty\hyphenpenalty|{"C0}|\penalty\hyphenpenalty|{"CA}|%
\penalty\hyphenpenalty|{Erd\H{o}s}| would be a trivial test.


\subsection{Font installation commands}
\label{Ssec:FontInstCmds}

The |\installfont|, |\installrawfont|, and |\installfontas| commands 
have the respective syntaxes
\begin{isyntax}
  |\installfont|\marg{font-name}\marg{metrics}\marg{etx-list}\penalty0
    \marg{encoding}\marg{family}\marg{series}\marg{shape}\marg{size}\\
  |\installrawfont|\marg{font-name}\marg{metrics}\marg{etx-list}%
    \penalty0
    \marg{encoding}\marg{family}\marg{series}\marg{shape}\marg{size}\\
  |\installfontas|\marg{font-name}\penalty0\marg{encoding}%
     \marg{family}\marg{series}\marg{shape}\marg{size}
\end{isyntax}
The \meta{font-name} argument and the last five arguments are common 
to all these commands. The first argument is the name of a \TeX\ font 
to install. The last five arguments are the NFSS attributes under which 
that font will be declared to \LaTeX---encoding, family, series, shape, 
and size. It is worth observing that encoding names are usually in 
upper case, whereas the family, series, and shape are usually in lower 
case. The size argument is either a shorthand (declared using 
|\declaresize|) for a particular font 
size (or range of font sizes), or an explicit list of font sizes or 
ranges of sizes, which is copied directly to the font declaration. 
The most common case is to let the size argument be empty, as that is 
declared as a shorthand for ``any size''.

The |\installfontas| command does not itself create the font, it just 
makes a note that the specified font declaration should be written to 
the proper FD file at |\end|\-|install|\-|fonts|. The 
|\install|\-|font| and |\install|\-|raw|\-|font| commands do however 
produce the font, in the sense that they write a VPL and PL 
respectively file for the font. It depends solely on the \meta{metrics} 
and \meta{etx-list} arguments what this font will contain. Many 
features of these arguments are new with \fontinst v\,1.9; 
therefore the complete syntaxes are described below.

Both arguments are comma-separated lists of basically file names (not 
including an extension). The files listed in the \meta{metrics} are 
font metric files which together build up a \emph{glyph base} 
(definitions of glyphs and metrics related to one or several glyphs), 
whereas the files listed in the \meta{etx-list} are encoding definition 
files that select a subset of the glyph base for turning into a 
\TeX\ font. The font metrics can be in either of the four formats 
MTX, PL, AFM, and VPL, which are considered in that order. If the 
metrics are not originally in MTX format then they will be converted 
to this format (a new file will be created) before they are used. 
The encoding definitions must be in ETX format. The files actually 
read will have a suffix \texttt{.mtx}, \texttt{.pl}, \texttt{.afm}, 
\texttt{.vpl}, or \texttt{.etx} appended to the name given, depending 
on which format is expected.

Within each element of the comma-separated list, the actual file name 
is followed by zero or more \emph{modifier clause}s. A \meta{modifier 
clause} consists of a \emph{keyword} followed by some number (usually 
one) of \emph{arguments}, separated by spaces. The whole thing looks 
a lot like the \meta{rule specifications} of e.g.\ the |\vrule| 
command, but here the spaces are mandatory. The currently defined 
\meta{modifier clause}s are
\begin{description}
  \item[\mdseries\textvisiblespace\texttt{option}\textvisiblespace
    \meta{string}]
    Available for metric and encoding files. This adds \meta{string} 
    to the list of options for this file, which may affect what code 
    the file executes. The file can then test, using the |\ifoption| 
    command, whether a specific string is one of the options it was 
    given.
  \item[\mdseries\textvisiblespace\texttt{scaled}\textvisiblespace
    \meta{factor}]
    Available for metric files. Causes the \texttt{rawscale} integer 
    variable to be set to the \meta{factor} (an integer expression) 
    while the file is being read. This scales glyphs and kerns that 
    are added to the glyph base by the \meta{factor}.
  \item[\mdseries\textvisiblespace\texttt{suffix}\textvisiblespace
    \meta{suffix}]
    Available for metric files. Causes \meta{suffix} to be appended 
    to every glyph name appearing in a glyph or kern that file adds 
    to the glyph base. Thus ``\texttt{suffix /2}'' effectively 
    changes a
    \begin{quote}
      |\setrawglyph{a}|\dots
    \end{quote}
    to a
    \begin{quote}
      |\setrawglyph{a/2}|\dots
    \end{quote}
  \item[\mdseries\textvisiblespace\texttt{encoding}\textvisiblespace
    \meta{etx-name}]
    Available for metric files, and forces \fontinst to 
    only consider the PL and VPL formats for this font. 
    As these file formats do not contain glyph names, an ETX file 
    is used to assign glyph names to the slots in the font. 
    This ETX file is usually selected according to the 
    \texttt{CODINGSCHEME} property of the PL or VPL (using the 
    correspondences set up via the |\declare|\-|encoding| command), 
    but that information is not always as one would want it (there 
    are even fonts for which it is quite wrong). An \texttt{encoding} 
    clause bypasses this automatic mechanism, so that the file 
    \meta{etx-name}\texttt{.etx} is used instead.
    
%     % The following is no longer true as of v1.926:
%     \textbf{Note:} The first time that a file in PL or VPL format is 
%     used in a \meta{metrics} argument, a corresponding MTX file is 
%     generated. This means that if the same file reference is used 
%     another time then the reference will be to the MTX file, not to 
%     the original PL or VPL, and thus \texttt{encoding} clauses on 
%     subsequent uses will have no effect. Each font only has one 
%     encoding, so it usually makes no sense to switch the ETX file 
%     used to interpret a font, but since MTX files are not 
%     automatically deleted between runs there is a risk that this 
%     takes away the intended effect of an \texttt{encoding} clause.
    
  \item[\mdseries\textvisiblespace\texttt{mtxasetx}]
    This is available for files in the \meta{etx-list}. The actual 
    function of a
    \begin{quote}
      \meta{file-name} \texttt{mtxasetx}
    \end{quote}
    item in the \meta{etx-list} is that the file 
    \meta{file-name}\texttt{.mtx} is inputted (\emph{not} 
    \meta{file-name}\texttt{.etx}) and that the correspondence 
    between glyph names and slot numbers set up in 
    |\set|\-|raw|\-|glyph| or |\set|\-|scaled|\-|raw|\-|glyph| 
    commands in this file is treated as if it had been set up by 
    |\setslot| commands in an ETX file. Provided the MTX file is 
    transformable, the glyph base will be unaffected.
    
    The purpose of this feature is to simplify quick and dirty 
    installations of odd fonts for which no suitable ETX file is 
    available. This can be useful in early stages of the design of 
    a new font, but is inferior to installation using proper ETX 
    files since one for example cannot specify any ligatures in 
    MTX files.
\end{description}
Furthermore there is a special exception for the \meta{metrics}: if 
the first token in one of the list items is the control sequence 
|\metrics|, then the rest of that item is interpreted as explicit 
metric commands to execute.

If the \meta{metrics} of two subsequent |\install|\-|font| 
or |\install|\-|raw|\-|font| commands are identical then the glyph 
bases will be identical as well. This creates an opportunity for 
optimization, which \fontinst makes use of by caching glyph 
bases from one installation command to the next so that the glyph 
base does not have to be rebuilt in these cases. A side-effect of 
this caching is that local assignments made between two font 
installation commands are cleared out with the glyph base, but 
|\setint| and similar \fontinst commands make global 
assignments when used in such positions.

Some examples might be in order. The first is an adaptation of an 
installation command from \texttt{mfnt-0.59} %~\cite{mfnt} 
by Matthias Clasen and Ulrik Vieth: the installation command for the 
8-bit math font \texttt{xma1000} (which can be thought of as being 
to \texttt{cmmi10} sort of as \texttt{ecrm1000} is to \texttt{cmr10}). 
The first three \texttt{encoding} clauses are more fine-tuning---without 
them, a few glyphs would get incorrect names---but the last two are 
quite essential, as the \texttt{msam10} and \texttt{msbm10} fonts 
incorrectly claim to have the coding scheme \texttt{TEX MATH 
SYMBOLS}.
\begin{verbatim}
  \installfont{xma1000}{%
     yma1000 encoding mcin,%
     cmr10 encoding ot1upright,%
     cmmi10,%
     cmsy10 encoding omscal,%
     msam10 encoding msam,%
     msbm10 encoding msbm,%
     mccmhax,mccmkern,mcmissing,%
     cmsy10-base,cmsy10-extra%
  }{mc}{MC}{cm}{m}{n}{<10->}
\end{verbatim}
Also note the explicit \LaTeX\ size specification for the range 
``10\,pt and up''.

The second example makes use of a \texttt{suffix} clause to combine 
the letters from one font with the digits from another.
\begin{verbatim}
  \installfont{msbrj8t}{msbr8r,msbrc8r suffix /2,latin}{digit2,t1}
    {T1}{msbj}{m}{n}{}
\end{verbatim}
In this case, the glyph base contains the glyphs of Monotype Sabon 
(SabonMT)---under names such as \texttt{A} for `A', \texttt{a} for 
`a', and \texttt{one} for a lining digit one---as well as the 
glyphs of Monotype Sabon Small Caps and Oldstyle Figures 
(SabonMT-SCOSF)---under names such as \texttt{A/2} for `A', 
\texttt{a/2} for `\textsc{a}', and \texttt{one/2} for a hanging 
digit one. The \texttt{digit2.etx} file simply makes the definition
\begin{verbatim}
  \setcommand\digit#1{#1/2}
\end{verbatim}
which causes \texttt{t1.etx} to put \texttt{zero/2} in slot 48 (digit 
zero), \texttt{one/2} in slot 49 etc., instead of as it normally 
would \texttt{zero} in slot 48, \texttt{one} in slot 49 and so on. 
The net effect is that the digits in the generated \texttt{msbrj8t} 
is from \texttt{msbrc8r} (SabonMT-SCOSF) but everything else is from 
\texttt{msbr8r} (SabonMT).

The third example makes use of an \texttt{mtxasetx} clause to install 
(with its default encoding) a font for which creating an appropriate 
ETX file seems not worth the trouble.
\begin{verbatim}
  \installrawfont{psyr}{psyr,\metrics 
     \setint{xheight}{\height{alpha}}
  }{txtfdmns,psyr mtxasetx}{U}{psy}{m}{n}{}
\end{verbatim}
The effect of the second \texttt{psyr} is that \texttt{psyr.mtx} is 
read (in case there was no \texttt{psyr.mtx} then it is created from 
(hopefully) \texttt{psyr.afm}) and the information in it will form 
the glyph base. Because of the |\metrics| control sequence, the rest 
of that item will be interpreted as explicit metric commands 
modifying the glyph base, and thus the |\setint| command can provide 
a value for the \texttt{xheight} variable (there doesn't seem to be 
such a value in the AFM). Once the glyph base is completed, the 
|\install|\-|raw|\-|font| starts writing the file \texttt{psyr.pl} 
(that's for the first \texttt{psyr}). The encoding of that font will, 
because of the \texttt{psyr mtxasetx}, be the same as that used in 
\texttt{psyr.mtx}. Finally, the \texttt{txtfdmns} is for 
\texttt{txtfdmns.etx}, an ETX file which sets fontdimens 1--16 as for 
a \texttt{T1} encoded font but does not set any slots. Since 
\texttt{psyr.mtx} reinterpreted as an ETX file sets slots but no 
fontdimens, these complement each other nicely.



\subsection{Bounding boxes}

Han The Thanh has created an implementation of bounding box support 
for \fontinst, and it is a modified form of that support 
is distributed with \fontinst as the file \texttt{bbox.sty}. To load 
this, begin your command file with
\begin{verbatim}
   \input fontinst.sty
   \input bbox.sty
\end{verbatim}
The reason for not making it default is that keeping track of bounding 
boxes increases some of \fontinst's memory requirements quite a lot.

One important 
characteristic of this implementation is that the dimensions of the 
bounding box are not bundled into the same data structure (the 
|\g-|\meta{glyph} macros) as the glyph's width, height, depth, and 
italic correction are, but stored in a separate data structure (the 
|\gb-|\meta{glyph} macros). A glyph doesn't need to have its bounding 
box set, it is simply a piece of information that \fontinst 
will store if you tell it to and which you can later retrieve.

The bounding box will be stored as coordinates of the sides in the 
normal AFM coordinate system. The commands for retrieving these 
coordinates are
\begin{center}
  \begin{tabular}{ll}
    \textbf{Command}& \textbf{Side}\\
    |\bbtop|\marg{glyph}& top ($y$-coordinate)\\
    |\bbbottom|\marg{glyph}& bottom ($y$-coordinate)\\
    |\bbleft|\marg{glyph}& left ($x$-coordinate)\\
    |\bbright|\marg{glyph}& right ($x$-coordinate)
  \end{tabular}
\end{center}
In Thanh's implementation the command names were |\ury|, |\lly|, 
|\llx|, and |\urx| respectively instead, but I think the former are 
easier to remember. If no bounding box has been set for a glyph then 
the above commands will instead report the corresponding coordinate of 
the glyph's \TeX\ box (i.e.\ |\height|\marg{glyph}, 
|\neg{\depth|\marg{glyph}|}|, |0|, and |\width|\marg{glyph} 
respectively).

The command for setting the bounding box of a glyph is
\begin{quote}
  |\setglyphbb|\marg{glyph}\marg{left}\marg{bottom}\marg{right}%
  \marg{top}
\end{quote}



\section{Miscellaneous notes}

\subsection{On verbatim, typewriter, and monowidth fonts}

The verbatim, typewriter, and monowidth concepts are common sources 
of confusion for those who use \fontinst to install fonts 
with \LaTeX; in particular 
there are many misconceptions about the relation between them. The 
official view (of which not much has actually been brought forward) 
is that these concepts are really about three quite different things. 

A font is a \emph{monowidth} (monospaced, fixed-pitch) font if all 
glyphs in it have exactly the same width. Some font formats make 
special provisions for such fonts; the most notable example is the 
AFM format, where a single \texttt{CharWidth} keyword specifies the 
width for all glyphs in the font. \Fontinst responds to this 
by including the command
\begin{quote}
  |\setint{monowidth}{1}|
\end{quote}
in the MTX file generated from an AFM, but that is everything that is 
hard-wired into the program. That a font is monowidth is however 
something that one should take note of when installing it for \TeX, 
as it means many of the glyphs in it have such a strange appearance 
that they are (pretty much) useless. The \texttt{endash} is for 
example usually only half as long as the \texttt{hyphen} and the 
letters in ligature glyphs are only half as wide as normal letters. 
Many of the ETX and MTX files that come with \fontinst 
contain special commands to avoid making use of such degenerate 
glyphs.

That a font is a \emph{typewriter} font really only means that it has 
a typewriterish look about it. The two most familiar typewriter fonts 
are probably Computer Modern Typewriter (\texttt{cmtt}) and Courier. 
Both of these fonts are monowidth, but there is no absolute rule about 
this. One of the standard \TeX\ fonts is for example Computer Modern 
Variable-width Typewriter (\texttt{cmvtt}), which is not a monowidth 
font, as Figure~\ref{Fig:TTvsVTT} shows.
\begin{figure}
  \begin{tabular}{ll}
    \texttt{cmtt}:& \fontfamily{cmtt}\selectfont
      The quick brown fox jumps over the lazy dog.\\
    \texttt{cmvtt}:& \fontfamily{cmvtt}\selectfont
      The quick brown fox jumps over the lazy dog.
  \end{tabular}
  \caption{Two typewriter fonts}
  \label{Fig:TTvsVTT}
\end{figure}

The verbatim concept has very little to do with fonts at all; in 
\LaTeX\ it is considered to be a property of the environment 
(\texttt{verbatim}, \texttt{macrocode}, etc.) rather than a property 
of the font. The connection there is with fonts is that the encoding 
of the font must contain visible ASCII (as defined in Appendix~C 
of \emph{The \TeX book}%~\cite{TeXbook}
) as a subset for the text to 
be rendered correctly. The \texttt{cmtt} family is the only one amongst 
the original Computer Modern fonts which meets this criterion and 
that is the primary grounds for the idea that these three concepts 
should be connected. Today that reason is at best a very weak one, as 
all \texttt{T1}-encoded fonts also meet the criterion of containing 
visible ASCII as a subset.

A circumstance which has probably added to the confusion is that 
\texttt{OT1} is usually claimed to be an encoding. In reality the 
Computer Modern fonts that are declared in \LaTeX\ as being 
\texttt{OT1} display as many as five different encodings, as shown in 
Table~\ref{Tab:OT1-fonts}.
\begin{table}
  \begin{tabular}{lccc}
    & \texttt{TEX TEXT}& 
      \begin{tabular}[b]{c}\texttt{TEX TEXT WITHOUT}\\
        \texttt{F-LIGATURES}\end{tabular}&
      \texttt{TEX TYPEWRITER TEXT}\\
      \noalign{\medskip}
    non-italic&
      \begin{tabular}{l}
        \texttt{cmb10}\\
        \texttt{cmbx5}--\texttt{12}\\
        \texttt{cmbxsl10}\\
        \texttt{cmdunh10}\\
        \texttt{cmff10}\\
        \texttt{cmfib8}\\
        \texttt{cmr6}--\texttt{17}\\
        \texttt{cmsl8}--\texttt{12}\\
        \texttt{cmss8}--\texttt{17}\\
        \texttt{cmssbx10}\\
        \texttt{cmssdc10}\\
        \texttt{cmssi8}--\texttt{17}\\
        \texttt{cmssq8}\\
        \texttt{cmssqi8}\\
        \texttt{cmvtt10}
      \end{tabular}&
      \begin{tabular}{l}
        \texttt{cmcsc8}--\texttt{10}\\
        \texttt{cmr5}
      \end{tabular}&
      \begin{tabular}{l}
        \texttt{cmsltt10}\\
        \texttt{cmtcsc10}\\
        \texttt{cmtt8}--\texttt{12}
      \end{tabular}\\
      \noalign{\medskip}
    italic&
      \begin{tabular}{l}
        \texttt{cmbxti10}\\
        \texttt{cmfi10}\\
        \texttt{cmti7}--\texttt{12}\\
        \texttt{cmu10}
      \end{tabular}&&
      \begin{tabular}{l}
        \texttt{cmitt10}
      \end{tabular}
  \end{tabular}
  \caption{``\texttt{OT1}-encoded'' Computer Modern fonts, collected 
    according to the actual font encoding}
  \label{Tab:OT1-fonts}
\end{table}
Since most monowidth fonts are only used for setting verbatim text, 
there is some code in \texttt{ot1.etx} which automatically chooses a 
\texttt{TEX TYPEWRITER TEXT} encoding for the font when the 
\texttt{monowidth} integer is set. The only reason for this is the 
guess that this is what the user wanted.



\section*{Acknowledgements}

I'd like to thank all of the \fontinst $\alpha$-testers, especially
Karl Berry, Damian Cugley, Steve Grahthwohl, Yannis Haralambous, Alan
Hoenig, Rob Hutchings, Constantin Kahn, Peter Busk Laursen, Ciar{\'a}n {\'O}
Duibh{\'\i}n, Hilmar Schlegel, Paul Thompson, Norman Walsh and John Wells,
who made excellent bug-catchers!

Thanks to Barry Smith, Frank Mittelbach, and especially Sebastian
Rahtz for many useful email discussions on how virtual fonts should
interact with \LaTeXe.

Thanks to Karl Berry and Damain Cugley for detailed comments on this
documentation.

Thanks to David Carlisle for the use of his \texttt{trig} macros for
calculating trigonometry.


% \section*{Warranty and distribution}
% 
% There is no warranty for the \fontinst package, to the extent
% permitted by applicable law. Except when otherwise stated in writing,
% the author provides the program `as is' without warranty of any kind,
% either expressed or implied, including, but not limited to, the
% implied warranties of merchantability and fitness for a particular
% purpose. The entire risk as to the quality and performance of the
% program is with you.  Should the package prove defective, you assume
% the cost of all necessary servicing, repair or correction.
% 
% In no event unless required by applicable law or agreed to in writing
% will the author be liable to you for damages, including any general,
% special, incidental or consequential damages arising out of the use or
% inability to use the program (including but not limited to loss of
% data or data being rendered inaccurate or losses sustained by you or
% third parties or a failure of the program to operate with any other
% programs), even if such holder or other party has been advised of the
% possibility of such damages.
% 
% Redistribution of unchanged files is allowed provided that all files
% listed in the \texttt{MANIFEST} file are distributed.
% 
% If you receive only some of these files from someone, or if you
% receieve altered files, then complain!

\end{document}
