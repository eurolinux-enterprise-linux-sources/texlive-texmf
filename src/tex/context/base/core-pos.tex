%D \module
%D   [       file=core-pos,
%D        version=1999.08.01,
%D          title=\CONTEXT\ Core Macros,
%D       subtitle=Positioning Support,
%D         author=Hans Hagen,
%D           date=\currentdate,
%D      copyright={PRAGMA / Hans Hagen \& Ton Otten}]
%C
%C This module is part of the \CONTEXT\ macro||package and is
%C therefore copyrighted by \PRAGMA. See mreadme.pdf for
%C details.

% needs a cleanup, things may change; we also need to move the mp
% related code to meta-pos

% shorter tags, ..:achtergrond:.. etc in pos actions

% dubbele text- * pos's eruit

% class pos -> als gelijk aan vorige, dan niet niet definieren
% en erven, maw:
%
% 1 -> opslaan
% 2 -> undef, dus == prev
% 3 -> undef, dus == prev
% 4 -> opslaan

\writestatus{loading}{Context Positioning Support}

% todo: topskip als optie voor eerste regel achtergrond
% todo: build pos layers on top of layers
% todo: positionlayer pos van text-1 etc delen

%D Although \TEX\ has a rather powerful channel to the outside
%D world, called \type {\special}, real communication with
%D other programs is complicated by the fact that no positional
%D information is available. Mid 1999, I discussed this with
%D \THANH, the author of \PDFTEX, and after some experiments,
%D \PDFTEX\ was extended with a simple but effective mechanism,
%D that provided positional information. The interesting
%D thought is that, although \TEX\ is frozen, similar
%D functionality could have been achieved with \type
%D {\specials} and an additional \DVI\ postprocessor.
%D
%D Since we want to be as compatible as can be, \CONTEXT\ will
%D support both methods, although the development is primarily
%D driven by the \PDFTEX\ way of doing things. Since the
%D mechanism is basically not limited to one application, for
%D the moment we stick to building the functionality around one
%D \CONTEXT\ special command, but at the same time we keep our
%D eyes open for extensions in other directions.
%D
%D A question that may arise when one reads this module, is to
%D what extend these macros are generic, in the sense that they
%D could be collected in a support module instead of a core
%D module. Since the mechanism described here will closely
%D cooperate with the \METAPOST\ support built in \CONTEXT,
%D which in turn will be tightly integrated with the \CONTEXT\
%D overlay mechanisms, I decided to write a core module instead
%D of a support one. This makes even more sense, when one takes
%D into account that this kind of support depends on special
%D drivers.

\unprotect

%D The first application of positional information was embedded
%D graphics. Since we are interacting with text, it made sense
%D to take the current line height and depth into account too.
%D This is why we have two basic position macros: one for
%D simple positions, and one for boxes.
%D
%D We could have sticked to one special, and actually did so in
%D earlier experiments, but for convenience, as well for
%D clearness, we now have two alternatives. This approach will
%D save us quite some bytes when storing large quantities of
%D positional information. We save as less information as
%D needed, that is, we save no dimensions, in a \METAPOST\
%D friendly way.
%D
%D The three specials involved are:
%D
%D \starttyping
%D \dosetposition          {identifier}
%D \dosetpositionwhd       {identifier} {width} {height} {depth}
%D \dosetpositionplus      {identifier} {width} {height} {depth} {list}
%D \dosetpositionpapersize {width} {height}
%D \stoptyping

\newbox\positionbox
\newif \ifpositioning

\def\POSprefix{POS::}

\let\setpospx     \gobblefourarguments  % suppress errors with mkii tuo file
\let\setpospxywhd \gobblesevenarguments % suppress errors with mkii tuo file
\let\setpospxyplus\gobbleeightarguments % suppress errors with mkii tuo file

%D This is real tricky! The page anchor is applied to the
%D page box and therefore flushed first. So, when present, it
%D is applied to all positions except itself.

\chardef\positionanchormode=0 % don't relocate page origin
\chardef\positionanchormode=1 % relocate page origin once

%D The core set macros.

\let\pospxy    \gobblefourarguments
\let\pospxywhd \gobblesevenarguments
\let\pospxyplus\gobbleeightarguments

%D Sometimes we want to trick the position handler a bit:

\let\replacepospxywhd\gobbleeightarguments

%D For postprocessing purposes, we save the number of
%D positions.

\newcounter\currentpositions  % current number of positions
\newcounter\totalnofpositions % total from previous run

\appendtoks
  \savecurrentvalue\totalnofpositions\currentpositions
\to \everybye

%D The next switch can be used to communicate a special
%D situation. Positioning and associated actions can be
%D executed any time. However, in for instance backgrounds
%D they can be collected in a layer, for instance the text
%D layer (especially the hidden text layer). In the case of
%D floats, we run into problems, since the page information is
%D not applicable when the content floats indeed. In such
%D situations one can treat positions and graphics local.

\newif\iflocalpositioning

%D Watch out: sometimes a pagebreak occurs inside a float
%D placement, so there we need to disable local mode.

\appendtoks
  \localpositioningtrue
\to \everyinsidefloat

\appendtoks
  \localpositioningfalse
\to \everypagebody

\def\checkpositions
  {\startnointerference
   \protectlabels
   \doutilities{positions}\jobname\empty\relax\relax
   \global\let\checkpositions\relax
   \stopnointerference}

%D Since the positional values are to be fully expandable, we
%D need to preload them as soon as possible, which is why we
%D load the data when we start a text.

\appendtoks \checkpositions \to \everystarttext

%D Positions are either generated at a delayed write time
%D (in \PDFTEX), or derived from the dvi file. The actual
%D method is implemented in a special driver. If needed, the
%D driver can fall back on the following macros.

\let\dolazysaveposition    \gobblefourarguments  % tag page x y
\let\dolazysavepositionwhd \gobblesevenarguments % tag page x y w h d
\let\dolazysavepositionplus\gobbleeightarguments % tag page x y w h d list
\let\dosaveposition        \gobblefourarguments  % tag page x y
\let\dosavepositionwhd     \gobblesevenarguments % tag page x y w h d
\let\dosavepositionplus    \gobbleeightarguments % tag page x y w h d list

%D \macros
%D   {MPp, MPx, MPy, MPw, MPh, MPd,
%D    MPxy, MPll, MPlr, MPur, MPul, MPpos}
%D
%D Access to the positional information is provided by macros
%D with short names that are clearly meant for \METAPOST.

\let\MPp  \!!zerocount
\def\MPx  \!!zeropoint
\def\MPy  \!!zeropoint
\def\MPw  \!!zeropoint
\def\MPh  \!!zeropoint
\def\MPd  \!!zeropoint
\def\MPxy \!!zeropoint
\def\MPll \!!zeropoint
\def\MPlr \!!zeropoint
\def\MPur \!!zeropoint
\def\MPul \!!zeropoint
\def\MPpos{\!!zerocount,\!!zeropoint,\!!zeropoint,\!!zeropoint,\!!zeropoint,\!!zeropoint}

%D \macros
%D  {MPplus, MPrest, MPv, MPvv}
%D
%D Since we will probably keep on extending, we provide a
%D general extension macro. The plus alternative takes an
%D extra argument, denoting what additional parameter to pick
%D up. So, the third extra is fetched with,
%D
%D \starttyping
%D \MPplus{identifier}{3}{default}
%D \stoptyping
%D
%D All extras (comma separated) are fetched with:
%D
%D \starttyping
%D \MPrest{identifier}
%D \stoptyping
%D
%D The extra parameters are not treated.

\def\MPplus#1#2{\!!zerocount}  \def\MPv {\MPplus}
\def\MPrest#1#2{#2}            \def\MPvv{\MPrest}

%D \macros
%D   {MPanchor}
%D
%D For readability we define a few synonyms:

\def\MPanchor{\MPpos}

%D \macros
%D   {POSp, POSx, POSy, POSh, POSd, POSw}
%D
%D and:

\def\POSp{\MPp} \def\POSx{\MPx} \def\POSy{\MPy}
\def\POSh{\MPh} \def\POSd{\MPd} \def\POSw{\MPw}

%D There are two low level positioning macros. Both store the
%D position as well as execute an action associated with that
%D position.

\def\initializenextposition
  {\ifpositioning \else
     \global\positioningtrue
     \dosetpositionpapersize
       {\printpaperwidth }%
       {\printpaperheight}%
   \fi
   \doglobal\increment\currentpositions}

\def\setpositiononly#1%
  {\iftrialtypesetting
      % nothing
   \else
     \initializenextposition
     \def\currentposition{#1}%
     \dosetposition\currentposition
   \fi}

\def\setposition#1%
  {\iftrialtypesetting
      % nothing
   \else
     \initializenextposition
     \def\currentposition{#1}%
     \dosetposition\currentposition
     \traceposstring\llap\green{\currentposition>}%
     \dopositionaction\currentposition
   \fi}

\def\setpositiondata#1#2#3#4%
  {\iftrialtypesetting \else
     \initializenextposition
     \hbox
       {\def\currentposition{#1}%
        \dosetpositionwhd\currentposition
          {\the\dimexpr#2\relax}%
          {\the\dimexpr#3\relax}%
          {\the\dimexpr#4\relax}%
        \traceposstring\llap\green{\currentposition>}%
        \dopositionaction\currentposition
        \hss}%
   \fi}

\def\setpositionbox#1%
  {\dowithnextbox
     {\iftrialtypesetting
        \flushnextbox
      \else
        \initializenextposition
        \hbox to \nextboxwd
          {\edef\currentposition{#1}%
           \dosetpositionwhd\currentposition
             {\the\nextboxwd}%
             {\the\nextboxht}%
             {\the\nextboxdp}%
           \traceposstring\llap\green{\currentposition>}%
           \setbox\positionbox\flushnextbox
           \dopositionaction\currentposition
           \box\positionbox
           \hss}%
      \fi}}

\def\setpositiondataplus#1#2#3#4#5%
  {\iftrialtypesetting \else
     \initializenextposition
     \hbox to \nextboxwd
       {\edef\currentposition{#1}%
        \dosetpositionplus\currentposition
          {\the\dimexpr#2\relax}%
          {\the\dimexpr#3\relax}%
          {\the\dimexpr#4\relax}%
          {#5}%
        \traceposstring\rlap\magenta{<\currentposition}%
        \dopositionaction\currentposition
        \hss}%
   \fi}

\def\setpositionplus#1#2%
  {\dowithnextbox
     {\iftrialtypesetting
        \flushnextbox
      \else
        \initializenextposition
        \hbox to \nextboxwd
          {\edef\currentposition{#1}%
           \dosetpositionplus\currentposition
             {\the\nextboxwd}%
             {\the\nextboxht}%
             {\the\nextboxdp}%
             {#2}%
           \traceposstring\rlap\magenta{<\currentposition}%
           \setbox\positionbox\flushnextbox
           \dopositionaction\currentposition
           \box\positionbox
           \hss}%
      \fi}}

\let\currentposition\s!unknown

%D A few more low level macros take care of defining and
%D recalling actions. We could save this information in the
%D position containers themselves, this would save hash
%D entries, but at the cost of much more time consuming
%D expansion. Actions are saved globally!

\newtoks\everypositionaction

\let\POSactionprefix\POSprefix

\def\dosetpositionaction#1%
  {\setgvalue{\POSactionprefix#1::}}

%D The lists can become quite long (also because there can
%D be lots of parameters passed on) so we provide a hook
%D to clean up the list afterwards.

\let\cleanuppositionaction\gobbleoneargument

%D The next one will be overloaded later.
%D
%D \starttyping
%D \def\dopositionaction#1% test saves hash entry in etex
%D   {\ifundefined{\POSactionprefix#1::}\else
%D      \ifnum\MPp{#1}>\zerocount % new
%D        \setbox\scratchbox\hbox \bgroup
%D          \traceposstring\clap\red{<#1>}%
%D          \the\everyinsertpositionaction
%D          \the\everypositionaction
%D          \ifcollectMPpositiongraphics
%D            % can save a lot of run time
%D            \pushMPdrawing
%D            \MPshiftdrawingtrue
%D            \resetMPdrawing
%D            \getvalue{\POSactionprefix#1::}%
%D            \ifMPdrawingdone
%D              \getMPdrawing
%D            \fi
%D            \resetMPdrawing
%D            \popMPdrawing
%D          \else
%D            \getvalue{\POSactionprefix#1::}%
%D          \fi
%D          \cleanuppositionaction{#1}%
%D        \egroup
%D        \smashedbox\scratchbox
%D      \else
%D        % shouldn't happen too often
%D        \traceposstring\clap\cyan{<#1>}%
%D      \fi
%D    \fi}
%D \stoptyping

\def\doifpositionaction#1%
  {\ifundefined{\POSactionprefix#1::}%
     \@EA\gobbleoneargument
   \else
     \@EA\firstofoneargument
   \fi}

\def\doifpositionactionelse#1%
  {\ifundefined{\POSactionprefix#1::}%
     \@EA\secondoftwoarguments
   \else
     \@EA\firstoftwoarguments
   \fi}

%D We can copy a position with:
%D
%D \starttyping
%D \copyposition {to} {from}
%D \stoptyping
%D
%D Again, this is a global action.

\let\copyposition\gobbletwoarguments

%D The fact that handling positions is a two pass operation, is
%D one of the reasons why we need to be able to test for
%D existence, using:
%D
%D \starttyping
%D \doifpositionelse {identifier} {found action} {not found action}
%D \stoptyping

\let\doifpositionelse\thirdofthreearguments

%D We have now arrived at a few macros that would make sense as
%D support macros, but ended up in the core.

%D \macros
%D   {xypos}
%D
%D We have several macros available to save positions. Later
%D we will see applications.
%D
%D \starttabulate[|l|l||]
%D \NC \type {\xypos} \NC    \NC simple position with no dimensions \NC \NR
%D \NC \type {\hpos}  \NC    \NC position and characteristics of a \type {\hbox}                    \NC \NR
%D \NC \type {\vpos}  \NC    \NC position and characteristics of a \type {\vbox}                    \NC \NR
%D \NC \type {\bpos}  \NC b: \NC begin point in a line \NC \NR
%D \NC \type {\epos}  \NC e: \NC end point in a line \NC \NR
%D \NC \type {\fpos}  \NC f: \NC begin point in a paragraph \NC \NR
%D \NC \type {\tpos}  \NC t: \NC end point in a paragraph \NC \NR
%D \stoptabulate
%D
%D Each macro takes an identifier as argument, and the \type
%D {\hpos} and \type {\vpos} also expect box content.

% \def\xypos{\initializenextposition\dosetposition}

\let\xypos\setpositiononly

\def\hpos#1{\dontleavehmode\setpositionbox{#1}\hbox}
\def\vpos#1{\setpositionbox{#1}\vbox}

\def\bpos#1{\hpos{b:#1}{\strut}\ignorespaces}
\def\epos#1{\removelastspace\hpos{e:#1}{\strut}}

\def\fpos#1%
  {\setpositionplus{b:#1}\parposcounter\horizontalstrut
   \ignorespaces}

\def\tpos#1%
  {\removelastspace
   \setpositionplus{e:#1}\parposcounter\horizontalstrut}

\def\ffpos#1%
  {\setpositionplus{b:#1}\parposcounter\horizontalstrut\wpos{#1}%
   \ignorespaces}

\def\ttpos#1%
  {\removelastspace
   \setpositionplus{e:#1}\parposcounter\horizontalstrut}

\def\wpos#1%
  {\dontleavehmode\vadjust % may disappear if buried
     {\setbox0\hbox{\raise\strutdp\hbox{\rawwpos{#1}}}%
      \rlap{\smashedbox0}}}

\def\wwpos#1% \hsmashed{\llap{\rawwpos{#1}}}
  {\rlap
     {\setbox0\hbox{\rawwpos{#1}}%
      \smashedbox0}}

\def\rawwpos#1%
  {\hpos{w:#1}
     {\strut
      \hskip-\leftskip
      \hskip\hsize
      \hskip-\rightskip}}

% the next macro disables par positions (in inner boxes) and
% only registers the width

\def\setinnerparpositions
  {\let\fpos\ffpos
   \let\tpos\ttpos
   \let\wpos\wwpos}

% example of usage: (see for application "techniek")
%
% \appendtoks
%   \setinnerparpositions
% \to \everytabulate

%D When we want to calculate more complex backgrounds, we
%D need to know what the current indentation scheme is. At
%D the cost of many positions and memory, we can keep track
%D of them. This mechanism is activated automatically
%D based on information collected in the previous pass.

\newcounter\parposcounter

\newif\ifpositioningpar

% we can check for used entries, and if not, then not add one

\def\registerparoptions
  {\ifpositioningpar \ifpositioning \iftrialtypesetting \else
     \ifinpagebody \else \ifmmode \else \ifinformula \else
       \ifprocessingverbatim
         \iflinepar \doregisterparoptions \fi
       \else
         \doregisterparoptions
       \fi
     \fi \fi \fi
   \fi \fi \fi}

\chardef\parposstrut=1 % 0 => no strut data, so fall backs used

\def\doregisterparoptions
  {\doglobal\increment\parposcounter
   \begingroup
     \leftskip 1\leftskip
     \rightskip1\rightskip
     \setpositiondataplus
       {p:\parposcounter}%         % identifier
       {\the\zeropoint}%
       {\the\strutht}%
       {\the\strutdp}%
       {\the\hsize     ,% 1
        \the\leftskip  ,% 2
        \the\rightskip ,% 3
        \the\hangindent,% 4
        \the\hangafter ,% 5 (num)
        \the\parindent }% 6
    %\normalhbox{\registerparsymbol}%
     \registerparsymbol
   \endgroup}

\newif\iftracepositions

\def\traceposstring#1#2#3%
  {\iftracepositions\smashedhbox{#1{\infofont#2#3}}\fi}

\def\registerparsymbol
  {\iftracepositions
     \smashedhbox to \zeropoint
       {\hss
        \startcolor[blue]%
        \llap{\infofont\number\parposcounter}%
        \scratchdimen\onepoint
        \vrule
          \!!width 4\scratchdimen
          \!!height2\scratchdimen
          \!!depth 2\scratchdimen
        \stopcolor
        \hss}%
   \fi}

% \appendtoks \registerparoptions \to \everypar

%D Eperimental code, don't use this yet: (must be sped up anyway)

\def\@@noden{node:n:}
\def\@@nodeo{node:o:}
\def\@@nodep{node:p:}

\def\doifelsenodelocation#1%
  {\ifcsname\@@noden#1\endcsname
     \expandafter\firstoftwoarguments
   \else
     \expandafter\secondoftwoarguments
   \fi}

\def\nextnodelocation#1%
  {\ifcsname\@@noden#1\endcsname\pluscounter{\@@noden#1}\fi}

\def\newnodelocation#1%
  {\ifcsname\@@noden#1\endcsname
     \setcounter{\@@noden#1}\zerocount
     \letgvalue {\@@nodeo#1}\!!zerocount
   \fi}

\def\tagnodelocation#1%
  {\ifcsname\@@noden#1\endcsname\xypos{\@@nodep#1:\countervalue{\@@noden#1}}\fi}

\def\getnodelocationp#1{\MPp{\@@nodep#1:\countervalue{\@@noden#1}}}
\def\getnodelocationx#1{\MPx{\@@nodep#1:\countervalue{\@@noden#1}}}
\def\getnodelocationy#1{\MPy{\@@nodep#1:\countervalue{\@@noden#1}}}

\def\numnodelocationp#1#2{\MPp{\@@nodep#1:\number#2}}
\def\numnodelocationx#1#2{\MPx{\@@nodep#1:\number#2}}
\def\numnodelocationy#1#2{\MPy{\@@nodep#1:\number#2}}

\def\getnodelocationn#1{\countervalue{\@@noden#1}}
\def\getnodelocationo#1{\getvalue    {\@@nodeo#1}}

\chardef\nodelocationmode\plusone

\def\analyzenodelocation#1%
  {\ifcsname\@@noden#1\endcsname
     \doanalyzenodelocation{#1}{\getnodelocationn{#1}}\zerocount
   \fi}

\def\doanalyzenodelocation#1#2#3% class n default
  {\begingroup
   \donefalse
   \ifcase\nodelocationmode
     % do nothing
   \else
     \edef\nodelocationselfn{#2}%
     \edef\nodelocationselfp{\numnodelocationp{#1}\nodelocationselfn}%
     \edef\nodelocationselfx{\numnodelocationx{#1}\nodelocationselfn}%
     \edef\nodelocationselfy{\numnodelocationy{#1}\nodelocationselfn}%
     \scratchcounter\plusone
     \doloop
       {\ifnum\recurselevel=\nodelocationselfn\relax
          \donetrue
        \else
          \edef\nodelocationotherp{\numnodelocationp{#1}\recurselevel}%
          \edef\nodelocationotherx{\numnodelocationx{#1}\recurselevel}%
          \edef\nodelocationothery{\numnodelocationy{#1}\recurselevel}%
          \ifcase\nodelocationmode
          \or
            % ok for single column
            \ifcase\nodelocationotherp\relax
              \exitloop
            \else\ifnum\nodelocationotherp<\nodelocationselfp\relax
              \donetrue \advance\scratchcounter\plusone
            \else\ifnum\nodelocationotherp>\nodelocationselfp\relax
              % skip
            \else\ifdim\nodelocationothery>\nodelocationselfy\relax
              \donetrue \advance\scratchcounter\plusone
            \else\ifdim\nodelocationothery<\nodelocationselfy\relax
              % skip
            \else\ifdim\nodelocationotherx<\nodelocationselfx\relax
              \donetrue \advance\scratchcounter\plusone
            \fi\fi\fi\fi\fi\fi
          \or
            % acceptable for double column
            \ifcase\nodelocationotherp\relax
              \exitloop
            \else\ifnum\nodelocationotherp<\nodelocationselfp\relax
              \donetrue \advance\scratchcounter\plusone
            \else\ifnum\nodelocationotherp>\nodelocationselfp\relax
              % skip
            \else\ifnum\recurselevel>\nodelocationselfn\relax
              \donetrue \exitloop
            \else
              \donetrue \advance\scratchcounter\plusone
            \fi\fi\fi\fi
        \else
          \exitloop
        \fi
        \fi}%
   \fi
   \ifdone \else
     \scratchcounter#3\relax
   \fi
   \setxvalue{\@@nodeo#1}{\the\scratchcounter}%
   \endgroup}

\unexpanded\def\shownodelocation#1%
  {\ifcsname\@@noden#1\endcsname
    \analyzenodelocation{#1}%
     (#1,%
      n:\getnodelocationn{#1},%
      p:\getnodelocationp{#1},%
      x:\getnodelocationx{#1},%
      y:\getnodelocationy{#1},%
      o:\getnodelocationo{#1})%
   \fi}

%D \macros
%D   {doifoverlappingelse}
%D
%D A first application of positional information, is to
%D determine if two boxes do overlap:
%D
%D \starttyping
%D \doifoverlappingelse{point a}{point b}
%D   {action when overlapping}
%D   {action when not overlapping}
%D \stoptyping

\def\overlappingmargin{-2\scaledpoint}

\def\doifoverlappingelse#1#2%
  {\begingroup
   \donefalse
   \edef\!!stringa{#1}\edef\!!stringb{#2}%
   \ifnum\MPp\!!stringa=\MPp\!!stringb\relax
     \!!dimena\MPx\!!stringa
     \!!dimenb\dimexpr\MPx\!!stringa+\MPw\!!stringa\relax
     \!!dimenc\dimexpr\MPy\!!stringa-\MPd\!!stringa\relax
     \!!dimend\dimexpr\MPy\!!stringa+\MPh\!!stringa\relax
     \!!dimene\MPx\!!stringb
     \!!dimenf\dimexpr\MPx\!!stringb+\MPw\!!stringb\relax
     \!!dimeng\dimexpr\MPy\!!stringb-\MPd\!!stringb\relax
     \!!dimenh\dimexpr\MPy\!!stringb+\MPh\!!stringb\relax
     \ifdim\overlappingmargin=\zeropoint\else
       \advance\!!dimena-\overlappingmargin
       \advance\!!dimenb+\overlappingmargin
       \advance\!!dimenc-\overlappingmargin
       \advance\!!dimend+\overlappingmargin
       \advance\!!dimene-\overlappingmargin
       \advance\!!dimenf+\overlappingmargin
       \advance\!!dimeng-\overlappingmargin
       \advance\!!dimenh+\overlappingmargin
     \fi
     % more often eh fb eg fg
     \def\checkone##1##2%
       {\ifdim##1<\!!dimena \else \ifdim##1>\!!dimenb \else
          \ifdim##2<\!!dimenc \else \ifdim##2>\!!dimend \else
            \donetrue
          \fi\fi
        \fi\fi}%
     \def\checktwo##1##2%
       {\ifdim##1<\!!dimene \else \ifdim##1>\!!dimenf \else
          \ifdim##2<\!!dimeng \else \ifdim##2>\!!dimenh \else
            \donetrue
          \fi\fi
        \fi\fi}%
     \checkone\!!dimene\!!dimeng \ifdone \else
     \checkone\!!dimene\!!dimenh \ifdone \else
     \checkone\!!dimenf\!!dimeng \ifdone \else
     \checkone\!!dimenf\!!dimenh \ifdone \else
     \checktwo\!!dimena\!!dimenc \ifdone \else
     \checktwo\!!dimena\!!dimend \ifdone \else
     \checktwo\!!dimenb\!!dimene \ifdone \else
     \checktwo\!!dimenb\!!dimenc \fi \fi \fi \fi \fi \fi \fi
   \fi
   \ifdone
     \endgroup\expandafter\firstoftwoarguments
   \else
     \endgroup\expandafter\secondoftwoarguments
   \fi}

%D \macros
%D   {doifpositionsonsamepageelse,
%D    doifpositionsonthispageelse}
%D
%D Instead of letting the user handle fuzzy expansion, we
%D provide a simple test on positione being on the same page.
%D
%D \starttyping
%D \doifpositionsonsamepageelse{point a}{point b}
%D   {action when on same page}
%D   {action when not on same page}
%D \doifpositionsonthispageelse{point a}{point b}
%D   {action when on this page}
%D   {action when not on this page}
%D \stoptyping

\def\dodoifpositionsonsamepageelse#1#2#3#4%
  {\bgroup
   \scratchcounter#1\donefalse
   \def\docommand##1%
     {\ifcase\scratchcounter
        \scratchcounter\MPp{##1}\donetrue
      \else
        \ifnum\scratchcounter=\MPp{##1}\relax\else\donefalse\fi
      \fi}%
   \rawprocesscommalist[#2]\docommand
   \ifdone\egroup#3\else\egroup#4\fi}

\def\doifpositionsonsamepageelse
  {\dodoifpositionsonsamepageelse{0}}

\def\doifpositionsonthispageelse#1#2#3%
  {\dodoifpositionsonsamepageelse\realfolio}

%D Plugins:

\loadmarkfile{core-pos}

\let\MPv     \MPplus
\let\MPvv    \MPrest

\let\MPanchor\MPpos

\let\POSp\MPp \let\POSx\MPx \let\POSy\MPy
\let\POSh\MPh \let\POSd\MPd \let\POSw\MPw

\protect \endinput
