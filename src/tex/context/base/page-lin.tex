%D \module
%D   [       file=page-lin, % copied from main-001
%D        version=1997.03.31,
%D          title=\CONTEXT\ Core Macros,
%D       subtitle=Line Numbering,
%D         author=Hans Hagen,
%D           date=\currentdate,
%D      copyright={PRAGMA / Hans Hagen \& Ton Otten}]
%C
%C This module is part of the \CONTEXT\ macro||package and is
%C therefore copyrighted by \PRAGMA. See mreadme.pdf for
%C details.

\writestatus{loading}{Context Core Macros / Line Numbering}

\unprotect

\newif\ifnumberinglines
\newif\iftypesettinglines

\newcount\linenumber
\newcount\linestepper

\chardef\linenumberlocation=0

\newtoks\beforeeverylinenumbering
\newtoks\aftereverylinenumbering

\def\setuplines
  {\dodoubleargument\getparameters[\??rg]}

\def\startlines
  {\@@rgbefore
   \pushmacro\checkindentation
   \whitespace
  %\page[\v!preference]} gaat mis na koppen, nieuw: later \nobreak
   \begingroup
   \setupindenting[\@@rgindenting]%
   \typesettinglinestrue
   \setupwhitespace[\v!none]%
   \obeylines
   \ignorespaces
   \gdef\afterfirstobeyedline % tzt two pass, net als opsomming
     {\gdef\afterfirstobeyedline
        {\nobreak
         \global\let\afterfirstobeyedline\relax}}%
   \def\obeyedline
     {\par
      \afterfirstobeyedline
      \futurelet\next\dobetweenthelines}%
   \activatespacehandler\@@rgspace
   \GotoPar}

\def\stoplines
  {\endgroup
   \popmacro\checkindentation
   \@@rgafter}

%D When spacing is active we need to handle commands in
%D a special way:
%D
%D \starttyping
%D \setuplines[space=on]
%D
%D \startlines
%D Let's talk about this{\ttsl\gobbleoneargument or}that.
%D \stoplines
%D
%D \startlines
%D Let's talk about this{\getvalue{ttsl}or}that.
%D \stoplines
%D \stoptyping
%D
%D One can indent in several ways:
%D
%D \starttyping
%D \setupindenting[medium] \setuplines[indenting=odd] % no yes odd even
%D
%D \startlines
%D first
%D second
%D third
%D fourth
%D \stoplines
%D \stoptyping

% this is not the natural place

\def\installspacehandler#1#2% needs to set \obeyedspace
  {\setvalue{\??sr#1}{#2}}

\installspacehandler \v!on
  {\obeyspaces
   \def\obeyedspace{\mathortext\normalspace{\dontleavehmode{\tt\controlspace}}}%
   \let\ =\obeyedspace}

\installspacehandler \v!yes
  {\obeyspaces
   \def\obeyedspace{\mathortext\normalspace{\dontleavehmode    \normalspace  }}%
   \let\ =\obeyedspace}

\installspacehandler \v!off
  {\normalspaces
   \let\obeyedspace\normalspace
   \let\ =\normalspace}

\def\activatespacehandler#1%
  {\executeifdefined{\??sr#1}{\activatespacehandler\v!off}}

\def\dobetweenthelines
  {\doifmeaningelse\next\obeyedline\@@rginbetween\donothing}

% het gebruik van \setlocalreference scheelt een hash entry

\def\dodoshowlinenumber % for use elsewhere, to be extended
  {\doplacelinenumber
   \global\advance\linenumber \plusone}

\def\completelinenumber
  {\@@rnleft\convertnumber\@@rnconversion\linenumber\@@rnright}

\def\dosetuplinenumbering[#1]%
  {\getparameters[\??rn][\c!start=1,\c!step=1,#1]%
   \global\linenumber\plusone}

\def\setuplinenumbering
  {\dosingleargument\dosetuplinenumbering}

\def\dostartnummerenLINE
  {\EveryPar{\placelinenumber}} % why not append to everypar ? better

\def\dostopnummerenLINE
  {\the\aftereverylinenumbering
   \egroup}

\def\dostartnummerenVERB
  {\EveryLine{\placelinenumber}}

\def\dostopnummerenVERB
  {\the\aftereverylinenumbering
   \egroup}

\newevery \everylinenumber \relax

\def\dodoplacelinenumber
  {% beware of em's, the font is already switched !
   \setbox\scratchbox\hbox
     {\setbox0\hbox{\@@rncommand{\completelinenumber}}\vsmashbox0%
      \ifcase\linenumberlocation
        \iftypesettinglines % hack
          \expandafter\llap
        \else
          \expandafter\rlap
        \fi{\hbox to \@@rnwidth{\box0\hss}}% was \llap, nog testen !!
      \or
        \inleftmargin
          {\forgetall
           \doifelse\@@rnwidth\v!margin
             {\hsize\leftmarginwidth}{\hsize\@@rnwidth}%
           \alignedline\@@rnalign\v!right{\box0\hskip\@@rndistance}}%
      \else
        \inrightmargin
          {\forgetall
           \doifelse\@@rnwidth\v!margin
             {\hsize\rightmarginwidth}{\hsize\@@rnwidth}%
           \alignedline\@@rnalign\v!left{\hskip\@@rndistance\box0}}%
      \fi}%
   \vsmashbox\scratchbox
   \box\scratchbox
   \the\everylinenumber}

\def\complexstartlinenumbering[#1]%
  {\doifnot{#1}\v!continue
     {\doifnumberelse{#1}
        {\global\linenumber#1\relax}
        {\doifelsenothing\@@rnstart
           {\global\linenumber\plusone}
           {\global\linenumber\@@rnstart}}}%
   \chardef\linenumberlocation\zerocount
   \processaction
     [\@@rnlocation]
     [  \v!inmargin=>\chardef\linenumberlocation\plusone,
          \v!inleft=>\chardef\linenumberlocation\plusone,
         \v!inright=>\chardef\linenumberlocation\plustwo,
          \v!margin=>\chardef\linenumberlocation\plusone]%
       %   \v!text=>\chardef\linenumberlocation\zerocount,
       %\s!unknown=>\chardef\linenumberlocation\zerocount,
       %\s!default=>\chardef\linenumberlocation\zerocount]%
   \doifnot\@@rnwidth\v!margin
     {\freezedimenmacro\@@rnwidth
      \ifcase\linenumberlocation % text
        \advance\leftskip\@@rnwidth\relax
      \fi}%
   \freezedimenmacro\@@rndistance
   \chardef\@@rn@@rnmethod
     \ifprocessingverbatim\zerocount\else\iftypesettinglines\plusone\else\plustwo\fi\fi
   \processaction
     [\@@rnmethod]
     [ \v!type=>\chardef\@@rn@@rnmethod\zerocount,
       \v!line=>\chardef\@@rn@@rnmethod\plusone,
       \v!text=>\chardef\@@rn@@rnmethod\plustwo,
       \v!file=>\chardef\@@rn@@rnmethod\plusthree]%
   \ifcase\@@rn@@rnmethod % verbatim, line by line
     \typesettinglinestrue
     \let\dostartnummeren\dostartnummerenVERB
     \let\stoplinenumbering\dostopnummerenVERB
     \def\placelinenumber
       {\doplacelinenumber
        \global\advance\linenumber \plusone}%
   \or % text, line by line
     \let\dostartnummeren\dostartnummerenLINE
     \let\stoplinenumbering\dostopnummerenLINE
     \def\placelinenumber
       {\doplacelinenumber
        \global\advance\linenumber \plusone}%
   \or % text, whole lot
     \let\dostartnummeren\dostartnummerenPAR
     \let\stoplinenumbering\dostopnummerenPAR
     \def\placelinenumber
       {\global\advance\linenumber \minusone
        \doplacelinenumber}%
   \or % verbatim, selective line by line
     \typesettinglinestrue
     \let\dostartnummeren\dostartnummerenVERB
     \let\stoplinenumbering\dostopnummerenVERB
     \def\placelinenumber
       {\global\linenumber\verbatimlinenumber
        \global\advance\linenumber \minusone
        \doplacelinenumber}%
   \fi
   \dostartnummeren}

\def\startlinenumbering
  {\bgroup
   \the\beforeeverylinenumbering
   \numberinglinestrue
   \complexorsimpleempty\startlinenumbering}

\def\donoplacelinenumber
  {\the\everylinenumber}

\def\doplacelinenumber
  {\ifnum\linenumber<\@@rnstart\relax
     \donoplacelinenumber
   \else
     \ifnum\numexpr(\linenumber/\@@rnstep)*\@@rnstep\relax=\linenumber
       \doattributes\??rn\c!style\c!color\dodoplacelinenumber
     \else
       \donoplacelinenumber
     \fi
   \fi}

\def\someline[#1]%
  {\dolinereference0[#1]\ignorespaces}

\def\startline[#1]%
  {\dolinereference1[#1]\ignorespaces}

\def\stopline[#1]%
  {\removelastspace\dolinereference2[#1]}

\def\inline#1[#2]%
  {\doifelsenothing{#1}
     {\doifinstringelse{--}\currenttextreference
        {\in{\leftlabeltext\v!lines}{\rightlabeltext\v!lines}[\@@rnprefix#2]}
        {\in{\leftlabeltext\v!line }{\rightlabeltext\v!line }[\@@rnprefix#2]}}
     {\in{#1}[\@@rnprefix#2]}}

\def\dostartnummerenPAR
  {\beginofshapebox
   \doglobal\newcounter\linereference}

% localcrossref heroverwegen

\def\setlinereference#1#2#3#4%
  {\setxvalue{lrf:#1}{\noexpand\dogetlinereference{#2}{#3}{#4}}}

\def\getlinereference#1%
  {\getvalue{lrf:#1}}

\def\dogetlinereference#1#2#3%
  {\edef\linereferencename{#1}%
   \edef\linereferenceline{#2}%
   \edef\linereferenceplus{#3}}

% 1 xxx xxx xxx xxx xxx xxx xxx
% 2 xxx yyy yyy yyy yyy yyy yyy <= start y
% 3 yyy yyy yyy yyy yyy yyy yyy
% 4 yyy yyy yyy yyy yyy xxx xxx <= stop y
% 5 xxx xxx xxx xxx xxx xxx xxx

\def\dolinereference#1[#2]%
  {\bgroup
   \dimen0=\strutdp
   \doif\@@rnreferencing\v!on
     {\doglobal\increment\linereference
      % start 1=>(n=y,l=0,p=1)
      % stop  2=>(n=y,l=0,p=2)
      \setlinereference{\linereference}{\@@rnprefix#2}{0}{#1}%
      \advance\dimen0 \linereference sp}%
   \prewordbreak
   \vrule \!!width \zeropoint \!!depth \dimen0 \!!height \zeropoint
   \prewordbreak
   \egroup}

\def\dostopnummerenPAR % dp's -> openstrutdepth
  {\endofshapebox
   \checkreferences
   \linestepper\zerocount
   \reshapebox{\global\advance\linestepper \plusone}%
   \global\advance\linenumber \linestepper
   \doifelse\@@rnreferencing\v!on
     {\reshapebox % We are going back!
        {\global\advance\linenumber \minusone
         \dimen0=\dp\shapebox
         \advance\dimen0 -\strutdp\relax
         \ifdim\dimen0>\zeropoint
           % 1=>4 | 2=>4 1=>2
           % start 1=>(n=y,l=2,p=1)
           % stop  2=>(n=y,l=4,p=2)
           \dostepwiserecurse\plusone{\number\dimen0}\plusone
             {\getlinereference\recurselevel
              \setlinereference\recurselevel
                {\linereferencename}{\the\linenumber}{\linereferenceplus}}%
         \fi}%
      \global\advance\linenumber \linestepper
      \ifnum\linereference>\zerocount % anders vreemde loop in paragraphs+recurse
        \dorecurse\linereference
          {\getlinereference\recurselevel
           \ifnum\linereferenceplus=2 % stop
             % ref y: text = 4 / Kan dit buiten referentie mechanisme om?
             \expanded{\setlocalcrossreference
               {\referenceprefix\linereferencename}{}{}{\linereferenceline}}%
           \fi}%
        \dorecurse\linereference
          {\getlinereference\recurselevel
           \ifnum\linereferenceplus<2 % start / lone
             \ifnum\linereferenceplus=1 % start
               \getreferenceelements\linereferencename % text = 4
               \ifnum\linereferenceline<0\currenttextreference\relax % 0 prevents error
                 \edef\linereferenceline{\linereferenceline--\currenttextreference}%
               \fi
             \fi
             \expanded{\setlocalcrossreference
               {\referenceprefix\linereferencename}{}{}{\linereferenceline}}%
           \fi}%
        \global\let\scratchline\linenumber  % We are going back!
        \reshapebox
          {\doglobal\decrement\scratchline
           \hbox
             {\dorecurse\linereference
                {\getlinereference\recurselevel
                 \getreferenceelements\linereferencename
                 \beforesplitstring\currenttextreference--\at--\to\firstline
                 \ifnum\firstline=\scratchline\relax
                   % beter een rawtextreference, i.e. expanded
                   % \textreference[\linereferencename]{\currenttextreference}%
                   \rawtextreference\s!lin\linereferencename\currenttextreference
                   \expanded{\setlocalcrossreference
                     {\referenceprefix\linereferencename}{}{}{0}}% ==done
                 \fi}%
              \dimen0\dp\shapebox
              \advance\dimen0 -\strutdp\relax
              \ifdim\dimen0>\zeropoint
                \dp\shapebox\strutdp
              \fi
              \placelinenumber\box\shapebox}}% no \strut !
      \else
        \reshapebox{\hbox{\placelinenumber\box\shapebox}}% no \strut !
      \fi}
     {\reshapebox{\global\advance\linenumber \minusone}%
      \global\advance\linenumber \linestepper
      \reshapebox{\hbox{\placelinenumber\box\shapebox}}}% no \strut !
   \global\advance\linenumber \linestepper
   \flushshapebox
   \the\aftereverylinenumbering
   \egroup}

\def\emptylines
  {\dosingleempty\doemptylines}

\def\doemptylines[#1]%
  {\endgraf\dorecurse{\iffirstargument#1\else3\fi}\crlf}

\newcount\internalparagraphnumber

\def\setupparagraphnumbering
  {\dosingleempty\dosetupparagraphnumbering}

\def\dosetupparagraphnumbering[#1]%
  {\getparameters
     [\??ph][#1]%
   \processaction
     [\@@phstate]
     [\v!start=>\let\showparagraphnumber\doshowparagraphnumberA,
       \v!stop=>\let\showparagraphnumber\relax,
       \v!line=>\let\showparagraphnumber\doshowparagraphnumberB,
      \v!reset=>\global\internalparagraphnumber\zerocount
                \let\showparagraphnumber\doshowparagraphnumberA]}

\def\dodoshowparagraphnumber
  {\global\advance\internalparagraphnumber \plusone
   \inleftmargin % \tf normalizes em
     {\tf{\doattributes\??ph\c!style\c!color{\the\internalparagraphnumber}}%
      \kern\@@phdistance}}

\def\doshowparagraphnumberA
  {\ifprocessingverbatim
     \iflinepar\dodoshowparagraphnumber\fi
   \else
     \dodoshowparagraphnumber
   \fi}

\def\doshowparagraphnumberB
  {\ifnumberinglines
     \doshowparagraphnumberA
   \fi}

\setuplinenumbering
  [\c!method=,
   \c!conversion=\v!numbers,
   \c!start=1,
   \c!step=1,
   \c!location=\v!margin,
   \c!style=,
   \c!color=,
   \c!width=2em,
   \c!prefix=,
   \c!referencing=\v!on]

% new

\setuplinenumbering
  [\c!width=\ifcase\linenumberlocation2em\else\v!margin\fi,
   \c!left=,
   \c!right=,
   \c!command=,
   \c!distance=\zeropoint,
   \c!align=\ifcase\linenumberlocation\v!right\or\v!right\or\v!left\fi]

\setupparagraphnumbering
  [\c!state=\v!stop,
   \c!style=,
   \c!color=,
   \c!distance=\ifcase\linenumberlocation2em\else\!!zeropoint\fi]

\setuplines
  [\c!before=\blank,
   \c!after=\blank,
   \c!inbetween=\blank,
   \c!indenting=\v!no,
   \c!space=\v!default]

\protect \endinput
