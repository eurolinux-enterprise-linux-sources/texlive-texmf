%D \module
%D   [       file=font-ini,
%D        version=1998.09.11, % (second)
%D        version=2001.02.20, % (third)
%D          title=\CONTEXT\ Font Macros,
%D       subtitle=Initialization,
%D         author=Hans Hagen,
%D           date=\currentdate,
%D      copyright={PRAGMA / Hans Hagen \& Ton Otten}]
%C
%C This module is part of the \CONTEXT\ macro||package and is
%C therefore copyrighted by \PRAGMA. See mreadme.pdf for
%C details.

\writestatus{loading}{Context Font Macros (ini)}

\unprotect

% \def\fontrange#1%
%   {\dofontrange{#1 =\bodyfontsize}}
%
% \def\dofontrange#1%
%   {\dodofontrange#1 \relax}% \fontstringA
%
% \def\dodofontrange#1 #2 %
%   {\ifdim\bodyfontsize#2%
%      #1\expandafter\gobbleuntilrelax
%    \else
%      \expandafter\dodofontrange
%    \fi}
%
% \definefont
%   [crap]
%   [\fontrange
%      {Regular        <10pt
%       RegularBold    <12pt
%       RegularSlanted <15pt
%       Regular} sa 1]
%
% may be better:
%
% \definefontrange
%   [crap]
%   [Regular        <10pt
%    RegularBold    <12pt
%    RegularSlanted <15pt]
%   [Regular sa 1]
%
%
% \dostepwiserecurse{2}{15}{1}
%   {{\switchtobodyfont[\recurselevel pt]\crap test}\endgraf}

% adapted, else wrong interlinespace

\def\setfontparameters
  {\synchronizefontsfalse
   \the\everybodyfont
   \synchronizefontstrue}

% handy

\newcounter\pushedfont

\def\savefont
  {\edef\savedfont{\the\font}%
   \pushmacro\savedfont
   \pushmacro\currentregime
   \pushmacro\charactermapping
   \pushmacro\characterencoding}

\def\restorefont
  {\popmacro\characterencoding
   \popmacro\charactermapping
   \popmacro\currentregime
   \popmacro\savedfont
   \savedfont}

\def\pushcurrentfont
  {\edef\popcurrentfont
     {\noexpand\def\noexpand\fontbody{\fontbody}%
      \noexpand\def\noexpand\fontstyle{\fontstyle}%
      \noexpand\dosetcurrentfontalternative{\fontalternative}%
      \noexpand\dosetcurrentfontsize{\fontsize}%
      \noexpand\synchronizefont}}

% \definetypeface[one][rm][serif][computer-roman][default]
% \definetypeface[two][rm][serif][computer-roman][default][rscale=.9]
%
% {\one \bf test \two test}
% {\one \bf test \pushcurrentfont \two \popcurrentfont test}

%D \macros
%D   {definedfont,startfont,doiffontcharelse}
%D
%D A couple of relatively new macros:

\newevery \everydefinedfont \relax

\def\dodefinedfont[#1]%
  {\iffirstargument\definefont[\string\thedefinedfont][#1]\fi
   \csname\string\thedefinedfont\endcsname % can be \relax
   \the\everydefinedfont}

\unexpanded\def\definedfont
  {\dosingleempty\dodefinedfont}

\unexpanded\def\startfont
  {\bgroup\definedfont}

\def\stopfont
  {\egroup}

\def\doiffontcharelse#1#2%
  {\bgroup
   \definedfont[#1]%
   \iffontchar\font#2\relax
     \egroup\expandafter\firstoftwoarguments
   \else
     \egroup\expandafter\secondoftwoarguments
   \fi}

%%% message 14 added

\startmessages  dutch  library: fonts
  title: korps
      1: codering --
      2: variant -- wordt geladen
      3: onbekende variant --
      4: korps -- is niet gedefinieerd
      5: stijl -- is niet gedefinieerd
      6: -- wordt geladen
      7: onbekend formaat --
      8: stijl -- gedefinieerd
%     9: mapping -- is geladen
     10: onbekende font file --
     14: korps -- is gedefinieerd (kan beter globaal plaatsvinden)
\stopmessages

\startmessages  english  library: fonts
  title: bodyfont
      1: coding --
      2: variant -- is loaded
      3: unknown variant --
      4: bodyfont -- is not defined
      5: style -- is not defined
      6: -- is loaded
      7: unknown format --
      8: style -- defined
%     9: mapping -- is loaded
     10: unknown font file --
     14: bodyfont -- is defined (can better be done global)
\stopmessages

\startmessages  german  library: fonts
  title: Fliesstext
      1: Kodierung --
      2: Variante -- ist geladen
      3: Unbekannte Variante --
      4: Fliesstext -- ist nicht definiert
      5: Stil -- ist nicht definiert
      6: -- ist geladen
      7: unbekanntes Format --
      8: Stil -- definiert
%     9: Map -- ist geladen
     10: unbekanntes Font --
     14: Fliesstext -- wurde definiert (besser waere globale Definition)
\stopmessages

\startmessages  czech  library: fonts
  title: zakladnifont
      1: kodovani --
      2: varianta -- je nactena
      3: neznama varianta --
      4: zakladni font -- neni definovan
      5: styl -- neni definovan
      6: -- je nacten
      7: neznamy format --
      8: styl -- definovan
%     9: mapovani -- je nacteno
     10: neznamy font --
     14: bodyfont -- is defined (can better be done global)
\stopmessages

\startmessages  italian  library: fonts
  title: font del corpo
      1: codifica --
      2: variante -- caricata
      3: variante sconosciuta --
      4: corpo del testo -- non definito
      5: stile -- non definito
      6: -- caricato
      7: formato sconosciuto --
      8: stile -- definito
%     9: mappatura -- caricata
     10: file di font sconosciuto --
     14: corpo del testo -- definito (sarebbe meglio globale)
\stopmessages

\startmessages  norwegian  library: fonts
  title: hovedfont
      1: koding --
      2: variant -- er lest inn
      3: ukjent variant --
      4: hovedfont -- er ikke definert
      5: stil -- er ikke definert
      6: -- er lest inn
      7: ukjent format --
      8: stil -- definert
%     9: avbildning -- er lest inn
     10: ukjent fontfil --
     14: bodyfont -- is defined (can better be done global)
\stopmessages

\startmessages  romanian  library: fonts
  title: corp de litere
      1: codificarea --
      2: varianta -- este incarcata
      3: varianta necunoscuta --
      4: corpul de litere -- nu este definit
      5: stilul -- nu este definit
      6: -- este incarcat
      7: format necunoscut --
      8: stilul -- definit
%     9: maparea -- este incarcat
     10: fisier font necunoscut --
     14: bodyfont -- is defined (can better be done global)
\stopmessages

\startmessages  french  library: fonts
  title: corps de texte
      1: encodage --
      2: la variante -- est chargée
      3: variante -- inconnue
      4: policecorps -- n'est pas définie
      5: le style -- n'est pas défini
      6: -- est chargé
      7: format -- inconnu
      8: style -- défini
%     9: mapping -- is loaded
     10: fichier de police -- inconnu
     14: policecorps -- est défini (une définition globale pourrait être plus adéquat)
\stopmessages

%D This module is one of the oldest modules of \CONTEXT. The
%D macros below evolved out of the \PLAIN\ \TEX\ macros and
%D therefore use a similar naming scheme (\type{\rm},
%D \type{\bf}, etc). This module grew out of our needs. We
%D started with the \PLAIN\ \TEX\ definitions, generalized the
%D underlaying macros, and extended those to a level at which
%D probably no one will ever recognize them.
%D
%D In 2001 we ran into a couple of projects where more than
%D one combined set of fonts was involved in a document. To
%D make definitions more readable, as well as to overcome the
%D problem of ever growing file name lists, and also because
%D we needed to scale fonts relative to each other, the low
%D level implementation was partly rewritten. Global
%D font assignments, relative scaling, font classes and alike
%D were added then. At the same time some macros were made a
%D bit more readable, and math support was extended to the
%D larger sizes.
%D
%D One important characteristic of the font mechanism presented
%D here is the postponing of font loading. This makes it
%D possible to distribute \type{fmt} files without bothering
%D about the specific breed of \type{tfm} files.
%D
%D Another feature implemented here is the massive switching
%D from roman to {\ss sans serif}, {\tt teletype} or else. This
%D means one doesn't have to take care of all kind of relations
%D between fonts.
%D
%D \page[bigpreference]

%D \macros
%D   {rm,ss,tt,hw,cg}
%D
%D Fonts are defined in separate files. When we define a font,
%D we distinguish between several styles. In most cases we will
%D use:
%D
%D \startlinecorrection
%D \starttable[|l||]
%D \HL
%D \NC roman regular serif    \NC \type{\rm} \NC\FR
%D \NC sansserif sans support \NC \type{\ss} \NC\MR
%D \NC type teletype mono     \NC \type{\tt} \NC\LR
%D \HL
%D \stoptable
%D \stoplinecorrection
%D
%D The number of styles is not limited to these three. When
%D using Lucida Bright we can for instance also define:
%D
%D \startlinecorrection
%D \starttable[|l||]
%D \HL
%D \NC handwritten  \NC \type{\hw} \NC\FR
%D \NC calligraphic \NC \type{\cg} \NC\LR
%D \HL
%D \stoptable
%D \stoplinecorrection
%D
%D Anyone who feels the need, can define additional ones, like
%D
%D \startlinecorrection
%D \starttable[|l||]
%D \HL
%D \NC faxfont    \NC \type{\ff} \NC\FR
%D \NC blackboard \NC \type{\bb} \NC\LR
%D \HL
%D \stoptable
%D \stoplinecorrection
%D
%D Or even
%D
%D \startlinecorrection
%D \starttable[|l||]
%D \HL
%D \NC hebrew \NC \type{\hb} \NC\SR
%D \HL
%D \stoptable
%D \stoplinecorrection
%D
%D Styles are grouped in font sets. At the moment there are
%D three main sets defined:
%D
%D \startlinecorrection
%D \starttable[|l|l||]
%D \HL
%D \NC Computer Modern Roman     \NC Knuth             \NC \type{cmr} \NC\FR
%D \NC Lucida Bright             \NC Bigelow \& Holmes \NC \type{lbr} \NC\MR
%D \NC Standard Postscript Fonts \NC Adobe             \NC \type{pos} \NC\LR
%D \HL
%D \stoptable
%D \stoplinecorrection
%D
%D There are also some Computer Modern Roman alternatives:
%D
%D \startlinecorrection
%D \starttable[|l|l||]
%D \HL
%D \NC Computer Modern Roman    \NC Knuth \& Sauter \NC \type{sau} \NC\FR
%D \NC Euler fonts              \NC Zapf            \NC \type{eul} \NC\MR
%D \NC Computer Modern Concrete \NC Knuth \& Zapf   \NC \type{con} \NC\LR
%D \HL
%D \stoptable
%D \stoplinecorrection
%D
%D All these definitions are ordered in files with names like
%D \type{font-cmr} and \type{font-pos}, where the last three
%D characters specify the name as known to \CONTEXT.
%D
%D Within such a font set (\type{cmr}) and style (\type{\rm})
%D we can define a number of text font alternatives:
%D
%D \startlinecorrection
%D \starttable[|l||]
%D \HL
%D \NC typeface    \NC \type{\tf} \NC\FR
%D \NC boldface    \NC \type{\bf} \NC\MR
%D \NC slanted     \NC \type{\sl} \NC\MR
%D \NC italic      \NC \type{\it} \NC\MR
%D \NC boldslanted \NC \type{\bs} \NC\MR
%D \NC bolditalic  \NC \type{\bi} \NC\MR
%D \NC smallcaps   \NC \type{\sc} \NC\LR
%D \HL
%D \stoptable
%D \stoplinecorrection

%D For old stylish Frans Goddijn we have:
%D
%D \startlinecorrection
%D \starttable[|l||]
%D \HL
%D \NC oldstyle \NC \type{\os} \NC\SR
%D \HL
%D \stoptable
%D \stoplinecorrection
%D
%D The availability of these alternatives depends on the
%D completeness of a font family and of course the definitions
%D in the font files.
%D
%D But let's not forget math. In addition to the previous \TEX\
%D families (the mysterious \type{\fam}'s) we've got some more:
%D
%D \startlinecorrection
%D \starttable[|l||]
%D \HL
%D \NC Math Roman  \NC \type{\mr} \NC\FR
%D \NC Math Italic \NC \type{\mi} \NC\MR
%D \NC Math Symbol \NC \type{\sy} \NC\MR
%D \NC Math Extra  \NC \type{\ex} \NC\MR
%D \NC Math A      \NC \type{\ma} \NC\MR
%D \NC Math B      \NC \type{\mb} \NC\MR
%D \NC Math C      \NC \type{\mc} \NC\LR
%D \HL
%D \stoptable
%D \stoplinecorrection
%D
%D Users can call for specific fonts in many ways. Switches to
%D other typefaces, like the switch from normal to bold, are as
%D intuitive as possible, which means that all dependant fonts
%D also switch. One can imagine that this takes quite some
%D processing time.
%D
%D Internally fonts are stored as combination of size, style
%D and alternative, e.g. \type{12pt}+\type{\ss}+\type{\bf}.
%D Users are not confronted with sizes, but use the style or
%D style+alternative to activate them.
%D
%D During the definition of a bodyfont one can also declare the
%D available larger alternatives:
%D
%D \starttyping
%D \tf \tfa \tfb \tfc ...
%D \bf \bfa \bfb \bfc ...
%D \sl \sla \slb \slc ...
%D \stoptyping
%D
%D The smaller ones are automatically supplied and derived from
%D the the bodyfont environment.
%D
%D \starttyping
%D \tfx \tfxx
%D \bfx \bfxx
%D \slx \slxx
%D \stoptyping
%D
%D There are only two smaller alternatives per style. The
%D larger alternatives on the other hand have no limitations.
%D
%D These larger alternatives are mostly used in chapter and
%D section titles or on title pages. When one switches to a
%D larger alternative, the bold an other ones automatically
%D adapt themselves:
%D
%D \startbuffer
%D \tfd Hi \bf there\sl, here \tfb I \bf am
%D \stopbuffer
%D
%S \startnarrower
%D \typebuffer
%S \stopnarrower
%D
%D therefore becomes:
%D
%D \startvoorbeeld
%D \getbuffer
%D \stopvoorbeeld
%D
%D Maybe this mechanism isn't always as logic, but as said
%D before, we tried to make it as intuitive as possible.
%D
%D So a specific kind of glyph can be characterized by:
%D
%D \startnarrower
%D family (cmr) + bodyfont (12pt) + style (rm) + alternative (bf) + size (a)
%D \stopnarrower
%D
%D The last component (the size) is optional.
%D
%D We introduced \type{\tf} as command to call for the current
%D normally sized typeface. This commands results in roman,
%D sans serif, teletype or whatever style is in charge. Such
%D rather massive switches of style sometimes take more
%D processing time than comfortable. Of course there is a
%D workaround for this: we can call fonts directly by means of
%D commands like:
%D
%D \starttyping
%D \rmtf \sssl \tttf \rmbsa
%D \stoptyping
%D
%D One should realize that this fast calls have limitations,
%D they lack for instance automatic super- and subscript
%D support.
%D
%D This leaves us two more commands: \type{\tx} and
%D \type{\txx}. These activate a smaller and even more smaller
%D font than the current one and adapt themselves to the
%D current alternative, so when \type{\bf} is active,
%D \type{\tx} gives a smaller boldface, which in turn can be
%D called directly by \type{\bfx}.
%D
%D These two smaller alternatives are specified by the bodyfont
%D environment and therefore not necessarily have similar sizes
%D as \type{\scriptsize} and \type{\scriptscriptsize}. The main
%D reason for this incompatibility (which can easily be undone)
%D lays in the fact that we often want a bit bigger characters
%D than in math mode. In \CONTEXT\ for instance the \type{\tx}
%D and \type{\txx} commands are used for surrogate
%D \cap{smallcaps} which support both nesting and alternatives,
%D like in {\bf\cap{a \cap{small} world}}, which was typeset by
%D
%D \starttyping
%D \bf\cap{a \cap{small} world}
%D \stoptyping
%D
%D And compare $\rm \scriptstyle THIS$ with the slightly larger
%D \cap{THIS}: \ruledhbox{$\rm \scriptstyle scriptstyle: THIS$}
%D or \ruledhbox{\cap{x style: THIS}} makes a big difference.

%D The \type{x..d} sizes should be used grouped. If you
%D don't group them, i.e. call them in a row, \CONTEXT\ will
%D not be able to sort out your intention (\type {x} inside
%D \type {d} inside \type {x}. etc.). The following table
%D demonstrates this:
%D
%D \def\FontState{\setstrut\ruledhbox{\strut Hello}}
%D
%D \starttabulate[|||||]
%D \HL
%D \NC \rlap{\quad\bf grouped} \NC \NC \type {\tx} \NC \type {\txx} \NC \NR
%D \HL
%D \NC \type{\tfx}  \NC \tfx  \FontState \NC \tfx \tx \FontState \NC \tfx \txx \FontState  \NC \NR
%D \NC \type{\tfxx} \NC \tfxx \FontState \NC \tfxx\tx \FontState \NC \tfxx\txx \FontState  \NC \NR
%D \NC \type{\tf}   \NC \tf   \FontState \NC \tf  \tx \FontState \NC \tf  \txx \FontState  \NC \NR
%D \NC \type{\tfa}  \NC \tfa  \FontState \NC \tfa \tx \FontState \NC \tfa \txx \FontState  \NC \NR
%D \NC \type{\tfb}  \NC \tfb  \FontState \NC \tfb \tx \FontState \NC \tfb \txx \FontState  \NC \NR
%D \NC \type{\tfc}  \NC \tfc  \FontState \NC \tfc \tx \FontState \NC \tfc \txx \FontState  \NC \NR
%D \NC \type{\tfd}  \NC \tfd  \FontState \NC \tfd \tx \FontState \NC \tfd \txx \FontState  \NC \NR
%D \NC \type{\tfx}  \NC \tfx  \FontState \NC \tfx \tx \FontState \NC \tfx \txx \FontState  \NC \NR
%D \NC \type{\tfxx} \NC \tfxx \FontState \NC \tfxx\tx \FontState \NC \tfxx\txx \FontState  \NC \NR
%D \HL
%D \stoptabulate
%D
%D \blank
%D
%D \starttabulate[|||||]
%D \HL
%D \NC \rlap{\quad\bf stacked} \NC \NC \type {\tx} \NC \type {\txx} \NC \NR
%D \HL
%D \NC \type{\tfx}
%D \NC \tfx                                   \FontState
%D \NC \tfx                              \tx  \FontState
%D \NC \tfx                              \txx \FontState
%D \NC \NR
%D \NC \type{\tfxx}
%D \NC \tfx\tfxx                              \FontState
%D \NC \tfx\tfxx                         \tx  \FontState
%D \NC \tfx\tfxx                         \txx \FontState
%D \NC \NR
%D \NC \type{\tf}
%D \NC \tfx\tfxx\tf                           \FontState
%D \NC \tfx\tfxx\tf                      \tx  \FontState
%D \NC \tfx\tfxx\tf                      \txx \FontState
%D \NC \NR
%D \NC \type{\tfa}
%D \NC \tfx\tfxx\tf\tfa                       \FontState
%D \NC \tfx\tfxx\tf\tfa                  \tx  \FontState
%D \NC \tfx\tfxx\tf\tfa                  \txx \FontState
%D \NC \NR
%D \NC \type{\tfb}
%D \NC \tfx\tfxx\tf\tfa\tfb                   \FontState
%D \NC \tfx\tfxx\tf\tfa\tfb              \tx  \FontState
%D \NC \tfx\tfxx\tf\tfa\tfb              \txx \FontState
%D \NC \NR
%D \NC \type{\tfc}
%D \NC \tfx\tfxx\tf\tfa\tfb\tfc               \FontState
%D \NC \tfx\tfxx\tf\tfa\tfb\tfc          \tx  \FontState
%D \NC \tfx\tfxx\tf\tfa\tfb\tfc          \txx \FontState
%D \NC \NR
%D \NC \type{\tfd}
%D \NC \tfx\tfxx\tf\tfa\tfb\tfd               \FontState
%D \NC \tfx\tfxx\tf\tfa\tfb\tfd          \tx  \FontState
%D \NC \tfx\tfxx\tf\tfa\tfb\tfd          \txx \FontState
%D \NC \NR
%D \NC \type{\tfx}
%D \NC \tfx\tfxx\tf\tfa\tfb\tfc\tfx           \FontState
%D \NC \tfx\tfxx\tf\tfa\tfb\tfc\tfx      \tx  \FontState
%D \NC \tfx\tfxx\tf\tfa\tfb\tfc\tfx      \txx \FontState
%D \NC \NR
%D \NC \type{\tfxx}
%D \NC \tfx\tfxx\tf\tfa\tfb\tfc\tfx\tfxx      \FontState
%D \NC \tfx\tfxx\tf\tfa\tfb\tfc\tfx\tfxx \tx  \FontState
%D \NC \tfx\tfxx\tf\tfa\tfb\tfc\tfx\tfxx \txx \FontState
%D \NC \NR
%D \HL
%D \stoptabulate

%D \macros
%D   {mf}
%D
%D Math fonts are a species in their own. They are tightly
%D hooked into smaller and even smaller ones of similar breed
%D to form a tight family. Let's first see how these are
%D related:
%D
%D \startbuffer
%D $\tf x^2+\bf x^2+\sl x^2+\it x^2+\bs x^2+ \bi x^2 =\rm 6x^2$
%D $\tf x^2+\bf x^2+\sl x^2+\it x^2+\bs x^2+{\bi x^2}=\rm 6x^2$
%D $\tf x^2+\bf x^2+\sl x^2+\it x^2+\bs x^2+ \bi x^2 =\tf 6x^2$
%D $\tf x^2+\bf x^2+\sl x^2+\it x^2+\bs x^2+{\bi x^2}=\tf 6x^2$
%D $\tf x^2+\bf x^2+\sl x^2+\it x^2+\bs x^2+ \bi x^2 =\bf 6x^2$
%D $\tf x^2+\bf x^2+\sl x^2+\it x^2+\bs x^2+{\bi x^2}=\bf 6x^2$
%D $\tf x^2+\bf x^2+\sl x^2+\it x^2+\bs x^2+ \bi x^2 =\sl 6x^2$
%D $\tf x^2+\bf x^2+\sl x^2+\it x^2+\bs x^2+{\bi x^2}=\sl 6x^2$
%D \stopbuffer
%D
%D \typebuffer
%D
%D Gives both an expected and unexpected result:
%D
%D \startvoorbeeld
%D \startlines
%D \getbuffer
%D \stoplines
%D \stopvoorbeeld
%D
%D We see here that the character shapes change accordingly to
%D the current family, but that the symbols are always typeset
%D in the font assigned to \type{\fam0}.
%D
%D \startbuffer
%D $\tf\mf x^2 + x^2 + x^2 + x^2 + x^2 + x^2 = 6x^2$
%D $\bf\mf x^2 + x^2 + x^2 + x^2 + x^2 + x^2 = 6x^2$
%D $\sl\mf x^2 + x^2 + x^2 + x^2 + x^2 + x^2 = 6x^2$
%D $\bs\mf x^2 + x^2 + x^2 + x^2 + x^2 + x^2 = 6x^2$
%D $\it\mf x^2 + x^2 + x^2 + x^2 + x^2 + x^2 = 6x^2$
%D $\bi\mf x^2 + x^2 + x^2 + x^2 + x^2 + x^2 = 6x^2$
%D \stopbuffer
%D
%D \startvoorbeeld
%D \startlines
%D \getbuffer
%D \stoplines
%D \stopvoorbeeld
%D
%D In this example we see a new command \type{\mf} surface
%D which means as much as {\em math font}. This commands
%D reactivates the last font alternative and therefore equals
%D \type{\bf}, \type{\sl} etc. but by default it equals
%D \type{\tf}:

\unexpanded\def\mf
  {\dodosetmathfont\fontalternative
   \csname\fontalternative\endcsname}

%D The previous example was typeset saying:
%D
%D \typebuffer
%D
%D Beware: the exact location of \type{\mf} is not that
%D important, we could as well has said
%D
%D \startbuffer
%D $\bf x^2 + x^2 + x^2 + x^2 + x^2 + x^2 = \mf 6x^2$
%D \stopbuffer
%D
%D \typebuffer
%D
%D This is due to the way \TEX\ handles fonts in math mode.
%D
%D Of course we'll have to redefine \type{\mf} every time we
%D change the current \type{\fam}.

%D \macros
%D   {mbox,enablembox,mathop}
%D
%D Now how can we put this to use? Will the next sequence
%D give the desired result?
%D
%D \startbuffer
%D $\bf x^2 + \hbox{\mf whatever} + \sin(2x)$
%D \stopbuffer
%D
%D \typebuffer
%D
%D It won't!
%D
%D \startvoorbeeld
%D \let\mathop=\normalmathop \getbuffer
%D \stopvoorbeeld
%D
%D The reason for this is that \type{\sin} is defined as:
%D
%D \starttyping
%D \def\sin{\mathop{\rm sin}\nolimits}
%D \stoptyping
%D
%D We can fix this by defining

\let\normalmathop=\mathop

\def\mathop
  {\normalmathop
   \bgroup
   \let\rm\mf
   \let\next=}

%D We can fix arbitrary horizontal boxes by redefining the
%D \TEX\ primitive \type{\hbox}:
%D
%D \starttyping
%D \let\normalhbox=\hbox
%D
%D \def\hbox{\ifmmode\mbox\else\normalhbox\fi}
%D \stoptyping
%D
%D with
%D
%D \starttyping
%D \def\mbox#1#%
%D   {\normalhbox#1\bgroup\mf\let\next=}
%D \stoptyping
%D
%D or more robust, that is, also accepting \type{\hbox\bgroup}:
%D
%D \starttyping
%D \def\mbox%
%D   {\normalhbox\bgroup\mf
%D    \dowithnextbox{\flushnextbox\egroup}%
%D    \normalhbox}
%D \stoptyping
%D
%D And now:
%D
%D \startbuffer
%D $\bf x^2 + \hbox{whatever} + \sin(2x)$
%D \stopbuffer
%D
%D \typebuffer
%D
%D Indeed gives:
%D
%D \startvoorbeeld
%D \enablembox\getbuffer
%D \stopvoorbeeld
%D
%D But, do we want this kind of trickery to be activated? No,
%D simply because we cannot be sure of incompatibilities,
%D although for instance unboxing goes ok. Therefore we
%D introduce:

% best can go to math-ini and make \mf a hook then

% better use \dowithnextboxcontent

\def\normalmbox
  {\normalhbox\bgroup\mf
   \dowithnextbox{\flushnextbox\egroup}\normalhbox}

% to test:
%
% \def\normalmbox
%   {\dowithnextboxcontent\mf\flushnextbox\normalhbox}

\def\mbox
  {\ifmmode\normalmbox\else\normalhbox\fi}

\def\enablembox
  {\appendtoks
     \ifx\normalhbox\undefined\let\normalhbox\hbox\fi
     \let\hbox\mbox
   \to\everymathematics}

%D So in fact one can enable this feature if needed. I would say:
%D go along, but use grouping if needed!

%D \macros
%D   {mrfam,mifam,syfam,exfam,
%D    bsfam,bifam,scfam,tffam,
%D    mafam,mbfam,msfam}
%D
%D After this short mathematical excursion, we enter the world
%D of fonts and fontswitching. We start with something very
%D \TEX: \type{\fam} specified font families. \TEX\ uses
%D families for managing fonts in math mode. Such a family has
%D three members: text, script and scriptscript: $x^{y^z}$. In
%D \CONTEXT\ we take a bit different approach than \PLAIN\
%D \TEX\ does. \PLAIN\ \TEX\ needs at least four families for
%D typesetting math. We use those but give them symbolic names.

\chardef\mrfam =  0 % (Plain TeX) Math Roman
\chardef\mifam =  1 % (Plain TeX) Math Italic
\chardef\syfam =  2 % (Plain TeX) Math Symbol
\chardef\exfam =  3 % (Plain TeX) Math Extra

%D \PLAIN\ \TEX\ also defines families for {\it italic}, {\sl
%D slanted} and {\bf bold} typefaces, so we don't have to
%D define them here.

\ifx\itfam\undefined

\chardef\itfam =  4 % (Plain TeX) Italic
\chardef\slfam =  5 % (Plain TeX) Slanted
\chardef\bffam =  6 % (Plain TeX) Boldface

\fi

%D Family~7 in \PLAIN\ \TEX\ is not used in \CONTEXT, because
%D we do massive switches from roman to sans serif, teletype or
%D other faces.

\ifx\ttfam\undefined

\chardef\ttfam =  7 % (Plain TeX) can be reused!

\fi

%D We define ourselves some more families for {\bs bold
%D slanted}, {\bi bold italic} and {\sc Small Caps}, so
%D we can use them in math mode too. Instead of separate
%D families for {\ss sans serif} and \type{teletype} we use the
%D more general \type{\tffam}, which stands for typeface.

\chardef\bsfam =  8 % (ConTeXt) BoldSlanted
\chardef\bifam =  9 % (ConTeXt) BoldItalic
\chardef\scfam = 10 % (ConTeXt) SmallCaps
\chardef\tffam = 11 % (ConTeXt) TypeFace

%D Because Taco needs a few more math families, we reuse
%D family~7 for all those typefaces that have no related
%D family, and therefore are grouped into one.

\chardef\nnfam =  7 % (ReUsed) NoName

%D Normally \type{\mrfam} equals \type{\tffam}, but a more
%D distinctive alternatives are possible, for instance the
%D Euler and Concrete Typefaces.
%D
%D After having defined all those in nature non||mathematical
%D families, we define ourselves some real math ones. These are
%D needed for the \AMS\ Symbol Fonts and Extended Lucida
%D Bright.

\chardef\mafam = 12 % (ConTeXt) Math A Fam (AmsTeX A)
\chardef\mbfam = 13 % (ConTeXt) Math B Fam (AmsTeX B)
\chardef\mcfam = 14 % (ConTeXt) Math C Fam (MathTime)
\chardef\mdfam = 15 % (ConTeXt) Math D Fam (MathTime)

%D Because there are 16~families and because \type{\ttfam}
%D is reused, at the moment we have no so many families
%D left. By default, we map any newly defined family on the
%D last one (F).

\def\newfam#1{\chardef#1=15 }

%D This hack is also needed because in \ETEX\ we are going
%D to reuse the \type {\newfam} allocation counter.

%D To ease the support of font packages, we als define
%D shortcuts to these familynames. This is necessary because
%D the family names are in fact \type{\chardef}'s, which means
%D that we're dealing with numbers (one can check this by
%D applying \type{\showthe} and \type{\show}). In the
%D specification of math symbols however we need hexadecimal
%D numbers, so we have to convert the \type{\fam}'s value.

\edef\hexmrfam {\hexnumber\mrfam}  \edef\hexbsfam {\hexnumber\bsfam}
\edef\hexmifam {\hexnumber\mifam}  \edef\hexbifam {\hexnumber\bifam}
\edef\hexsyfam {\hexnumber\syfam}  \edef\hexscfam {\hexnumber\scfam}
\edef\hexexfam {\hexnumber\exfam}  \edef\hextffam {\hexnumber\tffam}
\edef\hexitfam {\hexnumber\itfam}  \edef\hexmafam {\hexnumber\mafam}
\edef\hexslfam {\hexnumber\slfam}  \edef\hexmbfam {\hexnumber\mbfam}
\edef\hexbffam {\hexnumber\bffam}  \edef\hexmcfam {\hexnumber\mcfam}
\edef\hexnnfam {\hexnumber\nnfam}  \edef\hexmdfam {\hexnumber\mdfam}

%D \macros
%D   {uchar}
%D
%D This macro prepares \CONTEXT\ for \UNICODE\ support. By
%D defining it here, we have at least an safeguard for utility
%D file reading.

\ifx\uchar\undefined \unexpanded\def\uchar#1#2{[#1,#2]} \fi

%D We define some (very private) constants to improve speed,
%D memory usage and consistency.

\def\@size@       {@f@si@} % bodyfont size prefix (12pt etc)
\def\@style@      {@f@st@} % full style prefix (roman etc)
\def\@shortstyle@ {@f@sh@} % short style prefix (rm etc)
\def\@letter@     {@f@le@} % first alternative typeface
\def\@noletter@   {@f@no@} % second alternative typeface

%D The families can be grouped into math specific ones and
%D more text related families, although text ones can be
%D mapped onto the math ones to get for instance bold math.
%D
%D Both groups of families are handles by a couple of token
%D list tagged as strategies. This implementation makes
%D implementing extensions more comfortable.

\newtoks \textstrategies
\newtoks \mathstrategies
\newtoks \symbstrategies

\newif\ifsynchronizemathfonts \synchronizemathfontstrue

\def\synchronizetext % stylish text in mmode
  {\ifsynchronizemathfonts\the\textstrategies\fi} % \if...\fam\minusone\fi}

\def\synchronizemath % math stuff in mmode
  {\ifsynchronizemathfonts\the\mathstrategies\fi} % \if...\fam\minusone\fi}

\def\synchronizesymb % stylish math stuff in mmode
  {\ifsynchronizemathfonts\the\symbstrategies\fi} % \if...\fam\minusone\fi}

%D By not setting the family we can append a font switch to \tyep
%D {\everymath}. On the other hand, one never knows in what family
%D state the strategies brought us.
%D
%D \starttyping
%D {\bfa $\the\fam$} {\bfa \everymath{} $\the\fam$}
%D \stoptyping

%D \macros
%D   {textonly}
%D
%D We can inhibit this slow||downer with:

\def\textonly{\synchronizemathfontsfalse} % document this

\appendtoks
  \dosettextfamily\c!tf
  \dosettextfamily\c!bf
  \dosettextfamily\c!sl
  \dosettextfamily\c!it
  \dosettextfamily\c!bs
  \dosettextfamily\c!bi
  \dosettextfamily\c!sc
\to \textstrategies

\def\dosettextfamily#1% better pass fontbody to dodoset
  {\let\savedfontbody\fontbody
   \let\fontfamily#1%
   \let\fontbody\scriptscriptface\dodosettextfamily\scriptscriptfont
   \let\fontbody\scriptface      \dodosettextfamily      \scriptfont
   \let\fontbody\textface        \dodosettextfamily        \textfont
   \let\fontbody\savedfontbody}

% \def\s!nullfont{nullfont}

\beginTEX

\def\dodosettextfamily#1%
  {\@EA\ifx\csname\fontclass\fontbody\c!mm\fontfamily\fontsize\endcsname\relax
   \@EA\ifx\csname\fontclass\fontbody\c!mm\fontfamily\endcsname         \relax
   \@EA\ifx\csname\fontclass\fontbody\c!rm\fontfamily\fontsize\endcsname\relax
   \@EA\ifx\csname\fontclass\fontbody\c!rm\fontfamily\endcsname         \relax
   \@EA\ifx\csname          \fontbody\c!mm\fontfamily\fontsize\endcsname\relax
   \@EA\ifx\csname          \fontbody\c!mm\fontfamily\endcsname         \relax
   \@EA\ifx\csname          \fontbody\c!rm\fontfamily\fontsize\endcsname\relax
   \@EA\ifx\csname          \fontbody\c!rm\fontfamily\endcsname         \relax
   \autofontsizetrue            \nullfont                             \else
   \autofontsizetrue            \csname\fontbody\c!rm\fontfamily\endcsname         \fi\else
   \autofontsizefalse           \csname\fontbody\c!rm\fontfamily\fontsize\endcsname\fi\else
   \autofontsizetrue            \csname\fontbody\c!mm\fontfamily\endcsname         \fi\else
   \autofontsizefalse           \csname\fontbody\c!mm\fontfamily\fontsize\endcsname\fi\else
   \autofontsizetrue  \csname\fontclass\fontbody\c!rm\fontfamily\endcsname         \fi\else
   \autofontsizefalse \csname\fontclass\fontbody\c!rm\fontsize\endcsname\fi\else
   \autofontsizetrue  \csname\fontclass\fontbody\c!mm\fontfamily\endcsname         \fi\else
   \autofontsizefalse \csname\fontclass\fontbody\c!mm\fontfamily\fontsize\endcsname\fi
   #1\csname\fontfamily\s!fam\endcsname\font}

\endTEX

\beginETEX \ifcsname

%\def\dodosettextfamily#1%
%  {\ifcsname\fontclass\fontbody\c!mm\fontfamily\fontsize\endcsname \autofontsizefalse
%     \csname\fontclass\fontbody\c!mm\fontfamily\fontsize\endcsname \else
%   \ifcsname\fontclass\fontbody\c!mm\fontfamily\endcsname          \autofontsizetrue
%     \csname\fontclass\fontbody\c!mm\fontfamily\endcsname          \else
%   \ifcsname\fontclass\fontbody\c!rm\fontfamily\fontsize\endcsname \autofontsizefalse
%     \csname\fontclass\fontbody\c!rm\fontfamily\fontsize\endcsname \else
%   \ifcsname\fontclass\fontbody\c!rm\fontfamily\endcsname          \autofontsizetrue
%     \csname\fontclass\fontbody\c!rm\fontfamily\endcsname          \else
%   \ifcsname          \fontbody\c!mm\fontfamily\fontsize\endcsname \autofontsizefalse
%     \csname          \fontbody\c!mm\fontfamily\fontsize\endcsname \else
%   \ifcsname          \fontbody\c!mm\fontfamily\endcsname          \autofontsizetrue
%     \csname          \fontbody\c!mm\fontfamily\endcsname          \else
%   \ifcsname          \fontbody\c!rm\fontfamily\fontsize\endcsname \autofontsizefalse
%     \csname          \fontbody\c!rm\fontfamily\fontsize\endcsname \else
%   \ifcsname          \fontbody\c!rm\fontfamily\endcsname          \autofontsizetrue
%     \csname          \fontbody\c!rm\fontfamily\endcsname          \else
%                      \nullfont                                    \autofontsizetrue
%   \fi\fi\fi\fi\fi\fi\fi\fi
%   #1\csname\fontfamily\s!fam\endcsname\font}
%
% slightly faster:

\def\dodosettextfamily
  {\ifx\fontclass\empty
     \@EA\dodosettextfamilyA
   \else
     \@EA\dodosettextfamilyB
   \fi}

\def\dodosettextfamilyA#1%
  {\ifcsname          \fontbody\c!mm\fontfamily\fontsize\endcsname \autofontsizefalse
     \csname          \fontbody\c!mm\fontfamily\fontsize\endcsname \else
   \ifcsname          \fontbody\c!mm\fontfamily\endcsname          \autofontsizetrue
     \csname          \fontbody\c!mm\fontfamily\endcsname          \else
   \ifcsname          \fontbody\c!rm\fontfamily\fontsize\endcsname \autofontsizefalse
     \csname          \fontbody\c!rm\fontfamily\fontsize\endcsname \else
   \ifcsname          \fontbody\c!rm\fontfamily\endcsname          \autofontsizetrue
     \csname          \fontbody\c!rm\fontfamily\endcsname          \else
                      \nullfont                                    \autofontsizetrue
   \fi\fi\fi\fi
   #1\csname\fontfamily\s!fam\endcsname\font}

\def\dodosettextfamilyB#1%
  {\ifcsname\fontclass\fontbody\c!mm\fontfamily\fontsize\endcsname \autofontsizefalse
     \csname\fontclass\fontbody\c!mm\fontfamily\fontsize\endcsname \else
   \ifcsname\fontclass\fontbody\c!mm\fontfamily\endcsname          \autofontsizetrue
     \csname\fontclass\fontbody\c!mm\fontfamily\endcsname          \else
   \ifcsname\fontclass\fontbody\c!rm\fontfamily\fontsize\endcsname \autofontsizefalse
     \csname\fontclass\fontbody\c!rm\fontfamily\fontsize\endcsname \else
   \ifcsname\fontclass\fontbody\c!rm\fontfamily\endcsname          \autofontsizetrue
     \csname\fontclass\fontbody\c!rm\fontfamily\endcsname          \else
     \dodosettextfamilyA#1%
   \fi\fi\fi\fi
   #1\csname\fontfamily\s!fam\endcsname\font}

\endETEX

\def\mrfallback{\c!rm\c!tf}

\appendtoks
  \dosetmathfamily\mrfam\textface\scriptface\scriptscriptface\c!mr\mrfallback
  \dosetmathfamily\mifam\textface\scriptface\scriptscriptface\c!mi\empty
  \dosetmathfamily\syfam\textface\scriptface\scriptscriptface\c!sy\empty
  \dosetmathfamily\exfam\textface\textface  \textface        \c!ex\empty
  \dosetmathfamily\mafam\textface\scriptface\scriptscriptface\c!ma\empty
  \dosetmathfamily\mbfam\textface\scriptface\scriptscriptface\c!mb\empty
  \dosetmathfamily\mcfam\textface\scriptface\scriptscriptface\c!mc\empty
% \dosetmathfamily\mdfam\textface\scriptface\scriptscriptface\c!md\empty
  \dosetmathfamily\nnfam\textface\scriptface\scriptscriptface\c!nn\empty
\to \mathstrategies

\appendtoks
  \dosetskewchar\mifam\defaultskewcharmi % implemented later on
  \dosetskewchar\syfam\defaultskewcharsy % implemented later on
\to \mathstrategies

\def\dosetmathfamily#1#2#3#4#5#6%
  {\let\savedfontbody\fontbody % op hoger plan
   \let\fontfamily#5%
   \let\backfamily#6%
   \let\fontbody  #4\dodosetmathfamily\scriptscriptfont#1%
   \let\fontbody  #3\dodosetmathfamily      \scriptfont#1%
   \let\fontbody  #2\dodosetmathfamily        \textfont#1%
   \let\fontbody\savedfontbody}

\beginTEX

\def\dodosetmathfamily#1#2%
  {\@EA\ifx           \csname\fontclass\fontbody\c!mm\fontfamily\fontsize\endcsname\relax
   \@EA\ifx           \csname\fontclass\fontbody\c!mm\fontfamily         \endcsname\relax
   \@EA\ifx           \csname\fontclass\fontbody     \backfamily\fontsize\endcsname\relax
   \@EA\ifx           \csname\fontclass\fontbody     \backfamily         \endcsname\relax
   \@EA\ifx           \csname          \fontbody\c!mm\fontfamily\fontsize\endcsname\relax
   \@EA\ifx           \csname          \fontbody\c!mm\fontfamily         \endcsname\relax
   \@EA\ifx           \csname          \fontbody     \backfamily\fontsize\endcsname\relax
   \@EA\ifx           \csname          \fontbody     \backfamily         \endcsname\relax
   \autofontsizetrue  \nullfont                                                        \else
   \autofontsizetrue  \csname          \fontbody     \backfamily         \endcsname\fi \else
   \autofontsizefalse \csname          \fontbody     \backfamily\fontsize\endcsname\fi \else
   \autofontsizetrue  \csname          \fontbody\c!mm\fontfamily         \endcsname\fi \else
   \autofontsizefalse \csname          \fontbody\c!mm\fontfamily\fontsize\endcsname\fi \else
   \autofontsizetrue  \csname\fontclass\fontbody     \backfamily         \endcsname\fi \else
   \autofontsizefalse \csname\fontclass\fontbody     \backfamily\fontsize\endcsname\fi \else
   \autofontsizetrue  \csname\fontclass\fontbody\c!mm\fontfamily         \endcsname\fi \else
   \autofontsizefalse \csname\fontclass\fontbody\c!mm\fontfamily\fontsize\endcsname\fi
   #1#2\font}

\endTEX

\beginETEX \ifcsname

%\def\dodosetmathfamily#1#2%
%  {\ifcsname\fontclass\fontbody\c!mm\fontfamily\fontsize\endcsname \autofontsizefalse
%     \csname\fontclass\fontbody\c!mm\fontfamily\fontsize\endcsname \else
%   \ifcsname\fontclass\fontbody\c!mm\fontfamily\endcsname          \autofontsizetrue
%     \csname\fontclass\fontbody\c!mm\fontfamily\endcsname          \else
%   \ifcsname\fontclass\fontbody\backfamily\fontsize\endcsname      \autofontsizefalse
%     \csname\fontclass\fontbody\backfamily\fontsize\endcsname      \else
%   \ifcsname\fontclass\fontbody\backfamily\endcsname               \autofontsizetrue
%     \csname\fontclass\fontbody\backfamily\endcsname               \else
%   \ifcsname          \fontbody\c!mm\fontfamily\fontsize\endcsname \autofontsizefalse
%     \csname          \fontbody\c!mm\fontfamily\fontsize\endcsname \else
%   \ifcsname          \fontbody\c!mm\fontfamily\endcsname          \autofontsizetrue
%     \csname          \fontbody\c!mm\fontfamily\endcsname          \else
%   \ifcsname          \fontbody\backfamily\fontsize\endcsname      \autofontsizefalse
%     \csname          \fontbody\backfamily\fontsize\endcsname      \else
%   \ifcsname          \fontbody\backfamily\endcsname               \autofontsizetrue
%     \csname          \fontbody\backfamily\endcsname               \else
%                      \nullfont                                    \autofontsizetrue
%   \fi\fi\fi\fi\fi\fi\fi\fi
%   #1#2\font}
%
% sligthly faster:

\def\dodosetmathfamily
  {\ifx\fontclass\empty
     \@EA\dodosetmathfamilyA
   \else
     \@EA\dodosetmathfamilyB
   \fi}

\def\dodosetmathfamilyA#1#2%
  {\ifcsname          \fontbody\c!mm\fontfamily\fontsize\endcsname \autofontsizefalse
     \csname          \fontbody\c!mm\fontfamily\fontsize\endcsname \else
   \ifcsname          \fontbody\c!mm\fontfamily         \endcsname \autofontsizetrue
     \csname          \fontbody\c!mm\fontfamily         \endcsname \else
   \ifcsname          \fontbody     \backfamily\fontsize\endcsname \autofontsizefalse
     \csname          \fontbody     \backfamily\fontsize\endcsname \else
   \ifcsname          \fontbody     \backfamily         \endcsname \autofontsizetrue
     \csname          \fontbody     \backfamily         \endcsname \else
                      \nullfont                                    \autofontsizetrue
   \fi\fi\fi\fi
   #1#2\font}

\def\dodosetmathfamilyB#1#2%
  {\ifcsname\fontclass\fontbody\c!mm\fontfamily\fontsize\endcsname \autofontsizefalse
     \csname\fontclass\fontbody\c!mm\fontfamily\fontsize\endcsname \else
   \ifcsname\fontclass\fontbody\c!mm\fontfamily         \endcsname \autofontsizetrue
     \csname\fontclass\fontbody\c!mm\fontfamily         \endcsname \else
   \ifcsname\fontclass\fontbody     \backfamily\fontsize\endcsname \autofontsizefalse
     \csname\fontclass\fontbody     \backfamily\fontsize\endcsname \else
   \ifcsname\fontclass\fontbody     \backfamily         \endcsname \autofontsizetrue
     \csname\fontclass\fontbody     \backfamily         \endcsname \else
     \dodosetmathfamilyA#1#2%
   \fi\fi\fi\fi
   #1#2\font}

\endETEX

\appendtoks
  \dosetsymbfamily\mrfam\textface\scriptface\scriptscriptface\c!mr
  \dosetsymbfamily\mifam\textface\scriptface\scriptscriptface\c!mi
  \dosetsymbfamily\syfam\textface\scriptface\scriptscriptface\c!sy
  \dosetsymbfamily\exfam\textface\textface  \textface        \c!ex
  \dosetsymbfamily\mafam\textface\scriptface\scriptscriptface\c!ma
  \dosetsymbfamily\mbfam\textface\scriptface\scriptscriptface\c!mb
  \dosetsymbfamily\mcfam\textface\scriptface\scriptscriptface\c!mc
% \dosetsymbfamily\mdfam\textface\scriptface\scriptscriptface\c!md % also ?
\to \symbstrategies

\def\dosetsymbfamily#1#2#3#4#5%
  {\let\savedfontbody\fontbody
   \let\fontfamily#5%
   \let\fontbody  #4\dodosetsymbfamily\scriptscriptfont#1%
   \let\fontbody  #3\dodosetsymbfamily      \scriptfont#1%
   \let\fontbody  #2\dodosetsymbfamily        \textfont#1%
   \let\fontbody\savedfontbody}

\beginTEX

\def\dodosetsymbfamily#1#2%
  {\@EA\ifx\csname\fontclass\fontbody\c!mm\fontfamily\fontalternative\fontsize\endcsname\relax
     \@EA\ifx\csname\fontbody\c!mm\fontfamily\fontalternative\fontsize\endcsname\relax
     \else
       \csname\fontbody\c!mm\fontfamily\fontalternative\fontsize\endcsname
       #1#2\font
     \fi
   \else
     \csname\fontclass\fontbody\c!mm\fontfamily\fontalternative\fontsize\endcsname
     #1#2\font
   \fi}

\endTEX

\beginETEX \ifcsname

\def\dodosetsymbfamily#1#2%
  {\ifcsname\fontclass\fontbody\c!mm\fontfamily\fontalternative\fontsize\endcsname
     \csname\fontclass\fontbody\c!mm\fontfamily\fontalternative\fontsize\endcsname
     #1#2\font
   \else\ifcsname\fontbody\c!mm\fontfamily\fontalternative\fontsize\endcsname
     \csname\fontbody\c!mm\fontfamily\fontalternative\fontsize\endcsname
     #1#2\font
   \fi\fi}

\endETEX

%D All used styles, like rm, ss and tt, are saved in a comma
%D separated list. Appart from practical limitations one can
%D define as many styles as needed.

\let\stylelist=\empty

\def\fontsizelist{\s!text,\s!script,\s!scriptscript,\c!x,\c!xx,\c!big,\c!small}

%D \macros
%D   {magfactor,magfactorhalf}
%D
%D There are several ways to specify a font. Three of them are
%D pure \TeX\ ones, the fourth one is new:
%D
%D \starttyping
%D \font\name=cmr12
%D \font\name=cmr12 at 10pt
%D \font\name=cmr12 scaled \magstep2
%D \font\name=cmr12 sa 1.440
%D \stoptyping
%D
%D The non||\TEX\ alternative \type{sa} stands for {\em scaled
%D at}. This means as much as: scale the bodyfontsize with this
%D factor. The value 1.440 in this example is derived
%D from the \type{\magstep}'s as mentioned in
%D \in{table}[tab:magstep]. We therefore introduce
%D \type{\magfactor} as an alternative for \type{\magstep}.
%D
%D \placetable[here][tab:magstep]
%D   {Factors to be used with \type{sa.}}
%D \starttable[|c|c|c|]
%D \HL
%D \NC \bf magstep \NC \bf equivalent     \NC \bf factor \NC\SR
%D \HL
%D \NC 1           \NC \type{\magfactor1} \NC 1.200      \NC\FR
%D \NC 2           \NC \type{\magfactor2} \NC 1.440      \NC\MR
%D \NC 3           \NC \type{\magfactor3} \NC 1.728      \NC\MR
%D \NC 4           \NC \type{\magfactor4} \NC 2.074      \NC\MR
%D \NC 5           \NC \type{\magfactor5} \NC 2.488      \NC\LR
%D \HL
%D \stoptable

\def\magstep#1% \relax removed, otherwise space after it sticks, else added
  {\ifcase#1 \@m\or1200\or1440\or1728\or2074\or2488\or\@m\fi}

\def\magstephalf
  {1095}

\def\magfactor#1%
  {\ifcase#1 1.000\or1.200\or1.440\or1.728\or2.074\or2.488\or1\fi}

\def\magfactorhalf
  {1.095}

%D These macros enable the use of definitions like \type{sa
%D \magfactor3} which saves us both (mis|)|calculations and
%D potential mistypings.
%D
%D Because \type {sa} (scaled at) and \type {mo} (mapped on)
%D are not low level \TEX\ supported alternatives, we have to
%D test for it ourselves. In doing so, we need an auxiliary
%D \DIMENSION. We cannot use \type{\scratchdimen} because font
%D loading can happen at any moment due to postponed loading.
%D We could instead have used dirty grouping tricks, but this
%D one works too.

\let\defaultrelativefontsize \plusone
\let\localrelativefontsize   \plusone
\def\localabsolutefontsize  {\fontbody}

\let\relativefontsize \defaultrelativefontsize

\def\saverelativefontsize#1#2% #1=rm|ss|.. #2=waarde
  {\setxvalue{\fontclass#1\s!rscale}{#2}}

\beginTEX

\def\checkrelativefontsize#1%
  {\edef\relativefontsize
     {\@EA\ifx\csname\fontclass#1\s!rscale\endcsname\relax
        \defaultrelativefontsize
      \else
        \csname\fontclass#1\s!rscale\endcsname
      \fi}}

\endTEX

\beginETEX \ifcsname

\def\checkrelativefontsize#1%
  {\edef\relativefontsize
     {\ifcsname\fontclass#1\s!rscale\endcsname
        \csname\fontclass#1\s!rscale\endcsname
      \else
        \defaultrelativefontsize
      \fi}}

\endETEX

%D We also save:

\def\savemathtextstyle#1% #1=rm|ss|..
  {\doifsomething{#1}{\setxvalue{\fontclass\c!mm\s!text}{#1}}}

\def\currentmathtextstyle % we default to empty !
  {\executeifdefined{\fontclass\c!mm\s!text}\empty}

%D Scaling macros:

\newdimen\scaledfont

\let\somefontspec\empty

\def\@fs@{@fs@}

% \def\parsefontspec#1%
%   {\xdef\somefontspec{#1\space\relax}%
%    \@EA\doparsefontspec\somefontspec]% =>#1 \relax]
%    \donoparsefontspec}
%
% but, better:

\def\parsefontspec#1%
  {\xdef\somefontspec{#1\space}%
   \@EA\doparsefontspec\somefontspec\relax]% =>#1 \relax]
   \donoparsefontspec}

\def\doparsefontspec#1 #2%
  {\def\somefontname{#1}%
   \ifx\somefontname\empty % space before fontname
     \@EA\doparsefontspec
   \else\ifx#2\relax
     \@EAEAEA\setnaturalfontspec
   \else
     \@EAEAEA\setforcedfontspec
   \fi\fi
   #2}

\def\setnaturalfontspec#1]%
  {\@fs@scaled\plusthousand\relax}

\def\setforcedfontspec#1 #2\relax]%
  {\csname\@fs@#1\endcsname#2\relax}

%D For \XETEX\ we need to support a bit more:
%D
%D \starttyping
%D \definedfont[{cmr10}  at 40pt] test
%D \definedfont['cmss10' at 40pt] test
%D \definedfont["cmtt10" at 40pt] test
%D
%D \definedfont[{cmr10}  at 40pt] test
%D \definedfont['cmss10' at 40pt] test
%D \definedfont["cmtt10" at 40pt] test
%D
%D \definedfont["Gentium" at 40pt] test
%D \stoptyping

\beginXETEX \font

    % actually, we can use "" in pdftex> 1.20 as well

    \def\parsefontspec#1%
      {\xdef\somefontspec{#1\space}%
       \@EA\doparsefontspec\somefontspec\relax]% =>#1 \empty]
       \donoparsefontspec}

    \def\doparsefontspec#1%
      {\ifx"#1%
         \@EA\dodqparsefontspec
       \else\ifx'#1%
         \@EAEAEA\dosqparsefontspec
       \else
         \@EAEAEA\dodoparsefontspec
       \fi\fi#1}

    \def\dqfontquote{"}
    \def\sqfontquote{'}

    \def\dodqparsefontspec"#1"#2%
      {\def\somefontname{#1}%
       \let\somefontquote\dqfontquote
       \ifx#2\relax\@EA\setnaturalfontspec\else\@EA\setforcedfontspec\fi#2}

    \def\dosqparsefontspec'#1'#2%
      {\def\somefontname{#1}%
       \let\somefontquote\sqfontquote
       \ifx#2\relax\@EA\setnaturalfontspec\else\@EA\setforcedfontspec\fi#2}

    \def\dodoparsefontspec#1 #2%
      {\def\somefontname{#1}%
       \let\somefontquote\empty
       \ifx\somefontname\empty % space before fontname
         \@EA\doparsefontspec
       \else\ifx#2\relax
         \@EAEAEA\setnaturalfontspec
       \else
         \@EAEAEA\setforcedfontspec
       \fi\fi
       #2}

    \def\setnaturalfontspec#1]%
      {\@fs@scaled\plusthousand\relax}

    \def\setforcedfontspec#1 #2\relax]%
      {\csname\@fs@#1\endcsname#2\relax}

    \def\donoparsefontspec % #1 == \cs
      {\edef\fontfile{\somefontquote\truefontname\somefontname\somefontquote}%
       \ifx\fontfile\s!unknown \let\fontfile\defaultfontfile \fi
       \edef\lastfontname{\fontfile\somefontspec}%
       \ifx\fontclass\empty
         \expandafter\definefontlocal
       \else
         \expandafter\definefontglobal
       \fi} % #1 == \cs

\endXETEX

\def\@fs@scaled
  {\afterassignment\do@fs@scaled\scratchcounter}

\def\do@fs@scaled#1\relax
  {\scaledfont\number\scratchcounter\points
   \scaledfont\localrelativefontsize\scaledfont
   \ifautofontsize\scaledfont\currentfontbodyscale\scaledfont\fi
   \scratchcounter\scaledfont % \scaledfont is now pretty large
   \advance\scratchcounter \medcard
   \divide\scratchcounter \maxcard
   \xdef\somefontspec{ scaled \the\scratchcounter}}

\def\@fs@at
  {\afterassignment\do@fs@at\scaledfont}

\def\do@fs@at#1\relax
  {\scaledfont\localrelativefontsize\scaledfont
   \ifautofontsize\scaledfont\currentfontbodyscale\scaledfont\fi
   \xdef\somefontspec{ at \the\scaledfont}}

\def\@fs@sa
  {\scaledfont\localabsolutefontsize
   \setsamofontspec}

\def\@fs@mo
  {\scaledfont\setmappedfontsize\localabsolutefontsize
   \setsamofontspec}

\def\setsamofontspec#1\relax
  {\checkfontscale#1\end\scaledfont
   \scaledfont\localrelativefontsize\scaledfont
   \ifautofontsize\scaledfont\currentfontbodyscale\scaledfont\fi
   \xdef\somefontspec{ at \the\scaledfont}}

\def\getfontparameters
  {\expandafter\dogetfontparameter\@@fontdata,]=,}

\def\getglobalfontparameters
  {\expandafter\dogetglobalfontparameter\@@fontdata,]=,}

\def\dogetfontparameter#1=#2,%
  {\if]#1\else
     \expandafter\def\csname\??ff\@@fontfile#1\endcsname{#2}%
     \expandafter\dogetfontparameter
   \fi}

\def\dogetglobalfontparameter#1=#2,%
  {\if]#1\else
     \expandafter\gdef\csname\??ff\@@fontfile#1\endcsname{#2}%
     \expandafter\dogetglobalfontparameter
   \fi}

\let\@@fontencoding\empty
\let\@@fontmapping \empty
\let\@@fonthandling\empty
\let\@@skewchar    \empty
\let\@@hyphenchar  \empty % todo, will go to encoding

%D This brings down maps processing from 466 to 309 seconds
%D ($-33$\%) and mfonts from 42 to 34 seconds ($-15$\%).

\newif\ifskipfontcharacteristics \skipfontcharacteristicstrue

%D When fontclasses are used, we define the font global,
%D since namespaces are used. Otherwise we parse the specs
%D each time.

% wrong: this way we cannot set encoding etc
%
% \def\donoparsefontspec#1%
%   {\edef\fontfile{\truefontname\somefontname}%
%    \ifx\fontfile\s!unknown \let\fontfile\defaultfontfile \fi
%    \edef\lastfontname{\fontfile\somefontspec}%
%    \ifx\fontclass\empty\else\global\fi
%    \expandafter\font\csname#1\endcsname\lastfontname\relax
%    \relax}

\let\fontfile\s!unknown

\def\donoparsefontspec % #1 == \cs
  {\edef\fontfile{\truefontname\somefontname}%
   \ifx\fontfile\s!unknown \let\fontfile\defaultfontfile \fi
   \edef\lastfontname{\fontfile\somefontspec}%
   \ifx\fontclass\empty
     \expandafter\definefontlocal
   \else
     \expandafter\definefontglobal
   \fi} % #1 == \cs

\def\definefontlocal#1%
  {\expandafter\font\csname#1\endcsname\lastfontname\relax}

\def\definefontglobal#1% stores \somefontname=Mono and \fontfile=cmtt10
  {\expandafter\xdef\csname#1\endcsname
     {\noexpand\csname#1:\endcsname
      \noexpand\reactivatefont{\somefontname}{\fontfile}}%
   \global\expandafter\font\csname#1:\endcsname\lastfontname\relax}

\def\reactivatefont#1%#2%
  {\def\somefontname{#1}\def\fontfile}%{#2}}

% can be handy for tracing purposes
%
% \def\reportfontdefinition
%   {\bgroup
%    \expanded{\infofont
%      [\lastfontidentifier
%       ->\newfontidentifier
%       ->\fontname\csname\newfontidentifier\endcsname]}%
%    \endgraf
%    \egroup}

%D An additional the second \type {\font} definition can
%D prevent fuzzy font refs
%D
%D \starttyping
%D \expandafter\font\csname\dummyfontname\endcsname\lastfontname
%D \stoptyping
%D
%D but somehow they changed \TEX\ so that it does not matter
%D any more.

\def\defaultfontfile{cmtt10}
\edef\nullfontname  {\fontname\nullfont}
\edef\dummyfontname {font\strippedcsname\\}

%D \macros
%D  {everyfont,everyfontswitch}
%D
%D For special purposes, we provide a hook.

% \newevery \everyfont       \relax
% \newevery \everyfontswitch \relax

%  safer but sometimes introducing newlines in the log
%
%  \batchmode
%  \font\lastloadedfont\fontfile\somefontspec\relax
%  \errorstopmode
%  \edef\lastfontname{\fontname\lastloadedfont}%
%  \ifx\lastfontname\nullfontname
%    \showmessage\m!fonts{10}{\fontfile}%
%    \expandafter\font\csname#1\endcsname=\defaultfontfile\somefontspec\relax
%  \else
%    \expandafter\let\csname#1\endcsname\lastloadedfont
%  \fi

%D We also accept \type{sa a}||\type{sa d} as specification.

\beginETEX \ifcsname

\def\checkfontscale#1#2\end#3%
  {#3\ifcsname\??ft\s!default\noexpand#1\endcsname
     \csname\??ft\s!default\noexpand#1\endcsname\else#1#2%
   \fi#3}

\endETEX

\beginTEX

\def\checkfontscale#1#2\end#3%
  {#3\expandafter\ifx\csname\??ft\s!default\noexpand#1\endcsname\relax
     #1#2\else\csname\??ft\s!default\noexpand#1\endcsname
   \fi#3}

\endTEX

%D The duplicate font definition, using the ever the same dummy
%D font name, results in less fuzzy error messages. In the log
%D file, for instance when overfull boxes are reported, the
%D simple keyword `font' replaces the \TEX\ ordinated name. The
%D latter can be too misleading, due to the fact that \TEX\ has
%D a rather optimized font memory management. Thanks to Taco
%D for helping me sort this out.

%D \macros
%D   {definefontsynonym, doifelsefontsynonym,
%D    expandfontsynonym, truefontname, truefontdata}
%D
%D While looking for fonts, fontsynonyms are used for accessing
%D the files!
%D
%D \starttyping
%D \definefontsynonym[Serif][Lucida-Bright]
%D \definefontsynonym[Lucida-Bright][lbr][encoding=texnansi]
%D \stoptyping
%D
%D The definitions can be accessed by:
%D
%D \startlines
%D name: \type {\truefontname{Lucida-Bright}}
%D data: \type {\truefontdata{\truefontname{Lucida-Bright}}}
%D \stoplines

% \def\definefontsynonym
%   {\dotripleempty\dodefinefontsynonym}
%
% \def\dodefinefontsynonym[#1][#2][#3]%
%   {\@EA\edef\csname\??ff\fontclass#1\endcsname{#2}%
%    \ifthirdargument
%      \edef\@@fontdata{#3}%
%      \ifx\@@fontdata\empty \else
%        \edef\@@fontfile{#2}%
%        \ifx\fontclass\empty
%          \getfontparameters
%        \else
%          \getglobalfontparameters
%        \fi
%      \fi
%    \fi}
%
% slightly faster, noticable when loading many typefaces,
%
% \testfeatureonce{5000}{\definefontsynonym[somefont][somename]} \end


\def\classfont#1#2{#1#2} % \definefont[whatever][\classfont{xx}{yy} at 10pt]

\def\definefontsynonym[#1]#2[#3]%
  {\edef\@@fontfile{#3}%
   \@EA\let\csname\??ff\fontclass#1\endcsname\@@fontfile
   \doifnextcharelse[\dodefinefontsynonym\donothing}

\def\dodefinefontsynonym[#1]%
  {\edef\@@fontdata{#1}%
   \ifx\@@fontdata\empty \else \ifx\fontclass\empty
     \getfontparameters
   \else
     \getglobalfontparameters
   \fi \fi}

\let\definefontfile\definefontsynonym % dedicated to Taco Hoekwater

\def\setupfontsynonym
  {\dodoubleempty\dosetupfontsynonym}

\def\dosetupfontsynonym[#1][#2]%
  {\edef\@@fontdata{#2}%
   \ifx\@@fontdata\empty \else
     \edef\@@fontfile{#1}%
     \ifx\fontclass\empty
       \getfontparameters
     \else
       \getglobalfontparameters
     \fi
   \fi}

\beginTEX

\def\truefontdata#1#2%
  {\expandafter\ifx\csname\??ff#1#2\endcsname\relax
     \expandafter\ifx\csname\??ff\fontclass#1\endcsname\relax
       \expandafter\ifx\csname\??ff#1\endcsname\relax
         \expandafter\ifx\csname\??ff#2\endcsname\relax
         \else
           % raw(key)
           \csname\??ff#2\endcsname
         \fi
       \else
         % exp(Regular) raw(key)
         \expandafter\truefontdata\csname\??ff#1\endcsname#2%
       \fi
     \else
       % exp(palatino Regular) raw(key)
       \expandafter\truefontdata\csname\??ff\fontclass#1\endcsname#2%
     \fi
   \else
     % raw(Regular) raw(key)
     \csname\??ff#1#2\endcsname
   \fi}

\def\truefontname#1%
  {\expandafter\ifx\csname\??ff\fontclass#1\endcsname\relax
     \expandafter\ifx\csname\??ff#1\endcsname\relax
       #1%
     \else
       \@EA\truefontname\csname\??ff#1\endcsname
     \fi
   \else
     \@EA\truefontname\csname\??ff\fontclass#1\endcsname
   \fi}

\def\expandfontsynonym#1#2% #2 := onelevelexpansion(#1)
  {\@EA\ifx\csname\??ff\fontclass#2\endcsname\relax\else
     \expandafter\def\expandafter#1\expandafter{\csname\??ff\fontclass#2\endcsname}%
   \fi}

\def\doifelsefontsynonym#1%
  {\@EA\ifx\csname\??ff\fontclass#1\endcsname\relax
     \expandafter\secondoftwoarguments
   \else
     \expandafter\firstoftwoarguments
   \fi}

\endTEX

\beginETEX \ifcsname

\def\truefontdata#1#2%
  {\ifcsname\??ff#1#2\endcsname
     % raw(Regular) raw(key)
     \csname\??ff#1#2\endcsname
   \else\ifcsname\??ff\fontclass#1\endcsname
     % exp(palatino Regular) raw(key)
     \expandafter\truefontdata\csname\??ff\fontclass#1\endcsname#2%
   \else\ifcsname\??ff#1\endcsname
     % exp(Regular) raw(key)
     \expandafter\truefontdata\csname\??ff#1\endcsname#2%
   \else\ifcsname\??ff#2\endcsname
     % raw(key)
     \csname\??ff#2\endcsname
   \fi\fi\fi\fi}

\def\truefontname#1%
  {\ifcsname\??ff\fontclass#1\endcsname
     \@EA\truefontname\csname\??ff\fontclass#1\endcsname
   \else\ifcsname\??ff#1\endcsname
     \@EA\truefontname\csname\??ff#1\endcsname
   \else
     #1%
   \fi\fi}

\def\expandfontsynonym#1#2% #2 := onelevelexpansion(#1)
  {\ifcsname\??ff\fontclass#2\endcsname
     \expandafter\def\expandafter#1\expandafter{\csname\??ff\fontclass#2\endcsname}%
   \fi}

\def\doifelsefontsynonym#1%
  {\ifcsname\??ff\fontclass#1\endcsname
     \expandafter\firstoftwoarguments
   \else
     \expandafter\secondoftwoarguments
   \fi}

\endETEX

% \definetypeface[palatino][rm][serif][palatino,allbold][default]
%
% \startfontclass[palatino]
%   \definefontsynonym [Serif]        [SerifBold]
%   \definefontsynonym [SerifItalic]  [SerifBoldItalic]
%   \definefontsynonym [SerifSlanted] [SerifBoldSlanted]
%   \definefontsynonym [SerifCaps]    [SerifBold]
% \stopfontclass
%
% \setupbodyfont[palatino]

\def\startfontclass
  {\dosingleempty\dostartfontclass}

\def\dostartfontclass[#1]%
  {\pushmacro\fontclass
   \doifsomething{#1}{\def\fontclass{#1}}}

\def\stopfontclass
  {\popmacro\fontclass}

%D \macros
%D   {tracedfontname}
%D
%D A goody:

\beginTEX

\def\tracedfontencoding#1%
  {\expandafter\ifx\csname\??ff#1\s!encoding\endcsname\relax\else
     \space[\csname\??ff#1\s!encoding\endcsname]%
   \fi}

\def\tracedfontname#1%
  {\expandafter\ifx\csname\??ff\fontclass#1\endcsname\relax
     \expandafter\ifx\csname\??ff#1\endcsname\relax
       #1%
     \else
       #1\tracedfontencoding{#1}\space->\space
         \@EA\tracedfontname\csname\??ff#1\endcsname
     \fi
   \else
     #1\tracedfontencoding{\fontclass#1}\space->\space
       \@EA\tracedfontname\csname\??ff\fontclass#1\endcsname
   \fi}

\endTEX

\beginETEX

\def\tracedfontencoding#1%
  {\ifcsname\??ff#1\s!encoding\endcsname
     \space[\csname\??ff#1\s!encoding\endcsname]%
   \fi}

\def\tracedfontname#1%
  {\ifcsname\??ff\fontclass#1\endcsname
     #1\tracedfontencoding{\fontclass#1}\space->\space
       \@EA\tracedfontname\csname\??ff\fontclass#1\endcsname
   \else\ifcsname\??ff#1\endcsname
     #1\tracedfontencoding{#1}\space->\space
       \@EA\tracedfontname\csname\??ff#1\endcsname
   \else
     #1%
   \fi\fi}

\endETEX

%D \macros
%D   {getfontfileparameters}
%D
%D For special purposes, one can use the next macro to
%D access font file characteristics, for instance:
%D
%D \starttyping
%D \getfontfileparameters{Regular}
%D \stoptyping
%D
%D can result in:
%D
%D \starttyping
%D \def\currentfontfileencoding{texnansi}
%D \stoptyping

% \def\@@currentfontfile{currentfontfile}
%
% \def\getfontfileparameters#1%
%   {\edef\fontdata{\truefontdata{\truefontname{#1}}}%
%    \expanded{\getparameters[\@@currentfontfile][\fontdata]}}

\def\getfontfileparameters#1%
  {\edef\@@truefontname{\truefontname{#1}}%
   \edef\currentfontfileencoding{\truefontdata\@@truefontname\s!encoding}%
   \edef\currentfontfilemapping {\truefontdata\@@truefontname\s!mapping }%
   \edef\currentfontfilehandling{\truefontdata\@@truefontname\s!handling}}

%D \macros
%D   {definefont}
%D
%D Before we implement the main definition macro, we first show
%D one for local use:
%D
%D \starttyping
%D \definefont[Some][LucidaBright at 100pt]    \Some some
%D \definefont[More][LucidaBright scaled 3000] \More more
%D \definefont[Nice][LucidaBright mp 2]        \Nice nice
%D \definefont[Text][LucidaBright sa 5.4]      \Text last
%D \stoptyping
%D
%D The implementation one looks familiar:

% The simple implementation is:
%
% \def\definefont
%   {\dodoubleargument\dodefinefont}
%
% \def\dodefinefont[#1][#2]%
%   {\doifinstringelse{ }{#2}
%      {\unexpanded\setvalue{#1}{\dododefinefont{#1}{#2}}}
%      {\dodefinefont[#1][#2 sa 1]}}

\def\definefont
  {\dotripleempty\dodefinefont}

\def\dodefinefont[#1][#2][#3]% [name][spec][1.6 | line=10pt | setup_id]
  {\doifinstringelse{ }{#2}
     {\ifthirdargument
        \unexpanded\setvalue{#1}{\redodefinefont{#1}{#2}{#3}}%
      \else
        \unexpanded\setvalue{#1}{\dododefinefont{#1}{#2}}%
      \fi}
     {\definefont[#1][#2 sa *][#3]}}

%\def\redodefinefont#1#2#3%
%  {\dododefinefont{#1}{#2}%
%   \doifnumberelse{#3}
%     {\setupinterlinespace[#3]\setupspacing}
%     {\doifassignmentelse{#3}
%        {\setupinterlinespace[#3]\setupspacing}
%        {\setups[#3]}}} % so, we can have setups associated to fonts !

\def\redodefinefont#1#2#3%
  {\dododefinefont{#1}{#2}%
   \doifsetupselse{#3}
     {\setups[#3]} % don't forget to document this !
     {\setuplocalinterlinespace[#3]%
      \setupspacing}} % needed ?

% \def\defineclassfont
%   {\doquadrupleempty\dodefineclassfont}
%
% \def\dodefineclassfont[#1][#2][#3][#4]%  #2 = class
%   {\iffourthargument
%      \definefont[#1][#2#3][#4]%
%   %\else\ifthirdargument
%   %  \definefont[#1][#2#3]%
%    \else
%      \definefont[#1][#2]%
%    \fi}

%D The \type {*} makes the switch local, so that we can redefine a
%D logical name and/or change the size in between.

\newif\ifautofontsize \autofontsizetrue

\let\lastfontidentifier\empty

\def\rawfontidentifier{**\lastfontidentifier\fontsize**}
\def\newfontidentifier{*\fontclass\lastfontidentifier\fontsize*}

% best we can add the style as well because otherwise we get math encoding mixups
%
% \setupbodyfont   [8pt] [\currentencoding/\ccaron\scaron\zcaron]
% \blank
% \switchtobodyfont[7pt] [\currentencoding/\ccaron\scaron\zcaron]

\def\newfontidentifier{*\fontclass\lastfontidentifier\fontstyle\fontsize*}

\def\dododefinefont#1#2%
  {\edef\lastfontidentifier{#1}%
   \let\localrelativefontsize\defaultrelativefontsize
   \let\localabsolutefontsize\fontbody
   \parsefontspec{#2}\rawfontidentifier
   \let\localrelativefontsize\defaultrelativefontsize % not needed
   \csname\rawfontidentifier\endcsname
   \autofontsizefalse
   \setfontcharacteristics
   \the\everyfontswitch}

\beginTEX

\def\xxdododefinefont#1#2#3#4% \autofontsizetrue is set by calling routine
  {\edef\lastfontidentifier{#3}%
   \@EA\ifx\csname\newfontidentifier\endcsname\relax
     \def\localrelativefontsize{#1}%
     \def\localabsolutefontsize{#2}%
     \parsefontspec{#4}\newfontidentifier
     \let\localrelativefontsize\defaultrelativefontsize % not needed
   \fi
   \csname\newfontidentifier\endcsname
   \autofontsizefalse
  %\edef\lastfontidentifier{#3}%
   \ifskipfontcharacteristics \else
     \setfontcharacteristics
     \the\everyfontswitch
   \fi}

\endTEX

\beginETEX \ifcsname

\def\xxdododefinefont#1#2#3#4% \autofontsizetrue is set by calling routine
  {\edef\lastfontidentifier{#3}%
   \ifcsname\newfontidentifier\endcsname\else
     \def\localrelativefontsize{#1}%
     \def\localabsolutefontsize{#2}%
     \parsefontspec{#4}\newfontidentifier
     \let\localrelativefontsize\defaultrelativefontsize % not needed
   \fi
   \csname\newfontidentifier\endcsname
   \autofontsizefalse
  %\edef\lastfontidentifier{#3}%
   \ifskipfontcharacteristics \else
     \setfontcharacteristics
     \the\everyfontswitch
   \fi}

\endETEX

%D I considered checking for mistakenly use of \PLAIN's
%D \type{\magstep}'s but although it would take only a few
%D lines of code, this would not add to consistent use. I
%D therefore removed this check.

%D \macros
%D   {mapfontsize}
%D
%D For special purposes, like in math, you may want to use
%D slightly different sizes than the ones given. This happens
%D for instance with the Math Times fonts. Mapped font sizes
%D can be specified by using the \type {mo} key instead of
%D \type {sa} in font definitions.
%D
%D \startbuffer
%D \mapfontsize[10pt][11pt]
%D \mapfontsize[11pt][12pt]
%D \mapfontsize[12pt][13pt]
%D
%D \definefont[test][Serif]\test TEST \par
%D \definefont[test][Serif sa 5]\test TEST \par
%D \definefont[test][Serif mo 5]\test TEST \par
%D \definefont[test][Serif sa d]\test TEST \par
%D \definefont[test][Serif at 60pt]\test TEST \par
%D \definefont[test][Serif scaled 6000]\test TEST \par
%D \stopbuffer
%D
%D \typebuffer
%D
%D \startpacked
%D \getbuffer
%D \stoppacked

\def\mapfontsize
  {\dodoubleargument\domapfontsize}

\def\domapfontsize[#1][#2]%
  {\scratchdimen#1\relax % \relax is really needed here
   \setvalue{\??ft*\the\scratchdimen}{#2}}

\beginTEX

\def\setmappedfontsize#1%
  {\expandafter\ifx\csname\??ft*#1\endcsname\relax
     #1%
   \else
     \csname\??ft*#1\endcsname
   \fi}

\endTEX

\beginETEX \ifcsname

\def\setmappedfontsize#1%
  {\ifcsname\??ft*#1\endcsname
     \csname\??ft*#1\endcsname
   \else
     #1%
   \fi}

\endETEX

%D \macros
%D   {getfontname}
%D
%D The names of the fonts can be called with the rather simple
%D macro \type{\getfontname}. When for instance we pass
%D \type{12ptrmtf} as argument, we get \getfontname{12ptrmtf}.

\def\getfontname#1%
  {\csname\??ft#1\endcsname}

%D To be documented.

\let\sizelist\empty

\def\definefontsize[#1]% sneller met toks
  {\addtocommalist{#1}\sizelist
   \def\docommand##1%
     {\def\dodocommand####1%
        {\def\dododocommand########1%
          %{\checkbodyfont{}{########1}{####1}{##1}}%
           {\checkbodyfont{########1}{####1}{##1}}%
         \processcommacommand[\stylelist]\dododocommand}%
      \processcommacommand[\alternativelist]\dodocommand}%
   \processcommacommand[\sizelist]\docommand}

\def\alternativetextlist{\c!tf,\c!bf,\c!it,\c!sl,\c!bs,\c!bi,\c!sc}
\def\alternativemathlist{\c!mr,\c!mi,\c!sy,\c!ex,\c!ma,\c!mb}

\let\alternativelist\alternativetextlist % upward compatible

%\definefontsize[\c!a] \definefontsize[\c!b]
%\definefontsize[\c!c] \definefontsize[\c!d]

%D \macros
%D   {currentfontscale,currentfontbodyscale}
%D
%D Sometimes we need to have access to the font scale
%D including the \type{a}||\type{d} sizes. The next macro
%D returns the current scaling factor. Take a look at
%D \type {cont-log.tex} for an example of its use.

\beginTEX

\def\currentfontscale
  {\@EA\ifx\csname\??ft\s!default\xfontsize\endcsname\relax
     \@EA\ifx\csname\??ft\s!default\s!text\endcsname\relax
       1%
     \else
       \csname\??ft\s!default\s!text\endcsname
     \fi
   \else
     \csname\??ft\s!default\xfontsize\endcsname
   \fi}

\def\currentfontbodyscale
  {\@EA\ifx\csname\??ft\s!default\fontsize\endcsname\relax
     \@EA\ifx\csname\??ft\s!default\s!text\endcsname\relax
       1%
     \else
       \csname\??ft\s!default\s!text\endcsname
     \fi
   \else
     \csname\??ft\s!default\fontsize\endcsname
   \fi}

\endTEX

\beginETEX \ifcsname

\def\currentfontscale
  {\csname\??ft\s!default
     \ifcsname\??ft\s!default\xfontsize\endcsname \xfontsize \else
     \ifcsname\??ft\s!default\s!text   \endcsname \s!text    \fi\fi
   \endcsname}

\def\currentfontbodyscale
  {\csname\??ft\s!default
     \ifcsname\??ft\s!default\fontsize\endcsname  \fontsize  \else
     \ifcsname\??ft\s!default\s!text  \endcsname  \s!text    \fi\fi
   \endcsname}

\endETEX

\setvalue{\??ft\s!default}{1}

%D Both alternatives use \type {\xfontsize}, a macro that
%D expands to the current size in terms of \type {a} \unknown
%D \type {d}, nothing, or \type {x} \unknown \type {xx}.

\def\xfontsize{\ifcase\currentxfontsize\fontsize\or\c!x\else\c!xx\fi}

%D A typical use of this command is in commands that switch
%D to another font bypassing the font mechanism:
%D
%D \starttyping
%D \font\myfont=\truefontname{MyFont} at \currentfontscale\bodyfontsize
%D \stoptyping

%D Now we enter the area of font switching. The switching
%D mechanism has to take care of several situations, like:
%D
%D \startitemize[packed]
%D \item  changing the overal document fonts (including margins,
%D       headers and footers)
%D \item  changing local fonts (only the running text)
%D \item  smaller and even more smaller alternatives (super-
%D       and subscripts)
%D \stopitemize
%D
%D \TEX\ offers a powerfull family mechanism for super- and
%D subscripts in math mode. In text mode however, we don't use
%D families for the smaller alternatives, and therefore have
%D to take care of it otherwise.

%D \macros
%D   {definebodyfontenvironment,setupbodyfontenvironment}
%D
%D The relationship between the several sizes of a font, is
%D defined by:
%D
%D \showsetup{definebodyfontenvironment}
%D
%D Later on we will see how these parameters are used, so for
%D the moment we stick with an example:
%D
%D \starttyping
%D \definebodyfontenvironment
%D   [12pt]
%D   [        text=12pt,
%D          script=9pt,
%D    scriptscript=7pt,
%D               x=10pt,
%D              xx=8pt,
%D             big=12pt,
%D           small=10pt]
%D \stoptyping
%D
%D The first argument specifies the bodyfont size to which the
%D settings apply. All second parameters are specified in
%D dimensions and tell us more about related sizes.
%D
%D Afterwards, one can change values with
%D
%D \showsetup{setupbodyfontenvironment}
%D
%D Due to the fact that \type{\c!text} and \type{\s!text} can
%D have a similar meaning, and therefore can lead to an
%D unwanted loop, we temporary redefine \type{\c!text}. For
%D the moment this in only place that some trickery is needed
%D to fool the multilingual interface.
%D
%D When instead of a size the keyword \type{unknown} is
%D passed, fractions (relations) are used instead of fixed
%D sizes.

\let\bodyfontenvironmentlist\empty

\newcount\@@fontdefhack

\def\@@beginfontdef
  {\ifcase\@@fontdefhack
     \let\k!savedtext     \k!text      \let\k!text      \s!text
     \let\k!k!savedtext   \k!k!text    \let\k!k!text    \!!plusone
     \let\k!saveddefault  \k!default   \let\k!default   \s!default
     \let\k!k!saveddefault\k!k!default \let\k!k!default \!!plusone
   \fi
   \advance\@@fontdefhack \plusone }

\def\@@endfontdef
  {\advance\@@fontdefhack \minusone
   \ifcase\@@fontdefhack
     \let\k!k!default\k!k!saveddefault
     \let\k!default  \k!saveddefault
     \let\k!k!text   \k!k!savedtext
     \let\k!text     \k!savedtext
   \fi}

\def\definebodyfontenvironment
  {\dotripleempty\dodefinebodyfontenvironment}

\def\dodefinebodyfontenvironment[#1][#2][#3]% class size settings
  {\ifthirdargument
     \@@beginfontdef
     \doifelse{#2}\s!default
       {\getparameters[\??ft\s!default][#3]}
       {\normalizebodyfontsize#2\to\tempbodyfontsize
        \addtocommalist\tempbodyfontsize\bodyfontenvironmentlist
        \@EA\dododefinebodyfontenvironment\@EA[\tempbodyfontsize][#1][#3]}%
     \@@endfontdef
   \else
     \definebodyfontenvironment[\fontclass][#1][#2]% change */*
   \fi}

\def\dododefinebodyfontenvironment[#1][#2][#3]% size class settings
  {\@@beginfontdef
   \doifundefined{\??ft#2#1\c!em} % \s!text goes wrong in testing because
     {\def\docommand##1%         % the 12pt alternative will called when
        {\scratchdimen#1\relax    % typesetting the test (or so)
         \scratchdimen\csname\??ft\s!default##1\endcsname\scratchdimen
         \normalizebodyfontsize\scratchdimen\to\tempbodyfontsize
         \setevalue{\??ft#2#1##1}{\tempbodyfontsize}}%
     \processcommacommand[\fontsizelist]\docommand
     \copyparameters
       [\??ft#2#1][\??ft\s!default]
       [\c!interlinespace,\c!em]}%
   \getparameters[\??ft#2#1][#3]%
   \@@endfontdef
   % new code, see remark
   \ifloadingfonts \else            % only runtime
     \doifundefined{\@size@#1}      % only once
       {\letvalue{\@size@#1}\empty  % prevent loop
        \defineunknownfont{#1}}%    % safeguard
   \fi
   % so far
   \setvalue{\@size@#1}{\docompletefontswitch[#1]}}

%D {\bf Remark:} We need to cover the following cases,
%D otherwise users can get confused:
%D
%D \starttyping
%D \setupbodyfont[23pt]
%D
%D \definebodyfontenvironment[23pt]
%D \setupbodyfont[23pt]
%D
%D \definebodyfontenvironment[23pt]
%D \definebodyfont[23pt][rm,ss,tt][default]
%D \setupbodyfont[23pt]
%D \stoptyping

%D Beware: while some font defs can be global, the bodyfont
%D environment checks local. This means that multiple local
%D checks resulting in definitions are not that efficient.
%D So, apart from an occasional switch, one should define an
%D environment at the outer level.

\def\checkbodyfontenvironment[#1]%
  {\definebodyfontenvironment[\fontclass][#1][]}

% original
%
% \def\setupbodyfontenvironment
%   {\dotripleempty\dosetupbodyfontenvironment}
%
% \def\dosetupbodyfontenvironment[#1][#2][#3]% class size settings
%   {\@@beginfontdef
%    \ifthirdargument
%      \getparameters[\??ft#1#2][#3]%
%    \else
%      \getparameters[\??ft#1][#2]%
%    \fi
%    \@@endfontdef}
%
% this one already catches both define/setup

\def\setupbodyfontenvironment{\definebodyfontenvironment}

% officially, but not needed (yet):
%
% \def\dosetupbodyfontenvironment[#1][#2][#3]% class size settings
%   {\ifthirdargument
%      \localbodyfontsize#2\relax
%      \normalizebodyfontsize\localbodyfontsize\to\normalizedbodyfontsize
%      \doifundefinedelse{\??ft#1\normalizedbodyfontsize\c!em}
%        {\definebodyfontenvironment[#1][#2][#3]}%
%        {\getparameters[\??ft#1\normalizedbodyfontsize][#3]}%
%    \else
%      \localbodyfontsize#1\relax
%      \normalizebodyfontsize\localbodyfontsize\to\normalizedbodyfontsize
%      \doifundefinedelse{\??ft\normalizedbodyfontsize\c!em}
%        {\definebodyfontenvironment[#1][#2]}%
%        {\getparameters[\??ft\normalizedbodyfontsize][#2]}%
%    \fi}

%D Just a couple of interface macros:

\beginTEX

\def\bodyfontvariable#1%
  {\??ft\@EA\ifx\csname\??ft\fontclass#1\endcsname\relax\else\fontclass\fi#1}

\endTEX

\beginETEX \ifcsname

\def\bodyfontvariable#1%
  {\??ft\ifcsname\??ft\fontclass#1\endcsname\fontclass\fi#1}

\endETEX

\def\bodyfontcsname
  {\csname\bodyfontvariable}

\def\bodyfontinterlinespecs
  {\bodyfontvariable{\normalizedbodyfontsize\c!interlinespace}}

\def\bodyfontinterlinespace
  {\csname\bodyfontinterlinespecs\endcsname}

%D We default all parameters to the main bodyfont size (begin
%D \type{#1}), so the next setup is valid too:
%D
%D \starttyping
%D \definebodyfontenvironment[24pt]
%D \stoptyping
%D
%D All parameters can be redefined when needed, so one does
%D not have to stick to the default ones.

%D \macros
%D   {definebodyfont}
%D
%D The next step in defining a bodyfont involves the actual font
%D files, which can be recognized by their extension
%D \type{tfm}. Installing those file is often beyond the
%D scope of the user and up to the system administrator.
%D
%D \showsetup{definebodyfont}
%D
%D This commands takes three arguments: a (series of) bodyfont
%D size(s), the style group to which the definitions belong,
%D and an alternative, as specified by the \TEX\ (math) families,
%D extended with~a, b~\unknown.
%D
%D We show two examples, that show all the alternative
%D scaling options. The \type{\tfa} alternatives can be
%D extended with \type{\bfa}, \type{\slb}, etc. or even
%D \type{e} and higher alternatives.
%D
%D \starttyping
%D \definebodyfont [12pt] [rm]
%D   [tf=cmr12,
%D    bf=cmbx12,
%D    it=cmti12,
%D    sl=cmsl12,
%D    bi=cmbxti10 at 12pt,
%D    bs=cmbxsl10 at 12pt,
%D   tfa=cmr12    scaled \magstep1,
%D   tfb=cmr12    scaled \magstep2,
%D   tfc=cmr12    scaled \magstep3,
%D   tfd=cmr12    scaled \magstep4,
%D    sc=cmcsc10  at 12pt]
%D
%D \definebodyfont [12pt,11pt,10pt,9pt,8pt] [rm]
%D   [tf=lbr  sa 1,
%D    bf=lbd  sa 1,
%D    it=lbi  sa 1,
%D    sl=lbsl sa 1,
%D    bi=lbdi sa 1,
%D    bs=lbdi sa 1,
%D   tfa=lbr  sa 1.200,
%D   tfb=lbr  sa 1.440,
%D   tfc=lbr  sa 1.728,
%D   tfd=lbr  sa 2.074,
%D    sc=lbr  sa 0.833]
%D \stoptyping
%D
%D The second example shows that we can define more sizes at
%D once. The main difference between these examples is that the
%D Computer Modern Roman come in many design sizes. This means
%D that there we cannot define them in bulk using \type{sa}.
%D Instead of \type{rm} (roman) one can define \type{ss} (sans
%D serif), \type{tt} (teletype), \type{hw} (hand written),
%D \type{cg} (calygraphic) and whatever styles.
%D
%D The first argument may be a comma separated list. This,
%D combined with specifications using \type{sa} can save a lot
%D of typing. Although all arguments should be specified, we
%D treat the second argument as optional.
%D
%D Defining a bodyfont involves two actions: defining the
%D specific style related alternatives, like \type{\rma},
%D \type{\bfa} and \type{\rmsla}, and storing the definitions
%D of their bodyfont size related fonts. The first step is
%D bodyfont independant but executed every time. This permits
%D user definitions like \type{\tfw} or \type{\bfq} for real
%D large alternatives.

\def\definebodyfont
  {\doquadrupleempty\redefinebodyfont}

\def\redefinebodyfont[#1][#2][#3][#4]%
  {\iffourthargument
     \processcommacommand[#1]{\reredefinebodyfont[#2][#3][#4]}%
   \else
     \dodefinebodyfont[#1][#2][#3]%
   \fi}

\def\reredefinebodyfont[#1][#2][#3]#4%
  {\pushmacro\fontclass
   \doifelse{#4}\s!default
     {\let\fontclass\empty}
     {\def\fontclass{#4}}%
   \definebodyfont[#1][#2][#3]%
   \popmacro\fontclass}

\def\dodefinebodyfont[#1][#2][#3]% body|identifier style defs|identifier
  {\ifthirdargument
     \doifnumberelse{#1}
       {\doifassignmentelse{#3}
          {% [12pt] [style] [settings]
           \doifundefined{#2}{\expanded{\definefontstyle[#2][#2]}}% new
           \processcommalist[#1]{\dododefinebodyfont{#2}{#3}}}
          {% [12pt] [style] [identifier]
           \dodefinedefaultbodyfont[#1][#2][#3]}} % body style identifier
       {% [identifier] [style] [settings] % see ***
        \setvalue{\s!default#1#2}##1##2{\expanded{\xdodefinebodyfont[##1][##2][#3]}}}%
   \else\ifsecondargument
     \definebodyfont[#1][\c!rm][#2]%
   \else
     \doifundefined{\@size@#1} % Maybe there are default dependancies
       {\defineunknownfont{#1}}% defined which we can use ([unknown])
     \doifundefined{\@size@#1} % and if not, then we have at least to
       {\definebodyfont[#1][\c!rm][]}% make sure some basics are set up.
   \fi\fi}

\def\xdodefinebodyfont[#1][#2][#3]% body|identifier style defs|identifier
  {%\writestatus{[#1]}{[#2][#3]}%
   \checkrelativefontsize{#2}% rather new, inherit from other defs
   \ifundefined{#2}\expanded{\definefontstyle[#2][#2]}\fi % new
   \processcommalist[#1]{\dododefinebodyfont{#2}{#3}}%
   \let\relativefontsize\defaultrelativefontsize}

\def\dododefinebodyfont#1#2#3% style defs body
  {\checkbodyfontenvironment[#3]% just to be sure.
   \processcommalist[#2]{\dodododefinebodyfont{#1}{#3}}}

\def\dodododefinebodyfont#1#2#3% style body def
  {\dododododefinebodyfont{#1}{#2}[#3]}

\newif \ifresolvefontfile % protected by fontclass anyway

\def\resolvefontname#1 {\truefontname{#1} }

\def\iflocalclassfonts{\ifx\fontclass\empty}

\def\dododododefinebodyfont#1#2[#3#4#5=#6]% style body def
  {\ifundefined{#1#3#4#5}%
    %\checkbodyfont{#2}{#1}{#3#4}{#5}% not \definefontsize[#5]
     \checkbodyfont{#1}{#3#4}{#5}% not \definefontsize[#5]
   \fi
   \iflocalclassfonts
     \letbeundefined{*\fontclass#2#1#3#4#5*}%
     \scratchtoks{#6}%
     \expanded{\unexpanded\noexpand\setvalue{#2#1#3#4#5}%
       {\noexpand\xxdododefinefont{\relativefontsize}{#2}%
       {#2#1#3#4#5}{\the\scratchtoks}}}%
   \else
     %\expanded{\writestatus{defining}{[\fontclass][#2#1#3#4#5] \resolvefontname#6 }}%
     \global\letbeundefined{*\fontclass#2#1#3#4#5*}%
     \ifresolvefontfile
       \unexpanded\setxvalue{\fontclass#2#1#3#4#5}%
         {\noexpand\xxdododefinefont{\relativefontsize}{#2}%
         {#2#1#3#4#5}{\resolvefontname#6 }}%
     \else
       \scratchtoks{#6}%
       \expanded{\unexpanded\noexpand\setgvalue{\fontclass#2#1#3#4#5}%
         {\noexpand\xxdododefinefont{\relativefontsize}{#2}%
         {#2#1#3#4#5}{\the\scratchtoks}}}%
     \fi
   \fi}

% \def\checkbodyfont#1#2#3#4% body style alt size / gdef % #4 can be empty
%   {\def\c!!mm{#2}%
%    \ifx\c!!mm\c!mm % prevents \max and alike (re)defs
%      \unexpanded\setgvalue     {#2}{\setcurrentfontstyle               {#2}}%         \rm
%      \unexpanded\setgvalue     {#3}{\setcurrentfontalternative         {#3}}%         \sl
%    \else
%      \unexpanded\setgvalue   {#2#4}{\setcurrentfontstylesize           {#2}{#4}}%     \rma
%      \unexpanded\setgvalue   {#3#4}{\setcurrentfontalternativesize     {#3}{#4}}%     \sla
%      \unexpanded\setgvalue {#2#3#4}{\setcurrentfontstylealternativesize{#2}{#3}{#4}}% \rmsla
%      \unexpanded\setgvalue     {#2}{\setcurrentfontstyle               {#2}}%         \rm
%      \unexpanded\setgvalue     {#3}{\setcurrentfontalternative         {#3}}%         \sl
%      \unexpanded\setgvalue {#2\c!x}{\setcurrentfontxstylealternative   {#2}}%         \rmx
%      \unexpanded\setgvalue{#2\c!xx}{\setcurrentfontxxstylealternative  {#2}}%         \rmxx
%      \unexpanded\setgvalue {#3\c!x}{\setcurrentfontxalternative        {#3}}%         \slx
%      \unexpanded\setgvalue{#3\c!xx}{\setcurrentfontxxalternative       {#3}}%         \slxx
%      \unexpanded\setgvalue   {#2#3}{\setcurrentfontstylealternative    {#2}{#3}}%     \rmsl
%    \fi}
%
% leaner

\def\checkbodyfont#1% tests for ttsl mmbf
  {\def\c!!mm{#1}%
   \ifx\c!!mm\c!mm % prevents \max and alike (re)defs
     \expandafter\checkmathbodyfont
   \else
     \expandafter\checktextbodyfont
   \fi{#1}} % no \c!!mm, not expanded later on

\def\checkmathbodyfont#1#2#3% style alt size / gdef % #3 can be empty
  {%\message{!m #1 #2 #3!}%
   % #1 #2 #3 = signal
   \unexpanded\setgvalue {#1#2#3}{\setcurrentfontstylealternativesize{#1}{#2}{#3}}% \mmsla
   \unexpanded\setgvalue     {#1}{\setcurrentfontstyle               {#1}}%         \mm
   \unexpanded\setgvalue     {#2}{\setcurrentfontalternative         {#2}}}%        \sl

\def\checktextbodyfont#1#2#3% style alt size / gdef % #3 can be empty
  {%\message{!t #1 #2 #3!}%
   \unexpanded\setgvalue   {#1#3}{\setcurrentfontstylesize           {#1}{#3}}%     \rma
   \unexpanded\setgvalue   {#2#3}{\setcurrentfontalternativesize     {#2}{#3}}%     \sla
   \unexpanded\setgvalue {#1#2#3}{\setcurrentfontstylealternativesize{#1}{#2}{#3}}% \rmsla
   \unexpanded\setgvalue     {#1}{\setcurrentfontstyle               {#1}}%         \rm
   \unexpanded\setgvalue     {#2}{\setcurrentfontalternative         {#2}}%         \sl
   \unexpanded\setgvalue {#1\c!x}{\setcurrentfontxstylealternative   {#1}}%         \rmx
   \unexpanded\setgvalue{#1\c!xx}{\setcurrentfontxxstylealternative  {#1}}%         \rmxx
   \unexpanded\setgvalue {#2\c!x}{\setcurrentfontxalternative        {#2}}%         \slx
   \unexpanded\setgvalue{#2\c!xx}{\setcurrentfontxxalternative       {#2}}%         \slxx
   \unexpanded\setgvalue   {#1#2}{\setcurrentfontstylealternative    {#1}{#2}}}%    \rmsl

\def\dodefinedefaultbodyfont[#1][#2][#3]% sizes styles identifier
  {\def\dododefinedefaultbodyfont##1%
     {\def\dodododefinedefaultbodyfont####1%
        {\def\dododododefinedefaultbodyfont########1%
           {\doifdefined{\s!default########1####1}
              {% [12pt] [style] [identifier]
               \getvalue{\s!default########1####1}{##1}{####1}}}%
         \processcommalist[#3]\dododododefinedefaultbodyfont}%
      \processcommalist[#2]\dodododefinedefaultbodyfont}%
   \processcommalist[#1]\dododefinedefaultbodyfont}

%D Unknown families are added to the family list! For the
%D moment we also set the direct calls here. Some day a better
%D solution will be implemented. The good news is that unknown
%D fonts are defined automatically.

\newif\ifdefiningunknownfont

% \def\defineunknownfont#1%
%   {\let\c!savedtext\c!text
%    \let\c!text\s!text
%    \donefalse
%    \def\docommand##1%
%      {\doifdefined{\??ft\s!default##1}
%         {\donetrue
%          \scratchdimen#1\relax
%          \scratchdimen\csname\??ft\s!default##1\endcsname\scratchdimen
%          \normalizebodyfontsize\scratchdimen\to\!!stringa
%          \letvalue{\??ft#1##1}\!!stringa}}%
%    \processcommacommand[\fontsizelist]\docommand
%    \let\c!text\c!savedtext
%    \ifdone
%      \donefalse
%      \def\defineunknownbodyfont##1% see ***
%        %{\doifdefined{\s!default##1}
%        %   {\donetrue\getvalue{\s!default##1}{#1}}}%
%        {\doifdefined{\s!default\s!default##1}
%           {\donetrue\getvalue{\s!default\s!default##1}{#1}{##1}}}%
%      \processcommacommand
%        [\stylelist]
%        \defineunknownbodyfont
%      \ifdone
%        \setvalue{\@size@#1}{\docompletefontswitch[#1]}%
%        \ifdefiningunknownfont \else
%          \definingunknownfonttrue
%          \def\defineunknownsubfont##1%
%            {\doifundefined{\@size@\getvalue{\??ft#1##1}}
%               {\defineunknownfont{\getvalue{\??ft#1##1}}}}%
%          \processcommacommand[\fontsizelist]\defineunknownsubfont
%          \definingunknownfontfalse
%        \fi
%      \fi
%      \ifdone
%        \showmessage\m!fonts{14}{#1}%
%      \fi
%    \fi}

\def\dodefineunknownfont#1#2%
  {\doifdefined{\??ft\s!default#2}
     {\donetrue
      \scratchdimen#1\relax
      \scratchdimen\csname\??ft\s!default#2\endcsname\scratchdimen
      \normalizebodyfontsize\scratchdimen\to\!!stringa
      \letvalue{\??ft#1#2}\!!stringa}}

% \def\dodefineunknownbodyfont#1#2% see ***
%   {\doifdefined{\s!default\s!default#2}
%      {\donetrue
%       \getvalue{\s!default\s!default#2}{#1}{#2}}}
%
% but ... it needs to be as follows:
%
% \def\dodefineunknownbodyfont#1#2% see ***
%   {\doifdefined{\s!default\fontclass#2}% was \s!default\s!default (related to change */*)
%      {\donetrue
%       \getvalue{\s!default\fontclass#2}{#1}{#2}}}
%
% eh ... this does not work so we revert back to:

\def\dodefineunknownbodyfont#1#2% see ***
  {\doifdefined{\s!default\s!default#2}% somehow related to */*
     {\donetrue
      \getvalue{\s!default\s!default#2}{#1}{#2}}}

\def\dodefineunknownsubfont#1#2%
  {\doifundefined{\@size@\getvalue{\??ft#1#2}}
     {\donetrue
      \defineunknownfont{\getvalue{\??ft#1#2}}}}

\def\defineunknownfont#1%
  {\let\c!savedtext\c!text
   \let\c!text\s!text
   \donefalse
   \processcommacommand[\fontsizelist]{\dodefineunknownfont{#1}}%
   \let\c!text\c!savedtext
   \ifdone
     \donefalse
     \processcommacommand
       [\stylelist]
       {\dodefineunknownbodyfont{#1}}%
     \ifdone
     \donefalse
       \setvalue{\@size@#1}{\docompletefontswitch[#1]}%
       \ifdefiningunknownfont \else
         \definingunknownfonttrue
         \processcommacommand[\fontsizelist]{\dodefineunknownsubfont{#1}}%
         \definingunknownfontfalse
       \fi
     \fi
     \ifdone
       \showmessage\m!fonts{14}{#1}%
     \fi
   \fi}

% \def\defineunknownfontstyles#1%
%   {\def\defineunknownbodyfont##1% see ***
%      {\executeifdefined{\s!default\s!default##1}\gobbletwoarguments{#1}{##1}}%
%    \rawprocesscommacommand[\stylelist]\defineunknownbodyfont}

%D These macros show that quite some definitions take place.
%D Fonts are not loaded yet! This means that at format
%D generation time, no font files are preloaded.

%D A previous implementation used:
%D
%D \starttyping
%D \type {\setvalue{name}{\donottest...}}
%D \stoptyping
%D
%D instead of the more memory hungry:
%D
%D \starttyping
%D \unexpanded\setvalue{name}{...}
%D \stoptyping
%D
%D The first alternative saves about 500 hash entries (about
%D 2.5\% of the total number of entries used. The second
%D alternative is currently used, because that one can more
%D easily be made \ETEX\ aware.

%D \macros
%D   {everybodyfont,Everybodyfont}
%D
%D Every change in bodyfont size has conseqences for the baseline
%D distance and skips between paragraphs. These are initialized
%D in other modules. Here we only provide the hooks that
%D garantees their handling.

% \ifx\everybodyfont\undefined % permits reloading
%   \newevery \everybodyfont \EveryBodyFont
% \fi

%D At the system level one can initialize thing like:
%D
%D \starttyping
%D \appendtoks \setupspacing \to \everybodyfont
%D \stoptyping
%D
%D While users can add their own non standard commands like:
%D
%D \starttyping
%D \EveryBodyFont{\message{changing to bodyfont \the\bodyfontsize}}
%D \stoptyping
%D
%D Personnaly I never felt the need for such extensions, but
%D at least its possible.

%D \macros
%D   {globalbodyfontsize,localbodyfontsize}
%D
%D Next we'll do the tough job of font switching. Here we
%D have to distinguish between the global (overal) bodyfont
%D size and the local (sometimes in the textflow) size. We
%D store these dimensions in two \DIMENSION\ registers.

\newdimen\globalbodyfontsize  \globalbodyfontsize=12pt
\newdimen\localbodyfontsize   \localbodyfontsize =\globalbodyfontsize

%D \macros
%D   {bodyfontsize}
%D
%D These two registers are not to be misused in calculations.
%D For this purpose we keep a copy:

\newdimen\bodyfontsize  \bodyfontsize=\globalbodyfontsize

%D \macros
%D   {outputresolution}
%D
%D Sometimes (to be honest: not in this module) we need to
%D take the system resolution into account. Therefore we also
%D define a macro:

\def\outputresolution {600}

%D \macros
%D   {bodyfontfactor,bodyfontpoints}
%D
%D For multiplication purposes we keep an auxiliary counter
%D and macro (here the expansion is not explicitly needed):

\newcount\bodyfontpoints \dimensiontocount\bodyfontsize\bodyfontpoints

\edef\bodyfontfactor{\withoutpt\the\bodyfontsize}

%D When we assign for instance 12pt to a \DIMENSION\ register
%D the \type{\the}'d value comes out as 12.0pt, which is
%D often not the way users specify the bodyfont size. Therefore
%D we also store the normalized value.

\chardef\fontdigits=1

% \def\donormalizedbodyfontsize#1.#2#3#4\to#5%
%   {\edef#5%
%      {#1%
%       \ifcase\fontdigits\space
%       \or
%         \ifcase#2 \else.#2\fi % and not: \ifcase#2\else ...
%       \else
%         \ifcase#2#3 \else.#2\ifcase#3 \else#3\fi\fi % not: \ifcase#2#3\else ...
%       \fi
%       pt}}
%
% \def\normalizebodyfontsize#1\to#2%
%   {\scratchdimen#1\relax
%    \@EA\@EA\@EA\donormalizedbodyfontsize
%      \@EA\WITHOUTPT\the\scratchdimen00\to#2}

\def\normalizebodyfontsize#1\to#2%
  {\scratchdimen#1\relax
   \ifcase\fontdigits\advance\scratchdimen.5\points\fi
   \@EA\@EA\@EA\donormalizedbodyfontsize
     \@EA\WITHOUTPT\the\scratchdimen00\to#2}

\def\donormalizedbodyfontsize#1.#2#3#4\to#5% \points ?
  {\edef#5%
     {#1%
      \ifcase\fontdigits\or
        \ifcase#2 \else.#2\fi % and not: \ifcase#2\else ...
      \else
        \ifcase#2#3 \else.#2\ifcase#3 \else#3\fi\fi % not: \ifcase#2#3\else ...
      \fi
      \s!pt}}

\normalizebodyfontsize\bodyfontsize\to\normalizedglobalbodyfontsize
\normalizebodyfontsize\bodyfontsize\to\normalizedlocalbodyfontsize
\normalizebodyfontsize\bodyfontsize\to\normalizedbodyfontsize

%D To be internationalized:

\def\korpsgrootte {\bodyfontsize}
\def\korpspunten  {\bodyfontpoints}

%D some day.

%D \macros
%D   {fontstyle,fontalternative,fontsize}
%D
%D Within a bodyfont, fonts can come in different sizes. For
%D instance \type{\tf} is accompanied by \type{\tfa},
%D \type{\tfb} etc. The first two characters denote the
%D alternative, while the third character in these sequences
%D represents the size. The actual size is saved in a macro
%D
%D The style, being roman (\type{\rm}), sans serif (\type{\ss})
%D etc. is also available in a macro in \type{rm}, \type{ss}
%D etc. form:

\let\defaultfontalternative = \c!tf
\let\defaultfontstyle       = \empty
\let\defaultfontsize        = \empty

\let\fontalternative = \defaultfontalternative
\let\fontstyle       = \defaultfontstyle
\let\fontsize        = \defaultfontsize

%D {\em The following approach is obsolete.}
%D
%D All things related to fonts are grouped into files with
%D names like \type{font-cmr}. These files are loaded by:

\def\resetfontdefinitionfile[#1]%
  {\letbeundefined{\c!file\f!fontprefix#1}}

\newif\ifloadfontfileonce

\def\doreadfontdefinitionfile#1#2% #1 = set/switch state
  {\doifundefined{\c!file\f!fontprefix#2}%
     {\ifloadfontfileonce
        \letvalue{\c!file\f!fontprefix#2}\empty
      \fi
      \makeshortfilename[\truefilename{\f!fontprefix#2}]%
      \startreadingfile
        \readsysfile\shortfilename
          {\showmessage\m!fonts2{#2}}
          {\showmessage\m!fonts3{#2}}%
      \stopreadingfile}}

%D When \type {\loadfontfileoncetrue}, such files are
%D only loaded once! This permits redundant loading, but at
%D the same time forced grouping when we want continuously mix
%D all kind of font, which of course is a kind of
%D typographically sin. The \type{"} is made inactive if
%D needed to prevent problems with loading files that use this
%D character in numbers.

\def\doswitchpoints[#1]%
  {\expanded{\dodoswitchpoints{#1}}}

\def\dodoswitchpoints#1%
  {\doifundefined{\@size@#1}
     {\defineunknownfont{#1}}%
  %\defineunknownfontstyles{#1}%
   \doifdefinedelse{\@size@#1}
     {\getvalue{\@size@#1}%
      \localbodyfontsize#1\relax
      \normalizebodyfontsize\localbodyfontsize\to\normalizedbodyfontsize
      \checkbodyfontenvironment[\normalizedbodyfontsize]}
     {\showmessage\m!fonts4{#1}}}

\unprotected \def\doswitchstyle[#1]%
  {\doifdefinedelse{\@style@#1}
     {\getvalue{\@style@#1}%
      \edef\fontstyle{#1}%
      \ifmmode\mr\fi % in order to be compatible with \rm in math mode
      }% \the\everybodyfont}  % cleaner, in setting size as well as style
     {\showmessage\m!fonts5{#1}}}

%D \TEX\ loads font metric files like \type{cmr10.tfm} and
%D \type{tir.tfm} only once. In \PLAIN\ \TEX\ some font files
%D are {\em preloaded}. This means that the font files are
%D loaded, but not accessible yet by name. This is accomplished
%D by saying:
%D
%D \starttyping
%D \font\preloaded=cmr10 at 11pt
%D \stoptyping
%D
%D and using the name \type{\preloaded} again and again, so
%D fonts are indeed loaded, but unnamed, and therefore
%D unaccessible. In \CONTEXT\ we don't preload fonts, not even
%D the \PLAIN\ \TEX\ ones, although users can access them. Now
%D why is this done?

%D Defining fonts using \type{\definebodyfont} takes time, so we
%D prefer to predefine at least the Computer Modern Roman
%D fonts. However, loading all those fonts at definition time
%D would take both time and space. But even worse, once fonts
%D are loaded into memory, their encoding vector is fixed,
%D which is a handicap when we want to distribute the compact
%D \type{fmt} files. So what we want to do is defining fonts in
%D a way that postpones the loading. We accomplish this by only
%D loading the fonts when we switch to another bodyfont size.
%D Among the other alternatives, such as loading the font at
%D the moment of activation and redefining the activation
%D macro afterwards, this proved to be the most efficient
%D alternative.
%D
%D The next few macros take care of the one exeption on this
%D scheme. When at format generation time we load the default
%D font file, the one that defines the Computer Modern Fonts,
%D we don't want the fonts metrics to end up in the format
%D file, so we temporary prohibit loading. This means that at
%D runtime we have to load the default bodyfont size just before
%D we start typesetting.
%D
%D Therefore we have to signal the font switching macros that
%D we are preloading fonts. As long as the next boolean is,
%D true, no loading is done.

\newif\ifloadingfonts \loadingfontstrue

%D \macros
%D   {preloadfonts}
%D
%D Preloading is only called for once, during the startup
%D sequence of a session. After the loading job is done, the
%D macro relaxes itself and reset the signal.

\def\preloadfonts % never called, needs a clean up
  {\showmessage\m!fonts6{\normalizedbodyfontsize\normalspace\fontstyle}%
   \global\loadingfontsfalse
   \doswitchpoints[\normalizedbodyfontsize]%
   \doswitchstyle[\fontstyle]%
   \the\everybodyfont
   \saveinterlinespace
   \global\let\preloadfonts\relax}

% \prependtoks \preloadfonts \to \everydump % saves .1 s on a DELL P60 - 2GHZ

%D Here comes the main font switching macros. These macros
%D handle changes in size as well as returning to the global
%D bodyfont size.

\def\dosetfont#1#2%  #1 = set/switch state
  {\doifelse{#2}\v!global
     {\restoreglobalbodyfont}
     {\processcommacommand[#2]{\dodosetfont{#1}}% ##1 get also passed
      \ifloadingfonts\else
        \doswitchpoints[\normalizedbodyfontsize]%
        \doswitchstyle[\fontstyle]%
      \fi}%
   \chardef\currentxfontsize\zerocount}

\def\dodosetfont#1#2%  #1 = set/switch state | check fo rempty, else space
  {\doifsomething{#2}{\dododosetfont{#1}{#2}{\showmessage\m!fonts4{#2}}}}

\def\dododosetfont#1#2#3% #1 = set/switch state
  {\doifnumberelse{#2}
     {\dodododosetfont{#1}{#2}{#3}}
     {\doifdefinedelse{\??ft\normalizedbodyfontsize\interfaced{#2}}
        {\edef\fontstep{\bodyfontcsname\normalizedbodyfontsize\interfaced{#2}\endcsname}%
         \expanded{\dodododosetfont{#1}{\fontstep}}{#3}}
        {\doifelse{#2}\v!reset
           {\let\fontstyle\empty % new 31/7/2006
            \let\fontsize \empty}
           {\doifdefinedelse{\@style@#2}
              {\edef\fontstyle{#2}}
              {\doreadfontdefinitionfile{#1}{#2}}}}}}

\def\dodododosetfont#1#2#3% #1 = set/switch state
  {\scratchdimen#2\relax
   \normalizebodyfontsize\scratchdimen\to\normalizedsetfont
   \doifundefined{\@size@\normalizedsetfont}
     {\defineunknownfont{#2}}%
   \doifdefinedelse{\@size@\normalizedsetfont}
     {\localbodyfontsize\normalizedsetfont
      \let\normalizedbodyfontsize\normalizedsetfont}
     {#3\dosetsubstitutefont{#1}{#2}}}

%D In the previous macros we use \type{\currentxfontsize} to
%D hold the current x||size of the font. This enables us to
%D support for instance \type{\sl} inside a \type{\tx} switch.

\chardef\currentxfontsize=0

%D When users specify for instance a 13 point bodyfont while no
%D such bodyfont is defined, the system automatically tries to
%D find a best fit, that is the nearest smaller defined
%D bodyfontzize. A smaller one is definitely better than a larger
%D one, simply because otherwise a lot of overfull box messages
%D are more probable to occur. By taking a value slightly
%D smaller than half a point, we can use the next method.

\def\dosetsubstitutefont#1#2% #1 = set/switch state
  {\scratchdimen#2\relax
   \advance\scratchdimen .499\points
   \dimensiontocount\scratchdimen\scratchcounter
   \advance\scratchcounter \minusone
   \ifnum\scratchcounter>\plusthree
     \dododosetfont{#1}{\the\scratchcounter\s!pt}{}%
   \fi}

% The following bunch of macros deals with the (run time)
% expansion of names onto the definitions made by \type
% {\definebodyfont}.

% \let\fontbody       \empty % ... 10pt 11pt 12pt ...
% \let\fontstyle      \empty % rm ss tt mm hw cg ...
% \let\fontalternative\empty % tf bf sl it bs bi sc ...
% \let\fontsize       \empty % xy-abcd ...

\def\defaultfontbody{\normalizedbodyfontsize}

\let\fontbody\defaultfontbody

\let\fontclass\empty \let\globalfontclass\fontclass

\def\setcurrentfontclass#1%
  {\edef\fontclass{#1}}

\let\defaultfontstyle       \c!rm
\let\defaultfontalternative \c!tf
\let\defaultfontsize        \empty

%D \macros
%D   {bigmath,nobigmath}
%D
%D We can inhibit this slow||downer with:

% these can best be combined

% 0=never 1=everymath 2=always

\chardef\synchronizebigmathflag=1

\appendtoks
  \ifcase\synchronizebigmathflag
    % never
  \or
    \synchronizebigmath
  \or
    % always
  \fi
\to \everymathematics

\def\nobigmath{\chardef\synchronizebigmathflag 0 }
\def\bigmath  {\chardef\synchronizebigmathflag 2 \synchronizebigmath}

\let\bigmathfontsize\empty

\def\synchronizebigmath
  {\ifx\bigmathfontsize\fontsize
     % already in sync
   \else
     \let\bigmathfontsize\fontsize
     \synchronizemath \synchronizetext
   \fi}

\def\checkbigmathsynchronization
  {\ifcase\synchronizebigmathflag
     % never
   \or
     \ifmmode \synchronizebigmath \fi
   \or
     \synchronizebigmath
   \fi}

%D So far for synchronisation.

\def\dosetcurrentfontsize#1%
  {\edef\fontsize{#1}%
   \checkbigmathsynchronization}

\def\dosetcurrentfontalternative#1%
  {\edef\fontalternative{#1}%
   \ifmmode % maybe no test, or actually, an option
     \fam\csname\fontalternative\s!fam\endcsname
   \fi}

\def\setcurrentfont#1#2#3#4%
  {%\message{[1 #1 #2 #3 #4]}%
   \edef\fontbody{#1}%
   \edef\fontstyle{#2}%
   \dosetcurrentfontalternative{#3}%
   \dosetcurrentfontsize{#4}%
   \synchronizefont}

\def\setcurrentfontbody#1%
  {%\message{[2 #1]}%
   \edef\fontbody{#1}%
   \synchronizefont}

% \def\setcurrentfontstyle#1%
%   {%\message{[3 #1]}%
%    \edef\fontstyle{#1}%
%    \ifmmode\mr\fi % otherwise \rm not downward compatible
%    \synchronizefont}
%
% For Taco: optional fall backs:

\ifx\checkfontclass\undefined \let\checkfontclass\gobbleoneargument \fi % implemented in type-ini

\def\setcurrentfontstyle#1%
  {%\message{[3 #1]}%
   \checkfontclass{#1}%
   \edef\fontstyle{#1}%
   \ifmmode\mr\fi % otherwise \rm not downward compatible
   \synchronizefont}

\def\setcurrentfontbodyalternative#1#2%
  {%\message{[4 #1 #2]}%
   \edef\fontbody{#1}%
   \dosetcurrentfontalternative{#2}%
   \synchronizefont}

\def\setcurrentfontalternative#1%
  {%\message{[5 #1]}%
   \dosetcurrentfontalternative{#1}%
   \synchronizefont}

\def\setcurrentfontsize#1%
  {%\message{[6 #1]}%
   \dosetcurrentfontsize{#1}%
   \synchronizefont}

\def\setcurrentfontstylealternative#1#2% \rmsl
  {%\message{[7 #1 #2]}%
   \edef\fontstyle{#1}%
   \dosetcurrentfontalternative{#2}%
   \synchronizefont}

\def\setcurrentfontstylesize#1#2% \rmsla
  {%\message{[8 #1 #2]}%
   \edef\fontstyle{#1}%
   \dosetcurrentfontsize{#2}%
   \synchronizefont}

\def\setcurrentfontalternativesize#1#2% \sla
  {%\message{[9 #1 #2]}%
   \dosetcurrentfontalternative{#1}%
   \dosetcurrentfontsize{#2}%
   \synchronizefont}

\def\setcurrentfontstylealternativesize#1#2#3% \rmsla
  {%\message{[10 #1 #2 #3]}%
   \edef\fontstyle{#1}%
   \dosetcurrentfontalternative{#2}%
   \dosetcurrentfontsize{#3}%
   \synchronizefont}

%D In principle one can assign alternative fallback routines.
%D Some day we will.

\newtoks\fontstrategies
\newif\iftryingfont

% \def\synchronizefont
%   {\tryingfonttrue
%    \ifx\fontclass\empty\else
%      \let\fontstrategy\dofontclassstrategy
%      \the\fontstrategies \relax % \relax still needed ?
%    \fi
%    \iftryingfont
%      \let\fontstrategy\dofontstrategy
%      \the\fontstrategies \relax % \relax still needed ?
%    \fi
%    \ifskipfontcharacteristics
%      \setfontcharacteristics
%      \the\everyfontswitch
%    \fi}

\let\fontstrategy\relax

\def\synchronizefont
  {\tryingfonttrue
   \ifx\fontclass\empty\else
     \global\let\fontstrategy\dofontclassstrategy
     \the\fontstrategies \relax % \relax still needed ?
   \fi
   \iftryingfont
     \global\let\fontstrategy\dofontstrategy
     \the\fontstrategies \relax % \relax still needed ?
   \fi
   \ifskipfontcharacteristics
     \setfontcharacteristics
     \the\everyfontswitch
   \fi}

\beginTEX

\def\dofontstrategy#1#2#3#4#5%
  {\@EA\ifx\csname#2#3#4#5\endcsname\relax\else
     #1\csname#2#3#4#5\endcsname \tryingfontfalse
   \fi}

\def\dofontclassstrategy#1#2#3#4#5%
  {\@EA\ifx\csname\fontclass#2#3#4#5\endcsname\relax\else
     #1\csname\fontclass#2#3#4#5\endcsname \tryingfontfalse
   \fi}

\endTEX

\beginETEX \ifcsname

\def\dofontstrategy#1#2#3#4#5%
  {\ifcsname#2#3#4#5\endcsname
     #1\csname#2#3#4#5\endcsname \tryingfontfalse
   \fi}

\def\dofontclassstrategy#1#2#3#4#5%
  {\ifcsname\fontclass#2#3#4#5\endcsname
     #1\csname\fontclass#2#3#4#5\endcsname \tryingfontfalse
   \fi}

\endETEX

\appendtoks \iftryingfont \fontstrategy \autofontsizefalse % --- --- --- --- % pt tt bf a
  \fontbody \fontstyle \fontalternative \fontsize
\fi \to \fontstrategies

\appendtoks \iftryingfont \fontstrategy \autofontsizetrue  % --- --- --- def % pt tt bf
  \fontbody \fontstyle \fontalternative \defaultfontsize
\fi \to \fontstrategies

\appendtoks \iftryingfont \fontstrategy \autofontsizetrue  % --- --- def --- % pt tt tf a
  \fontbody \fontstyle \defaultfontalternative \fontsize
\fi \to \fontstrategies

\appendtoks \iftryingfont \fontstrategy \autofontsizetrue  % --- --- def def % pt tt tf
  \fontbody \fontstyle \defaultfontalternative \defaultfontsize
\fi \to \fontstrategies

\appendtoks \iftryingfont \fontstrategy \autofontsizefalse % --- def def def % pt rm tf
  \fontbody \defaultfontstyle \defaultfontalternative \defaultfontsize
\fi \to \fontstrategies

\appendtoks \iftryingfont \fontstrategy \autofontsizetrue  % def def def def % rm tf
  \defaultfontbody \defaultfontstyle \defaultfontalternative \defaultfontsize
\fi \to \fontstrategies

%D Let's synchronize:

\newif\ifsynchronizefonts \synchronizefontstrue

\prependtoks
  \ifsynchronizefonts
    \synchronizemath \synchronizetext \synchronizefont
  \fi
\to \everybodyfont

%D Setting the normal sized as well as the x and xx smaller
%D sizes is accomplished by the next set of macros. When in
%D math mode, the commands \type{\tx} and \type{\txx} are
%D just a switch to the script and double script styles, but
%D in text mode the values defined by the bodyfontenvironment are
%D used. Here we also set \type{\currentxfontsize}.

\beginTEX

\def\dosetcurrentfontxxxalternative#1#2#3#4%
  {\chardef\currentxfontsize#2\relax
   \ifmmode
     #4%
   \else\expandafter\ifx\csname\bodyfontvariable{\normalizedbodyfontsize#3}\endcsname\relax\else
     \setcurrentfontbodyalternative{\bodyfontcsname\normalizedbodyfontsize#3\endcsname}{#1}%
   \fi\fi}

\endTEX

\beginETEX \ifcsname

\def\dosetcurrentfontxxxalternative#1#2#3#4%
  {\chardef\currentxfontsize#2\relax
   \ifmmode
     #4%
   \else\ifcsname\bodyfontvariable{\normalizedbodyfontsize#3}\endcsname
     \setcurrentfontbodyalternative{\bodyfontcsname\normalizedbodyfontsize#3\endcsname}{#1}%
   \fi\fi}

\endETEX

\def\setcurrentfontxalternative#1%
  {\dosetcurrentfontxxxalternative{#1}1\c!x\scriptstyle
   \let\tx\txx}

\def\setcurrentfontxxalternative#1%
  {\dosetcurrentfontxxxalternative{#1}2\c!xx\scriptscriptstyle
   \let\tx\empty
   \let\txx\empty}

\def\checknestedxfontsize % option
  {\ifcase\currentxfontsize\else\ifx\fontsize\empty\else
     \chardef\currentxfontsize\zeropoint
     \let\fontsize\empty
     \let\tx\normaltx
     \let\txx\normaltxx
   \fi\fi}

\def\setcurrentfontxalternative#1%
  {\checknestedxfontsize
   \dosetcurrentfontxxxalternative{#1}1\c!x\scriptstyle
   \let\tx\txx}

\def\setcurrentfontxxalternative#1%
  {\checknestedxfontsize
   \dosetcurrentfontxxxalternative{#1}2\c!xx\scriptscriptstyle
   \let\tx\empty
   \let\txx\empty}

% This alterative is not really needed, but for old time's sake
% we keep it there. We can speed it up when needed.

% \def\setcurrentfontxstylealternative #1{\csname#1\endcsname\tfx}
% \def\setcurrentfontxxstylealternative#1{\csname#1\endcsname\tfxx}

\def\setcurrentfontxstylealternative #1{\csname#1\endcsname\tx}
\def\setcurrentfontxxstylealternative#1{\csname#1\endcsname\txx}

%D These macros also show us that when we call for \type{\tx},
%D this macro is redefined to be \type{\txx}. Therefore calls
%D like:
%D
%D \startbuffer
%D {small \tx  is \tx  beautiful}
%D {small \tx  is \txx beautiful}
%D {small \txx is \tx  beautiful}
%D {small \txx is \txx beautiful}
%D \stopbuffer
%D
%D \typebuffer
%D
%D result in:
%D
%D \startvoorbeeld
%D \startlines
%D \getbuffer
%D \stoplines
%D \stopvoorbeeld
%D
%D Setting the main size involves the style list and therefore
%D takes a bit more time. Keep in mind that the fontsize is
%D represented by a character or empty.

\unexpanded\def\tx {\setcurrentfontxalternative \fontalternative}
\unexpanded\def\txx{\setcurrentfontxxalternative\fontalternative}

\let\normaltx \tx
\let\normaltxx\txx

%D \macros
%D   {definefontstyle}
%D
%D When setting of switching the overall style we can use the
%D short identifier like rm and ss, but when defined we can
%D also use more verbose names like roman or sansserif. Such
%D names are defined by:
%D
%D \starttyping
%D \definefontstyle [serif,rm] [rm]
%D \definefontstyle [sansserif,ss] [ss]
%D \stoptyping

\def\dodefinefontstyle[#1][#2]%
  {\rawdoifinsetelse{#2}{\stylelist}
     {}%\debuggerinfo\m!fonts{unknown style #2}}
     {\addtocommalist{#2}\stylelist
      \showmessage\m!fonts8{#2 (#1)}}%
   % check kan hier
   \def\docommand##1%
     {\setvalue{\@shortstyle@##1}{#2}%
      \setvalue{\@style@##1}{\csname#2\endcsname}}%
   \processcommalist[#1]\docommand}

\def\definefontstyle
  {\dodoubleargument\dodefinefontstyle}

\def\setfontstyle#1#2%  #1:name (roman, romaan)  #2:style (rm)
  {\edef\fontstyle{#1}%
   \setcurrentfontstyle\normalizedbodyfontsize}

\chardef\defaultskewcharmi=127 % '177
\chardef\defaultskewcharsy= 48 % '60

% \def\dosetskewchar#1%
%   {\skewchar\font\ifx\@@fontskewchar\empty#1\else\@@fontskewchar\fi}

\def\dosetskewchar#1#2%
  {\ifx\@@fontskewchar\empty
     \skewchar\textfont        #1#2%
     \skewchar\scriptfont      #1#2%
     \skewchar\scriptscriptfont#1#2%
   \else
     \skewchar\textfont        #1\@@fontskewchar
     \skewchar\scriptfont      #1\@@fontskewchar
     \skewchar\scriptscriptfont#1\@@fontskewchar
   \fi}

%D The previous macros show that it's is not always
%D neccessary to define the whole bunch of fonts, take for
%D instance the sequence:
%D
%D \starttyping
%D \setupbodyfont
%D   [ams]
%D
%D \definebodyfont [24pt] [mm]
%D   [ma=msam10 at 24pt,
%D    mb=msbm10 at 24pt]
%D
%D \switchtobodyfont
%D   [24pt]
%D
%D This is a 24pt $\blacktriangleleft$
%D \stoptyping
%D
%D Here we didn't define the 24 point bodyfont environment, so
%D it's defined automatically. Of course one can always use the
%D \TEX\ primitive \type{\font} to switch to whatever font
%D needed.

%D When asking for a complete font switch, for instance from 10
%D to 12~points, the next macro does the job. First we
%D normalize the size, next we define the current range of
%D text, script and scriptscript sizes, then we set the text
%D fonts and the math families and finally we activate the
%D default typeface and also set the font specific parameters
%D assigned to \type{\everybodyfont}

\def\dosetbodyfontface#1#2%
  {\edef#1{\bodyfontcsname\normalizedbodyfontsize#2\endcsname}}

\def\docompletefontswitch[#1]%
  {\bodyfontsize#1\relax
   \dimensiontocount\bodyfontsize\bodyfontpoints
   \edef\bodyfontfactor{\withoutpt\the\bodyfontsize}%
   \normalizebodyfontsize\bodyfontsize\to\normalizedbodyfontsize
   \dosetbodyfontface \textface         \s!text
   \dosetbodyfontface \scriptface       \s!script
   \dosetbodyfontface \scriptscriptface \s!scriptscript}

\docompletefontswitch[12pt] % init

%D \macros
%D   {setupbodyfont,switchtobodyfont}
%D
%D The next two macros are user ones. With \type{\setupbodyfont}
%D one can set the document bodyfont size, font family, style
%D and/or options defined in files, for example:
%D
%D \starttyping
%D \setupbodyfont[cmr,ams,12pt,roman]
%D \stoptyping
%D
%D This command affects the document as a whole: text, headers
%D and footers. The second macro however affects only the text:
%D
%D \starttyping
%D \switchtobodyfont[10pt]
%D \stoptyping
%D
%D So we've got:
%D
%D \showsetup{setupbodyfont}
%D \showsetup{switchtobodyfont}
%D
%D Both macros look alike. The second one also has to take
%D all kind of keywords into account.

\ifx\saveinterlinespace   \undefined \let\saveinterlinespace   \relax \fi
\ifx\restoreinterlinespace\undefined \let\restoreinterlinespace\relax \fi

\chardef\bodyfontsetstate=0

\definecomplexorsimple\setupbodyfont

\def\simplesetupbodyfont
  {\restoreglobalbodyfont
   \saveinterlinespace}

\def\complexsetupbodyfont[#1]%
  {\doifsomething{#1}
     {\dosetfont1{#1}%
      \globalbodyfontsize\localbodyfontsize
      \normalizebodyfontsize\globalbodyfontsize\to\normalizedglobalbodyfontsize
      \let\globalfontstyle\fontstyle
      \ifloadingfonts\else
        \the\everybodyfont % indeed needed in case \dosetfont is not executed
        \saveinterlinespace
      \fi}}

\unexpanded\def\switchtobodyfont[#1]%
  {\doifsomething{#1}
     {\doifdefinedelse{\??ft\normalizedbodyfontsize\interfaced{#1}}
        {\setbodyfontstep{#1}} % so we have a fast [small] switch
        {\dosetfont0{#1}}%
      \the\everybodyfont}} % indeed needed in case nothing is executed

%D The following alternative is meant for math||to||text
%D switching and will be optimized.

\beginTEX

\def\fastswitchtobodyfont#1%
  {\@EA\ifx\csname\??ft\normalizedbodyfontsize#1\endcsname\else
     \edef\futurebodyfontsize
       {\csname\??ft\normalizedbodyfontsize#1\endcsname}%
     \@EA\ifx\csname\@size@\futurebodyfontsize\endcsname\else
       \csname\@size@\futurebodyfontsize\endcsname
       \localbodyfontsize\futurebodyfontsize\relax
     \fi
   \fi
   \csname\@style@\fontstyle\endcsname
   \the\everybodyfont}

\endTEX

\beginETEX \ifcsname

\def\fastswitchtobodyfont#1%
  {\ifcsname\??ft\normalizedbodyfontsize#1\endcsname
     \edef\futurebodyfontsize
       {\csname\??ft\normalizedbodyfontsize#1\endcsname}%
     \ifcsname\@size@\futurebodyfontsize\endcsname
       \csname\@size@\futurebodyfontsize\endcsname
       \localbodyfontsize\futurebodyfontsize\relax
     \fi
   \fi
   \csname\@style@\fontstyle\endcsname
   \the\everybodyfont}

\endETEX

%D Because the last macro can appear in arguments or be assigned
%D to parameters, we protect this one for unwanted expansion.

\def\dodosetmathfont#1%
  {\setcurrentfontalternative{#1}%
  % \doifdefinedelse{#1\s!fam} % adapted
  %   {\edef\mffam{\getvalue{#1\s!fam}}}
  %   {\edef\mffam{\getvalue{\c!nn\s!fam}}}%
   \textfont        \mrfam\textfont        \mffam
   \scriptfont      \mrfam\scriptfont      \mffam
   \scriptscriptfont\mrfam\scriptscriptfont\mffam}

\beginTEX

  \def\domffam#1%
    {\csname\ifundefined{#1\s!fam}\c!nn\else#1\fi\s!fam\endcsname}

\endTEX

\beginETEX \ifcsname

  \def\domffam#1%
    {\csname\ifcsname#1\s!fam\endcsname#1\else\c!nn\fi\s!fam\endcsname}

\endETEX

\def\mffam
  {\domffam\fontalternative}

\def\dosetmathfont
  {\def\rm{\fam\mrfam}\dodosetmathfont}

\def\enableencodinginmath
  {\appendtoks
     \everyhbox{\mr\everyhbox\emptytoks}%
     \everyvbox{\mr\everyvbox\emptytoks}%
   \to \everymathematics} % was \everymath

% \enableencodinginmath % too untested to enable by default

%D \startyping
%D $\cases{& \ccaron}$ $x=\hbox{\ccaron $x=\hbox{\ccaron}$}$
%D \stoptyping

%D The font specific features are bound to the filename.

\def\updatefontparameters
  {\edef\@@fontencoding{\truefontdata\fontfile    \s!encoding}%
   \edef\@@fontmapping {\truefontdata\fontfile    \s!mapping }%
   \edef\@@fonthandling{\truefontdata\somefontname\s!handling}%
   \edef\@@fontskewchar{\truefontdata\fontfile    \s!skewchar}}

\def\setfontcharacteristics
  {\updatefontparameters
   \fastenableencoding
     {\ifx\@@fontencoding\empty
        \s!default \else \@@fontencoding
      \fi}%
   \fastenablemapping
     {\ifx\@@fontmapping\empty
        \ifx\@@fontencoding\empty
          \s!default \else \@@fontencoding
        \fi
      \else
        \@@fontmapping
      \fi}%
   \fastenablehandling
     {\ifx\@@fonthandling\empty
        \s!default \else \@@fonthandling
      \fi}%
     {\lastfontidentifier}%
   \the\everyfont
   \synchronizepatterns}

%D The next auxilliary macro is an alternative to \type
%D {\fontname}.

% \def\purefontname#1{\expandafter\splitoffpurefontname\fontname#1 \\}
%
% extra level is needed:

\def\purefontname#1{\@EA\splitoffpurefontname\@EA{\@EA{\@EA\unstringed\fontname#1}} \\}

\def\splitoffpurefontname#1 #2\\{#1}

%D \macros
%D   {switchstyleonly}
%D
%D For switching a style but keeping the alternative, there
%D is:
%D
%D \starttyping
%D {\bf text \switchstyleonly\ss text}
%D {\bf text \switchstyleonly[ss]text}
%D {\sl text \switchstyleonly[sansserif]text}
%D \stoptyping

\definecomplexorsimple\switchstyleonly

\def\simpleswitchstyleonly#1% stupid version
  {\complexswitchstyleonly[\strippedcsname#1]}

\def\complexswitchstyleonly[#1]% todo : check
  {\setcurrentfontstyle{\getvalue{\@shortstyle@#1}}%
   \the\everybodyfont} % needed ?

%D \macros
%D   {os,frak, goth, cal}
%D
%D Old style numerals can be typeset with \type{\os} and look
%D like {\os 1234567890} instead of the more common looking
%D 1234567890.
%D
%D On behalf of {\frac Tobias Burnus}, we define some more of
%D these. Later we will link these names to real file names.

% older
%
% \definefont [os]   [OldStyle     sa *]
% \definefont [frak] [Fraktur      sa *]
% \definefont [goth] [Gothic       sa *]
% \definefont [cal]  [Calligraphic sa *]
% \definefont [bbd]  [Blackboard   sa *]
%
% newer

\def\os  {\mathortext{\fam\purefamily    {oldstyle}}{\symbolicfont    {OldStyle}}}
\def\frak{\mathortext{\fam\purefamily     {fraktur}}{\symbolicfont     {Fraktur}}}
\def\goth{\mathortext{\fam\purefamily      {gothic}}{\symbolicfont      {Gothic}}}
\def\cal {\mathortext{\fam\purefamily{calligraphic}}{\symbolicfont{Calligraphic}}}
\def\bbd {\mathortext{\fam\purefamily  {blackboard}}{\symbolicfont  {Blackboard}}}

\definefontsynonym [OldStyle]     [Serif]
\definefontsynonym [Fraktur]      [Serif]
\definefontsynonym [Gothic]       [Serif]
\definefontsynonym [Calligraphic] [Serif]
\definefontsynonym [Blackboard]   [Serif]

%D \macros
%D   {fraktur, gothic, calligraphic, blackboard}
%D
%D These macros assume that we use text fonts, and not math
%D families.

\ifx\mathtext\undefined \let\mathtext\hbox \fi

\def\fraktur     #1{\mathortext\domathtext\donothing{\frak#1}}
\def\gothic      #1{\mathortext\domathtext\donothing{\goth#1}}
\def\calligraphic#1{\mathortext\domathtext\donothing{\cal #1}}
\def\blackboard  #1{\mathortext\domathtext\donothing{\bbd#1}}

%D Torture test:
%D
%D \starttyping
%D \usetypescript[modern]  [texnansi]
%D \usetypescript[lucida]  [texnansi]
%D \usetypescript[palatino][texnansi]
%D \usetypescript[times]   [texnansi]
%D \usetypescript[fourier] [ec]
%D
%D \startbuffer
%D \section{\blackboard{T\high{\blackboard{T}}} \blackboard{E}\high{\blackboard{E}} \blackboard{X}\high{\blackboard{X}}}
%D
%D {\fontclass: 123 \os123 \cal TEX $\os 123$}
%D
%D $\blackboard{T}^{\blackboard{T}} \blackboard{E}^{\blackboard{E}} \blackboard{X}^{\blackboard{X}}$
%D \blackboard{T}\high{\blackboard{T}} \blackboard{E}\high{\blackboard{E}} \blackboard{X}\high{\blackboard{X}}
%D \stopbuffer
%D
%D {\setupbodyfont[lucida]   \getbuffer}
%D {\setupbodyfont[modern]   \getbuffer}
%D {\setupbodyfont[palatino] \getbuffer}
%D {\setupbodyfont[times]    \getbuffer}
%D {\setupbodyfont[fourier]  \getbuffer}
%D \stoptyping

%D \macros
%D   {definebodyfontswitch}
%D
%D \PLAIN\ \TEX\ defines some macro's like \type{\tenpoint}
%D to switch to a specific bodyfontsize. Just for the sake of
%D compatibility we can define them like:
%D
%D \starttyping
%D \definebodyfontswitch [twelvepoint] [12pt]
%D \stoptyping
%D
%D We don't support language specific synonyms here, mainly
%D because \PLAIN\ \TEX\ is english anyway.

\def\dodefinebodyfontswitch[#1][#2]%
  {\def\docommand##1{\setvalue{##1}{\switchtobodyfont[#2]}}%
   \processcommalist[#1]\docommand}

\def\definebodyfontswitch
  {\dodoubleargument\dodefinebodyfontswitch}

%D \macros
%D   {setsmallbodyfont,setmainbodyfont,setbigbodyfont}
%D
%D When we're typesetting at for instance 10pt, we can call for
%D the \type{small} as well as the \type{big} alternative,
%D related to this main size, using \type{\switchtobodyfont[small]}.
%D The three alternatives can be activated by the next three
%D system calls and are defined by the bodyfontenvironment.

\let\fontstep\empty % we can use \fontstep for tracing purposes

\def\setbodyfontstep#1%
  {\edef\fontstep{\bodyfontcsname\normalizedbodyfontsize\interfaced{#1}\endcsname}%
   \doswitchpoints[\fontstep]%
   \doswitchstyle[\fontstyle]}

\unexpanded\def\setsmallbodyfont{\setbodyfontstep\v!small\the\everybodyfont}
\unexpanded\def\setbigbodyfont  {\setbodyfontstep\v!big  \the\everybodyfont}

\unexpanded\def\setmainbodyfont
  {\doswitchpoints[\normalizedbodyfontsize]%
   \doswitchstyle[\fontstyle]%
   \the\everybodyfont
   \saveinterlinespace}

%D \macros
%D   {restoreglobalbodyfont}
%D
%D Users can set whatever font available while typesetting text.
%D Pagenumbers, footers, headers etc. however must be typeset
%D in the main bodyfont and style of the document. Returning to
%D the global state can be done with the next macro:

\let\mainfontclass\empty

\def\fullrestoreglobalbodyfont
  {\let\fontsize\defaultfontsize
   \let\fontbody\defaultfontbody
   \chardef\currentxfontsize\zerocount
   \let\fontclass\globalfontclass
   \doswitchpoints[\normalizedglobalbodyfontsize]%
   \doswitchstyle[\globalfontstyle]%
   \redoconvertfont % just in case a pagebreak occurs
   \tf \the\everybodyfont
   \saveinterlinespace}

\def\partialrestoreglobalbodyfont
  {\let\fontsize\defaultfontsize
   \let\fontbody\defaultfontbody
   \chardef\currentxfontsize\zerocount
   \redoconvertfont
   \tf
   \saveinterlinespace}

\def\restoreglobalbodyfont % ook style etc
  {\ifx\fontclass\globalfontclass
     \ifx\fontstyle\globalfontstyle
       \ifx\normalizedbodyfontsize\normalizedglobalbodyfontsize
         \partialrestoreglobalbodyfont
       \else
         \fullrestoreglobalbodyfont
       \fi
     \else
       \fullrestoreglobalbodyfont
     \fi
   \else
     \fullrestoreglobalbodyfont
   \fi}

% in case of troubles: \let\restorebodyfont\fullrestoreglobalbodyfont

%D This macro has to be called when entering the pagebody
%D handling routine as well as the footnote insert routine.
%D Users can access this feature |<|for instance when one wants
%D to typeset tables and alike in the main bodyfont and style
%D while the running text is temporary set to a smaller one|>|
%D by saying \type{\switchtobodyfont[global]}.

%D \macros
%D   {rasterfont}
%D
%D There are (at the moment) two situations in which we want to
%D have fast access to a particular font. When we are using
%D \TEX\ to typeset rasters, we use small {\rasterfont.}'s
%D (a rather small period indeed), the same as \PICTEX\ uses
%D for drawing purposes.

\definefont [rasterfont] [Serif at 5pt]

%D \macros
%D   {infofont}
%D
%D The second situation occurs when we enable the info mode,
%D and put all kind of status information in the margin. We
%D don't want huge switches to the main bodyfont and style, so
%D here too we use a direct method.

% old mechanism
%
% \def\infofont{\getvalue{\!!sixpoint\c!tt\c!tf}\the\everybodyfont}

\let\infofont\relax % satisfy dep checker

\definefont [infofont] [Mono at 6pt] % todo \the\everybodyfont

%D \macros
%D   {definealternativestyle}
%D
%D In the main modules we are going to implement lots of
%D parameterized commands and one of these parameters will
%D concern the font to use. To suit consistent use of fonts we
%D here implement a mechanism for defining the keywords that
%D present a particular style or alternative.
%D
%D \starttyping
%D \definealternativestyle [keywords] [\style] [\nostyle]
%D \stoptyping
%D
%D The first command is used in the normal textflow, while the
%D second command takes care of headings and alike. Consider
%D the next two definitions:
%D
%D \starttyping
%D \definealternativestyle [bold] [\bf]  []
%D \definealternativestyle [cap]  [\cap] [\cap]
%D \stoptyping
%D
%D A change \type{\bf} in a heading which is to be set in
%D \type{\tfd} does not look that well, so therefore we leave
%D the second argument of \type{\definealternativestyle} empty.
%D When we capatalize characters using the pseudo small cap
%D command \type{\cap}, we want this to take effect in both
%D text and headings, which is accomplished by assigning both
%D arguments.

\def\dodefinealternativestyle[#1][#2][#3]%
  {\def\docommand##1%
     {\doifundefined{##1}{\setvalue{##1}{\groupedcommand{#2}{}}}%
      \setvalue{\@letter@  ##1}{#2}%
      \setvalue{\@noletter@##1}{#3}}%
   \processcommalist[#1]\docommand}

\def\definealternativestyle
  {\dotripleempty\dodefinealternativestyle}

\def\definestyle{\definealternativestyle}

%D Maybe too geneneric, but probably ok is the following. (Maybe one
%D day we will use a dedicated grouped command for styles.)

\appendtoks
  \let\groupedcommand\thirdofthreearguments
\to \simplifiedcommands

%D This command also defines the keyword as command. This means
%D that the example definition of \type{bold} we gave before,
%D results in a command \type{\bold} which can be used as:
%D
%D \startbuffer
%D He's a \bold{bold} man with a {\bold head}.
%D \stopbuffer
%D
%D \typebuffer
%D
%D or
%D
%D \startvoorbeeld
%D \definealternativestyle[bold][\bf][]\getbuffer
%D \stopvoorbeeld
%D
%D Such definitions are of course unwanted for \type{\cap}
%D because this would result in an endless recursive call.
%D Therefore we check on the existance of both the command and
%D the substitution. The latter is needed because for instance
%D \type{\type} is an entirely diferent command. That command
%D handles verbatim, while the style command would just switch
%D to teletype font. This is just an example of a tricky
%D naming coincidence.

%D \macros
%D   {doconvertfont,noconvertfont,
%D    dontconvertfont,redoconvertfont}
%D
%D After having defined such keywords, we can call for them by
%D using
%D
%D \starttyping
%D \doconvertfont{keyword}{text}
%D \stoptyping
%D
%D We deliberately pass an argument. This enables us to
%D assign converters that handle one agrument, like
%D \type{\cap}.
%D
%D By default the first specification is used to set the style,
%D exept when we say \type{\dontconvertfont}, after which the
%D second specification is used. We can also directly call for
%D \type{\noconvertfont}. In nested calls, we can restore the
%D conversion by saying \type{\redoconvertfont}.

\def\@@dodoconvertfont#1{\csname\@letter@  #1\endcsname}
\def\@@donoconvertfont#1{\csname\@noletter@#1\endcsname}

\unexpanded\def\dodoconvertfont#1% #2% we need the protection
  {\doifdefinedelse{\@letter@#1}     % in testing
     {\doifelsenothing{#1}\gobbleoneargument\@@dodoconvertfont}
     {\doifdefinedelse{#1}\getvalue         \firstofoneargument}%
   {#1}} % {#2}}

\let\doconvertfont\dodoconvertfont

\def\noconvertfont#1% #2%
  {\doifdefinedelse{\@noletter@#1}
     {\doifelsenothing{#1}\gobbleoneargument\@@donoconvertfont}\gobbleoneargument
   {#1}} % {#2}}

%D Extras:

\unexpanded\def\dontconvertfont
  {\let\doconvertfont\noconvertfont}

\unexpanded\def\redoconvertfont
  {\let\doconvertfont\dodoconvertfont}

%D These commands are not grouped! Grouping is most probably
%D done by the calling macro's and would lead to unnecessary
%D overhead.

%D \macros
%D  {em,emphasistypeface,emphasisboldface}
%D
%D The next macro started as a copy of Donald Arseneau's
%D \type{\em} (\TUGNEWS\ Vol.~3, no.~1, 1994). His
%D implementation was a bit more sophisticated version of the
%D standard \LATEX\ one. We furter enhanced the macro, so now
%D it also adapts itself to boldface mode. Because we favor
%D {\sl slanted} type over {\it italic}, we made the emphasis
%D adaptable, for instance:
%D
%D \starttyping
%D \def\emphasistypeface {\it}
%D \def\emphasisboldface {\bi}
%D \stoptyping
%D
%D But we prefer:

\def\emphasistypeface {\sl}
\def\emphasisboldface {\bs}

%D or even better:

\def\doemphasistypeface#1#2%
  {\doifelsevalue{\??ft\fontclass\normalizedbodyfontsize\c!em}\v!slanted#1%
  {\doifelsevalue{\??ft\fontclass\normalizedbodyfontsize\c!em}\v!italic#2%
  {\doifelsevalue{\??ft          \normalizedbodyfontsize\c!em}\v!slanted#1%
  {\doifvalue    {\??ft          \normalizedbodyfontsize\c!em}\v!italic#2}}}}

\def\emphasistypeface{\doemphasistypeface\sl\it}
\def\emphasisboldface{\doemphasistypeface\bs\bi}

%D To be set with the default body font environment: \type
%D {em} being \type {slanted} or \type {italic}.

\newconditional\emneeded

\unexpanded\def\em
  {\relax
   \ifdim\slantperpoint>\zeropoint
     \settrue\emneeded
   \else
     \setfalse\emneeded
   \fi
   \setemphasisboldface % new
   \ifx\fontalternative\c!it      % \ifnum\fam=\itfam
     \def\emphasistypeface{\it}\tf
   \else\ifx\fontalternative\c!sl % \ifnum\fam=\slfam
     \def\emphasistypeface{\sl}\tf
   \else\ifx\fontalternative\c!bf % \ifnum\fam=\bffam
     \emphasisboldface
   \else\ifx\fontalternative\c!bs % \ifnum\fam=\bsfam
     \def\emphasisboldface{\bs}\bf
   \else\ifx\fontalternative\c!bi % \ifnum\fam=\bifam
     \def\emphasisboldface{\bi}\bf
   \else
     \emphasistypeface
   \fi\fi\fi\fi\fi
   \ifconditional\emneeded\relax
   \else
     \expandafter\aftergroup
   \fi
   \emphasiscorrection}

%D The next feature was not present in previous versions. It
%D takes care of \type {\em \bf ...} sitiations.

\def\setemphasisboldface
  {\let\normalbf\bf
   \let\setemphasisboldface\relax
   \unexpanded\def\bf
     {%\relax
      \let\bf\relax % new
      \ifx\fontalternative\c!it      % \ifnum\fam=\itfam
        \bi
      \else\ifx\fontalternative\c!sl % \ifnum\fam=\slfam
        \bs
      \else
        \normalbf
      \fi\fi
      \let\bf\normalbf}}

%D Donald's (adapted) macros take the next character into
%D account when placing italic correction. As a bonus we also
%D look for something that looks like a dash, in which case we
%D don't correct.

\let\italiccorrection=\/

\def\emphasiscorrection
  {\ifhmode
     \expandafter\emphasislook
   \fi}

\def\emphasislook
  {\begingroup
   \beginrobusttest
   \futurelet\next\emphasistest}

\def\emphasistest
  {\normalifcat\noexpand\next,%
     \endrobusttest\expandafter\doemphasiscorrection
   \normalelse
     \endrobusttest\expandafter\dododoemphasiscorrection
   \normalfi}

\def\doemphasiscorrection
  {\futurelet\next\dodoemphasiscorrection}

\def\dodoemphasiscorrection
   {\setbox\scratchbox\hbox{\next}%
    \ifdim\ht\scratchbox=\zeropoint % probably a space
      \expandafter\dododoemphasiscorrection
    \else\ifdim\ht\scratchbox<.3ex
      \expandafter\expandafter\expandafter\endgroup
    \else
      \expandafter\expandafter\expandafter\dododoemphasiscorrection
    \fi\fi}

\def\dododoemphasiscorrection
  {\scratchskip\lastskip
   \ifdim\scratchskip=\zeropoint\relax % == \ifzeropt\scratchskip
     \italiccorrection\relax
   \else
     \unskip\italiccorrection\hskip\scratchskip
   \fi
   \endgroup}

%D We end with some examples which show the behavior when
%D some punctuation is met. We also show how the mechanism
%D adapts itself to bold, italic and slanted typing.
%D
%D \startbuffer
%D test {test}test       \par
%D test {\sl test}test   \par
%D test {\em test}test   \par
%D test {\em test}--test \par
%D
%D test {test}, test     \par
%D test {\em test}, test \par
%D
%D test {\em test {\em test {\em test} test} test} test \par
%D test {\bf test {\em test {\em test} test} test} test \par
%D test {\sl test {\em test {\em test} test} test} test \par
%D test {\it test {\em test {\em test} test} test} test \par
%D \stopbuffer
%D
%D \typebuffer
%D
%D We get:
%D
%D \startvoorbeeld
%D \startpacked
%D \getbuffer
%D \stoppacked
%D \stopvoorbeeld

%D \macros
%D   {emphbf,emphit,emphsl,emphtf}
%D
%D The next emphasis alternatives are for \THANH. They adapt
%D their style as good as possible.

\def\emphbf{\groupedcommand{\bf\def\emphit{\bi}\def\emphsl{\bs}}{}}
\def\emphit{\groupedcommand{\it\def\emphbf{\bi}\def\emphsl{\sl}}{}}
\def\emphsl{\groupedcommand{\sl\def\emphbf{\bs}\def\emphit{\it}}{}}
\def\emphtf{\groupedcommand{\tf\def\emphbf{\bf}\def\emphit{\it}\def\emphsl{\sl}}{}}

%D \startbuffer
%D TEXT {\emphbf text \emphit text  \emphtf text  \emphsl text}  TEXT
%D TEXT  \emphbf{text \emphit{text} \emphtf{text} \emphsl{text}} TEXT
%D \stopbuffer
%D
%D \typebuffer
%D \startlines
%D \getbuffer
%D \stoplines

%D \macros
%D   {setfont}
%D
%D Every now and then we want to define a font directly, for
%D instance when we typeset title pages. The next macro saves
%D some typing:

\def\setfont% geen \dosetfont mogelijk
  {\def\next{\nextfont\setupinterlinespace}%
   \afterassignment\next\font\nextfont=}

%D One can call this macro as:
%D
%D \starttyping
%D \setfont cmr10 at 60pt
%D \stoptyping
%D
%D After which the font is active and the baselines and
%D struts are set.

%D \macros
%D   {showbodyfont}
%D
%D One can call for a rather simple overview of a bodyfont and the
%D relations between its alternative fonts.
%D
%D \showsetup{showbodyfont}
%D
%D The current bodyfont (here we omitted the argument) looks like:
%D
%D \showbodyfont
%D
%D The implementation is rather straightforward in using
%D \type{\halign}.

\fetchruntimecommand \showbodyfont  {\f!fontprefix\s!run}

%D \macros
%D   {showfontstrip, testminimalbaseline, showminimalbaseline}
%D
%D The next command can come in handy when combining
%D different fonts into a collection (typeface) and
%D determining optimal baseline distances.
%D
%D \showfontstrip \blank \showminimalbaseline

\fetchruntimecommand \showfontstrip       {\f!fontprefix\s!run}
\fetchruntimecommand \testminimalbaseline {\f!fontprefix\s!run}
\fetchruntimecommand \showminimalbaseline {\f!fontprefix\s!run}

%D \macros
%D   {showkerning}
%D
%D A goody is:
%D
%D \showkerning{Can you guess what kerning is?}

\fetchruntimecommand \showkerning         {\f!fontprefix\s!run}

%D \macros
%D   {showbodyfontenvironment}
%D
%D The current bodyfontenvironment is:
%D
%D \showbodyfontenvironment
%D
%D This overview is generated using:
%D
%D \showsetup{showbodyfontenvironment}

\fetchruntimecommand \showbodyfontenvironment {\f!fontprefix\s!run}

%D \macros
%D   {showfont,showfontstyle,showligatures}
%D
%D The following command generates a fontmap:
%D
%D \startbuffer
%D \showfont[SansBold at 12pt]
%D \stopbuffer
%D
%D \typebuffer
%D
%D Below the table the name, encoding, mapping and handling are
%D shown. Special characters like the \type {\skewchar} and
%D \type {\hyphenchar} als marked.
%D
%D \getbuffer

% to be internationalized

\fetchruntimecommand \showfont      {\f!fontprefix\s!run}
\fetchruntimecommand \showfontstyle {\f!fontprefix\s!run}
\fetchruntimecommand \showligature  {\f!fontprefix\s!run}
\fetchruntimecommand \showligatures {\f!fontprefix\s!run}
\fetchruntimecommand \showcharratio {\f!fontprefix\s!run}

%D \macros
%D   {getglyph, symbolicfont}
%D
%D Individual glyphs can be accessed by using
%D
%D \starttyping
%D \getglyph{fontname}{character}
%D \stoptyping
%D
%D This macro is used in for instance the symbol modules and
%D as one can see, it does obey the small and even smaller
%D sizes. The \type {\symbolicfont} macro can be used to
%D switch to a font named \type {fontname} (see \type
%D {cont-log} and \type {symb-eur} for examples of symbolic
%D definitions.

\def\fontstringA
  {\ifx\fontstyle\c!rm \s!Serif \else
   \ifx\fontstyle\c!ss \s!Sans  \else
   \ifx\fontstyle\c!tt \s!Mono  \else
                       \s!Serif \fi\fi\fi}

\def\fontstringB
  {\ifx\fontstyle\c!rm \s!Regular \else
   \ifx\fontstyle\c!ss \s!Support \else
   \ifx\fontstyle\c!tt \s!Type    \else
                       \s!Serif   \fi\fi\fi}

\def\fontstringC
  {\ifx\fontalternative\c!bf \s!Bold        \else
   \ifx\fontalternative\c!sl \s!Slanted     \else
   \ifx\fontalternative\c!it \s!Italic      \else
   \ifx\fontalternative\c!bs \s!BoldSlanted \else
   \ifx\fontalternative\c!bi \s!BoldItalic  \fi\fi\fi\fi\fi}

\def\fontstringD % default fontstyle
  {\expandafter\ifx\csname\??tf\fontclass\s!default\endcsname\c!rm \s!Serif \else
   \expandafter\ifx\csname\??tf\fontclass\s!default\endcsname\c!ss \s!Sans  \else
   \expandafter\ifx\csname\??tf\fontclass\s!default\endcsname\c!tt \s!Mono  \else
                                                                   \s!Serif \fi\fi\fi}

% potential generalization:

% \letvalue{\??ff:t:\c!rm}\s!Serif
% \letvalue{\??ff:t:\c!ss}\s!Sans
% \letvalue{\??ff:t:\c!tt}\s!Mono
%
% \letvalue{\??ff:s:\c!bf}\s!Bold
% \letvalue{\??ff:s:\c!sl}\s!Slanted
% \letvalue{\??ff:s:\c!it}\s!Italic
% \letvalue{\??ff:s:\c!bs}\s!BoldSlanted
% \letvalue{\??ff:s:\c!bi}\s!BoldItalic
%
% \letvalue{\??ff:a:\c!rm}\s!Regular
% \letvalue{\??ff:a:\c!ss}\s!Support
% \letvalue{\??ff:a:\c!tt}\s!Type
%
% \def\fontstringA{\executeifdefined{\??ff:t:\fontstyle}\s!Serif}
% \def\fontstringB{\executeifdefined{\??ff:a:\fontstyle}\s!Serif}
% \def\fontstringC{\executeifdefined{\??ff:s:\fontstyle}\empty}
% \def\fontstringD{\executeifdefined{\??ff:t:\csname\??tf\fontclass\s!default\endcsname}\s!Serif}

\beginETEX \ifcsname

\def\glyphfontfile#1%
  {#1%
   \ifcsname\??ff#1\fontstringA\fontstringC\endcsname
     \fontstringA\fontstringC
   \else\ifcsname\??ff#1\fontstringB\fontstringC\endcsname
     \fontstringB\fontstringC
   \else\ifcsname\??ff#1\fontstringA\endcsname
     \fontstringA
   \else\ifcsname\??ff#1\fontstringB\endcsname
     \fontstringB
   \else\ifcsname\??ff#1\fontstringC\endcsname
     \fontstringC
   \fi\fi\fi\fi\fi}

\endETEX

\beginTEX

\def\glyphfontfile#1%
  {#1%
   \expandafter\ifx\csname\??ff#1\fontstringA\fontstringC\endcsname\relax
     \expandafter\ifx\csname\??ff#1\fontstringB\fontstringC\endcsname\relax
       \expandafter\ifx\csname\??ff#1\fontstringA\endcsname\relax
         \expandafter\ifx\csname\??ff#1\fontstringB\endcsname\relax
           \expandafter\ifx\csname\??ff#1\fontstringC\endcsname\relax
           \else \fontstringC             \fi
           \else \fontstringB             \fi
           \else \fontstringA             \fi
           \else \fontstringB\fontstringC \fi
           \else \fontstringA\fontstringC \fi}

\endTEX

%D The next macro can be used to make decisions based on the shape:

\def\doifitalicelse#1#2%
  {\ifx\fontalternative\c!sl#1\else
   \ifx\fontalternative\c!it#1\else
   \ifx\fontalternative\c!bs#1\else
   \ifx\fontalternative\c!bi#1\else#2\fi\fi\fi\fi}

%D For an example of usage of the following command,
%D see \type {cont-log.tex}.
%D
%D \starttyping
%D \def\symbolicfont#1%
%D   {\definedfont[\glyphfontfile{#1} sa *]}
%D \stoptyping
%D
%D Since we know what scaling it to be applied, we can
%D implement a much faster alternative:

\let\thedefinedfont\relax

% \def\symbolicscaledfont#1#2%
%   {\scaledfont\fontbody
%    \scaledfont#1\scaledfont
%    \font\thedefinedfont\truefontname{\glyphfontfile{#2}} at
%      \currentfontbodyscale\scaledfont
%    \thedefinedfont}
%
% \unexpanded\def\symbolicfont
%   {\symbolicscaledfont\plusone}
%
% even more control (needed for mthsqrt etc)

\def\symbolicsizedfont#1#2#3%
  {\scaledfont#1%
   \scaledfont#2\scaledfont
   \font\thedefinedfont\truefontname{\glyphfontfile{#3}} at
     \currentfontbodyscale\scaledfont
   \thedefinedfont}

\def\symbolicscaledfont
  {\symbolicsizedfont\fontbody}

\unexpanded\def\symbolicfont
  {\symbolicsizedfont\fontbody\plusone}

\unexpanded\def\getglyph#1#2% slow, faster, much faster
 %{{\definefont[\s!dummy][\glyphfontfile{#1} sa \currentfontscale]\dummy#2}}
 %{{\definefont[\s!dummy][\glyphfontfile{#1} sa *]\dummy#2}}
 %{{\symbolicfont{#1}#2}}
  {{\symbolicfont{#1}\doifnumberelse{#2}\char\donothing#2}}

\unexpanded\def\getscaledglyph#1#2#3%
  {{\symbolicscaledfont{#1}{#2}\doifnumberelse{#3}\char\donothing#3}}

\unexpanded\def\getrawglyph#1#2% for simple symbols
  {{\scaledfont\fontbody
    \font\thedefinedfont=#1 at \currentfontbodyscale\scaledfont
    \thedefinedfont\doifnumberelse{#2}\char\donothing#2}}

%D The last implementation of \type {\getglyph} permits
%D definitions like:
%D
%D \starttyping
%D \definefontsynonym [EuroSans]            [eurose]
%D \definefontsynonym [EuroSansBold]        [euroseb]
%D \definefontsynonym [EuroSansItalic]      [eurosei]
%D \definefontsynonym [EuroSansSlanted]     [eurosei]
%D \definefontsynonym [EuroSansBoldItalic]  [eurosebi]
%D \definefontsynonym [EuroSansBoldSlanted] [eurosebi]
%D
%D \definesymbol [euro] [\getglyph{Euro}{\char160}]
%D
%D \def\euro{\symbol[euro]}
%D \stoptyping
%D
%D These definitions guarantee that the next calls work okay:
%D
%D \starttyping
%D \ss \tf\euro \bf\euro \sla\euro \itd\euro \bs\euro \bic\euro
%D \stoptyping
%D
%D The shape as well as the size is adapted to the current
%D environment.

%D Fonts can only be used when loaded. In \CONTEXT\ we
%D postpone the loading of fonts, even when we load \PLAIN.
%D This means that we have to redefine one of the \PLAIN\
%D macros. Let's tell that to the user first:

\writestatus{loading}{Postponed Plain TeX Font Definitions}

%D \macros
%D   {bordermatrix}
%D
%D In \PLAIN\ \TEX\ the width of a parenthesis is stored in
%D the \DIMENSION\ \type{\p@renwd}. This value is derived from
%D the width of \type{\tenrm B}, so let's take care of it now:

\let\normalbordermatrix=\bordermatrix

\def\bordermatrix%
  {\bgroup
   \setbox0\hbox{\getvalue{\textface\c!mm\c!ex}B}%
   \global\p@renwd\wd0\relax
   \egroup
   \normalbordermatrix}

%D Because we want to be as \PLAIN\ compatible as possible, we
%D make most of \PLAIN's font mechanisme available to the
%D \CONTEXT\ user.

\def\setplainfonts#1#2%
  {\setvalue  {ten#1}{\getvalue{\!!tenpoint  #2}}%
   \setvalue{seven#1}{\getvalue{\!!sevenpoint#2}}%
   \setvalue {five#1}{\getvalue{\!!fivepoint #2}}}

\setplainfonts {\c!rm} {\c!rm\c!tf}
\setplainfonts {\c!bf} {\c!rm\c!bf}
\setplainfonts {\c!sl} {\c!rm\c!sl}
\setplainfonts {\c!it} {\c!rm\c!it}
\setplainfonts {\c!tt} {\c!rm\c!tt}
\setplainfonts {\c!sy} {\c!mm\c!sy}
\setplainfonts {\c!ex} {\c!mm\c!ex}
\setplainfonts {\c!i}  {\c!mm\c!mi}

\let\setplainfonts=\undefined

%D \macros
%D   {ss, SS, sz}
%D
%D We are going to redefine \type{\ss} but for those wo still
%D want to have access to the german \SS, we save it's value in
%D \type{\SS}. Ok, I should have used \type{\sf} instead of
%D \type{\ss} in the first place.

\ifx\undefined\SS \let\SS=\ss \fi
\ifx\undefined\sz \let\sz=\ss \fi

%D \macros
%D   {xi}
%D
%D We are going to redefine \type{\xi}, but fortunately this
%D is a math mode character, so we can just say:

\let\normalxi=\xi

%D \macros
%D   {smashaccent}
%D
%D When we let \TEX\ put an accent on top of a character, such
%D composed characters can get more height that height of a
%D standard \type{\strut}. The next macro takes care of such
%D unwanted compositions.
%D
%D We need to reach over the number that specifies the accent,
%D and in doing so we use \type{\scratchcounter} as a placeholder
%D because it accepts 8 bit numbers in octal, decimal or
%D hexadecimal format. Next we set the height of the accented
%D character to the natural height of the character.

\unexpanded\def\smashaccent#1%
  {\dontleavehmode
   \bgroup
   \setbox\scratchbox\hbox{#1}%
   \ifdim\ht\scratchbox>\strutheight\relax\ht\scratchbox\strutheight\fi
   \ifdim\dp\scratchbox>\strutdepth \relax\dp\scratchbox\strutdepth \fi
   \box\scratchbox
   \egroup}

%D For instance we can say:
%D
%D \starttyping
%D \smashaccent{\"Uberhaupt}
%D \stoptyping
%D
%D But normally one will use it as a prefix in definitions.
%D The difference is in the height:
%D
%D \leavevmode\ruledhbox
%D   {\ruledhbox{\smashaccent{\"U}berhaupt}\quad
%D    oder\quad
%D    \ruledhbox{\"Uberhaupt}}

%D \macros
%D   {moveaccent}
%D
%D Exact positioning of accents can be realized by saying:
%D
%D \starttyping
%D \moveaccent{-.1ex}{\"u}berhaupt
%D \stoptyping
%D
%D Again, this one will mostly used as a prefix in definitions.
%D Here the difference is in the position:
%D
%D \leavevmode\ruledhbox
%D   {\ruledhbox{\moveaccent{-.1ex}{\"}Uberhaupt}\quad
%D    oder\quad
%D    \ruledhbox{\"Uberhaupt}}

\unexpanded\def\moveaccent#1#2%
  {\smashaccent
     {\dimen0\exheight
      \dimen2\dimen0
      \advance\dimen2 -#1%
      \exheight\dimen2
      #2\relax
      \exheight\dimen0}}

%D Personally I think that using \TEX\ is complicated by the
%D way fonts are handled. Apart from the many encodings, we
%D also deal with different naming schemes. Confronted with
%D this problem, I decided to change the definitions into:
%D
%D \starttyping
%D \definebodyfont [12pt] [rm] [tf=Times-Roman at 12pt]
%D \stoptyping
%D
%D combined with for instance:
%D
%D \starttyping
%D \definefontsynonym [Times-Roman] [tir]
%D \stoptyping

%D Now we're up to some definitions.

\definebodyfontenvironment
  [\s!default]
  [        \s!text=1.0,
         \s!script=0.7,
   \s!scriptscript=0.5,
              \c!a=1.200,
              \c!b=1.440,
              \c!c=1.728,
              \c!d=2.074,
                 *=\currentfontscale, % wildcard
              \c!x=0.8,
             \c!xx=0.6,
            \c!big=1.2,
          \c!small=0.8,
 \c!interlinespace=,
             \c!em=\v!slanted]

\definebodyfontenvironment
  [20.7pt]
  [        \s!text=20.7pt,
         \s!script=\!!fourteenpointfour,
   \s!scriptscript=\!!twelvepoint,
              \c!x=17.3pt,
             \c!xx=\!!fourteenpointfour,
            \c!big=20.7pt,    % !!!!
          \c!small=17.3pt]

\definebodyfontenvironment
  [17.3pt]
  [        \s!text=17.3pt,
         \s!script=\!!twelvepoint,
   \s!scriptscript=\!!tenpoint,
              \c!x=\!!fourteenpointfour,
             \c!xx=\!!twelvepoint,
            \c!big=20.7pt,
          \c!small=\!!fourteenpointfour]

\definebodyfontenvironment
  [\!!fourteenpointfour]
  [        \s!text=\!!fourteenpointfour,
         \s!script=\!!elevenpoint,
   \s!scriptscript=\!!ninepoint,
              \c!x=\!!twelvepoint,
             \c!xx=\!!tenpoint,
            \c!big=17.3pt,
          \c!small=\!!twelvepoint]

\definebodyfontenvironment
  [\!!twelvepoint]
  [        \s!text=\!!twelvepoint,
         \s!script=\!!ninepoint,
   \s!scriptscript=\!!sevenpoint,
              \c!x=\!!tenpoint,
             \c!xx=\!!eightpoint,
            \c!big=\!!fourteenpointfour,
          \c!small=\!!tenpoint]

\definebodyfontenvironment
  [\!!elevenpoint]
  [        \s!text=\!!elevenpoint,
         \s!script=\!!eightpoint,
   \s!scriptscript=\!!sixpoint,
              \c!x=\!!ninepoint,
             \c!xx=\!!sevenpoint,
            \c!big=\!!twelvepoint,
          \c!small=\!!ninepoint]

\definebodyfontenvironment
  [\!!tenpoint]
  [        \s!text=\!!tenpoint,
         \s!script=\!!sevenpoint,
   \s!scriptscript=\!!fivepoint,
              \c!x=\!!eightpoint,
             \c!xx=\!!sixpoint,
            \c!big=\!!twelvepoint,
          \c!small=\!!eightpoint]

\definebodyfontenvironment
  [\!!ninepoint]
  [        \s!text=\!!ninepoint,
         \s!script=\!!sevenpoint,
   \s!scriptscript=\!!fivepoint,
              \c!x=\!!sevenpoint,
             \c!xx=\!!fivepoint,
            \c!big=\!!elevenpoint,
          \c!small=\!!sevenpoint]

\definebodyfontenvironment
  [\!!eightpoint]
  [        \s!text=\!!eightpoint,
         \s!script=\!!sixpoint,
   \s!scriptscript=\!!fivepoint,
              \c!x=\!!sixpoint,
             \c!xx=\!!fivepoint,
            \c!big=\!!tenpoint,
          \c!small=\!!sixpoint]

\definebodyfontenvironment
  [\!!sevenpoint]
  [        \s!text=\!!sevenpoint,
         \s!script=\!!sixpoint,
   \s!scriptscript=\!!fivepoint,
              \c!x=\!!sixpoint,
             \c!xx=\!!fivepoint,
            \c!big=\!!ninepoint,
          \c!small=\!!fivepoint]

\definebodyfontenvironment
  [\!!sixpoint]
  [        \s!text=\!!sixpoint,
         \s!script=\!!fivepoint,
   \s!scriptscript=\!!fivepoint,
              \c!x=\!!fivepoint,
             \c!xx=\!!fivepoint,
            \c!big=\!!eightpoint,
          \c!small=\!!fivepoint]

\definebodyfontenvironment
  [\!!fivepoint]
  [        \s!text=\!!fivepoint,
         \s!script=\!!fivepoint,
   \s!scriptscript=\!!fivepoint,
              \c!x=\!!fivepoint,
             \c!xx=\!!fivepoint,
            \c!big=\!!sevenpoint,
          \c!small=\!!fivepoint]

\definebodyfontenvironment
  [\!!fourpoint]
  [        \s!text=\!!fourpoint,
         \s!script=\!!fourpoint,
   \s!scriptscript=\!!fourpoint,
              \c!x=\!!fourpoint,
             \c!xx=\!!fourpoint,
            \c!big=\!!sixpoint,
          \c!small=\!!fourpoint]

\definebodyfontswitch [fourteenpointfour] [\!!fourteenpointfour]
\definebodyfontswitch [twelvepoint]       [\!!twelvepoint]
\definebodyfontswitch [elevenpoint]       [\!!elevenpoint]
\definebodyfontswitch [tenpoint]          [\!!tenpoint]
\definebodyfontswitch [ninepoint]         [\!!ninepoint]
\definebodyfontswitch [eightpoint]        [\!!eightpoint]
\definebodyfontswitch [sevenpoint]        [\!!sevenpoint]
\definebodyfontswitch [sixpoint]          [\!!sixpoint]
\definebodyfontswitch [fivepoint]         [\!!fivepoint]
\definebodyfontswitch [fourpoint]         [\!!fourpoint]

\definebodyfontswitch [xii]  [\!!twelvepoint]
\definebodyfontswitch [xi]   [\!!elevenpoint]
\definebodyfontswitch [x]    [\!!tenpoint]
\definebodyfontswitch [ix]   [\!!ninepoint]
\definebodyfontswitch [viii] [\!!eightpoint]
\definebodyfontswitch [vii]  [\!!sevenpoint]
\definebodyfontswitch [vi]   [\!!sixpoint]

%D So far.

\definefontstyle [\c!rm,\v!roman,\v!serif,\v!regular]    [\c!rm]
\definefontstyle [\c!ss,\v!sansserif,\v!sans,\v!support] [\c!ss]
\definefontstyle [\c!tt,\v!teletype,\v!type,\v!mono]     [\c!tt]
\definefontstyle [\c!hw,\v!handwritten]                  [\c!hw]
\definefontstyle [\c!cg,\v!calligraphic]                 [\c!cg]

\definefontsize[\c!a] \definefontsize[\c!b]
\definefontsize[\c!c] \definefontsize[\c!d]

\definealternativestyle [\v!mediaeval]                            [\os]  []
\definealternativestyle [\v!normal]                               [\tf]  []
\definealternativestyle [\v!bold]                                 [\bf]  []
\definealternativestyle [\v!type]                                 [\tt]  []
\definealternativestyle [\v!mono]                                 [\tt]  []
\definealternativestyle [\v!slanted]                              [\sl]  []
\definealternativestyle [\v!italic]                               [\it]  []
\definealternativestyle [\v!boldslanted,\v!slantedbold]           [\bs]  []
\definealternativestyle [\v!bolditalic,\v!italicbold]             [\bi]  []
\definealternativestyle [\v!small,\v!smallnormal]                 [\tfx] []
\definealternativestyle [\v!smallbold]                            [\bfx] []
\definealternativestyle [\v!smalltype]                            [\ttx] []
\definealternativestyle [\v!smallslanted]                         [\slx] []
\definealternativestyle [\v!smallboldslanted,\v!smallslantedbold] [\bsx] []
\definealternativestyle [\v!smallbolditalic,\v!smallitalicbold]   [\bix] []

%D Slow but handy:

\definealternativestyle [\v!smallbodyfont] [\setsmallbodyfont] []
\definealternativestyle [\v!bigbodyfont]   [\setbigbodyfont]   []

%D We treat {\sc Small Caps} and \cap {Pseudo Caps} a bit
%D different. We also provide an \WORD {uppercase} style.

\definealternativestyle [\v!cap,\v!capital] [\smallcapped] [\smallcapped]
\definealternativestyle [\v!smallcaps]      [\sc]          [\sc]
\definealternativestyle [\v!WORD]           [\WORD]        [\WORD]

%D \macros
%D   {...math}
%D
%D New or old?

% tzt proper \define...
%
% watch out: \synchronizesymb resets the family so we need a second
% \mf (or maybe \mr): messy and to be sorted out

\def\tfmath{\tf\mf\synchronizesymb\mf}
\def\bfmath{\bf\mf\synchronizesymb\mf}
\def\slmath{\sl\mf\synchronizesymb\mf}
\def\itmath{\it\mf\synchronizesymb\mf}
\def\bsmath{\bs\mf\synchronizesymb\mf}
\def\bimath{\bi\mf\synchronizesymb\mf}
\def\scmath{\sc\mf\synchronizesymb\mf}
\def\nnmath{\nn\mf\synchronizesymb\mf}

\def\textmath    {\synchronizesymb}

%D \macros
%D   {fontstylesuffix}
%D
%D The next macro is used to map non latin fontnames on
%D fonts. See \type {font-uni} for an example of its use.

%\def\fontstylesuffix%
%  {\ifnum\fam=\tffam     \s!Regular \else
%   \ifnum\fam=\bffam        \s!Bold \else
%   \ifnum\fam=\slfam     \s!Slanted \else
%   \ifnum\fam=\itfam      \s!Italic \else
%   \ifnum\fam=\bsfam \s!BoldSlanted \else
%   \ifnum\fam=\bifam  \s!BoldItalic \else
%                         \s!Regular \fi\fi\fi\fi\fi\fi}%

\def\fontstylesuffix% why the \s!Regular ? see \getglyph
  {\ifx\fontalternative\c!tf     \s!Regular \else
   \ifx\fontalternative\c!bf        \s!Bold \else
   \ifx\fontalternative\c!sl     \s!Slanted \else
   \ifx\fontalternative\c!it      \s!Italic \else
   \ifx\fontalternative\c!bs \s!BoldSlanted \else
   \ifx\fontalternative\c!bi  \s!BoldItalic \else
   \ifx\fontalternative\c!sc        \s!Caps \else
                                 \s!Regular \fi\fi\fi\fi\fi\fi\fi}%

%D We still have to take care of \type{\xi}, so:

\def\xi{\ifmmode\normalxi\else\elevenpoint\fi}

%D \macros
%D   {definefontvariant,fontvariant,variant}
%D
%D A slightly adapted version of Adam Lindsays variant patches:
%D
%D \starttyping
%D \usetypescriptfile[type-psc] \loadmapfile[texnansi-public-fpl]
%D \usetypescript[palatino][texnansi] \setupbodyfont[palatino]
%D
%D \definefontvariant [Serif][osf] [OsF]
%D
%D \startlines
%D {\sc abcdefgHIJKlmnop}
%D 123{\Var[osf]456}789
%D {\Var[osf] 123{\Var[reset]456}789}
%D {\it  123{\Var[osf]456}789
%D {\Var[osf]123{\Var[reset]456}789}}
%D {\tfb\bf 123{\Var[osf]456}789
%D {\Var[osf] 123{\Var[reset]456}789}}
%D {\sc  123{\Var[osf]456}789
%D {\Var[osf]  123{\Var[reset]456}789}}
%D \stoplines
%D \stoptyping

\def\definefontvariant
  {\dotripleargument\dodefinefontvariant}

\def\dodefinefontvariant[#1][#2][#3]%
  {\setvalue{\??fv#1#2}{#3}}

\def\fontvariant#1#2{\executeifdefined{\??fv#1#2}\empty}

% original:
%
% \def\variant[#1]%
%   {\expanded{\definedfont
%      [\truefontname{\fontstringA\fontstylesuffix\fontvariant\fontstringA{#1}}
%         at \currentfontscale\bodyfontsize]}}
%
% \beginXETEX  \font
%
%   \def\variant[#1]%
%     {\font\variantfont\truefontname{\fontstringA\fontstylesuffix\fontvariant\fontstringA{#1}}
%        at \currentfontscale\bodyfontsize
%      \variantfont}
%
% \endXETEX
%
% better

\def\dosetscaledfont
  {\checkrelativefontsize\fontstyle
   \scaledfont\currentfontscale\bodyfontsize
   \scaledfont\relativefontsize\scaledfont}

\unexpanded\def\variant[#1]% slow
  {\dosetscaledfont
   \expanded{\definedfont
     [\truefontname{\fontstringA\fontstylesuffix\fontvariant\fontstringA{#1}}
        at \scaledfont]}%
   \ignoreimplicitspaces}

\beginXETEX  \font

  \unexpanded\def\variant[#1]%
    {\dosetscaledfont
     \font\variantfont\truefontname{\fontstringA\fontstylesuffix\fontvariant\fontstringA{#1}}
        at \scaledfont
     \variantfont}

\endXETEX

\ifx\Var\undefined \let\Var\variant \fi

%D By default we load the Computer Modern Roman fonts (but
%D not yet at this moment) and activate the 12pt roman
%D bodyfont. Sans serif and teletype are also available and
%D can be called for by \type{\ss} and \type{\tt}.

\setupbodyfont [unk, rm]

%D Also needed is:

\definefont[tinyfont][Mono at 1ex]

%D \macros
%D   {doiffontpresentelse}
%D
%D Some unused left||overs:
%D
%D \starttyping
%D \def\doiffontpresentelse#1#2#3%
%D   {\bgroup
%D    \batchmode\font\dummy=\truefontname{#1}\errorstopmode
%D    \edef\lastfontname{\fontname\dummy}%
%D    \ifx\lastfontname\nullfontname\egroup#3\else\egroup#2\fi}
%D
%D \def\abortiffontnotfound#1%
%D   {\doiffontpresentelse{#1}{}{\showmessage\m!fonts{10}{\truefontname{#1}}\endinput}}
%D \stoptyping
%D
%D We now provide (slow, but sometimes handy):
%D
%D \starttyping
%D \doiffontpresentelse{texnansi-lmr10}{YES}{NO}
%D \doiffontpresentelse{adam-lindsay-modern-serif}{YES}{NO}
%D \stoptyping

\beginTEX

\def\doiffontpresentelse#1%
  {\bgroup
   \batchmode\font\dummy=\truefontname{#1}\errorstopmode
   \edef\lastfontname{\fontname\dummy}%
   \ifx\lastfontname\nullfontname
     \egroup\expandafter\secondoftwoarguments
   \else
     \egroup\expandafter\firstoftwoarguments
  \fi}

\endTEX

\beginETEX \interactionmode

\def\doiffontpresentelse#1%
  {\bgroup
   \scratchcounter\interactionmode
   \batchmode
   \font\dummy=\truefontname{#1}\relax
   \interactionmode\scratchcounter
   \edef\lastfontname{\fontname\dummy}%
   \ifx\lastfontname\nullfontname
     \egroup\expandafter\secondoftwoarguments
   \else
     \egroup\expandafter\firstoftwoarguments
   \fi}

\endETEX

\protect \endinput
