%%
%% This is file `bpfats10.tex',
%% generated with the docstrip utility.
%%
%% The original source files were:
%%
%% stepe.dtx  (with options: `bpfats10')
%% 
%%     This work has been partially funded by the US government
%%  and is not subject to copyright.
%% 
%%     This program is provided under the terms of the
%%  LaTeX Project Public License distributed from CTAN
%%  archives in directory macros/latex/base/lppl.txt.
%% 
%%  Author: Peter Wilson (CUA and NIST)
%%          now at: peter.r.wilson@boeing.com
%% 
\ProvidesFile{bpfats10.tex}[2001/07/16 ATS ats clause boilerplate]
\typeout{bpfats10.tex [2001/07/16 ATS ats clause boilerplate]}

    Each abstract test case has a subclause for the preprocessor
test information and a subclause for each postprocessor
input specification and related test information.
The preprocessor and postprocessor input specifications
are mirror images of each other: they represent the same
semantic information. The preprocessor input model is presented
in the form of a table with five columns:
\begin{itemize}
\item The Id column contains an identifier for the application object
      instantiated in a particular row. The identifier may be
      referenced as the value of an application assertion.
      The identifier is the lowest-level subclause number from
      ISO 10303-\theAPpartno, 4.2 where the application
      element that appears in that row of the table is specified.

\item The V column specifies whether or not the element in that
      row of the table is assigned a verdict in this test case.
      A blank indicates that it is not assigned a verdict in this test case.
      A `*' indicates that it is assigned a verdict
      using a derived verdict criteria. The derived verdict criteria
      determine whether the semantics associated with the application
      element are preserved in the output of the IUT according to
      the reference paths specified in the mapping table defined
      in ISO 10303-\theAPpartno, 5.1. A number in the V column references
      a specific verdict criterion defined in the verdict criteria
      section that follows the preprocessor input specification table.

\item The Application Elements column identifies the particular
      application element instance that is being
      defined by the table. For assertions the role is specified
      in parenthesis.

\item The Value column specifies a specific value for the application
      element. For application objects and attributes the value column
      defines the semantic value for that element's instance in the
      input model. A `\#$<$number$>$' in the column is a reference
      to an entity instance name in the postprocessor input specification
      where the corresponding value is specified. For assertions, this
      column holds a link to the related application object.
      A `$<$not\_present$>$' indicates that the
      application element is not present in the
      input model.

\item The Req column specifies whether the value in the Value column
      is mandatory (M), suggested (S) or constrained (C$<$n$>$), where `n'
      is an integer referencing a note that follows the table.
      A suggested value may be changed by the test realizer.
      A mandatory value may not be changed due to rules in \Express,
      rules in the mapping \maptableorspec, or the requirements of the test
      purpose being assigned a verdict. Each constrained value references
      a note labelled C$<$number$>$ at the end of the preprocessor
      input model table and may be modified according to specific
      constraints specified in it.
\end{itemize}

    The postprocessor input specifications are defined using
ISO 10303-\theAPpartno. The values in the postprocessor specifications
are suggested unless declared mandatory or constrained by the
preprocessor input table.

    The abstract test case specifies all the verdict criteria that are
used to assign a verdict during testing. Special verdict criteria for
preprocessor and postprocessor testing are defined explicitly in each
abstract test case subclause. The relevant derived verdict criteria
for preprocessor and postprocessor testing are identified in the V
column of the preprocessor input table.

\endinput
%%
%% End of file `bpfats10.tex'.
