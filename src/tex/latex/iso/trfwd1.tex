%%
%% This is file `trfwd1.tex',
%% generated with the docstrip utility.
%%
%% The original source files were:
%%
%% isoe.dtx  (with options: `trfwd1')
%% 
%%      This work has been partially funded by the US government and is
%%  not subject to copyright.
%% 
%%      This program is provided under the terms of the
%%  LaTeX Project Public License distributed from CTAN
%%  archives in directory macros/latex/base/lppl.txt.
%% 
%%  Author: Peter Wilson (CUA and NIST)
%%          now at: peter.r.wilson@boeing.com
%% 
\ProvidesFile{trfwd1.tex}[2002/01/10 PAS/TS Foreword boilerplate]
   %% trfwd1.tex  Boilerplate for start of a tech rep Foreword clause

    ISO (the International Organization for Standardization) is a worldwide
federation of national standards bodies (ISO member bodies). The work
of preparing International Standards is normally carried out through
ISO technical committees. Each member body interested in a subject for
which a technical committee has been established has the right to be
represented on that committee. International organizations,
governmental and non-governmental, in liaison with ISO, also take part
in the work. ISO collaborates closely with the International
Electrotechnical Commission (IEC) on all matters of electrotechnical
standardization.

    International Standards are drafted in accordance with the rules
given in the ISO/IEC Directives, Part 3.

    The main task of technical committees is to prepare International
Standards. Draft International Standards adopted by the technical
committees are circulated to the member bodies for voting. Publication
as an International Standard requires approval by at least 75\% of the
member bodies casting a vote.

    In other circumstances, particularly when there is an urgent market
requirement for such documents, a technical committee may decide to
publish other types of normative document:
\begin{itemize}
\item an ISO Publicly Available Specification (ISO/PAS) represents an
agreement between technical experts in an ISO working group and is
accepted for publication if it is approved by more than 50\% of the
members of the parent committee casting a vote;

\item an ISO Technical Specification (ISO/TS) represents an agreement
between the members of a technical committee and is accepted for
publication if it is approved by 2/3 of the members of the committee
casting a vote.
\end{itemize}

    An ISO/PAS or ISO/TS is reviewed every three years with a view to
deciding whether it can be transformed into an International Standard.
\par

\endinput
%%
%% End of file `trfwd1.tex'.
