%% BEGIN pst-plot.tex
%%
%% Plots and axes with PSTricks 97.
%% See the PSTricks User's Guide for documentation.
%%
\def\fileversion{97 patch 2}
\def\filedate{1999/12/12}
%%
%% COPYRIGHT 1993, 1994, 1999 by Timothy Van Zandt, tvz@nwu.edu.
%%
%% This program can be redistributed and/or modified under the terms
%% of the LaTeX Project Public License Distributed from CTAN
%% archives in directory macros/latex/base/lppl.txt.
%%
\message{ v\fileversion, \filedate}

\csname PSTplotLoaded\endcsname
\let\PSTplotLoaded\endinput

\ifx\PSTricksLoaded\endinput\else
  \def\next{\input pstricks.tex }
  \expandafter\next
\fi

\ifx\MultidoLoaded\endinput\else
  \def\next{\input multido.tex }
  \expandafter\next
\fi

\edef\TheAtCode{\the\catcode`\@}
\catcode`\@=11

% Using lists of data is optimized for \dataplot and \fileplot
% Here is the tricky part. As each line is read from file,
% we want to ignore trailing delimiters, and convert arbitrary
% strings of non-trailing delimiters to _D_.
% We end up with
%   D x1 D y1 D x2 D y2 ... D xn D yn
%
\begingroup
\catcode`\{=13
\catcode`\}=13
\catcode`\(=13
\catcode`\)=13
\catcode`\,=13
\catcode`\!=1
\catcode`\*=2
\catcode`\ =13
\catcode`\_=13
\catcode`\^^M=13
\gdef\pst@datadelimiters!% Begin def
\catcode`\{=13%
\catcode`\}=13%
\catcode`\(=13%
\catcode`\)=13%
\catcode`\,=13%
\catcode`\ =13%
\catcode`\^^M=13%
\def,##1!%
\ifcat\noexpand,\noexpand##1%
\expandafter##1%
\else\space%
D\space##1%
\fi*%
\let(,\let),\let{,\let},\let ,\let^^M,\let_\@empty*% End def
\endgroup
\begingroup
\catcode`\,=13
\catcode`\_=13
\gdef\savedata@#1[#2]{%
  \xdef\pst@tempg{#2_}%
  \endgroup
  \let#1\pst@tempg
  \global\let\pst@tempg\relax
  \ignorespaces}
\gdef\readdata@{%
  \read1 to \pst@tempa
  \expandafter\readdata@@\pst@tempa_\@nil
  \ifeof1\else\expandafter\readdata@\fi}
\gdef\pst@@readfile#1#2\@nil{\addto@pscode{,#1#2}}%
\gdef\readdata@@#1#2\@nil{\xdef\pst@tempg{\pst@tempg,#1#2}}%
\endgroup

\def\readdata#1#2{%
  \openin1=#2
  \begingroup
  \def\pst@tempg{}%
  \ifeof1
    \@pstrickserr{Data file `#2' not found.}\@ehpa
  \else
    \pst@datadelimiters
    \catcode`\[=1
    \catcode`\]=2
    \readdata@%
  \fi
  \endgroup
  \let#1\pst@tempg
  \global\let\pst@tempg\relax
  \ignorespaces}

\def\pst@readfile#1{{\let\readdata@@\pst@@readfile\readdata\pst@tempg{#1}}}
\def\pst@altreadfile#1{%
  \openin1=#1
  \ifeof1
    \@pstrickserr{Data file `#1' not found.}\@ehpa
  \else
    \catcode`\{=10
    \catcode`\}=10
    \catcode`\(=10
    \catcode`\)=10
    \catcode`\,=10
    \catcode`\^^M=10
    \catcode`\[=1
    \catcode`\]=2
    \pst@@altreadfile
  \fi}
\def\pst@@altreadfile{%
  \read1 to \pst@tempg
  \expandafter\pst@@@altreadfile\pst@tempg\@empty\@nil
  \ifeof1\else\expandafter\pst@@@altreadfile\fi}
\def\pst@@@altreadfile#1#2\@nil{\addto@pscode{#1#2}}%

\def\savedata#1{\begingroup\pst@datadelimiters\savedata@{#1}}

\def\beginplot@line{\begin@OpenObj}
\def\endplot@line{\psline@ii}
\def\beginplot@polygon{\begin@ClosedObj}
\def\endplot@polygon{\pspolygon@ii}
\def\beginplot@curve{\begin@OpenObj}
\def\endplot@curve{\pscurve@ii}
\def\beginplot@ecurve{\begin@OpenObj}
\def\endplot@ecurve{\psecurve@ii}
\def\beginplot@ccurve{\begin@ClosedObj}
\def\endplot@ccurve{\psccurve@ii}
\def\beginplot@dots{\begin@SpecialObj}
\def\endplot@dots{\psdots@ii}
\def\beginplot@bezier{\begin@OpenObj}
\def\endplot@bezier{\psbezier@ii}
\def\beginplot@cbezier{\begin@ClosedObj}
\def\endplot@cbezier{\pscbezier@ii}

\def\psset@plotstyle#1{%
  \@ifundefined{beginplot@#1}%
    {\@pstrickserr{Plot style `#1' not defined}\@eha}%
    {\edef\psplotstyle{#1}}}
\psset@plotstyle{line}

\def\psset@plotpoints#1{%
  \pst@cntg=#1\relax
  \ifnum\pst@cntg<2
    \@pstrickserr{plotpoints parameter must be at least 2}\@ehpa
  \else
    \advance\pst@cntg-1
    \edef\psk@plotpoints{\the\pst@cntg\space}%
  \fi}
\psset@plotpoints{50}

% For quick plots, define:
%   \beginqp@<foo>   : What to do to first point (PS code only).
%   \doqp@<foo>      : What to do to subsequent points (PS code only).
%   \endqp@<foo>     : How to end plot.
%   \testqp@<foo>    : Set \@psttrue if OK to use quick plot.

\def\beginqp@line{\pst@oplineto}
\def\doqp@line{L }
\def\endqp@line{\end@OpenObj}%
\def\testqp@line{%
  \ifdim\pslinearc>\z@\else
    \ifshowpoints\else
      \ifx\psk@arrowA\@empty
        \ifx\psk@arrowB\@empty
          \@psttrue
        \fi
      \fi
    \fi
  \fi}

\def\beginqp@polygon{moveto }
\def\doqp@polygon{L }
\def\endqp@polygon{%
  \addto@pscode{closepath}%
  \end@ClosedObj}
\def\testqp@polygon{%
  \ifdim\pslinearc>\z@\else
    \ifshowpoints\else
      \@psttrue
    \fi
  \fi}

\def\beginqp@dots{%
  \psk@dotsize
  \@nameuse{psds@\psk@dotstyle}
% DG/SR modification begin - Dec. 12, 1999 - Patch 2
%  /TheDot { gsave \psk@dotangle \psk@dotscale Dot grestore } def
%  TheDot }
  Dot }
%\def\doqp@dots{TheDot }
\def\doqp@dots{Dot }
% DG/SR modification end
\def\endqp@dots{\end@SpecialObj}
\def\testqp@dots{\@psttrue}

\def\beginqp@bezier{/n 0 def \pst@oplineto}
\def\doqp@bezier{/n n 1 add def n 3 mod 0 eq { curveto } if }
\def\endqp@bezier{%
  \addto@pscode{n 3 mod { pop pop } repeat}
  \end@OpenObj}%
\def\testqp@bezier{%
  \ifshowpoints\else
    \ifx\psk@arrowA\@empty
      \ifx\psk@arrowB\@empty
        \@psttrue
      \fi
    \fi
  \fi}

\def\beginqp@cbezier{/n 0 def moveto }
\def\doqp@cbezier{\doqp@bezier}
\def\endqp@cbezier{%
  \addto@pscode{n 3 mod { pop pop } repeat closepath}
  \end@ClosedObj}%
\def\testqp@cbezier{\ifshowpoints\else\@psttrue\fi}

\def\dataplot{\def\pst@par{}\pst@object{dataplot}}
\def\dataplot@i#1{%
  \pst@killglue
  \begingroup
    \use@par
    \@pstfalse
    \@nameuse{testqp@\psplotstyle}%
    \if@pst
      \dataplot@ii{\addto@pscode{#1}}%
    \else
      \listplot@ii{\addto@pscode{#1}}%
    \fi
  \endgroup
  \ignorespaces}
\def\dataplot@ii#1{%
  \@nameuse{beginplot@\psplotstyle}%
    \addto@pscode{%
      /Dx { \pst@number\psxunit mul /D { Dy } def } def
      /Dy { \pst@number\psyunit mul Do /D { Dx } def } def
      /D { /D { Dx } def } def
      /Do {
        \@nameuse{beginqp@\psplotstyle}%
        /Do { \@nameuse{doqp@\psplotstyle}} def
      } def}%
    #1%
    \addto@pscode{D}%
  \@nameuse{endqp@\psplotstyle}}

\def\fileplot{\def\pst@par{}\pst@object{fileplot}}
\def\fileplot@i#1{%
  \pst@killglue
  \begingroup
    \use@par
    \@pstfalse
    \@nameuse{testqp@\psplotstyle}%
    \if@pst
      \dataplot@ii{\pst@readfile{#1}}%
    \else
      \listplot@ii{\pst@altreadfile{#1}}%
    \fi
  \endgroup
  \ignorespaces}

\pst@def{ScalePoints}<%
  /y ED /x ED
  counttomark dup dup cvi eq not { exch pop } if
  /m exch def /n m 2 div cvi def
  n { y mul m 1 roll x mul m 1 roll /m m 2 sub def } repeat>

\def\listplot{\def\pst@par{}\pst@object{listplot}}
\def\listplot@i#1{\listplot@ii{\addto@pscode{#1}}}
\def\listplot@ii#1{%
  \@nameuse{beginplot@\psplotstyle}%
    \addto@pscode{/D {} def mark}%
    #1%
    \addto@pscode{\pst@number\psxunit \pst@number\psyunit \tx@ScalePoints}%
  \@nameuse{endplot@\psplotstyle}}

% \psplot

\def\psplotinit#1{\xdef\psplot@init{#1 }}
\def\psplot@init{}

\def\psplot{\def\pst@par{}\pst@object{psplot}}
\def\psplot@i#1#2#3{%
  \pst@killglue
  \begingroup
    \use@par
    \@nameuse{beginplot@\psplotstyle}%
    \addto@pscode{%
      \psplot@init
      /x #1 def
      /x1 #2 def
      /dx x1 x sub \psk@plotpoints div def
      /xy {
        x \pst@number\psxunit mul
        #3 \pst@number\psyunit mul
      } def}%
    \gdef\psplot@init{}%
    \@pstfalse
    \@nameuse{testqp@\psplotstyle}%
    \if@pst
      \psplot@ii
    \else
      \psplot@iii
    \fi
  \endgroup
  \ignorespaces}
\def\psplot@ii{%
    \addto@pscode{%
      xy \@nameuse{beginqp@\psplotstyle}
      \psk@plotpoints 1 sub {
        /x x dx add def
        xy \@nameuse{doqp@\psplotstyle}
      } repeat
      /x x1 def
      xy \@nameuse{doqp@\psplotstyle}}%
  \@nameuse{endqp@\psplotstyle}}
\def\psplot@iii{%
    \addto@pscode{%
      mark
      /n 2 def
      \psk@plotpoints {
        xy
        n 2 roll
        /n n 2 add def
        /x x dx add def
      } repeat
      /x x1 def
      xy
      n 2 roll}%
  \@nameuse{endplot@\psplotstyle}}

\def\parametricplot{\def\pst@par{}\pst@object{parametricplot}}
\def\parametricplot@i#1#2#3{%
  \pst@killglue
  \begingroup
    \use@par
    \@nameuse{beginplot@\psplotstyle}%
    \addto@pscode{%
      \psplot@init
      /t #1 def
      /t1 #2 def
      /dt t1 t sub \psk@plotpoints div def
      /xy {
        #3
        \pst@number\psyunit mul exch
        \pst@number\psxunit mul exch
      } def}%
    \gdef\psplot@init{}%
    \@pstfalse
    \@nameuse{testqp@\psplotstyle}%
    \if@pst
      \parametricplot@ii
    \else
      \parametricplot@iii
    \fi
  \endgroup
  \ignorespaces}
\def\parametricplot@ii{%
    \addto@pscode{%
      xy \@nameuse{beginqp@\psplotstyle}
      \psk@plotpoints 1 sub {
        /t t dt add def
        xy \@nameuse{doqp@\psplotstyle}
      } repeat
      /t t1 def
      xy \@nameuse{doqp@\psplotstyle}}%
  \@nameuse{endqp@\psplotstyle}}
\def\parametricplot@iii{%
    \addto@pscode{%
      mark
      /n 2 def
      \psk@plotpoints {
        xy
        n 2 roll
        /n n 2 add def
        /t t dt add def
      } repeat
      /t t1 def
      xy
      n 2 roll}%
  \@nameuse{endplot@\psplotstyle}}

% These axes macros are complicated. Be careful.

% \pst@ticks{angle}{dx}{n}{int}
% int=1 if ticks appear on top of axes, 0 otherwise.
\def\pst@ticks#1#2#3#4{%
  \begin@SpecialObj
    \addto@pscode{%
      #1 rotate
      /n #3 def
      /dx #2 def
      n 0 lt { /dx dx neg def /n n neg def } if
      /y2 \psk@ticksize CLW 2 div add def
      /y1 y2 neg def
      \ifnum\psk@tickstyle=1
        \ifdim#4<\z@ /y2 \else /y1 \fi 0 def
      \else
        \ifnum\psk@tickstyle=-1
          \ifdim#4<\z@ /y1 \else /y2 \fi 0 def
        \fi
      \fi
      /x dx def
      n { x y1 moveto x y2 lineto stroke /x x dx add def } repeat}%
  \end@SpecialObj}


\def\psset@ticksize#1{\pst@getlength{#1}\psk@ticksize}
\psset@ticksize{3pt}

\def\psset@tickstyle#1{\pst@expandafter\psset@@tickstyle{#1}\@nil}
\def\psset@@tickstyle#1#2\@nil{%
  \ifx#1f\let\psk@tickstyle\z@\else
    \ifx#1t\let\psk@tickstyle\@ne\else
      \ifx#1b\let\psk@tickstyle\m@ne\else
        \@pstrickserr{Bad tick style: `#1#2'}\@ehpa
  \fi\fi\fi}
\psset@tickstyle{full}

\def\psset@ticks#1{\pst@expandafter\psset@@ticks{#1}\@nil\psk@ticks}
\def\psset@@ticks#1#2\@nil#3{%
  \ifx#1a\let#3\z@\else
    \ifx#1x\let#3\@ne\else
      \ifx#1y\let#3\tw@\else
        \ifx#1n\let#3\thr@@\else
          \@pstrickserr{Bad argument: `#1#2'}\@ehpa
  \fi\fi\fi\fi}
\psset@ticks{all}

\def\psset@labels#1{\pst@expandafter\psset@@ticks{#1}\@nil\psk@labels}
\psset@labels{all}

\def\psset@Ox#1{\edef\psk@Ox{#1}}
\psset@Ox{0}
\def\psset@Dx#1{\edef\psk@Dx{#1}}
\psset@Dx{1}
\def\psset@dx#1{%
  \pssetxlength\pst@dimg{#1}%
  \edef\psk@dx{\number\pst@dimg}}
\psset@dx{0}

\def\psset@Oy#1{\edef\psk@Oy{#1}}
\psset@Oy{0}
\def\psset@Dy#1{\edef\psk@Dy{#1}}
\psset@Dy{1}
\def\psset@dy#1{%
  \pssetylength\pst@dimg{#1}%
  \edef\psk@dy{\number\pst@dimg}}
\psset@dy{0}

\newif\ifshoworigin
\def\psset@showorigin#1{\@nameuse{showorigin#1}}
\psset@showorigin{true}


\def\psaxes{\def\pst@par{}\pst@object{psaxes}}
\def\psaxes@i{\pst@getarrows\psaxes@ii}
\def\psaxes@ii(#1){\@ifnextchar({\psaxes@iii(#1)}{\psaxes@iv(0,0)(0,0)(#1)}}
\def\psaxes@iii(#1)(#2){%
  \@ifnextchar(%
    {\psaxes@iv(#1)(#2)}%
    {\psaxes@iv(#1)(#1)(#2)}}
\def\psaxes@iv(#1,#2)(#3,#4)(#5,#6){%
  \setbox\pst@hbox=\hbox\bgroup
    \use@par
    \pssetxlength\pst@dimg{#1}% o-x
    \pssetylength\pst@dimh{#2}% o-y
    \pssetxlength\pst@dima{#3}% bl-x
    \pssetylength\pst@dimb{#4}% bl-y
    \pssetxlength\pst@dimc{#5}% ur-x
    \pssetylength\pst@dimd{#6}% ur-y
% Whole thing will be translated to origin:
    \advance\pst@dima-\pst@dimg % Dist. from bl-x to o-x
    \advance\pst@dimb-\pst@dimh % Dist. from bl-y to o-y
    \advance\pst@dimc-\pst@dimg % Dist. from ur-x to o-x
    \advance\pst@dimd-\pst@dimh % Dist. from ur-y to o-y
% Make lines/arrows or frame:
    \@nameuse{psxs@\psk@axesstyle}%
% "\pslabelsep" should be from the edge of the axis.
    \advance\pslabelsep.5\pslinewidth
% Now the ticks and labels. Start by checking for "\multido".
% !!Need to fix this so that does nothing when there are 0 ticks.!!
    \begingroup
      \ifdim\pst@dimb=\z@\else\showoriginfalse\fi
      \ifnum\psk@dx=\z@
        \pst@dimg=\psk@Dx\psxunit
        \edef\psk@dx{\number\pst@dimg}%
      \fi
      \ifnum\psk@ticks<\tw@
        \ifnum\psk@tickstyle>\z@\else
          \advance\pslabelsep\psk@ticksize\p@
        \fi
      \fi
      \pst@hlabels\pst@dimc\psk@arrowB
      \pst@hlabels\pst@dima\psk@arrowA
    \endgroup
    \begingroup
      \ifdim\pst@dima=\z@\else\showoriginfalse\fi
      \ifnum\psk@dy=\z@
         \pst@dimg=\psk@Dy\psyunit
         \edef\psk@dy{\number\pst@dimg}%
      \fi
      \ifodd\psk@ticks\else
        \ifnum\psk@tickstyle>\z@\else
          \advance\pslabelsep\psk@ticksize\p@
        \fi
      \fi
      \pst@vlabels\pst@dimd\psk@arrowB
      \pst@vlabels\pst@dimb\psk@arrowA
    \endgroup
% Now close "\pst@hbox" (which is 0-dimensional), and put it at the origin.
  \egroup
  \pssetxlength\pst@dimg{#1}%
  \pssetylength\pst@dimh{#2}%
  \leavevmode\psput@cartesian\pst@hbox
  \ignorespaces}

\def\psxs@axes{%
  \psxs@@axes\pst@dima\pst@dimc{}%
  \psxs@@axes\pst@dimb\pst@dimd{exch}}
\def\psxs@@axes#1#2#3{%
  \begin@SpecialObj
    \ifdim#1=\z@
      \def\psk@arrowA{C}%
    \else
      \ifdim#2=\z@
        \def\psk@arrowB{C}%
      \fi
    \fi
    \let\pst@linetype\pst@arrowtype
    \pst@addarrowdef
    \addto@pscode{%
      \pst@number#2 0 #3
      \pst@number#1 0 #3
      ArrowA
      CP 4 2 roll
      ArrowB
      L
      pop pop}%
    \pst@stroke
  \end@SpecialObj}

\def\psxs@frame{%
  \begin@SpecialObj
    \addto@pscode{%
      0 0 moveto \pst@number\pst@dimc 0 L
      0 \pst@number\pst@dimd 2 copy rlineto L closepath}%
    \pst@stroke
    \psk@fillstyle
  \end@SpecialObj
  \let\psk@arrowA\@empty
  \let\psk@arrowB\@empty}

\def\psset@axesstyle#1{%
  \@ifundefined{psxs@#1}%
    {\@pstrickserr{Axes style `#1' not defined}\@eha}%
    {\edef\psk@axesstyle{#1}}}
\psset@axesstyle{axes}

\def\psxs@none{\let\psk@arrowA\@empty\let\psk@arrowB\@empty}

% The origin is never the only label.
\def\pst@hlabels#1#2{%
  \ifdim#1=\z@\else
    \ifx#2\empty\else
      \advance#1\ifdim#1>\z@-\fi7\pslinewidth
    \fi
    \pst@cnta=#1\relax                % Distance (in sp) to end.
    \divide\pst@cnta\psk@dx\relax     % Number of ticks/labels
    \ifnum\pst@cnta=\z@\else
      \pst@dimb=\psk@dx sp            % Space between ticks.
      \ifnum\psk@ticks<\tw@
        \pst@ticks{0}{\pst@number\pst@dimb}{\the\pst@cnta}{\pst@dimd}%
      \fi
      \ifnum\psk@labels<\tw@ \pst@@hlabels\fi
      \showoriginfalse
    \fi
  \fi}

% Knows \pst@dimb and \pst@cnta
\def\pst@@hlabels{%
  \vbox to\z@{%
    \ifdim\pst@dimd>\z@\vskip\pslabelsep\else\vss\fi
    \ifnum\pst@cnta<\z@
      \pst@dimb=-\pst@dimb
    \fi
    \hbox to\z@{%
      \ifshoworigin\hbox to \z@{\hss\pshlabel{\psk@Ox}\hss}\fi
      \mmultido
        {\n=\psk@Ox+\psk@Dx}%
        {\pst@cnta}%
        {\hskip\pst@dimb\hbox to \z@{\hss\pshlabel{\n}\hss}}%
      \hss}%
    \ifdim\pst@dimd>\z@\vss\else\vskip\pslabelsep\fi}}%

\def\pshlabel#1{$#1$}

\def\pst@vlabels#1#2{%
  \ifdim#1=\z@\else
    \ifx#2\empty\else
      \advance#1\ifdim#1>\z@-\fi7\pslinewidth
    \fi
    \pst@cnta=#1\relax                % Distance (in sp) to end.
    \divide\pst@cnta\psk@dy\relax     % Number of ticks/labels
    \ifnum\pst@cnta=\z@\else
      \pst@dima=\psk@dy sp            % Space between ticks.
      \ifodd\psk@ticks\else
        \pst@ticks{90}{\pst@number\pst@dima}{\the\pst@cnta}{-\pst@dimc}%
      \fi
      \ifodd\psk@labels\else\pst@@vlabels\fi
      \showoriginfalse
    \fi
  \fi}

% Knows \pst@dima and \pst@cnta
\def\pst@@vlabels{%
  \vbox to\z@{%
    \ifnum\pst@cnta>\z@
      \pst@dima=-\pst@dima
    \fi
    \offinterlineskip
    \ifshoworigin
      \vbox to \z@{\vss\hbox to\z@{%
        \ifdim\pst@dimc>\z@\hss\else\hskip\pslabelsep\fi
        \psvlabel{\psk@Oy}%
        \ifdim\pst@dimc>\z@\hskip\pslabelsep\else\hss\fi}\vss}%
    \fi
    \mmultido
      {\n=\psk@Oy+\psk@Dy}%
      {\pst@cnta}%
      {\vbox to\pst@dima{\vss}\vbox to \z@{\vss\hbox to\z@{%
        \ifdim\pst@dimc>\z@\hss\else\hskip\pslabelsep\fi
        \psvlabel{\n}%
        \ifdim\pst@dimc>\z@\hskip\pslabelsep\else\hss\fi}\vss}}%
    \vss}}

\def\psvlabel#1{$#1$}

\catcode`\@=\TheAtCode\relax
\endinput
%%
%% END pst-plot.tex
