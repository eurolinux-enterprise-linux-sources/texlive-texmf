%% BEGIN pst-grad.tex
%%
%% Gradient fillstyle with PSTricks.
%% See the PSTricks User's Guide for description.
%% This uses the header file `pst-grad.pro'.
%%
%% Based on some EPS files by leeweyr!bill@nuchat.sccsi.com (W. R. Lee).
%%
%%
%% COPYRIGHT 1993, 1994, 1999 by Timothy Van Zandt, tvz@nwu.edu.
%%           2006 Herbert Voss, hvoss@tug.org
%%
%% This program can be redistributed and/or modified under the terms
%% of the LaTeX Project Public License Distributed from CTAN
%% archives in directory macros/latex/base/lppl.txt.
%%
%% This defines a new fill style, "gradient", for use with PSTricks,
%% which has gradiated color. The following parameters are used:
%%
%%    gradbegin=color     : Beginning color.
%%    gradend=color       : Final color.
%%    gradlines=int       : Number of lines to use. The higher the number,
%%                           the slower the rendering.
%%    gradmidpoint=num    : Gradient color goes from gradbegin to gradend,
%%                           and then back to beginning. Midpoint (point
%%                           where "gradend" color appears, is gradmidpoint
%%                           from the top.  (0 <= Gmidpoint <= 1).
%%    gradangle=angle     : Rotate image by angle.
%%    GradientCircle=true : Instead of a linear a circled gradient is build.
%%                          (version 1.04)
%%    GradientPos=(x,y) :   the center of the circled gradient
%%                          (version 1.04)
%%    GradientScale=float : scaling factor of the circled gradient
%%                          (version 1.04)
%%    GradientHSB	  : Use HSB color model
%%
\csname GradientLoaded\endcsname
\let\GradientLoaded\endinput
\ifx\PSTricksLoaded\endinput\else
  \def\next{\input pstricks.tex }\expandafter\next
\fi
\ifx\PSTXKeyLoaded\endinput\else\input pst-xkey \fi

\def\fileversion{1.05}
\def\filedate{2006/11/04}
\message{`pst-plot' v\fileversion, \filedate\space (tvz,dg,hv)}

\edef\TheAtCode{\the\catcode`\@}
\catcode`\@=11
\pst@addfams{pst-grad}

\pstheader{pst-grad.pro}

\newrgbcolor{gradbegin}{0 .1 .95}
\newrgbcolor{gradend}{0 1 1}
\define@key[psset]{pst-grad}{gradbegin}{\pst@getcolor{#1}\psgradbegin}
\define@key[psset]{pst-grad}{gradend}{\pst@getcolor{#1}\psgradend}
\define@key[psset]{pst-grad}{gradlines}{%
  \pst@getint{#1}\psgradlines
  \ifnum\psgradlines<2
    \@pstrickserr{gradlines must be at least 2}\@epha
    \def\psgradlines{2 }%
  \fi}
\define@key[psset]{pst-grad}{gradmidpoint}{\pst@checknum{#1}\psgradmidpoint}
\define@key[psset]{pst-grad}{gradangle}{\pst@getangle{#1}\psk@gradangle}
\psset[pst-grad]{gradangle=0,gradlines=300,gradend=gradend,gradbegin=gradbegin,
    gradmidpoint=0.9}

% Denis Girou - April 1998  ------- patch 2 (hv)
% To define the gradient as linear or as circle
\define@boolkey[psset]{pst-grad}[Pst@]{GradientCircle}[true]{}
% Position of the center of the gradient
\define@key[psset]{pst-grad}{GradientPos}{\psset@@GradientPos#1}%
  \def\psset@@GradientPos(#1){\edef\ps@GradientPos{#1}}
% Scale factor
\define@key[psset]{pst-grad}{GradientScale}{\def\ps@GradientScale{#1}}
\psset[pst-grad]{GradientScale=1,GradientPos={(0,0)},GradientCircle=false}
%
\define@boolkey[psset]{pst-grad}[Pst@]{gradientHSB}[true]{}
\psset[pst-grad]{gradientHSB=false}
%
\def\psfs@gradient{%
  \ifPst@gradientHSB
    \addto@pscode{%
      gsave
      gsave \pst@usecolor\psgradbegin currenthsbcolor grestore
      gsave \pst@usecolor\psgradend currenthsbcolor grestore
      \psgradlines
      \psgradmidpoint
      \psk@gradangle
%  hv 2004-05-05 begin    fixed a spurious blank
      tx@GradientDict begin GradientFillHSB end grestore%
    }%
%    tx@GradientHSBDict begin GradientFillHSB end grestore}
%  hv 2004-05-05 end
  \else%
%    hv 2004-06-25 begin    fixed a spurious blank
    \pst@getcoor{\ps@GradientPos}{\pst@tempa}% <- "%"  hv 2004-06-23 
%    hv end
    \addto@pscode{%
      gsave
      \ifPst@GradientCircle true \else false \fi
      \ps@GradientScale\space
      \pst@tempa\space
      gsave \pst@usecolor\psgradbegin currentrgbcolor grestore
      gsave \pst@usecolor\psgradend currentrgbcolor grestore
      \psgradlines
      \psgradmidpoint
      \psk@gradangle
      tx@GradientDict begin GradientFill end grestore%
    }%
  \fi%
}
%
\catcode`\@=\TheAtCode\relax
%
\endinput
%%
%% END pst-grad.tex
